% -*- mode: context; -*-
\chapter{Miscellaneous Problems}
\startitemize[n, 1*broad]
\item Let the numbers be $a$ and $b$ and harmonic means are $h_1$ and $h_1$. Let $d$ be the c.d.\ of the
  corresponding A.P. Then

  $\frac{1}{b} = \frac{1}{a} + 3d\Rightarrow d = \frac{a - b}{3ab}$

  $\frac{1}{h_1} = \frac{1}{a} + \frac{a - b}{3ab} = \frac{a + 2b}{3ab}$.

  Given that $\frac{3ab}{a + 2b} = \sqrt{ab}\Rightarrow 9a^2b^2 = (a + 2b)^2ab \Rightarrow 9ab = a^2 + 4b^2
  + 4ab$

  $\Rightarrow a^2 - 5ab + 4b^2 = 0 \Rightarrow (a - b)(a - 4b) = 0$.

  So either the numbers are equal which will make H.M.'s also equal to the numbers or one of the number is
  four times the other number.
\item Let the numbers be $a$ and $b$ and harmonic mean be $h$. Then

  $\frac{1}{h} = \frac{\frac{1}{a} + \frac{1}{b}}{2} \Rightarrow h = \frac{2ab}{a + b} = 4 \Rightarrow 2a +
  2b - ab = 0$

  Given that $2A + G^2 = 27 \Rightarrow a + b + ab = 27 \Rightarrow a + b = 9$.

  Solving we get $a, b = 6, 3$.
\item Since $n$ is odd, the no.\ of positive term of the given series is $\frac{n + 1}{2}$ and no.\ of
  negative terms of negative terms of the series is $\frac{n - 1}{2}$.

  Sum of positive terms is $\frac{n + 1}{4}\left[2a + \frac{n - 1}{2}d\right]$ and sum of negative terms is
  $\frac{n - 1}{4}\left[2(a + d) + \frac{n - 3}{2}d\right]$.

  Adding we get the required sum.

  Alternatively we observe that first $\frac{n- 1}{2}$ pairs will have a sum of $-\frac{(n - 1)d}{2}$ and
  $n$th term is $a + (n- 1)d$. So the sum is $a + \frac{n - 1}{2}d$.
\item $x = 2n\pi\pm \cos^{-1}y$. So we see that it is a periodic function which has a period of $2\pi$. If
  we move only in one(positive/negative) direction then all values of $y$ will give an A.P., however, if we
  want to move in both directions then $x$ must be $0, \pi$ making $y = \pm1$.
\item Let $a$ be the first term and $r$ be the common ratio of the G.P. Sum of all terms is given by
  $\frac{a}{1 - r}$ and sum of odd terms is given by $\frac{a}{1 - r^2}$.

  Given that $\frac{a}{1 - r} = \frac{5a}{1 - r^2}\Rightarrow 1 + r = 5\Rightarrow r = 4$.
\item Let $S_p$ represent sum of $p$ terms of the given A.P., then $t_p = S_p - S_{p - 1} = a + (2p - 1)b$.

  $\Rightarrow t_4 = a + 7b$.
\item We see that c.d.\ is $5$. Either we can apply the $t_n = 3 + (n - 1)5$ and solve for each or on simple
  observation we find that last digit should be $3$ and hence $303$ is the number, which will be a term in
  the given sequence.
\item Let $a$ be the first term and $d$ be the common ratio of the A.P., then $A = a, B = a + d, c = a +
  2d$.

  Substituting the points in the given equation we find that the point $(1, -2)$ satisfies the
  equation. Hence, the given straight line will pass through $(1, -2)$.
\item Let $S_n$ represent the sum of $n$ terms of the A.P., then $S_n = \frac{3n^2 + 5n}{36}$.

  $t_n = S_n - S_{n - 1} = \frac{6n - 3 + 5}{36} = \frac{3n + 1}{18}$.

  c.d. $= t_n - t_{n - 1} = \frac{1}{6}$.
\item Given, $\frac{a - b}{b - c} = \frac{a}{c} \Rightarrow b = \frac{2ac}{a + c}$, so the given sequence is
  a harmonic sequence.
\item Common ratio of the given series is $\frac{2x}{x + 3}$. For the sum to be definite c.r should be less
  that $|1|$. Therefore,

  $-1 < \frac{2x}{x + 3} < 1 \Rightarrow -1 < x < 3$.
\item Let the numbers be $x_1, x_2, \ldots, x_n$, then using A.M.-G.M. relation we have

  $\frac{x_1 + x_2 + \cdots + x_n}{n}\geq \sqrt[n]{x_1.x_2.\ldots x_n} \Rightarrow x_1 + x_2 + \cdots +
  x_n\geq n$.
\item Let $a$ be the first term and $d$ be the c.d. of the given A.P. Given that $S_{2n} = 3S_n$ so we have

  $n[2a + (2n - 1)d] = \frac{3n}{2}[2a + (n - 1)d] \Rightarrow 4a + (4n - 2)d = 6a + 3(n - 1)d$

  $\Rightarrow 2a = (n + 1)d$

  $\frac{S_{3n}}{S_n} = \frac{3[2a + (3n - 1)d]}{2a + (n - 1)d} = \frac{12nd}{2nd} = 6$.
\item Each term consistes of two integers which are in two different A.P. The $n$th term is given by $t_n =
  [1 + (n - 1)2].[3 + (n - 1)2] = (2n - 1)(2n + 1) = 4n^2 - 1$

  Thus, sum $= S_n = \displaystyle\sum_{i = 1}^n(4i^2 - 1) = 4.\frac{n(n + 1)(2n + 1)}{6} - 1$.
\item Let $S = 1 + \frac{4}{5} + \frac{7}{5^2} + \frac{10}{5^3} + \cdots$. We observe that numerator is in
  A.P. with first term as $1$ and c.d.\ as $3$ while denominator is in G.P.\ with first terms as $1$ and
  c.r.\ as $\frac{1}{5}$.

  So using the technique for AGP we have $S - \frac{S}{5} = 1 + \frac{3}{5} + \frac{3}{5^2} + \cdots$

  $\Rightarrow \frac{4S}{5} = 1 + \frac{3}{5}.\frac{1}{1 - \frac{1}{5}} = 1 + \frac{3}{4}
  = \frac{7}{4} \Rightarrow S = \frac{35}{16}$.
\item From the given information we can write $2b = a + c; x^2 = ab$ and $y^2 = bc$

  $\Rightarrow x^2 + y^2 = b(a + c) = 2b^2$ and thus, $x^2, b^2, y^2$ are in A.P.
\item Given, $2\log_{10}(2^x - 1) = \log_{10}2 + \log_{10}(2^x + 3)$

  $\Rightarrow (2^x - 1)^2 = 2.2^x + 6 \Rightarrow 2^{2x} -4.2^x - 5 = 0 \Rightarrow 2^x = 5, -1$, however,
  $2^x$ cannot be $-1$.

  $\therefore 2^x = 5 \Rightarrow x = \log_25$.
\item Since $p, q, r$ are in A.P.\ so we can write $q - p = r - q$. Now

  $\frac{m^{7q}}{m^{7p}} = m^{7(q - p)} = m^{7(r - q)} = \frac{m^{7r}}{m^{7q}}$, and thus, $m^{7p}, m^{7q},
  m^{7r}$ are in G.P.
\item $x = \frac{1}{1 - \cos^2\phi} = \csc^2\phi; y = \frac{1}{1 - \sin^2\phi} = \sec^2\phi$ and $z
  = \frac{1}{1 - \cos^2\phi\sin^2\phi}$

  $xyz = \frac{1}{\sin^2\phi\cos^2\phi(1 - \cos^2\phi\sin^2\phi)}$

  $xy + z = \frac{1}{\sin^2\phi\cos^2\phi} + \frac{1}{1 - \cos^2\phi\sin^2\phi}
  = \frac{1}{\sin^2\phi\cos^2\phi(1 - \cos^2\phi\sin^2\phi)}$.

  Thus, $xyz = xy + z$.
\item Rewriting we have given series as $\frac{1}{\sqrt{2}}(2 + 4 + 6 + 8 + \cdots) = \frac{n(n +
  1)}{\sqrt{2}}$.
\item Since $a, b, c$ are in A.P., we can write $a - b + c = b \Rightarrow a - b + c = \frac{b - a + c + a +
  b - c}{2}$

  Thus, $\frac{b -a + c}{2}, \frac{a - b + c}{2}, \frac{a + b - c}{2}$ are in A.P.

  $\Rightarrow s - a, s - b, s - c$ are in A.P. $\Rightarrow \frac{s(s - a)}{\Delta}, \frac{s(s -
    b)}{\Delta}, \frac{s(s - c)}{\Delta}$ are in A.P.

  $\Rightarrow \frac{\Delta}{s(s - a)}, \frac{\Delta}{s(s - b)}, \frac{\Delta}{s(s - c)}$ are in H.P.

  $\Rightarrow \tan\frac{A}{2}, \tan\frac{B}{2}, \tan\frac{C}{2}$ are in H.P.
\item The ex-radii are given by $\frac{\Delta}{s - a}, \frac{\Delta}{s - b}$ and $\frac{\Delta}{s - c}$,
  which are in A.P.

  So $\frac{s - a}{\Delta}, \frac{s - b}{\Delta}, \frac{s - c}{\Delta}$ are in A.P.

  We know that if we multiply an A.P.\ by a constant then resulting series is in A.P. Multiplying given
  terms by $\Delta$, we have

  $s - a, s - b, s - c$ are in A.P. We also know that if we subtract a constant from A.P. then resulting
  series remains in A.P. So subtracting given terms from $s$ we have desired result.
\item Let $x$ be the first term and $y$ be the $2n - 1$st term. Then $n$th term of A.P.\ will be A.M.\ of
  $x$ and $y$, which is $a = \frac{x + y}{2}$; $n$th term of the G.P.\ will be G.M.\ of $x$ and $y$, which
  is $b = \sqrt{xy}$; and $n$th term of H.P.\ will be H.M.\ of $x$ and $y$ which is $c = \frac{2xy}{x + y}$.

  Clearly, $ac - b^2 = 0$.
\item Let $a, b, c$ are the roots; then from Vieta's relations we have $a + b + c = 12; ab + bc + ca = 39,
  abc = 28$.

  From first relation $3b = 12 \Rightarrow b = 4$ so the two other relations become $4a + 4c + ca = 39$ and
  $4ca = 28 \Rightarrow \ca = 7$

  $\Rightarrow a + c = 8 \Rightarrow a + \frac{7}{a} = 8 \Rightarrow a^2 - 8a + 7 = 0 \Rightarrow a = 1, 7$
  and $c = 7, 1$ so common difference is $\pm 3$.
\item Given, $(a^2 + b^2 + c^2)p^2 - 2(ab + bc + cd)p + (b^2 + c^2 + d^2)\leq 0$

  $\Rightarrow (a^2p^2 - 2abp + b^2) + (b^2p^2 - 2bcp + c^2) + (c^2p^2 - 2cdp + d^2)\leq 0$

  $\Rightarrow (ap - b)^2 + (bp - c)^2 + (cp - d)^2\leq 0$

  Clearly, only equality is possible as square of numbers cannot be negative. Hence, $p = \frac{b}{a}
  = \frac{c}{b} = \frac{d}{c}$, making $a, b, c, d$ form a G.P.
\item Let $d$ be the c.d.\ of A.P., then numbers are $6, 6 - d, 6 -2d, 3 - d$ such that $\frac{6 - 2d}{6 -
  d} = \frac{1}{2} \Rightarrow 12 - 4d = 6 - d \Rightarrow d = 2$.

  Thus, numbers are $6, 4, 2, 1$.
\item Given $\frac{44}{9} = 3 + 5r + 7r^2 + 9r^3 + \cdots$ to $\infty \Rightarrow \frac{44}{9}r = 3r + 5r^2
  + 7r^3 + \cdots$ to $\infty$

  Subtracting terms with corresponding powers of $r$, we get

  $\frac{44}{9}(1 - r) = 3 + 2r + 2r^2 + 2r^3 + \cdots $ to $\infty = 3 + \frac{2r}{1 - r} = \frac{3 - r}{1
    - r}$

  $\Rightarrow 44(1 - r)^2 = 27 - 9r \Rightarrow 44r^2 - 79r + 17 = 0 \Rightarrow (4r - 1)(11r - 17) = 0$

  For the given series $r$ has to be less than $|1|$ so $r = \frac{1}{4}$.
\item Let $S = 1^2 + 2^2x + 3^2x^2 + 4^2x^3 + \cdots$ to $\infty$. $\Rightarrow xS = 1^2x + 2^2x^2 + 3^2x^3
  + \cdots$ to $\infty$.

  Subtracting terms with corresponsding powers of $x$, we get

  $S(1 - x) = 1 + 3x + 5x^2 + 7x^3 + \cdots$ to $\infty$. $\Rightarrow Sx(1 - x) = x + 3x^2 + 5x^3 + \cdots$
  to $\infty$.

  Subtracting terms with corresponsding powers of $x$, we get

  $S(1 - x)^2 = 1 + 2x + 2x^2 + 2x^3 + \cdots$ to $\infty = 1 + \frac{2x}{1 - x} = \frac{1 + x}{1 - x}$

  $\Rightarrow S = \frac{1 + x}{(1 - x)^3}$.
\item $n$th term is given by $t_n = 1^2 + 2^2 + \cdots + n^2 = \frac{n(n + 1)(2n + 1)}{6} = \frac{2n^3 +
  3n^2 + n}{6}$

  Thus, sum is $S = \displaystyle\sum_{n = 0}^k\frac{2n^3 + 3n^2 + n}{6} = \frac{1}{3}\left(\frac{k(k +
    1)}{2}\right)^2 + \frac{1}{2}k(k + 1)(2k + 1) + \frac{k(k + 1)}{12}$

  Putting $k = 22$, we get sum as $S = 23276$.
\item Let $a$ be the first term and $d$ be the c.d. Then given $S_m:S_n = m^2:n^2$

  $\Rightarrow \frac{\frac{m}{2}[2a + (m - 1)d]}{\frac{n}{2}[2a + (n - 1)d]} = \frac{m^2}{n^2}$

  $\Rightarrow 2an + (m - 1)nd = 2am + (n - 1)md \Rightarrow 2a(m - n) = (m - n)d \Rightarrow a
  = \frac{d}{2}$

  $\Rightarrow \frac{t_m}{t_n} = \frac{a + (m - 1)d}{a + (n - 1)d} = \frac{2m - 1}{2n - 1}$.
\item Let $a$ be the first term and $d$ be the c.d.\ of the first A.P.\ and $x$ be the first term and $y$ be
  the c.d.\ of the second A.P. Also, let $S_n$ be the sum of $n$ terms of first A.P.\ and $t_{11}$ be the
  $11$th term of the first A.P., while $S_n'$ be the sum of the $n$ terms of the second A.P.\ and $t_{11}'$
  be the $11$th term of the second A.P.

  $S_n = (7n + 1)k = 2a + (n - 1)d \Rightarrow d = 7k, a = 4k$. $S_n' = (4n + 27)k = 2x + (n -
  1)y\Rightarrow y = 4k, x = \frac{31}{2}k$

  $\Rightarrow \frac{t_{11}}{t_{11}'} = \frac{a + (11 - 1)d}{x + (11 - 1)y} = \frac{4k + 70k}{\frac{31}{2}k +
    40k} = \frac{148k}{111k} = \frac{4}{3}$.
\item Let the distance between the stations be $x$ km. Total time taken $= \frac{x}{40} + \frac{x}{60}
  = \frac{x}{24}$ hours.

  $\therefore$ average speed $= \frac{2x}{\frac{x}{24}} = 48$ kmph.
\item Because $a, b, c$ are in A.P., therefore $2b = a + c$. Also, because $a^2, b^2, c^2$ are in H.P.,
  therefore $b^2 = \frac{2a^2c^2}{a^2 + c^2} \Rightarrow \frac{(a + c)^2}{4} = \frac{2a^2c^2}{a^2 + c^2}$

  $\Rightarrow a^4 + 2a^3c + 2ac^3 + 2a^2c^2 + c^4 = 8a^2c^2$, which gives $a = c$ upon solving making $a =
  b = c$.
\item $t_n = \frac{1}{1 + 2 + 3 + \cdots + n} = \frac{2}{n(n + 1)} = 2\left[\frac{1}{n} - \frac{1}{n +
    1}\right]$

  Thus, $S = 2.1 = 2$.
\item Let the roots be $\alpha$ and $\beta$. Then from question

  $\alpha + \beta = \frac{1}{\alpha^2} + \frac{1}{\beta^2} = \frac{(\alpha + \beta)^2 -
  2\alpha\beta}{\alpha^2\beta^2}$

  $\Rightarrow -\frac{b}{a} = \frac{\frac{b^2}{a^2} - 2\frac{c}{a}}{\frac{c^2}{a^2}} = \frac{b^2 -
  2ca}{c^2}$

  $\Rightarrow 2ca^2 - ab^2 = bc^2\Rightarrow ab^2, ca^2, bc^2$ are in A.P.
\item Let $a$ be the first term and $d$ be the c.d.\ of the A.P., then according to question

  $pt_p = qt_q$, where $t_p$ is the $p$th term and $t_Q$ be the $q$th term of the given A.P.

  $\Rightarrow ap + p(p - 1)d = aq + q(q - 1)d \Rightarrow a(p - q) = (q^2 - p^2)d + (p - q)d$

  $\Rightarrow a = d +(p + q)d = d(1 - p - q)$

  $\therefore t_{p + q} = a + (p + q - 1)d = 0$.
\item Let $r$ be the c.r.\ then $r = \frac{b}{a}$, and $c = ar^{n - 1}$, where $n$ is the number of terms.

  Sum of the series is $S = \frac{a(r^n - 1)}{r - 1} = \frac{c\frac{b}{a} - a}{\frac{b}{a} - 1} = \frac{bc -
    a^2}{b - a}$.
\item Let the triangle be $\triangle ABC$ such that $A$ is largest angle and $B$ is second largest. Given
  that angles are in A.P., $\Rightarrow a -d + a + a + d = 180^\circ \Rightarrow a = 60^\circ$ which would
  be $\angle B$. Let the unknown side be $x$, which is in front of $\angle C$.


  $\cos B = \frac{x^2 + 10^2 - 9^2}{20x} = \frac{1}{2}\Rightarrow x = 5\pm\sqrt{6}$.
\item Let the quantities be $a$ and $b$, then $G = \sqrt{ab}$. Let $d$ be the common difference then $d
  = \frac{b - a}{3}\Rightarrow p = \frac{2a + b}{3}, q = \frac{a + 2b}{3}$.

  $2p - q = a, 2q - p = b \Rightarrow G^2 = (2p - q)(2q - p)$.
\item Let $d$ be the common difference, then $b - a = c - b = a = d$. Thus, $b = 2d, c = 3d$. Hence, $a:b:c
  = 1:2:3$.
\item Let $d$ be the c.d.\ and $n$ be the number of terms then $l = a + (n - 1)d \Rightarrow n = \frac{l -
  a}{d} + 1$.

  $s = \frac{n}{2}[a + l] \Rightarrow 2s = \left(\frac{l - a}{d} + 1\right)(a + l)$

  $\Rightarrow \frac{2s}{a + l} = \frac{l - a}{d} + 1 \Rightarrow \frac{2s - a - l}{a + l} = \frac{l -
  a}{d} \Rightarrow d = \frac{l^2 - a^2}{2s - (l + a)}$.
\item Let $a$ be the first term and $d$ be the c.d.\ of the A.P., then

  $\frac{1}{n} = a + (m - 1)d$ and $\frac{1}{m} = a + (n - 1)d$

  Subtracting we have $\frac{m - n}{mn} = (m - n)d \Rightarrow d = \frac{1}{mn} \Rightarrow a = \frac{1}{n}
  - \frac{m - 1}{mn} = \frac{1}{mn}$.

  $\Rightarrow T_{mn} = \frac{1}{mn} + (mn - 1).\frac{1}{mn} = 1$.
\item Given $\frac{\cos\theta}{\sin\theta} = \frac{\tan\theta}{\cos\theta} \Rightarrow \sin^2\theta
  = \cos^3\theta$

  $\cot^6\theta - \cot^2\theta = \cot^2\theta(\cot^2\theta - 1)(\cot^2\theta + 1)
  = \frac{1}{\cos\theta}\left(\frac{1 - \cos\theta}{\cos\theta}\right)\left(\frac{1
    + \cos\theta}{\cos\theta}\right)$

  $= 1$.
\item Given $a + b + c = k(\sin A + \sin B + \sin C) = 6\frac{\sin A + \sin B + \sin C}{3} \Rightarrow k =
  2$.

  $\Rightarrow \frac{a}{\sin A} = \frac{1}{2} \Rightarrow A = 30^\circ$.
\item Given, $1.1! + 2.2! + 3.3! + \cdots + n.n! = (2 - 1).1! + (3 - 1)2! + (4 - 1)3! + \cdots + (n + 1 -
  1)n!$

  $= 2! - 1! + 3! - 2! + 4! - 3! + \cdots + (n + 1)! - n! = (n + 1)! - 1$.
\item Let $y = x + d$ and $x = a + 2d$. Then, $\frac{1}{\sqrt{y} + \sqrt{x}} = \frac{\sqrt{y}
  - \sqrt{x}}{d}, \frac{1}{\sqrt{z} + \sqrt{x}} = \frac{\sqrt{z} - \sqrt{x}}{2d}, \frac{1}{\sqrt{z}
  + \sqrt{y}} = \frac{\sqrt{z} - \sqrt{y}}{d}$.

  Now $\frac{1}{\sqrt{z} + \sqrt{x}} - \frac{1}{\sqrt{y} + \sqrt{x}} = \frac{\sqrt{z} - \sqrt{x} - 2\sqrt{y}
    + 2\sqrt{x}}{2d}$

  and $\frac{1}{\sqrt{z} + \sqrt{y}} - \frac{1}{\sqrt{z} + \sqrt{x}} = \frac{2\sqrt{z} - 2\sqrt{y}
    - \sqrt{z} + \sqrt{x}}{2d}$

  Hence the required terms are in A.P. as well.
\item $S = 1 + \frac{1.3}{3} + \frac{3.5}{2!.3^2} + \frac{3.5.7}{3!3^3} + \cdots$

  $= 1 + \frac{3}{1!}.\frac{1}{3} + \frac{3(3 + 2)}{2!}.\frac{1}{3^2} + \frac{3(3 + 2)(3 +
  2.2)}{3!}.\frac{1}{3^3} + \cdots$

  $= \left(1 - \frac{2}{3}\right)^{-\frac{3}{2}} = 3\sqrt{3}$.
\item $S = \frac{1}{1 - \cos\alpha} = 2 - \sqrt{2} \Rightarrow 1 - \cos\alpha = \frac{1}{2 - \sqrt{2}}
  = \frac{2 + \sqrt{2}}{4 - 2} = 1 + \frac{1}{\sqrt{2}}\Rightarrow \cos\alpha = -\frac{1}{\sqrt{2}}$

  $\Rightarrow \alpha = \frac{3\pi}{4}$.
\item Let $x_1$ and $x_2$ be roots of the equation. Then H.M. $= \frac{2x_1x_2}{x_1 + x_2} = \frac{2.\frac{8
    + 2\sqrt{5}}{5 + \sqrt{2}}}{\frac{4 + \sqrt{5}}{5 + \sqrt{2}}} = \frac{16 + 4\sqrt{5}}{4 + \sqrt{5}} = 4$.
\item Applying $C_3\rightarrow C_3 - (C_1 + C_2)$, we have

  $\startdeterminant\NC a\NC b\NC 0\NR\NC b\NC c\NC 0\NR\NC a + b\NC b + c\NC -(a + 2b + c)\NR\stopdeterminant$

  Expanding against $3$rd row and $3$rd column, we have

  $(a + 2b + c)[ac - b^2] = 0 \Rightarrow ac = b^2$, and thus, $a, b, c$ are in G.P.
\item We have $225a^2 + 9b^2 + 25c^2 - 75ac - 45ab - 15bc = 0\Rightarrow \frac{1}{2}[(15a - 3b)^2 + (3b -
  5c)^2 + (5c - 15a)^2] = 0$

  $\therefore 15a = 3b = 5c \Rightarrow \frac{a}{1} = \frac{b}{5} = \frac{c}{3} = \lambda$

  $\therefore b, c, a$ are in A.P.
\item $m$th term of the first progression $= 1 + (m - 1)3 = 3m - 2$. $n$th term of the second progression $=
  2 + (n - 1)5 = 5n - 3$. $r$th term of the third progression $= 3 + (r - 1)7 = 7r - 4$.

  Terms between the first and the second progression will be same when $m = \frac{5n - 1}{3}\Rightarrow n =
  2, 5, 11, \ldots$. Similarly, the terms between the second and the third progression will be same when $r
  = \frac{5n + 1}{7}\Rightarrow n = 4, 11, \ldots$.

  So the first common term will be $n = 11\Rightarrow a = 2 + (11 - 1)5 = 52$. Common difference will be LCM
  of common differences which is $105$.
\item Let the sides be $a - d, a, a + d$. Then from area $a(a - d) = 48$ and $(a - d)^2 + a^2 = (a +
  d)^2\Rightarrow a = 4d \Rightarrow a = 8, d = 2$.

  Thus, smallest side is $a - d = 6$.
\item We have $[-x] = -[x] - 1$ if $x\not\in\mathbb{I}$ and $[x] + \left[x + \frac{1}{n}\right] + \left[x
  + \frac{2}{n}\right] + \cdots + \left[x + \frac{n - 1}{n}\right] = [nx]$, where $n\in{N}$.

  So given series $= -1\times 100 - \left[\frac{1}{3}\times 100\right] = -133$.
\item Sum of first $n$ terms is given by $\frac{n}{2}[2a + (n - 1)d]$, where each term has the usual
  meaing. Given that this sum is $50n + \frac{n(n - 7)}{2}A$.

  Comparing corresponding terms we have $a = 50 - 3A$ and $d = A$.
\item Numbers which leave a remainder of $2$ are given by $7n + 2$. Smallest two digit number is
  $16$. Largest such number is $93$. No.\ of such numbers is $12$.

  Thus, sum is $\frac{12}{2}[16 + 93] = 654$.

  Similarly, number which leaves a remainder of $5$ are given by $7n + 5$. Smallest two digit number is
  $12$. Largest such number is $96$. No.\ of such numbers is $13$.

  Thus, sum is $\frac{13}{2}[12 + 96] = 702$. Thus, total sum is $1356$.
\item Let $d$ be the c.d., then $S = 15[2a_1 + 29d], T = \frac{15}{2}[2a_1 + 28d] \Rightarrow 2T = 15[2a_1 +
  28d]$.

  $\therefore S - 2T = 15d = 75 \Rightarrow d = 5$. $a_{10} = a_5 + 5d = 27 + 25 = 52$.
\item If $\log_eb_1, \log_eb_2, \ldots, \log_eb_{101}$ are in A.P. with common difference of $\log_e2$, then
  $b_1, b_2, \ldots, b_{101}$ are in G.P., with a common ratio of $2$.

  $\Rightarrow b_2 = 2b_1, b_3 = 2^2b_1, \ldots, b_{101} = 2^{100}b_1$.

  Also given that $a_1, a_2, \ldots, a_{101}$ are in A.P. Given $a_1 = b_1$ and $a_{51} = b_{51}\Rightarrow
  a_1 + 50d = 2^{50}b_1 = 2^{50}a_1$.

  $t = b_1 + b_2 + \cdots + b_{51} = b_1(2^{51} - 1) < 2^{51}a_1, s = a_1 + a_2 + \cdots + a_{51}
  = \frac{51}{2}[2a_1 + 50d] = \frac{51}{2}[a_1 + a_1 + 50d] = \frac{51}{2}[a_1 + 2^{20}a_1] > 2^{51}a_1$

  Hence, $s > t$. Now, $a_{101} = a_1 + 100d = a_1 + 100.\left(\frac{2^{50}a_1 - a_1}{50}\right), b_{101} =
  2^{100}b_1 = 2^{100}a_1$

  $\Rightarrow a_{101} = a_1 + 2^{51}a_1 - 2a_1 = 2^{51}a_1 - a_1$, and thus, $a_{101} < b_{101}$.
\item Let $S_n = cn^2 \Rightarrow t_n = S_n - S_{n - 1} = 2cn - c\Rightarrow t_n^2 = 4c^2n^2 + c^2 - 4c^2n$

  Sum of squares of terms is $\sum t_n^2 = 4c^2.\frac{n(n + 1)(2n + 1)}{6} + c^2n - 4c^2.\frac{n(n + 1)}{2}
  = \frac{nc^2(4n^2 - 1)}{3}$.
\item $V_r = \frac{r}{2}[2r + (r - 1)(2r - 1)] = \frac{1}{2}(2r^3 - r^2 + r)$

  $\sum V_r = \frac{1}{2}\left[2\left(\frac{n(n + 1)}{2}\right)^2 - \frac{n(n + 1)(2n + 1)}{6} + \frac{n(n +
    1)}{2}\right] = \frac{1}{12}n(n + 1)(3n^2 + n + 2)$.

  $T_r = V_{r + 1} - V_r = (r + 1)(3r - 1)$, which is a composite number.

  $Q_r = T_{r + 1} - T_r = 6r + 5$, therefore common difference is $6$.
\item From given conditions we write equations for sum of roots and product of roots as $p + q = 2, pq = A,
  r + s = 18, rs = B$.

  Also, given that $p, q, r, s$ form an A.P., so let $p = a - 3d, q = a - d, r = a + d, s = a + 3d$.

  Adding $4a = p + q + r + s = 20 \Rightarrow a = 5 \Rightarrow d = 2$ so $p, q, r, s$ are $-1, 3, 7, 11$
  respectively. Therefore, $A = -3, B = 77$.
\item Given $1^2 + 2.2^2 + 3^2 + 2.4^2 + 5^2 + \cdots$ upto $n$ terms $= \frac{n(n + 1)^2}{2}$, when $n$ is
  even.

  When $n$ is odd the series becomes $1^2 + 2.2^2 + 3^2 + 2.4^2 + 5^2 + \cdots + 2(n - 1)^2 + n^2
  = \frac{n(n  1)^2}{2} + n^2 = \frac{n^2(n + 1)}{2}$.
\item Let the numbers are $a - 3d, a - d, a + d, a - 3d$. All these will be integers and also $2d$ will be
  an integer.

  Now $(a - 3d)^2(a - d)^2(a + d^2)(a + 3d)^2 + (2d)^4 = a^4 - 10a^2d^2 + 9d^4 + 16d^4 = a^4 - 10a^2d^2 +
  25d^4 = (a^2 - 5d^2)^2$.

  Now $a^2 - 5d^2 = a^2 - 9d^2 + 4d^2 = (a - 3d)(a + 3d) + (2d)^2$. All the terms in this expression are
  integers, and hence proven.
\item From Vieta's relation $\sum x_i = 1, \sum x_ix_j = \beta$, and $\prod x_i = -\gamma$

  Let $x_1 = a - d, x_2 = a$, and $x_3 = a + d$. $\Rightarrow 3a = 1\Rightarrow a = \frac{1}{3}$.

  From second relation $(a - d)a + (a + d)a + (a + d)(a - d) = \beta \Rightarrow 3a^2 - d^2 = \beta$

  $\Rightarrow \frac{1}{3} - \beta = d^2 \Rightarrow \frac{1}{3} - \beta \geq
  0 \Rightarrow \beta \leq \frac{1}{3}$.

  From third relation $a(a^2 - d^2) = -\gamma \Rightarrow \frac{1}{3}\left(\frac{1}{9} - d^2\right)
  = \gamma$

  $\Rightarrow \gamma + \frac{1}{27} = \frac{d^2}{3}\Rightarrow \gamma\geq -\frac{1}{27}$.
\item Given, $\frac{S_7}{S_{11}} = \frac{6}{11} \Rightarrow \frac{7(a + 6d)}{a + 10d} = 6 \Rightarrow a =
  9d$.

  Also given, $130 < t_7 < 140 \Rightarrow 130 < a + 6d < 140 \Rightarrow 130 < 15d < 140 \Rightarrow d =
  9$.
\item From question $\frac{n(n + 1)}{2} - k - k - 1 = 1224 \Rightarrow n^2 + n - 2450 = 4k\Rightarrow (n +
  50)(n - 49) = 4k$

  $\therefore n > 49$. Let $n = 50\Rightarrow k = 25$.
\item $\frac{S_m}{S_n} = \frac{S_{5n}}{S_n} = \frac{5n[2a_1 + (5n - 1)d]}{n[2a_1 + (n - 1)d]} = \frac{5[(6 -
    d) + 5n]}{6 - d + n}$

  If $6 - d = 0$ then the fraction is independent of $n$. Hence, $d = 6$.
\item Given, $a_k = 2a_{k - 1} - a_{k - 2}\Rightarrow a_1, a_2, a_3, \ldots, a_{11}$ are in an A.P.

  $\Rightarrow \frac{a_1^2 + a_2^2 + \cdots + a_{11}^2}{90} = \frac{11a^2 + 35\times11d^2 + 10ad}{11} = 90$

  $\Rightarrow 35d^2 + 150d + 135 = 0 \Rightarrow d = -3, -\frac{9}{7}$. Given $a_2 < \frac{27}{2}$ implies
  $d = -3$ and $d\neq -\frac{9}{7}$.

  $\Rightarrow \frac{a_1 + a_2 + \cdots + a_{11}}{11} = \frac{11}{2}[30 - 10\times 3] = 0$.
\item Clearly $b = ar$ and $c = ar^2$. Also, because $3a, 7b$ and $15c$ are in A.P., therefore

  $14b = 3a + 15c\Rightarrow 14r = 3 + 15r^2 \Rightarrow r = \frac{1}{3}$ from given condition on $r$.

  Thus, common difference is $7b - 3a = -\frac{2a}{3}$.

  So, the $4$th terms of the A.P.\ is $3a + 3.-\frac{2a}{3} = a$.
\item Let $r$ be the c.r.\ of the G.P., then $a + ar + ar^2 = xar \Rightarrow r^2 + (1 - x)r + 1 = 0$

  Because $x$ is real the discriminant of the above equation must be greater that zero.

  $\Rightarrow (1 - x)^2 - 4\geq 0 \Rightarrow x^2 - 2x - 3 \geq 0 \Rightarrow (x - 3)(x + 1)\geq 0$

  $\Rightarrow x \leq -1$ and $x\geq 3$.
\item Given, $(\alpha^2 + \beta^2)^2 = (\alpha + \beta)(\alpha^3 + \beta^3)\Rightarrow \alpha^4 + \beta^4 +
  2\alpha^2\beta^2 = \alpha^4 + \beta^4 + \alpha^3\beta + \alpha\beta^3$

  $\Rightarrow \alpha^2\beta(\alpha - \beta) + \alpha\beta^2(\beta - \alpha) = 0$

  Thus, $\alpha = \beta\Rightarrow \Delta = 0$.
\item Let $S = (x + y) + (x^2 + xy + y^2) + (x^3 + x^2y + xy^2 + y^2) + \cdots$

  $(x - y)S = (x^2 - y^2) + (x^3 - y^3) + (x^4 - y^4) + \cdots = (x^2 + x^3 + x^4 + \cdots) - (y^2 + y^3 +
  y^4 + \cdots)$

  $= \frac{x^2}{1 - x} - \frac{y^2}{1 - y}\Rightarrow S = \frac{x + y - xy}{(1 - x)(1 - y)}$.
\item $S_n = \frac{1 - q^n}{1 - q}$ and $T_n = \frac{1 - \left(\frac{q + 1}{2}\right)^n}{1 - \frac{q +
    1}{2}} = \frac{1 - (q + 1)^n}{2^{n - 1}.(1 - q)}$.

  L.H.S. $= \frac{1}{1 - q}[C_1^^{101} + C_2^^{101} + \cdots + C_{101}^^{101} - C_1^^{101}q +
    C_2^^{101}q^2 + \cdots + C_{101}^^{101}q^{101}] = \alpha.T_{100}$

  $\Rightarrow \frac{1}{1 - q}[(2^{101} - 1) - (1 + q)^{101} + 1] = \alpha.T_{100}$

  On simplification we get $\Rightarrow \alpha = 2^{100}$.
\item $k = 1 + 2.\left(\frac{11}{10}\right) + 3.\left(\frac{11}{10}\right)^2 + \cdots +
  10.\left(\frac{11}{10}\right)^9$

  $\frac{11}{10}k = 1.\frac{11}{10} + 2.\left(\frac{11}{10}\right)^2 + 3.\left(\frac{11}{10}\right)^3
  + \cdots + 10.\left(\frac{11}{10}\right)^{10}$

  $\Rightarrow k\left(1 - \frac{11}{10}\right) = 1 + \frac{11}{10} + \left(\frac{11}{10}\right)^2 + \cdots +
  \left(\frac{11}{10}\right)^9 - \left(\frac{11}{10}\right)^{10}$

  On simplification we get $k = 100$.
\item Let us prove these one by one.
  \startitemize[a]
  \item Roots of the given equation are $\alpha = \frac{-1 + \sqrt{5}}{2}$ and $\beta = \frac{-1
    - \sqrt{5}}{2}$. $\Rightarrow \alpha + \beta = 1, \alpha\beta = -1$.

    $a_n = \frac{\alpha^n - \beta^n}{\alpha - beta} \Rightarrow a_{n + 1} = \frac{\alpha^{n + 1} - \beta^{n
      + 1}}{\alpha - \beta}$

    $a_{n + 1} = \alpha^n + \alpha^{n - 1}\beta + \alpha^{n - 2}\beta^2 + \cdots + \alpha\beta^{n - 1}
    + \beta^n$

    $= \alpha^n + \beta^n - \left(\alpha^{n - 2} + \alpha^{n - 3}\beta + \cdots + \beta^{n -
      2}\right)[\because \alpha\beta = -1]$

    $\Rightarrow a_{n + 1} + a_{n - 1} = \alpha^n + \beta^n = b_n$.
  \item $\because \alpha^2 = \alpha + 1$ and $\beta^2 = \beta + 1$

    $\Rightarrow alpha^{n + 2} = \alpha^{n + 1} + \alpha^n$ and $\beta^{n + 2} = \beta^{n + 1} + \beta^n$

    $\alpha^{n + 2}\beta^{n + 2} = (\alpha^{n + 1} + \beta^{n + 1}) + (\alpha^n + \beta^n)$

    $\Rightarrow a_{n + 2} = a_{n + 1} + a_n\Rightarrow a_{n + 1} = a_n + a_{n - 1} \cdots a_3 = a_2 + a_1$

    $\Rightarrow a_{n + 2} = (a_n + a_{n - 1} + \cdots + a_2 + a_1) + a_2[\because a_2 = \frac{\alpha^2
      - \beta^2}{\alpha - \beta} = 1]$

    $\Rightarrow a_1 + a_2 + \cdots + a_n = a_{n + 2} - 1$.
  \item $\displaystyle\sum_{n = 1}^\infty\frac{a_n}{10^n} = \sum_{n = 1}^\infty\frac{\alpha^n
    - \beta^n}{(\alpha -\beta)10^n} = \frac{1}{\alpha - \beta}\sum_{n =
    1}^\infty\left[\left(\frac{\alpha}{10}\right)^n - \left(\frac{\beta}{10}\right)^n\right]$

    On simplification it gives the value $\frac{10}{89}$.
  \stopitemize
\item We have to find an odd $n$ such that $B_n = 1 - A_n > A_n \Rightarrow A_n < \frac{1}{2}$.

  $A_n = \frac{3}{4}\frac{\left[1 - \left(-\frac{3}{4}\right)^n\right]}{1
  + \frac{3}{4}}\Rightarrow \left(-\frac{3}{4}\right)^n > -\frac{1}{6}$.

  For all even values of $n$ required condition is fulfilled, however, we have to find odd values for it.

  We see that $n = 7, \left(-\frac{3}{4}\right)^7 = -\frac{2187}{12288} < -\frac{}{6}$, which is minimum
  odd value for $n$ for satisfying the required condition.
\item Given, $S_k = \frac{\frac{k - 1}{k!}}{1- \frac{1}{k}} = \frac{1}{(k - 1)!}$

  Now, $(k^2 - 3k + 1)S_k = [(k - 2)(k - 1) - 1]\frac{1}{(k - 1)!} = \frac{1}{(k - 3)!} - \frac{1}{(k -
    1)!}$

  $\Rightarrow \displaystyle\sum_{k = 1}^{100}|(k^2 - 3k + 1)S_k| = 1 + 1 + 2 - \left(\frac{1}{99!}
  + \frac{1}{98!}\right) = 4 - \frac{100^2}{100!}$.

  Thus, $\frac{100^2}{100!} + \displaystyle\sum_{k = 1}^{100}|(k^2 - 3k + 1)S_k| = 4$.
\item From given condition $\cos x = \frac{2\cos(x - y)\cos(x + y)}{\cos(x - y) + \cos(x + y)}$

  $\Rightarrow \cos x(2\cos x\cos y) = 2[\cos^2x - \sin^2y]\Rightarrow \cos^2x(1 - \cos y) = \sin^2y$

  $\Rightarrow \cos^2x.2\sin^2\frac{y}{2} = 4\sin^2\frac{y}{2}\cos^2\frac{y}{2}\Rightarrow \cos
  x\sec\frac{y}{2} = \pm\sqrt{2}$.
\item $\displaystyle\sum_{n = 1}^{15} = \frac{1^3 + 2^3 + \cdots + n^3}{1 + 2 + \cdots + n} = \sum_{n =
  1}^{15}\frac{\left(\frac{n(n + 1)}{2}\right)^2}{\frac{n(n + 1)}{2}}$

  $= \frac{1}{2}\displaystyle\sum_{n = 1}^{15}(n^2 + n) = \frac{1}{2}\left[\frac{15\times16\times 31}{6}
  + \frac{15\times 16}{2}\right] = 680$

  $\frac{1}{2}(1 + 2 + \cdots + 15) = \frac{15\times 16}{4} = 60$.

  $\therefore$ Required sum is $680 - 60 = 620$.
\item $t_n = \frac{(2n + 1)\left(\frac{n(n + 1)}{2}\right)^2}{\frac{n(n + 1)(2n + 1)}{6}} = \frac{3}{2}n(n +
  1)$

  $S_{10} = \frac{3}{2}\displaystyle\sum_{n = 1}^^{10}(n^2 + n) = \frac{3}{2}\left[\frac{10\times 11\times
      21}{6} + \frac{10\times 11}{2}\right] = 660$.
\item Let $S = \left(\frac{3}{4}\right)^3 + \left(\frac{6}{4}\right)^3 + \left(\frac{9}{4}\right)^3
  + \left(\frac{12}{4}\right)^3 + \cdots$ upto $15$ terms

  $= \left(\frac{3}{4}\right)^3\left[1^3 + 2^3 + \cdots + 15^3\right]
  = \left(\frac{3}{4}\right)^3\left(\frac{15\times 16}{2}\right)^2 = 27\times 225$

  Thus, $k = 27$.
\item $S_k = \frac{k + 1}{2}\Rightarrow S_k^2 = \frac{k^2 + 2k + 1}{4}$

  $\therefore S_1^2 + S_2^2 + \cdots + S_{10}^2 = \displaystyle\sum_{k = 1}^{10}\left(\frac{k^2 + 2k +
  1}{4}\right)$

  $= \frac{1}{4}\left[\frac{10\times 11\times 21}{6} + 10\times 11 + 10\right] = \frac{505}{4}\Rightarrow A
  = 303$.
\item The expression is $\frac{x^my^n}{\left(1 + x^{2m}\right)\left(1 + y^{2n}\right)} = \frac{1}{\left(x^m
  + x^{-m}\right)\left(y^n + y^{-n}\right)}$

  Since $x, y$ are real numbers and $m, n$ are positive integers so using A.M.-G.M.\ inequality we have

  $x^m + x^{-m}\geq 2$ and $y^n + y^{-n}\geq 2$. So maximum value of given expression is $\frac{1}{4}$.
\item $t_n = \frac{3n(1^2 + 2^2 + \cdots + n^2)}{2n + 1} = \frac{3n.n(n + 1)(2n + 1)}{6(2n + 1)}
  = \frac{n^2(n + 1)}{2}$

  $\therefore S_n = \frac{1}{2}\left[\left(\frac{n(n + 1)}{2}\right)^2 + \frac{n(n + 1)(2n + 1)}{6}\right]$

  $\Rightarrow S_{15} = 7820$.
\item Let $a$ be the first term and $d$ be the c.d.\ of the given A.P. Given $\displaystyle\sum_{k =
  0}^{12}a_{4k + 1} = 416$ and $a_9 + a_{43} = 66$.

  $\Rightarrow \frac{13}{2}[2a + 48d] = 416 \Rightarrow a + 24d = 32$ and $2a + 50d = 66 \Rightarrow a + 25d
  = 33$

  Thus, $d = 1, a = 8$ from the two equations.

  Given $a_1^2 + a_2^2 + \cdots + a_{17}^2 = 140m \Rightarrow 8^2 + 9^2 + \cdots + 24^2 = 140m$

  $\Rightarrow (1^2 + 2^2 + \cdots + 24^2) - (1^2 + 2^2 + \cdots + 7^2) = 140m\Rightarrow \frac{24.25.49}{6}
  - \frac{7.8.15}{6} = 140m\Rightarrow m = 34$.
\item $A = (1^2 + 2^2 + \cdots + 20^2) + (2^2 + 4^2 + \cdots + 20^2) = \frac{20\times21\times41}{6}
  + \frac{4\times10\times11\times 21}{6} = \frac{20\times41\times 63}{6}$

  $B = (1^2 + 2^2 + \cdots + 40^2) + (2^2 + 4^2 + \cdots + 40^2) = \frac{40\times\41\times 123}{6}$.

  Solving the equation $B - 2A = 100\lambda$ gives us $\lambda = 248$.
\item $S_{10} = \left(\frac{8}{5}\right)^2 + \left(\frac{12}{5}\right)^2 + \left(\frac{16}{5}\right)^2
  + \left(\frac{20}{5}\right)^2 + \cdots$ upto $10$ terms

  $= \frac{4^2}{5^2}[2^2 + 3^2 + \cdots + 10^2] = \frac{4^2}{5^2}\left[\frac{10\times11\times21}{6} -
    1\right] = \frac{16}{25}\times505 = \frac{16m}{5}\Rightarrow m = 101$.
\item Given, $m$ is the A.M.\ of $l$ and $n\Rightarrow l + n = 2m$.

  $l, G_1, G_2, G_3, n$ are in G.P. Let $r$ be the c.r.\ of the G.P. $\Rightarrow r =
  \left(\frac{n}{l}\right)^{\frac{1}{4}}$.

  $\therefore G_1^4 + 2G_2^4 + G_3^4 = 4lm^2n$.
\item $t_n = \frac{1^3 + 2^3 + \cdots + n^3}{1 + 3 + \cdots + (2n - 1)} = \left(\frac{n + 1}{2}\right)^2
  = \frac{n^2 + 2n + 1}{4}$

  $S_n = \frac{1}{4}\left[\frac{n(n + 1)(2n + 1)}{6} + n(n + 1) + n\right] \Rightarrow S_{9} = 96$.
\item Using A.M.-G.M.\ inequality we have

  $\frac{\sqrt{x^2 + x} + \frac{\tan^2\alpha}{\sqrt{x^2 + x}}}{2}\geq \tan\alpha\Rightarrow \sqrt{x^2 + x}
  + \frac{\tan^2\alpha}{\sqrt{x^2 + x}}\geq 2\tan\alpha$.
\item Given $a_1a_2\ldots a_n = c\Rightarrow a_1a_2\ldots a_{2n - 1}(2a_n) = 2c$

  $\therefore \frac{a_1 + a_2 + \cdots + 2a_n}{n}\geq\sqrt[n]{a_1a_2\ldots(2a_n)} = \sqrt[n]{2c}$

  Therefore, minimum value is $n\sqrt[n]{2c}$.
\item Using A.M.-G.M.\ relationshiop, we have $\frac{(a + b) + (c + d)}{2}\geq \sqrt{(a + b)(c + d)}
  \Rightarrow M\leq 1$. Also, $(a + b + c + d) > 0 \Rightarrow M > 0$.
\item Let the roots be $\alpha$ and $\beta$, then we have $\alpha + \beta = \frac{4 + \sqrt{5}}{5 +
  \sqrt{2}}$ and $\alpha\beta = \frac{8 + 2\sqrt{5}}{5 + \sqrt{2}}$.


  The harmonic mean is given by $\frac{2\alpha\beta}{\alpha + \beta} = \frac{2\left(8 + 2\sqrt{5}\right)}{4
    + \sqrt{5}} = 4$.
\item Using A.M.-G.M.\ between $(a + b - c)$ and $(b + c - a)$, we have

  $\frac{a + b - c + b + c - a}{2}\geq \sqrt{(a + b - c)(b + c - a)}\Rightarrow b\geq \sqrt{(a + b - c)(b +
  c - a)}$.

  Similarly, $c\geq \sqrt{(b + c - a)(c + a - b)}$ and $a\geq \sqrt{(c + a - b)(a + b - c)}$

  Multiplying the three relationships $abc \geq (a + b - c)(b + c - a)(c + a - b)$, and hence proven.
\item We have $(1 + a)(1 + b)(1 + c) = 1 + a + b + c + ab + bc + ca + abc$

  Using A.M.-G.M.\ inequality, we have $\frac{1 + a + b + c + ab + bc + ca + abc}{7}\geq
  \sqrt[7]{a^4b^4c^4}$

  $\Rightarrow 1 + a + b + c + ab + bc + ca + abc\geq 7\sqrt[7]{a^4b^4c^4}$

  $\Rightarrow \left[(1 + a)(1 + b)(1 + c)\right]^7 > 7^7a^4b^4c^4$.
\item Let $G_m$ be the G.M.\ of $G_1, G_2, \ldots, G_n$.

  $\Rightarrow G_m = \left(G_1G_2\ldots G_n\right)^{1/n} =
  \left[(a_1).(a_1a_1r)^{1/2}(a_1a_1ra_1r^2)^{1/3}\ldots (a_1a_1r\ldots a_1r^{n - 1})^{1/n}\right]$, where
  $r$ is the common ratio of the G.P.

  $= \left[a_1^nr^{\frac{1}{2} + 1 + \frac{3}{2} + \cdots + \frac{n - 1}{2}}\right]^{\frac{1}{n}} =
  a_1r^{\frac{n - 1}{4}}$

  $A_n = \frac{a_1 + a_2 + \cdots + a_n}{n} = \frac{a_1(1 - r^n)}{n(1 - r)}$

  $H_n = \frac{n}{\frac{1}{a_1} + \frac{1}{a_2} + \cdots + \frac{1}{a_n}} = \frac{a_1n(1 - r)^{n - 1}}{1 -
    r^n}$

  $A_nH_n = a_1^2r^{n - 1}\Rightarrow \displaystyle\prod_{i = 1}^nA_iH_i = \prod_{i = 1}^n(a_1^2r^{n - 1}) =
  \left[a_1r^{\frac{n - 1}{4}}\right]^{2n} = G_m^{2n}$

  $\Rightarrow G_m = \left[\displaystyle\prod_{i= 1}^nA_iH_i\right]^{1/2n}$.
\item Since $y_1, y_2, y_3$ are real number so $3^{y_1}, 3^{y_2}, 3^{y_3}$ will be real numbers. Using
  A.M.-G.M.\ inequality, we have

  $\frac{3^{y_1} + 3^{y_2} + 3^{y_3}}{3}\geq \left(3^{y_1}.3^{y_2}.3^{y_3}\right)^{\frac{1}{3}}\Rightarrow
  3^{y_1} + 3^{y_2} + 3^{y_3} \geq 3\left(3^{y_1}.3^{y_2}.3^{y_3}\right)^{\frac{1}{3}}$

  Taking $\log$ of both sides with base $3$, we have

  $\log_3\left(3^{y_1} + 3^{y_2} + 3^{y_3}\right)\geq \left[1 + \frac{1}{3}(y_1 + y_2 + y_3)\right] = 4$

  Thus, $m = 4$.

  Using A.M.-G.M.\ inequality for $x_1, x_2, x_3$, we have

  $\frac{x_1 + x_2 + x_3}{3}\geq \left(x_1 + x_2 + x_3\right)^{\frac{1}{3}}$

  Taking $\log$ of both sides with base $3$, we have

  $\log_3\left(\frac{x_1 + x_2 + x_3}{3}\right)\geq \frac{1}{3}\left(\log_3 x_1 + \log_3x_2 +
  \log_3x_3\right)$

  $\Rightarrow M = 3$.

  Now, $\log_2m^3 + \log_3M^2 = 3\log_24 + 2\log_33 = 8$.
\item Given equality is $2(a_1 + a_2 + \cdots + a_n) = b_1 + b_2 + \cdots + b_n$

  $\Rightarrow 2.\frac{n}{2}[2c + (n - 1)2] = c\left(2^n- 1\right)$

  $\Rightarrow c\left[2^n - 2n - 1\right] = 2n^2 - 2n$

  $\because c\in\mathbb{N}\Rightarrow 2n^2 - 2n \geq 2^n - 2n - 1\Rightarrow n\leq 6$ and also $c >
  0\Rightarrow n > 2$

  The possible values of $n$ are $3, 4, 5, 6$.

  Only for $n = 3, c$ has integral value of $12$ and for other $n$ it has fractional value. So $c =12, n =
  3$.
\item Given $a, b, c$ are in G.P. Let $r$ be the c.r.\ of this G.P. Given that A.M.\ of $a, b, c$ is $b +
  2$. Thus,

  $\frac{a + b + c}{3} = b + 2 \Rightarrow a + ar + ar^2 = 3ar + 6 \Rightarrow (r - 1)^2 = \frac{6}{a}$.

  Since $\frac{6}{a}$ must be a perfect square and $a\in\mathbb{N}$, so $a$ can be $6$ only.

  $\Rightarrow r - 1 = \pm 1\Rightarrow r = 2$.

  $\Rightarrow \frac{a^2 + a - 14}{a + 1} = 4$.
\item Using A.M.-G.M., we have

  $\frac{a^{-5} + a^{-4} + 3a^{-3} + 1 + a^8 + a^{10}}{8}\geq
  \left(a^{-5}.a^{-4}.(a^{-3})^3.a^8.a^{10}\right)^{\frac{1}{8}}$

  $\Rightarrow a^{-5} + a^{-4} + 3a^{-3} + 1 + a^8 + a^{10}\geq 8.1 = 8$. Hence, minimum value is $8$.
\item Given, $a_1 = b_1 = 1$. $a_2 = a_1 + 2 = 3, a_3 = a_2 + 2 = 5\Rightarrow a_n = 2n - 1$. $b_2 = a_2 +
  b_1 = 4, b_3 = a_3 + b_2 = 9\Rightarrow b_n = n^2$

  $\displaystyle\sum_{i = 1}^{15}a_n.b_n = \sum_{i = 1}^{15}\left(2n^3 - n^2\right) = 2.\frac{n^2(n +
    1)^2}{4} - \frac{n(n + 1)(2n + 1)}{6} = 27560$.
\item Given, $\frac{1}{2.3^{10}} + \frac{1}{2^2.3^9} + \cdots + \frac{1}{2^{10}.3} =
  \frac{K}{2^{10}.3^{10}}$

  $\Rightarrow K = 2^9 + 2^8.3 + 2^73^2 + \cdots + 3^9 = \frac{2^9\left(\frac{3^{10}}{2^{10}} -
    1\right)}{\frac{3}{2} - 1} = 3^{10} - 2^{10} = \left(3^5 - 2^5\right)\left(3^5 + 2^5\right)$

  $= 211\times275 = (35\times6 + 1)(45\times6 + 5) = 6\lambda + 5$

  Hence, the remainder is $5$.
\item Given sequence can be written as $\left(1 - \frac{2}{3}\right) + \left(1 - \frac{4}{9}\right) +
  \left(1 - \frac{8}{27}\right) + \left(1 - \frac{16}{81}\right) + \cdots$ for $100$ terms

  $= 100 - \left[\frac{2}{3} + \frac{2^2}{3^2} + \cdots\right] = 100 - \frac{\frac{2}{3}\left(1 -
    \frac{2^{100}}{3^{100}}\right)}{1 - \frac{2}{3}}$

  $= 98 + 2.\frac{2^{100}}{3^{100}}$, so the greatest integer is $98$.
\item According to question, $\big(\left(2.2^2.\ldots2^{60}\right)\left(4.4^2\ldots4^n\right))^{\frac{1}{60
    + n}} = 2^{\frac{225}{8}}$

  $\Rightarrow \big(2^{30\times61}4^{\frac{n(n + 1)}{2}}) = 2^{\frac{225}{8}}\Rightarrow 2^{1830 + n^2 + n}
  = 2^{\frac{225(60 + n)}{8}}\Rightarrow 8n^2 - 217n + 1140 = 0$

  $\Rightarrow n = 20, \frac{57}{8}$, we discard fractional value as $n$ can only be an integer.

  Now $\displaystyle\sum_{k = 1}^n(nk -k^2) = \frac{n^2(n + 1)}{2} - \frac{n(n + 1)(2n + 1)}{6} = 1330$.
\item Given, $\displaystyle\sum_{k = 1}^{10}\frac{k}{k^4 + k^2 + 1} = \frac{m}{n}$

  $\Rightarrow \frac{1}{2}\displaystyle\sum_{k = 1}^{10}\frac{1}{k}\left(\frac{k^2 + k + 1 - (k^2 - k + 1)}{(k^2 + k + 1)(k^2
  - k + 1)}\right) = \frac{1}{2}\displaystyle\sum_{k = 1}^{10}\frac{1}{k}\left(\frac{1}{k^2 + k + 1} - \frac{1}{k^2
  - k + 1}\right)$

  $\Rightarrow \frac{55}{111} = \frac{m}{n}\Rightarrow m + n = 166$.
\item We have $d = \frac{l - a}{n - 1}$, where the terms have usual meanings. $d = \frac{99}{n - 1}$.

  The factors of $99$ are $3, 9, 11, 33$. Clearly, $n = 4, 10, 12$ are the ones, which will give integral
  common difference. These will be $33, 11, 9$ and hence the sum will be $53$.
\item $A = \frac{1}{2} + \frac{1}{4^2} + \frac{1}{2^3} + \frac{1}{4^4} + \cdots \infty$

  $= \frac{\frac{1}{2}}{1 - \frac{1}{4}} + \frac{\frac{1}{4^2}}{1 - \frac{1}{16}} = \frac{11}{15}$

  $B = -\frac{1}{2} + \frac{1}{4^2} - \frac{1}{2^3} + \frac{1}{4^4} + \cdots \infty$

  $= \frac{-\frac{1}{2}}{1 - \frac{1}{4}} + \frac{\frac{1}{16}}{1 - \frac{1}{16}} = -\frac{9}{15}$

  $\therefore \frac{A}{B} = -\frac{11}{9}$.
\item Let $r$ be the common ratio of the G.P., then

  $3a_2 + a_3 = 2a_4 \Rightarrow 2r^2 - r + 3 = 0 \Rightarrow r = -1, \frac{3}{2}$

  $a_2 + a_4 = 2a_3 + 1 \Rightarrow a(r + r^3 - 2r^2) = 1\Rightarrow a = \frac{8}{3}$ when $r =
  \frac{3}{2}$.

  When $r = -1, a = -\frac{1}{4}$, which is discarded as $a_1 > 0$.

  $\Rightarrow a_2 + a_4 + 2a_5 = 40$.
\item $t_n = \frac{n}{4n^4 + 1} = \frac{n}{(2n^2 + 1)^2 - (2n)^2} = \frac{n}{(2n^2 + 2n + 1)(2n^2 - 2n + 1)}
  = \frac{1}{4}\left[\frac{1}{2n^2 - 2n + 1} - \frac{1}{2n^2 + 2n + 1}\right]$

  $S_{10} = \frac{1}{4}\left[1 - \frac{1}{221}\right] = \frac{55}{221} = \frac{m}{n}$

  $\therefore m  + n = 276$.
\item Given, $S = 2 + \frac{6}{7} + \frac{12}{7^2} + \frac{20}{7^3} + \frac{30}{7^4} + \cdots \infty$

  $\frac{S}{7} = \frac{2}{7} + \frac{6}{7^2} + \frac{12}{7^3} + \frac{20}{7^4} + \cdots \infty$

  Subtracting, we have $\frac{6S}{7} = 2 + \frac{4}{7} + \frac{6}{7^2} + \frac{8}{7^3} + \frac{10}{7^4} +
  \cdots \infty$

  $\frac{6S}{7^2} = \frac{2}{7} + \frac{4}{7^2} + \frac{6}{7^3} + \frac{8}{7^4} + \cdots \infty$

  Subtracting again, we have

  $\frac{6^2S}{7^2} = 2 + \frac{2}{7} + \frac{2}{7^2} + \frac{2}{7^3} + \cdots \infty$

  $= \frac{2}{1 - \frac{1}{7}} = \frac{7}{3}\Rightarrow 4S = \left(\frac{7}{3}\right)^3$.
\item $a_{n + 2}a_{n + 1} - a_{n + 1}a_n = 2\Rightarrow \frac{a_n + \frac{1}{a_{n + 1}}}{a_{n + 2}} =
  \frac{a_{n + 2} - \frac{1}{a_{n + 1}}}{a_{n + 2}} = 1 - \frac{1}{a_{n + 1}a_{n + 2}} = 1 - \frac{1}{2(r +
    1)} = \frac{2r + 1}{2(r + 1)}$

  Product is given by $\displaystyle\prod_{r = 1}^{30}\frac{2r + 1}{2(r + 1)} = \frac{1.3.5\ldots
    61}{2^{30}(2.3\ldots 31)}$

  $= \frac{61!}{2^{60}31!30!}$, and thus, $\alpha$ is $-60$.
\item The terms in A.P.\ are $C_2^^4\times\frac{\beta^2}{6}, -6\beta, -C_3^^6\times\frac{\beta^3}{8}$.

  Thus, $\beta^2 - \frac{5}{2}\beta^3 = -12\beta\Rightarrow \beta = \frac{12}{5}, -2\therefore \beta =
  \frac{12}{5}$ as it is given that $\beta > 0$.

  $\Rightarrow d = -\frac{72}{5} - \frac{144}{25} = -\frac{504}{25}$.

  $\therefore 50 - \frac{2d}{\beta^2} = 57$.
\item Given, $A_1A_3A_5A_7 = \frac{1}{1296}\Rightarrow A_4^4 = \frac{1}{1296}\Rightarrow A_4 = \frac{1}{6}$

  Also given, $A_2 + A_4 = \frac{7}{36}\Rightarrow A_2 = \frac{1}{36}\Rightarrow A_6 = 1\Rightarrow A_8 = 6
  \Rightarrow A_{10} = 36$

  $\therefore A_6 + A_8 + A_{10} = 43$.
\item Let $d$ be the c.d., then $d = \frac{100 - a}{n - 1}$. $A_1 = a + d, A_n = 100 - d$.

  Given $\frac{A_1}{A_n} = \frac{1}{7} = \frac{a + d}{100 - d}\Rightarrow 7a + 8\left(\frac{100 - a}{n +
    1}\right) = 100$.

  Also given, $a + n = 33$. Thus, $7n^2 - 132n - 667 = 0\Rightarrow n = 23$.
\item $S_n = \frac{n^2}{1 - \frac{1}{(n + 1)^2}} = \frac{n(n + 1)^2}{n + 2} = n^2 + 1 - \frac{2}{n + 2}$

  $\frac{1}{26} + \displaystyle\sum_{n = 1}^{50}\left(S_n + \frac{2}{n + 1} - n - 1\right) = \frac{1}{26} +
  \sum_{n = 1}^{50}\left[(n^2 - n) - 2\left(\frac{1}{n + 2} - \frac{1}{n + 1}\right)\right] = 41651$.
\item $S = \frac{a_1}{2} + \frac{a_2}{2^2} + \frac{a_3}{2^3} + \cdots$

  $\frac{S}{2} = \frac{a_1}{2^2} + \frac{a_2}{2^3} + \cdots$

  Subtracting, we have

  $\frac{S}{2} = \frac{a_1}{2} + d\left(\frac{1}{2^2} + \frac{1}{2^3} + \cdots\right) = \frac{a_1}{2} +
  d\left(\frac{\frac{1}{4}}{1 - \frac{1}{2}}\right) = \frac{a_1}{2} + \frac{d}{2} = 4$

  $\Rightarrow a_1 + d = a_2 = 4$.
\item Given, $\frac{1}{2.3.4} + \frac{1}{3.4.5} + \frac{1}{4.5.6} + \cdots + \frac{1}{100.101.102}
  = \frac{k}{102}\Rightarrow \frac{4 - 2}{2.3.4} + \frac{5 - 3}{3.4.5} + \cdots + \frac{102 -
    100}{100.101.102} = \frac{2k}{101}$

  $\Rightarrow \frac{1}{2.3} - \frac{1}{3.4}+ \frac{1}{3.4} - \frac{1}{4.5} + \frac{1}{4.5} + \cdots
  + \frac{1}{100.101} - \frac{1}{101.102} = \frac{2k}{101}$

  $\Rightarrow \frac{1}{2.3} - \frac{1}{101.102} = \frac{2k}{101}\Rightarrow 34k = 286$.
\item From given relations, $a_{n + 2} - a_{n + 1} = 2(a_{n + 1} - a_n) + 1$ so we have

  $a_2 - a_1 = 2(a_1 - a_0) + 1; a_3 - a_2 = 2(a_2 - a_1) + 1, \ldots$

  Thus, $(a_{n + 2} - a_1) - 2(a_{n + 1} - a_0) - (n + 1) = 0\Rightarrow a_{n + 2} = 2a_{n + 1} + (n + 1)$

  Setting $n = n - 2, a_n - 2a_{n - 1} = n - 1$.

  Thus, $a_{25}a_{23} - 2a_{25}a_{22} - 2a_{23}a_{24} + 4a_{22}a_{24} = (a_{25} - 2a_{24})(a_{23} - a_{22})
  = 24\times 22 = 528$.
\item $a_{n + 2} - a_{n + 1} = a_{n + 1} - a_n + 1$ and thus we have $a_2 - a_1 = a_1 - a_0 + 1; a_3 - a_2 =
  a_2 - a_1 + 1, \ldots$

  Adding we have $a_{n + 2} - a_{n + 1} = n + 1$ and thus we have $a_2 - a_1 = 1; a_3 - a_2 = 2; \ldots$ and
  thus we have $a_n = \frac{n(n - 1)}{2}$.

  So $S = \frac{1}{7^2} + \frac{3}{7^3} + \frac{6}{7^4} + \cdots$ and $\frac{S}{7} = \frac{1}{7^3}
  + \frac{3}{7^4} + \frac{6}{7^5} + \cdots$

  Subtracting we have $\frac{6S}{7} = \frac{1}{7^2} + \frac{2}{7^3} + \frac{3}{7^4} + \cdots$ and
  $\frac{6S}{7^2} = \frac{1}{7^3} + \frac{2}{7^4} + \frac{3}{7^5} + \cdots$

  $\frac{6^2S}{7^2} = \frac{1}{7^2} + \frac{1}{7^3} + \cdots =\frac{1/7^2}{1 - 1/7} \Rightarrow S
  = \frac{7}{216}$.
\item $A_{51} - A_{50} = 1000\Rightarrow l_{51}w_{51} - l_{50}w_{50} = 1000 \Rightarrow (l_1 + 50d_1)(w_1 +
  50d_2) - (l_1 + 49d_1)(w_1 + 49d_2) = 1000$

  $\Rightarrow l_1d_2 + w_1d_1 = 10$

  Now, $A_{100} - A_{90} = l_{100}w_{100} - l_{90}w_{90} = (l_1 + 99d_1)(w_1 + 99d_2) - (l_1 + 89d_1)(w_1 +
  89d_2)$

  $= 10(l_1d_2 + w_1d_1) + (99^2 - 89^2)d_1d_2 = 10.10 + (99^2 - 89^2)10 = 18900$.
\item $a_1 = 7, d = 8. T_{n + 1} - T_n = a_n\forall n\geq 1$.

  $\startalign[n=3, location=packed]
    \NC S = \NC T_1 + \NC T_2 + T_3 + \cdots + T_n\NR
    \NC S = \NC \NC T_1 + T_2 + \cdots + T_{n - 1} + T_n\NR
  \stopalign$

  On subtraction we get $T_n = T_1 + a_1 + a_2 + \cdots + a_{n - 1}\Rightarrow T_n = 3 + (n - 1)(4n - 1) =
  4n^2 - 5n + 4$

  Thus, $\displaystyle\sum_{i = 1}^{20}T_i = 1504$.
\item Using partial fraction, $t_i = \frac{(i^2 + i + 1) - (i^2 - i + 1)}{2(1 + i^2 + i^4)}
  = \frac{1}{2}\left[\frac{1}{i^2 - i + 1} - \frac{1}{i^2 + i + 1}\right]$

  $t_1 = \frac{1}{2}\left[\frac{1}{1} - \frac{1}{3}\right]; t_2 = \frac{1}{2}\left[\frac{1}{3}
    - \frac{1}{7}\right]; t_3 = \frac{1}{2}\left[\frac{1}{7} - \frac{1}{13}\right]; \cdots; t_{10}
  = \frac{1}{2}\left[\frac{1}{91} - \frac{1}{111}\right]$

  Thus, $\displaystyle\sum_{i = 1}^{10} = \frac{1}{2}\left[1 - \frac{1}{111}\right] = \frac{55}{111}$.
\item $\displaystyle\sum_{n = 1}^\infty\frac{2n^2 + 3n + 4}{2n!} = \frac{1}{2}\sum_{i = 0}^\infty\frac{2n(2n
  - 1) + 8n + 8}{2n!} = \frac{1}{2}\sum_{i = 1}^\infty\left[\frac{1}{(2n - 2)!} + \frac{2}{(2n - 1)!}
  + \frac{4}{2n!}\right]$

  We know that $e = 1 + 1 + \frac{1}{2!} + \frac{1}{3!} + \cdots$ and $e^{-1} = 1 - 1 + \frac{1}{2!}
  - \frac{1}{3!} + \cdots$

  Thus, $e + e^{-1} = 2\left[1 + \frac{1}{2!} + \frac{1}{4!} + \cdots\right]$ and $e - e^{-1} = 2\left[1
    + \frac{1}{3!} + \frac{1}{5!} + \cdots\right]$

  Thus given sum becomes $\displaystyle\frac{1}{2}\left[\frac{e + \frac{1}{e}}{2}\right] + 2\left[\frac{e
      - \frac{1}{e}}{2}\right] + 4\left[\frac{e + \frac{1}{e} - 2}{2}\right]$

  $= \frac{13}{4}e + \frac{5}{4e} - 4$.
\item Let $S = 20^{19} + 2.21.20^{18} + 3.21^2.20^{17} + \cdots + 20(21)^{19} = k.20^{19}$

  $\frac{21S}{20} = 21.20^{18} + 2.(21)^220^{17} + 3.21^3.20^{16} + \cdots + 21^{20}$

  $\Rightarrow S - \frac{21S}{20} = 20^{19} + 21.20^{18} + 21^2.20^{17} + \cdots + 21^{19} - 21^{20}$

  $-\frac{S}{20} = 20^{19}\displaystyle\left[\frac{1 - \left(\frac{21}{20}\right)^{20}}{1
    - \frac{21}{20}}\right] - 21^{20}$

  $-\frac{S}{20} = 21^{20} - 20^{20} - 21^{20}\Rightarrow S = 20^{21} = k.20^{19}\Rightarrow k = 400$.
\item \startframed[align=normal,frame=off,location=high]
  $\startalign[n=3, location=packed]
  \NC S = \NC 5 + \NC 8 + 14 + 23 + 35 + \cdots\NR
  \NC S = \NC \NC 5 + 8 + 14 + 23 + 35 + \cdots\NR
\stopalign$\stopframed
\\

  On subtraction we have $a_n = 5 + [3 + 6 +  + 9 + \cdots + (3n - 3)] = \frac{3n^2 - 3n + 10}{2}$

  $\Rightarrow a_{40} = 2345$.

  $S_n = \displaystyle\sum_{i = 1}^na_i = \frac{1}{2}\sum_{i = 1}^n(3i^2 - 3i + 10)$

  $\Rightarrow S_{30} = 13635 \Rightarrow S_{30} - a_{40} = 11290$.
\item We have $a = 3, d = 5, t_n = 373 = a + (n - 1)d \Rightarrow n = 75$.

  $S = \frac{n}{2}[2a + (n - 1)d] = \frac{75}{2}[2.3 + 74.5] = 14100$.

  The numbers divisible by $3$ are $3, 18, 33, \ldots, 363$. $363 = 3 + (k - 1)15\Rightarrow k = 25$.

  $S' = \frac{25}{2}[2.3 + (25 - 1)15] = 4575$.

  Therefore, required sum is $S - S' = 14100 - 4575 = 9525$.
\item \startframed[align=normal,frame=off,location=high]
  $\startalign[n=3, location=packed]
  \NC S = \NC 4 + \NC 11 + 21 + 34 + 50 + \cdots + t_n\NR
  \NC S = \NC \NC \;\;4 + 11 + 21 + 34 + \cdots + t_{n - 1} + t_n\NR\stopalign$\stopframed
  \\

  Subtracting, we get $t_n = 4 + 7 + 10 + 13 + \cdots + (3n + 1) = \frac{n}{2}(3n + 5) = \frac{3}{2}n^2
  + \frac{5}{2}n$

  $S_n = \sum t_n = \frac{3}{2}.\frac{n(n + 1)(2n + 1)}{6} - \frac{5}{2}.\frac{n(n + 1)}{2}$

  $\Rightarrow \frac{1}{60}(S_{29} - S_9) = 223$.
\item Let $a$ be the first term  of the A.P.\ and $d$ be its c.d. Then the terms would be $\frac{a - 2d}{4},
  \frac{a - d}{2}, a, 2(a + d), 4(a + 4d)$. Given that $a = 2$.

  From question $\frac{2 - 2d}{4} + \frac{2 - d}{2} + 2 + 2(2 + d) + 4(2 + 2d) = \frac{49}{2}\Rightarrow d =
  1$

  Thus, $a_4 = 4(a + 2d) = 16$.
  % 129
\item Using A.M.-G.M.\ inequality, we have

  $\frac{\frac{a}{5} + \frac{a}{5} + \frac{a}{5} + \frac{a}{5} + \frac{a}{5} + \frac{b}{3} + \frac{b}{3}
  + \frac{b}{3} + \frac{c}{2} + \frac{c}{2} +
  d}{11}\geq \left(\frac{a^5b^3c^2d}{5^53^32^2}\right)^{\frac{1}{11}}$

  $\Rightarrow a^5b^3c^2d\leq 5^53^32^2$, and thus, required maximum value is $337500$.
  %130
\item Given $10 = 1 + \frac{4}{k} + \frac{8}{k^2} + \frac{13}{k^3} + \frac{19}{k^4} + \cdots$

  $\Rightarrow 9 = \frac{4}{k} + \frac{8}{k^2} + \frac{13}{k^3} + \frac{19}{k^4} + \cdots$

  $\Rightarrow \frac{9}{k} = \frac{4}{k^2} + \frac{8}{k^3} + \frac{13}{k^4} + \frac{19}{k^5} + \cdots$

  On subtraction $S = 9\left(1 - \frac{1}{k}\right) = \frac{4}{k} + \frac{4}{k^2} + \frac{5}{k^3}
  + \frac{6}{k^4} + \cdots$

  $\frac{S}{k} = \frac{4}{k^2} + \frac{4}{k^3} + \frac{5}{k^4} + \frac{6}{k^5} + \cdots$

  $\Rightarrow \left(1 - \frac{1}{k}\right)S = \frac{4}{k} + \frac{1}{k^3} + \frac{1}{k^4} + \frac{1}{k^5} +
  \cdots$

  $\Rightarrow 9\left(1 - \frac{1}{k}\right)^2 = \frac{4}{k} + \frac{\frac{1}{k^3}}{1
    - \frac{1}{k}}\Rightarrow k = 2$.
  %131
\item $a_n = S_n - S_{n - 1} = \frac{n^2 + 3n}{(n + 1)(n + 2)} - \frac{(n - 1)(n + 2)}{n(n + 1)}
  = \frac{4}{n(n + 1)(n + 2)}$

  Thus, $\displaystyle28\sum_{i = 1}^{10}\frac{1}{a_i} = 28\sum_{i = 1}^{10}\frac{i(i + 1)(i + 2)}{4}
  = \frac{7}{4}\sum_{i = 1}^{10}[i(i + 1)(i + 2)(i + 3) - (i - 1)i(i + 1)(i + 2)] = 2.3.5.7.11.13$.

  So $m = 6$.
  %132
\item $S_i$ will have $i$ as first term and $2i - 1$ as c.d. So $S_i = \frac{12}{2}[2i + (12 - 1)(2i - 1)] =
  144i - 66$

  $\therefore \displaystyle\sum_{i = 1}^{10}S_i = 144.\frac{10.11}{2} - 660 = 7260$.
  %133
\item Sum is $(2^2 - 3^2 + 4^2 - 5^2 + \cdots 20$ terms$) + (2^2 + 4^2 + \cdots 10)$ terms

  $= -(2 + 3 + 4 + 5 + \cdots + 11) + 4(1^2 + 2^2 + \cdots + 10^2) = -\left[\frac{21.22}{2} - 1\right] +
  4.\frac{10.11.21}{6} = 1310$.
  %134
\item Let $r$ be the c.r.\ of the given G.P.\ and $a$ be the first term. Given $a_3a_5
  = \frac{1}{9}\Rightarrow ar^2.ar^4 = \frac{1}{9}\Rightarrow ar^3 = \frac{1}{3}$.

  Also given $a_6 + a_8 = 2 \Rightarrow ar^5 + ar^7 = 2$.

  Solving the two equations we have $a = \frac{1}{6\sqrt{2}}$ and $r = \sqrt{2}$. Thus, $6(a_2 + a_4)(a_4 +
  a_6) = 3$.
  %135
\item Clearly, $A_1 + A_2 = a + b$. Let $r$ be the c.r.\ of the G.P., then $G_1 = \sqrt[4]{a^3b}, G_2
  = \sqrt[4]{a^2b^2}, G_3 = \sqrt[4]{ab^3}$

  $G_1^4 + G_2^4 + G_3^4 + G_1^2G_3^2 = a^3b + a^2b^2 + ab^3 + a^2b^2 = ab(a + b)^2 = G_1G_3(A_1 + A_2)^2$.
  %136
\item Let $P = \left(\frac{1}{2} - \frac{1}{3}\right) + \left(\frac{1}{2^2}
  - \frac{1}{2.3} + \frac{1}{3^2}\right) + \left(\frac{1}{2^2} - \frac{1}{2^2.3} + \frac{1}{2.3^2}
  - \frac{1}{3^3}\right) + \left(\frac{1}{2^4} - \frac{1}{2^3.3} + \frac{1}{2^2.3^2} - \frac{1}{2.3^3}
  + \frac{1}{3^4}\right) + \cdots$

  $P\left(\frac{1}{2} + \frac{1}{3}\right) = \left(\frac{1}{2^2} - \frac{1}{3^2}\right)
  + \left(\frac{1}{2^3} - \frac{1}{3^3}\right) + \left(\frac{1}{2^4} - \frac{1}{3^4}\right) + \cdots$

  $\frac{5P}{6} = \frac{\frac{1}{4}}{1 - \frac{1}{2}} - \frac{\frac{1}{9}}{1 - \frac{1}{3}} = \frac{5}{12}$

  $\Rightarrow P = \frac{1}{2}\Rightarrow p + 3q = 7$.
  %137
\item $1^3 + 2^3 + 3^3 + \cdots + n^3 = \left(\frac{n(n + 1)}{2}\right)^2$

  $1.3 + 2.5 + 3.7 + \cdots + n(2n + 1) = \displaystyle\sum_{i = 1}^ni(2i + 1) = 2.\frac{n(n + 1)(2n +
  1)}{6} + \frac{n(n + 1)}{2} = \frac{n(n + 1)(4n + 5)}{6}$

  Thus, $\frac{1^3 + 2^3 + 3^3 + \cdots \mathrm{\;upto\;} n \mathrm{\;terms}}{1.3 + 2.5 + 3.7
    + \cdots \mathrm{\;upto\;} n \mathrm{\;terms}} = \frac{9}{5}$

  $\Rightarrow \frac{5n(n + 1)}{2} = \frac{9(4n + 5)}{3}\Rightarrow n = 5, -\frac{6}{5}$, however, $n$ has
  to be integral so $n = 5$.
  %138
\item Let $x^{pq^2} = y^{qr} = z^{p^2r} = k \Rightarrow pq^2 = \log_xk; qr = \log_yk; p^2r = \log_zk$

  $\Rightarrow \log_yx = \frac{qr}{pq^2} = \frac{r}{pq}; \log_xz = \frac{pq^2}{p^2r}
  = \frac{q^2}{pr}, \log_zy = \frac{p^2}{q}$

  Thus, $3, \frac{3r}{pq}, \frac{3p^2}{q}, \frac{7q^2}{pr}$ are in A.P. $\Rightarrow \frac{3r}{pq} - 3
  = \frac{1}{2}\Rightarrow r = \frac{7}{6}pq$

  Given that $r = pq + 1 \Rightarrow pq = 6$ and $r = 7 \Rightarrow \frac{3p^2}{q} = 4 \Rightarrow p = 2, q
  = 3\Rightarrow r - p - q = 2$.
  %139
\item Let $a$ be the first term and $r$ be the c.r.\ which is $\frac{1}{m}$ of this G.P.

  $t_4 = \frac{a}{m^3} = 500$. We know that $S_n - S_{n - 1} = ar^{n - 1} = \frac{a}{m^{n - 1}}$.

  Given $S_6 > S_5 + 1 \Rightarrow S_6 - S_5 > 1 \Rightarrow \frac{a}{m^5} > 1 \Rightarrow \frac{500}{m^2} >
  1 \Rightarrow m^2 < 500 \Rightarrow m < 23$.

  $S_7 - S_6 < \frac{1}{2}\Rightarrow \frac{a}{m^6} < \frac{1}{2}\Rightarrow m^3 > 10^3 \Rightarrow m > 10$.

  Thus, total possible no.\ of values of $m$ is $12$.
  %140
\item Given $\startdeterminant\NC A + 6d\NC 7\NC 1\NR\NC 2(A + 1 + 8d)\NC 17\NC 1\NR\NC A + 2 + 16d\NC
  17\NC 1\NR\stopdeterminant + 70 = 0$

  Solving this gives us $A = -7$ and $d = 6\therefore c - a - b = 20\Rightarrow S_{20} = 495$.
  %141
\item $a, b, \frac{1}{18}$ are in G.P., which gives us $\frac{a}{18} = b^2$. Also, $\frac{1}{a},
  10, \frac{1}{b}$ are in A.P., which gives us $\frac{1}{a} + \frac{1}{b} = 20$.

  Solving the two equations gives us $a = \frac{1}{8}, b = \frac{1}{12}$. Thus, $16a + 12b = 3$.
  %142
\item \startframed[align=normal,frame=off,location=high]
  $\startalign[n=3, location=packed]
  \NC S = \NC\frac{b_1}{2} \NC + \frac{b_2}{2^2} + \cdots + \frac{b_9}{2^9} + \frac{b_{10}}{2^{10}}\NR
  \NC S = \NC \NC\;\;\;\; \frac{b_1}{2^2} + \cdots + \frac{b_8}{2^9} + \frac{b_9}{2^{10}} + \frac{b_{10}}{2^{11}}\NR
  \stopalign$
  \stopframed
  \\
  \vskip{1cm}
  Subtracting, we get $S = b_1 - \frac{b_{10}}{2^{10}} + \left(\frac{a_1}{2} + \frac{a_2}{2^2} + \cdots
  + \frac{a_9}{2^9}\right)$

  $\frac{S}{2} = \frac{b_1}{2} - \frac{b_{10}}{2^{11}} + \left(\frac{a_1}{2^2} + \frac{a_2}{2^3} + \cdots
  + \frac{a_9}{2^{10}}\right)$

  Subtracting we get

  $\frac{S}{2} = \frac{b_1}{2} - \frac{b_{10}}{2^{11}} + \left(\frac{a_1}{2} - \frac{a_9}{2^{10}}\right)
  + \left(\frac{1}{2^2} + \frac{2}{2^3} + \cdots + \frac{8}{2^9}\right)$

  $\Rightarrow \frac{S}{2} = \frac{a_1 + b_1}{2} - \frac{b_{10} + 2a_9}{2^{11}} + \frac{T}{4}\Rightarrow 2S
  = 2(a_1 + b_1) - \frac{b_{10} + 2a_9}{2^9} + T$

  $2^7(2S - T) = 2^8(a_1 + b_1) - \frac{b_{10} + 2a_9}{4}$

  Given $a_n - a_{n - 1} = n - 1\therefore a_2 - a_1 = 1; a_3 - a_2 = 2; \ldots; a_9 - a_8= 8 \Rightarrow
  a_9 = 37$

  Also, $b_n - b_{n - 1} = a_{n - 1}\therefore b_{10} - b_1 = a_1 + a_2 + \cdots + a_9 \Rightarrow b_{10} =
  130$

  $\Rightarrow 2^7(2S - T) = 461$.
  %143
\item $c_2 = a_2 + b_2 = 5 \Rightarrow r_1 + r_2 = \frac{5}{4}$ and $c_3 = a_3 + b_3
  = \frac{13}{4}\Rightarrow r_1^2 + r_2^2 = \frac{13}{4}\Rightarrow r_1 = \frac{1}{2}\& r_2 = \frac{3}{4}$.

  $\displaystyle\sum_{k = 1}^\infty c_k - (12a_6 + 8b_4) = \frac{4}{1 - r_1} + \frac{4}{1 - r_2}
  - \left(\frac{48}{32} + \frac{27}{2}\right) = 9$.
  %144
\item $a_1 + a_2 + \cdots + a_{25} = \displaystyle\sum_{n = 1}^{25}a_n = \sum_{n = 1}^{25}\left(\frac{1}{2n
  - 3} - \frac{1}{2n - 5}\right) = \frac{1}{47} - \frac{1}{-3} = \frac{50}{141}$.
  %145
\item $\because a^3, b^3, c^3$ are in A.P. $\Rightarrow a^3 + c^3 = 2b^3$. $\log_ab, \log_ca$ and $\log_bc$
  are in G.P.

  $\Rightarrow \frac{\log b}{\log a}.\frac{\log c}{\log b} = \left(\frac{\log a}{\log
    c}\right)^2 \Rightarrow (\log a)^3 = (\log c)^3\Rightarrow a = c\Rightarrow a = b = c$.

  $T_1 = \frac{a + 4b + c}{3} = 2a; d = \frac{a - 8b + c}{10} = -\frac{3}{5}a$

  $S_{20} = \frac{20}{2}\left[4a + 19.\frac{-3a}{5}\right] = -74a = -444 \Rightarrow a = 6 \Rightarrow abc =
  216$.
  %146
\item Let $a$ be the first term and $r$ be the c.r.\ of the G.P. Given $a.ar.ar^2.ar^3 = 1296 \Rightarrow
  a^4r^6 = 1296\Rightarrow a = \frac{6}{r^{3/2}}$.

  Also given, $a + ar + ar^2 + ar^3 = 126 \Rightarrow \frac{1}{r^{3/2}} + \frac{1}{r^{1/2}} + r^{1/2} +
  r^{3/2} = 21$

  Let $r^{1/2} + r^{-1/2} = A \Rightarrow r^{-3/2} + r^{3/2} + 3A = A^3 \Rightarrow A^3 - 3A + A =
  21 \Rightarrow A = 3 \Rightarrow \sqrt{r} + \frac{1}{\sqrt{r}} = 3$

  $\Rightarrow r^2 - 7r + 1 = 0$. Sum of roots is $7$ which is the required answer.
  %147
\item $2a_7 = a_5 \Rightarrow 2(a_1 + 6d) = a_1 + 4d \Rightarrow a_1 + 8d = 0$. Also given, $a_{11} =
  18 \Rightarrow a_1 + 10d = 18$

  Solving the two equations we have $a_1 = -72, d = 9\Rightarrow a_{18} = 81, a_{10} = 9$.

  $12\left(\frac{1}{\sqrt{a_{10}} + \sqrt{a_{11}}} + \frac{1}{\sqrt{a_{11}} + \sqrt{a_{12}}} + \cdots
  + \frac{1}{\sqrt{a_{17}} + \sqrt{a_{18}}}\right) = 12\left(\frac{\sqrt{a_{11}} - \sqrt{a_{10}}}{d}
  + \frac{\sqrt{a_{12}} - \sqrt{a_{11}}}{d} + \cdots + \frac{\sqrt{a_{18}} - \sqrt{a_{17}}}{d}\right)$

  $= 12.\frac{\sqrt{a_{18}} - \sqrt{a_{10}}}{d} = 12.\frac{9 - 3}{9} = 8$.
  %148
\item $S = (1.1^2 + 3.5^2 + \cdots + 15.29^2) - (2.3^2 + 4.7^2 + \cdots + 14.27^2)$

  $= \displaystyle\sum_{n = 1}^8(2n - 1)(4n - 3)^2 - \sum_{n - 1}^72n.(4n - 1)^2$

  $= 6952$.
  %149
\item \startframed[align=normal,frame=off,location=high]
  $\startalign[n=3, location=packed]
  \NC S = \NC 77\NC + 757 + 7557 + \cdots + 7\underbrace{55\ldots55}_{98 \;\mathrm{times}}7\NR
  \NC 10S = \NC \NC 770 + 7570 + \cdots + 7\underbrace{55\ldots55}_{97 \;\mathrm{times}}70 +
  7\underbrace{55\ldots55}_{98 \;\mathrm{times}}70\NR
  \stopalign$
  \stopframed
  \\
  \vskip{0.8cm}
  Subtracting, we have $9S = -77 + \underbrace{13 + 13 + \cdots + 13}_{98 \;\mathrm{times}} +
  7\underbrace{55\ldots55}_{98 \;\mathrm{times}}70$

  $= -77 + 13.98 + 7\underbrace{55\ldots55}_{99 \;\mathrm{times}}7 + 13$

  $= \frac{7\overbrace{55\ldots55}^{99 \;\mathrm{times}}7 + 1210}{9}\Rightarrow m = 1210, n = 9\Rightarrow m
  + n = 1219$.
  %150
\item $3, a, b, c$ are in A.P. $\Rightarrow a - 3 = b - a\Rightarrow 2a = b + 3$.

  $3, a - 1, b + 1$ are in G.P. $\Rightarrow (a - 1)^2 = 3(b + 1)\Rightarrow (a - 7)(a - 1) = 0\Rightarrow a
  = 1, 7$

  If $a = 7$, then $b = 11$, and if $a =1$, then $b = -1$.

  We also have $c - b = a - 3\Rightarrow c = 15, -3$.

  So A.M.s are $11, -1$.
  %151
\item $8 = 3 + \frac{3}{4} + \frac{3}{4^2} + \frac{3}{4^3} + \cdots \infty + 3 + \frac{3p}{4} +
  \frac{6p}{4^2} + \frac{9p}{4^3} + \cdots \infty$

  Let $S = \frac{p}{4} + \frac{2p}{4^2} + \frac{3p}{4^3} + \cdots \infty$

  $\frac{S}{4} = \frac{p}{4^2} + \frac{2p}{4^3} + \cdots \infty$

  Subtarcting $S\left(1 - \frac{1}{4}\right) = \frac{p}{4} + \frac{p}{4^2} + \frac{p}{4^3} + \cdots \infty$

  $\Rightarrow S\left(1 - \frac{1}{4}\right) = \frac{\frac{p}{4}}{1 - \frac{1}{4}}$

  Thus, $8 = \frac{3}{1 - \frac{1}{4}} + \frac{\frac{p}{4}}{\left(1 - \frac{1}{4}\right)^2}\Rightarrow p=
  9$.
  %152
\item Given that $\log_ea, \log_eb, \log_ec$ are in A.P. $\Rightarrow 2\log_eb = \log_ea + \log_ec
  \Rightarrow b^2 = ac$.

  Also given that $\log_e\frac{a}{2b}, \log_e\frac{2b}{3c}, \log_e\frac{3c}{a}$ are in A.P. $\Rightarrow
  \left(\frac{2b}{3c}\right)^2 = \frac{a}{2b}.\frac{3c}{a} \Rightarrow \frac{b}{c} = \frac{3}{2}$

  Thus, we conclude $a:b:c = 9:6:4$.
  %153
\item Given that $\alpha = 1^2 + 4^2 + 8^2 + \cdots$. Let $t_n = an^2 + bn + c$.

  $1 = a + b + c; 4 = 4a + 2b + c; 8 = 9a + 3b + c$

  Solving these three equations, we have $a = \frac{1}{2}, b = \frac{3}{2}, c = -1$

  $\alpha = \displaystyle\sum_{n = 1}^{10}\left(\frac{n^2}{2} + \frac{3n}{2} - 1\right)^2\Rightarrow 4\alpha
  = \sum_{n = 1}^{10}(n^2 + 3n - 2)^2$

  $4\alpha - \beta = \displaystyle\sum_{n = 1}^{10}(6n^3 + 5n^2 - 12n + 4) = 55.353 + 40\Rightarrow k =
  353$.
  %154
\item First we consider the first G.P.\ with c.r.\ $r_1$. $b = ar_1^2\Rightarrow r_1^2 = \frac{b}{a}$

  $t_{11} = ar_1^{10} = a\left(\frac{b}{a}\right)^5$

  Now we consider the second G.P.\ with c.r.\ $r_2$. $b= ar_2^4 \Rightarrow r_2^4 = \frac{b}{a}$

  $t_p = ar_2^{p - 1} = a\left(\frac{b}{a}\right)^{(p - 1)/4} = a\left(\frac{b}{a}\right)^5$

  $\Rightarrow 5 = \frac{p - 1}{4}\Rightarrow p = 21$.
  %155
\item $S_n = 3 + 7 + 11 + \cdots + n$ terms $= \frac{n}{2}[6 + (n - 1)4] = 2n^2 + n$.

  $\displaystyle\sum_{i = 1}^nS_i = \sum_{i = 1}^n(2n^2 + n) = \frac{n(n + 1)(2n + 1)}{3} + \frac{n(n +
  1)}{2} = \frac{n(n + 1)(4n + 5)}{6}$

  $\Rightarrow 40 < \frac{6}{n(n + 1)}\displaystyle\sum_{k = 1}^nS_k < 42\Rightarrow 40 < 4n + 5 < 42
  \Rightarrow n = 9$.
  %156
\item First we find general term of the sequence. $t_r = \frac{1}{1 - 3r^2 + r^4} = \frac{1}{1 - 2r^2 + r^4
  - r^2} = \frac{1}{(r^2 - 1)^2 - r^2} = \frac{1}{(r^2 - r - 1)(r^2 + r -1)}$

  $= \frac{1}{2}\left[\frac{1}{r^2 - r - 1} - \frac{1}{r^2 + r - 1}\right]$

  $\Rightarrow \displaystyle\sum_{r = 1}^{10} = \frac{1}{2}\left[\frac{1}{-1} - \frac{1}{109}\right] =
  -\frac{55}{109}$.
  %157
\item Let $a$ be the length of the side of smallest side of the triangle then the other two sides have
  lengths of $ar$ and $ar^2$.

  We know that sum of lengths of any two sides of a triangle is greater than the length of the third
  side. Hence,

  $a + ar > ar^2, a + ar^2 > ar$ and $ar + ar^2 > a$.

  Thus, $r^2 - r - 1 < 0\Rightarrow r\in\left(\frac{1 - \sqrt{5}}{2}, \frac{1 + \sqrt{5}}{2}\right)$

  Also, $r^2 - r + 1 > 0$ is always true.

  Also, $r^2 + r - 1> 0\Rightarrow r\in\left(-\infty, \frac{-1 - \sqrt{5}}{2}\right)\cup\left(\frac{-1 +
    \sqrt{5}}{2}, \infty\right)$

  Taking intersection of the two results above we have $r\in\left(\frac{-1 + \sqrt{5}}{2}, \frac{1 +
    \sqrt{5}}{2}\right)$

  Given that $r > 1$, therefore $r\in\left(1, \frac{1 + \sqrt{5}}{2}\right)$

  $\therefore [r] = 1, [-r] = -2\Rightarrow 3[3] + [-r] = 1$.
  %158
\item $\frac{1\times2^2 + 2\times3^2 + \cdots + 100\times101^2}{1^2\times2 + 2^2\times3 + \cdots +
  100^2\times101} = \displaystyle\frac{\sum_{r = 1}^{100}r(r + 1)^2}{\sum_{r = 1}^{100}r^2(r + 1)}$

  $= \displaystyle\frac{\sum_{r = 1}^{100}(r^3 + 2r^2 + r)}{\sum_{r = 1}^{100}(r^3 + r^2)} =
  \frac{\left(\frac{n(n+ 1)}{2}\right)^2 + \frac{n(n + 1)(2n + 1)}{3} + \frac{n(n + 1)}{2}}{\left(\frac{n(n
      + 1)}{2}\right)^2 + \frac{n(n + 1)(2n+ 1)}{6}} = \frac{305}{301}$.
  %159
\item Let $a$ be the first term and $d$ be the common difference of the A.P. We have

  $a_1^2 - a_2^2 = -2ad - d^2; a_3^2 - a_4^2 = -2ad - 5d^2, \ldots a_{2k - 1}^2 - a_{2k}^2 = -2ad - (4k -
  3)d^2$

  $\Rightarrow A_k = \frac{k}{2}[-4ad - 2d^2 + (k - 1).(-4d^2)] = -kd[2a + (2k - 1)d]$

  $\Rightarrow 153 = 13d[2a + 5d]$ and $435 = 5d[2a + 9d]$. Solving the two equations we have $a = 1, d =
  3$.

  $a_{17} - A_7 = 910$.
  %160
\item $K = 4\left(4^x + \frac{1}{4^x}\right) + \left(4^{2x} + \frac{1}{4^{2x}}\right)$

  We know that for any real number $x, x + \frac{1}{x}\geq 2$.

  Hence $K \geq 4.2 + 2 = 10$. So the least value of $K$ is $10$.
  %161
\item $t_2 = 3t_1 + 6^2 = 3.6 + 6^2; t_3 = 3t_2 + 6^3 = 3^2.6 + 3.6^2 + 6^3; \ldots$.

  $\Rightarrow t_r = 3^{r - 1}6 + 3^{r - 2}6^2 + \cdots + 6^r = 3^{r - 1}.6\left[1 + \frac{6}{3} +
  \left(\frac{6}{3}\right)^2 + \cdots + \left(\frac{6}{3}\right)^{r - 1}\right]$

  $= 3^{r - 1}.6.\frac{1 - 2^r}{-1} = \frac{6.3^r}{3}(2^r - 1) = 2(6^r - 3^r)$

  $S_n = 2\left[\frac{6(6^n - 1)}{5} - \frac{3(3^n - 1)}{2}\right] = \frac{3}{5}[4.6^n - 5.3^n + 1]$

  $\therefore n^2 - 12n + 39 = 3\Rightarrow n = 6$.
  %162
\item The first sum has $1$ number, second has $2$ number and so on. Thus, first $k - 1$ sums will have
  $\frac{k - 1}{2}[2 + k - 2] = \frac{k(k - 1)}{2}$ numbers.

  So $(k + 1)$th sum will start with $\frac{k(k - 1)}{2} + 1$ and will have $k$ consecutive numbers.

  We find that $n = 103, \frac{k(k - 1)}{2} + 1 = 5254$ and $n = 104, \frac{k(k - 1)}{2} + 1 = 5357$. So the
  sum $S_{103}$ will contain the number $5310$.
  %163
\item The first terms in each row is $2, 5, 11, 20, 32, \ldots$. We compute the first general term which is
  found to be $t_n = \frac{3n^2 - 3n + 4}{2}$.

  So the first term of $10^{\mathrm{th}}$ row is $\frac{3.100 - 3.10 + 4}{2} = 137$.

  $S_{10} = \frac{10}{2}[2.137 + 9.3] = 1505$.
  %164
\item $\left(\frac{1}{\alpha + 1} + \frac{1}{\alpha + 2} + \cdots + \frac{1}{\alpha + 1012}\right)
  - \left\{\left(1 - \frac{1}{2}\right) + \left(\frac{1}{3} - \frac{1}{4}\right) + \cdots
  + \left(\frac{1}{2023} - \frac{1}{2024}\right)\right\} = \frac{1}{2024}$

  $\Rightarrow \left(\frac{1}{\alpha + 1} + \frac{1}{\alpha + 2} + \cdots + \frac{1}{\alpha + 1012}\right) -
  \left\{\left(1 + \frac{1}{2} + \frac{1}{3} + \frac{1}{4} + \cdots + \frac{1}{2023}\right) - \frac{1}{2024}
  - 2\left(\frac{1}{2} + \frac{1}{4} + \cdots + \frac{1}{2022}\right)\right\} = \frac{1}{2024}$

  $\Rightarrow \left(\frac{1}{\alpha + 1} + \frac{1}{\alpha + 2} + \cdots + \frac{1}{\alpha + 1012}\right) -
  \left(1 + \frac{1}{2} + \frac{1}{3} + \frac{1}{4} + \cdots + \frac{1}{2023}\right) + \frac{1}{2024}
  + \left(1 + \frac{1}{2} + \frac{1}{3} + \cdots + \frac{1}{1011}\right) = \frac{1}{2024}$

  $\Rightarrow \frac{1}{\alpha + 1} + \frac{1}{\alpha + 2} + \cdots + \frac{1}{\alpha + 1012}
  = \frac{1}{1012} + \frac{1}{1013} + \cdots + \frac{1}{2023}$

  $\Rightarrow \alpha = 1011$
  %165
\item Given that $f(x) = \frac{2^x}{2^x + \sqrt{2}}$. $f(x) + f(1 - x) = \frac{2^x}{2^x + \sqrt{2}}
  + \frac{2^{1 - x}}{2^{1 - x} + \sqrt{2}} = \frac{2^x}{2^x + \sqrt{2}} + \frac{2}{2 + 2^x\sqrt{2}} = 1$

  $\displaystyle\sum_{k = 1}^{81}f\left(\frac{k}{82}\right) = \left[f\left(\frac{1}{82}\right) + f\left(1
    - \frac{1}{82}\right)\right] + \left[f\left(\frac{2}{82}\right) + f\left(1 - \frac{2}{82}\right)\right]
  + \cdots 40$ times $+ f\left(\frac{41}{82}\right)$

  $= 40 + \frac{\sqrt{2}}{\sqrt{2} + \sqrt{2}} = \frac{81}{2}$.
  %166
\item From given relation $2a_{n + 2} - 2a_{n + 1} = 3a_{n + 1} - 3a_n$, which leads to $2a_2 - 2a_1 = 3a_1
  - 3a_0; 2a_3 - 2a_2 = 3a_2 - 3a_1, \ldots$

  Adding these gives us $2a_{n + 2} = 3a_{n + 1} + 1$, and thus, $a_n = \frac{3}{2}a_{n - 1} + \frac{1}{2}$

  $\Rightarrow a_1 = \frac{1}{2}, a_2 = \frac{1}{2} + \frac{1}{2}.\frac{3}{2}, a_3 = \frac{1}{2}
  + \frac{1}{2}.\frac{3}{2} + \frac{1}{2}.\frac{3^2}{2^2}, \ldots, a_n = \frac{1}{2}
  + \frac{1}{2}.\frac{3}{2} + \frac{1}{2}.\frac{3^3}{2^3} + \cdots + \frac{1}{2}.\frac{3^{n - 1}}{2^{n - 1}}
  = -1 + \left(\frac{3}{2}\right)^n$.

  Now it is trivial to calculate $\displaystyle\sum_{k = 1}^{100}a_k$ which comes out to be $3a_{100} -
  100$.
  %167
\item Let $a$ be the first term and $d$ be the c.d.\ of the given A.P.

  $T_m = a + (m - 1)d = \frac{1}{25}; T_{25} = a + 24d = \frac{1}{20}$

  $20\displaystyle\sum_{r = 1}^{25}T_r = 20.\frac{25}{2}\left[a + \frac{1}{20}\right] = 13\Rightarrow a
  = \frac{1}{500}$

  Also, $20S_{25} = 20.\frac{25}{2}[2a + 24d] = 13 \Rightarrow d = \frac{1}{500}$

  $\frac{1}{500} + (m - 1)\frac{1}{500} = \frac{1}{25}\Rightarrow m = 20$

  Now it is trivial to calculate $4m\displaystyle\sum_{r = m}^{2m}T_r$ which turns out to be $126$.
  %168
\item $\frac{1}{a_n} = \frac{4}{n^2 + 5n + 6} = \frac{4}{n + 2} - \frac{4}{n + 3}$

  $S_n = 4\left(\frac{1}{3} - \frac{1}{4} + \frac{1}{4} - \frac{1}{5} + \cdots + \frac{1}{n + 2}
  - \frac{1}{n + 3}\right) = 4\left(\frac{1}{3} - \frac{1}{n + 3}\right) = \frac{4n}{3(n + 3)}$

  $\therefore 507S_{2025} = \frac{4.507.2025}{3.2028} = 675$.
  %169
\item Given that $S_3 = 3a + 3d = 54 \Rightarrow a + d = 18$.

  Also given that $S_{20} = 10(2a + 19d) = 10(36 + 17d)\Rightarrow 1600 < 10(36 + 17d) < 1800 \Rightarrow
  7\frac{5}{17} < d < 8\frac{8}{17}$

  Since the A.P.\ consists of positive integers the c.d.\ will be aan integer. Thus, $d = 8\Rightarrow a =
  10$.

  $t_{11} = a + 10d = 90$.
  %170
\item The sum of terms equidistant from ends is equal. Therefore, $a_1 + a_{2024} = a_5 + a_{2020} = a_{10}
  + a_{2015} = \cdots$. This constitutes $203$ pairs.

  $\Rightarrow 203(a_1 + a_{2024}) = 2233 \Rightarrow a_1 + a_{2024} = 11$

  Hence, $\displaystyle\sum_{i = 1}^{2024}a_i = 1012\times11 = 11132$.
  %171
\item $t_i = \frac{4.i}{1 + 4.i^4} = \frac{4i}{(2i^2 + 2i + 1)(2i^2 - 2i + 1)} = \frac{1}{2i^2 - 2i + 1}
  - \frac{1}{2i^2 + 2i + 1}$

  $t_1 = 1 - \frac{1}{5}$

  $t_2 = \frac{1}{5} - \frac{1}{13}$

  $\cdots$

  $t_n = \frac{1}{2n^2 - 2n + 1} - \frac{1}{2n^2 + 2n + 1}$

  $S_n = \frac{2n^2 + 2n}{2n^2 + 2n + 1}$.
  %172
\item $t_i = \frac{4i}{4 + 3i^2 + i^4} = \frac{4i}{(i^2 + i + 2)(i^2 - i + 2)} = 2\left(\frac{1}{i^2 - i +
  2} - \frac{1}{i^2 + i + 2}\right)$

  $t_1 = 2\left(\frac{1}{2} - \frac{1}{4}\right)$

  $t_2 = 2\left(\frac{1}{4} - \frac{1}{8}\right)$

  $t_3 = 2\left(\frac{1}{8} - \frac{1}{14}\right)$

  $\cdots$

  $t_n = 2\left(\frac{1}{n^2 - n + 2} - \frac{1}{n^2 + n + 2}\right)$

  $S_n = 2\left(1 - \frac{1}{n^2 + n + 2}\right) = \frac{n^2 + n}{n^2 + n + 2}$.
  %173
\item Given that $\frac{1}{1^4} + \frac{1}{2^4} + \frac{1}{3^4} + \frac{1}{4^4} + \cdots \infty
  = \frac{\pi^4}{90}$ and $\frac{1}{2^4} + \frac{1}{4^4} + \frac{1}{6^4} + \cdots \infty = \beta$

  So $\beta = \frac{1}{16}\left[\frac{1}{1^4} + \frac{1}{2^4} + \frac{1}{3^4} + \frac{1}{4^4}
    + \cdots \infty\right] = \frac{1}{16}.\frac{\pi^4}{90}$

  Also given that $\alpha = \frac{1}{1^4} + \frac{1}{3^4} + \frac{1}{5^4} + \cdots \infty$

  $= \left(\frac{1}{1^4} + \frac{1}{2^4} + \frac{1}{3^4} + \frac{1}{4^4} + \cdots \infty\right) - \beta$

  $= \frac{\pi^4}{90}\left(1 - \frac{1}{16}\right) = \frac{\pi^4}{96}$

  $\therefore \frac{\alpha}{\beta} = 15$.
  %174
\item From given equation we have $x^2 + 9y^2 + 25z^2 = 15yz + 5zx + 3xy \Rightarrow (x - 5z)^2 + (3y -
  5z)^2 + (x - 3y)^2 = 0$

  $\Rightarrow x:y:z = 1:1/3:1/5 \Rightarrow \frac{1}{x}:\frac{1}{y}:\frac{1}{z} = 1:3:5$, and hence, $x,
  y, z$ are in H.P.
  %175
\item Given that $G = 6$. We know that $AH = G^2 \Rightarrow H = \frac{36}{A}$ so the given equation becomes

  $90A + \frac{180}{A} = 918 \Rightarrow 10 A + \frac{20}{A} = 102\Rightarrow 10A^2 - 102A + 20 = 0
  \Rightarrow (10A - 100)(10A - 2) = 0 \Rightarrow A = \frac{1}{5}, 10$.
  %176
\item From given equations $|z - 5i| = |z + 5i|\Rightarrow x^2 + y^2 - 10y + 25 = x^2 + y^2 + 10y + 25
  \Rightarrow y= 0$, which is the equation of $x$-axis.
  %177
\item Given that $|z - 4| < |z - 2| \Rightarrow x^2 + y^2 - 8y + 16 < x^2 + y^2 - 4y + 4\Rightarrow y > 3$.
  %178
\item Given that $\Im(z) = 10$ so we have a complex number of the form $z = x + 10i$. Also given that

  $\frac{2z - n}{2z + n} = 2i - i \Rightarrow 2x + 20i - n = (2i - 1)(2x + 20i + n)$

  $\Rightarrow (2x - n) + 20i = (-2x - n - 40) + (4x + 2n - 20)i$

  Comparing real and imaginary parts, we have

  $2x - n = -2x - n - 40$ and $20 = 4x + 2n - 20$

  Solving the two equations we have $n = 40$ and $x = -10$.
  %179
\item Let $x + iy = \frac{\alpha + i}{\alpha - i} = \frac{(\alpha + i)(\alpha + i)}{\alpha^2 + 1} =
  \frac{\alpha^2 - 1}{\alpha^2 + 1} + \frac{2\alpha i}{\alpha^2 + 1}$

  Comparing real and imaginary parts, we have $x = \frac{\alpha^2 - 1}{\alpha^2 + 1}, y =
  \frac{2\alpha}{\alpha^2 + 1}$

  We find that $x^2 + y^2 = 1$, which is the equation of a circle with center at $(0, 0)$ and radius $1,$
  and thus, we have the desired locus.
  %180
\item Given complex number $\omega = \frac{5 + 3z}{5(1 - z)}\Rightarrow 5\omega - 5\omega z = 5 + 3z$

  $\Rightarrow (3 + 5\omega)z = 5\omega - 5\Rightarrow |3 + 5\omega ||z| = |5\omega - 5|$(Taking modulus and
  using $|z_1z_2| = |z_1||z_2|$)

  $\Rightarrow |3 + 5\omega| > |5\omega - 5|(\because |z| < 1)\Rightarrow \left|\omega + \frac{3}{5}\right|
  > |\omega - 1|$

  Let $\omega = x + iy \Rightarrow \left(x + \frac{3}{5}\right)^2 + y^2 > (x - 1)^2 + y^2 \Rightarrow x >
  \frac{1}{5} \Rightarrow \Re(\omega) > \frac{1}{5}$.
  %181
\item We have $\frac{x + iy}{27} = \left[-\frac{1}{3}(6 + i)\right]^3 = -\frac{1}{27}(216 + 108i + 18i^2 +
  i^3)$

  $\Rightarrow x + iy = -(198 + 107i)\Rightarrow y - x = 91$.
  %182
\item Let $z = \frac{3 + 2i\sin\theta}{1 - 2i\sin\theta} = \frac{(3 + 2i\sin\theta)(1 + 2i\sin\theta)}{(1 -
  2i\sin\theta)(1 + 2i\sin\theta)}$

  $= \frac{3 - 4\sin^2\theta}{1 + 4\sin^2\theta} + \frac{8\sin\theta}{1 + 4\sin^2\theta}i$

  Given that the elements of the set are imaginary so real part must be zero. This means $\sin\theta =
  \pm\frac{\sqrt{3}}{2}$

  $\Rightarrow \theta = \left\{-\frac{\pi}{3}, \frac{\pi}{3}, \frac{2\pi}{3}\right\}$ are principal
  values. Thus, sum of these elements is $\frac{2\pi}{3}$.
  %183
\item Let $z = \frac{2 + 3i\sin\theta}{1 - 2i\sin\theta} = \frac{(2 + 3i\sin\theta)(1 + 2i\sin\theta)}{(1 -
  2i\sin\theta)(1 + 2i\sin\theta)}$

  $= \frac{2 - 6\sin^2\theta}{1 + 4\sin^2\theta} + \frac{7\sin\theta}{1 + 4\sin^2\theta}i$

  Since it is required to be purely imaginary, so the real part must be zero, which gives us:

  $\sin^2\theta = \frac{1}{3}\Rightarrow \theta = \sin^{-1}\left(\pm\frac{1}{\sqrt{3}}\right)$.
  %184
\item Given $\startdeterminant\NC6i\NC-3i\NC1\NR\NC 4\NC 3i\NC -1\NR\NC 20\NC 3\NC i\NR\stopdeterminant = x +
  iy\Rightarrow -3i\startdeterminant\NC6i\NC1\NC1\NR\NC 4\NC -1\NC -1\NR\NC 20\NC i\NC i\NR\stopdeterminant
  = 0 = x + iy$(because $C_2$ and $C_3$ are identical)

  $\Rightarrow x = 0, y = 0$.
  %185
\item $\displaystyle\sum_{n = 1}^{13}(i^n + i^{n + 1}) = (1 + i)\sum_{n = 1}^{13}i^n$

  $= (1 + i)(i + i^2 + \cdots + i^{13}) = (1 + i)\left[\frac{i - (1 - i^{13})}{1 - i}\right] = i - 1$.
  %186
\item $\frac{1 + i}{1 - i} = \frac{(1 + i)(1 + i)}{(1 - i)(1 + i)} = \frac{2i}{2} = i$.

  The minimum power of $i$ for which $i^n$ is $1$ is $4$.
  %187
\item $\frac{az + b}{z + 1} = \frac{a(x + iy) + b}{x + 1 + iy} = \frac{[(ax + b) + iay](x + 1 - iy)}{(x + 1
  + iy)(x + 1 - iy)}$

  $\Rightarrow \Im\left(\frac{az + b}{z + 1}\right) = \frac{-(ax + b)y + ay(x + 1)}{(x + 1)^2 + y^2} = y$

  $\Rightarrow a - b = 1$ and $(x + 1)^2 + y^2 = 1\Rightarrow x = -1\pm\sqrt{1 - y^2}$.
  %188
\item Given that $|z - i| = |z - 1|\Rightarrow x^2 + (y - 1)^2 = (x - 1)^2 + y^2\Rightarrow y = x$.

  Thus, the given equation represents the straight line passing through origin with a slope of $1$.
  %189
\item Given that $z = \frac{(1 + i)^2}{a - i} = \frac{(1 - 1 + 2i)(a + i)}{(a - i)(a + i)} = \frac{-2 +
  2ai}{a^2 + 1}$

  Also given that $|z| = \sqrt{\frac{2}{5}}\Rightarrow \frac{2}{\sqrt{a^2 + 1}} =
  \sqrt{\frac{2}{5}}\Rightarrow \frac{4}{1 + a^2} = \frac{2}{5}\Rightarrow a = 3$.

  Therefore, $\overline{z} = -\frac{1}{5} - \frac{3i}{5}$.
  %190
\item We know that $|z_1| = 9$ represents a circle having center $C_1(0, 0)$ having radius $9$ and $|z_2 - 3
  - 4i| = 4$ represents a circle having center $C_2(3, 4)$ having radius $4$.

  $C_1C_2 = \sqrt{(3 - 0)^2 + (4 - 0)^2} = 5$ and $|r_1 - r_2| = |9 - 4| = 5$.

  Thus, circles touch each other internally. Hence, $|z_1 - z_2|_{\mathrm{min}} = 0$.

  \startplacefigure[location=force]
    \startMPcode
      draw fullcircle scaled 1.8cm;
      draw fullcircle scaled .8cm shifted (.3cm, .4cm);
      drawdblarrow (-2cm, 0) -- (2cm, 0);
      drawdblarrow (0, -2cm) -- (0, 2cm);
      label.llft("$C_1$", origin);
      drawdot(.3cm, .4cm) withpen pencircle scaled 1pt;
      label.llft("$C_2$", (.3cm, .4cm));
    \stopMPcode
  \stopplacefigure
  %191
\item Let $z = x + iy$ so the complex number $\frac{z - \alpha}{z + \alpha}$ becomes $\frac{(x - \alpha) +
  iy}{(x + \alpha) + iy}$.

  Now $\Re\left(\frac{(x - \alpha) + iy}{(x + \alpha) + iy}\right) = x^2 - \alpha^2 + y^2$ after
  rationalization should be zero because it is given that it is purely imaginary number.

  Also, given that $|z| = 2 \Rightarrow x^2 + y^2 = 4 \Rightarrow \alpha^2 = 4 \Rightarrow \alpha = \pm 2$.
  %192
\item Let $z = x + iy$. Given that $|z| + z = 3 + i \Rightarrow \sqrt{x^2 + y^2} + x + iy = 3 + i$

  Comparing real and imaginary parts, we get

  $y = 1$ and $\sqrt{x^2 + y^2} + x = 3 \Rightarrow \sqrt{x^2 + 1} = 3 - x\Rightarrow x^2 + 1 = x^2 - 6x +
  9$

  $\Rightarrow x = \frac{4}{3}\Rightarrow z = \frac{4}{3} + i\Rightarrow |z| = \sqrt{\frac{25}{9}} =
  \frac{5}{3}$.
  %193
\item Given $z_2$ is not unimodular i.e. $|z_2|\neq 1$ and $\frac{z_1 - 2z_2}{2 - z_1\overline{z_2}}$ is
  unimodular.

  $\Rightarrow \frac{z_1 - 2z_2}{2 - z_1\overline{z_2}} = 1 \Rightarrow |z_1 - 2z_2| = |2 -
  z_1\overline{z_2}|$

  $\Rightarrow (z_1 - 2z_2)(\overline{z_1} - 2\overline{z_2}) = (2 - z_1\overline{z_2})(2 -
  \overline{z_1}z_2)(\because z\overline{z} = |z|^2)$

  $\Rightarrow |z_1|^2 + 4|z_2|^2 - 2\overline{z_1}z_2 - 2z_1\overline{z_2} = 4 + |z_1|^2|z_2|^2 -
  2\overline{z_1}z_2 - 2z_1\overline{z_2}\Rightarrow (|z_2|^2 - 1)(|z_1|^2 - 4) = 0$

  Given that $|z_2|\neq 1$, therefore, $|z_1| = 2\Rightarrow x^2 + y^2 = 2^2$

  Thus, it is proven that $z_1$ lies on center of radius $2$.
  %194
\item $|z|\geq 2$ is the region on or outside circle whose center is $(0, 0)$ and radius is $2$. Minimum
  $\left|z + \frac{1}{2}\right|$ is distance of $z$, which lie on circle $|z| = 2$ from $\left(-\frac{1}{2},
  0\right)$.

  Minimum $\left|z + \frac{1}{2}\right| = $ Distance of $\left(-\frac{1}{2}, 0\right)$ from $(-2, 0) =
  \sqrt{\left(-2 + \frac{1}{2}\right)^2 + 0} = \frac{3}{2}$.
  %195
\item Let $\alpha = x + iy$ then $1/\overline{\alpha} = \frac{x + iy}{x^2 + y^2} = \frac{x +
  iy}{|\alpha|^2}$

  From first given equation $x^2 - 2xx_0 + x_0^2 + y^2 - 2yy_0 + y_0^2 = r^2 \Rightarrow |\alpha|^2 - 2xx_0
  - 2yy_0 + z_0^2 = r^2$

  Similarly, from second given equation $|\alpha|^2 - 2xx_0^2|\alpha|^2 - 2yy_0|\alpha|^2 + |z_0|^2|\alpha|^4
  = 4r^2|\alpha|^4$

  $\Rightarrow 1 - 2xx_0 - 2yy_0 + |z_0|^2|\alpha|^2 = 4r^2|\alpha|^2$

  Subtracting we have $|\alpha^2| - 1 + |z_0|^2 - |z_0|^2|\alpha|^2 = r^2(1 - 4|\alpha|^2)$

  $\Rightarrow (|\alpha|^2 - 1)(1 - |z_0|^2) = r^2(1 - 4|\alpha|^2)$, however, it is given that $2|z_0|^2 =
  r^2 + 2$. Using this relation in the obtained equation

  $(|\alpha|^2 - 1)(1 - \frac{r^2 + 2}{2}) = r^2(1 - 4|\alpha|^2)\Rightarrow 1 - |\alpha|^2 = 2 - 8|\alpha|^2$

  $\Rightarrow |\alpha| = \frac{1}{\sqrt{7}}$.
  %196
\item Let $z = x + iy$ such that $y\neq 0$. Given that $a = z^2 + z + 1 = x^2 - y^2 + 2ixy + x + iy + 1$ is
  purely real.

  This means that $2xy + y = 0 \Rightarrow (2x + 1) = 0\Rightarrow x = -\frac{1}{2}$.

  Now $a = x^2 + x + 1 - y^2 = \frac{1}{4} - \frac{1}{2} + 1 - y^2 = \frac{3}{4} - y^2$. Now we are given
  that $y$ is non-zero so $a$ cannot be $\frac{3}{4}$.
  %197
\item Given that $z\overline{z}^3 + z^3\overline{z} = 350\Rightarrow z\overline{z}(z^2 + \overline{z}^2) =
  350$

  $\Rightarrow (x^2 + y^2)(x^2 - y^2) = 175$

  The only integral solutions of this gives us $x^2 + y^2 = 35$ and $x^2 - y^2 = 7$. Solving this pair of
  equations gives us $x = \pm4$ and $y = \pm3$.

  So the area of the rectangle are $8\times 6 = 48$ sq.\ units.
  %198
\item $\frac{z}{1 - z^2} = \frac{z}{z\overline{z} - z^2} = \frac{1}{z - \overline{z}}$, which is equation of
  imaginary $y$-axis.
  %199
\item Given that $\frac{w - \overline{w}z}{1 - z}$ is real so it will be equal to its conjugate.

  $\Rightarrow \frac{w - \overline{w}z}{1 - z} = \frac{\overline{w} - w\overline{z}}{1 -
  \overline{z}}\Rightarrow w - w\overline{z} - \overline{w}z + \overline{w}z.\overline{z} = \overline{w} -
  z\overline{w} - w\overline{z} + wz.\overline{z}$

  $\Rightarrow (w - \overline{w})(1 - |z|^2) = 0\Rightarrow |z|^2 = 0(\because w - \overline{w}\neq 0,
  \because \beta\neq 0)$

  $\Rightarrow |z| = 1$ and $|z|\neq 1$.
  %200
\item Since it is given that $|z| = 1$, therefore it is convenient to use $z = \cos\theta + i\sin\theta$.

  $w = \frac{z - 1}{z + 1} = \frac{(\cos\theta - 1 + i\sin\theta)(\cos\theta + 1 - i\sin\theta)}{(\cos\theta
  + 1 + i\sin\theta)(\cos\theta + 1 - i\sin\theta)}$

  $\Re(w) = \frac{\cos^2\theta - 1 + \sin^2\theta }{(\cos\theta + 1)^2 + \sin^2\theta} = 0$.
  %201
\item Given that $|z_1| = |z_2| = |z_3| = 1 \Rightarrow z_1\overline{z_1} = z_2\overline{z_2} =
  z_3\overline{z_3} = 1$

  Also given that $\left|\frac{1}{z_1} + \frac{1}{z_2} + \frac{1}{z_3}\right| = 1\Rightarrow |\overline{z_1}
  + \overline{z_2} + \overline{z_3}| = 1 \Rightarrow |\overline{z_1 + z_2 + z_3}| = 1\Rightarrow |z_1 + z_2 +
  z_2| = 1$.
  %202
\item Given expression can be written as $(1 + i)^{n_1} + (1 - i)^{n_1} + (1 + i)^{n_2} + (1 - i)^{n_2}$.

  We see that $1 + i = \overline{1 - i}$.

  We know that $z + \overline{z} = \Re(z)$, and thus, the above expression is a real number.
  %203
\item For the given numbers to be a conjugate pair $\overline{\sin x + i\cos2x} = \cos x -i\sin 2x$

  $\Rightarrow \sin x = \cos x$ and $\cos 2x = \sin 2x\Rightarrow \tan x = 1$ and $\tan 2x = 1$

  $\Rightarrow x = \\frac{\pi}{4}$ and $x = \frac{\pi}{8}$; both of which are not possible at the same
  time. Hence, no solution exists.
  %204
\item Given that $|z^2 + z + 1| = 1 \Rightarrow \left|\left(z + \frac{1}{2}\right)^2 + \frac{3}{4}\right| =
  1$

  We know that $|z_1 - z_2|\geq |z_1| - |z_2|$, therefore $\left|\left(z + \frac{1}{2}\right)^2
  -\left(-\frac{3}{4}\right)\right|\geq \left|\left|z + \frac{1}{2}\right|^2 -
  \left|-\frac{3}{4}\right|\right|$

  $\Rightarrow -1\leq \left|z + \frac{1}{2}\right| - \frac{3}{4}\leq 1 \Rightarrow -\frac{1}{4}\leq \left|z
  + \frac{1}{2}\right|^2\leq \frac{7}{4}$

  $\Rightarrow 0\leq \left|z + \frac{1}{2}\right|\leq \frac{\sqrt{7}}{2}$, using $|z_1 - z_2|\geq ||z_1| -
  |z_2||$

  $\therefore \left|z + \frac{1}{2}\right|\geq \left||z| - \frac{1}{2}\right|\Rightarrow
  -\frac{\sqrt{7}}{2}\leq z - \frac{1}{2}\leq \frac{\sqrt{7}}{2}$

  $\Rightarrow |z|\leq \frac{1 + \sqrt{7}}{2}$

  Also, $\left|\left(z + \frac{1}{2}\right)^2 + \frac{3}{4}\right|\leq \left|\left|z + \frac{1}{2}\right|^2
  + \frac{3}{4}\right|\Rightarrow \left|z + \frac{1}{2}\right|\geq \frac{1}{2}$.
  %205
\item Given that $sz + t\overline{z} + r = 0$. Taking conjugate gives us $\overline{s}\overline{z} +
  \overline{t}z + \overline{r} = 0$

  Solving these two equations we have $z = \frac{\overline{r}t - r\overline{s}}{|s|^2 - |t|^2}$

  For unique solutions the denominator must not be zero, and hence, $|s|\neq |t|$.

  If $L$ represents a set of straight line, then the line and given circle intersect at a maximum of $2$
  points. If $L$ represents a line, then it has infinte points on its locus.
  %206
\item $\frac{z_1 + z_2}{z_1 - z_2} = \frac{(z_1 + z_2)(\overline{z_1} - \overline{z_2})}{\overline{z_1} -
  \overline{z_2}} = \frac{z_1\overline{z_1} - z_1\overline{z_2} + z_2\overline{z_1} -
  z_2\overline{z_2}}{|z_1 - z_2|^2}$

  $= \frac{|z_1|^2 - z_1\overline{z_2} + z_2\overline{z_1} - |z_2|^2}{|z_1 - z_2|^2} = \frac{-
  z_1\overline{z_2} + z_2\overline{z_1}}{|z_1 - z_2|^2}[\because |z_1|^2 = |z_2|^2 = 1]$

  We know that $z - \overline{z} = \Im(z)\therefore z_2\overline{z_1} - z_1\overline{z_2} =
  2i\Im(z_2\overline{z_1})$

  $\therefore \frac{z_1 + z_2}{z_1 - z_2} = \frac{2i\Im(z_2\overline{z_1})}{|z_1 - z_2|^2}$, which is either
  zero or purely imaginary.
  %207
\item The solutions are given below:
  \startitemize[i]
  \item $|w - 2 - i| < 3\Rightarrow |w| - |2 + i| < 3 \Rightarrow -3 + \sqrt{5}< w < 3 + \sqrt{5}\Rightarrow
    -3 - \sqrt{5} < -|w|< 3 - \sqrt{5}$

    Similarly, $|z - 2 - i| = 3\Rightarrow -3 + \sqrt{5} \leq |z|\leq 3 + \sqrt{5}$

    $\Rightarrow -3< |z| - |w| + 3< 9$
  \item $|z + 1 - i|^2 + |z + 5 - i|^2 = (x + 1)^2 + (y - 1)^2 + (x + 5)^2 + (y - 1)^2 = 2(x^2 + y^2 - 4x -
    2y) + 28$, where $z = x  iy$.

    Given that $|z - 2 - i| = 3 \Rightarrow x^2 + y^2 - 4x - 2y = 4$.

    Thus, $|z + 1 - i|^2 + |z + 5 - i|^2 = 2.4 + 28 = 36$.
  \item Set $A$ represents the region $y\geq 1$. Set $B$ consists of the points lying on the circle having
    center at $(2, 1)$ and radius $3$. Set $C$ consists of points lying on the line $x + y = \sqrt{2}$.

    \startplacefigure[location=force]
      \startMPcode
        draw fullcircle scaled 1.5cm shifted (.5cm, .25cm);
        drawdblarrow (-2cm, 0cm) -- (2cm, -0cm);
        drawdblarrow (0, -2cm) -- (0, 2cm);
        label.urt("$(2, 1)$", (.5cm, .25cm));
        draw (-2cm, .25cm) -- (2cm, .25cm);
        label.top("$y = 1$", (2cm, .25cm));
        draw (-1cm, 1.35cm) -- (1.5cm, -1.15cm);
        drawdot(.5cm, .25cm) withpen pencircle scaled 2pt;
        drawdot(0.35cm, 0)  withpen pencircle scaled 2pt;
        label.llft("$(\sqrt{2}, 0)$", (0.35cm, 0));
        path c;
        c = fullcircle scaled 1.5cm shifted (.5cm, .25cm);
        path l;
        l = (-1cm, 1.35cm) -- (1.5cm, -1.15cm);
        label.ulft("$P$", l intersectionpoint c);
        drawdot(l intersectionpoint c) withpen pencircle scaled 2pt;
      \stopMPcode
    \stopplacefigure

    Clearly, there is only one point of intersection for the given conditions.
  \stopitemize
\item The solutions are given below:
  \startitemize[i]
  \item Minimum of $|1 - 3i - z|$ is perpendicular distance of point $(1, -3)$ from the line $\sqrt{3}x + y
    = 0$, which is

    $\frac{|\sqrt{3} - 3|}{\sqrt{3 + 1}} = \frac{3 - \sqrt{3}}{2}$.
  \item $\frac{z - 1 + \sqrt{3}i}{1 - \sqrt{3}i} = \frac{[z - (1 - \sqrt{3}i)](1 + \sqrt{3}i)}{1 + 3}$

    $= \frac{[x + iy - (1 - \sqrt{3}i)](1 + \sqrt{3}i)}{4}$ Imaginary part is given by $\frac{\sqrt{3}xi +
    y}{4}$, which is given to be greater than $0$

    The area given by $S_1\cap S_2\cap s_3$ is shown below:

    \startplacefigure[location=force]
      \startMPcode
        numeric u;
        u:= 0.4cm;

        % Define points of intersection
        pair A, B;
        A := (2u, -2u*sqrt(3));
        B := (2u, 2u*sqrt(3));

        % Define circular arc from A to B (going CCW on the circle)
        path arcAB;
        arcAB := (2u, -2u*sqrt(3))
        .. (4u, 0)
        .. (0, 4u)
        -- (0, 0)
        -- (2u, -2u*sqrt(3));

        % Define the line segment from B to A (along y = -sqrt(3)x)
        path chord;
        chord := B -- A;

        % Fill the region bounded by arc and chord
        fill arcAB -- chord -- cycle withcolor (0.8, 0.8, 0.8);

        % draw the circle
        draw fullcircle scaled (8u);

        % draw the boundary lines
        draw (-3u, 3u*sqrt(3)) -- (3u, -3u*sqrt(3)); % line y = -sqrt(3)

        % Draw labeled axes
        drawdblarrow (-5u, 0) -- (5u, 0);
        drawdblarrow (0, -5u) -- (0, 5u);
        label.llft("$O$", origin);
        label.lft("$X'$", (-5u, 0));
        label.rt("$X$", (5u, 0));
        label.top("$Y$", (0, 5u));
        label.bot("$Y'$", (0, -5u));
        label.lrt("$\sqrt{3}x + y = 0$", (3u, -3u*sqrt(3)));
        draw angle_mark((2u, -2u*sqrt(3)), (0, 0), (0, 2u), 16);
        label.urt("$150^\circ$", origin + (1u,  1u));
      \stopMPcode
    \stopplacefigure

    $S = \frac{1}{2}r^2\theta = \frac{1}{2}.4^2.\frac{5\pi}{6} = \frac{20\pi}{3}$.
  \stopitemize
  %209
\item Let $z = x + it$. Given, $|z - i|z|| = |z + i|z|| \Rightarrow |x + i(y - \sqrt{x^2 + y^2})| = |x +
  i(y + \sqrt{x^2 + y^2})|$

  $\Rightarrow y\sqrt{x^2 + y^2} = 0$, from which we can conclude that $y = 0 \Rightarrow \Im(z) = 0$ or
  all real numbers is the set of points.
  %210
\item Let $z = x + iy$. Given that $|z + 4| + |z - 4| = 0$.

  The above equation is sum of two positive quantities being equal to zero, which is impossible to
  satisfy. So there are no such points.
  %211
\item Let $w = 2(\cos\theta + i\sin\theta)\Rightarrow z = 2(\cos\theta + i\sin\theta) -
  \frac{1}{2(\cos\theta + i\sin\theta)} = 2(\cos\theta + i\sin\theta) - \frac{1}{2}(\cos\theta -
  i\sin\theta)$

  $= \frac{3}{2}\cos\theta - i\frac{5}{2}\sin\theta\Rightarrow x = \frac{3}{2}\cos\theta, y =
  \frac{5}{2}\sin\theta$

  $\Rightarrow \left(\frac{2x}{3}\right)^2 + \left(\frac{2y}{5}\right)^2 = 1 \Rightarrow \frac{x^2}{9/4} +
  \frac{y^2}{25/4} = 1$.

  Thus, $z$'s locus is that of an ellipse.
  %212
\item $\alpha = p^{1/3}, \beta = p^{1/3}\omega, \gamma = p^{1/3}\omega^2$

  $\Rightarrow \frac{x\alpha + y\beta + z\gamma}{x\beta + y\gamma + z\alpha} = \frac{xp^{1/3} +
  yp^{1/3}\omega + zp^{1/3}\omega^2}{xp^{1/3}\omega + yp^{1/3}\omega^2 + zp^{1/3}}
  = \frac{xp^{1/3} + yp^{1/3}\omega + zp^{1/3}\omega^2}{\omega(xp^{1/3} + yp^{1/3}\omega +
    zp^{1/3}\omega^2)} = \omega^2$.
  %213
\item $|az_1 - bz_2|^2 + |bz_1 + az_2|^2 = a^2|z_1|^2 + b^2|z_2|^2 - 2ab\Re(z_1\overline{z_2}) + b^2|z_1|^2
  + a^2|z_2|^2 + 2ab\Re(z_1\overline{z_2})$

  $= (a^2 + b^2)(|z_1|^2 + |z_2|^2)$.
  %214
\item Given that $|z - \alpha|^2 = k^2|z - \beta|^2 \Rightarrow (z - \alpha)(\overline{z} -
  \overline{\alpha}) = k(z - \beta)(\overline{z} - \overline{\beta})$

  $\Rightarrow |z|^2 - \frac{(\alpha - k^2\beta)}{1 - k^2} - \frac{\overline{\alpha} -
    \overline{\beta}k^2}{1 - k^2}z + \frac{|\alpha|^2 - k^2|\beta|^2}{1 - k^2} = 0$

  Equation of a circle is given by $|z|^2 + a\overline{z} + \overline{a}z + b = 0$ whose center is given by
  $-a$ and radius $\sqrt{|a|^2 - b}$.

  Comparing the two equations we have center as $\frac{\alpha - k^2\beta}{1 - k^2}$ and radius
  $\sqrt{\left(\frac{\alpha - k^2\beta}{1 - k^2}\right)\left(\frac{\overline{\alpha} -
      k^2\overline{\beta}}{1 - k^2}\right) - \frac{\alpha\overline{\alpha} - k^2\beta\overline{\beta}}{1 -
      k^2}} = \left|\frac{k(\alpha - \beta)}{1 - k^2}\right|$.
  %215
\item Given that $a_1z + a_2z^2 + \cdots + a_nz^n = 1$ and $|z| < \frac{1}{3}$.

  $\therefore |a_1z + a_2z^2 + \cdots + a_nz^n| = 1\Rightarrow |a_1z| + |a_2z^2| + \cdots + |z_nz^n| \geq 1$

  $\Rightarrow 2[|z| + |z|^2 + \cdots + |z|^n] > 1(\because |a_r| < 2)\Rightarrow \frac{2|z|}(1 - |z|^n){1 -
  |z|} > 1$

  $\Rightarrow 2|z| - 2|z|^{n + 1} > 1 - |z| \Rightarrow 3|z| > 1 + 2|z|^{n + 1}\Rightarrow |z| >
  \frac{1}{3}$, which contradicts with the given condition. Thus, there exists no such $z$, which satisfies
  the given condition.
  %216
\item Given that $\left|\frac{1 - z_1\overline{z_2}}{z_1 - z_2}\right| < 1 \Rightarrow |1 -
  z_1\overline{z_2}| < |z_1 - z_2|$

  Squaring both sides and using the fact that $|z|^2 = z\overline{z}$, we have

  $(1 - z_1\overline{z_2})(1 - \overline{z_+1}z_2) < (z_1 - z_2)(\overline{z_1} - \overline{z_2})$

  $\Rightarrow 1 - z_1\overline{z_2} - \overline{z_1}z_2 + |z_1|^2|z_2|^2 < |z_1|^2 - z_1\overline{z_2} -
  \overline{z_1}z_2 + |z_2|^2 $

  $\Rightarrow (1 - |z_1|^2)(1 - |z_2|^2) < 0$, which is true for given condition of $|z_1| < 1 < |z_2|$.
  %217
\item Given that $|z|^2w - |w|^2z = z - w \Rightarrow |z|^2w - |w|^2z - z + w = 0 \Rightarrow (|z|^2 + 1)w =
  (|w|^2 + 1)z$

  $\Rightarrow \frac{z}{w} = \frac{|z|^2 + 1}{|w|^2 + 1}$, and hence, $\frac{z}{w}$ is purely real.


  $\Leftrightarrow \frac{z}{w} = \frac{\overline{z}}{\overline{w}} \Rightarrow z\overline{w} =
  \overline{z}w$

  Again, $|z|^2w - |w|^2z = z - w \Rightarrow z\overline{z}w - w\overline{w}z - z + w = 0 \Rightarrow (z -
  w)(z\overline{w} - 1) = 0[\because z\overline{w} = \overline{z}w]$

  $\Rightarrow z = w$ or $z\overline{w} = 1$.
  %218
\item We know that $e^{i\theta} = \cos\theta + i\sin\theta$, which leads to $e^{i\pi/2} = i$.

  Thus, $i^i = \left(e^{i\pi/2}\right)^i = e^{-\pi/2}$, which is real.
  %219
\item Let $z = x + iy$. We have to prove that $\overline{z} = iz^2 \Rightarrow x - iy = i(x^2 + y^2) -2xy$

  Comparing real and imaginary parts, we get

  $x = -2xy$ and $-y = x^2 - y^2\Rightarrow x(1 + 2y) = 0$ and $x^2 - y^2 + y = 0\Rightarrow x = 0$ or $y =
  -\frac{1}{2}$

  When $x = 0, x^2 - y^2 + y = 0 \Rightarrow y = 0, 1$

  When $y = -\frac{1}{2}, x^2 - y^2 + y = 0 \Rightarrow x = \pm\frac{\sqrt{3}}{2}$

  Thus, non-zero complex numbers are $i, \pm\frac{\sqrt{3}}{2} - \frac{i}{2}$.
  %220
\item Given that $iz^3 + z^2 - z + i = 0 \Rightarrow iz^3 - i^2z^2 - z + i = 0\Rightarrow iz^2(z - i) - (z -
  i) = 0 \Rightarrow (iz^2 - 1)(z - i) = 0$

  If $z - i = 0 \Rightarrow z = i \Rightarrow |z| = |i| = 1$.

  If $iz^2 - 1 = 0 \Rightarrow iz^2 = 1\Rightarrow |z|^2 = |-i| = 1\Rightarrow |z| = 1$.
  %221
\item Here $z_1Rz_2\Leftrightarrow \frac{z_1 - z_2}{z_1 + z_2}$ is real.

  $z_1Rz_1\Leftrightarrow \frac{z_1 - z_1}{z_1 + z_1} = 0$, which is purely real. Hence, the relation is
  reflexive.

  $z_1Rz_2\Leftrightarrow \frac{z_1 - z_2}{z_1 + z_2}$ is real. $z_2Rz_1\Leftrightarrow \frac{z_2 - z_1}{z_1
    + z_2}$ is real. Hence, the relation is symmetric.

  $z_1Rz_2\Leftrightarrow \frac{z_1 - z_2}{z_1 + z_2}$ is real.

  $\frac{z_1 - z_2}{z_1 + z_2} = \frac{(x_1 - x_2) + i(y_1 - y_2)}{(x_1 + x_2) + i(y_1 + y_2)} = \frac{[(x_1
      - x_2) + i(y_1 - y_2)][(x_1 + x_2) - i(y_1 + y_2)]}{[(x_1 + x_2) + i(y_1 + y_2)][(x_1 + x_2) - i(y_1 +
      y_2)]}$

  $\Rightarrow (y_1 - y_2)(x_1 + x_2) + (x_1 - x_2)(y_1 + y_2) = 0 \Rightarrow 2x_2y_1 - 2x_1y_2 = 0
  \Rightarrow \frac{x_1}{y_1} = \frac{x_2}{y_2}$

  Similarly, $z_2Rz_3 \Leftrightarrow \frac{x_2}{y_2} = \frac{x_3}{y_3}$

  $\Rightarrow z_1Rz_3 \Leftrightarrow \frac{x_1}{y_1} = \frac{x_3}{y_3}$. Hence, the relation is trnsitive.

  And hence, $R$ is an equivalence relation.
  %222
\item Given that $\frac{(1 + i)x - 2i}{3 + i} + \frac{(2 - 3i)y + i}{3 - i} = i$

  $\Rightarrow (1 + i)(3 - i)x - 2i(3 - i) + (2 - 3i)(3 - i)y + i(3 - i) = 10i\Rightarrow 4x + 2ix - 6i - 2
  + 9y - 7iy + 3i - 1 = 10i$

  Comparing real and imaginary parts yields

  $4x + 9y - 3 = 0$ and $2x - 7y - 3 = 10$. Solving the two equations gives us $x = 3$ and $y = -1$.
  %223
\item We have to maaximize $|2z - 6 + 5i|$ or $2\left|z - 3 + \frac{5}{2}i\right|$

  We know that $||z_1| - |z_2||\leq |z_1 + z_2|$ from triangle inequality.

  $\therefore \left|z - 3 + \frac{5}{2}i\right| = \left|(z - 3 + 2i) + \frac{9}{2}i\right|\geq \left|2 -
  \frac{9}{2}\right|\geq \frac{5}{2}$.

  Hence, the maximum value is $5$.
  %224
\item Let the complex numbers be $z_1 = x_1 + iy_1$ and $z_2 = x_2 + iy_2$.

  Given that $\Re(z_1) = |z_1 - 1| \Rightarrow x_1^2 = (x_1 - 1)^2 + y_1^2 \Rightarrow y_1^2 + 1 = 2x_1$

  Also given that $\Re(z_2) = |z_2 - 1|\Rightarrow y_2^2 + 1 = 2x_2$

  and $\arg(z_1 - z_2) = \frac{\pi}{6}\Rightarrow \frac{y_1 - y_2}{x_1 - x_2} = \frac{1}{\sqrt{3}}$

  From these three equations $y_1^2 - y_2^2 = 2(x_1 - x_2) \Rightarrow \frac{1}{\sqrt{3}}(y_1 + y_2) = 2
  \Rightarrow y_1 + y_2 = 2\sqrt{3}\Rightarrow \Im(z_1 + z_2) = 2\sqrt{3}$.
  %225
\item Given that $3|z_1| = 4|z_2|\Rightarrow \frac{|z_1|}{|z_2|} = \frac{4}{3}\Rightarrow \frac{z_1}{z_2} =
  \frac{4}{2}e^{i\theta}$ and $\frac{z_2}{z_1} = \frac{3}{4}e^{-i\theta}$

  $\Rightarrow z = \frac{5}{2}\cos\theta + \frac{3}{2}i\sin\theta$

  $\Rightarrow |z| = \sqrt{\frac{17}{2}}$.
  %226
\item Let $z = e{i\theta}\Rightarrow \overline{z} = e^{-i\theta}$. We see that $|z|^2 = 1 =
  z\overline{z}\Rightarrow z = \frac{1}{\overline{z}}$

  Now $\arg\left(\frac{1 + z}{1 + \overline{z}}\right) = \arg\left(\frac{1 + z}{1 + \frac{1}{z}}\right) =
  \arg(z) = \theta$.
  %227
\item Since $\arg(z) < 0 \Rightarrow \arg(z) = -\theta\Rightarrow z = r[\cos(-\theta) + i\sin(-\theta)] =
  r(\cos\theta - i\sin\theta)$

  $\Rightarrow -z = -r[\cos\theta - i\sin\theta] = r[\cos(\pi - \theta) + i\sin(\pi - \theta)]\Rightarrow
  \arg(-z) = \pi - \theta$

  $\therefore \arg(-z) - \arg(z) = \pi$

  {\bf Alternative Solution:} $\arg(-z) - \arg(z) = \arg\left(\frac{-z}{z}\right) = \arg(-1) = \pi$.
  %228
\item Given that $|z + iw| = |z - i\overline{w}| = 2\Rightarrow |z| + |iw| = |z| + |w|\leq 2$

  $|z| + |w| = 2$ holds when $\arg(z) - \arg(iw) = 0$ i.e. $\arg\left(\frac{z}{iw}\right) = 0\Rightarrow
  \frac{z}{iw}$ is purely real i.e. $\frac{z}{w}$ is purely imaginary.

  Similarly, when $|z - i\overline{w}| = 2$, then $\frac{z}{\overline{w}}$ is purely imaginary.

  Now putting $w = i$, we get $|z + i^2| = 2 \Rightarrow |z - 1| = 2\Rightarrow z = -1$.

  Putting $w = -i$, we get $|z - i^2| = 2\Rightarrow z = 1$. So $z = \pm1$.
  %229
\item Given that $|z| = |w|$ and $\arg(z) = \pi - \arg(w)$. Let $w = re^{i\theta}$, then $\overline{w} =
  re^{-i\theta}$

  $\therefore z = re^{i(\pi - \theta)} = e^{i\pi}.e^{-i\theta} = -\overline{w}$.
  %230
\item Given that $|z_1 + z_2| = |z_!| + |z_2|$. Squaring both sides we get

  $|z_1|^2 + |z_2|^2 + 2|z_1||z_2|\cos[\arg(z_1) - \arg(z_2)] = |z_1|^2 + |z_2|^2 + 2|z_1||z_2|$

  $\Rightarrow \cos[\arg(z_1) - \arg(z_2)] = 1\Rightarrow \arg(z_1) - \arg(z_2) = 0$.
  %231
\item Given that $z = \frac{(1 - t)z_1 + tz_2}{(1 - t) + t}$. Thus, $z$ divides $z_1$ and $z_2$ in the ratio
  of $t: 1- t$.

  \startplacefigure[location=force]
    \startMPcode
      numeric u;
      u := 1cm;
      draw (0, 0) -- (3u, 0);
      drawdot(0, 0) withpen pencircle scaled 2pt;
      drawdot(u, 0) withpen pencircle scaled 2pt;
      drawdot(3u, 0) withpen pencircle scaled 2pt;
      label.top("$A$", (0, 0));
      label.bot("$z_1$", (0, 0));
      label.top("$P$", (u, 0));
      label.bot("$z$", (u, 0));
      label.top("$B$", (3u, 0));
      label.bot("$z_2$", (3u, 0));
      label.bot("$t$", (0.5u, 0));
      label.bot("$1 - t$", (2u, 0));
    \stopMPcode
  \stopplacefigure

  $\Rightarrow AP + PB = AB\Rightarrow |z - z_1| + |z - z_2| = |z_1 - z_2|$ and $\arg(z - z_1) = \arg(z_2 -
  z) = \arg(z_2 - z_1)$

  $\Rightarrow \arg\left(\frac{z - z_1}{z_2 - z_!}\right) = 0\Rightarrow \frac{z - z_1}{z_2 - z_1}$ is
  purely real.

  $\Rightarrow \frac{z - z_1}{z_2 - z_1} = \frac{\overline{z} - \overline{z_1}}{\overline{z_2}
    -\overline{z_1}}$

  $\Rightarrow \startdeterminant\NC z - z_1\NC \overline{z} - \overline{z_1}\NR\NC z_2 -
  z_1\NC \overline{z_2} - \overline{z_1}\NR\stopdeterminant = 0$
  %232
\item $|z - w|^2 = |z|^2 + |w|^2 -2|z||w|\cos[\arg(z) - \arg(w)] = |z|^2 + |w|^2 - 2|z||2| + 2|z||w| -
  2|z||w|\cos[\arg(z) - \arg(w)]$

  $= \left(|z| - |w|\right)^2 + 2|z||w|\sin^2\left[\frac{\arg(z) - \arg(2)}{2}\right]$

  $\Rightarrow |z - w|^2\leq \left(|z| - |w|\right)^2 + 4.1.1\left[\frac{\arg(z)
      - \arg(w)}{2}\right]^2[\because \sin\theta\leq \theta]$

  $\Rightarrow |z - w|^2\leq \left(|z| - |w|\right)^2 + [\arg(z) - \arg(w)]^2$.
  %233
\item Let $z = x + iy$, then $\frac{z - z_1}{z - z_2} = \frac{(x - 10) + i(y - 6)}{(x - 4) + i(y - 6)}$

  Rationalizing gives us, $\frac{[(x - 10) + i(y - 6)][(x - 4) - i(y - 6)]}{(x - 4)^2 + (y - 6)^2}$

  $= \frac{(x - 10)(x - 4) + (y - 6)^2 + i(y - 6)[x - 4 - x + 10]}{(x - 4)^2 + (y - 6)^2}$

  Given that $\arg\left(\frac{z - z_1}{z - z_2}\right) = \theta = \frac{\pi}{4}$ so $\tan\theta = 1$
  i.e. real and imaginary parts will be equal.

  $\Rightarrow x^2 + y^2 - 14x - 18y = 112 \Rightarrow |z - 7 - 9i| = 3\sqrt{2}$.
  %234
\item The diagram is given below:
  \startplacefigure[location=force]
    \startMPcode
      numeric u;
      u := 0.6cm;
      path c;
      c = fullcircle scaled(2u*sqrt(5)) shifted (2u, -u);
      draw c;
      drawdblarrow (-u, 0) -- (5u, 0);
      drawdblarrow (0, u) -- (0, -4u);
      path p;
      p = (-u, 2u) -- (2u, -u);
      path x;
      x = (0, 0) -- (5u, 0);
      drawdot(x intersectionpoint p) withpen pencircle scaled 2pt;
      drawdot(p intersectionpoint c) withpen pencircle scaled 2pt;
      drawdot(2u, -u) withpen pencircle scaled 2pt;
      draw (2u, -u) -- (p intersectionpoint c);
      label.llft("$O$", origin);
      label.lft("$X'$", (-u, 0));
      label.rt("$X$", (5u, 0));
      label.top("$Y$", (0, u));
      label.bot("$Y'$", (0, -4u));
      label.top("$z_0$", (p intersectionpoint c));
      label.bot("$C(2, -1)$", (2u, -u));
      label.urt("$(1, 0)$", (u, 0));
    \stopMPcode
  \stopplacefigure

  The complex number satisfying $|z - 2 + i|\geq \sqrt{5}$, which represents the reegion outside the
  circle(including the circumference which is equality) having center $(2, -1)$ and radius $\sqrt{5}$
  units. Now we have $\frac{1}{|z_0 - 1|}$ to be maximized. This means that $z_0$ is collinear with points
  $(2, -1)$ and $(1, 0)$ and lies on the circumference of the circle.

  Let $z_0 = x + iy$ and from figure $0 < x < 1$ and $y > 0$.

  $\Rightarrow \frac{4 - z_0 - \overline{z_0}}{z_0 - \overline{z_0} + 2i} = i\frac{2 - x}{y + 1}$, which is
  a pure imaginary number.

  Thus, $\arg\left(\frac{4 - z_0 - \overline{z_0}}{z_0 - \overline{z_0} + 2i}\right) = \frac{\pi}{2}$.
  %235
\item The diagram is given below:
  \startplacefigure[location=force]
    \startMPcode
      numeric u;
      u := 0.3cm;
      drawdot(u, 2u) withpen pencircle scaled 2pt;
      drawdot(6u, 2u) withpen pencircle scaled 2pt;
      drawdot(6u, 5u) withpen pencircle scaled 2pt;
      drawdot(7u, 6u) withpen pencircle scaled 2pt;
      drawdot(-6u, 7u) withpen pencircle scaled 2pt;

      drawdblarrow (-7u, 0) -- (8u, 0);
      drawdblarrow (0, 10u) -- (0, -u);

      draw(u, 2u) -- (6u, 2u) -- (6u, 5u) -- (7u, 6u) -- origin -- (-6u, 7u);

      path c;
      c = fullcircle scaled(2u*sqrt(85)) cutbefore (7u, 6u) cutafter(-6u, 7u);
      draw c;
      draw unitsquare scaled 5 rotated angle ((7u, 6u) - origin) shifted origin withcolor 1/2;

      label.llft("$O$", origin);
      label.lft("$X'$", (-7u, 0));
      label.rt("$X$", (8u, 0));
      label.top("$Y$", (0, 10u));
      label.bot("$Y'$", (0, -u));
      label.top("$z_0(1, 2)$", (u, 2u));
      label.rt("$(6, 2)$", (6u, 2u));
      label.rt("$z_1(6, 5)$", (6u, 5u));
      label.rt("$(7, 6)$", (7u, 6u));
      label.ulft("$z_2(-6, 7)$", (-6u, 7u));
      label.bot("$5$", (3.5u, 2u));
      label.rt("$3$", (6u, 3.5u));
      label.ulft("$\sqrt{2}$", (6.5u, 5.5u));
    \stopMPcode

  \stopplacefigure

  As shown in the diagram $z_2' = \left(6 + \sqrt{2}\cos45^\circ, 5 + \sqrt{2}\sin45^\circ\right) = 7 + 6i$

  Now, we rotate this by $\pi/2$ about origin

  $z_2 = (7 + 6i)e^{i\pi/2} = -6 + 7i$.
  %236
\item Let $A$ be the point obtained after moving $3$ units towards North-East. So this would be
  $3e^{i\pi/4}$.

  The complex number associated with $P(z)$ would be

  $\frac{z - 3e^{i\pi/4}}{0 - 3e^{i\pi/4}} = \frac{4}{3}e^{-i\pi/2} = -\frac{4i}{3}\Rightarrow z = (3 +
  4i)e^{i\pi/4}$.
  \startplacefigure[location=force]
    \startMPcode
      numeric u;
      u := 0.5cm;

      draw (0, 0) -- (u*3/sqrt(2), u*3/sqrt(2)) -- (-u/sqrt(2), 7*u/sqrt(2)) -- cycle;
      draw unitsquare scaled 5 rotated angle ((-u/sqrt(2), 7*u/sqrt(2)) - (u*3/sqrt(2), u*3/sqrt(2)))
      shifted (u*3/sqrt(2), u*3/sqrt(2)) withcolor 1/2;

      drawdblarrow (-2u, 0) -- (3u, 0);
      drawdblarrow (0, -u) -- (0, 5u);
      label.llft("$O$", origin);
      label.lft("$X'$", (-2u, 0));
      label.rt("$X$", (3u, 0));
      label.top("$Y$", (0, 5u));
      label.bot("$Y'$", (0, -u));
      label.rt("$A(3.e^{i\pi/4})$", (u*3/sqrt(2), u*3/sqrt(2)));
      label.ulft("$P[(3 + 4i)e^{i\pi/4}]$", (-u/sqrt(2), 7*u/sqrt(2)));
    \stopMPcode
  \stopplacefigure
  %237
\item Since $|PQ| = |PS| = |PR| = 2. \therefore$ Shaded part represents the external part of circle having
  center $(-1, 0)$ and radius $2$.

  We know that equation of circle having center $z_0$ and radius $r$ is given by $|z - z_0| = r$, so the
  equation of this circle is $|z + 1|> 2$.

  Also, argument of $z + 1$ w.r.t.\ positive direction of $X$-axis is $\pi/4$.

  $\therefore \arg(z + 1)\leq \frac{\pi}{4}$

  and argument of $z + 1$ in anticlockwise direction is $-\frac{\pi}{4}$.

  $\therefore -\frac{\pi}{4}\leq\arg(z + 1)$

  Thus, $-\frac{\pi}{4}\leq \arg(z + 1)\leq \frac{\pi}{4}\Rightarrow |\arg(z + 1)|\leq \frac{\pi}{4}$.
  %238
\item Given that $\frac{z_1 - z_3}{z_2 - z_3} = \frac{1 - i\sqrt{3}}{2} = \frac{2}{1 + i\sqrt{3}}$

  $\Rightarrow \frac{z_2 - z_3}{z_1 - z_3} = \frac{1 + i\sqrt{3}}{2} = \cos\frac{\pi}{3} +
  i\sin\frac{\pi}{3}$

  $\Rightarrow \left|\frac{z_2 - z_3}{z_1 - z_3}\right| = 1$ and $\arg\left(\frac{z_2 - z_3}{z_1 -
    z_3}\right) = \frac{\pi}{3}$.
  %239
\item Given that $x + iy = \frac{1}{a + ibt}.\frac{a - ibt}{a - ibt} = \frac{a - ibt}{a^2 + b^2t^2}$.

  Let $a, b\neq 0$. $\therefore x = \frac{a}{a^2 + b^2t^2}, y = \frac{-b}{a^2 + b^2t^2}\Rightarrow t
  = \frac{ay}{bx}$

  Thus, $x^2 + y^2 - \frac{x}{a} = 0 \Rightarrow \left(x - \frac{1}{2a}\right)^2 + y^2 = \frac{1}{4a^2}$,
  whihc is equation of a circle with center $\left(\frac{1}{2a}, 0\right)$ and radius $\frac{1}{2a}$.

  If $a\neq 0, b = 0$, then $x + iy = \frac{1}{a}\Rightarrow x = \frac{1}{a}, y = 0$, i.e. $z$ lies on
  $X$-axis.

  If $a = 0, b\neq 0$, then $x + iy = \frac{1}{ibt}\Rightarrow y = \frac{1}{bt}$, i.e. $z$ lies on $Y$-axis.
  %240
\item Given that $P = W^n = \left(\cos\frac{\pi}{6} + i\sin\frac{\pi}{6}\right)^n = \cos\frac{n\pi}{6} +
  i\sin\frac{n\pi}{6}$

  $\therefore P\cap H_1$ represents those points for which $\cos\frac{n\pi}{6}$ is $+$ve. Hence, it belongs
  to either first or fourth quadrant.

  $\Rightarrow z_1 = \frac{\sqrt{3}}{2} + \frac{i}{2}$ or $\frac{\sqrt{3}}{2} - \frac{i}{2}$. So $n$ is
  either $1$ or $11$.

  Similarly, $P\cap H_2$ represents those points for which $\cos\frac{n\pi}{6}$ is $-$ve. Hence, it belongs
  to either second or third quadrant. So $n$ is either $5$ or $7$.

  Thus, $z_1Oz_2 = \frac{2\pi}{3}, \frac{5\pi}{6}, \pi$.
  %241
\item Let $z_1 = r(\cos\theta + i\sin\theta)$. Now since this point lies on a circle so $|z_1| =
  2\Rightarrow r = 2, \theta = \frac{\pi}{3}$.

  The triangle is given to be an equilateral triangle  so the sides $z_1z_2$ and $z_1z_3$ will make an angle
  of $2\pi/3$ at the center.

  $\angle POz_2 = \frac{\pi}{3} + \frac{2\pi}{3} = \pi, \angle POz_3 = \pi + \frac{2\pi}{3}
  = \frac{5\pi}{3}$

  $z_2 = 2(\cos\pi + i\sin\pi) = -2$ and $z_3 = 1 - i\sqrt{3}$.
  \startplacefigure[location=force]
    \startMPcode
      numeric u;
      u := 0.7cm;

      draw fullcircle scaled(4u);
      drawdot (u, u*sqrt(3)) withpen pencircle scaled 2pt;
      drawdot (u, -u*sqrt(3)) withpen pencircle scaled 2pt;
      drawdot (-2u, 0) withpen pencircle scaled 2pt;
      draw (u, u*sqrt(3)) -- (u, -u*sqrt(3)) -- (-2u, 0) --cycle;

      drawdblarrow (-3u, 0) -- (3u, 0);
      drawdblarrow (0, 3u) -- (0, -3u);

      label.urt("$z_1(1 + \sqrt{3}i)$", (u, u*sqrt(3)));
      label.ulft("$z_2(-2, 0)$", (-2u, 0));
      label.lrt("$z_3(1 - \sqrt{3}i)$", (u, -u*sqrt(3)));
      label.llft("$O$", (0, 0));
      label.rt("$X$", (3u, 0));
      label.lft("$X'$", (-3u, 0));
      label.top("$Y$", (0, 3u));
      label.bot("$Y'$", (0, -3u));
      label.lrt("$P(2, 0)$", (2u, 0));
    \stopMPcode
  \stopplacefigure
  %242
\item The diagram is given below:

    \startplacefigure[location=force]
    \startMPcode
      numeric u;
      u := 0.7cm;

      draw fullcircle scaled(2u*sqrt(2)) shifted (u, 0);
      drawdot(2u, u*sqrt(3)) withpen pencircle scaled 2pt;
      drawdot(0, -u*sqrt(3)) withpen pencircle scaled 2pt;
      drawdot(u*(1 - sqrt(3)), u) withpen pencircle scaled 2pt;
      drawdot(u*(1 + sqrt(3)), -u) withpen pencircle scaled 2pt;
      drawdot(u, 0) withpen pencircle scaled 2pt;

      drawdblarrow (-3u, 0) -- (3u, 0);
      drawdblarrow (0, 3u) -- (0, -3u);
      draw (2u, u*sqrt(3)) -- (u*(1 - sqrt(3)), u) -- (0, -u*sqrt(3)) -- (u*(1 + sqrt(3)), -u) -- cycle;

      label.lrt("$C$", (u, 0));
      label.llft("$O$", (0, 0));
      label.rt("$X$", (3u, 0));
      label.lft("$X'$", (-3u, 0));
      label.top("$Y$", (0, 3u));
      label.bot("$Y'$", (0, -3u));
      label.urt("$z_1$", (2u, u*sqrt(3)));
      label.ulft("$z_2$", (u*(1 - sqrt(3)), u));
      label.llft("$z_3$", (0, -u*sqrt(3)));
      label.lrt("$z_4$", (u*(1 + sqrt(3)), -u));
    \stopMPcode
  \stopplacefigure

  Here, center of circle is $z_0(1, 0)$ and point of intersection of diagonals.

  $\therefore \frac{z_1 + z_3}{2} = z_0\Rightarrow z_3 = -\sqrt{3}i$

  and $\frac{z_2 - 1}{z_1 - 1} = e^{\pm i\pi/2}\Rightarrow z_2 = (1 - \sqrt{3}) + i$ and $z_4 = (1 +
  \sqrt{3}) - i$.
  %243
\item Given that $z^4 - |z|^4 = 4iz^2 \Rightarrow z^4 - z^2\overline{z}^2 = 4iz^2 \Rightarrow z^2(z^2
  - \overline{z}^2) = 4iz^2$

  $\Rightarrow z^2 = 0$ or $z^2 - \overline{z}^2 = 4i$

  $z^2 = 0$ is not possible due to given conditions. Thus, $z^2 - \overline{z}^2 = 4i$. Let $z = x + iy$,
  then

  $2iy.2x = 4i\Rightarrow xy = 1$, which is equation of a hyperbola. For minimum of $|z_1 - z_2|^2, z_1$ and
  $z_2$ must be vertices of a rectangular hyperbols.

  $\therefore z_1 = 1 + i$ and $z_2 = -1 - i\Rightarrow |z_1 - z_2|^2 = 8$.
  %244
\item Given that $\alpha_k = \cos\left(\frac{2\pi}{7}\right) + i\sin\left(\frac{2\pi}{7}\right)
  = \cos\left(\frac{2k\pi}{14}\right) + i\sin\left(\frac{2k\pi}{14}\right)$

  $\therefore \alpha_k$ are vertices of a regular polygon having $14$ sides. Let $a$ be the length of each
  side. Then, $|\alpha_{k + 1} - \alpha_k| = |\alpha_{4k - 1} - \alpha_{4k - 2}| = a$.

  Thus, $\displaystyle\frac{\sum_{k = 1}^{12}|\alpha_{k + 1} - \alpha_k|}{\sum_{k = 1}^3|\alpha_{4k - 1}^{4k
      - 2}|} = \frac{12a}{3a} = 4$
  %245
\item We have to evaluate $\displaystyle\left(\frac{1 + \sin\frac{2\pi}{9} + i\cos\frac{2\pi}{9}}{1
  + \sin\frac{2\pi}{9} - i\cos\frac{2\pi}{9}}\right)^3$

  $= \displaystyle\left[\frac{\left(\sin^2\frac{2\pi}{9} - i^2\cos^2\frac{2\pi}{9}\right)
    + \left(\sin\frac{2\pi}{9} + i\cos\frac{2\pi}{9}\right)}{1 + \sin\frac{2\pi}{9}} -
  i\cos\frac{2\pi}{9}\right]^3$

  $= \displaystyle\left[\frac{\left(\sin\frac{2\pi}{9} + i\cos\frac{2\pi}{9}\right)\left(\sin\frac{2\pi}{9}
    - i\cos\frac{2\pi}{9} + 1\right)}{1 + \sin\frac{2\pi}{9} - i\cos\frac{2\pi}{9}}\right]^3$

  $= \displaystyle\left(\sin\frac{2\pi}{9} + i\cos\frac{2\pi}{9}\right)^3 = \left(-i^2\sin\frac{2\pi}{9} +
  i\cos\frac{2\pi}{9}\right)^3$

  $= i^3\left(\cos\frac{2\pi}{9} - i\sin\frac{2\pi}{9}\right)^3 = -i \left(\cos\frac{2\pi}{3} -
  i\sin\frac{2\pi}{3}\right) = -\frac{1}{2}(\sqrt{3} - i)$.
  %246
\item Given that $|zw| = 1\Rightarrow |z||w| = 1$ and $\arg(z) = \frac{\pi}{2} + \arg(w)$.

  Let $|z| = r\Rightarrow |w| = \frac{1}{r}$ and $\arg(w) = \theta \Rightarrow \arg(z) = \frac{\pi}{2}
  + \theta$

  $\Rightarrow z = re^{i(\pi/2 + \theta)}$ and $w = \frac{1}{r}e^{i\theta}$

  $\Rightarrow \overline{z}w = e^{-i\pi/2} = -i$.
  %247
\item Given that $z = \frac{\sqrt{3}}{2} + \frac{i}{2} = \cos\frac{\pi}{6} + i\sin\frac{\pi}{6} =
  e^{i\pi/6}$

  $\therefore \left(1 + iz + z^5 + iz^8\right)^9 = \left(1 + ie^{i\pi/6} + e^{i5\pi/6} +
  ie^{i8\pi/6}\right)^9$

  $= \left(1 + e^{i\pi/2}.e^{i\pi/6} + e^{i5\pi/6} + e^{i\pi/2}e^{i4\pi/3}\right)^9\left[\because e^{i\pi/2}
    = i\right]$

  $= \left(1 + e^{i2\pi/3} + e^{i\pi/6} + e^{i11\pi/6}\right)^9 = \left[1 + \left(\cos\frac{2\pi}{3} +
    i\sin\frac{2\pi}{3}\right) + \left(\cos\frac{5\pi}{6} + i\sin\frac{5\pi}{6}\right) + \left(\cos\frac{11\pi}{6}
    + i\sin\frac{11\pi}{6}\right)\right]^9$

  $= \left(1 - \frac{1}{2} + \frac{\sqrt{3}}{2} - \frac{\sqrt{3}}{2} + \frac{1}{2}i + \frac{\sqrt{3}}{2}
  - \frac{i}{2}\right)^9$

  $= \left(\frac{1}{2} + \frac{\sqrt{3}i}{2}\right)^9 = \left(\cos\frac{\pi}{3} +
  i\sin\frac{\pi}{3}\right)^9$

  $ = \cos3\pi + i\sin3\pi = -1$.
  %248
\item Given tha $z_0$ be a root of the equation $x^2 + x + 1 = 0\Rightarrow z_0 = \omega, \omega^2$, where
  $\omega$ and $\omega^2$ are cube root of unity.

  $\Rightarrow z = 3 + 6iz_0^{81} - 3iz_0^{93} = 3 + 6i - 3i[\because w^{3n} = 1]$

  $\Rightarrow \arg(z) = \tan^{-1}\frac{3}{3} = \frac{\pi}{4}$.
  %249
\item $\displaystyle\sum_{m = 1}^{15}\Im\left(z^{2m - 1}\right) = \sin\theta + \sin3\theta + \sin5\theta
  + \cdots + \sin29\theta$

  $= \frac{\sin\left(\frac{\theta +
      29\theta}{2}\right)\sin\left(\frac{15\times2\theta}{2}\right)}{\sin\frac{2\theta}{2}}
  = \frac{\sin15\theta.\sin15\theta}{\sin\theta} = \frac{1}{4\sin2^\circ}$.
  %250
\item Given that $z = |a + b\omega + c\omega^2|\Rightarrow z^2 = (a^2 + b^2 + c^2 - ab - bc - ca)
  = \frac{1}{2}\left[(a - b)^2 + (b - c)^2 + (c - a)^2\right]$

  Thus, $z$ will attain minimum value when $a = b = c$, however, it is given that all three are not
  equal. So we make two of them equal. Let $a = b$ and $a\neq c, b\neq c$.

  Thus, minimum difference between $c - a$ and $b - c$ would be $\pm1$.

  Thus, $z^2 \geq 1\Rightarrow |z|_{\min} = 1$.
  %251
\item Given that $(1 + \omega^2)^n = (1 + \omega^4)^n\Rightarrow -\omega^n = -\omega^2n \Rightarrow \omega^n
  = 1$

  Thus, least possible value of $n$ is $3$.
  %252
\item $(1 + \omega)^7 = (1 + \omega)(1 + \omega)^6 = (1 + \omega)(-\omega^2)^6 = 1 + \omega = A + B\omega$

  Thus, $A, B$ are $1, 1$.
  %253
\item Let $A = \startdeterminant\NC 1\NC \omega \NC \omega^2\NR\NC \omega \NC \omega^2 \NC
  1\NR\NC \omega^2\NC 1\NC \omega\NR\stopdeterminant = 0$

  So $\startdeterminant\NC z + 1\NC \omega \NC \omega^2\NR\NC \omega \NC z + \omega^2 \NC
  1\NR\NC \omega^2\NC 1\NC z + \omega\NR\stopdeterminant = A + zI = 0$

  $\Rightarrow z = 0$, hence, there is only one complex number which satisfies the given determinant.
  %254
\item $t_r = (r - 1)(r - \omega)(r - \omega^2) = r^3 - 1$.

  Thus, given expression is $\sum_{r = 1}^n(r^3 - 1) = \left[\frac{n(n + 1)}{2}\right]^2 - n$.
  %255
\item Given that $z^{p + q} - z^p - z^q + 1 = 0 \Rightarrow \left(z^p - 1\right)(z^q - 1) = 0$

  Since $p, q$ are different primes so either $z^p - 1 = 0$ or $z^q - 1 = 0 \Rightarrow \alpha^p - 1 = 0$ or
  $\alpha^q - 1 = 0$. It is given that $\alpha \neq 1$.

  So dividing the two equations by $\alpha - 1$ we obtain the desired condition.
  %256
\item Given that $1, a_1, a_2, \ldots, a_{n - 1}$ are the $n$ roots of unity. Therefore,

  $x^n - 1 = (x - 1)(x - a_1)(x - a_2) \cdots (x - a_{n - 1})$

  $\Rightarrow x^{n - 1} + x^{n - 2} + \cdots + x^2 + x + 1 = (x - a_1)(x - a_2) \cdots (x - a_{n - 1})$

  Putting $x = 1$, we obtain the desired equqation.
  %257
\item $|a + b\omega + c\omega^2|^2 = \frac{1}{2}[(a - b)^2 + (b - c)^2 + (c - a)^2]$

  For minimum value $a = 1, b = 2, c = 3\Rightarrow |a + b\omega + c\omega^2|_{\min}^2 = 3$.
  %258
\item It is a simple matter of putting the values and solving the equation which gives $3$ as the answer.
  %259
\item Given that $\alpha, \beta, \gamma, \delta$ are the roots of the equation $x^4 + x^3 + x^2 + x + 1 =
  0$.

  $x^4 + x^3 + x^2 + x + 1 = 0\Rightarrow x^5 - 1 = 0$. Thus, $\alpha, \beta, \gamma, \delta$ are fifth
  roots of unity. Thus, $\alpha^5 = \beta^5 = \gamma^5 = \delta^5 = 1$.

  Thus, $\alpha^{2021} + \beta^{2021} + \gamma^{2021} + \delta^{2021} = \alpha + \beta + \gamma + \delta =
  -1$ from Vieta's relations.
  %260
\item $S_n: |z - (3 - 2i)| = \frac{n}{4}$ is a circle having center at $C_1(3, -2)$ and radius
  $\frac{n}{4}$.

  $T_n: |z - (2 - 3i)| = \frac{1}{n}$ is a circle having center at $C_2(2, -3)$ and radius $\frac{1}{n}$.

  $S_n\cap T_n = \phi$ has the meaning that these two circles do not intersect.

  {\bf Case I:} $C_1C_2 > \frac{n}{4} + \frac{1}{n}$

  $\sqrt{2} > \frac{n}{4} + \frac{1}{n}$ then $n = 1, 2, 3, 4$

  {\bf Case II:} $C_1C_2 < \left|\frac{n}{4} - \frac{1}{n}\right|$

  $\Rightarrow \sqrt{2} <\left|\frac{n^2 - 4}{4n}\right|$, which gives us infinte solutions.
  %261
\item Center of the first circle is $C_1(3\sqrt{2}, 0)$ and center of the second circle is $C_2(0,
  p\sqrt{2})$. The minimum distance will be distance between centers.

  $\Rightarrow \sqrt{18 + 2p^2} = 5\sqrt{2}\Rightarrow p = 4$.
  %262
\item The diagram is given below:

  \startplacefigure[location=force]
    \startMPcode
      numeric u;
      u := 0.5cm;

      draw fullcircle scaled (10u);
      drawdot (3u, 4u) withpen pencircle scaled 2pt;
      drawdot (4u, 3u) withpen pencircle scaled 2pt;
      drawdot (0, 5u) withpen pencircle scaled 2pt;
      draw (4u, 3u) -- (0, 5u);
      draw (3u, 4u) -- (-u*7/5, -u*24/5);
      drawdblarrow (-6u, 0) -- (6u, 0);
      drawdblarrow (0, -6u) -- (0, 6u);

      path p;
      p = (3u, 4u) -- (-u*7/5, -u*24/5);
      path q;
      q = (4u, 3u) -- (0, 5u);

      draw unitsquare scaled 5 rotated angle ((p intersectionpoint q) - (3u, 4u))
      shifted (p intersectionpoint q) withcolor 1/2;

      label.lft(("$X'$"), (-6u, 0));
      label.rt(("$X$"), (6u, 0));
      label.top(("$Y$"), (0, 6u));
      label.bot(("$Y'$"), (0, -6u));
      label.rt(("$z_1(3, 4)$"), (3u, 4u));
      label.rt(("$z_2(4, 3)$"), (4u, 3u));
      label.urt(("$z_3(0, 5)$"), (0, 5u));
      label.lft(("$z(-7/5, -24/5)$"), (-u*7/5, -u*24/5));
    \stopMPcode
  \stopplacefigure

  Slope of $z_2z_3 = \frac{3 - 5}{4 - 0} = -\frac{1}{2}$. Therefore, slope of $zz_2 = 2$ [because product of
    slopes of perpendicular lines is $-1$]

  Equation of $zz_2: y - 4 = 2(x - 3)\Rightarrow y = 2(x - 1)$. $z$ lies on the circle $x^2 + y^2 = 25
  \Rightarrow x^2 + 4(x - 1)^2 = 25 \Rightarrow x = -\frac{7}{5}, y = -\frac{24}{5}$

  $\arg(z) = \tan^{-1}\frac{24}{7} - \pi$.
  %263
\item Let $\arg(z_2) = \theta$. Given that $\overline{z_!} = i\overline{z_2}\Rightarrow z_1 = -iz_2$.

  Also given that $\arg\left(\frac{z_1}{\overline{z_2}}\right) = \pi \Rightarrow
  \arg\left(-i\frac{z_2}{\overline{z_2}}\right) = \pi$

  $\Rightarrow -\pi/2 + \theta + \theta = \pi \Rightarrow \theta = \frac{3\pi}{4}$

  $\Rightarrow \arg(z_1) = \pi/4$.
  %264
\item Given that $\left|\frac{z + 1}{z - 1}\right| < 1 \Rightarrow (x + 1)^2 + y^2 < (x - 1)^2 +
  y^2\Rightarrow x < 0$

  Also given that $\arg\left(\frac{z - 1}{z + 1}\right) = \frac{2\pi}{3}\Rightarrow \tan^{-1}\frac{y}{x - 1}
  - \tan^{-1}\frac{y}{x + 1} = \frac{2\pi}{3}$

  $\Rightarrow x^2 + y^2 + \frac{2y}{\sqrt{3}} - 1 = 0 \Rightarrow x^2 + \left(y +
  \frac{1}{\sqrt{3}}\right)^2 = \frac{4}{3}$

  Thus, $z$ has a locus of potion of the above circle in second quadrant only.

  \startplacefigure[location=force]
    \startMPcode
      numeric u;
      u := 2cm;

      draw fullcircle scaled (4u/sqrt(3)) shifted(0, -u/sqrt(3));
      drawdot(-u/sqrt(3), u*(1 - 1/sqrt(3))) withpen pencircle scaled 2pt;
      drawdot(u, 0) withpen pencircle scaled 2pt;
      drawdot(-u, 0) withpen pencircle scaled 2pt;
      draw (-u, 0) -- (-u/sqrt(3), u*(1 - 1/sqrt(3))) -- (u, 0);
      draw angle_mark((-u, 0), (-u/sqrt(3), u*(1 - 1/sqrt(3))), (u, 0), 10);

      drawdblarrow (-2u, 0) -- (2u, 0);
      drawdblarrow (0, -2u) -- (0, u);
      label.lft(("$X'$"), (-2u, 0));
      label.rt(("$X$"), (2u, 0));
      label.top(("$Y$"), (0, u));
      label.bot(("$Y'$"), (0, -2u));
      label.bot(("$2\pi/3$"), (-u/sqrt(3), u*(1 - 1/sqrt(3))) - (0, .2u));
      label.top(("$z$"), (-u/sqrt(3), u*(1 - 1/sqrt(3))));
      label.ulft(("$(-1, 0)$"), (-u, 0));
      label.urt(("$(1, 0)$"), (u, 0));
    \stopMPcode
  \stopplacefigure
  %265
\item Given that $z^2 + z + 1 = 0\Rightarrow z = \omega, \omega^2$

  We have to find $\left|\displaystyle\sum_{n = 1}^{15}\left(z^n +
  (-1)^n\frac{1}{z^n}\right)^2\right| = \left|\displaystyle\sum_{n = 1}^{15}\left(z^{2n} +
  \frac{1}{z^{2n} + 2(-1)^n}\right)\right|$

  $= \left|\frac{\omega^2(1 - \omega^{20})}{1 - \omega^2} + \frac{\frac{1}{\omega^2}\left(1 -
    \frac{1}{\omega^{30}}\right)}{1 - \frac{1}{\omega62}} -2\right|= 2$
  %266
\item $z_0 = \left(\frac{0 + 3 + 0}{3}, \frac{0 + 0 + 6}{3}\right) = (1, 2)$

  $\nu_0 = |1 + 2i|^2 + |1 + 2i - 3|^2 + |1 + 2i - 6|^2 = 30$

  $|2z_0^2 - \overline{z_0}^3 + 3|^2 + \nu_0^2 = |2(1 + 2i)^2 - (1 - 2i)^3 + 3|^2 + 900 = 1000$.
  %267
\item $z^2 + \overline{z} = 0 \Rightarrow x^2 - y^2 + x - i(2xy - y) = 0$

  $\Rightarrow x^2 - y^2 + x = 0$ and $2xy - y = 0 \Rightarrow y = 0, x = \frac{1}{2}$

  If $y = 0; x = 0, -1$ and if $x = \frac{1}{2}; y = \pm\frac{\sqrt{3}}{2}$

  $\therefore \displaystyle\sum_{\mstack z\in S}\Re(z) + \Im(z) = -1 + \frac{1}{2} + \frac{1}{2} +
  \frac{\sqrt{3}}{2} - \frac{\sqrt{3}}{2} = 0$.
  %268
\item $\frac{1 - i\sin\alpha}{1 + 2i\sin\alpha}$ is a purely imaginary number. Now

  $\frac{1 - i\sin\alpha}{1 + 2i\sin\alpha}.\frac{1 - 2i\sin\alpha}{1 - 2i\sin\alpha} \Rightarrow \frac{1 -
  2\sin^2\alpha}{1 + 4\sin^2\alpha} = 0 \Rightarrow \sin\alpha= \pm\frac{1}{\sqrt{2}}$

  $\Rightarrow \alpha = \left\{\frac{5\pi}{4}, \frac{7\pi}{4}\right\}$

  Also, $\frac{1 + i\cos\beta}{1 - 2i\cos\beta} = \frac{1 + i\cos\beta}{1 - 2i\cos\beta}.\frac{1 +
    2i\cos\beta}{1 + 2i\cos\beta} \Rightarrow 3\cos\beta = 0$ because it is given as purely real.

  $\Rightarrow \beta = \frac{3\pi}{2}$

  So $z_{\alpha\beta} = \pm 1 - i$ for two combinations of $\alpha$ and $\beta$

  $\therefore \displaystyle\sum_{\mstack (\alpha, \beta)\in S}\left(iz_{\alpha\beta} +
  \frac{1}{i\overline{z}_{\alpha\beta}}\right) = \left[i(1 - i) + \frac{1}{i(1 + i)}\right] + \left[i(-1 -
    i) + \frac{1}{i(-1 + i)}\right] = 1$.
  %269
\item Given that $\bigl|z_2 - |z_2 + 1|\bigr| = \bigl|z_2 + |z_2 - 1|\bigr|$

  $\Rightarrow (z_2 - |z_2 + 1|)(\overline{z_2} - |z_2 + 1|) = (z_2 + |z_2 - 1|)(\overline{z_2} + |z_2 -
  1|)$

  $\Rightarrow |z_2|^2 - (z_2 + \overline{z_2})|z_2 + 1| + |z_2 + 1|^2 = |z_2|^2 + (z_2 +
  \overline{z_2})|z_2 - 1| + |z_2 - 1|^2$

  $\Rightarrow (z_2 + \overline{z_2})(|z_2 + 1| + |z_2 - 1|) = |z_2 + 1|^2 - |z_2 - 1|^2 = 2(z_2 +
  \overline{z_2})$

  $\Rightarrow (z_2 + \overline{z_2})[|z_2 + 1| + |z_2 - 1| - 2] = 0$

  So $z_2$ is entire imaginary axis or on real axis within range $(1, -1)$.

  $|z_1 - 3| = \frac{1}{2}$ is the circle with center $(3, 0)$ and radius $\frac{1}{2}$.

  The diagram is shown below:

  \startplacefigure[location=force]
    \startMPcode
      numeric u;
      u := 1cm;

      draw fullcircle scaled (u) shifted(3u, 0);
      drawdot(u, 0) withpen pencircle scaled 2pt;
      drawdot(-u, 0) withpen pencircle scaled 2pt;
      drawdot(u*5/2, 0) withpen pencircle scaled 2pt;
      drawdot(u*7/2, 0) withpen pencircle scaled 2pt;


      drawdblarrow (-2u, 0) -- (4u, 0);
      drawdblarrow (0, -u) -- (0, u);
      label.lft("$X'$", (-2u, 0));
      label.rt("$X$", (4u, 0));
      label.top("$Y$", (0, u));
      label.bot("$Y'$", (0, -u));
      label.bot("$(-1, 0)$", (-u, 0));
      label.bot("$(1, 0)$", (u, 0));
      label.llft("$(5/2, 0)$", (u*5/2, 0));
      label.lrt("$(7/2, 0)$", (u*7/2, 0));
    \stopMPcode
  \stopplacefigure

  Clearly, the minimum distance is $\frac{5}{2} - 1 = \frac{3}{2}$.
  %270
\item Given that $z^2 = \overline{z}.2^{1 - |z|} \Rightarrow |z^2| = |\overline{z}|.2^{1 - |z|}$

  $\Rightarrow |x^2 - y^2 + 2xyi| = |x - iy|.2^{1 - |z|}$

  Squaring gives us $(x^2 + y^2)^2 = (x^2 + y^2).2^{2 - 2|z|}\Rightarrow x^2 + y^2 = \frac{4}{2^{2(x^2 +
      y^2)}}$

  If $x^2 + y^2 < 1$ or $x^2 + y^2 > 1$ then the equality is not satisfied. Thus, $|z| = 1\Rightarrow z^2
  = \overline{z}\Rightarrow z^3 = 1\Rightarrow z = \omega, \omega^2$

  $\omega^n = (1 + \omega)^2 = (-\omega^2)^n$, which gives $6$ as least possible natural number.
  %271
\item Given that $1 < |z - 3 + 2i| < 4 \Rightarrow 1 < (a - 3)^2 + (b + 2)^2 < 16$. There are $40$
  complex numbers, which satisfy the above inequality.
  %272
\item $A = \startbmatrix\NC1 + i\NC 1\NR -i\NC 0\NR\stopbmatrix\Rightarrow A^2 = \startbmatrix\NC i\NC  1 +
  i\NR\NC -i + 1\NC -i\NR\stopbmatrix$

  $\Rightarrow A^4 = \startbmatrix\NC1\NC 0\NR\NC 0\NC 1\NR\stopbmatrix = I \therefore A^{4n + 1} = A$

  Thus, total no.\  of required elements in the set would be $25$.
  %273
\item Let $z = x + iy$ and we have been given that $\overline{z} = iz^2 + z^2 - z \Rightarrow z
  + \overline{z} = iz^2 + z^2\Rightarrow 2x = (i + 1)(x^2 - y^2 + 2xyi)$

  Comparing real and imaginary parts gives us

  $2x = x^2 - y^2 + 2xy$ and $x^2 - y^2 - 2xy = 0$. Thus, $2x = -4xy \Rightarrow x = 0, y = -\frac{1}{2}$

  When $x = 0, y = 0$ and when $y = -\frac{1}{2}, x= \frac{1\pm\sqrt{2}}{2}$

  Now we have $2$ non-zero complex numbers, whose sum of squares of modulii is $\frac{(1 + \sqrt{2})^2 +
    1}{4} + \frac{(1 - \sqrt{2})^2 + 1}{4} = 2$.
  %274
\item Let $|z| = r$. Now given that $\left|z - \frac{1}{z}\right| = 2 \Rightarrow \left||z|
  - \frac{1}{|z|}\right|\leq \left|z - \frac{1}{z}\right|\leq |z| + \frac{1}{|z|}$

  $\Rightarrow \left|r - \frac{1}{r}\right|\leq 2\leq r + \frac{1}{r}\Rightarrow \left|r
  - \frac{1}{r}\right|\leq 2$ and $r + \frac{1}{r}\geq 2$

  $\Rightarrow r - \frac{1}{r}\geq -2$ and $r - \frac{1}{r}\leq 2$. Thus, we have $r -
  1\leq \sqrt{2} \Rightarrow |z|_{\min} = 1 + \sqrt{2}$.
  %275
\item $|z - 1 + i| \geq |z| \Rightarrow x - y \leq 1, |z| < 2 \Rightarrow x^2 + y^2 < 4$ and $|z + i| = |z -
  1| \Rightarrow x + y = 0$

  Combining $\Rightarrow x \leq \frac{1}{2}$.

  Given that $w = 2x + iy\Rightarrow 2x\leq \frac{1}{2}\Rightarrow x\leq \frac{1}{4}$

  and $(2x)^2 + (2x)^2 < 4\Rightarrow x^2 < \frac{1}{2}\Rightarrow
  x\in\left(-\frac{1}{\sqrt{2}}, \frac{1}{4}\right]$.
  %276
\item   Given that $|z - 2|\leq 1$, which is interior of the circle $(x - 1)^2 + y^2 = 1$.

  Also given that $z(1 + i) + \overline{z}(1 - i)\leq 2 \Rightarrow x - y\leq 1$

  $\therefore PA  = \sqrt{17}, PB = \sqrt{13}$. Maximum is $PA$ and minimum is $PD$.

  Let $D = (2 + \cos\theta, \sin\theta) \Rightarrow \tan\theta = -2$

  $\Rightarrow z_1 = 2 - \frac{1}{\sqrt{5}} + \frac{2i}{\sqrt{5}}$ and $z_2 = 1$

  $\Rightarrow 5(|z_1|^2 + |z_2|^2) = 30 - 4\sqrt{5}\Rightarrow \alpha + \beta = 26$.

  The diagram is given below:

  \startplacefigure[location=force]
    \startMPcode
      numeric u;
      u := 1cm;

      draw fullcircle scaled (2u) shifted(2u, 0);
      path c;
      c = fullcircle scaled(2u) shifted(2u, 0) cutbefore (2u, u) cutafter(u, 0);
      fill c -- (2u, u) -- cycle withcolor (0.8, 0.8, 0.8);;
      drawdot(u, 0) withpen pencircle scaled 2pt;
      drawdot(2u, 0) withpen pencircle scaled 2pt;
      drawdot(2u, u) withpen pencircle scaled 2pt;
      draw(0, 4u) -- (2u, u) -- (u, 0) -- cycle;
      draw(0, 4u) -- (2u, 0);

      drawdblarrow (-2u, 0) -- (4u, 0);
      drawdblarrow (0, -u) -- (0, 5u);
      label.lft("$X'$", (-2u, 0));
      label.rt("$X$", (4u, 0));
      label.top("$Y$", (0, 5u));
      label.bot("$Y'$", (0, -2u));
      label.llft("$A(1, 0)$", (u, 0));
      label.urt("$B(2, 1)$", (2u, u));
      label.bot("$C(2, 0)$", (2u, 0));
      label.urt("$P(0, 4)$", (0, 4u));
      label.ulft("$D$", (u + u/sqrt(2), u/sqrt(2)) - (.3u, 0));
    \stopMPcode
  \stopplacefigure
  %277
\item Let $z = x + iy$, then $|z| = 3$ represents the circle with center at $(0, 0)$ and radius $3$.

  $\arg\left(\frac{z - 1}{z + 1}\right) = \frac{\pi}{4} \Rightarrow \arg\left(\frac{z - 1}{z + 1}\right) =
  1 \Rightarrow \arg\left(\frac{[(x - 1) + iy][(x + 1) - iy]}{(x + 1)^2 + y^2}\right) = 1$

  $\Rightarrow \frac{2y}{x^2 + y^2 - 1} = 1\Rightarrow x^2 + (y - 1)^2 = 2$, which represents a circle with
  center $(0, 1)$ and radius $\sqrt{2}$.

  The diagram is given below:

  \startplacefigure[location=force]
    \startMPcode
      numeric u;
      u := .6cm;

      draw fullcircle scaled (6u);
      draw fullcircle scaled (2u*sqrt(2)) shifted(0, u);
      drawdot(0, u) withpen pencircle scaled 2pt;
      drawdblarrow (-3u, 0) -- (3u, 0);
      drawdblarrow (0, -3u) -- (0, 3u);
      label.lft("$X'$", (-3u, 0));
      label.rt("$X$", (3u, 0));
      label.top("$Y$", (0, 3u));
      label.bot("$Y'$", (0, -3u));
      label.llft("$O(0, 0)$", (0, 0));
      label.urt("$C(0, u)$", (0, u));
    \stopMPcode
  \stopplacefigure

  From the digram it is clear that no.\ of points of intersection is zero.
  %278
\item Let $\alpha = \frac{z + 3z + 4z^2}{2 - 3z + 4z^2} = \frac{z - 3z + 4z^2 + 6z}{2 - 3z + 4z^2} = 1
  + \frac{6z}{2 - 3z + 4z^2}$

  Given that $\alpha = \overline{\alpha}$ because it is a real number.

  $\Rightarrow \frac{6z}{2 - 3z + 4z^2} = \frac{6\overline{z}}{2 - 3\overline{z} +
    4\overline{z}^2} \Rightarrow (z - \overline{z})(2 - 4z\overline{z}) = 0$

  Since $z\neq \overline{z}$ because it is not purely real number therefore
  $z\overline{z}= \frac{1}{2} \Rightarrow |z|^2 = \frac{1}{2}$.
  %279
\item Given that $\overline{z} - z^2 = i(\overline{z} + z^2)\Rightarrow (1 - i)\overline{z} = (1 +
  i)z^2\Rightarrow \frac{1 - i}{1 + i}\overline{z} = z^2$

  $\Rightarrow -\frac{2i}{2}\overline{z} = z^2 \Rightarrow -i(x - iy) = (x^2 - y^2) + 2ixy$, where $z = x +
  iy$

  Comparing real and imaginary parts gives us

  $x^2 - y^2 + y = 0$ and $(2y + 1)x = 0\Rightarrow x =0$ or $y = -\frac{1}{2}$

  When $x = 0 \Rightarrow y - y^2 = 0\Rightarrow y = 0, 1$

  When $y = -\frac{1}{2}\Rightarrow x^2 - \frac{1}{4} - \frac{1}{2} = 0 \Rightarrow x
  = \pm\frac{\sqrt{3}}{2}$

  Thus, we have $4$ total solutions.
  %280
\item Given that $\frac{1 + ai}{b + i}$ is of unit modulus. $\Rightarrow \left|\frac{1 + ai}{b + i}\right| =
  1\Rightarrow |1 + ai| = |b + i| \Rightarrow a^2 = b^2 \Rightarrow a = -b$ because it is given that $ab <
  0$.

  Also given that $a + ib$ lies on $|z - 1| = |2z| \Rightarrow (a - 1)^2 + b^2 = 4(a^2 + b^2)\Rightarrow (a
  - 1)^2 + a^2 = 8a^2 \Rightarrow 6a^2 + 2a - 1 = 0\Rightarrow a = \frac{\sqrt{7} - 1}{6}, b = \frac{1
    - \sqrt{7}}{6}$ or $a = \frac{-1 -\sqrt{7}}{6}$

  Now it is trivial to find the answer.
  %281
\item We know that $z + \overline{z} = 2x$ and therefore $(az^2 + bz) + (a\overline{z}^2 + b\overline{z}) =
  2a$ and $(bz^2 + az) + (b\overline{z}^2 + a\overline{z}) = 2b$

  Adding and subtracting the two equations gives us

  $(a + b)(z^2 + z + \overline{z}^2 + \overline{z}) = 2(a + b)$ and $(a - b)(z^2 - z + \overline{z}^2
  - \overline{z}) = 2(a - b)$

  If $a + b \neq 0$ then from the two obtained equations we have $x^2 - y^2 + x = 1$ and $x^2 - y^2 - x = 1$

  $\Rightarrow 2x = 0\Rightarrow y^2 = -1$, which is not possible.

  If $a + b = 0$, then there are infinite solutions possible.
  %282
\item Given that $\frac{z^2 + 8iz - 15}{z^2 - 3iz - 2}\in\mathbb{R}\Rightarrow 1 + \frac{11iz - 13}{z^2 -
  3iz - 2}\in\mathbb{R}$

  Putting $z = \alpha - \frac{13i}{11}$ makes $11zi - 13$ acquire a value of $11\alpha i$ and therefore $z^2
  - 3iz -2$ must be imaginary.

  Now we put $z = x + iy$ and we have $z^2 - 3iz - 2 = x^2 - y^2 + 2zyyi - 3ix + 3y - 2$ which should be
  imaginary.

  Thus, $x^2 - y^2 + 3y - 2 = 0\Rightarrow x^2 = (y - 1)(y - 2)$ but $z = \alpha - \frac{13i}{11}$ so we put
  $x = \alpha$ and $y = \frac{13}{11}$ to obtain $\alpha^2 = \frac{24\times 35}{121}$.
  %283
\item $\omega = z\overline{z} = k_1z + k_2iz + \lambda(1 + i)\Rightarrow \Re(\omega) = x^2 + y^2 + k_1x -
  k_2y + \lambda = 0$

  Given that the circle has radius $1$ and it touches the line $y = 1$ and $y$-axis, therefore, the center
  of circle is $(1, 2) \Rightarrow k_1 = -2, k_2 = 4$

  Radius $=1\Rightarrow \lambda = 4$

  $\Im(\omega) = 0 \therefore 2x - y + 2 = 0$. Distance of center of circle from this line is
  $\frac{2}{\sqrt{5}}$.

  $\therefore \frac{AB^2}{4} = \left(1^2 - \frac{4}{5}\right)\Rightarrow 30AB^2 = 24$.
  %284
\item Let $z_1 = \frac{z - \overline{z} + z\overline{z}}{2 - 3z + 5\overline{z}}$, also let $z = 3 +
  iy$ then $\overline{z} = 3 - iy$

  $\therefore z_1 = \frac{2iy + 9 + y^2}{2 - 3(3 + iy) + 5(3 - iy)} = \frac{9 + y^2 + 2iy}{8(1 - iy)}$

  $\Re(z_1) = \frac{9 + y^2 - 2y^2}{8(1 + y^2)} = \frac{9 - y^2}{8(1 + y^2)} = \frac{1}{8}\left[\frac{10}{1
      + y^2} - 1\right]$

  $1 + y^2\in[1, \infty]\Rightarrow \frac{1}{1 + y^2}\in(0, 1]\Rightarrow \frac{10}{1 + y^2}\in(0, 10]$

  $\Rightarrow \Re(z_1)\in \left(-\frac{1}{8}, \frac{9}{8}\right]\Rightarrow \alpha =
  -\frac{1}{8}, \beta = \frac{9}{8}$

  $\Rightarrow \beta  - \alpha = \frac{10}{8}$
  %285
\item Let $x + iy$ then $S = (x -  2)^2 + (y - 3)^2 - (x - 3)^2 - (y - 4)^2 = 2$

  $= 2x + 2y -12 = 2\Rightarrow x + y = 7$.
  %286
\item $(z - 2i)(\overline{z} + 2i) = 4(z + i)(\overline{z} - i)[\because |z|^2 = z\overline{z}]$

  $\Rightarrow 3z\overline{z} - 6i(z - \overline{z}) = 0 \Rightarrow x^2 + y^2 + 4y = 0$.
  %287
\item Let $z = x + iy\Rightarrow \alpha z = (8x + 14y) + i(-14x + 8y)$

  $A: \frac{2i(-14x + 8y)}{i(4xy -112)}= 1\Rightarrow (x - 4)(y + 7) = 0$

  $B: x^2 + (y + 3)^2 = 16$

  When $x = 4, y = -3$ and when $y = -7, x = 0$

  $\Rightarrow \displaystyle\sum_{\mstack z\in A\cap B} \Re(z) - \Im(z) = 4 - (-3) + 0 - (-7) = 14$.
  %288
\item Let $z = x = iy$, then $|z - 1| = 1\Rightarrow (x - 1)^2 + y^2 = 1$ and $(\sqrt{2} - 1)(z
  + \overline{z}) - i(z - \overline{z}) = 2\sqrt{2} \Rightarrow (\sqrt{2} - 1)x + y = \sqrt{2}$

  Solving the two equations we have $x = 1, \frac{1}{2 - \sqrt{2}}$ which gives $y = 1$ and $y = \sqrt{2}
  - \frac{1}{\sqrt{2}}$ respectively.

  $\Rightarrow |\sqrt{2}z_1 - z_2|^2 = 2$.
  %289
\item $|z + 2 - 3i|\leq 1$ represents inner region of a circle centered at $(-2, 3)$ with radius $1$.

  Given that $z(1 + i) + \overline{z}(1 - i)\leq -8\Rightarrow (x + iy)(1 + i) + (x - iy)(1 - i)\leq -8$

  $\Rightarrow x - y + ix + iy + x - y -ix -y \leq -8 \Rightarrow x - y \leq -4$

  $z - 3 + 2i$ is a straight line $x + y - 1 = 0$

  The diagram is shown below:

  \startplacefigure[location=force]
    \startMPcode
      numeric u;
      u := .6cm;

      path c;
      c = fullcircle scaled(2u) shifted(-2u, 3u) cutbefore (-u, 3u) cutafter(-2u, 2u);
      fill c -- (-u, 3u) -- cycle withcolor (0.8, 0.8, 0.8);;
      draw fullcircle scaled (2u) shifted (-2u, 3u);
      drawdot(3u, -2u) withpen pencircle scaled 2pt;
      draw (u, 5u) -- (-4u, 0);
      draw (-4u, 5u) -- (4u, -3u);
      drawdblarrow (-5u, 0) -- (4u, 0);
      drawdblarrow (0, -4u) -- (0, 5u);

      drawdot(-2u - u/sqrt(2), 3u + u/sqrt(2)) withpen pencircle scaled 2pt;
      drawdot (-3u/2, 5u/2) withpen pencircle scaled 2pt;

      label.lft("$X'$", (-5u, 0));
      label.rt("$X$", (4u, 0));
      label.top("$Y$", (0, 5u));
      label.bot("$Y'$", (0, -4u));
      label.llft("$O(0, 0)$", (0, 0));
      label.top("$z_1$", (-2u - u/sqrt(2), 3u + u/sqrt(2)));
      label.bot("$z_2$", (-3u/2, 5u/2) - (0, .3u));
      label.top ("$x - y = -4$", (-3u, u)) rotated angletoright((-2u, 2u), (-3u, u)) shifted (-.4u, 2.4u);
      label.top ("$x + y = 1$", (4u, -3u)) rotated angletoright((4u, -3u), (3u, -2u)) shifted (2u, 2u);
    \stopMPcode
  \stopplacefigure

  Solving the equation $x - y\leq -4$ with circle gives us $z_1 = \left(-2 -\frac{1}{\sqrt{2}}, 3
  + \frac{1}{\sqrt{}}\right)$ and solving the two lines we get $z_2
  = \left(-\frac{3}{2}, \frac{5}{2}\right)$

  Thus, $|z_1|^2 + 2|z_2|^2 = 31 + 5\sqrt{2}\Rightarrow \alpha + \beta = 36$.
  %290
\item Shaded area represents the area asked for.

  Shaded area is given by $=$ Area of semi circle - Area of sector

  $= \frac{1}{2}\pi.2^2 - \frac{\pi}{2} = \frac{3\pi}{2}$

  The diagram is given below:

  \startplacefigure[location=force]
    \startMPcode
      numeric u;
      u := .75cm;

      path c;
      c = fullcircle scaled(4u) shifted(u, 0) cutbefore (u + u*sqrt(2), u*sqrt(2)) cutafter(-u, 0);
      fill c -- (u, 0) -- (u + u*sqrt(2), u*sqrt(2)) -- cycle withcolor (0.8, 0.8, 0.8);;
      drawdot(3u, 2u) withpen pencircle scaled 2pt;
      draw fullcircle scaled (4u) shifted (u, 0);
      draw (-2u, -3u) -- (3u, 2u);
      drawdblarrow (-2.2u, 0) -- (3.2u, 0);
      drawdblarrow (0, -2.2u) -- (0, 2.2u);

      label.lft("$X'$", (-2.2u, 0));
      label.rt("$X$", (3.2u, 0));
      label.top("$Y$", (0, 2.2u));
      label.bot("$Y'$", (0, -2.2u));
      label.llft("$O(0, 0)$", (0, 0));
      label.bot("$C(1, 0)$", (u, 0));
      label.top ("$x - y = 1$", (-2u, -3u)) rotated angletoright((-2u, -3u), (3u, 2u)) shifted (-1.4u, 1u);
      draw angle_mark((3u, 0), (u, 0), (2u, u), 10);
      label.urt("$45^\circ$", (u, 0) + (.3u, 0.1u));
    \stopMPcode
  \stopplacefigure
  %291
\item $S_1: x^2 + y^2\leq 25, S_2: \Im\left(\frac{z + 1 - \sqrt{3}i}{1 - \sqrt{3}i}\right)
  = \Im\left(\frac{(x + iy)(1 + \sqrt{3}i)}{4}\right)\geq 0\Rightarrow \sqrt{3}x + y\geq 0$

  $S_3: x \geq 0$

  The diagram is given below:

    \startplacefigure[location=force]
    \startMPcode
      numeric u;
      u := .4cm;

      path c;
      c := fullcircle scaled(10u) cutbefore (5u/2, -5u*sqrt(3)/2) cutafter (0, 5u);
      fill c -- (0, 0) -- (5u/2, -5u*sqrt(3)/2) -- cycle withcolor (0.8, 0.8, 0.8);;
      c := fullcircle scaled(10u) cutbefore (5u, 0) cutafter (0, 5u);
      fill c -- (0, 0) -- cycle withcolor (0.8, 0.8, 0.8);;
      draw fullcircle scaled (10u) shifted (0, 0);
      drawdblarrow (-6u, 0) -- (6u, 0);
      drawdblarrow (0, -6u) -- (0, 6u);
      draw (6u/2, -6u*sqrt(3)/2) -- (-6u/2, 6u*sqrt(3)/2);

      label.lft("$X'$", (-6u, 0));
      label.rt("$X$", (6u, 0));
      label.top("$Y$", (0, 6u));
      label.bot("$Y'$", (0, -6u));
      label.llft("$O(0, 0)$", (0, 0));
      label ("$y = \sqrt{3}x$", (-5u/2, 5u*sqrt(3)/2)) rotated angletoright((5u/2, -5u*sqrt(3)/2),
      (-5u/2, 5u*sqrt(3)/2)) shifted (-2u +u, -2.5u*sqrt(3) - 2u*sqrt(3));
      draw angle_mark((5u/2, -5u*sqrt(3)/2), (0, 0), (u, 0), 10);
      label.lrt("$60^\circ$", origin + (.6u, -0.2u));
    \stopMPcode
  \stopplacefigure
  %292
\item $S = 1 + \frac{x}{2\sqrt{3}} + \frac{x^2}{18} + \frac{x^3}{36\sqrt{3}} + \frac{x^4}{180} + \cdots$ to
  $\infty$

  Let $\frac{x}{\sqrt{3}} = t$, where $x = \sqrt{3} - \sqrt{2}$, then $S$ becomes

  $S = 1 + \frac{t}{2} + \frac{t^2}{6} + \frac{t^3}{12} + \frac{t^4}{20} + \cdots$

  $= 1 + \left(1 - \frac{1}{2}\right)t + \left(\frac{1}{2} - \frac{1}{3}\right)t^2 + \left(\frac{1}{3}
  - \frac{1}{4}\right)t^3 + \left(\frac{1}{4} - \frac{1}{5}\right)t^4 + \cdots$

  $= \left(t + \frac{t^2}{2} + \frac{t^3}{3} + \cdots\right) - \frac{1}{t}\left(t + \frac{t^2}{2}
  + \frac{t^3}{2} + \cdots\right) +2$

  $= 2 + \left(1 - \frac{1}{t}\right)[-\log(1 - t)] = \left(\frac{1}{t} - 1\right)\log(1 - t) + 2$

  $= 2 + \left(\sqrt{\frac{3}{2}} + 1\right)\log_e\frac{2}{3}\Rightarrow a, b = 2, 3$.
  %293
\item $4x^4 + 8x^3 - 17x^2 - 12x + 9 = 4(x - x_1)(x - x_2)(x - x_3)(x - x_4) =0$

  Putting $x = \pm 2i$ and multuplying equations

  $(4 + x_1^2)(4 + x_2)^2(4 + x_3^2)(4 + x_4)^2 = \frac{141^2 + 88^2}{16}$

  Thus, $\frac{125m}{16} = \frac{141^2 + 88^2}{16}\Rightarrow m = 221$.
  %294
\item Given that $|z + 2| = 1$ and $\Im\left(\frac{z + 1}{z + 2}\right)= \frac{1}{5}$

  Let $z + 2 = \cos\theta + i\sin\theta\Rightarrow \frac{1}{z + 2} = \cos\theta - i\sin\theta$

  $\Rightarrow \frac{z + 1}{z + 2} = 1 - \frac{1}{z + 2} = (1 - \cos\theta) + \i\sin\theta$

  $\Rightarrow \Im\left(\frac{z + 1}{z + 2}\right) = \sin\theta\Rightarrow \sin\theta = \frac{1}{5}$

  $\Rightarrow \cos\theta = \pm\frac{2\sqrt{6}}{5}\Rightarrow |\Re(\overline{z + 2})|
  = \frac{2\sqrt{6}}{5}$.
  %295
\item Given $a + 5b = 42$ and $a, b\in\mathbb{N}$. We see that $b$ can have $8$ values from $1$ through $8$.

  Now $\displaystyle\sum_{n = 1}^n\left(1 - i^{n!}\right) = x + iy$. For $n \geq 4, 1 - i^{n!} = 0$

  So $x + iy = (1- i) + \left(1 - i^2\right) + \left(1 - i^6\right) = 5 - i\Rightarrow x = 5, y = -1$

  $\therefore m + x + y = 12$.
  %296
\item $|z - 1|\leq 1\Rightarrow (x - 1)^2 + y^2\leq 1$, which represnts the equation of a circle with center
  $(1, 0)$ and radius $1$.

  $|z - 5|\leq |z - 5i|\Rightarrow x\geq y$

  Solving the twp equations $x = 0$ or $x = 1$ which gives us $y = 0$ and $y = 1$.

    \startplacefigure[location=force]
    \startMPcode
      numeric u;
      u := 1cm;

      path c;
      c := fullcircle scaled(2u) shifted (u, 0) cutbefore (0, 0) cutafter (u, u);
      fill c .. (u, u) -- cycle withcolor (0.8, 0.8, 0.8);;
      draw fullcircle scaled (2u) shifted (u, 0);
      drawdblarrow (-2u, 0) -- (2.5u, 0);
      drawdblarrow (0, -2u) -- (0, 2u);
      draw (2u, 2u) -- (-u, -u);

      label.lft("$X'$", (-2u, 0));
      label.rt("$X$", (2.5u, 0));
      label.top("$Y$", (0, 2u));
      label.bot("$Y'$", (0, -2u));
      label.llft("$O(0, 0)$", (0, 0));
      label.bot("$C(1, 0)$", (u, 0));
      label.rt("$y = x$", (2u, 2u));
    \stopMPcode
  \stopplacefigure

  Since $a, b\in\mathbb{Z}$ the points are $(0, 0), (1, 0), (2, 0), (1, 1), (1, -1)$

  So sum of squares of modulus comes out to be $9$.
  %297
\item Since $\frac{z - 2i}{z + 2i}$ has real part as zero so $\frac{z - 2i}{z + 2i} + \frac{\overline{z} +
  2i}{\overline{z} - 2i} = 0$

  $\Rightarrow z\overline{z} - 2i\overline{z} - 2iz + 4(-1) = 0\Rightarrow |z| = 2$

  $\therefore |z - (6 + 8i)|_{\max} = |z| + |6 + 8i| = 12$.
  %298
\item The given points in the equation represents points $(0, 1), (0, -1)$ and $(1, 0)$ i.e. a
  triangle. Only circumcenter of this triangle is the point equidistant from the three points. Hence, $n(S)
  = 1$.
  %299
\item $|z - z_0|^2 = 4\Rightarrow \left(\alpha - z_0\right)\left(\overline{\alpha} - \overline{z_0}\right) =
  4$

  $\Rightarrow |\alpha|^2 - \alpha\overline{z_0} - \overline{\alpha}z_0 = 2$

  Also, $|z - z_0|^2 = 16 \Rightarrow \left(\frac{1}{\overline{\alpha}} - z_0\right)\left(\frac{1}{\alpha}
  - \overline{z_0}\right) = 16$

  $\Rightarrow 1 - \overline{\alpha}z_0 - \alpha\overline{z_0} = 14|\alpha|^2$

  Combining the two equations obtained $5|\alpha|^2 = 1\Rightarrow 100|\alpha|^2 = 20$.
  %300
\item Given that $z^3 + 2z^2 + 2z + 1 = 0 \Rightarrow (z + 1)(z^2 + z + 1) = 0$, which has three roots.

  $z = -1, \omega, \omega^2$ out of which $\omega$ and $\omega^2$ satisfy both the equations.
  %301
\item Given that $|1 - i|^x = 2^x \Rightarrow \left(\sqrt{2}\right)^x = 2^x\Rightarrow x =
  0 \Rightarrow \alpha = 1$

  Also given that $z = \frac{\pi}{4}(1 + i)^4\left(\frac{1 - \sqrt{\pi}i}{\sqrt{\pi} + i} + \frac{\sqrt{\pi}
    - i}{1 + \sqrt{\pi}i}\right)$

  $= \frac{\pi}{4}(1 + i)^4\left[\frac{\sqrt{\pi} - \pi i - i - \sqrt{\pi}}{\pi + 1} + \frac{\sqrt{\pi} - i
      - \pi i-\sqrt{\pi}}{\pi + 1}\right]$

  $= \frac{\pi}{4}\left(1 + 4i + 6i^2 + 4i^3 + 1\right) = 2\pi i$

  Thus, $\beta = 4$

  Distance of $(1, 4)$ from $4x - 3y = 7$ is $3$.
  %302
\item Given that $z_1 + z_2 = 5$ and $z_1^3 + z_2^3 = 20 + 15i\Rightarrow (z_1 + z_2)^3 = 20 + 15i +
  3z_1z_2(z_1 + z_2)$

  $\Rightarrow 125 = 20 + 15i - 15z_1z_2 \Rightarrow z_1z_2 = 7 - i$

  $\Rightarrow (z_1 + z_2)^2 = 25\Rightarrow z_1^2 + z_2^2 = (z_1 + z_2)^2 - 2z_1z_2 = 25 - 14 + 2i = 11 +
  2i$

  $\Rightarrow \left(z_1^2 + z_2^2\right)^2 = 117 + 44i \Rightarrow \left|z_1^4 + z_2^4\right| = 75$.
  %303
\item Given that $\frac{2 + k^2z}{k + \overline{z}} = kz\Rightarrow |z|^2k = 2 \Rightarrow k = 2$.

  So the point is $P(2, 4)$. The given circle has center $C(1, 2)$ and radius of $1$.

  So the maximum distance of point $P$ from circle is $OP + r = \sqrt{1 + 4} + 1 = \sqrt{5} + 1$.
  %304
\item Given that $\frac{z^2 + 3i}{z - 2 + i} = 2 + 3i\Rightarrow z^2 - z(2 + 3i) + 7 + 7i = 0$

  This is a quadratic equation which has two roots $z_1$ and $z_2$ such that

  $z_1 + z_2 = 2 + 3i$ and $z_1z_2 = 7 + 7i$

  Thus, $z_1^2 + z_2^2 = \left(z__1^2 + z_2^2\right) = \left(z_1 + z_2\right)^2 - 2z_1z_2 = -19 -2i$.
  %305
\item We know that for an equilateral triangle $z_1^2 + z^2 + z_3^2 = z_1z_2 + z_2z_3 + z_3z_1$

  Given that $z_1 + z_2 + z_2 = 3z_0 \Rightarrow \left(z_1 + z_2 + z_3\right)^2 = 9z_0^2 \Rightarrow z_1^2 +
  z_2^2 + z_3^2 = 3z_0^2$

  $\displaystyle\sum_{k = 1}^3\left(z_k - z_0\right)^2 = z_1^2 + z_2^2 + z_3^2 + 3z_0^2 - 2\left(z_1 + z_2 +
  z_3\right)z_0 = 6z_0^2 - 6z_0^2 = 0$.
  %306
\item Let $z = x + iy$. Gven that $A: = |z - 2 - i| = 3\Rightarrow (x - 2)^2 + (y - 1)^2 = 9$

  Also given that $B: \Re(z - iz) = 2 \Rightarrow x + y = 2$

  Solving the two equations we have $x = \frac{3\pm \sqrt{17}}{2}, y = \frac{1\mp\sqrt{17}}{2}$

  $\therefore \displaystyle\sum_{\mstack m\in S} |z|^2 = \frac{1}{4}(2\times 26 + 2\times 16) = 22$.
  %307
\item $\omega_1\omega_2 = i\left(65\sin^2\theta + 120\sin\theta\cos\theta + 65\cos^2\theta\right)$

  $\Rightarrow \alpha +i\beta = i(65 + 60\sin2\theta)$

  $\Rightarrow \alpha = 0, \beta_{\max} = 125$ and $\beta_{\min} = 5$

  $\therefore p + q = 130$.
  %308
\item $\displaystyle\sum_{k = 1}^n\left(\omega^{2k} + \omega^k + 2\right) = 20$

  $\Rightarrow \left(\omega^2 + \omega^4 + \omega^6 + \cdots + \omega^{2n}\right)  + \left(\omega
  + \omega^2 + \omega^3 + \cdots + \omega^n\right) + 2n = 20$

  If $n = 3m, m\in\mathbb{I}$ then $0 + 0 + 2n = 20 \Rightarrow n = 10\neq 3m$

  If $n = 3m + 1$ then $\omega^2 + \omega + 2n = 20 \Rightarrow n = \frac{21}{2}$, which is not possible.

  If $n = 3m + 2$ then $2n = 22 \Rightarrow n = 11$, which is possible.
  %309
\item $|z| = 1$ and $\frac{z - 1}{z + 1} + \frac{\overline{z} - 1}{\overline{z} + 1} = 0$

  $\Rightarrow |z|^2 + \left(z - \overline{z}\right) - 1 + |z|^2 - \left(z - \overline{z}\right) - 1 = 0$

  $\Rightarrow |z|^2 = 1$. So there are infinite such points.
  %310
\item We observe that $\frac{z - 1}{2z + i} = \frac{\overline{\overline{z} + 1}}{\overline{2\overline{z} -
    i}}$

  $\therefore \Re\left(\frac{z - 1}{2z + i}\right) = 1$

  $\Re\left(\frac{x - 1 + iy}{2x + i(2y + 1)}\right) = 1 \Rightarrow \frac{2x(x - 1) + y(2y + 1)}{4x^2 + (2y
  + 1)^2} = 1$

  $\Rightarrow x^2 + y^2 + x + \frac{3}{2}y + \frac{1}{2} = 0$

  Thus, center of the circle is $\left(-\frac{1}{2}, -\frac{3}{4}\right)$ and radius $\frac{\sqrt{5}}{4}$.
  %311
\item Since $|z| = 1$ we can write $z_1 = e^{-i\frac{\pi}{4}}, z_2 = 1, z_3 = e^{i\frac{\pi}{4}}$

  $\therefore |z_1z_2 + z_2z_3 + z_1z_3|^2 = \left|e^{-i\frac{\i}{4}} + e^{i\frac{\pi}{4}} + 1\right|^2$

  $= \left|\frac{1}{\sqrt{2}} -\frac{i}{\sqrt{2}} + \frac{1}{\sqrt{2}} + \frac{i}{\sqrt{2}} + 1\right|^2 =
  \left|1 + \sqrt{2}\right|^2 = 3 + 2\sqrt{2}$

  Thus, $\alpha = 3, \beta = 2\Rightarrow \alpha^2 + \beta^2 = 13$.
  %312
\item The diagram is given below:

  \startplacefigure[location=force]
    \startMPcode
      numeric u;
      u := 1cm;

      path c;
      c := fullcircle scaled(2u) shifted (3u, 0) cutbefore (3u, u) cutafter (2u, 0);
      fill c -- (3u, u) -- cycle withcolor (0.8, 0.8, 0.8);;
      draw fullcircle scaled (2u) shifted (3u, 0);
      drawdblarrow (-u, 0) -- (5u, 0);
      drawdblarrow (0, -u) -- (0, 2u);
      draw (3.5u, 1.5u) -- (1.5u, -.5u);
      drawdot(3u, 0) withpen pencircle scaled 2pt;

      label.lft("$X'$", (-u, 0));
      label.rt("$X$", (5u, 0));
      label.top("$Y$", (0, 2u));
      label.bot("$Y'$", (0, -u));
      label.bot("$C(3, 0)$", (3u, 0));
      label.ulft("$(3, 1)$", (3u, u));
      label.ulft("$(2, 0)$", (2u, 0));
      label.rt("$x - y = 2$", (3.5u, 1.5u));
    \stopMPcode
  \stopplacefigure

  Let $z = x + iy \Rightarrow (x + iy)(1 + i) + (x - iy)(1 - i) = 4\Rightarrow x - y = 2$

  $|z - 3|\leq 1\Rightarrow (x - 3)^2 + y^2\leq 1$, which is inner part of a circle with center $(3, 0)$ and
  radius $1$.

  Area of shaded region is $\frac{\pi}{4}.1^2 - \frac{1}{2}1.1 = \frac{\pi}{4} - \frac{1}{2}$

  Area of unshaded region inside the circle is $\frac{3\pi}{4}.1^2 + \frac{1}{2}.1.1 = \frac{3\pi}{4}
  + \frac{1}{2}$

  $\therefore$ Difference of area is $\frac{\pi}{2} + 1$.
  %313
\item Given that $\left|\frac{\overline{z} - i}{2\overline{z} + i}\right|
  = \frac{1}{3}\Rightarrow \left|\frac{\overline{z - i}}{\overline{z} + \frac{i}{2}}\right| = \frac{2}{3}$

  $\Rightarrow 3|x - iy - i| = 2\left|x - iy + \frac{i}{2}\right|\Rightarrow x^2 + y^2 + \frac{22}{5}y
  + \frac{8}{5} = 0$

  Thus, center is at $\left(0, -\frac{11}{5}\right)$

  Area of triangle $\frac{1}{2}\startdeterminant\NC 0\NC 0\NC 1\NR\NC 0\NC -11/5\NC 1\NR\NC \alpha\NC 0\NC
  1\NR\stopdeterminant = 11$

  $\Rightarrow \alpha^2 = 100$.
  %314
\item Since $|z| = 1 \Rightarrow z = e^{i\theta}\Rightarrow \frac{z}{\overline{z}} = e^{i2\theta}$

  $\therefore \left|\frac{z}{\overline{z}} + \frac{\overline{z}}{z}\right| =
  1\Rightarrow \left|e^{i2\theta} + e^{-i2\theta}\right| = 1\Rightarrow |\cos2\theta| = \frac{1}{2}$

  There are $8$ solutions for the given condition in $(0, 2\pi)$.
  %315
\item From given equation $\alpha + \beta = a, \alpha\beta = b$

  $P_6 = aP_5 + bP_4 \Rightarrow 45 = 11a - 3b$ and $P_5 = aP_4 + bP_3 \Rightarrow 11 = -3a - 5b$

  Thus, $a = 3, b = -4\Rightarrow \left|\alpha^4 + \beta^4\right| = 31$.
  %316
\item Given $2z^2 - 3z - 2i = 0\Rightarrow 2\left(z - \frac{i}{z}\right) = 3\Rightarrow \alpha
  - \frac{i}{\alpha} = \frac{3}{2}$

  $\Rightarrow \alpha^2 - \frac{1}{\alpha^2} - 2i = \frac{9}{4}\Rightarrow \alpha^4 + \frac{1}{\alpha^4}
  = \frac{49}{16} + 9i = \beta^4 + \frac{1}{\beta^4}$

  $\frac{\alpha^{19} + \beta^{19} + \alpha^{11} + \beta^{11}}{\alpha^{15} + \beta^{15}}
  = \frac{\alpha^{15}\left(\alpha^4 + \frac{1}{\alpha^4}\right) + \beta^{15}\left(\beta^4
    + \frac{1}{\beta^4}\right)}{\alpha^{15} + \beta^{15}}$

  $= \frac{\left(\alpha^{15} + \beta^{15}\right)\left(\frac{49}{16} + 9i\right)}{\alpha^{15} + \beta^{15}} =
  \frac{49}{16} + 9i$

  $\therefore \Re\left(\frac{\alpha^{19} + \beta^{19} + \alpha^{11} + \beta^{11}}{\alpha^{15}
  + \beta^{15}}\right).\Im\left(\frac{\alpha^{19} + \beta^{19} + \alpha^{11} + \beta^{11}}{\alpha^{15}
    + \beta^{15}}\right) = \frac{441}{16}$.
  %317
\item $z_2 = \frac{|z_2|}{|z_1|}.z_1.e^{i\pi/6} = \frac{(3 - 2\sqrt{2}) + i(2\sqrt{6}
  + \sqrt{3})}{2\sqrt{3}}$

  We find that $|z_1 - z_2| = |z_2| \Rightarrow \triangle ABO$ is an isosceles triangle with angles
  $\pi/6, \pi/6, 2\pi/3$.
  %318
\item $\alpha = 1 + \displaystyle\sum_{k = 1}^6(-3)^{r - 1}C_{2r - 1}^^{12} = 1 + \displaystyle\sum_{r =
  1}^6C_{2r - 1}^^{12}\frac{\left(\sqrt{3}i\right)^{2r - 1}}{\sqrt{3}i}$

  $= 1 + \frac{1}{\sqrt{3}i}\displaystyle\sum_{r = 1}^6\left(C_1^^{12}x + C_{2}^^{12}x^2 + \cdots +
  C_{11}^^{12}x^{11}\right)$

  $= 1 + \frac{1}{\sqrt{3}i}\left(\frac{\left(1 + \sqrt{3}i\right)^{12} - \left(1
    - \sqrt{3}i\right)^{12}}{2}\right) = 1 + \frac{1}{\sqrt{3}i}\left(\frac{\left(-2\omega^2\right)^{12}
    - \left(2\omega\right)^{12}}{2}\right) = 1$

  So distance of $\left(12, \sqrt{3}\right)$ from $x - \sqrt{3}y + 1 = 0$ is $\frac{12 - \sqrt{3}.\sqrt{3} +
    1}{1 + \left(\sqrt{3}\right)^2} = 5$.
  %319
\item Let the centers be $A(8, 2)$ and $B(2, -6)$. The radii are $1$ and $2$ respectively.

  $AB = \sqrt{(8 - 2)^2 + (2 + 6)^2} = \sqrt{100} = 10$

  $\Rightarrow |z_1 - z_2|_{\min} = 10 - 2 - 1 = 7$.

  The diagram is given below:

  \startplacefigure[location=force]
    \startMPcode
      numeric u;
      u := .5cm;

      draw fullcircle scaled (2u) shifted (8u, 2u);
      draw fullcircle scaled (4u) shifted (2u, -6u);
      drawdblarrow (-u, 0) -- (8u, 0);
      drawdblarrow (0, -8u) -- (0, 3u);
      draw (8u, 2u) -- (2u, -6u);
      drawdot(8u, 2u) withpen pencircle scaled 2pt;
      drawdot(2u, -6u) withpen pencircle scaled 2pt;

      label.lft("$X'$", (-u, 0));
      label.rt("$X$", (8u, 0));
      label.top("$Y$", (0, 3u));
      label.bot("$Y'$", (0, -8u));
      label.bot("$A(8, 2)$", (8u, 2u));
      label.ulft("$B(2, -6)$", (2u, -6u));
    \stopMPcode
  \stopplacefigure
  %320
\item We have been given $|z - a| = |z + b|$ and $\startdeterminant\NC z +
  1\NC \omega\NC \omega^2 \NR\NC \omega\NC z + \omega^2 \NC 1\NR\NC \omega^2\NC 1\NC z
  + \omega\NR\stopdeterminant = 1$

  $\Rightarrow \startdeterminant\NC z \NC z\NC z \NR\NC \omega\NC z + \omega^2 \NC 1\NR\NC \omega^2\NC 1\NC
  z + \omega\NR\stopdeterminant = 1[R_1\rightarrow R_1 + R_2 + R_3]$

  $\Rightarrow z\startdeterminant\NC 1\NC 0\NC 0\NR\NC \omega\NC z + \omega^2 - \omega\NC 1
  - \omega\NR\NC \omega^2 \NC 1 - \omega^2\NC z + \omega -\omega^2\NR\stopdeterminant = 1[C_2\rightarrow C_2
    - C_1; C_3\rightarrow C_3 - C_1]$

  $\Rightarrow z^3 = 1 \Rightarrow z = 1, \omega, \omega^2$

  But we have $|z - a| = |z + b|$ along with the fact that $a, b\in\mathbb{I}$.

  So we have $10$ ordered pairs, which can be enumerated under the condition $a,b\in[-3, 3]$.
  %321
\item Given that $ |x + iy - z_1| = 2|x + iy - z_2| \Rightarrow (x - 1)^2 + (y - 2)^2 = 4 [x^2 + (y - 3)^2]$

  $\Rightarrow x^2 - 2x + 1 + y^2 - 4y + 4 = 4x^2 + 4y^2 - 24y + 36 \Rightarrow 3x^2 + 3y^2 + 2x - 20y + 35
  = 0$

  This is equation of a circle with center at $\left(-\frac{1}{3}, \frac{10}{3}\right)$ having radius
  $\frac{2\sqrt{2}}{3}$.
  %322
\item Discriminant of the given equation is given by $D = \frac{25}{4}(l + m)^2 + 8(l - m)^2$, which is sum
  of two squares and hence will be greater than zero because $l\neq m$. Thus, roots will be real and
  unequal.
  %323
\item If $x > 0$ the equation is $x^2 - 3x + 2 = 0$, which has two real roots $1$ and $2$.

  If $x < 0$ the equation is $x^2 + 3x + 2 = 0$, which also will have two real roots because $D = 1 \geq 0$.
  Thus the given equation has four real solutions.
<<<<<<< HEAD
  %324
\item Let $z = x + iy$ then $az^2 + bz = a\left(x^2 - y^2 + 2xyi\right) + b(x + iy)$

  $\Re\left(az^2 + bz\right) = a \Rightarrow a\left(x^2 - y^2\right) + bx = a$

  and $bz^2 + az = b\left(x^2 - y62 + 2xyi\right) + a(x + iy)$

  $\Re\left(bz^2 + az\right) = b \Rightarrow b\left(x^2 - y^2\right) + ax = b$

  Multiplying the obtained equations with $b$ and $a$ and subtracting we obtain

  $b^2x - a^2x = 0 \Rightarrow x = a^2 - b^2$. We also have $2xya + by = 0 \Rightarrow y = 0$ from given
  conditions.

  So we have one solution for the given set of conditions.
  %325
=======
>>>>>>> b94a7b0766448d1968b0570a496b041ba224dcb4
\item $(a + b + c)^2 = a^2 + b^2 + c^2 + 2ab + 2bc + 2ca = 1 + 2(ab + bc + ca)\geq 0$

  $\Rightarrow ab + bc + ca\geq -\frac{1}{2}$

  Also, $(a - b)^2 + (b - c)^2 + (c - a)^2\geq 0 \Rightarrow 2\left(a^2 + b^2 + c^2\right) - 2(ab + bc +
  ca)\geq 0$

  $\Rightarrow ab + bc + ca \leq 1$

  So the range is $\left[-\frac{1}{2}, 1\right]$.
  %326
\item We have been given that $x^{3/4\left(\log_2x\right)^2 + \log_2x - 5/4} = \sqrt{2}$

  Taking $\log$ of both sides with base $2$ and putting $\log_2x = y$

  $\frac{3}{4}y^3 + y^2 - \frac{5}{4}y = \frac{1}{2}\Rightarrow 3y^3 + 4y^2 - 5y - 2 = 0$

  We observe that sum of coefficients is zero, so $y = 1$ is one of the roots.

  $\Rightarrow 3y^3 - 3y^2 + 7y^2 - 7y + 2y - 2 = 0\Rightarrow (y - 1)\left(3y^2 + 7y + 2\right) = 0$

  $\Rightarrow (y - 1)(y + 2)(3y + 1) = 0$

  So $\log_2x = 1, -2, -\frac{1}{3}\Rightarrow x = 2, \frac{1}{4}, \frac{1}{\sqrt[3]{2}}$.
  %327
\item We can write $\frac{p^2 + p + 1}{p}$ as $p + 1 + \frac{1}{p}$. We know that $x + \frac{1}{x}\geq 2$
  using A.M.-G.M.\ inequality.

  So minimum value of $p + 1 + \frac{1}{p}$ is $3$. Thus, minimum value of given expression is $81$.
  %328
\item Let the roots of the first equation are $\alpha, \beta$ and that of second equation are $\alpha +
  k, \beta + k$.

  $\Rightarrow \alpha + \beta = b, \alpha\beta = c, \alpha + \beta + 2k = c = \alpha\beta, (\alpha +
  k)(\beta + k) = b = \alpha + \beta$

  $b + c = \alpha + \beta + \alpha\beta = \alpha + \beta + 2k + (\alpha + k)(\beta + k)$

  $\Rightarrow k(k + 2 + \alpha + \beta) = 0$. Now since $k\neq 0 \Rightarrow k = -2 - \alpha - \beta = -2 -
  b$

  We have $\alpha + \beta + 2k = c \Rightarrow b - 2b - 4 = c \Rightarrow b + c = -4$.
  %329
\item Let $\alpha, \beta$ be the of the given equation. $\Rightarrow \alpha + \beta =
  -\frac{m}{2l}, \alpha\beta = \frac{n}{l}$

  Let $\gamma$ be the A.M.\ of the roots. $\Rightarrow \gamma = -\frac{m}{2l}$, and $\delta$ is the H.M.\ of
  the roots. $\Rightarrow \delta = \frac{2\alpha\beta}{\alpha + \beta} = -\frac{2n}{m}$

  Equation whose roots are $\gamma, \delta$ is $x^2 - \left(\gamma + \delta\right)x + \gamma\delta = 0$

  $\Rightarrow 2lmx^2 + \left(m^2 + 4nl\right)x + 2mn = 0$.
  %330
\item Since the roots are of opposite sign the product of the roots will be negative.

  $\Rightarrow \frac{a^2 - 4a}{2} < 0 \Rightarrow 0 < a < 4$.
  %331
\item $D = 324 \geq 0$ and coefficient of $x^2$ is negative. Thus, $7 + 10x - 8x^2 \leq 0\Rightarrow (7 -
  4x)(1 + 2x)\leq 0$

  Thus, maximum value of $x$ is $\frac{7}{4}$.
  %332
\item Given $\alpha + \beta = 2\Rightarrow \alpha^2 + \beta^2 + 2\alpha\beta = 4\Rightarrow \alpha^2
  + \beta^2 = 4 - 2\alpha\beta$

  $\Rightarrow \left(\alpha^2 + \beta^2\right)^2 - 2\alpha^2\beta^2 = 272 \Rightarrow \left(4 -
  2\alpha\beta\right)^2 - 2\alpha^2\beta^2 = 272$

  $\Rightarrow 16 - 16\alpha\beta + 2\alpha^2\beta^2 = 272 \Rightarrow \alpha^2\beta^2 - 8\alpha\beta - 128
  = 0$

  $\Rightarrow \alpha\beta = 16, -8$

  Thus, we have two possible equations $x^2 - 2x + 16 = 0$ and $x^2 - 2x - 8 = 0$

  However, $x^2 - 2x + 16 = 0$ has non-real roots. So $x^2 - 2x - 8 = 0$ is only possible equation.
  %333
\item We have been given that $(x - a)(x - b) + (x - b)(x - c) + (x - c)(x - a) = 0$

  $\Rightarrow x^2 - 2(a + b + c)x + ab + bc + ca = 0$

  Discriminant $D = 4(a + b + c)^2 - 4(ab + bc + ca) = 2[(a - b)^2 + (b - c)^2 + (c - a)^2]\geq 0$

  Since discriminant is always greater than zero, therefore, both the roots will be real.
  %334
\item Let $\alpha, \beta$ be two roots. Then $\alpha, \beta = \frac{3\pm \sqrt{9 - 4k}}{2}$

  Since roots are real, therefore, $9 - 4k \geq 0$. The extreme values of $D$ are $9$ and $0$.

  If $D = 9, \alpha, \beta = 3, 0$ and if $D = 0, \alpha, \beta = \frac{3}{2}$

  If $k = 2, \alpha\beta = 2, 1$. Thus we see that if $k$ is low the higher side of roots goes above $3$ and
  if $k$ is high then lower side of the roots goes above $0$.

  Thus, there is no such $k$ for which roots will lie in the range $(0, 1)$.
  %335
\item From the question $a + b = -20, ab = -2020, c + d = 20, cd = 2020$

  $ac(a - c) + ad(a - d) + bc(b - c) + bd(b - d) = a^2(c + d) - a\left(c^2 + d^2\right) + b^2(c + d) - b(c^2
  + d^2)$

  $= (c + d)\left[\left(a + b\right)^2 - 2ab\right] - (a + b)\left[\left(c + d\right)^2 - 2cd\right]$

  $= 16000$.
  %336
\item Given that $\alpha$ is a common root for the given equations, so both equations must satisfy

  $\alpha^2 - \alpha + 2\lambda = 0$ and $3\alpha^2 - 3\alpha + 6\lambda = 0$

  Eliminating term with $\alpha^2$, we have $\alpha = 3\lambda$

  $\Rightarrow 9\lambda^2 - 3\lambda + 2\lambda = 0\Rightarrow \lambda = \frac{1}{9}[\because \lambda\neq
    0]\Rightarrow \alpha = \frac{1}{3}$

  $\Rightarrow \alpha\beta = 2\lambda \Rightarrow \beta = \frac{2}{3}$ and $\alpha\gamma = 9\lambda
  \Rightarrow \gamma = 3$

  $\Rightarrow \frac{\beta\gamma}{\lambda} = 18$.
  %337
\item Since root has to be real the other root will also be real as it cannot be complex or irrational.

  $\therefore D > 0 \Rightarrow 16\lambda^2 - 4\left(\lambda^2 + 1\right)2 > 0 \Rightarrow \lambda^2 >
  1\Rightarrow \lambda\in(-\infty, 1)\cup (1, \infty)$

  Because root has to lie in $(0, 1)\Rightarrow f(0)f(1) < 0 \Rightarrow 2\left(\lambda^2 + 1 - 4\lambda +
  2\right) < 0$

  $\Rightarrow \lambda\in(1, 3)$

  Combining both the sets we have $\lambda\in(1, 3)$.
  %338
\item From the questions $\alpha + \beta = -\sin\theta, \alpha\beta = 2\sin\theta = 2(\alpha + \beta)$

  We have to find $\dfrac{\left(\alpha^{12} +\beta^{12}\right)}{\left(\alpha^{-12} +
    \beta^{-12}\right)(\alpha - \beta)^{24}}$

  $= \dfrac{\alpha^{12} + \beta^{12}}{\left(\frac{1}{\alpha^{12}} + \frac{1}{\beta^{12}}\right)(\alpha -
    \beta)^{24}} = \left[\frac{\alpha\beta}{(\alpha - \beta)^2}\right]^{12}$

  $= \left(\frac{\alpha\beta}{(\alpha + \beta)^2 - 4\alpha\beta}\right)^{12} = \frac{2^{12}}{(\sin\theta +
    8)^{12}}$.
  %339
\item Let $\alpha, \beta$ be roots of the equation. Then, $\alpha + \beta = \frac{3}{m^2 + 1}$.

  It is given that the sum has to be greatest that implies that $m^2 + 1$ has to minimum. Thus, $m = 0$

  Then $\alpha + \beta = 3, \alpha\beta = 1$

  Absolute difference of the cubes of roots $= \left|\alpha^3 - \beta^3\right| = |\alpha -
  \beta|\left|\alpha^2 + \beta^2 + \alpha\beta\right|$

  $= \sqrt{(\alpha + \beta)^2 - 4\alpha\beta}\left|(\alpha + \beta)^2 - \alpha\beta\right| = 8\sqrt{5}$.
  %340
\item Given equation is $x^2 - 2x - 2 = 0 \Rightarrow (x - 1)^2 = -1 \Rightarrow x = 1\pm i$

  Now $\left(\frac{\alpha}{\beta}\right)^n = \left(\frac{1 + i}{1 - i}\right)^n = \left(\frac{(1 + i)^2}{1 +
    1}\right)^n = i^n$, which will be $1$ for $n = 4$.
  %341
\item $D = [-2(1 + 3m)]^2 - 4(1 + m^2)(1 + 8m) = -8m(2m - 1)^2$

  Therefore, $D$ will be less than zero, which is the condition for having no real root, if $m > 0$.

  Thus, there are infinite integral values for $m$, which satisfy the requirement of the equation.
  %342
\item For the given expression to be positive $1 + 2m > 0$ and $D < 0$

  $\Rightarrow (1 + 3m)^2 - (2m + 1)(1 + m) < 0 \Rightarrow m^2 - 6m - 3 < 0$

  $\Rightarrow 3 - \sqrt{12} < m < 3 + \sqrt{12}$

  Combining both the inequalities obtained we find that $m = 0, 1, 2, 3, 4, 5, 6$ i.e. $7$ integral values
  satisfy the requirement.
  %343
\item Let $\alpha, \beta$ be the roots of the given quadratic equation. Then $\alpha + \beta = \frac{m(m -
  4)}{3m^2}$ and $\alpha\beta = \frac{2}{3m^2}$

  Given that $\lambda = \frac{\alpha}{\beta}$ and we have to satisfy that $\lambda + \frac{1}{\lambda} = 1$

  $\Rightarrow \frac{\alpha}{\beta} + \frac{\beta}{\alpha} = 1\Rightarrow \alpha^2 + \beta^2 = \alpha\beta$

  $\Rightarrow (\alpha + \beta)^2 = 3\alpha\beta \Rightarrow \frac{m^2(m - 4)^2}{9m^4} = \frac{6}{3m^2}$

  $\Rightarrow (m - 4)^2 = 18[\because m\neq 0$ else it won't remain a quadratic equation$]$

  $\Rightarrow m = 4\pm3\sqrt{}2$.  So the least value is $4 - 3\sqrt{2}$.
  %344
\item Let $\alpha, \beta$ be the roots of the given quadratic equation. Then $\alpha +\beta = \lambda - 3$
  and $\alpha\beta = 2 - \lambda$

  $\alpha^2 + \beta^2 = (\alpha + \beta)^2 - 2\alpha\beta = (\lambda - 3)^2 - 2(2 - \lambda) = (\lambda -
  2)^2 + 1$

  Thus, minimum value of $\lambda = 2$.
  %345
\item For the roots to be rattional, the discriminant must be perfect square.

  $D = 11^2 - 24k$. We observe that for $k = 3, 4, 5$ the discriminant is a perfect square.
  %346
\item Given equation is $x^2 + 2x + 2 = 0$, its roots are $x = -1\pm i = -\sqrt{2}\left(\cos\frac{\pi}{4}
  \pm\sin \frac{\pi}{4}\right)$

  Thus, $\alpha^{15} + \beta^{15} = -\left(\sqrt{2}\right)^{15}\left[2\cos\frac{15\pi}{4}\right] = -2^8 =
  -256$.
  %347
\item Let $\sqrt{x} - 3 = y \Rightarrow \sqrt{x} = y + 3$. Therefore, the given equation becomes

  $2|y| + (y + 3)(y - 3) + 6 = 0 \Rightarrow 2|y| + y^2 - 3 = 0\Rightarrow |y|^2 + 2|y| - 3 = 0$

  $\Rightarrow |y| = 1\Rightarrow x = 16, 4$.
  %348
\item Roots of $x^2 - x + 1 = 0$ are $-\omega$ and $-\omega^2$, where $\omega$ is a cube root of unity.

  $\therefore \alpha^{101} + \beta^{107} = -\omega^{101} -\omega^{214} = -\omega^2 - \omega = -(-1) = 1$.
  %349
\item Given quadratic equation is $x(x + 1) + (x + 1)(x + 2) + \cdots + (x + n - 1)(x + n) = 10n$

  $\Rightarrow \left(x^2 + x^2 + \cdots + x^2\right) + \left(1 + 3 + 5 + \cdots + 2n - 1\right)x + \left(1.2
  + 2.3 + \cdots + (n - 1)n \right) = 10n$

  $\Rightarrow nx^2 + n^2x + \frac{n\left(n^2 - 1\right)}{3} - 10n = 0\Rightarrow x^2 + nx + \frac{n^2 -
    1}{3} - 10 = 0$

  Let $\alpha$ and $\beta$ the roots of the given quadratic equation. Given that they are consecutive
  integegrs.

  $\Rightarrow |\alpha - \beta|^2 = 1\Rightarrow (\alpha - \beta)^2 = (\alpha + \beta)^2 - 4\alpha\beta$

  $\Rightarrow 1 = \left(-n\right)^2 - 4\left(\frac{n^2 - 31}{3}\right)\Rightarrow n = 11$.
  %350
\item Given that $\left(x^2 - 5x + 5\right)^{\left(x^2 + 4x - 60\right)} = 1$. This is possible when $x^2 +
  4x - 60 = 0$ and $x^2 - 5x + 5\neq 0$ or $x^2 - 5x + 5 = 1$ or $x^2 - 5x + 5 = -1$ and $x^2 + 4x - 60 =$
  an even integer.

  When $x^2 + 4x - 60 = 0\Rightarrow x = -10, 6$, note that for these two values $x^2 - 5x + 5 neq 0$

  When $x^2 - 5x + 5 = 1\Rightarrow x = 4, 1$

  When $x^2 - 5x + 5 = -1 \Rightarrow x = 3, 2$. When $x = 2, x^2 + 4x - 60 = -48$, which is an even
  integer. Hoever, for $x = 3, x^2 + 4x - 60 = -39$, which is not an even integer.

  Thus, sum of all possible values of $x = -10 + 6 + 4 + 1 = 2 = 3$.
  %351
\item $\alpha_1, \beta_1 = \sec\theta\pm\tan\theta\Rightarrow \alpha_1 = \sec\theta - \tan\theta, \beta_1
  = \sec\theta + \tan\theta[\because \alpha_1 > \beta_1]$ and $\theta\in$ fourth quadrant.

  $\alpha_2, \beta_2 = -\tan\theta + \sec\theta, -\tan\theta - \sec\theta[\because \alpha_2 > \beta_2]$

  $\therefore \alpha_1 + \beta_2 = -2\tan\theta$.
\item Since $\alpha, \beta$ are the roots of $x^2 - 6x - 2 = 0 \Rightarrow \alpha^2 -2 = 6\alpha$ and
  $\beta^2 -2 = 6\beta$.

  $a_{10} = \alpha^{10} - \beta^{10}, a_9 = \alpha^9 - \beta^9, a_8 = \alpha^8 - \beta^8$

  $\Rightarrow \frac{a_{10} - 2a_8}{2a_9} = \frac{\alpha^8\left(\alpha^2 - 2\right) - \beta^8\left(\beta^2
  - 2\right)}{2\left(\alpha^9 - \beta^9\right)}$

  $= \frac{\alpha^8.2\alpha - \beta^8.2\beta}{2\left(\alpha^9 - \beta^9\right)} = 3$.
  %353
\item Since $p(x) = 0$ has purely imaginary roots, therefore, it has the form of $ax^2 + b = 0$, where $a,
  b$ are real coefficients and $a, b$ are of same sign.

  $\therefore p[p(x)] = a\left(ax^2 + b\right)^2 + b = b\neq 0$. Thus, $p[p(x)]$ has neihter real nor
  imaginary roots.
  %354
\item From given conditions $\alpha + \beta = -\frac{q}{p}, \alpha\beta = \frac{r}{p}$ and $2q = p + r$

  $\frac{1}{\alpha} + \frac{1}{\beta} = 4 \Rightarrow \alpha + \beta = 4\alpha\beta\Rightarrow -\frac{q}{p}
  = \frac{4r}{p}$

  $\Rightarrow -q = 4r\Rightarrow 2(-4r) = p + r \Rightarrow p= 9r$

  $\Rightarrow \alpha + \beta = -\frac{4}{9}, \alpha\beta = -\frac{1}{9}$

  $\Rightarrow (\alpha - \beta)^2 = (\alpha + \beta)^2 - 4\alpha\beta = \frac{52}{81}$

  $\Rightarrow |\alpha - \beta| = \frac{2\sqrt{13}}{9}$.
  %355
\item Sum of roots $= \frac{\alpha^2 + \beta^2}{\alpha\beta}$ and product $= 1$

  Given $\alpha + \beta = -p$ and $\alpha^3 + \beta^3 = q\Rightarrow (\alpha + \beta)\left(\alpha^2
  - \alpha\beta + \beta^2\right) = q$

  $\Rightarrow \alpha^2 - \alpha\beta + \beta^2 = -\frac{q}{p}$ and $(\alpha + \beta)^2 = p^2$

  Thus, $\alpha^2 + \beta^2 = \frac{p^3 - 2q}{3p}$ and $\alpha\beta = \frac{p^3 + q}{3p}$

  Therefore, required equation is $x^2 - \frac{p^3 - 2q}{3p}x + (p^3 + q) = 0$.
  %356
\item Since roots of given equations are real $\therefore D\geq 0$

  $\Rightarrow 4(a + b + c)^2 - 12\lambda(ab + bc + ca)\geq 0\Rightarrow a^ + b^2 + c^2\geq (ab + bc +
  ca)(3\lambda - 2)$

  $\Rightarrow 3\lambda - 2\leq \frac{a^2 + b^2 + c^2}{ab + bc + ca}$

  Also given that $a, b, c$ are sides of a triangle, therefore $\cos A = \frac{b^2 + c^2 - a^2}{2bc} < 1$

  $\Rightarrow b^2 + c^2 - a^2 < 2bc$ and similarly $c^2 + a^2 - b^2 < 2ca$ and $a^2 + b^2 - c^2 < 2ab$

  $\Rightarrow \frac{a^2 + b^2 + c^2}{ab + bc + ca} < 2$

  Thus, $3\lambda - 2 < 2\Rightarrow \lambda < \frac{4}{3}$.
  %357
\item Given $c < 0 < b\Rightarrow \alpha\beta < 0$ so $\alpha, \beta$ have opposite signs.

  $d > 0 \Rightarrow -b < 0\Rightarrow \alpha + \beta < 0$

  So modulus of negative quantity is greater than the modulus of positive quantity. Thus, we have the
  desired relation.
  %358
\item According to question $\alpha + \beta = -p, \alpha\beta = q, \alpha^4 + \beta^4 = r$ and
  $\alpha^4\beta^4 = s$

  Let the roots of $x^2 - 4qx + 2q^2 - r = 0$ be $\alpha'$ and $\beta'$, then

  $\alpha'\beta' = (2q^2 - r) = 2(\alpha\beta)^2 - \left(\alpha^4 + \beta^4\right) = -\left(\alpha^2
  - \beta^2\right) < 0$

  Thus, roots are real and of opposite sign.
  %359
\item Given that $x - \frac{2}{x - 1} = 1 - \frac{2}{x - 1}\Rightarrow x = 1$

  However, the equation is not defined at $x = 1$. Therefore, the equation has no roots.
  %360
\item Given $|x - 2|^2 + |x - 2| - 2 = 0\Rightarrow \left(|x - 2| + 2\right)\left(|x - 2| - 1\right) = 0$

  $\Rightarrow |x - 2| = -2, 1\Rightarrow x = 3, 1$, we neglect $|x - 2| = -2$.

  Thus, sum of the roots is $4$.
  %361
\item Given that $2e^{2\log k} - 1 = 7 \Rightarrow e^{2\log k} = 4 \Rightarrow k^2 = 4 \Rightarrow k = 2$

  The other value is $k = -2$ for which $\log$ is not defined, hence, neglected.
  %362
\item Coefficient of $x^{99} = -(1 + 2 + 3 + \cdots + 100) = -5050$.
  %363
\item $a + b = 10c$ and $c + d = 10a, \Rightarrow a - c + b - d = 10(c - a)\Rightarrow b - d = 11(c - a)$

  Since $c$ is a root of $x^2 - 10ax - 11b = 0\Rightarrow c^2 - 10ac - 11b = 0$ and since $a$ is a root
  $x^2 - 10cx - 11d = 0 \Rightarrow a^2 - 11ac - 11d = 0$

  $\Rightarrow c^2 - a^2 = 11(b - d)\Rightarrow c + a(c - a) = 121(c - a)\Rightarrow c + a = 121$

  $\therefore a + b + c + d = 10c + 10a = 10(c + a) = 1210$.
  %364
\item We have to prove that $\frac{b^2 - 4ac}{a^2} = \frac{B^2 -
  4AC}{A^2} \Rightarrow \left(-\frac{b}{a}\right)^2 - \frac{4c}{a} = \left(-\frac{B}{A}\right)^2
  - \frac{4C}{A}$

  $\Rightarrow (\alpha + \beta)^2 - 4\alpha\beta = (\alpha + \delta + \beta + \delta)^2 - 4(\alpha
  + \delta)(\beta + \delta)$

  $\Rightarrow (\alpha - \beta)^2 = (\alpha + \delta - \beta - \delta)^2 = (\alpha - \beta)^2$.

  Hence proved.
  %365
\item Suppose that $f(x) = Ax^2 + Bx + C$ is an integer whenever $x$ is an integer. Therefore, $f(0), f(1),
  f(-1)$ are integers.

  $\Rightarrow C, A + B + C, A - B + C$ are integers. $\Rightarrow C, A + B, A - B$ are
  integers. $\Rightarrow C, A + B, A+ B + (A - B) = 2A$ are integers.

  Cconversely suppose that $2A, A + B, C$ are integers. Let $n$ be any integer. Then

  $f(n) = An^2 + Bn + C = 2A\frac{n(n - 1)}{2} + (A + B)n + C$, which is an integer for all $n$.
  %366
\item {\bf Case I:} When $y\in(-\infty, 0]$

  $\therefore 2^{-y} + 2^{y - 1} - 1 = 2^{y - 1} + 1\Rightarrow 2^{-y} = 2 \Rightarrow y = -1$

  {\bf Case II:} When $y\in(0, 1]$

  $\therefore 2^y + 2^{y - 1} - 1 = 2^{y - 1} + 1\Rightarrow 2^y = 2 \Rightarrow y = 1$

  {\bf Case III:} When $y\in(1, \infty)$

  $\therefore 2^y - 2^{y - 1} + 1 = 2^{y - 1} + 1\Rightarrow 0 = 0$, which is true for all $y > 1$
  %367
\item We have $\frac{2x}{2x^2 + 5x + 2} > \frac{1}{x + 1} \Rightarrow \frac{2x}{2x^2 + 5x + 2} - \frac{1}{x
  + 1} > 0$

  $\Rightarrow -\frac{(3x + 2)}{(2x + 1)(x + 1)(x + 2)} > 0$

  Thus, the solution set is $(-2, -1)\cup\left(-\frac{2}{3}, -\frac{1}{2}\right)$.
  %368
\item The diagram is given below:

  \startplacefigure[location=foce]
    \startMPcode
      numeric u; u := 0.3cm;
      numeric tmin, tmax, dt;
      tmin := -3.2; tmax := 3.2; dt := 0.1;

      vardef X(expr t) = t*t enddef;       % x(t) = t^2
      vardef Y(expr t) = 2*t enddef;       % y(t) = 2t

      % Build the parabola path by sampling t
      path P;
      P := (u*X(tmin), u*Y(tmin))
      for t=tmin+dt step dt until tmax:
      .. (u*X(t), u*Y(t))
      endfor;

      % Axes (adjust view window as desired)
      pair Xmin, Xmax, Ymin, Ymax;
      Xmin := (-u, 0);  Xmax := (10u, 0);
      Ymin := (0, -8u); Ymax := (0, 8u);
      drawarrow Xmin -- Xmax;
      drawarrow Ymin -- Ymax;
      label.rt(btex $x$ etex, Xmax);
      label.top(btex $y$ etex, Ymax);

      % Draw the parabola
      draw P;

      % Mark the points (4,-4) and (9,6)
      pair A, B, X;
      A := (4u, -4u);
      B := (9u, 6u);
      X := (4u, 4u);

      drawdot A withpen pencircle scaled 2bp;
      drawdot B withpen pencircle scaled 2bp;
      drawdot X withpen pencircle scaled 2bp;

      label.llft("$Q(4,-4)$", A);
      label.ulft("$P(9, 6)$", B);
      label.llft("$O$", origin);
      label.ulft("$X(t^2, 2t)$", X);

      draw A -- B -- X -- cycle dashed evenly;

      % Optional: label the curve
      label.rt("$y^2=4x$", (10u, 7u));
    \stopMPcode
  \stopplacefigure

  $\Delta PXQ = \frac{1}{2}\left|\startdeterminant\NC 4\NC -4\NC 1\NR\NC t^2\NC 2t\NC 1\NR\NC 9\NC 6\NC
  1\NR\stopdeterminant\right|$

  $= \left|5t^2 - 5t - 30\right|$

  Now, as $X$ is any point on the arc $POQ$ of the parabola, therefore ordinate of the point $X, 2t\in(-4,
  6)\Rightarrow t\in(-2, 3)$

  $\Delta PXQ = -5t^2 + 5t + 30[\because |x - a| = -(x - a)$ if $x < a]$

  Thus, maximum area is $\frac{125}{4}$. [Maximum area of expression $ax^2 + bx + c$ is $-D/4a$]
  %369
\item The diagrams are given below:

  \startplacefigure[location=force]
    \startMPcode
      numeric u; u := 1.5cm;     % scale (points per unit)
      numeric xmin, xmax, dx;
      xmin := 0.5; xmax := 3.5; dx := 0.1;

      vardef Y(expr x) = (x**2 - 4*x + 15/4) enddef;
      vardef Z(expr x) = (-(x**2) + 4*x - 15/4) enddef;

      % Build the parabola path by sampling x
      path P;
      P := (u*xmin, u*Y(xmin))
      for x=xmin+dx step dx until xmax:
      .. (u*x, u*Y(x))
      endfor;

      % Axes
      drawarrow (-u,0)--(4u,0);
      drawarrow (0,-u)--(0,2u);
      label.rt("$x$", (4u,0));
      label.top("$y$", (0,2u));
      draw(2u, -0.1*u) -- (2u, 0.1*u);
      draw(3u, -0.1*u) -- (3u, 0.1*u);
      label.bot("$2$", (2u, -0.1*u));
      label.bot("$3$", (3u, -0.1*u));

      % Draw the parabola
      draw P;

      % Optional: vertex point (2, -1/4)
      pair V;
      V := (2u, (2**2 - 4*2 + 15/4)*u);
      drawdot V withpen pencircle scaled 2pt;

      path Q;
      Q := (u*xmin, u*Z(xmin))
      for x=xmin+dx step dx until xmax:
      .. (u*x, u*Z(x))
      endfor;

      draw Q;
    \stopMPcode
  \stopplacefigure

  According to the problem, the graph of $f(x)$ will be either of two ways as shown above. In both cases
  $f(0).f(2) < 0$ and $f(2)f(3) < 0$.

  First we consider $f(0)f(2) < 0$

  $\Rightarrow (c - 4)[4(c - 5) - 4c + c - 4] < 0 \Rightarrow (c - 4)(c - 24) < 0\Rightarrow c\in(4, 24)$

  Now we consider $f(2)f(3) < 0$ which leads to $c\in\left(\frac{49}{4}, 24\right)$

  Combining both the solutions we have integral values of $c$ from $13$ through $24$ i.e. $11$ possible
  values.
  %370
\item The diagram is given below:

  The following condition must satisfy.

  {\bf Cond. I:} $D > 0 \Rightarrow b^2 - 4ac > 0 \Rightarrow m^2 - 16 > 0\Rightarrow m\in(-\infty,
  -4)\cup(4, \infty)$

  {\bf Cond. II:} The vertex of the parabola must lie between $1$ and $5$

  $\therefore -\frac{b}{2a}\in(1, 5)\Rightarrow 1 < \frac{m}{2} < 5\Rightarrow m\in(2, 10)$

  {\bf Cond. III:} $f(1) > 0\Rightarrow 1 - m + 4 > 0 \Rightarrow m < 5 \Rightarrow m\in(-\infty, 5)$

  {\bf Cond. IV:} $f(5) > 0 \Rightarrow 25 - 5m + 4 > 0 \Rightarrow m\in\left(-\infty, \frac{29}{5}\right)$

  Combining all the conditions we have $m\in(4, 5)$.

  \startplacefigure[location=force]
    \startMPcode
      numeric u; u := 1cm;     % scale (points per unit)
      numeric xmin, xmax, dx;
      xmin := 1; xmax := 5; dx := 0.1;

      vardef Y(expr x) = (x**2 - 6*x + 8) enddef;

      % Build the parabola path by sampling x
      path P;
      P := (u*xmin, u*Y(xmin))
      for x=xmin+dx step dx until xmax:
      .. (u*x, u*Y(x))
      endfor;
      % Draw the parabola
      draw P;

      % Axes
      drawarrow (-u,0)--(6u,0);
      drawarrow (0,-u)--(0,4u);
      label.rt("$x$", (6u,0));
      label.top("$y$", (0,4u));
      draw(u, -0.1*u) -- (u, 0.1*u);
      draw(5u, -0.1*u) -- (5u, 0.1*u);
      label.bot("$1$", (u, -0.1*u));
      label.bot("$5$", (5u, -0.1*u));
    \stopMPcode
  \stopplacefigure
  %371
\item Putting $t = x - [x] = \left\{X\right\}$, which is a fractional part function and lie between
  $0\leq \left\{X\right\} < 1$.

  $\Rightarrow -3t^2 + 2t + a = 0 \Rightarrow t = \frac{1\pm\sqrt{1 + 3a^1}}{3}$

  $\Rightarrow 0\leq \frac{1\pm\sqrt{1 + 3a^2}}{3} < 1$

  We take positive sign as negative sign will be less than zero.

  $\Rightarrow \sqrt{1 + 3a^2} < 2\Rightarrow 1 + 3a^2 < 4\Rightarrow a^2 - 1 < 0 \Rightarrow a\in(-1,
  0)\cup (0, 1)$.
  %372
\item We know that $ax^2 + bx + c > 0\forall\; x\in\mathbb{R}$ if $a > 0$ and $D < 0$

  $\Rightarrow 4x^2 - 4(10 - 3a) < 0 \Rightarrow a^2 + 3a - 10 < 0$

  $(a + 5)(a - 2) < 0 \Rightarrow \a\in(-5, 2)$, however, $a > 0\Rightarrow a\in(0, 2)$.
  %373
\item The diagram is given below.

  From the diagram it is clear that one of the roots of $(x - a)(x - b) - 1 = 0$ lies in the range $(\infty,
  a)$ and the other root lies in the range $(b, \infty)$.

  \startplacefigure[location=force]
    \startMPcode
      numeric u; u := 1cm;     % scale (points per unit)
      numeric xmin, xmax, dx;
      xmin := 0.5; xmax := 2.5; dx := 0.05;
      numeric xmins, xmaxs;
      xmins := 0.5; xmaxs :=2.5;

      vardef Y(expr x) = (x**2 - 3*x + 2) enddef;
      vardef Z(expr x) = (x**2 - 3*x + 1) enddef;

      % Build the parabola path by sampling x
      path P;
      P := (u*xmin, u*Y(xmin))
      for x=xmin+dx step dx until xmax:
      .. (u*x, u*Y(x))
      endfor;
      % Draw the parabola
      draw P;
      path Q;
      Q := (u*xmins, u*Z(xmins))
      for x=xmins+dx step dx until xmaxs:
      .. (u*x, u*Z(x))
      endfor;
      % Draw the parabola
      draw Q;

      % Axes
      drawarrow (-u,0)--(3u,0);
      drawarrow (0,-u)--(0,2u);
      label.rt("$x$", (3u,0));
      label.top("$y$", (0,2u));
      draw(u, -0.1*u) -- (u, 0.1*u);
      draw(2u, -0.1*u) -- (2u, 0.1*u);
      label.bot("$a$", (u, -0.1*u));
      label.bot("$b$", (2u, -0.1*u));
      label.rt("$(x - a)(x - b) = 0$", (2.5u, 0.75u));
      label.rt("$(x - a)(x - b) - 1 = 0$", (2.5u, -0.25u));
    \stopMPcode
  \stopplacefigure
  %374
\item Following conditions must be satisfied.

  {\bf Cond. I:} Since both the roots are less than $3\Rightarrow \alpha < 3, \beta < 3$

  $\Rightarrow \alpha + \beta < 6 \Rightarrow \frac{2a}{2} < 3\Rightarrow a < 3$

  {\bf Cond. II:} Product $\alpha\beta < 9 \Rightarrow a^2 + a - 3 < 9 \Rightarrow a^2 + a - 12 <0$

  $\Rightarrow -4 < a < 3$

  {\bf Cond. III:} $D \geq 0 \Rightarrow 4a^2 - 4(a^2 + a - 3)\geq 0\Rightarrow a\leq 3$

  {\bf Cond. IV:} $f(3) > 0 \Rightarrow a^2 - 5a + 6 > 0 \Rightarrow a\in(-\infty, 2)\cup (3, \infty)$

  Combining all four conditions we have $a\in(-4, 2)$.
  %375
\item Let $f(x) = ax^2 + bx + c$. It is given that $f(x) > 0\;\forall\;x\in\mathbb{R}\Rightarrow a > 0$ and
  $b^2 - 4ac < 0$

  Given that $g(x) = f(x) + f'(x) + f''(x) = ax^2 + bx + c + 2ax + b + 2a$

  Its discriminant $D' = (b + 2a)^2 - 4a(c + b + 4a) = (b^2 - 4ac) - 4a^2 < 0$

  Thus, $g(x) > 0\;\forall\;x\in\mathbb{R}$.
  %376
\item $x^2 + (a - b)x + (1 - a - b) = 0$ has real and unequal roots.

  $\Rightarrow D > 0 \Rightarrow (a - b)^2 - 4(1 - a - b) > 0 \Rightarrow a^2 + b^2 - 2ab - 4 + 4a + 4b > 0$

  We have to find all values of $a$ for which the given equation has real and unequal roots for all real
  values of $b$ i.e. above equation is true for all real values of $b$'

  $\Rightarrow b^2 + b(4 - 2a) + (a^2 + 4a - 4) > 0$ for all real $b$

  $\Rightarrow (4 - 2a)^2 - 4(a^2 + 4a - 4) < 0 \Rightarrow a > 1$.
  %377
\item Refer to the diagram given in the chapter of theory of equations where $a > 0$ and $a < 0$. It is
  clear that if $a > 0$ then $f(-1) < 0$ and $f(1) < 0$, and if $a < 0, f(-1) > 0$ and $f(1) > 0$. In both
  the cases

  $af(-1) < 0$ and $af(1) < 0\Rightarrow a(a - b + c) < 0$ and $a(a + b + c) < 0$

  Dividing by $a^2$ we have $1 - \frac{b}{a} + \frac{c}{a} < 0$ and $1 + \frac{b}{a} + \frac{c}{a} < 0$

  $\Rightarrow 1 + \frac{c}{a} + \left|\frac{b}{a}\right| < 0$.
  %378
\item $D > 0 \Rightarrow 64\left\{k^2 - (k^2 - k + 1)\right\}\Rightarrow k > 1$

  $-\frac{b}{2a} > 4\Rightarrow \frac{8k}{2} > 4\Rightarrow k > 1$

  $f(4)\geq 0 \Rightarrow k^2 -3k + 2 \geq 0 \Rightarrow k\leq 1$ or $k\geq 2$

  From these three conditions we have $k = 2$.
  %379
\item Given equation is $5 + \left|2^x - 1\right| = 2^x(2^x - 2)$

  {\bf Case I:} If $2^x - 1\geq 0\Rightarrow x\geq 0$ then $5 + 2^x - 1 = 2^x(2^x - 2)$

  Let $t = 2^x$ then the equation becomes $t + 4 = t^2 - 2t\Rightarrow t^2 - 3t - 4 = 0$

  $\Rightarrow t = 4 \Rightarrow x = 2 > 0\Rightarrow x = 2$ is a solution.

  {\bf Case II:} If $2^x - 1 < 0 \Rightarrow x < 0$ so the given equation becomes

  $5 + 1 - 2^x = 2^x\left(2^x - 2\right)$. We put $2^x = y$ and rewrite to get $y^2 - y - 6 = 0$

  $\Rightarrow y = 3, -2$. However, $y\neq 2$ as $2^x < 1\Rightarrow 2^x = -2 \Rightarrow
  x\not\in\mathbb{R}$

  Thus, we have one real solution.
  %380
\item Given inequality is $2^{\sqrt{\sin^2x - 2\sin x + 5}}.\frac{1}{4\sin^2y}\leq 1$

  $\Rightarrow 2^{\sqrt{\sin^2x - 2\sin x + 5}}\leq 2^2\sin^2y$

  $\Rightarrow \sqrt{(\sin x - 1)^2 + 4}\leq 2\sin^2y$

  Rnage of $\sqrt{(\sin x - 1)^2 + 4}$ is $\left[2, 2\sqrt{2}\right]$ and range of $2\sin^2y$ is $\left[0,
    2\right]$.

  Thus, we see that inequality is achieved at $2$.

  $\Rightarrow \sqrt{(\sin x - 1)^2 + 4} = 2 = 2\sin^2y \Rightarrow \sin x = 1 = |\sin y|$

  Thus, $x = \frac{\pi}{2}, y = 0, \pi$.
  %381
\item Given equation is $\left|\sqrt{x} - 2\right| + \sqrt{x}\left(\sqrt{x} - 4\right) + 2 = 0$

  $\Rightarrow \left|\sqrt{x} - 2\right| + \left(\sqrt{x} - 2\right)^2 = 2$

  Let $\sqrt{x} - 2 = y$ then the equation becomes $y^2 + y - 2 = 0\Rightarrow y = 1, -2$(reject $-2$ as
  $y \geq 0$)

  $\Rightarrow x = 9, 1$.
  %382
\item Differentiating given equation w.r.t.\ $x$ we obtain

  $f'(x) = 6x^2 + 3$, which is a strictly increasing function. $\Rightarrow f(x) = 0$ has only one root.
  %383
\item From given conditions $a^2\alpha^2 + b\alpha + c = 0$ and $a^2\beta^2 - b\beta - c = 0$

  Let $f(x) = a^2x^2 + 2bx + 2c$, then $f(\alpha) = a^2\alpha^2 + 2b\alpha + 2c = a^2\alpha^2 - 2a^2\alpha^2
  = -a^2\alpha^2$

  and $f(\beta) = a^2\beta^2 + 2\beta + 2c = a^2\alpha^2 + 2a^2\beta^2 = 3a^2\beta^2$

  $\Rightarrow f(\alpha)f(\beta) < 0$, which implies that $f(x)$ must have a root lying between $\alpha$ and
  $\beta$.

  $\Rightarrow \alpha < \gamma <\beta$.
  %384
\item Let $f(x) = ax^3 + bx^2 + cx + d = 0\Rightarrow f(0) = d$ and $f(1) = a + b + c + d = d[\because a + b
  + c = 0]$

  $\therefore f(0) = f(1)$. $f$ is continuous in the interval $[0, 1]$ and $f$ is derivable in the open
  interval $(0, 1)$.

  By Rolle's theorem $f'(\alpha) = 0$ for $0 < \alpha < 1$

  $\Rightarrow f'(x) = 3ax^2 + 2bx + c\Rightarrow f'(\alpha) = 3a\alpha^2 + 2b\alpha + c = 0$

  Thus, given equation has at least one root in $(0, 1)$.
  %385
\item There are three cases:

  {\bf Case I:} When $x\leq 0\Rightarrow x^{12}\geq 0, -x^{9} > 0, x^4 > 0, -x > 0$

  $\therefore x^{12} - x^{9} + x^4 - x + 1 > 0\;\forall\;x\leq 0$.

  {\bf Case II:} When $0 <x \leq 1\Rightarrow x^9 < x^4$ and $x < 1\Rightarrow -x^9 + x^4 > 0$ and $1 - x >
  0$

  $\therefore x^{12} - x^{9} + x^4 - x + 1 > 0\;\forall\;0< x\leq 1$

  {\bf Case III:} When $x > 1\Rightarrow x^{12} - x^9 > 0$ and $x^4 - x > 0$

  $\therefore x^{12} - x^{9} + x^4 - x + 1 > 0\;\forall\;x > 1$.

  Thus, the given equation holds for all real values of $x$.
  %386
\item Given that $x_1$ and $x_2$ are roots of $\alpha x^2 - x + \alpha = 0$

  $x_1 + x_2 = \frac{1}{\alpha}$ and $x_1x_2 = 1$

  Also given that $\left|x_1 - x_2\right| < 1\Rightarrow \left(x_1 - x_2\right)^2 < 1$

  $\Rightarrow \left(x_1 + x_2\right)^2 - 4x_1x_2 < 1\Rightarrow \frac{1}{\alpha^2} < 5$

  $\Rightarrow 5\alpha^2 - 1 > 0\Rightarrow \alpha\in\left(-\infty,
  -\frac{1}{\sqrt{5}}\right)\cup \left(\frac{1}{\sqrt{5}}, \infty\right)$

  Also, $D > 0\Rightarrow 1 - 4\alpha^2 > 0 \Rightarrow \alpha\in\left(-\frac{1}{2}, \frac{1}{2}\right)$

  Therefore, combining both the set obtained gives us

  $S = \left(-\frac{1}{\sqrt{5}}, -\frac{1}{2}\right)\cup\left(\frac{1}{2}, \frac{1}{\sqrt{5}}\right)$.
  %387
\item Since $\alpha$ is a repeated root of $ax^2 - 2bx + 15 = 0$, therefore, we have

  $2\alpha = \frac{2b}{a}\Rightarrow \alpha = \frac{b}{a}$ and $\alpha^2 = \frac{15}{a}$

  $\Rightarrow \alpha = \frac{15}{b}$

  Now since $\alpha$ is a root of $x^2 - 2bx + 21 = 0$

  $\Rightarrow \left(\frac{15}{b}\right)^2 - 2b.\frac{15}{b} + 21 = 0 \Rightarrow b^2 = 25$

  $\Rightarrow \alpha + \beta = 2b, \alpha\beta = 21$

  $\alpha^2 + \beta^2 = (\alpha + \beta)^2 - 2\alpha\beta = 58$.
  %388
\item Given that $x^5\left(x^3 - x^2 - x + 1\right) + x\left(3x^3 - 4x^2 - 2x + 4\right) - 1 = 0$

  $\Rightarrow (x - 1)^2(x + 1)\left(x^5 + 3x - 1\right) = 0$

  Let $f(x) = x^5 + 3x - 1 \Rightarrow f'(x) = 5x^4 + 3 > 0\;\forall\;x\in\mathbb{R}$, which is strictly
  increasing. Therefore, $f(x)$ can have only one root.

  Thus, given equation has $3$ distinct, real roots.
  %389
\item Let $\alpha, beta$ be the roots of the given equation.

  $\alpha^2 + \beta^2 = (\alpha + \beta)^2 - 2\alpha\beta = (3 - a)^2 - 2(1 - 2a)$

  $= (a - 1)^2 + 6\Rightarrow \left(\alpha + \beta\right)_{\min}^2 = 6$.
  %390
\item Clearly $x = 0$ is not a solution so we can divide the given equation by $x^2$

  $\Rightarrow x^2 - 3x - 2+ \frac{3}{x} + \frac{1}{x^2} = 0 \Rightarrow x^2 + \frac{1}{x^2} - 2 + 2 -
  3\left(x - \frac{1}{x}\right) - 2 = 0$

  $\Rightarrow \left(x - \frac{1}{x}\right)^2 - 3\left(x - \frac{1}{x}\right) = 0$

  $\Rightarrow x - \frac{1}{x} = 0, 3 \Rightarrow \alpha, \beta = \pm 1$ and $x^2 - 3x + 1 = 0$

  $\Rightarrow \gamma + \delta = 3;,\gamma\delta = 1$, where $\gamma, \delta$ are roots of $x^2 - 3x + 1 = 0$

  $\alpha^3 + \beta^3 + \gamma^3 + \delta^3 = 1 - 1 + (\gamma + \delta)[(\gamma + \delta)^2 -
    3\gamma\delta]$

  $= 36$.
  %391
\item Given that $p + q = 3$ and $p^4 + q^4 = 369\Rightarrow (p + q)^2 = 9 \Rightarrow p^2 + q^2 = 9 - 2pq$

  $\frac{1}{\left(\frac{1}{p} + \frac{1}{q}\right)^2} = \frac{(qp)^2}{(q + p)^2} = \frac{(qp)^2}{9}$

  Now $p^4 + q^4 = \left(p^2 + q^2\right)^2 - 2p^2q^2 \Rightarrow 360 = \left(9 - 2pq\right)^2 - 2(pq)^2$

  $\Rightarrow 144 = p^2q^2 - 18pq \Rightarrow pq = 24, -6$(we reject $pq = 24$ because it makes $p^2 + q^2
  < 0$)

  $\Rightarrow \frac{(pq)^2}{9} = 4$.
  %392
\item Given $f(x) = 2x^2 - x - 1$ and $|f(x)|\leq 800 \Rightarrow n^2 -\frac{n}{2} - 801\leq 0$

  $\Rightarrow \left(n - \frac{1}{4} - \frac{\sqrt{6409}}{4}\right)\left(n - \frac{1}{4} +
  \frac{\sqrt{6409}}{16}\right) \leq 0$

  $\Rightarrow n = \left\{-19, -18, -17, \ldots, 20\right\}$

  $\Rightarrow \displaystyle\sum_{\mstack n\in S} = 2\left(19^2 + 18^2 + \cdots + 1^2 + 1^2 + \cdots +
  20^2\right) - (-19 - 18 -\cdots + 19 + 20) - 40$

  $= 10620$.
  %393
\item $3^{\sqrt{\log_35}} = 3^{\sqrt{\log_35}.\sqrt{\log_35}\sqrt{\log_53}} = 3^{\log_35\sqrt{\log_53}} =
  5^{\sqrt{\log_53}}$

  Similarly $3^{\sqrt[3]{\log_35}} = 3^{\log_35\sqrt[3]{\left(\log_53\right)^2}} =
  5^{\sqrt[3]{\left(\log_53\right)^2}}$

  Thus, the equation is $x^2 - 5x - 3 =0\Rightarrow \alpha + \beta = 5, \alpha\beta = -3$

  Now $\alpha + \frac{1}{\beta} = \frac{\alpha\beta + 1}{\beta} = -\frac{2}{\beta}$ and $\beta +
  \frac{1}{\alpha} = \frac{\alpha\beta + 1}{\alpha} = -\frac{2}{\alpha}$

  Let $-\frac{2}{\alpha} = t\Rightarrow \alpha = -\frac{2}{t}$

  Given $\alpha^2 - 5\alpha - 3 = 0\Rightarrow \frac{4}{t^2}+ \frac{10}{t} - 3 = 0$

  $\Rightarrow 3t^2 - 10t - 4 = 0\Rightarrow 3x^2 - 10x -4 = 0$
  %394
\item Let $f(x) = x^4 - 4x + 1\Rightarrow f'(x) = 4x^3 - 4\Rightarrow f'(x) = 0 \Rightarrow x = 1$

  $x = 1$ si the point of minima. $f(1) = -2, f(0) = 1$

  Hence, it has two distinct real roots.

  The plot is given below:

  \startplacefigure[location=force]
    \startMPcode
      numeric u; u := 1cm;     % scale (points per unit)
      numeric xmin, xmax, dx;
      xmin := -0.2; xmax := 1.6; dx := 0.05;
      vardef Y(expr x) = (x**4 - 4*x + 1) enddef;

      % Build the parabola path by sampling x
      path P;
      P := (u*xmin, u*Y(xmin))
      for x=xmin+dx step dx until xmax:
      .. (u*x, u*Y(x))
      endfor;
      % Draw the curve
      draw P;

      % Axes
      drawarrow (-u,0)--(3u,0);
      drawarrow (0,-u)--(0,2u);
      label.rt("$x$", (3u,0));
      label.top("$y$", (0,2u));
      draw(u, -0.1*u) -- (u, 0.1*u);
      draw(2u, -0.1*u) -- (2u, 0.1*u);
      label.bot("$1$", (u, -0.1u));
      label.bot("$2$", (2u, -0.1u));
    \stopMPcode
  \stopplacefigure
  %395
\item Given equation is $e^{2x} - 11e^2 - 45e^{-x} + \frac{81}{2} = 0$

  $\Rightarrow e^{3x} - 11e^{2x} - 45 + \frac{18e^x}{x} = 0$. Putting $e^x = t$ gives us

  $2t^3 - 22t^2 + 81t - 90 = 0\Rightarrow t_1t_2t_3 = 45\Rightarrow e^{x_1 + x_2 + x_3} = 45$

  $\Rightarrow x_1 + x_2 + x_3 = \log_e45 = \log_eP \Rightarrow P = 45$.
  %396
\item $\frac{3x^2 - 9x + 17}{x^2 + 3x + 10} = \frac{5x^2 - 7x + 19}{3x^2 + 5x + 12}$

  $\Rightarrow 1 + \frac{2x^2 - 12x + 7}{x^2 + 3x + 10} = 1 + \frac{2x^2 - 12x + 7}{3x^2 + 5x + 12}$

  $(2x^2 - 12x + 7)\left(\frac{1}{x^2 + 3x + 10} - \frac{1}{3x^2 + 5x + 12}\right) = 0$

  $\Rightarrow 2x^2 - 12x + 7 = 0$ or $2x^2 + 2x + 2 = 0$

  $x = \frac{12\pm\sqrt{D}}{4}$ or $D < 0$ i.e. non-real roots.

  So sum of roots is $6$.
  %397
\item When $x \leq -3$. $\frac{|x + 3| - 1}{|x| - 2} = \frac{-x - 4}{-x - 2} = \frac{x + 4}{x + 2}\geq 0$

  When $-3 < x < -2$. $\frac{|x + 3| - 1}{|x| - 2} = \frac{x + 2}{-x -2} \leq 0$

  When $-2 < x < 0$. $\frac{|x + 3| - 1}{|x| - 2} = \frac{x + 2}{-x - 2}\leq 0$

  When $0 < x < 2$. $\frac{|x + 3| - 1}{|x| - 2} = \frac{x + 2}{x - 2} \leq 0$

  When $2 < x \leq 3$. $\frac{|x + 3| - 1}{|x| - 2} = \frac{x + 2}{x - 2} \geq 0$

  Thus, $S = \left\{-5, -4, 3\right\}$

  When $x < 0$. $x^2 - 7x + 9 \leq 0\Rightarrow D > 0$, thus it is true for all $x < 0$.

  When $x = 3$. $\Rightarrow 9 - 21 + 9 < 0$.

  Thus, $S\cap T = \left\{-5, -4, 3\right\}$.
  %398
\item $a = -\frac{1}{\alpha^2} - \frac{1}{\beta^2} - 2, b = \frac{1}{\alpha^2} + \frac{1}{\beta^2} + 1 +
  \frac{1}{\alpha^2\beta^2}$

  $a + b = \frac{1}{\alpha^2\beta^2} - 1 = \frac{1}{6} - 1 = -\frac{5}{6}$

  $\Rightarrow x^2 \left(-\frac{5}{6} - 2\right)x + \left(2 - \frac{5}{6}\right) = 0$

  $\Rightarrow 6x^2 + 17x + 7 = 0\Rightarrow x = -\frac{7}{3}. -\frac{1}{2}$.
  %399
\item Let $f(x) = x^7 + 5x^3 + 3x + 1 = 0\Rightarrow f'(x) = 7x^6 + 15x^2 + 3 > 0$

  $\therefore f(x)$ is a strictly increasing function.

  \startplacefigure[location=force]
    \startMPcode
      numeric u; u := 0.5cm;     % scale (points per unit)
      numeric xmin, xmax, dx;
      xmin := -0.5; xmax := 0.5; dx := 0.05;
      vardef Y(expr x) = (x**7 + 5*(x**3) + 3*x + 1) enddef;

      path P;
      P := (u*xmin, u*Y(xmin))
      for x=xmin+dx step dx until xmax:
      .. (u*x, u*Y(x))
      endfor;
      % Draw the curve
      draw P;

      % Axes
      drawarrow (-u,0)--(3u,0);
      drawarrow (0,-u)--(0,2u);
      label.rt("$x$", (3u,0));
      label.top("$y$", (0,2u));
      draw(u, -0.1*u) -- (u, 0.1*u);
      draw(2u, -0.1*u) -- (2u, 0.1*u);
      label.bot("$1$", (u, -0.1u));
      label.bot("$2$", (2u, -0.1u));
    \stopMPcode
  \stopplacefigure
  %400
\item Given that $e^{4x} + 4e^{3x} - 58e^{2x} + 4e^x + 1 = 0$

  $\Rightarrow \left(e^{2x} + \frac{1}{e^{2x}}\right) + 4\left(e^x + \frac{1}{e^x}\right) - 58 = 0$

  Let $t = e^x + \frac{1}{e^x}$, then the above equation becomes $t^2 + 4t - 58 = 0$

  $\Rightarrow t = -2 + 2\sqrt{62}$ because $t = -2 - 2\sqrt{62}$ is not possible.

  $\Rightarrow t \geq 0 \Rightarrow e^x + \frac{1}{e^x} = -2 + 2\sqrt{62}$

  for which both the roots are positive. Hence, we have two real roots.
  %401
\item Since $x = -1$ is a root of $f(x)$, we can write $f(x) = (x + 1)(ax + b)$

  Given that $f(-2) + f(3) = 14a + 3b = 0 \Rightarrow -\frac{b}{a} = \frac{14}{3}$

  $\therefore$ Sum of roots is $-1 - \frac{b}{a} = \frac{11}{3}$.
  %402
\item  L.H.S.\ $\leq 2$ and R.H.S.\ $\geq 2\Rightarrow 2\cos\left(\frac{x^2 + x}{6}\right) = 4^x + 4^{-x} =
  2$

  For this $x = 0$ is the only solution.
  %403
\item $S: \frac{(x - 3)^2}{16} + \frac{(y - 4)^2}{9}\leq 1$ and $T: (x - 7)^2 + (y - 4)^2\leq 36$

  Let $x - 3 = x$ and $y - 4 = y$(effectively we are shifting origin), then the equations become

  $\frac{x^2}{16} + \frac{y^2}{9}\leq 1$ and $(x - 4)^2 + y^2\leq 36$

  The diagram is given below:

  \startplacefigure[location=force]
    \startMPcode
      numeric u, a, b;
      u := 0.3cm;        % unit length for convenient sizing
      a := 8u;         % semi-major axis (x-direction)
      b := 6u;         % semi-minor axis (y-direction)

      path E;
      E := fullcircle xscaled a yscaled b;
      draw fullcircle scaled(12u) shifted(4u, 0);

      draw E;

      % axes
      drawarrow (-5u,0) -- (11u,0);
      drawarrow (0,-5u) -- (0,5u);

      label.rt("$x$", (11u,0));
      label.top("$y$", (0,5u));
      label.bot("$(-2, 0)$", (-2u, 0));
      label.ulft("$(0, 2\sqrt(5))$", (0, 2*sqrt(5)*u));
      label.llft("$(0, -2\sqrt(5))$", (0, -2*sqrt(5)*u));
      label.urt("$(0, 3)$", (0, 3u));
      label.lrt("$(0, -3)$", (0, -3u));
      label.bot("$(4, 0)$", (4u, 0));
      label.bot("$(10, 0)$", (10u, 0));
    \stopMPcode
  \stopplacefigure

  $S\cap T = (-2, 0), (-1, 0), \ldots, (4, 0)\rightarrow (7)$

  $(-1, 1), (0, 1), \ldots, (3, 1) \rightarrow (5)$

  $(-1, -1), (0, -1), \ldots, (3, -1) \rightarrow (5)$

  $(-1, 2), (0, 2), (1, 2), (2, 2)\rightarrow (4)$

  $(-1, -2), (0, -2), (1, -2), (2, -2)\rightarrow (4)$

  $(0, 3), (0, -3) \rightarrow (2)$

  Thus, a total of $27$ points.
  %404
\item Taking $\log$ of both sides with base $5$ and putting $\log_5x = t$ gives us

  $16t^4 - 68t^2 + 16 = 0\Rightarrow 4t^4 - 17t^2 + 4 = 0$

  $\Rightarrow t_1t_2t_3t_4 = 0$, where $t_1, t_2, t_3, t_4$ are the roots of the equation above.

  $\Rightarrow \log_5x_1 + \log_5x_2 + \log_5x_3 + \log_5x_4 = 0 \Rightarrow x_1x_2x_3x_4 = 1$, where $x_1,
  x_2, x_3, x_4$ are the roots of the given equation.
  %405
\item Let $\left(\sqrt{3} + \sqrt{2}\right)^{x^2 - 4} = t \Rightarrow t + \frac{1}{t} = 10\Rightarrow t =
  5 \pm2\sqrt{6}$

  $\Rightarrow x^2 - 4 = \pm2 \Rightarrow x^2 = 6, 2\Rightarrow x = \pm\sqrt{2}, \pm\sqrt{6}$

  $\Rightarrow n(S) = 4$.
  %406
\item Clearly, $f(1)f(2) < 0$ and $f(2)f(3) < 0$

  $\Rightarrow (k - 6)(k - 8) < 0$ and $(k - 8)(k - 6) < 0$

  $\Rightarrow k\in(6, 8)$, so the integral value of $k$ is $7$.

  The diagram is given below:

  \startplacefigure[location=force]
    \startMPcode
      numeric u; u := 1cm;     % scale (points per unit)
      numeric xmin, xmax, dx;
      xmin := 1; xmax := 3; dx := 0.1;

      vardef Y(expr x) = (2*(x**2) - 8*x + 7) enddef;

      % Build the parabola path by sampling x
      path P;
      P := (u*xmin, u*Y(xmin))
      for x=xmin+dx step dx until xmax:
      .. (u*x, u*Y(x))
      endfor;
      % Draw the parabola
      draw P;

      % Axes
      drawarrow (-u,0)--(4u,0);
      drawarrow (0,-u)--(0,2u);
      label.rt("$x$", (4u,0));
      label.top("$y$", (0,2u));
      draw(u, -0.1*u) -- (u, 0.1*u);
      draw(2u, -0.1*u) -- (2u, 0.1*u);
      draw(3u, -0.1*u) -- (3u, 0.1*u);
      label.bot("$1$", (u, -0.1*u));
      label.bot("$2$", (2u, -0.1*u));
      label.bot("$3$", (3u, -0.1*u));
    \stopMPcode
  \stopplacefigure
  %407
\item The diagram is given below:

  \startplacefigure[location=force]
    \startMPcode
      numeric u; u := 1cm;     % scale (points per unit)
      numeric xmin, xmax, dx;
      xmin := 2; xmax := 3.0001; dx := 0.1;

      vardef Y(expr x) = abs(x**2 - 8*x + 15) enddef;

      % Build the parabola path by sampling x
      path P;
      P := (u*xmin, u*Y(xmin))
      for x=xmin+dx step dx until xmax:
      .. (u*x, u*Y(x))
      endfor;
      % Draw the parabola
      draw P;

      numeric xmins, xmaxs;
      xmins := 3; xmaxs := 5;

      path Q;
      Q := (u*xmins, u*Y(xmins))
      for x=xmins+dx step dx until xmaxs:
      .. (u*x, u*Y(x))
      endfor;
      % Draw the parabola
      draw Q;

      numeric xmine, xmaxe;
      xmine := 5; xmaxe :=6.8;

      path R;
      R := (u*xmine, u*Y(xmine))
      for x=xmine+dx step dx until xmaxe:
      .. (u*x, u*Y(x))
      endfor;
      % Draw the parabola
      draw R;

      draw(7u, 7u) -- (3u, -u);
      draw (4u, u) -- (4u, 0) dashed evenly;
      draw ((5 + sqrt(3))*u, 3u + 2*sqrt(3)*u) -- ((5 + sqrt(3))*u, 0) dashed evenly;

      % Axes
      drawarrow (-u,0)--(8u,0);
      drawarrow (0,-u)--(0,4u);
      label.rt("$x$", (8u,0));
      label.top("$y$", (0,4u));
      draw(3u, -0.1*u) -- (3u, 0.1*u);
      draw(4u, -0.1*u) -- (4u, 0.1*u);
      draw(5u, -0.1*u) -- (5u, 0.1*u);
      draw((5 + sqrt(3))*u, -0.1*u) -- ((5 + sqrt(3))*u, 0.1*u);
      label.bot("$3$", (3u, -0.1*u));
      label.bot("$4$", (4u, -0.1*u));
      label.bot("$5$", (5u, -0.1*u));
      label.bot("$(5 + \sqrt{3})$", ((5 + sqrt(3))*u, 0));
    \stopMPcode
  \stopplacefigure

  Given that $|x^2 - 8x + 15| = 2x - 7\Rightarrow x^2 - 8x + 15 = 2x - 7$ or $x^2 - 8x + 15 = 7 - 2x$

  $\Rightarrow x = 5 \pm\sqrt{3}$ or $x = 4, 2$

  We reject $x = 5 - \sqrt{3}$ and $x = 2$ as we can see from diagram that the line $2x - 7 = 0$ does not
  cut $|x^2 - 8x + 15|$ at those points.
  %408
\item Putting $\alpha = -1$ gives $-1 - b + c = 0 \Rightarrow c - b = 1$

  Also, $\alpha\beta\gamma = -c \Rightarrow -1 = -c \Rightarrow c = 1\Rightarrow b = 0$

  So the equation is reduced to $x^3 + 1 = 0$ and other roots are $\omega, \omega^2$.

  Thus, $b^3 + 2c^3 - 3\alpha^3- 6\beta^3 -8\gamma^3 = 19$.
  %409
\item Given $x^2 - 12x +[x] + 31 = 0\Rightarrow \{x\} = x^2 - 11x + 31\Rightarrow 0\leq x^2 - 11x + 31\leq
  1$

  Consider $x^2 - 11x + 31 < 0\Rightarrow x\in(5, 6)$ $[x] = 5\Rightarrow x^2 - 12x + 36 = 0 \Rightarrow x =
  6$ but $x\in(5, 6)\Rightarrow x = \phi$

  If $x\geq -2$ then $x^2 - 5x - 14 = 0 \Rightarrow x = 7, -2$

  If $x\leq -2$ then $x^2 + 5x + 6 = 0 \Rightarrow x = -3, -2$

  So $m = 0, n = 3\Rightarrow m^2 + mn + n^2 = 9$.
  %410
\item $-6 < n^2 - 10n + 19 < 6$. First we consider $n^2 - 10n + 25 > 0 \Rightarrow (n - 5)^2 > 0\Rightarrow
  n\in\mathbb{Z} - \left\{5\right\}$

  Also, $n^2 - 10n + 13 < 0 \Rightarrow n\in\left(5\pm2\sqrt{3}\right)\Rightarrow n\in\left\{2, 3, 4, 5, 6,
  7, 8\right\}$

  Thusm $n\in{2, 3, 4, 6, 7, 8}$. So number of elements in the set is $6$.
  %411
\item Given that $\log_{\left(x + \frac{7}{2}\right)}\left(\frac{x - 7}{2x - 3}\right)^2\geq 0$

  $\Rightarrow x + \frac{7}{2}\geq 0 \Rightarrow x > -\frac{7}{2}$ and $x + \frac{7}{2}\neq 1 \Rightarrow
  x\neq -\frac{5}{2}$

  And $\frac{x - 7}{2x - 3}\neq 0$ and $2x - 3\neq 0$

  $\Rightarrow x\neq 7$ and $x\neq \frac{3}{2}$. Thus, $x\in\left(-\frac{7}{2}, \infty\right)
  - \left\{-\frac{5}{2}, \frac{3}{2}, 7\right\}$

  Now $\log_ab\geq 0$ if $a > 1$ and $b\geq 1$

  {\bf Case I:} $x + \frac{7}{2} > 1$ and $\left(\frac{x - 7}{2x - 3}\right)^2\geq 1$

  $\Rightarrow x > -\frac{5}{2}$ and $(2x - 3)^2 - (x - 7)^2 < 0 \Rightarrow
  x\in\left[-4, \frac{10}{3}\right]$

  $\Rightarrow x\in\left(-\frac{5}{2}, \frac{10}{3}\right]$

  {\bf Case II:} $0 < x + \frac{7}{2}  < 1; \left(\frac{x - 7}{2x - 3}\right)^2\leq 1$

  $\Rightarrow x\in\left(-\infty, -4\right)\cup \left(\frac{10}{3}, \infty\right)$

  Combining all results we have integral solutions as $\left\{-2, -1, 0, 1, 2, 3\right\}$.

  Thus, we have total six soolutions.
  %412
\item Let $e^{2x} = t$ then the equation becomes $t^4 - t^3 -3t^2 - t + 1 = 0$

  $\Rightarrow t^2 - t + 3 - \frac{1}{t} + \frac{1}{t^2} = 0 \Rightarrow \left(t  + \frac{1}{t}\right)^2
  - \left(t + \frac{1}{t}\right) - 5 = 0$

  $\Rightarrow t + \frac{1}{t} = \frac{1 + \sqrt{21}}{2}$ (we take only positive value because $e^{2x} > 0$)

  $\Rightarrow D > 0$ so we have two values for $t$, where the curve cuts $x-$axis.
  %413
\item Given equation is $x^2 + \sqrt{6}x + 3 = 0 \Rightarrow x = \frac{-\sqrt{6}\pm\sqrt{-6}}{2}
  = \sqrt{3}e^{\pm3\pi i/4}$

  $\Rightarrow \frac{\alpha^{23} + \beta^{23} + \alpha^{14} + \beta^{14}}{\alpha^{15} + \beta^{15}
    + \alpha^{10} + \beta^{10}} = \frac{\left(\sqrt{3}\right)^{23}\left(2\cos\frac{69\pi}{4}\right)
    + \left(\sqrt{3}\right)^{15}\left(2\cos\frac{42\pi}{4}\right)}{\left(\sqrt{3}\right)^{15}\left(2\cos\frac{45\pi}{4}\right)
    + \left(\sqrt{3}\right)^{10}\left(2\cos\frac{30\pi}{4}\right)}$

  $= \left(\sqrt{3}\right)^8 = 81$.
  %414
\item Given that $x^2 - \sqrt{2}x + 2 = 0 \Rightarrow x = \alpha, \beta = -\sqrt{2}\omega, -\sqrt{2}\omega^2$

  $\alpha^{14} + \beta^{14} = 2^7\left(\omega^{14} + \omega{28}\right) = -128$.
  %415
\item {\bf Case I:} $x\in[-\infty, -2)\Rightarrow -x^2 + 5x + 16 = 0\Rightarrow x = \frac{5 - \sqrt{89}}{2}$
  we reject $x = \frac{5 + \sqrt{89}}{2}$ as it does not fall in range

  {\bf Case II:} $x\in[-2, -0)\Rightarrow -x^2 - 5x - 4 = 0\Rightarrow x = -1, -4\Rightarrow x = -1$

  {\bf Case III:} $x\in[0, \infty]\Rightarrow x^2 - 5x - 4 = 0\Rightarrow x = \frac{5 + \sqrt{41}}{2}$.
  %416
\item Given that $3\left(x^2 + \frac{1}{x^2}\right) - 2\left(x + \frac{1}{x}\right) + 5  = 0$

  $\Rightarrow 3\left[\left(x + \frac{1}{x}\right)^2 - 2\right] - 2\left(x + \frac{1}{x}\right) + 5 = 0$

  Let $x + \frac{1}{x} = t$, then the equation becomes

  $3t^2 - 2t - 1 = 0 \Rightarrow t = 1, -\frac{1}{3}$, which leads to no real solution.
  %417
\item Given that $\left(1 - \sqrt{3}i\right)^{200} = 2^{199}(p + iq)\Rightarrow
  2^{200}\left(\cos\frac{\pi}{3} - i\sin\frac{\pi}{3}\right)^{200} = 2^{199}(p + iq)$

  $\Rightarrow p = -1, q = -\sqrt{3}\Rightarrow p + q + q^2 = 2 - \sqrt{3}$ and $p - q + q^2 = 2 + \sqrt{3}$

  Thus, equation whose roots are $p + q + q^2$ and $p - q + q^2$ is $x^2 - 4x + 1 = 0$.
  %418
\item we have $\log_2\left(9^{2\alpha - 4} + 13\right) - \log_2\left(\frac{5}{2}3^{2\alpha - 4} + 1\right) =
  2$

  $\Rightarrow \frac{9^{2\alpha - 4} + 13}{\frac{5}{2}3^{2\alpha - 4} + 1} = 4$

  Let $3^{2\alpha - 4} = y\Rightarrow \frac{y^2 + 13}{\frac{5}{2}y + 1} = 4 \Rightarrow y^2 + 13 = 10y + 4$

  $\Rightarrow y^2 - 10y + 9 = 0 \Rightarrow y = 1, 9\Rightarrow 2\alpha - 4 = 0, 2\Rightarrow \alpha
  = 2, 3$

  $\Rightarrow \displaystyle x^2 - 2\left(\sum_{\mstack \alpha\in S}\alpha\right)^2x
  + \sum_{\mstack \alpha\in S}(\alpha + 1)^2\beta = 0$ becomes

  $\Rightarrow x^2 - 50x + 25\beta = 0 \Rightarrow D \geq 0[\because x\in\mathbb{R}]\Rightarrow \beta\leq
  25$

  Hence, $\beta_{\max} = 25$.
  %419
\item $\alpha + \beta = -\sqrt[4]{60}$ and $\alpha\beta = a$. Given that $\alpha^4 + \beta^4 = -30$

  $\Rightarrow \left[\left(\alpha + \beta\right)^2 - 2\alpha\beta\right]^2 - 2(\alpha\beta)^2 = -30$

  $\Rightarrow 60 + 4a^2- 4a\sqrt{60} - 2a^2 = -30\Rightarrow 2a^2 - 4\sqrt{60}a + 90 = 0$

  Porduct of roots i.e. values of $a$ is $45$.
  %420
\item According to question n$\alpha + \beta = \frac{31}{14}, \alpha\beta = \frac{3\lambda}{14}, \alpha
  + \gamma = \frac{53}{35}, \alpha\gamma = \frac{4\lambda}{35}$

  $\Rightarrow \frac{\beta}{\gamma} = \frac{15}{8}$ and $\beta - \gamma = \frac{7}{10}$

  $\Rightarrow \gamma = \frac{4}{5}, \beta = \frac{3}{2}, \alpha = \frac{5}{7}, \lambda = 5$

  Now it is trivial to find the required equation which is $49x^2 - 245x + 250 = 0$.
  %421
\item Given equation is $x\left(x^6 + 3x^4 - 13x^2 - 15\right) = 0$. Clearly, $x = 0$ is a root. Putting
  $x^2 = t$ transforms the equation to

  $t^3 + 3t^2 - 13t - 15 = 0 \Rightarrow (t - 3)(t^2 + 6t + 5) = 0 \Rightarrow x = \pm\sqrt{3}, \pm
  i, \pm\sqrt{5}i$

  Clearly, $\alpha_7 = 0, \alpha_1,\alpha_2 = \pm i, \alpha_3,\alpha_4 = \pm\sqrt{3}$ and
  $\alpha_5, \alpha_6 = \pm\sqrt{5}i$

  Thus, $\alpha_1\alpha_2 - \alpha_3\alpha_4 + \alpha_5\alpha_6 = 1 - (-3) + 5 = 9$.
  %422
\item Given equation can be written as $\sqrt{(x - 1)(x - 3)} + \sqrt{(x - 3)(x + 3)} = \sqrt{(x - 3)(4x -
  2)}$

  Thus, $x = 3$ is a root.

  Now $\sqrt{x - 1} + \sqrt{x + 3} = \sqrt{4x - 2}$, squaring, $2\sqrt{(x - 1)(x + 3)} = 2x - 4$

  $\Rightarrow x^2 + 2x - 3 = x^2 - 4x + 4 \Rightarrow x = \frac{7}{6}$(rejected, it has come due to
  squaring).
  %423
\item Let $e^x = y$, then the equation can be written as

  $y^4 + 8y^3 + 13y^2 - 8y + 1 = 0$. Since $y\neq 0$, we can divide the equation by $y^2$ to obtain

  $y^2 + \frac{1}{y^2} + 8\left(y - \frac{1}{y}\right) + 13 = 0$

  $\Rightarrow \left(y - \frac{1}{y}\right)^2 + 2 + 8\left(y - \frac{1}{y}\right) + 13 = 0$

  Again let $y - \frac{1}{y} = z$, then

  $z^2 + 8z + 15 = 0 \Rightarrow z = -3, -5\Rightarrow y = \frac{\sqrt{13} - 3}{2}, \frac{\sqrt{29} - 5}{2}$

  $x = \log_e\left(\frac{\sqrt{13} - 3}{2}\right), \log_e\left(\frac{\sqrt{29} - 5}{2}\right)$.
  %424
\item $|2a - 1| = [a] + 2a$

  {\bf Case I:} $a > \frac{1}{2}\Rightarrow 2a - 1 = [a] + 2a \Rightarrow [a] = -1$(rejected)

  {\bf Case II:} $a < \frac{1}{2}\Rightarrow -2a + 1 = [a] + 2a$

  Let $a = I + f\Rightarrow I= 0, f= \frac{1}{4}\Rightarrow a = \frac{1}{4}$

  $\Rightarrow \displaystyle\sum_{\mstack a\in S}a = \frac{1}{4}$.
  %425
\item {\bf Case I:} $x^2 + 2x - 5x - 5 - 1= 0 \Rightarrow x^2 - 3x - 6 = 0$

  $\Rightarrow x = \frac{3\pm\sqrt{33}}{2}$, so we have one positive root from this.

  {\bf Case II:} $-1\leq x < 0\Rightarrow x^2 + 7x + 6 = 0 \Rightarrow x = -1, -6$(rejected)

  {\bf Case III:} $-2\leq x < -1\Rightarrow x^2 - 3x - 4 = 0$, we have no root in range.

  {\bf Case IV:} $x < -2\Rightarrow x^2 + 7x + 4 = 0 \Rightarrow x = \frac{-7 - \sqrt{33}}{2}$, the other
  root is rejected as it is outside the range.

  Thus, we have a total of $3$ real roots.
  %426
\item Let $y = \sqrt{8x - x^2 - 12}\Rightarrow y^2 = -(x - 4)^2 + 16 - 12\Rightarrow (x - 4)^2 + y^2 = 4$

  We see that $x^2 - 8x + 12 = (x - 6)(x - 2)$. The following diagram depicts the problem.

  \startplacefigure[location=force]
    \startMPcode
      numeric u;
      u := 0.6cm;

      path c;
      c = fullcircle scaled(4u) shifted(4u, 0) cutbefore(6u, 0) cutafter (2u, 0);
      draw c;
      draw (7u, 4u) -- (4u, 0);
      drawdblarrow(-u, 0) -- (7u, 0);
      drawdblarrow(0, -u) -- (0, 4u);

      label.rt("$x$", (7u, 0));
      label.top("$y$", (0, 4u));
      label.bot("$(2, 0)$", (2u, 0));
      label.bot("$(4, 0)$", (4u, 0));
      label.bot("$(6, 0)$", (6u, 0));
      label.top("$(7, 4)$", (7u, 4u));

      drawdot(2u, 0) withpen pencircle scaled 2pt;
      drawdot(4u, 0) withpen pencircle scaled 2pt;
      drawdot(6u, 0) withpen pencircle scaled 2pt;
      drawdot(7u, 6u) withpen pencircle scaled 2pt;
    \stopMPcode
  \stopplacefigure

  Thus, $m = \left(\sqrt{(7 - 4)^2 + (4 - 0)^2} - 2\right)^2 = 9, M = (7 - 2)^2 + (4 - 0)^2 = 41$

  $M^2 - m^2 = 1600$.
  %427
\item {\bf Case I:} $x\geq -5\Rightarrow x^2 + 7x + 12 = 0\Rightarrow x = 3, -4$

  {\bf Case II:} $-7 < x < -5\Rightarrow x^2 + 3x - 12 = 0\Rightarrow x = \frac{-3 - \sqrt{57}}{2}$(positive
  root is not in range)

  {\bf Case III:} $x\leq -7\Rightarrow x^2 + 7x + 16 = 0 \Rightarrow D < 0$(roots are not real)

  Thus, we have three solutions.
  %428
\item $\frac{U_{10} + \sqrt{12}U_9}{2U_8} = \frac{\alpha^{10} + \beta^{10} + \sqrt{2}\left(\alpha^9
  + \beta^9\right)}{2\left(\alpha^8 + \beta^8\right)}$

  $= \frac{\alpha^8\left(\alpha^2 + \sqrt{2}\alpha\right) + \beta^8\left(\beta^2
  + \sqrt{2}\beta\right)}{2\left(\alpha^8 + \beta^8\right)}$

  $= \frac{8\alpha^8 + 8\beta^8}{2\left(\alpha^8 + \beta^8\right)} = 4$.
  %429
\item {\bf Case I:} $x\leq -3\Rightarrow (x + 1)(x + 3)  + 4(x + 2) + 5 = 0\Rightarrow (x + 4)^2 =
  0 \Rightarrow x = -4$

  {\bf Case II:} $-3\leq x\leq -2\Rightarrow -x^2 - 4x - 3 + 4x + 8 + 5 = 0 \Rightarrow -x^2 + 10 =
  0 \Rightarrow x = \pm\sqrt{10}$

  {\bf Case III:} $-2\leq x\leq -1\Rightarrow -x^2 - 4x - 3 - 4x - 8 + 5 = 0\Rightarrow x^2 + 8x + 6 =
  0\Rightarrow x = -4\pm\sqrt{10}$

  {\bf Case IV:} $x\geq -1\Rightarrow x^2 = 0 \Rightarrow x = 0$
  %430
\item $\alpha + \beta = -2\sqrt{2}, \alpha\beta = -1\Rightarrow \alpha^4 + \beta^4 = \left(\alpha^2
  + \beta^2\right)^2 - 2\alpha^2\beta^2$

  $= \left[\left(\alpha + \beta\right)^2 - 2\alpha\beta\right]^2 - 2\left(\alpha\beta\right)^2 = (8 + 2)^2 -
  2(-1)^2 = 98$

  $\alpha^6 + \beta^6 = \left(\alpha^3 + \beta^3\right)^2 - 2\alpha^3\beta^3 = [(\alpha
    + \beta)\left\{(\alpha + \beta)^2 - 3\alpha\beta\right\}] - 2(\alpha\beta)^3$

  $= 970 \Rightarrow \frac{1}{10}\left(\alpha^6 + \beta^6\right) = 97$

  Thus, the required equation is $x^2 - 195x + 9506 = 0$.
  %431
\item According to question $\alpha^{n + 2} - \sqrt{2}\alpha^{n + 1} - \sqrt{3}\alpha^n = 0$ and $\beta^{n +
  2} - \sqrt{2}\beta^{n + 1} - \sqrt{3}\beta^n = 0$

  Subtracting $P_{n + 2}- \sqrt{2}P_{n + 1} - \sqrt{3}P_n = 0$

  We put $n = 10$ and $n = 9$ to obtain $P_{12} - \sqrt{2}P_{11} - \sqrt{3}P_{10} = 0$ and $P_{11}
  - \sqrt{2}P_{10} - \sqrt{3}P_9 = 0$

  $\Rightarrow 11\left(\sqrt{3}P_{10} + \sqrt{2}P_{11} - P_{11}\right) - 10\left(\sqrt{2}P_{10} -
  P_{11}\right) = 0 - 10\left(-\sqrt{3}P_9\right) = 10\sqrt{3}P_9$.
  %432
\item Since $\alpha, \beta$ are roots of $x^2 - x + 2 = 0 \Rightarrow \alpha^2 = \alpha - 2$ and $\beta^2
  = \beta - 2$

  $\alpha^6 + \alpha^4 + \beta^4 - 5\alpha^2 = \alpha^4(\alpha - 2) + \alpha^4 - 5\alpha^2 + (\beta - 2)^2$

  $= \alpha^5 - \alpha^4 - 5\alpha^2 + \beta^2 - 4\beta + 4 = \alpha^3(\alpha - 2) - \alpha^4 - 5\alpha^2
  + \beta - 2 - 4\beta + 4$

  $= -2\alpha^3 - 5\alpha^2 - 3\beta + 2 = -2\alpha(\alpha - 2) - 5\alpha^2 - 3\beta + 2$

  $= -7\alpha^2 + 4\alpha - 3\beta + 2 = -7(\alpha - 2) + 4\alpha - 3\beta + 2 = -3\alpha - 3\beta + 16 =
  13$.
  %433
\item $\alpha, \beta = \frac{\sqrt{6}}{2}(1\pm i) = \sqrt{3}\left(e^{\pm i\pi/4}\right)$

  $\frac{\alpha^{99}}{\beta} + \alpha^{98} = \alpha^{98}\frac{(\alpha + \beta)}{\beta} =
  3^{49}e^{i99\pi/4}.\sqrt{2}$

  $= 3^{49}(-1 + i) = 3^n(a + ib) \Rightarrow n + a + b = 49 - 1 + 1 = 49$.
  %434
\item $\alpha + \beta = 70, \alpha\beta = \lambda \Rightarrow \lambda = 325, \alpha = 5, \beta = 65$

  $\frac{\left(\sqrt{\alpha - 1} + \sqrt{\beta -1}\right)(\lambda + 35)}{|\alpha - \beta|}
  = \frac{\left(\sqrt{64} + \sqrt{4}\right)(360)}{60} = 60$.
  %435
\item $f(1) = 0$, so $x = 1$ is the second root.

  $\alpha.1 = \frac{c + a - 2b}{a + b - 2c}$

  If $-1 < \alpha < 0\Rightarrow -1 < \frac{c + a - 2b}{a + b - 2c} < 0 \Rightarrow b + c < 2a$ and $b
  > \frac{a + c}{2}$

  so $b$ cannot be the G.M. between $a$ and $c$.

  If $0 < \alpha < 1\Rightarrow 0 < \frac{c + a - 2b}{a + b - 2c} < 1\Rightarrow b > c$ and $b < \frac{a +
    c}{2}$

  so $b$ maybe the G.M. between $a$ and $c$.
  %436
\item $x^2 - 8x + 31 > 0$ because coefficient of $x^2 > 0$ and $D < 0$.

  Coefficient of $x^2$ is $a$, which should be less than zero for the fraction to be less than zero. Thus,
  we have zero elements in the set $S$.
  %437
\item $a = 3\sqrt{2}\Rightarrow a^2 = 18; 1080 = 5.6^3\Rightarrow \sqrt[6]{5}\sqrt{6} = (1080)^{1/6}
  = \frac{1}{b}\Rightarrow 1080^{1/2} = \frac{1}{b^3}$

  $\therefore 3x + 2y = \log_a\left(a^2\right)^{5/4} = \frac{5}{2}$ and $2x - y = \log_b\frac{1}{b^3} = -3$

  $\Rightarrow 4x + 5y = 8$.
  %438
\item $P_1 = \alpha + \beta = 1\Rightarrow \alpha^2 + \beta^2 = (\alpha + \beta)^2 - 2\alpha\beta = 1 -
  2\alpha\beta$

  Let $y = \alpha\beta \Rightarrow P_2 = 1 - 2y \Rightarrow P_4 = (1 - 2y)^2 - 2y^2 \Rightarrow P_8 = 47 =
  [(1 - 2y)^2 - 2y^2]^2 - 2y^4\Rightarrow y = -1 = \alpha\beta$

  Thus, $\frac{1}{\alpha} + \frac{1}{\beta} = -1$ and $\frac{1}{\alpha\beta} = -1$

  So the equation is $x^2 + x - 1 = 0$.
  %439
\item Because both the roots are positive $\Rightarrow D\geq 0$

  $\Rightarrow 9(a - 3)^2 - 9(1 - a)\geq 0\Rightarrow a^2 + 3a\geq 0 \Rightarrow a\in(-\infty, -3]\cup
  [0, \infty)$

  $-\frac{b}{2a} > 0\Rightarrow \frac{a - 3}{a - 1} > 0\Rightarrow a\in(-\infty, 1)\cup (3, \infty)$

  $f(0) = 9 > 0\Rightarrow \alpha\in(-\infty, -3]\cup[0, 1)$

  $2\alpha + \beta + \gamma = 6 + 0 + 1 = 7$.
  %440
\item Since $\alpha, \beta$ are the roots of $x^2 + \sqrt{3}x - 6 = 0$ and $P_n = \alpha^n + \beta^n$

  $\Rightarrow P_n + \sqrt{3}P_{n - 1} - 16P_{n - 2} = 0\Rightarrow P_{25} + \sqrt{3}P_{24} - 16P_{23} = 0$

  $\Rightarrow \frac{P_{25} + \sqrt{3}P_{24}}{2P_{23}} = 8$

  Similarly, $Q_{25} - Q_{23} = \gamma^{25} + \delta^{25} - \gamma^{23} - \delta^{23}
  = \gamma^{23}\left(\gamma^2 0 1\right) + \delta^{23}\left(\delta^2 - 1\right)$

  $= \gamma^{23}(-3\gamma) + \delta^{23}(-3\delta) = -3\left(\gamma^{24} + \delta^{24}\right) =
  -3Q_{24}\Rightarrow \frac{Q_{25} - Q_{23}}{Q_{24}} = -3$

  Hence, $\frac{P_{25} + \sqrt{3}P_{24}}{2P_{23}} + \frac{Q_{25} - Q_{23}}{Q_{24}} = 8 - 3 = 5$.
  %441
\item $x^2 + 4x + 4 = n + 4\Rightarrow (x + 2)^2 = n + 4 \Rightarrow x = -2 \pm\sqrt{n + 4}$

  Given that $20\leq n\leq 100 \Rightarrow \sqrt{24}\leq \sqrt{n + 4}\leq \sqrt{104}$

  $\Rightarrow n\in\left\{5, 6, 7, 8, 9, 10\right\}$ i.e. we have six values for $n$ under the given
  conditions.
  %442
\item Using the location of the roots:

    \startplacefigure[location=force]
    \startMPcode
      numeric u; u := 1cm;     % scale (points per unit)
      numeric xmin, xmax, dx;
      xmin := 0.5; xmax := 2.5; dx := 0.05;
      numeric xmins, xmaxs;
      xmins := 0.5; xmaxs :=2.5;

      vardef Y(expr x) = (x**2 - 3*x + 2) enddef;
      vardef Z(expr x) = (x**2 - 3*x + 1) enddef;

      % Build the parabola path by sampling x
      path P;
      P := (u*xmin, u*Y(xmin))
      for x=xmin+dx step dx until xmax:
      .. (u*x, u*Y(x))
      endfor;
      % Draw the parabola
      draw P;
      path Q;
      Q := (u*xmins, u*Z(xmins))
      for x=xmins+dx step dx until xmaxs:
      .. (u*x, u*Z(x))
      endfor;
      % Draw the parabola
      draw Q;

      % Axes
      drawarrow (-u,0)--(3u,0);
      drawarrow (0,-u)--(0,2u);
      label.rt("$x$", (3u,0));
      label.top("$y$", (0,2u));
      draw(u, -0.1*u) -- (u, 0.1*u);
      draw(2u, -0.1*u) -- (2u, 0.1*u);
      label.bot("$a$", (u, -0.1*u));
      label.bot("$b$", (2u, -0.1*u));
      label.rt("$(x - \alpha)(x - \beta) = 0$", (2.5u, 0.75u));
    \stopMPcode
  \stopplacefigure

  We have to satisfy following conditions:

  $D\geq 0; -\frac{b}{2a} < 0;$ and $a.f(0) > 0$

  $\Rightarrow (p + 2)^2 - 4(2p + 2)\geq 0 \Rightarrow (p + 4)(p - 8)\geq 0\;\mathrm{and}\; p + 2 <
  0\;\mathrm{and}\; 2p + 9 > 0\Rightarrow p\in\left(-\frac{9}{2}, -4\right]$

  $\therefore \beta - 2\alpha = -4 + 9 = 5$.
  %443
\item {\bf Case I:} $x < 2\Rightarrow -x^2 + 2x - 3x + 9 + 1 = 9 \Rightarrow x^2 + x - 10 = 0\Rightarrow x =
  \frac{-1 + \sqrt{41}}{2}, \frac{-1 - \sqrt{41}}{2}$(first one is out of range)

  {\bf Case II:} $2\leq x < 3\Rightarrow x^2 - 2x - 3x + 9 + 1 = 0 \Rightarrow x^2 - 5x + 10 = 0\Rightarrow
  D < 0$. Thus we have no real roots.

  {\bf Case III:} $x\geq 3\Rightarrow x^2 + x - 8 = 0 \Rightarrow x = \frac{-1\pm \sqrt{32}}{2}$(both are
  out of range)

  Thus, we have one real root.
  %444
\item $|x + 2|^2 + |x - 2| - 2 = 0\Rightarrow \left(|x - 2| + 2\right)\left(|x - 2| - 1\right) = 0$

  $\Rightarrow |x - 2| = 1\Rightarrow x = 3, 1$. Thus, sum of squares of the roots is $10$.

  If $x - 3\geq 0 \Rightarrow x^2 - 2x + 1= 0 \Rightarrow x = 1$, which is out of range.

  If $x < 3 \Rightarrow x^2 + 2x - 11 = 0 \Rightarrow D > 0, f(3) > 0, -\frac{b}{2a} = -1 < 3$

  So both roots are less than $3$, and thus, both are acceptable.

  Sum of square of roots is $4 + 22 = 26$. Final sum is $36$.
  %445
\item Given that $e^{5\left(\log_ex\right)^2 + 3} = x^8\Rightarrow 5\left(\log_ex\right)^2 + 3 = 8\log_ex$

  Let $\log_ex = y$, then the equation transforms to $5y^2 + 3 - 8y = 0$

  Sum of roots, $y_1 + y_2 = \frac{8}{5}\Rightarrow \log_ex_1x_2 = \frac{8}{5}\Rightarrow x_1x_2 = e^{8/5}$.
  %446
\item Clearly, $x = 1$ is a root as sum of coefficients is zero. Let $\alpha, \beta$ be its roots. Since
  both roots are equal, therefore, both roots are $1$.

  $\alpha + beta = -\frac{b(c - a)}{a(b - c)} = 2\Rightarrow -bc + ab = 2ab - 2ca\Rightarrow 2ac = b(a +
  c)\Rightarrow 2ac = 15b$

  $2ac = 108 \Rightarrow ac = 54, a + c = 15\Rightarrow a^2 + c^2 = 117$.
  %447
\item $S_{11} = \frac{11}{2}(2a + 10d) = 88 \Rightarrow a + 5d = 8\Rightarrow a = \frac{1}{2}$

  Roots are $t_{10} = a + 9d = 14, t_{11} = a + 10d = \frac{31}{2}$

  $t_{10} + t_{11} = \frac{p}{3}\Rightarrow p = \frac{177}{2}$

  $\frac{q}{3} = t_{10}.t_{11} = 217 \Rightarrow q = 651$

  $q - 2p = 474$.
  %448
\item $\left(x^2 - 9x + 11\right)^2 - \left(x^2 - 9x + 20\right) = 3$. Let $x^2 - 9x = y$ to transform the
  equation to

  $(y + 11)^2 - (y + 20) = 3 \Rightarrow y^2 + 21y + 98 = 0 \Rightarrow y = -7, -14$

  $\Rightarrow x = \frac{9\pm\sqrt{53}}{2}, \frac{9\pm5}{2}$

  Thus, product of rational roots is $7\times 2 = 14$.
  %449
\item $\log_{2/\pi}|\sin x| + \log_{2/\pi}|\cos x| = 2\Rightarrow |\sin 2x| = \frac{8}{\pi^2}$

  The plot of the $y = |\sin 2x|$ and $y = \frac{8}{\pi^2}$ is given below:

  \startplacefigure[location=force]
    \startMPcode
      numeric u, pi;
      u = 1cm;
      pi = 3.141592654;
      numeric xmin, xmax, dx;
      xmin := -0; xmax := pi/2; dx := 0.05;

      vardef Y(expr x) = (abs(sin(2*x))) enddef;

      path P;
      P := (u*xmin, u*Y(xmin))
      for x=xmin+dx step dx until xmax:
      .. (u*x, u*Y(x))
      endfor;
      draw P;
      path Q;
      P := (u*pi/2, u*Y(pi/2))
      for x=pi/2+dx step dx until pi:
      .. (u*x, u*Y(x))
      endfor;
      draw P;

      drawarrow(-u, 0) -- (4u, 0);
      drawarrow(0, -u) -- (0, 2u);
      draw(-u, 8/(pi**2)*u) -- (4u, 8/(pi**2)*u);

      draw (pi/2*u, -0.1u) -- (pi/2*u, 0.1u);
      draw (pi*u, -0.1u) -- (pi*u, 0.1u);

      label.bot("$\pi/2$", (pi/2*u, -0.1u));
      label.bot("$\pi$", (pi*u, -0.1u));
      label.rt("$x$", (4*u, 0));
      label.top("$y$", (0, 2u));
    \stopMPcode
  \stopplacefigure

  There are four points of intersection so there are four solutions.

  Let $\sqrt{x} = t < 2$, which transforms $B$ into

  $t^2 - 4t + 3t - 6 + 6 = 0\Rightarrow t = 0, 1 \Rightarrow x = 0, 1$

  Again let $\sqrt{x} = t > 2\Rightarrow t^2 - 7t + 12 = 0 \Rightarrow x = 9, 16$

  Thus, $n(A\cup B) = 4 + 4 = 8$.
  %450
\item {\bf Case I:} $x\geq \frac{3}{2}\Rightarrow x^2 + 2x - 7 = 0 \Rightarrow x = 2\sqrt{2} - 1$

  {\bf Case II:} $x\leq \frac{3}{2}\Rightarrow x^2 - 2x - 1= 0 \Rightarrow x = 1 - \sqrt{2}$

  Then sum is $\left(2\sqrt{2} - 1\right)^2 + \left(1 - \sqrt{2}\right)^2 = 6\left(2 - \sqrt{2}\right)$.
  %451
\item We rewrite the equation by letting $\frac{1}{\sqrt{x}} = y$ to

  $\left(9y^2 - 9y + 2\right)(2y^2 - 7y + 3) = 0\Rightarrow (3y - 2)(3y - 1)(y - 3)(2y - 1) = 0$

  $\Rightarrow y = \frac{1}{3}, \frac{1}{2}, \frac{2}{3}, 3\Rightarrow x = 9, 4, \frac{9}{4}, \frac{1}{9}$,
  thus we have four solutions.
  %452
\item Total no.\  of five letter words made with ten letters without any restriction is $10^5$.

  No.\ of words of five letters out of ten letters without repetition is $C_5^^{10} = 252$.

  Therefore, required no.\ of words is $10^5 - 252 = 99748$.
  %453
\item Two women can occupy chairs from $1$ to $4$ in $P_2^^4$ ways. Three men can occupy from remaining $6$
  chairs in $P_3^^6$ ways.

  Thus, total no.\ of seating arrangements is $P_2^^4\times P_3^^6$.
  %454
\item There are four odd digits with two pairs of repeated numbers. Thus, no.\ of ways of filling odd
  positions is $\frac{4!}{2!2!} = 6$

  There are three even places which are to be filled by $2$ unique digits with one of them being repeated is
  $\frac{3!}{2!} = 3$

  Thus, total no.\ of numbers formed is $6\times3 = 18$.
  %455
\item In the word INDEPENDENCE we have $5$ vowels out of which one is \quote{I} and $4$ are \quote{E}. We
  can treat these vowels as one letter but they can be arranged in $\frac{5!}{4!} = 5$ ways among
  themselves.

  For consonants we have $3$ \quote{n}, $2$ \quote{d} and $1$ \quote{p} and \quote{c}. We will also need to
  include all vowels as one letter. Thus, we have a total of $8$, which can be arranged in $\frac{8!}{3!2!}
>  = 3360$ ways.

  Thus, total no.\  of words that can be formed is $3360\times 5 = 16,800$.
  %456
\item Four out of five smaller animals can be arranged in $P_4^^5 = 120$ ways.

  Remaining six cages can be filled in $6! = 720$ ways. Thus, total no.\ of ways is $120\times720 = 86,400$.
  %457
\item {\bf Case I:} When two points are on the given straight line. No.\ of ways of constructing a triangle
  is $C_2^^7\times C_1^^5 = 105$

  {\bf Case II:} When one point is on the given straight line. No.\ of ways of constructing a triangle is
  $c_1^^7\times C_2^^5 = 70$

  {\bf Case III:} When none of the points from straight line is taken. No.\ of ways of constructing a
  triangle is $C_3^^{5} = 10$.

  Thus, total no.\ of possible triangles are $185$.
  %458
\item Each bulb can be on or off. Thus, total no.\ of permutations is $2^{10} = 1024$. But in one of these
  all the bulbs are off. So no.\ of ways of illuminating the room is $1023$.
  %459
\item {\bf Case I:} If first three digits are $432$, then unit's place can be filled in $4$ ways using one
  of $2, 3, 4, 5$.

  {\bf Case II:} If first two digits are $43$, then ten's placce can be filled with one of $3, 4, 5$ and
  unit's place can be filled with any digits. Thus, total no.\ of numbers is $3\times 6 = 18$

  {\bf Case III:} If first digits is $4$. Hundred's place can be filled by one of $4, 5$ and remaining two
  places can be filled by any digit. Thus, total no.\ of numbers is $2\times6\times6 = 72$

  {\bf Case IV:} If first digit is $5$. Remaining places can be filled by any digits in $6^3 = 216$ ways.

  Thus, total no.\ of numbers is $4 + 18 + 72 + 216 = 310$.
  %460
\item Consider the matrix $M = \startbmatrix\NC a_1\NC a_2\NC a_3\NR\NC a_4\NC a_5\NC a_6\NR\NC a_7\NC a_8\NC
  a_9\NR\stopbmatrix$ then

  $M^T = \startbmatrix\NC a_1\NC a_4\NC a_7\NR\NC a_2\NC a_5\NC a_8\NR\NC a_3\NC a_6\NC a_9\NR\stopbmatrix$
  then we consider only diagonal entries

  $M^TM = \startbmatrix\NC a_1^2 + a_4^2 + a_7^2\NC \NC\NR\NC\NC
  a_2^2 + a_5^2 + a_8^2\NC\NR\NC\NC\NC a_3^2 + a_6^2 + a_9^2\NR\stopbmatrix$

  Thus, we see that sum of diagonal entries is $\displaystyle\sum a_i^2$.

  Thus, we have two cases $\left\{2, 1, 0, 0, 0, 0, 0, 0, 0\right\}$ and $\left\{1, 1, 1, 1, 1, 0, 0, 0,
  0\right\}$.

  Total permutations are $\frac{9!}{7!} + \frac{9!}{4!5!} = 198$.
  %461
\item {\bf Case I:} When the number is of four digits. Thousand's place can be filled by $6, 7, 8$ i.e. in
  $3$ ways. Remaining $3$ digits can be filled by remaining $4$ digits in $P_3^^4 = 24$ ways.

  Thus, total no.\ of integers in this case is $72$.

  {\bf Case II:} When the integer is of five digits. Such numbers can be made in $P_5^^5 = 120$ ways.

  $\therefore $ Total no.\ of integers is $192$.
  %462
\item {\bf Case I:} Five ones, one two and one three. Total no.\ of such integers is $\frac{7!}{5!} = 42$

  {\bf Case II:} Four ones and three twos. Total no.\ of such integers is $\frac{7!}{4!3!} = 35$

  $\therefore $ Total no.\ of seven-digit integers is $77$.
  %463
\item Consider the no.\ as $OEOEOEOEO$ where $O$ denotes odd positions and $E$ denotes even places. We have
  four odd digits $33$ and $55$ and also four even positions.

  The odd digits can be placed at even positions in $\frac{4!}{2!2!} = 6$ ways.

  Five remaining digits can be put at odd positions in $\frac{5!}{2!3!} = 10$ ways.

  Thus, total no.\ of required nine-digit integers is $10\times 6 = 60$.
  %464
\item Distinct $n$-digit numbers that can be formed using $3$ digits is $3^n\geq 900 \Rightarrow n = 7$.
  %465
\item Let there be $n$ newspapaers. Then $60n = 300\times 5 \Rightarrow n = 25$.
  %466
\item Since the number has to be divisible by $3$ the sum of digits must be divisible by $3$.

  {\bf Case I:} When the number is made by $0, 1, 2, 4, 5$.

  Total no.\ of such numbers is $4.4! = 96$.

  {\bf Case II:} When the number is made by $1, 2, 3, 4, 5$.

  Total no.\ of such numbers is $5! = 120$.

  Thus, total no.\ of required five-digit numbers is $216$.
  %467
\item Four guests who want to be seated on a particular side can be seated in $P_4^^9$ ways and three other
  can be seated on the other side in $P_3^^9$ ways.

  Remaining $11$ guests can be seated in $11!$ ways.

  Thus, total no.\ of seating arrangements is $P_4^^9\times P_3^^9\times11!$.
  %468
\item Questions can be answered by picking $3, 1, 1$ or $2, 2, 1$ questions from each paper. Now each
  combination can be rotated in $3$ ways.

  Thus, total no.\ of ways is $3\left(C_3^^5\times C_1^^5\times C_1^^5\right) + 3\left(C_2^^5\times
  C_2^^5\times C_1^^5\right) = 2250$.
  %469
\item We can choose $10$ objects out of $31$ in following way:

  Number of ways for $0$ identical $+ 10$ distinct is $1\times C_{10}^{21}$

  Number of ways for $1$ identical $+ 9$ distinct is $1\times C_{9}^{21}$

  $\ldots$

  Number of ways for $10$ identical $+ 0$ distinct is $1\times C_{0}^{21}$

  Let $\Rightarrow C_0^^{21} + C_1^^{21} + \cdots + C_{10}^^{21} = x$

  We know that $C_r^^n = C_{n - r}^^n \Rightarrow C_{11}^^{21} + C_{12}^^{21} + \cdots + C_{21}^^{21} = x$

  Adding $2x = C_0^^{21} + C_1^^{21} + \cdots + C_{21}^^{21} = 2^{21}\Rightarrow x = 2^{20}$.
  %470
\item Total no.\ of beams is total no.\ of diagonals of $20$-side polygon.

  Total no.\ of diagonals is $C_2^^{20} - 20 = 170$, which is the number of beams.
  %471
\item Let $n$ be the no.\ of balls on each side of the equilateral triangle. Then according to question

  $\frac{n(n + 1)}{2} + 99 = (n - 2)^2 \Rightarrow n = 19$.

  Thus, total no.\ of balls used to make the equilateral triangle is $\frac{19\times20}{2} = 190$.
  %472
\item No.\ of games played by men among themselves is $2\times C_2^^m$

  No.\ of games played between men and women is $2\times C_1^^m\times C_1^^2$

  According to the condition given in the question $2\times C_2^^m = 2\times C_1^^m \times C_1^^2 + 84$

  $\Rightarrow m(m - 1) = 4m + 84 \Rightarrow m^2 - 5m - 84 = 0 \Rightarrow (m - 12)(m + 7) = 0$

  $m = 12$. We reject $m = -7$ as no.\ of men players cannot be negative.
  %473
\item Given $\displaystyle\sum_{r = 0}^{25}C_r^^{50}.C_{25 - r}^^{50 - r} = K.C_{25}^^{50}$

  $\Rightarrow \displaystyle\sum_{r = 0}^{25}\left(\frac{50!}{r!(50 - r)!}.\frac{(50 - r)!}{(25 -
  r)!25!}\right) = K.C_{25}^^{50}$

  $\Rightarrow \displaystyle\sum_{r = 0}^{25}\left(\frac{50!}{25!25!}.\frac{25!}{r!(25 - r!)}\right) =
  K.C_{25}^^{50}$

  $\Rightarrow C_{25}^^{\50}\displaystyle\sum_{r = 0}^{25}C_{r}^^{25} = K.C_{25}^^{50} \Rightarrow K
  = \sum_{r = 9}^^{25}C_r^^{25}$

  $K = 2^{25}$.
  %474
\item Given $\displaystyle\sum_{i = 1}^{20}\left(\frac{C_{i - 1}^^{20}}{C_i^^{20} + C_{i -
    1}^^{20}}\right)^3 = \frac{k}{21}\Rightarrow \sum_{i = 1}^{20}\left(\frac{C_{i -
    1}^^{20}}{C_i^^{21}}\right)^3$

  $\Rightarrow \displaystyle\sum_{i = 1}^{20}\left(\frac{C_{i - 1}^^{20}}{\frac{21}{i}C_{i -
    1}^^{20}}\right)^3 = \frac{k}{21}$

  $\Rightarrow \frac{1}{21^3}\displaystyle\sum_{i = 1}^{20}i^3 = \frac{k}{21} =
  \frac{1}{21^3}\left[\frac{n(n + 1)}{2}\right]^2$

  $\Rightarrow k = \frac{21}{21^3}\left(\frac{21\times20}{2}\right)^2 = 100$.
  %475
\item Man can invite $0$ men and $3$ ladies or $1$ man and $2$ ladies or $2$ men and $1$ lady or all $3$ as
  men. Similarly, the wife can invite $0$ men and $3$ ladies or $1$ man and $2$ ladies or $2$ men and $1$
  lady or all $3$ as men in a complementary manner.

  Thus, total no.\ of ways $= C_3^^3\times C_0^^4\times C_0^^4\times C_3^^3 + C_2^^3\times C_1^^4\times
  C_1^^4\times C_2^^3 + C_1^^3\times C_2^^4\times C_2^^4\times C_1^^3 + C_0^^3\times C_3^^4\times
  C_3^^4\times C_0^^3$

  $= 1 + 144 + 324 + 16 = 485$.
  %476
\item There are $5$ odd digits in $S$ and $4$ even digits. $N_1$ will have $1$ odd digit and $4$ even
  digits.

  Thus, no.\ of ways to form $N_1 = C_1^^5\times C_4^^4 = 5$

  Similarly, no.\ of ways to form $N_2 = C_2^^5\times C_3^^4 = 40$

  Similarly, no.\ of ways to form $N_3 = C_3^^5\times C_2^^4 = 60$

  Similarly, no.\ of ways to form $N_4 = C_4^^5\times C_1^^4 = 20$

  Similarly, no.\ of ways to form $N_5 = C_5^^5\times C_0^^4 = 1$

  Thus, total no.\ of ways is $126$.
  %477
\item There are two ways to form the team. Either with all $4$ members as girls or $1$ boy and $3$ girls as
  we can have at most $1$ boy in the team.

  Thus, no.\ of ways of selecting members is $C_4^^6 + C_3^^6\times C_1^^4 = 95$

  The captain can be selected from $4$ team members in $C_1^^4 = 4$ ways.

  Thus, total no.\ of ways of forming the team is $95\times4 = 380$.
  %478
\item If there are $n$ vertices of a regular polygon then no.\ of triangles possible would be $T_n = C_3^^n
  = \frac{n(n - 1)(n - 2)}{6}$

  Given $T_{n + 1} - T_n = 10 \Rightarrow \frac{n(n - 1)}{6}[n + 1 - n + 2] = 10 \Rightarrow n^2 - n - 20 =
  0\Rightarrow n = 5$ as $n$ cannot be negative.
  %479
\item We can pick $(p, q)$ to have factors or $r$ as $\left(1, r^2\right), \left(r, r^2\right), \left(r^2,
  r^2\right), \left(r^2, r\right), \left(r^2, 1\right)$ i.e. $5$ ways.

  Similarly, we can pick $s$ in $9$ ways and $t$ in $5$ ways.

  Thus, total no.\  of ordered pairs are $5\times9\times 5 = 225$.
  %480
\item Since the persons are seated in a circular fashion we can distribute the hats in the combination of
  $2R + 2B + 1G$ or $2R + 2G + 1B$ or $2B + 2G + 1R$ i.e. three wyas.

  We can pick a person from $5$ in $5$ ways. Once we have picked person we can go either clockwise or
  counter-clockwise.

  Thus, total no.\ of ways are $3\times5\times 2 = 30$.
  %481
\item Adjacent points form the sides of an $n$-sided polygon. Thus, no.\ of blue lines is $n$. Red lines are
  the diagonals of this polygon, which is given by $C_2^^n - n$.

  According to the qeuestion $C_2^^n - n = n \Rightarrow \frac{n(n - 1)}{2} = 2n \Rightarrow n^2 - 5n = 0
  \Rightarrow n = 5$.
  %482
\item The number of students who gave wrong answer to exactly one question $=$ No.\ of students who gave at
  least one wrong answer - No.\ of students who gave at least two wrong answers $= a_1 - a_2$

  Similarly, number of students who gave wrong answer to exactly one question $= a_2 - a_3$, and so on.

  Total no.\ of wrong answers is $a_1 - a_2 + 2\left(a_2 - a_3\right) + 3\left(a_3 - a_4\right) + \cdots +
  (k - 1)(a_{k - 1} - a_k) + k(a_k - 0)$(the last term is $0$ because no students gave more than $k$ wrong
  answers)

  $= a_1 + a_2 + a_3 + \cdots + a_k$.
  %483
\item The sum of given digits is $24$. Let the number be $abcdef$, where $a, b, c, d, e, f$ are one of the
  given digits. Since the number has to be divisible by $11$ the sum of difference of digits at odd and even
  places should be either $0$ or multiple of $11$.

  Hence, possible case is $a + c + e = 12 = b + d + f$.

  {\bf Case I:} Set $\left\{a, c, e\right\} = \left\{0, 5, 7\right\}$ and $\left\{b, d, f\right\}
  = \left\{1, 2, 9\right\}$

  So no.\ of possible six-digit numbers are $2\times2!\times 3! = 24$[$\because 0$ cannot be put at most
    significant position]

  {\bf Case II:} Set $\left\{a, c, e\right\} = \left\{1, 2, 9\right\}$ and $\left\{b, d, f\right\}
  = \left\{0, 5, 7\right\}$

  No.\ of possible six-digit numbers are $3!\times 3! = 36$

  Thus, total no.\ of possible six-digit numbers is $60$.
  %484
\item We can have $6, 7$ or $8$ males for $m$, which will cause selection of $5, 4$ and $3$ females.

  Thus, $m = C_6^^8\times C_5^^5 + C_7^^8\times C_4^^5  + C_8^^8\times C_3^^5 = 78$

  Similarly, when at least $3$ females are chosen, will have the same no.\ of ways. Essentially $m$ and $n$
  are same. Therefore, $m = n = 78$.
  %485
\item The no.\ of ways of choosing balls such that $n_21 < n_2 < n_3$ is same as selecting $3$ balls out of
  $10$, which is $C_3^^{10}$.
  %486
\item Number of one-digit numbers which can be formed is $4$.

  For two-digit numbers ten's place can be filled in $4$ ways($0$ cannot occupy most significant plance) and
  unit's place in $5$ ways. Thus, number of two-digit numbers is $4\times5 = 20$.

  For three-digit numbers hundred's place can be filled in $4$ ways and ten's and unit's place can be filled
  in $5^2$ ways. Thus, number of three-digit numbers is $4\times 5^2 = 100$.

  For four-digit numbers thousands's place can be filled in $2$ ways either with $7$or with $9$ and
  hundred's, ten's and unit's place can be filled in $5^3$ ways. Thus, number of three-digit numbers is
  $2\times 5^3 = 250$.

  Thus, total no.\ of required numbers is $374$.
  %487
\item Let us call this boys $x$ and $y$. No.\ of ways in which $x$ is selected by $y$ is not is
  $C_2^^5.C_2^^5 = 100$

  Similarly, no.\ of ways in which $y$ is selected by $x$ is not is $100$

  No.\ of ways in which neither is selected is $C_2^^5.C_2^^5 = 100$

  Thus, total no.\ of ways is $300$.
  %488
\item No.\ of days when the engineer does not visit the factory is $11$.

  To ensure no two visits are on consecutive days, the $4$ visited days must be placed in the gaps created
  by the $11$ unvisited days. If there are $n$ unvisited days, there are $n + 1$ possible gaps where the
  visited days can be placed. In this case, with $11$ unvisited days, there are $12$ gaps.

  No.\ of ways of choosing $4$ gaps out of $12$ is $C_4^^{12} = 495$.
  %489
\item $x = 10!, y = C_1^^{10}\times C_8^^9\times \frac{10!}{2!} = 45\times10!$.
  %490
\item We can form the queue as $BGBGBGBGBG$ or $GBGBGBGBGB$. So we have arrangement like $-B-B-B-B-B-$,
  where $-$ shows the vacant positions to be filled by girls.

  Out of $5$ girls $4$ are together and $1$ is seprate, To select $2$ positions out of $6$ can be done in
  $C_2^^6$ ways. $4$ girls can be selected out of $5$ in $C_4^^5$ ways.

  $2$ groups of girls can be arranged in $2!$ ways. The group of $4$ girls and $5$ boys can be arranged in
  $4!\times5!$ ways.

  Thus, $m = C_2^^6\times C_4^^5\times 2\times 4!\times5!$

  $n = 5!\times6!\Rightarrow \frac{m}{n} = 5$.
  %491
\item As $n_1\geq 1, n_2\geq 2, n_3\geq 3, n_4\geq 4, n_5\geq 5$, let $n_1 -1 = x_1\geq 0, n_2 - 2 = x_2\geq
  0, n_3 - 3 = x_3\geq 0, n_4 - 4 = x_4\geq 0, n_5 - 5 = x_5\geq 0$

  The given eqution becomes $x_1 + x_2 + x_3 + x_4 + x_5 = 5$

  \setupTABLE[column][align=middle, width=1cm]
  \midaligned{
    \bTABLE[split=yes] % allow splitting over page boundaries
    \bTR \bTD $x_1$ \eTD \bTD $x_2$\eTD \bTD $x_3$ \eTD \bTD $x_4$\eTD\bTD $x_5$\eTD\eTR
    \bTR \bTD[width=1cm] $0$ \eTD \bTD $0$\eTD \bTD $0$ \eTD \bTD $0$\eTD\bTD $5$\eTD\eTR
    \bTR \bTD $0$ \eTD \bTD $0$\eTD \bTD $0$ \eTD \bTD $1$\eTD\bTD $4$\eTD\eTR
    \bTR \bTD $0$ \eTD \bTD $0$\eTD \bTD $0$ \eTD \bTD $2$\eTD\bTD $3$\eTD\eTR
    \bTR \bTD $0$ \eTD \bTD $0$\eTD \bTD $1$ \eTD \bTD $1$\eTD\bTD $3$\eTD\eTR
    \bTR \bTD $0$ \eTD \bTD $0$\eTD \bTD $1$ \eTD \bTD $2$\eTD\bTD $2$\eTD\eTR
    \bTR \bTD $0$ \eTD \bTD $1$\eTD \bTD $1$ \eTD \bTD $1$\eTD\bTD $2$\eTD\eTR
    \bTR \bTD $1$ \eTD \bTD $1$\eTD \bTD $1$ \eTD \bTD $1$\eTD\bTD $1$\eTD\eTR
    \eTABLE
  }

  Thus, we see that there are $7$ possible arrangements.
  %492
\item The no.\ of solutions of $x_1 + x_2 + \cdots = x_k = n = $

  Coefficient of $t^n$ in $\left(t + t^2 + t^3 + \cdots\right)\left(t^2 + t^3 + \cdots\right)\ldots\left(t^k
  + t^{k + 1} + \cdots\right)$

  $\Rightarrow 1 + 2 + \cdots + k = \frac{k(k + 1)}{2}= p$(let) and $1 + t + t^2 + \cdots = \frac{1}{1 - t}$

  Thus, the number of required solutions $=$ Coefficient of $t^{n - p}$ in $\left(1 - t\right)^{-k}$

  $= C_{n - p}^^{k + n - p - 1}$, where

  $k + n - p - 1 = k + n - 1 - \frac{k(k + 1)}{2} = \frac{1}{2}(2n - k^2 + k - 2)$.
  %493
\item Let vertex on $x$-axis be $(a, 0)$ and on $y$-axis be $(0, b)$.

  Now the area would be $\frac{1}{2}ab = 50 \Rightarrow ab = 100$. Given that $(a, b)$ are integers. So we
  have following combinations $(1, 100), (2, 50), (4, 25), (5, 20), (10, 10), (20, 5), (25, 4), (50, 2),
  (100, 1)$.

  Now there will be four combinations as these can be negative and we can have hypotenuse in all four
  quadrants. Thus, total no.\ of elements in set $S$ is $36$.
  %494
\item We can select $4$ novels out of $6$ in $C_4^^6 = 15$ ways. Also, number of ways of selecting $1$ from
  $3$ dictionaries is $C_1^^3 = 3$.

  Total number of arrangements of $4$ novels and $1$ dcitionary such that dictionary is always in the middle
  is $15\times3\times 4! = 1080$.
  %495
\item If we distribute $n^2$ objects in $n$ group such that each group contains $n$ identical objects, then
  number of arrangements is given by:

  $C_n^^{n^2}.C_n^^{n^2 - n}\ldots C_n^^n = \frac{(n^2)!}{(n!)^n}$, which will always be an integer.
  %496
\item $240 = 2^4.3.5$, therefore, total no.\ of divisors is $5.2.2 = 20$.

  Out of these $2, 6, 10$ and $30$ are of the form $4n + 2$.
  %497
\item This is a straight forward problem on which direct formula o derangements can be applied.

  No.\ of ways is $4!\left(1 - \frac{1}{1!} + \frac{1}{2!} - \frac{1}{3!} + \frac{1}{4!}\right) = 9$.
  %498
\item If the word starts with \quote{A}, then we have $\frac{6!}{2!} = 360$ words. Similarly, if the word
  starts with \quote{D}, \quote{I} and \quote{K} the no.\ of words would be $360$ for each of them.

  No.\ of words starting with \quote{MAD}, \quote{MAI} or \quote{MAK} would be $\frac{4!}{2!} = 12$ each.

  No.\ of words starting with \quote{MAND} or \quote{MANI} would be $3! = 6$ each.

  No.\ of words starting with \quote{MANKD} is $2! = 2$. Finally, we have two more words \quote{MANKIDN} and
  the last word would be \quote{MANKIND} itself.

  Thus, position of the word would be $360\times 4 + 12\times 3 + 6\times 2 + 2 + 2 = 1492$.
  %499
\item Let the number be $100x + 10y + z$. As the number is odd $z$ can be $1, 3, 5, 7, 9$.

  Possible sums of digits is $7, 14, 21$ because maximum sum could be $27$.

  So $x + y$ can assume values of $2, 4, 6, 5, 7, 9, 11, 13, 12, 14, 16, 18$ for the sum to be divisible by
  $7$.

  If $x + y = 2$ then we have two solutions $(1, 1)$ and $(2, 0)$.

  if $x + y = 4$ then we have four solutions $(1, 3), (3, 1), (2, 2)$ and $(4, 0)$.

  Similarly, we can enumerate solutions for other values to find total as $63$.
  %500
\item If $0$ occurs twice then it cannot be at hundred's position, which can be filled in $C_1^^9$ ways,
  making total no.\ of such number $9$.

  If $0$ appears once then it can occupy either unit's place or ten's place, while the other number can be
  picked in $C_1^^9 = 9$ ways. Thus, total no.\ of such numbers is $18$.

  If $0$ is not included then we can pick two numbers in $C_1^^9\times C_1^^8$ ways and the digits can be
  arranged in $\frac{3!}{2!} = 3$ ways. Thus, total no.\ of such numbers is $9\times8\times3 = 216$.

  Thus, required answer is $9 + 18 + 216 = 243$.
  %501
\item $36 = 2^2.3^2$, thus we have following possibile combination of digits. $(1,1,1,4,9), (1,1,1,6,6),
  (1,1,2,2,9), (1,1,3,3,4), (1,2,2,3,3), (1,1,2,3,6)$.

  Thus, total no.\ of such numbers is $\times\frac{5!}{3!} + \frac{5!}{2!3!} + 3\times \frac{5!}{2!2!} +
  \frac{5!}{2!} = 180$.
  %502
\item Since the numbers are between $1000$ and $3000$ we have two cases. First case is when thousand's place
  is occupied by $1$

  In this case, last two digits can be $24, 32, 36, 52, 56, 64$. Thus, no.\ of numbers is $3\times6 = 18$
  because remaining $1$ position can be filled by $3$ remaining nummbers.

  In the second case, thousand's place is occupied by $2$. The last two digits in this case can be $16, 36,
  56, 64$. And hundred's place can be filled in $3$ ways like first case. Thus, no.\ of such numbers is
  $3\times4 = 12$

  Hence, we have a total of $30$ required numbers.
  %503
\item Total no.\ of possible groups is $C_3^^{10}\times C_3^^5 = 1200$

  If we keep the two particular boys in the group toogether, then no.\ of possible groups is $C_1^^8\times
  C_3^^5 = 80$

  Thus, no.\ of groups when the two particular boys are not in the group together os $1120$.
  %504
\item $36 = 2^2.3^3$, so the number should be odd multiple of $2$ and does not have factor of $3$ and $9$.

  Odd multiples of $2$ are $102, 106, 110, \ldots, 998$, which are $225$ in count.

  No.\ of $3$ multiples are $102, 114, 126, \ldots, 990$, which are $75$ in count and also include the
  multiple of $9$.

  Thus, required answer is $225 - 75 = 150$.
  %505
\item If five cubes are places in a row then there are six places where blue cubes can be places. Two are at
  the end and four are between the red cubes.

  Let $x_1, x_2, x_3, x_4, x_5, x_6$ be the number of blue cubes placed in these positions. Then according
  to question

  $x_1 + x_2 + x_3 + x_4 + x_5 + x_6 = 11$, where $x_1, x_6\geq 0$ and $x_2, x_3, x_4, x_5\geq 2$.

  Let $x_2 = t_1 + 2, x_3 = t_2 + 2, x_4 = t_3 + 2$ and $x_5 = t_4 + 2$, then the new equation becomes

  $x_1 + t_1 + t_2 + t_3 + t_4 + x_6 = 3$, where each term is greater than equal to zero.

  Thus, possible combinations are $C_3^^{6 + 3 - 1} = C_3^^8 = 56$.
  %506
\item Total no.\ of password is $10^6 + 10^7 + 10^8$ for $6, 7, 8$ characters. No.\ of passwords when they
  do not have any character from $\left\{1, 2, 3, 4, 5\right\}$ is $5^6 + 5^7 + 5^8$.

  Thus, total no.\ of required passwords is $10^6 - 5^6 + 10^7 - 5^7 + 10^8 - 5^8 = 10^6\times 111 -
  5^6 \times 31$

  $= 5^6\times 7073$.
  %507
\item Required answer is coefficient of $x^{30}$ in $\left(1 + x + x^2 + \cdots + x^{30}\right)^2\left(x^4
  + x^5 + x^6 + x^7\right)\left(x^2 + x^3 + x^4 + x^5 + x^6\right)$

  $= x^6\left(\frac{1 - x^{31}}{1 - x}\right)^2\left(1 + x + x^2 + x^3\right)\left(1 + x + x^2 + x^3 +
  x^4\right)$

  $= x^6\left(1 - x^{31}\right)^2\left(1 - x^4\right)\left(1 - x^5\right)(1 - x)^{-4}$

  $= x^6\left(1 - x^4 - x^5 + x^9\right)(1 - x)^{-4} = C_3^^{27} - C_3^^{23} - C_3^^{22} + C_3^^{18} = 430$.
  %508
\item $\displaystyle\sum_{k = 1}^{10}k^2\left(C_k^^{10}\right)^2 = \sum_{k =
  1}^{10}\left(k.C_k^^{10}\right)^2 = \sum_{k= 1}^{10}\left(10.C_{k -1}^^9\right)^2$

  $= 100\displaystyle\sum_{k = 1}^{10}C_{k - 1}^^9.C_{10 - k}^^9 = 100\sum_{k = 1}^{10}C_9^^{18} = 4862000$.
  %509
\item We pick a set of $b_1b_2b_3$ as consecutive integers for which we have $98$ sets and then fourth
  element can be chosen in $97$ ways from remaining integers. Let it be $n(A)$.

  This is also true for $b_2b_3b_4$; we call it $n(B)$. For numbers where we have $b_1b_2b_3b_4$ we have
  $97$ sets; it would be $n(A\cap B)$.

  Thus, required answer is $n(A\cup B) = n(A) + n(B) - n(A\cap B) = 18915$.
  %510
\item Let the number be $abcd$, where $abc$ are to be divisble by $d$.

  If $d = 1$, we can choose $a$ in $9$ ways, and $bc$ in $10$ ways each. Thus, total no.\ of these numbers
  is $9\times10\times10 = 900$.

  If $d = 2$, we can put $a\in\left\{2, 4, 6, 8\right\}$ i.e. $4$ ways and $bc$ in $5$ ways each. Thus,
  total no.\ of these numbers is $4\times5\times5 = 100$.

  Similarly, proceeding for all digits we find the final answer to be $1086$.
  %511
\item If we fix first three digits to $202$ then unit't place can be filled in $5$ ways using $2, 3, 4, ,6,
  7$.

  If we fix first two digits to $20$ then remaining two places can be filled using $(3, 4, 6, 7)$ and any of
  the given digits.

  Similarly if we fix first digit to $2$, hundred's place can be filled using $(2, 3, 4, 6, 7)$ and remainng
  two places in $6\times 6$ ways.

  If thousand's place is occupied by $3$ we can fill remaining places in $6\times6\times6$ ways.

  If thousand's place is occupied by $4$, we can fill hundred's place using $0, 2, 3, 4$ and remaining
  places in $6\times6$ ways.

  Thus, total no.\ of such numbers would be $5 + 24 + 180 + 216 + 144 = 569$.
  %512
\item {\bf Case I:} When one bad gives four balls. This bag can be selected in $C_1^^4$ ways. Four balls can
  be chosen in two ways. A combination of $3$ red and $1$ blue ball or $2$ of each i.e. $C_3^^3\times C_1^^2
  + C_2^^3\times C_2^^2$

  From $3$ bags one ball of each color can be chosne in $\left(C_1^^3 + C_1^^2\right)^3$ ways.

  Thus, total no.\ of ways is $C_1^^4\left(C_3^^3\times C_1^^2 + C_2^^3\times C_2^^2\right)\left(C_1^^3\times
  C_1^^2\right)^3$

  {\bf Caase II:} We pick $3$ balls from two bags and $2$ from two bags. Ways of choosing $2$ bags out of
  $4$ is $C_2^^4$.

  We can pick $3$ balls in two ways. A combination of $2$ balls of one color and $1$ one ball of other color
  or vice-versa. From other two bags we pick $2$ balls like case I.

  Thus, total no.\ of ways in this case is $C_2^^4\left(C_2^^3\times C_1^^2 + C_1^^3\times
  C_2^^2\right)^2\left(C_1^^3\times C_1^^2\right)^3$

  Final sum would be $21816$.
  %513
\item The $3$ digit numbers are from $100$ to $999$ i.e. $900$ in count.

  Divisible by $2$ are $450$. Divisible by $3$ are $300$. Divisible by $7$ are $128$. Divisible by both $2$
  and $7$ i.e. by $14$ are $64$. Divisible by $3$ and $7$ i.e. $21$ are $43$. Divisible by $2$ and $3$
  i.e. $6$ are $150$. Divisible by $2, 3$ and $7$ i.e. $42$ are $21$.

  Thus, required no.\ is $450 + 300 - 150 - 64 - 43 + 21 = 514$.
  %514
\item Since the numbers has to be divisible by $6$ so the sum of digits should be divisible by $3$ and
  unit's place is occupied by an even number for which we have only $4$.

  {\bf Case I:} The number is $444444$ which is unique and count is $1$.

  {\bf Case II:} Number is formed using $4, 4, 4, 4, 5, 9$ i.e. $\frac{5!}{3!} = 20$ because unit's place is
  fixed for $4$.

  {\bf Case III:} Number is formed using $4, 4, 5, 5, 5$ i.e $\frac{5!}{2!3!} = 10$

  {\bf Case IV:} Number is formed using $4, 4, 9, 9, 9$ i.e $\frac{5!}{2!3!} = 10$

  {\bf Case V:} Number is formed using $4, 5, 5, 9, 9$ i.e $\frac{5!}{2!2!} = 30$

  {\bf Case VI:} Number is formed using $5, 9, 9, 9, 9$ i.e $\frac{5!}{4!} = 5$

  {\bf Case VII:} Number is formed using $5, 5, 5, 5, 9$ i.e $\frac{5!}{4!} = 5$

  Thus, required answer is $81$.
  %515
\item Required no.\ of ways $=$ Total no.\ of ways $-$ (one child receives no orange $+$ two children
  receive no orange)

  $= 3^{20} - \left(C_1^^3(2^{20} - 2) + C_2^^3.1^{20}\right) = 3483638676$.
  %516
\item No.\ of words beginning with $B, C, I, L$ is $4\times5! = 480$

  No. of words beginning with $PB, PC, PI, PL$ is $4\times 4! = 96$

  No. of words begnning with $PUBC, PUBI$ is $2\times2! = 4$

  Then, we will have $PUBLC$ followed by $PIBLIC$ i.e. $2$

  Thus, rank of the word \quote{PUBLIC} would be $582$.
  %517
\item $7$ boys can be seated in $6!$ ways around a round table and there will be $7$ seats between them. We
  can pick $5$ seats for the girls in $C_5^^7$ ways and arrange the girls in $5!$ ways.

  Thus, total no.\ of possible arrangements are $6!\times C_5^^7\times 5! = 126\times(5!)^2$.
  %518
\item {\it Legendre's Formula}: The exponent of prime $p$ in $n!$ is given by the floor of $\frac{n}{p^k}$
  for all positive integers $k$

  $n = \displaystyle\sum_{k = 1}^{\infty}\left\lfloor\frac{66}{3^k}\right\rfloor$

  $\Rightarrow n = \left\lfloor\frac{66}{3}\right\rfloor + \left\lfloor\frac{66}{3^2}\right\rfloor
  + \left\lfloor\frac{66}{3^3}\right\rfloor + 0 + 0 + 0 + \cdots \infty$

  $= 22 + 7 + 2 = 31$.
  %519
\item Required number $=$ Total $-$ Words in which \quote{C} and \quote{S} come together

  $= \frac{11!}{2!2!21} - \frac{10!}{2!2!2!}2! = 5670.6!$.
  %520
\item Number of ways to choose $2$ men from $n$ couples is $C_2^^n$ and number of ways of choosing $2$ women
  from remaining $n - 2$ couples is $C_2^^{n - 2}$.

  Now with $2$ men and $2$ women we can form two distinct groups of men and women.

  Thus, $C_2^^n\times C_2^^{n - 2}\times 2 = 840\Rightarrow n = 16$.
  %521
\item When $1$ is at unit's place number of numbers is $\frac{3!}{2!} = 3$, when $2$ is at unit's place
  number of numbers is $3! = 6$ and when $3$ is at unit's place number of numbers is $3$ again.

  Thus, sum of digits at unit's place is $24$. The sum will remain same at all places as there is no zero or
  any restriction on placement.

  Thus, sum of numbers is $24000 + 2400 + 240 + 24 = 26664$.
  %522
\item This is a straight-forward problem on derangements and answer is given by

  $5!\left(1 - \frac{1}{1!} + \frac{1}{2!} - \frac{1}{3!} + \frac{1}{4!} - \frac{1}{5!}\right) = 44$.
  %523
\item When the first alphabet is $A, H, M, S$ no.\ of words is $4!$ thus a total of $4\times4!$.

  If $TA$ is the first two characters then no.\ of words is $3!$. The next word would be $THAMS$. So the
  position is $96 + 6 + 1 = 103$.
  %524
\item The possible arrangements are $5xxx0, 7xxx0, 7xxx5, 9xxx0, 9xxx5$.

  Thus required no.\ is $5\times P_3^^4 = 120$.
  %525
\item Let the seven digits be $x_1, x_2, x_3, x_4, x_5, x_6, x_7$ then according to question $x_1 + x_2 +
  x_3 + x_4 + x_5 + x_6 + x_7 = 12$

  Now all these have minimum value of $1$. Let $x_i = y_i + 1$ so that $y_i \geq 0$, then the equation
  becomes $\displaystyle\sum_{i = 1}^7 y_i = 5$.

  The generating function for one $y_i$ is $1 + z + z^2 + z^3$ as it can have the value of $0, 1, 2, 3$
  i.e. $\frac{1 - z^4}{1 - z}$

  Coefficient of $z^5$ in $\left(\frac{1 - z^4}{1 - z}\right)^7 = C_6^^{11} - \frac{7!}{6!} - \frac{7!}{5!}
  = 413$.
  %526
\item Words beginning with $A$ or $D$ are $2\times 5! = 240$

  Words beginning with $MA, MD, MN$ are $3\times 4! = 72$

  Words beginning with $MOA, MOD$ are $2\times 3! = 12$

  Words beginning with $MONA$ are $2!= 2$ and next words is $MONDAY$.

  So position is $240 + 72 + 12 + 2 + 1 = 327$.
  %527
\item Being divisible by $6$ means that sum of digits must be divisible by $3$ and the number should be
  even. This implies that we can have either $2$ or $4$ at unit's place.

  If $2$ is at unit's place we can fill remaining two positions with $(1,3), (3,1), (2,2), (2,5), (5,2),
  (3,4), (4,3), (5,5)$ i.e. we can form $8$ numbers.

  If $4$ is at unit's place we can fillremaining two positions with $(1,1), (1,4), (4,1), (2,3), (3,2)
  (4,4), (3,5), (5,3)$ i.e. $8$ numbers.

  Thus, total no.\ of numbers is $16$.
  %528
\item Since the numbers have to be divisible by $3$ the sum of digits must be divisible by $3$.

  The possible combinations are $(1,1,1), (3,3,3), (5,5,5), (8,8,8), (5,5,8), (8,8,5), (1,3,5), (1,3,8)$.

  Thus, total no.\ of permutations are $1 + 1 + 1 + 1 + \frac{3!}{2!} + \frac{3!}{2!} + 3! + 3! = 22$.
  %529
\item It is given that sum of first two digits is equal to sum of last two digits.

  {\bf Case I:} When the sum is $7$. Let first two digits be $7, 0$ or $0, 7$. Then last two digits can be
  $(1, 6), (2, 5), (3, 4), (4, 3), (5, 2), (6, 1)$.

  Thus, total possible codes are $2\times 12 = 24$.

  {\bf Case II:} When the sum is $8$. Let the first two digits be $7, 1$ or $1, 7$. Then the last two digits
  can be $(2, 6), (3, 5), (5, 3), (6, 2)$.

  Thus, total possible codes are $2\times 8 = 16$.

  Proceeding similarly for sum $9$, we have $16$ possible codes, for sum $10$, we have $8$ possible codes,
  and for sum $11$ we have $8$ possible codes.

  Thus, maximum no.\ of trials for correct code is $72$.
  %530
\item Since the number has to be greater than $7000$ so it has to be at least of $4$-digits. For such
  $4$-digit numbers thousand's place can be filled by eithe $7$ or $8$. Remaining three positions can be
  filled in $4\times3\times2$ ways.

  Thus, we have a total of $48$ four-digit numbers.

  All five-digit numbers would be greater than $7000$. No.\ of five-digit numbers is $5! = 120$.

  Thus, required no.\ of numbers is $168$.
  %531
\item There are $4$ even digits, a pair of $2$'s and a pair of $4$'s. There are $4$ even places. Thus,
  no.\ of possible arrangements is $\frac{4!}{2!2!}$.

  There are five odd places to be filled with $3$ ones and $2$ threes. Thus, no.\ of possible arrangement of
  odd digits is $\frac{5!}{3!21}$.

  Thus, required no. of numbers is $\frac{4!}{2!2!}\times\frac{5!}{3!2!} = 60$.
  %532
\item He can chhose $2$ language courses from $5$ language course along with $3$ courses from $7$ other
  courses or $1$ language course from $5$ language course along with $4$ courses from $7$ other courses or
  all $5$ courses from $7$ other courses.

  Thus, required number of ways of taking course is $C_2^^5\times C_3^^7 + C_1^^5\times C_4^^7 + C_4^^7 =
  546$.
  %533
\item Elements of type $3k = 1$, elements of type $3k + 1 = 1, 7, 10$ and elements of type $3k + 2 = 2, 5,
  11$.

  Subset containing one element is $1$. Subsets containing two elements is $C_1^^3\times C_1^^3 =
  9$. Subsets containing three elements is $C_1^^3 + C_1^^3 + 1 + 1 = 11$. Subsets containing four elements
  is $C_3^^3 + C_3^^3 + C_2^^3\times C_2^^3 = 11$. Subsets containing five elements is $C_2^^3\times C_2^^3
  = 9$. Subsets containing six elements is $1$. Subsets containing seven elements is $1$.

  Thus, a total of $43$ subsets are possible.
  %534
\item Possible selections are $(2O, 1RA, 2WA), (2O, 2RA, 1WA), (3O, 1RA, 1WA)$.

  Thus, no.\ of ways is $C_2^^8\times C_1^^7\times C_2^^5 + C_2^^8\times C_2^^7\times C_1^^5 + C_3^^8\times
  C_1^^7\times C_1^^5 = 6860$.
  %535
\item If first two digits are $12$ then remaining $4$ positions can be filled in $C_4^^7 = 35$ ways.

  If first two digits are $13$ then remaining $4$ positions can be filled in $C_4^^6 = 15$ ways.

  If first two diigts are $14$ then remaining $4$ positions can be filled in $C_4^^5 = 5$ ways.

  If first two diigts are $15$ then remaining $4$ positions can be filled in $C_4^^4 = 1$ way.

  If first two diigts are $23$ then remaining $4$ positions can be filled in $C_4^^6 = 15$ ways.

  So far we have $71$ numbers so $72$nd number would be $245678$, whose sum of digits is $32$.
  %536
\item Total $3$-digit numbers is $900$. Numbers divisible by $3$ is $300$. Numbers divisible by $4$ is
  $225$. Numbers divisible by both $3$ and $4$ i.e. $12$ is $75$.

  Therefore, numbers divisible by either $3$ or $4$ is $300 + 225 - 75 = 450$.

  We remove numbers divisible by $48$, which are $144, 192, \ldots $ i.e. $18$ in number.

  Thus, required answer is $432$.
  %537
\item $54 = 2.3^3$ so the numbers should be divisible by $2$ but not by $3$.

  Required number $=$ Numbers divisible by $2 -$ Numbers not divisible by $6 = \frac{9000}{2}
  - \frac{9000}{6} = 3000$.
  %538
\item For the number to be divisible by $15$ it should be divisible by both $3$ and $5$ i.e. sum of digits
  should be divisible by $3$ and unit's place must be $5$.

  Possible combinations are $1215, 2235, 3315, 1155, 2355, 3555$

  No.\ of arrangements are $\frac{3!}{2!}, \frac{3!}{2!}, \frac{3!}{2}, \frac{3!}{2!}, 3!, \frac{3!}{2!}$
  respectively.

  Thus, total numbers are $21$.
  %539
\item Clearly, $a + b\in\left\{25, 71, 117, 163\right\}$

  If $a + b = 25$, no.\ of ordered pairs is $12$. If $a + b = 71$, no.\ of ordered pairs is $35$. If $a + b
  = 117$ no.\ of ordered pairs is $42$. If $a + b = 163$, no.\ of ordered pairs is $19$.

  Thus, we have a total of $108$ pairs.
  %540
\item If unit's digit is $5$, then no.\ of numbers is $\frac{6!}{3!2!}$

  If unit's digit is $3$, then no.\ of numbers is $\frac{6!}{3!}$

  If unit's digit is $1$, then no.\ of numbers is $\frac{6!}{3!2!}$

  Therefore, required no.\ of numbers is $60 + 60 + 120 = 240$.
  %541
\item If $2$ or $3$ occupies most significant place then we can fill remaining $4$ places in $6$ ways
  each. Thus, no.\ of numbers is $2\times 6^4 = 2592$.

  If $40$ occupies first two places, then remaining places can be filled in $6^3 = 216$ ways.

  If first three places are filled by $420, 422, 423, 424, 427$ then remaining places can be filled in $6^2$
  ways. Thus, no.\ of numbers in this case is $5\times36 = 180$.

  If $4290$ takes first four places then unit's place can be filled in $6$ ways.

  Then we have $42920, 42922, 42923$.

  Thus, position of $42923$ is $2997$.
  %542
\item First $4$ digit number divisible by $3$ is $1002$ and last is $2799$ in the given range. Thus, $1002 +
  (n - 1)3 = 2799 \Rightarrow n = 600$ i.e. we have $600$ numbers divisible by $3$.

  Similarly, we have $166$ numbers divisible by $11$ and $54$ numbers divisible by $33$.

  Thus, required no.\ is $600 + 164 - 33 = 710$.
  %543
\item Given that $a_{11} + a_{12} + a_{21} + a_{22} \in(3, 5, 7, 11)$.

  If sum is $3$ the solution is given by coefficient of $x^3$ in $\left(1 + x + x^2 + x^3 + x^4\right)^4
  = \left(\frac{1 - x^5}{1 - x}\right)^4$

  $= C_3^^{4 + 3 - 1} = 20$

  We proceed similarly for sum to be $5, 7, 11$ to get final answer as $568$.
  %544
\item We can partition $5$ into $4$ offices in following ways:

  $5, 0, 0, 0$, which gives us $1$ way because all five can go to any of the four offices but offices are
  indistinguiishable.

  $4, 1, 0, 0$, which gives us $\frac{5!}{4!} = 5$ ways.

  $3, 2, 0, 0$, which gives us $\frac{5!}{3!2!} = 10$ ways.

  $2, 2, 0, 1$, which gives us $\frac{5!}{2!2!2!} = 15$ ways.

  $2, 1, 1, 1$, which gives us $\frac{5!}{2!3!} = 10$ ways.

  $3, 1, 1, 0$, which gives us $\frac{5!}{3!2!} = 10$ ways.

  Thus, we have a total of $51$ ways.
  %545
\item {\bf Case I:} When $4$ men are selected from Group A and $4$ women are selected from Group B. No.\ of
  ways is $C_4^^4\times C_4^^4 = 1$

  {\bf Case II:} When $3$ men and $1$ woman are selected from Group A and $1$ man and $3$ woman are selected
  from Group B. No.\ of ways is $C_3^^4\times C_1^^5\times C_1^^6\times C_3^^4 = 400$

  {\bf Case III:} When $2$ men and $2$ women are selected from Group A and $2$ men and $2$ women are
  selected from Group B. No.\ of ways is $C_2^^4\times C_2^^5\times C_2^^5\times C_2^^4 = 3600$

  {\bf Case IV:} When $1$ man and $3$ women are selected from Group A and $3$ men and $1$ woman are selected
  from Group B. No.\ of ways is $C_1^^4\times C_3^^5\times C_3^^5\times C_1^^4 = 1600$

  {\bf Case V:} When $4$ women are selected from Group A and $4$ men are selected from Group B. No.\ of ways
  is $C_4^^5\times C_4^^5 = 25$.

  Thus, we have a total of $5626$ ways.
  %546
\item When the words begins with \quote{B}, no.\ of words is $4! = 24$. When the words begin with \quote{H},
  no.\ of words is $\frac{4!}{2!} = 12$. When the words begin with \quote{J}, no.\ of words is
  $\frac{4!}{2!} = 12$.

  Then we have \quote{OBBHJ}, which is $49$-th word and $50$-th word is \quote{OBBJH}.
  %547
\item Let $n(m)$ represent numbers multiple of $m$. Then $n(3)$ will have $200$ elements and $n(4)$ will
  have $151$. $n(3\cap 4) = n(12) = 50$.

  $\Rightarrow n(3\cup 4) = n(3) + n(4) - n(12) = 301$

  $n\left(\overline{3\cup 4}\right) = A - n(3\cup 4) = 300$.
  %548
\item Total possible triangles is $C_3^^8 = 56$. Triangles whose two sides are common with the
  octagon. These triangles are formed by three consecutive vertices and no.\ of such triangles is $8$.

  Triangles with one side common with the octagon. We pick any side. Then we pick a third vertex which is
  not adjacent to any of the vertices to the sides. Thus, for any side we have $4$ non-adjacent vertices. So
  no.\ of triangles is $8\times 4 = 32$.

  Thus, required answer is $56 - 8 - 32 = 16$.
  %549
\item Words beginning wiht $A$ and $G$ are $2\times5! = 240$ in number. Words begnning with $NA, NG, NP$ are
  $3\times4! = 72$ in number. This gives us $312$ numbers.

  After thna we have \quote{NRAGPU}, \quote{NRAGUP} and \quote{NRAPGU}. The last of these is at $315$-th
  position.
  %550
\item Total no.\ of three digit numbers, which are not divisible by $3$, will be formed using the following
  combination of digits:

  $(4, 5, 7), (3, 4, 7), (2, 5, 7), (2, 4, 7), (2, 4, 5), (2, 3, 5)$

  Thus, number of numbers is $6\times3! = 36$.
  %551
\item The word has \quote{AA, MM, TT, H, I, C, S, E}.

  {\bf Case I:} When all chosen alphabets are distinct. No.\ of ways is $C_5^^8 = 56$

  {\bf Case II:} $2$ same and $3$ different, No.\ of ways is $C_1^^3\times C_3^^7 = 105$

  {\bf Case III:} Two pairs of same characters and $5$-th alphabet different. No.\ of ways is $C_2^^3\times
  C_1^^6 = 18$

  Thus, we have a total of $179$ possible combinations.
\stopitemize
