% -*- mode: context; -*-
\chapter{Miscellaneous Problems}
\startitemize[n, 1*broad]
\item Let the numbers be $a$ and $b$ and harmonic means are $h_1$ and $h_1$. Let $d$ be the c.d.\ of the
  corresponding A.P. Then

  $\frac{1}{b} = \frac{1}{a} + 3d\Rightarrow d = \frac{a - b}{3ab}$

  $\frac{1}{h_1} = \frac{1}{a} + \frac{a - b}{3ab} = \frac{a + 2b}{3ab}$.

  Given that $\frac{3ab}{a + 2b} = \sqrt{ab}\Rightarrow 9a^2b^2 = (a + 2b)^2ab \Rightarrow 9ab = a^2 + 4b^2
  + 4ab$

  $\Rightarrow a^2 - 5ab + 4b^2 = 0 \Rightarrow (a - b)(a - 4b) = 0$.

  So either the numbers are equal which will make H.M.'s also equal to the numbers or one of the number is
  four times the other number.
\item Let the numbers be $a$ and $b$ and harmonic mean be $h$. Then

  $\frac{1}{h} = \frac{\frac{1}{a} + \frac{1}{b}}{2} \Rightarrow h = \frac{2ab}{a + b} = 4 \Rightarrow 2a +
  2b - ab = 0$

  Given that $2A + G^2 = 27 \Rightarrow a + b + ab = 27 \Rightarrow a + b = 9$.

  Solving we get $a, b = 6, 3$.
\item Since $n$ is odd, the no.\ of positive term of the given series is $\frac{n + 1}{2}$ and no.\ of
  negative terms of negative terms of the series is $\frac{n - 1}{2}$.

  Sum of positive terms is $\frac{n + 1}{4}\left[2a + \frac{n - 1}{2}d\right]$ and sum of negative terms is
  $\frac{n - 1}{4}\left[2(a + d) + \frac{n - 3}{2}d\right]$.

  Adding we get the required sum.

  Alternatively we observe that first $\frac{n- 1}{2}$ pairs will have a sum of $-\frac{(n - 1)d}{2}$ and
  $n$th term is $a + (n- 1)d$. So the sum is $a + \frac{n - 1}{2}d$.
\item $x = 2n\pi\pm \cos^{-1}y$. So we see that it is a periodic function which has a period of $2\pi$. If
  we move only in one(positive/negative) direction then all values of $y$ will give an A.P., however, if we
  want to move in both directions then $x$ must be $0, \pi$ making $y = \pm1$.
\item Let $a$ be the first term and $r$ be the common ratio of the G.P. Sum of all terms is given by
  $\frac{a}{1 - r}$ and sum of odd terms is given by $\frac{a}{1 - r^2}$.

  Given that $\frac{a}{1 - r} = \frac{5a}{1 - r^2}\Rightarrow 1 + r = 5\Rightarrow r = 4$.
\item Let $S_p$ represent sum of $p$ terms of the given A.P., then $t_p = S_p - S_{p - 1} = a + (2p - 1)b$.

  $\Rightarrow t_4 = a + 7b$.
\item We see that c.d.\ is $5$. Either we can apply the $t_n = 3 + (n - 1)5$ and solve for each or on simple
  observation we find that last digit should be $3$ and hence $303$ is the number, which will be a term in
  the given sequence.
\item Let $a$ be the first term and $d$ be the common ratio of the A.P., then $A = a, B = a + d, c = a +
  2d$.

  Substituting the points in the given equation we find that the point $(1, -2)$ satisfies the
  equation. Hence, the given straight line will pass through $(1, -2)$.
\item Let $S_n$ represent the sum of $n$ terms of the A.P., then $S_n = \frac{3n^2 + 5n}{36}$.

  $t_n = S_n - S_{n - 1} = \frac{6n - 3 + 5}{36} = \frac{3n + 1}{18}$.

  c.d. $= t_n - t_{n - 1} = \frac{1}{6}$.
\item Given, $\frac{a - b}{b - c} = \frac{a}{c} \Rightarrow b = \frac{2ac}{a + c}$, so the given sequence is
  a harmonic sequence.
\item Common ratio of the given series is $\frac{2x}{x + 3}$. For the sum to be definite c.r should be less
  that $|1|$. Therefore,

  $-1 < \frac{2x}{x + 3} < 1 \Rightarrow -1 < x < 3$.
\item Let the numbers be $x_1, x_2, \ldots, x_n$, then using A.M.-G.M. relation we have

  $\frac{x_1 + x_2 + \cdots + x_n}{n}\geq \sqrt[n]{x_1.x_2.\ldots x_n} \Rightarrow x_1 + x_2 + \cdots +
  x_n\geq n$.
\item Let $a$ be the first term and $d$ be the c.d. of the given A.P. Given that $S_{2n} = 3S_n$ so we have

  $n[2a + (2n - 1)d] = \frac{3n}{2}[2a + (n - 1)d] \Rightarrow 4a + (4n - 2)d = 6a + 3(n - 1)d$

  $\Rightarrow 2a = (n + 1)d$

  $\frac{S_{3n}}{S_n} = \frac{3[2a + (3n - 1)d]}{2a + (n - 1)d} = \frac{12nd}{2nd} = 6$.
\item Each term consistes of two integers which are in two different A.P. The $n$th term is given by $t_n =
  [1 + (n - 1)2].[3 + (n - 1)2] = (2n - 1)(2n + 1) = 4n^2 - 1$

  Thus, sum $= S_n = \displaystyle\sum_{i = 1}^n(4i^2 - 1) = 4.\frac{n(n + 1)(2n + 1)}{6} - 1$.
\item Let $S = 1 + \frac{4}{5} + \frac{7}{5^2} + \frac{10}{5^3} + \cdots$. We observe that numerator is in
  A.P. with first term as $1$ and c.d.\ as $3$ while denominator is in G.P.\ with first terms as $1$ and
  c.r.\ as $\frac{1}{5}$.

  So using the technique for AGP we have $S - \frac{S}{5} = 1 + \frac{3}{5} + \frac{3}{5^2} + \cdots$

  $\Rightarrow \frac{4S}{5} = 1 + \frac{3}{5}.\frac{1}{1 - \frac{1}{5}} = 1 + \frac{3}{4}
  = \frac{7}{4} \Rightarrow S = \frac{35}{16}$.
\item From the given information we can write $2b = a + c; x^2 = ab$ and $y^2 = bc$

  $\Rightarrow x^2 + y^2 = b(a + c) = 2b^2$ and thus, $x^2, b^2, y^2$ are in A.P.
\item Given, $2\log_{10}(2^x - 1) = \log_{10}2 + \log_{10}(2^x + 3)$

  $\Rightarrow (2^x - 1)^2 = 2.2^x + 6 \Rightarrow 2^{2x} -4.2^x - 5 = 0 \Rightarrow 2^x = 5, -1$, however,
  $2^x$ cannot be $-1$.

  $\therefore 2^x = 5 \Rightarrow x = \log_25$.
\item Since $p, q, r$ are in A.P.\ so we can write $q - p = r - q$. Now

  $\frac{m^{7q}}{m^{7p}} = m^{7(q - p)} = m^{7(r - q)} = \frac{m^{7r}}{m^{7q}}$, and thus, $m^{7p}, m^{7q},
  m^{7r}$ are in G.P.
\item $x = \frac{1}{1 - \cos^2\phi} = \csc^2\phi; y = \frac{1}{1 - \sin^2\phi} = \sec^2\phi$ and $z
  = \frac{1}{1 - \cos^2\phi\sin^2\phi}$

  $xyz = \frac{1}{\sin^2\phi\cos^2\phi(1 - \cos^2\phi\sin^2\phi)}$

  $xy + z = \frac{1}{\sin^2\phi\cos^2\phi} + \frac{1}{1 - \cos^2\phi\sin^2\phi}
  = \frac{1}{\sin^2\phi\cos^2\phi(1 - \cos^2\phi\sin^2\phi)}$.

  Thus, $xyz = xy + z$.
\item Rewriting we have given series as $\frac{1}{\sqrt{2}}(2 + 4 + 6 + 8 + \cdots) = \frac{n(n +
  1)}{\sqrt{2}}$.
\item Since $a, b, c$ are in A.P., we can write $a - b + c = b \Rightarrow a - b + c = \frac{b - a + c + a +
  b - c}{2}$

  Thus, $\frac{b -a + c}{2}, \frac{a - b + c}{2}, \frac{a + b - c}{2}$ are in A.P.

  $\Rightarrow s - a, s - b, s - c$ are in A.P. $\Rightarrow \frac{s(s - a)}{\Delta}, \frac{s(s -
    b)}{\Delta}, \frac{s(s - c)}{\Delta}$ are in A.P.

  $\Rightarrow \frac{\Delta}{s(s - a)}, \frac{\Delta}{s(s - b)}, \frac{\Delta}{s(s - c)}$ are in H.P.

  $\Rightarrow \tan\frac{A}{2}, \tan\frac{B}{2}, \tan\frac{C}{2}$ are in H.P.
\item The ex-radii are given by $\frac{\Delta}{s - a}, \frac{\Delta}{s - b}$ and $\frac{\Delta}{s - c}$,
  which are in A.P.

  So $\frac{s - a}{\Delta}, \frac{s - b}{\Delta}, \frac{s - c}{\Delta}$ are in A.P.

  We know that if we multiply an A.P.\ by a constant then resulting series is in A.P. Multiplying given
  terms by $\Delta$, we have

  $s - a, s - b, s - c$ are in A.P. We also know that if we subtract a constant from A.P. then resulting
  series remains in A.P. So subtracting given terms from $s$ we have desired result.
\item Let $x$ be the first term and $y$ be the $2n - 1$st term. Then $n$th term of A.P.\ will be A.M.\ of
  $x$ and $y$, which is $a = \frac{x + y}{2}$; $n$th term of the G.P.\ will be G.M.\ of $x$ and $y$, which
  is $b = \sqrt{xy}$; and $n$th term of H.P.\ will be H.M.\ of $x$ and $y$ which is $c = \frac{2xy}{x + y}$.

  Clearly, $ac - b^2 = 0$.
\item Let $a, b, c$ are the roots; then from Vieta's relations we have $a + b + c = 12; ab + bc + ca = 39,
  abc = 28$.

  From first relation $3b = 12 \Rightarrow b = 4$ so the two other relations become $4a + 4c + ca = 39$ and
  $4ca = 28 \Rightarrow \ca = 7$

  $\Rightarrow a + c = 8 \Rightarrow a + \frac{7}{a} = 8 \Rightarrow a^2 - 8a + 7 = 0 \Rightarrow a = 1, 7$
  and $c = 7, 1$ so common difference is $\pm 3$.
\item Given, $(a^2 + b^2 + c^2)p^2 - 2(ab + bc + cd)p + (b^2 + c^2 + d^2)\leq 0$

  $\Rightarrow (a^2p^2 - 2abp + b^2) + (b^2p^2 - 2bcp + c^2) + (c^2p^2 - 2cdp + d^2)\leq 0$

  $\Rightarrow (ap - b)^2 + (bp - c)^2 + (cp - d)^2\leq 0$

  Clearly, only equality is possible as square of numbers cannot be negative. Hence, $p = \frac{b}{a}
  = \frac{c}{b} = \frac{d}{c}$, making $a, b, c, d$ form a G.P.
\item Let $d$ be the c.d.\ of A.P., then numbers are $6, 6 - d, 6 -2d, 3 - d$ such that $\frac{6 - 2d}{6 -
  d} = \frac{1}{2} \Rightarrow 12 - 4d = 6 - d \Rightarrow d = 2$.

  Thus, numbers are $6, 4, 2, 1$.
\item Given $\frac{44}{9} = 3 + 5r + 7r^2 + 9r^3 + \cdots$ to $\infty \Rightarrow \frac{44}{9}r = 3r + 5r^2
  + 7r^3 + \cdots$ to $\infty$

  Subtracting terms with corresponding powers of $r$, we get

  $\frac{44}{9}(1 - r) = 3 + 2r + 2r^2 + 2r^3 + \cdots $ to $\infty = 3 + \frac{2r}{1 - r} = \frac{3 - r}{1
    - r}$

  $\Rightarrow 44(1 - r)^2 = 27 - 9r \Rightarrow 44r^2 - 79r + 17 = 0 \Rightarrow (4r - 1)(11r - 17) = 0$

  For the given series $r$ has to be less than $|1|$ so $r = \frac{1}{4}$.
\item Let $S = 1^2 + 2^2x + 3^2x^2 + 4^2x^3 + \cdots$ to $\infty$. $\Rightarrow xS = 1^2x + 2^2x^2 + 3^2x^3
  + \cdots$ to $\infty$.

  Subtracting terms with corresponsding powers of $x$, we get

  $S(1 - x) = 1 + 3x + 5x^2 + 7x^3 + \cdots$ to $\infty$. $\Rightarrow Sx(1 - x) = x + 3x^2 + 5x^3 + \cdots$
  to $\infty$.

  Subtracting terms with corresponsding powers of $x$, we get

  $S(1 - x)^2 = 1 + 2x + 2x^2 + 2x^3 + \cdots$ to $\infty = 1 + \frac{2x}{1 - x} = \frac{1 + x}{1 - x}$

  $\Rightarrow S = \frac{1 + x}{(1 - x)^3}$.
\item $n$th term is given by $t_n = 1^2 + 2^2 + \cdots + n^2 = \frac{n(n + 1)(2n + 1)}{6} = \frac{2n^3 +
  3n^2 + n}{6}$

  Thus, sum is $S = \displaystyle\sum_{n = 0}^k\frac{2n^3 + 3n^2 + n}{6} = \frac{1}{3}\left(\frac{k(k +
    1)}{2}\right)^2 + \frac{1}{2}k(k + 1)(2k + 1) + \frac{k(k + 1)}{12}$

  Putting $k = 22$, we get sum as $S = 23276$.
\item Let $a$ be the first term and $d$ be the c.d. Then given $S_m:S_n = m^2:n^2$

  $\Rightarrow \frac{\frac{m}{2}[2a + (m - 1)d]}{\frac{n}{2}[2a + (n - 1)d]} = \frac{m^2}{n^2}$

  $\Rightarrow 2an + (m - 1)nd = 2am + (n - 1)md \Rightarrow 2a(m - n) = (m - n)d \Rightarrow a
  = \frac{d}{2}$

  $\Rightarrow \frac{t_m}{t_n} = \frac{a + (m - 1)d}{a + (n - 1)d} = \frac{2m - 1}{2n - 1}$.
\item Let $a$ be the first term and $d$ be the c.d.\ of the first A.P.\ and $x$ be the first term and $y$ be
  the c.d.\ of the second A.P. Also, let $S_n$ be the sum of $n$ terms of first A.P.\ and $t_{11}$ be the
  $11$th term of the first A.P., while $S_n'$ be the sum of the $n$ terms of the second A.P.\ and $t_{11}'$
  be the $11$th term of the second A.P.

  $S_n = (7n + 1)k = 2a + (n - 1)d \Rightarrow d = 7k, a = 4k$. $S_n' = (4n + 27)k = 2x + (n -
  1)y\Rightarrow y = 4k, x = \frac{31}{2}k$

  $\Rightarrow \frac{t_{11}}{t_{11}'} = \frac{a + (11 - 1)d}{x + (11 - 1)y} = \frac{4k + 70k}{\frac{31}{2}k +
    40k} = \frac{148k}{111k} = \frac{4}{3}$.
\item Let the distance between the stations be $x$ km. Total time taken $= \frac{x}{40} + \frac{x}{60}
  = \frac{x}{24}$ hours.

  $\therefore$ average speed $= \frac{2x}{\frac{x}{24}} = 48$ kmph.
\item Because $a, b, c$ are in A.P., therefore $2b = a + c$. Also, because $a^2, b^2, c^2$ are in H.P.,
  therefore $b^2 = \frac{2a^2c^2}{a^2 + c^2} \Rightarrow \frac{(a + c)^2}{4} = \frac{2a^2c^2}{a^2 + c^2}$

  $\Rightarrow a^4 + 2a^3c + 2ac^3 + 2a^2c^2 + c^4 = 8a^2c^2$, which gives $a = c$ upon solving making $a =
  b = c$.
\item $t_n = \frac{1}{1 + 2 + 3 + \cdots + n} = \frac{2}{n(n + 1)} = 2\left[\frac{1}{n} - \frac{1}{n +
    1}\right]$

  Thus, $S = 2.1 = 2$.
\item Let the roots be $\alpha$ and $\beta$. Then from question

  $\alpha + \beta = \frac{1}{\alpha^2} + \frac{1}{\beta^2} = \frac{(\alpha + \beta)^2 -
  2\alpha\beta}{\alpha^2\beta^2}$

  $\Rightarrow -\frac{b}{a} = \frac{\frac{b^2}{a^2} - 2\frac{c}{a}}{\frac{c^2}{a^2}} = \frac{b^2 -
  2ca}{c^2}$

  $\Rightarrow 2ca^2 - ab^2 = bc^2\Rightarrow ab^2, ca^2, bc^2$ are in A.P.
\item Let $a$ be the first term and $d$ be the c.d.\ of the A.P., then according to question

  $pt_p = qt_q$, where $t_p$ is the $p$th term and $t_Q$ be the $q$th term of the given A.P.

  $\Rightarrow ap + p(p - 1)d = aq + q(q - 1)d \Rightarrow a(p - q) = (q^2 - p^2)d + (p - q)d$

  $\Rightarrow a = d +(p + q)d = d(1 - p - q)$

  $\therefore t_{p + q} = a + (p + q - 1)d = 0$.
\item Let $r$ be the c.r.\ then $r = \frac{b}{a}$, and $c = ar^{n - 1}$, where $n$ is the number of terms.

  Sum of the series is $S = \frac{a(r^n - 1)}{r - 1} = \frac{c\frac{b}{a} - a}{\frac{b}{a} - 1} = \frac{bc -
    a^2}{b - a}$.
\item Let the triangle be $\triangle ABC$ such that $A$ is largest angle and $B$ is second largest. Given
  that angles are in A.P., $\Rightarrow a -d + a + a + d = 180^\circ \Rightarrow a = 60^\circ$ which would
  be $\angle B$. Let the unknown side be $x$, which is in front of $\angle C$.


  $\cos B = \frac{x^2 + 10^2 - 9^2}{20x} = \frac{1}{2}\Rightarrow x = 5\pm\sqrt{6}$.
\item Let the quantities be $a$ and $b$, then $G = \sqrt{ab}$. Let $d$ be the common difference then $d
  = \frac{b - a}{3}\Rightarrow p = \frac{2a + b}{3}, q = \frac{a + 2b}{3}$.

  $2p - q = a, 2q - p = b \Rightarrow G^2 = (2p - q)(2q - p)$.
\item Let $d$ be the common difference, then $b - a = c - b = a = d$. Thus, $b = 2d, c = 3d$. Hence, $a:b:c
  = 1:2:3$.
\item Let $d$ be the c.d.\ and $n$ be the number of terms then $l = a + (n - 1)d \Rightarrow n = \frac{l -
  a}{d} + 1$.

  $s = \frac{n}{2}[a + l] \Rightarrow 2s = \left(\frac{l - a}{d} + 1\right)(a + l)$

  $\Rightarrow \frac{2s}{a + l} = \frac{l - a}{d} + 1 \Rightarrow \frac{2s - a - l}{a + l} = \frac{l -
  a}{d} \Rightarrow d = \frac{l^2 - a^2}{2s - (l + a)}$.
\item Let $a$ be the first term and $d$ be the c.d.\ of the A.P., then

  $\frac{1}{n} = a + (m - 1)d$ and $\frac{1}{m} = a + (n - 1)d$

  Subtracting we have $\frac{m - n}{mn} = (m - n)d \Rightarrow d = \frac{1}{mn} \Rightarrow a = \frac{1}{n}
  - \frac{m - 1}{mn} = \frac{1}{mn}$.

  $\Rightarrow T_{mn} = \frac{1}{mn} + (mn - 1).\frac{1}{mn} = 1$.
\item Given $\frac{\cos\theta}{\sin\theta} = \frac{\tan\theta}{\cos\theta} \Rightarrow \sin^2\theta
  = \cos^3\theta$

  $\cot^6\theta - \cot^2\theta = \cot^2\theta(\cot^2\theta - 1)(\cot^2\theta + 1)
  = \frac{1}{\cos\theta}\left(\frac{1 - \cos\theta}{\cos\theta}\right)\left(\frac{1
    + \cos\theta}{\cos\theta}\right)$

  $= 1$.
\item Given $a + b + c = k(\sin A + \sin B + \sin C) = 6\frac{\sin A + \sin B + \sin C}{3} \Rightarrow k =
  2$.

  $\Rightarrow \frac{a}{\sin A} = \frac{1}{2} \Rightarrow A = 30^\circ$.
\item Given, $1.1! + 2.2! + 3.3! + \cdots + n.n! = (2 - 1).1! + (3 - 1)2! + (4 - 1)3! + \cdots + (n + 1 -
  1)n!$

  $= 2! - 1! + 3! - 2! + 4! - 3! + \cdots + (n + 1)! - n! = (n + 1)! - 1$.
\item Let $y = x + d$ and $x = a + 2d$. Then, $\frac{1}{\sqrt{y} + \sqrt{x}} = \frac{\sqrt{y}
  - \sqrt{x}}{d}, \frac{1}{\sqrt{z} + \sqrt{x}} = \frac{\sqrt{z} - \sqrt{x}}{2d}, \frac{1}{\sqrt{z}
  + \sqrt{y}} = \frac{\sqrt{z} - \sqrt{y}}{d}$.

  Now $\frac{1}{\sqrt{z} + \sqrt{x}} - \frac{1}{\sqrt{y} + \sqrt{x}} = \frac{\sqrt{z} - \sqrt{x} - 2\sqrt{y}
    + 2\sqrt{x}}{2d}$

  and $\frac{1}{\sqrt{z} + \sqrt{y}} - \frac{1}{\sqrt{z} + \sqrt{x}} = \frac{2\sqrt{z} - 2\sqrt{y}
    - \sqrt{z} + \sqrt{x}}{2d}$

  Hence the required terms are in A.P. as well.
\item $S = 1 + \frac{1.3}{3} + \frac{3.5}{2!.3^2} + \frac{3.5.7}{3!3^3} + \cdots$

  $= 1 + \frac{3}{1!}.\frac{1}{3} + \frac{3(3 + 2)}{2!}.\frac{1}{3^2} + \frac{3(3 + 2)(3 +
  2.2)}{3!}.\frac{1}{3^3} + \cdots$

  $= \left(1 - \frac{2}{3}\right)^{-\frac{3}{2}} = 3\sqrt{3}$.
\item $S = \frac{1}{1 - \cos\alpha} = 2 - \sqrt{2} \Rightarrow 1 - \cos\alpha = \frac{1}{2 - \sqrt{2}}
  = \frac{2 + \sqrt{2}}{4 - 2} = 1 + \frac{1}{\sqrt{2}}\Rightarrow \cos\alpha = -\frac{1}{\sqrt{2}}$

  $\Rightarrow \alpha = \frac{3\pi}{4}$.
\item Let $x_1$ and $x_2$ be roots of the equation. Then H.M. $= \frac{2x_1x_2}{x_1 + x_2} = \frac{2.\frac{8
    + 2\sqrt{5}}{5 + \sqrt{2}}}{\frac{4 + \sqrt{5}}{5 + \sqrt{2}}} = \frac{16 + 4\sqrt{5}}{4 + \sqrt{5}} = 4$.
\item Applying $C_3\rightarrow C_3 - (C_1 + C_2)$, we have

  $\startdeterminant\NC a\NC b\NC 0\NR\NC b\NC c\NC 0\NR\NC a + b\NC b + c\NC -(a + 2b + c)\NR\stopdeterminant$

  Expanding against $3$rd row and $3$rd column, we have

  $(a + 2b + c)[ac - b^2] = 0 \Rightarrow ac = b^2$, and thus, $a, b, c$ are in G.P.
\item We have $225a^2 + 9b^2 + 25c^2 - 75ac - 45ab - 15bc = 0\Rightarrow \frac{1}{2}[(15a - 3b)^2 + (3b -
  5c)^2 + (5c - 15a)^2] = 0$

  $\therefore 15a = 3b = 5c \Rightarrow \frac{a}{1} = \frac{b}{5} = \frac{c}{3} = \lambda$

  $\therefore b, c, a$ are in A.P.
\item $m$th term of the first progression $= 1 + (m - 1)3 = 3m - 2$. $n$th term of the second progression $=
  2 + (n - 1)5 = 5n - 3$. $r$th term of the third progression $= 3 + (r - 1)7 = 7r - 4$.

  Terms between the first and the second progression will be same when $m = \frac{5n - 1}{3}\Rightarrow n =
  2, 5, 11, \ldots$. Similarly, the terms between the second and the third progression will be same when $r
  = \frac{5n + 1}{7}\Rightarrow n = 4, 11, \ldots$.

  So the first common term will be $n = 11\Rightarrow a = 2 + (11 - 1)5 = 52$. Common difference will be LCM
  of common differences which is $105$.
\item Let the sides be $a - d, a, a + d$. Then from area $a(a - d) = 48$ and $(a - d)^2 + a^2 = (a +
  d)^2\Rightarrow a = 4d \Rightarrow a = 8, d = 2$.

  Thus, smallest side is $a - d = 6$.
\item We have $[-x] = -[x] - 1$ if $x\not\in\mathbb{I}$ and $[x] + \left[x + \frac{1}{n}\right] + \left[x
  + \frac{2}{n}\right] + \cdots + \left[x + \frac{n - 1}{n}\right] = [nx]$, where $n\in{N}$.

  So given series $= -1\times 100 - \left[\frac{1}{3}\times 100\right] = -133$.
\item Sum of first $n$ terms is given by $\frac{n}{2}[2a + (n - 1)d]$, where each term has the usual
  meaing. Given that this sum is $50n + \frac{n(n - 7)}{2}A$.

  Comparing corresponding terms we have $a = 50 - 3A$ and $d = A$.
\item Numbers which leave a remainder of $2$ are given by $7n + 2$. Smallest two digit number is
  $16$. Largest such number is $93$. No.\ of such numbers is $12$.

  Thus, sum is $\frac{12}{2}[16 + 93] = 654$.

  Similarly, number which leaves a remainder of $5$ are given by $7n + 5$. Smallest two digit number is
  $12$. Largest such number is $96$. No.\ of such numbers is $13$.

  Thus, sum is $\frac{13}{2}[12 + 96] = 702$. Thus, total sum is $1356$.
\item Let $d$ be the c.d., then $S = 15[2a_1 + 29d], T = \frac{15}{2}[2a_1 + 28d] \Rightarrow 2T = 15[2a_1 +
  28d]$.

  $\therefore S - 2T = 15d = 75 \Rightarrow d = 5$. $a_{10} = a_5 + 5d = 27 + 25 = 52$.
\item If $\log_eb_1, \log_eb_2, \ldots, \log_eb_{101}$ are in A.P. with common difference of $\log_e2$, then
  $b_1, b_2, \ldots, b_{101}$ are in G.P., with a common ratio of $2$.

  $\Rightarrow b_2 = 2b_1, b_3 = 2^2b_1, \ldots, b_{101} = 2^{100}b_1$.

  Also given that $a_1, a_2, \ldots, a_{101}$ are in A.P. Given $a_1 = b_1$ and $a_{51} = b_{51}\Rightarrow
  a_1 + 50d = 2^{50}b_1 = 2^{50}a_1$.

  $t = b_1 + b_2 + \cdots + b_{51} = b_1(2^{51} - 1) < 2^{51}a_1, s = a_1 + a_2 + \cdots + a_{51}
  = \frac{51}{2}[2a_1 + 50d] = \frac{51}{2}[a_1 + a_1 + 50d] = \frac{51}{2}[a_1 + 2^{20}a_1] > 2^{51}a_1$

  Hence, $s > t$. Now, $a_{101} = a_1 + 100d = a_1 + 100.\left(\frac{2^{50}a_1 - a_1}{50}\right), b_{101} =
  2^{100}b_1 = 2^{100}a_1$

  $\Rightarrow a_{101} = a_1 + 2^{51}a_1 - 2a_1 = 2^{51}a_1 - a_1$, and thus, $a_{101} < b_{101}$.
\item Let $S_n = cn^2 \Rightarrow t_n = S_n - S_{n - 1} = 2cn - c\Rightarrow t_n^2 = 4c^2n^2 + c^2 - 4c^2n$

  Sum of squares of terms is $\sum t_n^2 = 4c^2.\frac{n(n + 1)(2n + 1)}{6} + c^2n - 4c^2.\frac{n(n + 1)}{2}
  = \frac{nc^2(4n^2 - 1)}{3}$.
\item $V_r = \frac{r}{2}[2r + (r - 1)(2r - 1)] = \frac{1}{2}(2r^3 - r^2 + r)$

  $\sum V_r = \frac{1}{2}\left[2\left(\frac{n(n + 1)}{2}\right)^2 - \frac{n(n + 1)(2n + 1)}{6} + \frac{n(n +
    1)}{2}\right] = \frac{1}{12}n(n + 1)(3n^2 + n + 2)$.

  $T_r = V_{r + 1} - V_r = (r + 1)(3r - 1)$, which is a composite number.

  $Q_r = T_{r + 1} - T_r = 6r + 5$, therefore common difference is $6$.
\item From given conditions we write equations for sum of roots and product of roots as $p + q = 2, pq = A,
  r + s = 18, rs = B$.

  Also, given that $p, q, r, s$ form an A.P., so let $p = a - 3d, q = a - d, r = a + d, s = a + 3d$.

  Adding $4a = p + q + r + s = 20 \Rightarrow a = 5 \Rightarrow d = 2$ so $p, q, r, s$ are $-1, 3, 7, 11$
  respectively. Therefore, $A = -3, B = 77$.
\item Given $1^2 + 2.2^2 + 3^2 + 2.4^2 + 5^2 + \cdots$ upto $n$ terms $= \frac{n(n + 1)^2}{2}$, when $n$ is
  even.

  When $n$ is odd the series becomes $1^2 + 2.2^2 + 3^2 + 2.4^2 + 5^2 + \cdots + 2(n - 1)^2 + n^2
  = \frac{n(n  1)^2}{2} + n^2 = \frac{n^2(n + 1)}{2}$.
\item Let the numbers are $a - 3d, a - d, a + d, a - 3d$. All these will be integers and also $2d$ will be
  an integer.

  Now $(a - 3d)^2(a - d)^2(a + d^2)(a + 3d)^2 + (2d)^4 = a^4 - 10a^2d^2 + 9d^4 + 16d^4 = a^4 - 10a^2d^2 +
  25d^4 = (a^2 - 5d^2)^2$.

  Now $a^2 - 5d^2 = a^2 - 9d^2 + 4d^2 = (a - 3d)(a + 3d) + (2d)^2$. All the terms in this expression are
  integers, and hence proven.
\item From Vieta's relation $\sum x_i = 1, \sum x_ix_j = \beta$, and $\prod x_i = -\gamma$

  Let $x_1 = a - d, x_2 = a$, and $x_3 = a + d$. $\Rightarrow 3a = 1\Rightarrow a = \frac{1}{3}$.

  From second relation $(a - d)a + (a + d)a + (a + d)(a - d) = \beta \Rightarrow 3a^2 - d^2 = \beta$

  $\Rightarrow \frac{1}{3} - \beta = d^2 \Rightarrow \frac{1}{3} - \beta \geq
  0 \Rightarrow \beta \leq \frac{1}{3}$.

  From third relation $a(a^2 - d^2) = -\gamma \Rightarrow \frac{1}{3}\left(\frac{1}{9} - d^2\right)
  = \gamma$

  $\Rightarrow \gamma + \frac{1}{27} = \frac{d^2}{3}\Rightarrow \gamma\geq -\frac{1}{27}$.
\item Given, $\frac{S_7}{S_{11}} = \frac{6}{11} \Rightarrow \frac{7(a + 6d)}{a + 10d} = 6 \Rightarrow a =
  9d$.

  Also given, $130 < t_7 < 140 \Rightarrow 130 < a + 6d < 140 \Rightarrow 130 < 15d < 140 \Rightarrow d =
  9$.
\item From question $\frac{n(n + 1)}{2} - k - k - 1 = 1224 \Rightarrow n^2 + n - 2450 = 4k\Rightarrow (n +
  50)(n - 49) = 4k$

  $\therefore n > 49$. Let $n = 50\Rightarrow k = 25$.
\item $\frac{S_m}{S_n} = \frac{S_{5n}}{S_n} = \frac{5n[2a_1 + (5n - 1)d]}{n[2a_1 + (n - 1)d]} = \frac{5[(6 -
    d) + 5n]}{6 - d + n}$

  If $6 - d = 0$ then the fraction is independent of $n$. Hence, $d = 6$.
\item Given, $a_k = 2a_{k - 1} - a_{k - 2}\Rightarrow a_1, a_2, a_3, \ldots, a_{11}$ are in an A.P.

  $\Rightarrow \frac{a_1^2 + a_2^2 + \cdots + a_{11}^2}{90} = \frac{11a^2 + 35\times11d^2 + 10ad}{11} = 90$

  $\Rightarrow 35d^2 + 150d + 135 = 0 \Rightarrow d = -3, -\frac{9}{7}$. Given $a_2 < \frac{27}{2}$ implies
  $d = -3$ and $d\neq -\frac{9}{7}$.

  $\Rightarrow \frac{a_1 + a_2 + \cdots + a_{11}}{11} = \frac{11}{2}[30 - 10\times 3] = 0$.
\item Clearly $b = ar$ and $c = ar^2$. Also, because $3a, 7b$ and $15c$ are in A.P., therefore

  $14b = 3a + 15c\Rightarrow 14r = 3 + 15r^2 \Rightarrow r = \frac{1}{3}$ from given condition on $r$.

  Thus, common difference is $7b - 3a = -\frac{2a}{3}$.

  So, the $4$th terms of the A.P.\ is $3a + 3.-\frac{2a}{3} = a$.
\item Let $r$ be the c.r.\ of the G.P., then $a + ar + ar^2 = xar \Rightarrow r^2 + (1 - x)r + 1 = 0$

  Because $x$ is real the discriminant of the above equation must be greater that zero.

  $\Rightarrow (1 - x)^2 - 4\geq 0 \Rightarrow x^2 - 2x - 3 \geq 0 \Rightarrow (x - 3)(x + 1)\geq 0$

  $\Rightarrow x \leq -1$ and $x\geq 3$.
\item Given, $(\alpha^2 + \beta^2)^2 = (\alpha + \beta)(\alpha^3 + \beta^3)\Rightarrow \alpha^4 + \beta^4 +
  2\alpha^2\beta^2 = \alpha^4 + \beta^4 + \alpha^3\beta + \alpha\beta^3$

  $\Rightarrow \alpha^2\beta(\alpha - \beta) + \alpha\beta^2(\beta - \alpha) = 0$

  Thus, $\alpha = \beta\Rightarrow \Delta = 0$.
\item Let $S = (x + y) + (x^2 + xy + y^2) + (x^3 + x^2y + xy^2 + y^2) + \cdots$

  $(x - y)S = (x^2 - y^2) + (x^3 - y^3) + (x^4 - y^4) + \cdots = (x^2 + x^3 + x^4 + \cdots) - (y^2 + y^3 +
  y^4 + \cdots)$

  $= \frac{x^2}{1 - x} - \frac{y^2}{1 - y}\Rightarrow S = \frac{x + y - xy}{(1 - x)(1 - y)}$.
\item $S_n = \frac{1 - q^n}{1 - q}$ and $T_n = \frac{1 - \left(\frac{q + 1}{2}\right)^n}{1 - \frac{q +
    1}{2}} = \frac{1 - (q + 1)^n}{2^{n - 1}.(1 - q)}$.

  L.H.S. $= \frac{1}{1 - q}[C_1^^{101} + C_2^^{101} + \cdots + C_{101}^^{101} - C_1^^{101}q +
    C_2^^{101}q^2 + \cdots + C_{101}^^{101}q^{101}] = \alpha.T_{100}$

  $\Rightarrow \frac{1}{1 - q}[(2^{101} - 1) - (1 + q)^{101} + 1] = \alpha.T_{100}$

  On simplification we get $\Rightarrow \alpha = 2^{100}$.
\item $k = 1 + 2.\left(\frac{11}{10}\right) + 3.\left(\frac{11}{10}\right)^2 + \cdots +
  10.\left(\frac{11}{10}\right)^9$

  $\frac{11}{10}k = 1.\frac{11}{10} + 2.\left(\frac{11}{10}\right)^2 + 3.\left(\frac{11}{10}\right)^3
  + \cdots + 10.\left(\frac{11}{10}\right)^{10}$

  $\Rightarrow k\left(1 - \frac{11}{10}\right) = 1 + \frac{11}{10} + \left(\frac{11}{10}\right)^2 + \cdots +
  \left(\frac{11}{10}\right)^9 - \left(\frac{11}{10}\right)^{10}$

  On simplification we get $k = 100$.
\item Let us prove these one by one.
  \startitemize[a]
  \item Roots of the given equation are $\alpha = \frac{-1 + \sqrt{5}}{2}$ and $\beta = \frac{-1
    - \sqrt{5}}{2}$. $\Rightarrow \alpha + \beta = 1, \alpha\beta = -1$.

    $a_n = \frac{\alpha^n - \beta^n}{\alpha - beta} \Rightarrow a_{n + 1} = \frac{\alpha^{n + 1} - \beta^{n
      + 1}}{\alpha - \beta}$

    $a_{n + 1} = \alpha^n + \alpha^{n - 1}\beta + \alpha^{n - 2}\beta^2 + \cdots + \alpha\beta^{n - 1}
    + \beta^n$

    $= \alpha^n + \beta^n - \left(\alpha^{n - 2} + \alpha^{n - 3}\beta + \cdots + \beta^{n -
      2}\right)[\because \alpha\beta = -1]$

    $\Rightarrow a_{n + 1} + a_{n - 1} = \alpha^n + \beta^n = b_n$.
  \item $\because \alpha^2 = \alpha + 1$ and $\beta^2 = \beta + 1$

    $\Rightarrow alpha^{n + 2} = \alpha^{n + 1} + \alpha^n$ and $\beta^{n + 2} = \beta^{n + 1} + \beta^n$

    $\alpha^{n + 2}\beta^{n + 2} = (\alpha^{n + 1} + \beta^{n + 1}) + (\alpha^n + \beta^n)$

    $\Rightarrow a_{n + 2} = a_{n + 1} + a_n\Rightarrow a_{n + 1} = a_n + a_{n - 1} \cdots a_3 = a_2 + a_1$

    $\Rightarrow a_{n + 2} = (a_n + a_{n - 1} + \cdots + a_2 + a_1) + a_2[\because a_2 = \frac{\alpha^2
      - \beta^2}{\alpha - \beta} = 1]$

    $\Rightarrow a_1 + a_2 + \cdots + a_n = a_{n + 2} - 1$.
  \item $\displaystyle\sum_{n = 1}^\infty\frac{a_n}{10^n} = \sum_{n = 1}^\infty\frac{\alpha^n
    - \beta^n}{(\alpha -\beta)10^n} = \frac{1}{\alpha - \beta}\sum_{n =
    1}^\infty\left[\left(\frac{\alpha}{10}\right)^n - \left(\frac{\beta}{10}\right)^n\right]$

    On simplification it gives the value $\frac{10}{89}$.
  \stopitemize
\item We have to find an odd $n$ such that $B_n = 1 - A_n > A_n \Rightarrow A_n < \frac{1}{2}$.

  $A_n = \frac{3}{4}\frac{\left[1 - \left(-\frac{3}{4}\right)^n\right]}{1
  + \frac{3}{4}}\Rightarrow \left(-\frac{3}{4}\right)^n > -\frac{1}{6}$.

  For all even values of $n$ required condition is fulfilled, however, we have to find odd values for it.

  We see that $n = 7, \left(-\frac{3}{4}\right)^7 = -\frac{2187}{12288} < -\frac{}{6}$, which is minimum
  odd value for $n$ for satisfying the required condition.
\item Given, $S_k = \frac{\frac{k - 1}{k!}}{1- \frac{1}{k}} = \frac{1}{(k - 1)!}$

  Now, $(k^2 - 3k + 1)S_k = [(k - 2)(k - 1) - 1]\frac{1}{(k - 1)!} = \frac{1}{(k - 3)!} - \frac{1}{(k -
    1)!}$

  $\Rightarrow \displaystyle\sum_{k = 1}^{100}|(k^2 - 3k + 1)S_k| = 1 + 1 + 2 - \left(\frac{1}{99!}
  + \frac{1}{98!}\right) = 4 - \frac{100^2}{100!}$.

  Thus, $\frac{100^2}{100!} + \displaystyle\sum_{k = 1}^{100}|(k^2 - 3k + 1)S_k| = 4$.
\item From given condition $\cos x = \frac{2\cos(x - y)\cos(x + y)}{\cos(x - y) + \cos(x + y)}$

  $\Rightarrow \cos x(2\cos x\cos y) = 2[\cos^2x - \sin^2y]\Rightarrow \cos^2x(1 - \cos y) = \sin^2y$

  $\Rightarrow \cos^2x.2\sin^2\frac{y}{2} = 4\sin^2\frac{y}{2}\cos^2\frac{y}{2}\Rightarrow \cos
  x\sec\frac{y}{2} = \pm\sqrt{2}$.
\item $\displaystyle\sum_{n = 1}^{15} = \frac{1^3 + 2^3 + \cdots + n^3}{1 + 2 + \cdots + n} = \sum_{n =
  1}^{15}\frac{\left(\frac{n(n + 1)}{2}\right)^2}{\frac{n(n + 1)}{2}}$

  $= \frac{1}{2}\displaystyle\sum_{n = 1}^{15}(n^2 + n) = \frac{1}{2}\left[\frac{15\times16\times 31}{6}
  + \frac{15\times 16}{2}\right] = 680$

  $\frac{1}{2}(1 + 2 + \cdots + 15) = \frac{15\times 16}{4} = 60$.

  $\therefore$ Required sum is $680 - 60 = 620$.
\item $t_n = \frac{(2n + 1)\left(\frac{n(n + 1)}{2}\right)^2}{\frac{n(n + 1)(2n + 1)}{6}} = \frac{3}{2}n(n +
  1)$

  $S_{10} = \frac{3}{2}\displaystyle\sum_{n = 1}^^{10}(n^2 + n) = \frac{3}{2}\left[\frac{10\times 11\times
      21}{6} + \frac{10\times 11}{2}\right] = 660$.
\item Let $S = \left(\frac{3}{4}\right)^3 + \left(\frac{6}{4}\right)^3 + \left(\frac{9}{4}\right)^3
  + \left(\frac{12}{4}\right)^3 + \cdots$ upto $15$ terms

  $= \left(\frac{3}{4}\right)^3\left[1^3 + 2^3 + \cdots + 15^3\right]
  = \left(\frac{3}{4}\right)^3\left(\frac{15\times 16}{2}\right)^2 = 27\times 225$

  Thus, $k = 27$.
\item $S_k = \frac{k + 1}{2}\Rightarrow S_k^2 = \frac{k^2 + 2k + 1}{4}$

  $\therefore S_1^2 + S_2^2 + \cdots + S_{10}^2 = \displaystyle\sum_{k = 1}^{10}\left(\frac{k^2 + 2k +
  1}{4}\right)$

  $= \frac{1}{4}\left[\frac{10\times 11\times 21}{6} + 10\times 11 + 10\right] = \frac{505}{4}\Rightarrow A
  = 303$.
\item The expression is $\frac{x^my^n}{\left(1 + x^{2m}\right)\left(1 + y^{2n}\right)} = \frac{1}{\left(x^m
  + x^{-m}\right)\left(y^n + y^{-n}\right)}$

  Since $x, y$ are real numbers and $m, n$ are positive integers so using A.M.-G.M.\ inequality we have

  $x^m + x^{-m}\geq 2$ and $y^n + y^{-n}\geq 2$. So maximum value of given expression is $\frac{1}{4}$.
\item $t_n = \frac{3n(1^2 + 2^2 + \cdots + n^2)}{2n + 1} = \frac{3n.n(n + 1)(2n + 1)}{6(2n + 1)}
  = \frac{n^2(n + 1)}{2}$

  $\therefore S_n = \frac{1}{2}\left[\left(\frac{n(n + 1)}{2}\right)^2 + \frac{n(n + 1)(2n + 1)}{6}\right]$

  $\Rightarrow S_{15} = 7820$.
\item Let $a$ be the first term and $d$ be the c.d.\ of the given A.P. Given $\displaystyle\sum_{k =
  0}^{12}a_{4k + 1} = 416$ and $a_9 + a_{43} = 66$.

  $\Rightarrow \frac{13}{2}[2a + 48d] = 416 \Rightarrow a + 24d = 32$ and $2a + 50d = 66 \Rightarrow a + 25d
  = 33$

  Thus, $d = 1, a = 8$ from the two equations.

  Given $a_1^2 + a_2^2 + \cdots + a_{17}^2 = 140m \Rightarrow 8^2 + 9^2 + \cdots + 24^2 = 140m$

  $\Rightarrow (1^2 + 2^2 + \cdots + 24^2) - (1^2 + 2^2 + \cdots + 7^2) = 140m\Rightarrow \frac{24.25.49}{6}
  - \frac{7.8.15}{6} = 140m\Rightarrow m = 34$.
\item $A = (1^2 + 2^2 + \cdots + 20^2) + (2^2 + 4^2 + \cdots + 20^2) = \frac{20\times21\times41}{6}
  + \frac{4\times10\times11\times 21}{6} = \frac{20\times41\times 63}{6}$

  $B = (1^2 + 2^2 + \cdots + 40^2) + (2^2 + 4^2 + \cdots + 40^2) = \frac{40\times\41\times 123}{6}$.

  Solving the equation $B - 2A = 100\lambda$ gives us $\lambda = 248$.
\item $S_{10} = \left(\frac{8}{5}\right)^2 + \left(\frac{12}{5}\right)^2 + \left(\frac{16}{5}\right)^2
  + \left(\frac{20}{5}\right)^2 + \cdots$ upto $10$ terms

  $= \frac{4^2}{5^2}[2^2 + 3^2 + \cdots + 10^2] = \frac{4^2}{5^2}\left[\frac{10\times11\times21}{6} -
    1\right] = \frac{16}{25}\times505 = \frac{16m}{5}\Rightarrow m = 101$.
\item Given, $m$ is the A.M.\ of $l$ and $n\Rightarrow l + n = 2m$.

  $l, G_1, G_2, G_3, n$ are in G.P. Let $r$ be the c.r.\ of the G.P. $\Rightarrow r =
  \left(\frac{n}{l}\right)^{\frac{1}{4}}$.

  $\therefore G_1^4 + 2G_2^4 + G_3^4 = 4lm^2n$.
\item $t_n = \frac{1^3 + 2^3 + \cdots + n^3}{1 + 3 + \cdots + (2n - 1)} = \left(\frac{n + 1}{2}\right)^2
  = \frac{n^2 + 2n + 1}{4}$

  $S_n = \frac{1}{4}\left[\frac{n(n + 1)(2n + 1)}{6} + n(n + 1) + n\right] \Rightarrow S_{9} = 96$.
\item Using A.M.-G.M.\ inequality we have

  $\frac{\sqrt{x^2 + x} + \frac{\tan^2\alpha}{\sqrt{x^2 + x}}}{2}\geq \tan\alpha\Rightarrow \sqrt{x^2 + x}
  + \frac{\tan^2\alpha}{\sqrt{x^2 + x}}\geq 2\tan\alpha$.
\item Given $a_1a_2\ldots a_n = c\Rightarrow a_1a_2\ldots a_{2n - 1}(2a_n) = 2c$

  $\therefore \frac{a_1 + a_2 + \cdots + 2a_n}{n}\geq\sqrt[n]{a_1a_2\ldots(2a_n)} = \sqrt[n]{2c}$

  Therefore, minimum value is $n\sqrt[n]{2c}$.
\item Using A.M.-G.M.\ relationshiop, we have $\frac{(a + b) + (c + d)}{2}\geq \sqrt{(a + b)(c + d)}
  \Rightarrow M\leq 1$. Also, $(a + b + c + d) > 0 \Rightarrow M > 0$.
\item Let the roots be $\alpha$ and $\beta$, then we have $\alpha + \beta = \frac{4 + \sqrt{5}}{5 +
  \sqrt{2}}$ and $\alpha\beta = \frac{8 + 2\sqrt{5}}{5 + \sqrt{2}}$.


  The harmonic mean is given by $\frac{2\alpha\beta}{\alpha + \beta} = \frac{2\left(8 + 2\sqrt{5}\right)}{4
    + \sqrt{5}} = 4$.
\item Using A.M.-G.M.\ between $(a + b - c)$ and $(b + c - a)$, we have

  $\frac{a + b - c + b + c - a}{2}\geq \sqrt{(a + b - c)(b + c - a)}\Rightarrow b\geq \sqrt{(a + b - c)(b +
  c - a)}$.

  Similarly, $c\geq \sqrt{(b + c - a)(c + a - b)}$ and $a\geq \sqrt{(c + a - b)(a + b - c)}$

  Multiplying the three relationships $abc \geq (a + b - c)(b + c - a)(c + a - b)$, and hence proven.
\item We have $(1 + a)(1 + b)(1 + c) = 1 + a + b + c + ab + bc + ca + abc$

  Using A.M.-G.M.\ inequality, we have $\frac{1 + a + b + c + ab + bc + ca + abc}{7}\geq
  \sqrt[7]{a^4b^4c^4}$

  $\Rightarrow 1 + a + b + c + ab + bc + ca + abc\geq 7\sqrt[7]{a^4b^4c^4}$

  $\Rightarrow \left[(1 + a)(1 + b)(1 + c)\right]^7 > 7^7a^4b^4c^4$.
\item Let $G_m$ be the G.M.\ of $G_1, G_2, \ldots, G_n$.

  $\Rightarrow G_m = \left(G_1G_2\ldots G_n\right)^{1/n} =
  \left[(a_1).(a_1a_1r)^{1/2}(a_1a_1ra_1r^2)^{1/3}\ldots (a_1a_1r\ldots a_1r^{n - 1})^{1/n}\right]$, where
  $r$ is the common ratio of the G.P.

  $= \left[a_1^nr^{\frac{1}{2} + 1 + \frac{3}{2} + \cdots + \frac{n - 1}{2}}\right]^{\frac{1}{n}} =
  a_1r^{\frac{n - 1}{4}}$

  $A_n = \frac{a_1 + a_2 + \cdots + a_n}{n} = \frac{a_1(1 - r^n)}{n(1 - r)}$

  $H_n = \frac{n}{\frac{1}{a_1} + \frac{1}{a_2} + \cdots + \frac{1}{a_n}} = \frac{a_1n(1 - r)^{n - 1}}{1 -
    r^n}$

  $A_nH_n = a_1^2r^{n - 1}\Rightarrow \displaystyle\prod_{i = 1}^nA_iH_i = \prod_{i = 1}^n(a_1^2r^{n - 1}) =
  \left[a_1r^{\frac{n - 1}{4}}\right]^{2n} = G_m^{2n}$

  $\Rightarrow G_m = \left[\displaystyle\prod_{i= 1}^nA_iH_i\right]^{1/2n}$.
\item Since $y_1, y_2, y_3$ are real number so $3^{y_1}, 3^{y_2}, 3^{y_3}$ will be real numbers. Using
  A.M.-G.M.\ inequality, we have

  $\frac{3^{y_1} + 3^{y_2} + 3^{y_3}}{3}\geq \left(3^{y_1}.3^{y_2}.3^{y_3}\right)^{\frac{1}{3}}\Rightarrow
  3^{y_1} + 3^{y_2} + 3^{y_3} \geq 3\left(3^{y_1}.3^{y_2}.3^{y_3}\right)^{\frac{1}{3}}$

  Taking $\log$ of both sides with base $3$, we have

  $\log_3\left(3^{y_1} + 3^{y_2} + 3^{y_3}\right)\geq \left[1 + \frac{1}{3}(y_1 + y_2 + y_3)\right] = 4$

  Thus, $m = 4$.

  Using A.M.-G.M.\ inequality for $x_1, x_2, x_3$, we have

  $\frac{x_1 + x_2 + x_3}{3}\geq \left(x_1 + x_2 + x_3\right)^{\frac{1}{3}}$

  Taking $\log$ of both sides with base $3$, we have

  $\log_3\left(\frac{x_1 + x_2 + x_3}{3}\right)\geq \frac{1}{3}\left(\log_3 x_1 + \log_3x_2 +
  \log_3x_3\right)$

  $\Rightarrow M = 3$.

  Now, $\log_2m^3 + \log_3M^2 = 3\log_24 + 2\log_33 = 8$.
\item Given equality is $2(a_1 + a_2 + \cdots + a_n) = b_1 + b_2 + \cdots + b_n$

  $\Rightarrow 2.\frac{n}{2}[2c + (n - 1)2] = c\left(2^n- 1\right)$

  $\Rightarrow c\left[2^n - 2n - 1\right] = 2n^2 - 2n$

  $\because c\in\mathbb{N}\Rightarrow 2n^2 - 2n \geq 2^n - 2n - 1\Rightarrow n\leq 6$ and also $c >
  0\Rightarrow n > 2$

  The possible values of $n$ are $3, 4, 5, 6$.

  Only for $n = 3, c$ has integral value of $12$ and for other $n$ it has fractional value. So $c =12, n =
  3$.
\item Given $a, b, c$ are in G.P. Let $r$ be the c.r.\ of this G.P. Given that A.M.\ of $a, b, c$ is $b +
  2$. Thus,

  $\frac{a + b + c}{3} = b + 2 \Rightarrow a + ar + ar^2 = 3ar + 6 \Rightarrow (r - 1)^2 = \frac{6}{a}$.

  Since $\frac{6}{a}$ must be a perfect square and $a\in\mathbb{N}$, so $a$ can be $6$ only.

  $\Rightarrow r - 1 = \pm 1\Rightarrow r = 2$.

  $\Rightarrow \frac{a^2 + a - 14}{a + 1} = 4$.
\item Using A.M.-G.M., we have

  $\frac{a^{-5} + a^{-4} + 3a^{-3} + 1 + a^8 + a^{10}}{8}\geq
  \left(a^{-5}.a^{-4}.(a^{-3})^3.a^8.a^{10}\right)^{\frac{1}{8}}$

  $\Rightarrow a^{-5} + a^{-4} + 3a^{-3} + 1 + a^8 + a^{10}\geq 8.1 = 8$. Hence, minimum value is $8$.
\item Given, $a_1 = b_1 = 1$. $a_2 = a_1 + 2 = 3, a_3 = a_2 + 2 = 5\Rightarrow a_n = 2n - 1$. $b_2 = a_2 +
  b_1 = 4, b_3 = a_3 + b_2 = 9\Rightarrow b_n = n^2$

  $\displaystyle\sum_{i = 1}^{15}a_n.b_n = \sum_{i = 1}^{15}\left(2n^3 - n^2\right) = 2.\frac{n^2(n +
    1)^2}{4} - \frac{n(n + 1)(2n + 1)}{6} = 27560$.
\item Given, $\frac{1}{2.3^{10}} + \frac{1}{2^2.3^9} + \cdots + \frac{1}{2^{10}.3} =
  \frac{K}{2^{10}.3^{10}}$

  $\Rightarrow K = 2^9 + 2^8.3 + 2^73^2 + \cdots + 3^9 = \frac{2^9\left(\frac{3^{10}}{2^{10}} -
    1\right)}{\frac{3}{2} - 1} = 3^{10} - 2^{10} = \left(3^5 - 2^5\right)\left(3^5 + 2^5\right)$

  $= 211\times275 = (35\times6 + 1)(45\times6 + 5) = 6\lambda + 5$

  Hence, the remainder is $5$.
\item Given sequence can be written as $\left(1 - \frac{2}{3}\right) + \left(1 - \frac{4}{9}\right) +
  \left(1 - \frac{8}{27}\right) + \left(1 - \frac{16}{81}\right) + \cdots$ for $100$ terms

  $= 100 - \left[\frac{2}{3} + \frac{2^2}{3^2} + \cdots\right] = 100 - \frac{\frac{2}{3}\left(1 -
    \frac{2^{100}}{3^{100}}\right)}{1 - \frac{2}{3}}$

  $= 98 + 2.\frac{2^{100}}{3^{100}}$, so the greatest integer is $98$.
\item According to question, $\big(\left(2.2^2.\ldots2^{60}\right)\left(4.4^2\ldots4^n\right))^{\frac{1}{60
    + n}} = 2^{\frac{225}{8}}$

  $\Rightarrow \big(2^{30\times61}4^{\frac{n(n + 1)}{2}}) = 2^{\frac{225}{8}}\Rightarrow 2^{1830 + n^2 + n}
  = 2^{\frac{225(60 + n)}{8}}\Rightarrow 8n^2 - 217n + 1140 = 0$

  $\Rightarrow n = 20, \frac{57}{8}$, we discard fractional value as $n$ can only be an integer.

  Now $\displaystyle\sum_{k = 1}^n(nk -k^2) = \frac{n^2(n + 1)}{2} - \frac{n(n + 1)(2n + 1)}{6} = 1330$.
\item Given, $\displaystyle\sum_{k = 1}^{10}\frac{k}{k^4 + k^2 + 1} = \frac{m}{n}$

  $\Rightarrow \frac{1}{2}\displaystyle\sum_{k = 1}^{10}\frac{1}{k}\left(\frac{k^2 + k + 1 - (k^2 - k + 1)}{(k^2 + k + 1)(k^2
  - k + 1)}\right) = \frac{1}{2}\displaystyle\sum_{k = 1}^{10}\frac{1}{k}\left(\frac{1}{k^2 + k + 1} - \frac{1}{k^2
  - k + 1}\right)$

  $\Rightarrow \frac{55}{111} = \frac{m}{n}\Rightarrow m + n = 166$.
\item We have $d = \frac{l - a}{n - 1}$, where the terms have usual meanings. $d = \frac{99}{n - 1}$.

  The factors of $99$ are $3, 9, 11, 33$. Clearly, $n = 4, 10, 12$ are the ones, which will give integral
  common difference. These will be $33, 11, 9$ and hence the sum will be $53$.
\item $A = \frac{1}{2} + \frac{1}{4^2} + \frac{1}{2^3} + \frac{1}{4^4} + \cdots \infty$

  $= \frac{\frac{1}{2}}{1 - \frac{1}{4}} + \frac{\frac{1}{4^2}}{1 - \frac{1}{16}} = \frac{11}{15}$

  $B = -\frac{1}{2} + \frac{1}{4^2} - \frac{1}{2^3} + \frac{1}{4^4} + \cdots \infty$

  $= \frac{-\frac{1}{2}}{1 - \frac{1}{4}} + \frac{\frac{1}{16}}{1 - \frac{1}{16}} = -\frac{9}{15}$

  $\therefore \frac{A}{B} = -\frac{11}{9}$.
\item Let $r$ be the common ratio of the G.P., then

  $3a_2 + a_3 = 2a_4 \Rightarrow 2r^2 - r + 3 = 0 \Rightarrow r = -1, \frac{3}{2}$

  $a_2 + a_4 = 2a_3 + 1 \Rightarrow a(r + r^3 - 2r^2) = 1\Rightarrow a = \frac{8}{3}$ when $r =
  \frac{3}{2}$.

  When $r = -1, a = -\frac{1}{4}$, which is discarded as $a_1 > 0$.

  $\Rightarrow a_2 + a_4 + 2a_5 = 40$.
\item $t_n = \frac{n}{4n^4 + 1} = \frac{n}{(2n^2 + 1)^2 - (2n)^2} = \frac{n}{(2n^2 + 2n + 1)(2n^2 - 2n + 1)}
  = \frac{1}{4}\left[\frac{1}{2n^2 - 2n + 1} - \frac{1}{2n^2 + 2n + 1}\right]$

  $S_{10} = \frac{1}{4}\left[1 - \frac{1}{221}\right] = \frac{55}{221} = \frac{m}{n}$

  $\therefore m  + n = 276$.
\item Given, $S = 2 + \frac{6}{7} + \frac{12}{7^2} + \frac{20}{7^3} + \frac{30}{7^4} + \cdots \infty$

  $\frac{S}{7} = \frac{2}{7} + \frac{6}{7^2} + \frac{12}{7^3} + \frac{20}{7^4} + \cdots \infty$

  Subtracting, we have $\frac{6S}{7} = 2 + \frac{4}{7} + \frac{6}{7^2} + \frac{8}{7^3} + \frac{10}{7^4} +
  \cdots \infty$

  $\frac{6S}{7^2} = \frac{2}{7} + \frac{4}{7^2} + \frac{6}{7^3} + \frac{8}{7^4} + \cdots \infty$

  Subtracting again, we have

  $\frac{6^2S}{7^2} = 2 + \frac{2}{7} + \frac{2}{7^2} + \frac{2}{7^3} + \cdots \infty$

  $= \frac{2}{1 - \frac{1}{7}} = \frac{7}{3}\Rightarrow 4S = \left(\frac{7}{3}\right)^3$.
\item $a_{n + 2}a_{n + 1} - a_{n + 1}a_n = 2\Rightarrow \frac{a_n + \frac{1}{a_{n + 1}}}{a_{n + 2}} =
  \frac{a_{n + 2} - \frac{1}{a_{n + 1}}}{a_{n + 2}} = 1 - \frac{1}{a_{n + 1}a_{n + 2}} = 1 - \frac{1}{2(r +
    1)} = \frac{2r + 1}{2(r + 1)}$

  Product is given by $\displaystyle\prod_{r = 1}^{30}\frac{2r + 1}{2(r + 1)} = \frac{1.3.5\ldots
    61}{2^{30}(2.3\ldots 31)}$

  $= \frac{61!}{2^{60}31!30!}$, and thus, $\alpha$ is $-60$.
\item The terms in A.P.\ are $C_2^^4\times\frac{\beta^2}{6}, -6\beta, -C_3^^6\times\frac{\beta^3}{8}$.

  Thus, $\beta^2 - \frac{5}{2}\beta^3 = -12\beta\Rightarrow \beta = \frac{12}{5}, -2\therefore \beta =
  \frac{12}{5}$ as it is given that $\beta > 0$.

  $\Rightarrow d = -\frac{72}{5} - \frac{144}{25} = -\frac{504}{25}$.

  $\therefore 50 - \frac{2d}{\beta^2} = 57$.
\item Given, $A_1A_3A_5A_7 = \frac{1}{1296}\Rightarrow A_4^4 = \frac{1}{1296}\Rightarrow A_4 = \frac{1}{6}$

  Also given, $A_2 + A_4 = \frac{7}{36}\Rightarrow A_2 = \frac{1}{36}\Rightarrow A_6 = 1\Rightarrow A_8 = 6
  \Rightarrow A_{10} = 36$

  $\therefore A_6 + A_8 + A_{10} = 43$.
\stopitemize
