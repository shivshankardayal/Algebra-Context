% -*- mode: context; -*-
\chapter{Miscellaneous Problems}
\startitemize[n, 1*broad]
\item Let the numbers be $a$ and $b$ and harmonic means are $h_1$ and $h_1$. Let $d$ be the c.d.\ of the
  corresponding A.P. Then

  $\frac{1}{b} = \frac{1}{a} + 3d\Rightarrow d = \frac{a - b}{3ab}$

  $\frac{1}{h_1} = \frac{1}{a} + \frac{a - b}{3ab} = \frac{a + 2b}{3ab}$.

  Given that $\frac{3ab}{a + 2b} = \sqrt{ab}\Rightarrow 9a^2b^2 = (a + 2b)^2ab \Rightarrow 9ab = a^2 + 4b^2
  + 4ab$

  $\Rightarrow a^2 - 5ab + 4b^2 = 0 \Rightarrow (a - b)(a - 4b) = 0$.

  So either the numbers are equal which will make H.M.'s also equal to the numbers or one of the number is
  four times the other number.
\item Let the numbers be $a$ and $b$ and harmonic mean be $h$. Then

  $\frac{1}{h} = \frac{\frac{1}{a} + \frac{1}{b}}{2} \Rightarrow h = \frac{2ab}{a + b} = 4 \Rightarrow 2a +
  2b - ab = 0$

  Given that $2A + G^2 = 27 \Rightarrow a + b + ab = 27 \Rightarrow a + b = 9$.

  Solving we get $a, b = 6, 3$.
\item Since $n$ is odd, the no.\ of positive term of the given series is $\frac{n + 1}{2}$ and no.\ of
  negative terms of negative terms of the series is $\frac{n - 1}{2}$.

  Sum of positive terms is $\frac{n + 1}{4}\left[2a + \frac{n - 1}{2}d\right]$ and sum of negative terms is
  $\frac{n - 1}{4}\left[2(a + d) + \frac{n - 3}{2}d\right]$.

  Adding we get the required sum.

  Alternatively we observe that first $\frac{n- 1}{2}$ pairs will have a sum of $-\frac{(n - 1)d}{2}$ and
  $n$th term is $a + (n- 1)d$. So the sum is $a + \frac{n - 1}{2}d$.
\item $x = 2n\pi\pm \cos^{-1}y$. So we see that it is a periodic function which has a period of $2\pi$. If
  we move only in one(positive/negative) direction then all values of $y$ will give an A.P., however, if we
  want to move in both directions then $x$ must be $0, \pi$ making $y = \pm1$.
\item Let $a$ be the first term and $r$ be the common ratio of the G.P. Sum of all terms is given by
  $\frac{a}{1 - r}$ and sum of odd terms is given by $\frac{a}{1 - r^2}$.

  Given that $\frac{a}{1 - r} = \frac{5a}{1 - r^2}\Rightarrow 1 + r = 5\Rightarrow r = 4$.
\item Let $S_p$ represent sum of $p$ terms of the given A.P., then $t_p = S_p - S_{p - 1} = a + (2p - 1)b$.

  $\Rightarrow t_4 = a + 7b$.
\item We see that c.d.\ is $5$. Either we can apply the $t_n = 3 + (n - 1)5$ and solve for each or on simple
  observation we find that last digit should be $3$ and hence $303$ is the number, which will be a term in
  the given sequence.
\item Let $a$ be the first term and $d$ be the common ratio of the A.P., then $A = a, B = a + d, c = a +
  2d$.

  Substituting the points in the given equation we find that the point $(1, -2)$ satisfies the
  equation. Hence, the given straight line will pass through $(1, -2)$.
\item Let $S_n$ represent the sum of $n$ terms of the A.P., then $S_n = \frac{3n^2 + 5n}{36}$.

  $t_n = S_n - S_{n - 1} = \frac{6n - 3 + 5}{36} = \frac{3n + 1}{18}$.

  c.d. $= t_n - t_{n - 1} = \frac{1}{6}$.
\item Given, $\frac{a - b}{b - c} = \frac{a}{c} \Rightarrow b = \frac{2ac}{a + c}$, so the given sequence is
  a harmonic sequence.
\item Common ratio of the given series is $\frac{2x}{x + 3}$. For the sum to be definite c.r should be less
  that $|1|$. Therefore,

  $-1 < \frac{2x}{x + 3} < 1 \Rightarrow -1 < x < 3$.
\item Let the numbers be $x_1, x_2, \ldots, x_n$, then using A.M.-G.M. relation we have

  $\frac{x_1 + x_2 + \cdots + x_n}{n}\geq \sqrt[n]{x_1.x_2.\ldots x_n} \Rightarrow x_1 + x_2 + \cdots +
  x_n\geq n$.
\item Let $a$ be the first term and $d$ be the c.d. of the given A.P. Given that $S_{2n} = 3S_n$ so we have

  $n[2a + (2n - 1)d] = \frac{3n}{2}[2a + (n - 1)d] \Rightarrow 4a + (4n - 2)d = 6a + 3(n - 1)d$

  $\Rightarrow 2a = (n + 1)d$

  $\frac{S_{3n}}{S_n} = \frac{3[2a + (3n - 1)d]}{2a + (n - 1)d} = \frac{12nd}{2nd} = 6$.
\item Each term consistes of two integers which are in two different A.P. The $n$th term is given by $t_n =
  [1 + (n - 1)2].[3 + (n - 1)2] = (2n - 1)(2n + 1) = 4n^2 - 1$

  Thus, sum $= S_n = \displaystyle\sum_{i = 1}^n(4i^2 - 1) = 4.\frac{n(n + 1)(2n + 1)}{6} - 1$.
\item Let $S = 1 + \frac{4}{5} + \frac{7}{5^2} + \frac{10}{5^3} + \cdots$. We observe that numerator is in
  A.P. with first term as $1$ and c.d.\ as $3$ while denominator is in G.P.\ with first terms as $1$ and
  c.r.\ as $\frac{1}{5}$.

  So using the technique for AGP we have $S - \frac{S}{5} = 1 + \frac{3}{5} + \frac{3}{5^2} + \cdots$

  $\Rightarrow \frac{4S}{5} = 1 + \frac{3}{5}.\frac{1}{1 - \frac{1}{5}} = 1 + \frac{3}{4}
  = \frac{7}{4} \Rightarrow S = \frac{35}{16}$.
\item From the given information we can write $2b = a + c; x^2 = ab$ and $y^2 = bc$

  $\Rightarrow x^2 + y^2 = b(a + c) = 2b^2$ and thus, $x^2, b^2, y^2$ are in A.P.
\item Given, $2\log_{10}(2^x - 1) = \log_{10}2 + \log_{10}(2^x + 3)$

  $\Rightarrow (2^x - 1)^2 = 2.2^x + 6 \Rightarrow 2^{2x} -4.2^x - 5 = 0 \Rightarrow 2^x = 5, -1$, however,
  $2^x$ cannot be $-1$.

  $\therefore 2^x = 5 \Rightarrow x = \log_25$.
\item Since $p, q, r$ are in A.P.\ so we can write $q - p = r - q$. Now

  $\frac{m^{7q}}{m^{7p}} = m^{7(q - p)} = m^{7(r - q)} = \frac{m^{7r}}{m^{7q}}$, and thus, $m^{7p}, m^{7q},
  m^{7r}$ are in G.P.
\item $x = \frac{1}{1 - \cos^2\phi} = \csc^2\phi; y = \frac{1}{1 - \sin^2\phi} = \sec^2\phi$ and $z
  = \frac{1}{1 - \cos^2\phi\sin^2\phi}$

  $xyz = \frac{1}{\sin^2\phi\cos^2\phi(1 - \cos^2\phi\sin^2\phi)}$

  $xy + z = \frac{1}{\sin^2\phi\cos^2\phi} + \frac{1}{1 - \cos^2\phi\sin^2\phi}
  = \frac{1}{\sin^2\phi\cos^2\phi(1 - \cos^2\phi\sin^2\phi)}$.

  Thus, $xyz = xy + z$.
\item Rewriting we have given series as $\frac{1}{\sqrt{2}}(2 + 4 + 6 + 8 + \cdots) = \frac{n(n +
  1)}{\sqrt{2}}$.
\item Since $a, b, c$ are in A.P., we can write $a - b + c = b \Rightarrow a - b + c = \frac{b - a + c + a +
  b - c}{2}$

  Thus, $\frac{b -a + c}{2}, \frac{a - b + c}{2}, \frac{a + b - c}{2}$ are in A.P.

  $\Rightarrow s - a, s - b, s - c$ are in A.P. $\Rightarrow \frac{s(s - a)}{\Delta}, \frac{s(s -
    b)}{\Delta}, \frac{s(s - c)}{\Delta}$ are in A.P.

  $\Rightarrow \frac{\Delta}{s(s - a)}, \frac{\Delta}{s(s - b)}, \frac{\Delta}{s(s - c)}$ are in H.P.

  $\Rightarrow \tan\frac{A}{2}, \tan\frac{B}{2}, \tan\frac{C}{2}$ are in H.P.
\item The ex-radii are given by $\frac{\Delta}{s - a}, \frac{\Delta}{s - b}$ and $\frac{\Delta}{s - c}$,
  which are in A.P.

  So $\frac{s - a}{\Delta}, \frac{s - b}{\Delta}, \frac{s - c}{\Delta}$ are in A.P.

  We know that if we multiply an A.P.\ by a constant then resulting series is in A.P. Multiplying given
  terms by $\Delta$, we have

  $s - a, s - b, s - c$ are in A.P. We also know that if we subtract a constant from A.P. then resulting
  series remains in A.P. So subtracting given terms from $s$ we have desired result.
\item Let $x$ be the first term and $y$ be the $2n - 1$st term. Then $n$th term of A.P.\ will be A.M.\ of
  $x$ and $y$, which is $a = \frac{x + y}{2}$; $n$th term of the G.P.\ will be G.M.\ of $x$ and $y$, which
  is $b = \sqrt{xy}$; and $n$th term of H.P.\ will be H.M.\ of $x$ and $y$ which is $c = \frac{2xy}{x + y}$.

  Clearly, $ac - b^2 = 0$.
\item Let $a, b, c$ are the roots; then from Vieta's relations we have $a + b + c = 12; ab + bc + ca = 39,
  abc = 28$.

  From first relation $3b = 12 \Rightarrow b = 4$ so the two other relations become $4a + 4c + ca = 39$ and
  $4ca = 28 \Rightarrow \ca = 7$

  $\Rightarrow a + c = 8 \Rightarrow a + \frac{7}{a} = 8 \Rightarrow a^2 - 8a + 7 = 0 \Rightarrow a = 1, 7$
  and $c = 7, 1$ so common difference is $\pm 3$.
\item Given, $(a^2 + b^2 + c^2)p^2 - 2(ab + bc + cd)p + (b^2 + c^2 + d^2)\leq 0$

  $\Rightarrow (a^2p^2 - 2abp + b^2) + (b^2p^2 - 2bcp + c^2) + (c^2p^2 - 2cdp + d^2)\leq 0$

  $\Rightarrow (ap - b)^2 + (bp - c)^2 + (cp - d)^2\leq 0$

  Clearly, only equality is possible as square of numbers cannot be negative. Hence, $p = \frac{b}{a}
  = \frac{c}{b} = \frac{d}{c}$, making $a, b, c, d$ form a G.P.
\item Let $d$ be the c.d.\ of A.P., then numbers are $6, 6 - d, 6 -2d, 3 - d$ such that $\frac{6 - 2d}{6 -
  d} = \frac{1}{2} \Rightarrow 12 - 4d = 6 - d \Rightarrow d = 2$.

  Thus, numbers are $6, 4, 2, 1$.
\item Given $\frac{44}{9} = 3 + 5r + 7r^2 + 9r^3 + \cdots$ to $\infty \Rightarrow \frac{44}{9}r = 3r + 5r^2
  + 7r^3 + \cdots$ to $\infty$

  Subtracting terms with corresponding powers of $r$, we get

  $\frac{44}{9}(1 - r) = 3 + 2r + 2r^2 + 2r^3 + \cdots $ to $\infty = 3 + \frac{2r}{1 - r} = \frac{3 - r}{1
    - r}$

  $\Rightarrow 44(1 - r)^2 = 27 - 9r \Rightarrow 44r^2 - 79r + 17 = 0 \Rightarrow (4r - 1)(11r - 17) = 0$

  For the given series $r$ has to be less than $|1|$ so $r = \frac{1}{4}$.
\item Let $S = 1^2 + 2^2x + 3^2x^2 + 4^2x^3 + \cdots$ to $\infty$. $\Rightarrow xS = 1^2x + 2^2x^2 + 3^2x^3
  + \cdots$ to $\infty$.

  Subtracting terms with corresponsding powers of $x$, we get

  $S(1 - x) = 1 + 3x + 5x^2 + 7x^3 + \cdots$ to $\infty$. $\Rightarrow Sx(1 - x) = x + 3x^2 + 5x^3 + \cdots$
  to $\infty$.

  Subtracting terms with corresponsding powers of $x$, we get

  $S(1 - x)^2 = 1 + 2x + 2x^2 + 2x^3 + \cdots$ to $\infty = 1 + \frac{2x}{1 - x} = \frac{1 + x}{1 - x}$

  $\Rightarrow S = \frac{1 + x}{(1 - x)^3}$.
\item $n$th term is given by $t_n = 1^2 + 2^2 + \cdots + n^2 = \frac{n(n + 1)(2n + 1)}{6} = \frac{2n^3 +
  3n^2 + n}{6}$

  Thus, sum is $S = \displaystyle\sum_{n = 0}^k\frac{2n^3 + 3n^2 + n}{6} = \frac{1}{3}\left(\frac{k(k +
    1)}{2}\right)^2 + \frac{1}{2}k(k + 1)(2k + 1) + \frac{k(k + 1)}{12}$

  Putting $k = 22$, we get sum as $S = 23276$.
\item Let $a$ be the first term and $d$ be the c.d. Then given $S_m:S_n = m^2:n^2$

  $\Rightarrow \frac{\frac{m}{2}[2a + (m - 1)d]}{\frac{n}{2}[2a + (n - 1)d]} = \frac{m^2}{n^2}$

  $\Rightarrow 2an + (m - 1)nd = 2am + (n - 1)md \Rightarrow 2a(m - n) = (m - n)d \Rightarrow a
  = \frac{d}{2}$

  $\Rightarrow \frac{t_m}{t_n} = \frac{a + (m - 1)d}{a + (n - 1)d} = \frac{2m - 1}{2n - 1}$.
\item Let $a$ be the first term and $d$ be the c.d.\ of the first A.P.\ and $x$ be the first term and $y$ be
  the c.d.\ of the second A.P. Also, let $S_n$ be the sum of $n$ terms of first A.P.\ and $t_{11}$ be the
  $11$th term of the first A.P., while $S_n'$ be the sum of the $n$ terms of the second A.P.\ and $t_{11}'$
  be the $11$th term of the second A.P.

  $S_n = (7n + 1)k = 2a + (n - 1)d \Rightarrow d = 7k, a = 4k$. $S_n' = (4n + 27)k = 2x + (n -
  1)y\Rightarrow y = 4k, x = \frac{31}{2}k$

  $\Rightarrow \frac{t_{11}}{t_{11}'} = \frac{a + (11 - 1)d}{x + (11 - 1)y} = \frac{4k + 70k}{\frac{31}{2}k +
    40k} = \frac{148k}{111k} = \frac{4}{3}$.
\item Let the distance between the stations be $x$ km. Total time taken $= \frac{x}{40} + \frac{x}{60}
  = \frac{x}{24}$ hours.

  $\therefore$ average speed $= \frac{2x}{\frac{x}{24}} = 48$ kmph.
\item Because $a, b, c$ are in A.P., therefore $2b = a + c$. Also, because $a^2, b^2, c^2$ are in H.P.,
  therefore $b^2 = \frac{2a^2c^2}{a^2 + c^2} \Rightarrow \frac{(a + c)^2}{4} = \frac{2a^2c^2}{a^2 + c^2}$

  $\Rightarrow a^4 + 2a^3c + 2ac^3 + 2a^2c^2 + c^4 = 8a^2c^2$, which gives $a = c$ upon solving making $a =
  b = c$.
\item $t_n = \frac{1}{1 + 2 + 3 + \cdots + n} = \frac{2}{n(n + 1)} = 2\left[\frac{1}{n} - \frac{1}{n +
    1}\right]$

  Thus, $S = 2.1 = 2$.
\item Let the roots be $\alpha$ and $\beta$. Then from question

  $\alpha + \beta = \frac{1}{\alpha^2} + \frac{1}{\beta^2} = \frac{(\alpha + \beta)^2 -
  2\alpha\beta}{\alpha^2\beta^2}$

  $\Rightarrow -\frac{b}{a} = \frac{\frac{b^2}{a^2} - 2\frac{c}{a}}{\frac{c^2}{a^2}} = \frac{b^2 -
  2ca}{c^2}$

  $\Rightarrow 2ca^2 - ab^2 = bc^2\Rightarrow ab^2, ca^2, bc^2$ are in A.P.
\item Let $a$ be the first term and $d$ be the c.d.\ of the A.P., then according to question

  $pt_p = qt_q$, where $t_p$ is the $p$th term and $t_Q$ be the $q$th term of the given A.P.

  $\Rightarrow ap + p(p - 1)d = aq + q(q - 1)d \Rightarrow a(p - q) = (q^2 - p^2)d + (p - q)d$

  $\Rightarrow a = d +(p + q)d = d(1 - p - q)$

  $\therefore t_{p + q} = a + (p + q - 1)d = 0$.
\item Let $r$ be the c.r.\ then $r = \frac{b}{a}$, and $c = ar^{n - 1}$, where $n$ is the number of terms.

  Sum of the series is $S = \frac{a(r^n - 1)}{r - 1} = \frac{c\frac{b}{a} - a}{\frac{b}{a} - 1} = \frac{bc -
  a^2}{b - a}$.
\stopitemize
