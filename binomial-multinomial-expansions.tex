% -*- mode: context; -*-
\chapter{Binomials, Multinomials and Expansions}
An algebraic expression containing one term is called {\it monomial}, two terms is callled {\it binomial} and more than two
is called is called {\it multinomial}. Examples of a monomial expressions are $2x, 4y$, examples of binomial expressions are $a
+ b, x^2 + y^2, x^3 + y^3, x + \frac{1}{y}$ and exaamples of multinomial expressions are $1 + x + x^2, a^2 + 2a + b^2, a^3 + 3a^2b
+ 3ab^2 + b^3$.

\section{Binomial Theorem}
Newton gave binomial theorem, by which we can expand any opwer of a binomial expression as a series. First we consider only
positive integral values of exponent. For positive integral exponent the formula has the following form:
\startformula (a + x)^n = {}^nC_0a^nx^0 + {}^nC_1a^{n - 1}x^1 + {}^nC_2a^{n - 2}x^2 + \ldots + {}^nC_na^0x^n\stopformula

\subsection{Proof by Mathematical Induction}
Let \startformula P(n) = (a + x)^n = {}^nC_0a^nx^0 + {}^nC_1a^{n - 1}x^1 + {}^nC_2a^{n - 2}x^2 + \ldots + {}^nC_na^0x^n\stopformula

When $n = 1, P(1) = a + x ={}^1C_0a + {}^1C_1x$. When $n = 2, P(2) = a^2 + 2ax + x^2 = {}^2C_0a^2 + {}^2C_1ax + {}^2C_2x^2$. Thus
we see that $P(n)$ holds good for $n = 1$ and $n = 2$. Let $P(n)$ is true for $n = k$ i.e.
\startformula P(k) = (a + x)^k = {}^kC_0a^kx^0 + {}^kC_1a^{k - 1}x^1 + {}^kC_2a^{k - 2}x^2 + \ldots + {}^kC_ka^0x^k\stopformula
Multiplying both sides with $(a + x)$
\startformula P(k + 1) = (a + x)^{k + 1} = {}^kC_0a^{k + 1}x^0 + {}^kC_1a^kx + {}^kC_2a^{k - 1}x^2 + \ldots + {}^kC_kax^k +\stopformula \startformula {}^kC_0a^kx +
{}^kC_1a^{k - 1}x^2 + {}^kC_2a^{k - 2}x^3 + \ldots + {}^kC_kx^{k + 1}\stopformula

Combining terms with equal powers of $a$ and $x$, using the formula ${}^nC_r + {}^nC_{r + 1} = {}^{n + 1}C_{r + 1}$ and rewriting
${}^kC_0$ and ${}^kC_k$ as ${}^{k + 1}C_0$ and ${}^{k + 1}C_{k + 1}$, we get
\startformula P(k + 1) = {}^{k + 1}C_0a^{k + 1}x^0 + {}^{k + 1}C_1a^{k}x^1 + {}^{k + 1}C_2a^{k - 1}x^2 + \cdots + {}^{k + 1}C_{k + 1}a^0x^{k + 1}\stopformula
Thus, we see that $P(n)$ holds good for $n = k + 1$ and we have proven binomial theorem by mathemtical induction.

\subsection{Proof by Combination}
We know that $(a + x)^n = (a + x)(a + x)\cdots[n\;\text{factors}]$. If see only $a$, then we see that $a^n$ exists and hence, $a^n$
is a term in the final product. This is the term $a^n$, which can be written as ${}^nC_0a^nx^0$. If we take the letter $a, n - 1$
times and $x$ once then we observe ttat $x$ can be taken in ${}^nC_1$ ways. Thus, we can say that the term in final product is
${}^nC_1a^{n - 1}x$. Similarly, if we choose $a, n - 2$ times and $x$ twice then the term will be ${}^nC_2a^{n - 2}x^2$. Finally,
like $a^n, x^n$ will exist and can be written as ${}^nC_nx^n$ for consistency. Thus, we have proven binomial theorem by
combination.

\section{Special Forms of Binomial Expansion}
We have
\placeformula[eq:be1]
\startformula
(a + x)^n = {}^nC_0a^nx^0 + {}^nC_1a^{n - 1}x^1 + {}^nC_2a^{n - 2}x^2 + \ldots + {}^nC_na^0x^n
\stopformula

\startitemize[n]
\item Putting $-x$ instead of $x$
  \startformula (a - x)^n = {}^nC_0a^nx^0 - {}^nC_1a^{n - 1}x^1 + {}^nC_2a^{n - 2}x^2 - \ldots + (-1)^n{}^nC_na^0x^n\stopformula
\item Putting $a = 1$ in \in{Eq.}[eq:be1]
  \startformula (1 + x)^n = {}^nC_0 + {}^nC_1x + {}^nC_2x^2 + \ldots + {}^nC_nx^n\stopformula
\item Putting $x = -x$ in above equation
  \startformula (1 - x)^n = {}^nC_0 - {}^nC_1x + {}^nC_2x^2 - \ldots + (-1)^n{}^nC_nx^n\stopformula
\stopitemize

\section{General Term of a Binomial Expansion}
We see that first term is $t_1 = {}^nC_0a^nx^0$, second term is $t_2 = {}^nC_1a^{n - 1}x^1$ so general term will be \startformula t_r =
{}^nC_{r - 1}a^{n - r + 1}x^{r - 1}\stopformula

\section{Middle Term of a Binomial Expansion}
When $n$ is an even number, i.e. $n = 2m,\;m\in\mathbb{P}$. Middle term will be $m + 1$th term i.e. \startformula t_{m + 1} =
{}^nC_ma^mmx^m.\stopformula When $n$ an odd number, i.e. $n = 2m + 1\;m\in\mathbb{N}$. There will be two middle terms i.e. $m + 1$th and $m +
2$th terms will be middle terms. So \startformula t_{m + 1} = {}^nC_ma^{m +1}x^m, t_{m + 2} = {}^nC_{m + 1}a^{m}x^{m + 1}\stopformula

The middle terms have the largest coefficient. In case of two middle terms the coefficients of both the middle terms are equal.

\section{Equidistant Coefficients}
Binomial coefficients equidistant from start and end are equal. Coefficients of first term from start and end are ${}^nC_0$ and
${}^nC_n$ which are equal. Coefficients of second term from start and end are ${}^nC_1$ and ${}^nC_{n-1}$ which are
equal. Similarly, coefficient of $r$th term from start is ${}^nC_{r - 1}$ and from end is ${}^nC_{n - r +1}$. From combinations we
know that ${}^nC_{r - 1} = {}^nC_{n - r + 1}$. Thus, it is prove that coefficients of terms equidistant from start and end are
equal.

\section{Properties of Binomial Coefficients}
We have proven earlier that \startformula (1 + x)^n = {}^nC_0 + {}^nC_1x + {}^nC_2x^2 + \cdots + {}^nC_nx^n.\stopformula Putting $x = 1$, we get \startformula 2^n =
{}^nC_0 + {}^nC_1 + {}^nC_2 + \cdots + {}^nC_n.\stopformula Putting $x = -1$, we get \startformula 0 = {}^nC_0 - {}^nC_1 + {}^nC_2 - \cdots
+(-1)^n{}^nC_n.\stopformula Adding the last two, we have \startformula 2^n = 2[{}^nC_0 + {}^nC_2 + {}^nC_4 + \cdots]\stopformula \startformula 2^{n - 1} \ {}^nC_0 + {}^nC_1 +
{}^nC_2 + \cdots\stopformula Subtracting, we get \startformula 2^{n - 1} = {}^nC_1 + {}^nC_3 + {}^nC_5 + \cdots\stopformula

\section{Multinomial Theorem}
Consider the multinomila $(x_1 + x_2 + \cdots + x_n)^p$, where $n$ and $p$ are positive integers. The general term of such a
multinomial is givenby \startformula \frac{p!}{p_1!p_2!\ldots p_n!}x_1^{p_1}x_2^{p_2}\cdots x_n^{p_n}\stopformula such that $p_1, p_2, \ldots, p_n$ are
non-negative integers and $p_1 + p_2 + \cdots + p_n = p.$

We can find the general term using the binomial theorem itself. General term in the expansion $[x_1 + (x_2 + x_3 + \cdots +
  x_n)]^n$ is \startformula \frac{n!}{p!(n - p_1)!}x_1^{p_1}(x_2 + x_3 + \cdots + x_n)^{n- p_1}.\stopformula General term in expansion of $(x_2 + x_3 +
\cdots + x_n)^{n - p_1}$ is \startformula \frac{(n - p_1)!}{p_2!(n - p_1 - p_2)!}x_2^{p_2}(x_3 + x_4 + \cdots + x_n)^{n - p_1 -p_2}.\stopformula
Proceding in this manner we obtain the general term given above.

\subsection{Som Results on Multinomial Expansions}
\startitemize[n]
\item No.\ of terms in the multinomial $(x_1 + x_2 + \cdots + x_n)^p$ is number of non-negative integral solution of the equation
  $p_1 + p_2 + \cdots + p_n = p$ i.e. ${}^{n + p - 1}C_{p}$ or ${}^{n + p - 1}C_{n - 1}$.
\item Largest coeff. in $(x_1 + x_2 + \cdots + x_n)^p$ is $\frac{n!}{(q!)^{n - r}[(q + 1)!]^r}$, where $q$ is the quotient and $r$
  is the remainder of $p/n$.
\item Coefficient of $x^r$ in $(a_0 + a_1x + a_2x^2 + \cdots + a_nx^n)^p$ is $\sum\frac{n!}{p_0!p_1!p_2!\ldots
  p_n!}a_0^{p_0}a_1^{p_1}a_n^{p_n}$ where $p_0, p_1, \cdots, p_n$ are non-negative integers satisfying the equation $p_0 + p_1 +
  \ldots + p_n = n$ and $p_1 + 2p_2 + \cdots + np_n = r.$
\stopitemize

\section{Binomial Theorem for Any Index}
\subsection{Fractional Index}
Let $f(m) = (1 + x)^m = 1 + mx + \frac{m(m- 1)}{1.2}x^2 + \frac{m(m - 1)(m - 2)}{1.2.3}x^3 + \cdots$, where $m\in R$ then, $f(n) =
(1 + x)^n = 1 + nx + \frac{n(n - 1)}{1.2}x^2 + \frac{n(n - 1)(n - 2)}{1.2.3}x^3 + \cdots$

\startformula f(m)f(n) = (1 + x)^{m + n} = f(m + n)\stopformula
\startformula f(m)f(n)\ldots\text{~to~}k\text{~factos~}=f(m + n + \ldots)\text{~to~}k\text{~terms~}\stopformula

Let $m, n, \ldots$ each equal to $\frac{j}{k}$

\startformula \Rightarrow \left[f\left(\frac{j}{k}\right)\right]^k = f(j)\stopformula

but $j$ is a positive integer, $f(j) = (1 + x)^j$

\startformula \therefore (1 + x)^{\frac{j}{k}} = f\left(\frac{j}{k}\right)\stopformula

\startformula \therefore (1 + x)^{\tfrac{j}{k}} = 1 + \frac{j}{k}x + \frac{\frac{j}{k}\left(\frac{j}{k} - 1\right)}{1.2}x^2 + \ldots\stopformula

And thus, we have proven binomial theorem for fractional index.

\subsection{Negative Index}
We can write \startformula f(n)f(-n) = f(0) = 1\stopformula
\startformula \Rightarrow f(-n) = \frac{1}{f(n)} = (1 + x)^{-n} = 1 - nx + \frac{n(n - 1)}{1.2}x^2 - \ldots\stopformula

\section{General Term in Binomial Theorem for Any Index}
General term is given by \startformula \frac{n.(n - 1)\ldots(n - r + 1)}{r!}x^r\stopformula

The above expansion does not hold true when $|x| > 1$ which can be quickly proved by making $r$ arbitrarily large. For example, $(1
- x)^{-1} = 1 + x + x^2 + x^3 + \ldots$. However, if we put $x = 2$, then we have $(-1)^{-1} = 1 + 2 + 2^2 + \ldots$ which shows
that when $x > 1$ the above formula does not hold true.

From G.P. we know that $1 + x + x^2 + \ldots$ for $r$ terms is \startformula \frac{1}{1 - x} - \frac{x^r}{1 - x}\stopformula

Thus, if $r$ is very large and $|x| < 1$, we can ignore the second fraction but not when $|x| > 1$.

\section{General Term for Negative Index}
The $r + 1$th term is given by \startformula \frac{-n(-n - 1)\ldots(-n - r + 1)}{r!}(-x)^r\stopformula
\startformula = \frac{n(n + 1)\ldots(n + r - 1)}{r!}x^r\stopformula

\section{Exponential and Logrithmic Series Expansions}
Following expansions are useful for solving problem related to exponential and logarithmic series:
\startitemize[n]
\item $e^x = 1 + \frac{x}{1!} + \frac{x^2}{2!} + \frac{x^3}{3!} + \ldots$ to $\infty$, where $x$ is any number. $e$ lies between
  $2$ and $3$.
\item If $a > 0, a^x = e^{x\log_ea} = 1 + \frac{x\log_ea}{1!} + \frac{(x\log_ea)^2}{2!} + \ldots$
\item $\log_e(1 + x) = x - \frac{x^2}{2} + \frac{x^3}{3} - \frac{x^4}{4} + \ldots$ to $\infty$ where $-1< x\leq 1$.
\stopitemize
\pagebreak
\section{Problems}

\startitemize[n, 1*broad]
\item Expand $\left(x + \frac{1}{x}\right)^5$.
\item Use the bonimial theorem to find the exact value of $(10.1)^5$.
\item Simplify $(x + \sqrt{x - 1})^6 + (x - \sqrt{x - 1})^6$.
\item If $A$ be the sum of odd terms and $B$ be the sum of even terms in the expansion of $(x + a)^n$, prove that $A^2 - B^2 = (x^2
  - a^2)^n$.
\item If $n$ is a positive integer, prove that the integral part of $(7 + 4\sqrt{3})^n$ is an odd number.
\item If $(7 + 4\sqrt{3})^n = \alpha + \beta$, where $\alpha$ is a positive integer and $\beta$ is a proper fraction, then prove
  that $(1 - \beta)(\alpha + \beta) = 1$.
\item Find the coefficient of $\frac{1}{y^2}$ in $\left(y + \frac{c^3}{y^2}\right)^{10}$.
\item Find the coefficient of $x^9$ in $(1 + 3x + 3x^2 + x^3)^{15}$.
\item Find the term independent of $x$ in $\left(\frac{3}{2}x^2 - \frac{1}{3x}\right)^9$.
\item Find the term independent of $x$ in $(1 + x)^m\left(x + \frac{1}{x}\right)^n$.
\item Find the coefficient of $x^{-1}$ in $(1 + 3x^2 + x^4)\left(1 + \frac{1}{x}\right)^8$.
\item If $a_r$ denotes the coefficient of $x^r$ in the expansion $(1 - x)^{2n - 1}$, then prove that $a_{r - 1} + a_{2n - r} = 0$.
\item Find the vallue of $k$ so that the term independent of $x$ in $\left(\sqrt{x} + \frac{k}{x^2}\right)^{10}$ is $405$.
\item Show that there will be no term containing $x^{2r}$ in the expansion $(x + x^{-2})^{n - 3}$, if $n - 2r$ is positive but not
  a multiple of $3$.
\item Show that there will be a term independent of $x$ in the expansion $(x^a + x^{-b})^n$, only if $an$ is a multiple of $a + b$.
\item Expand $\left(x + \frac{1}{x}\right)^7$ using binomial theorem.
\item Use binomial theorem to expand $\left(\frac{2x}{3} - \frac{3}{2x}\right)^6$.
\item If $(1 + ax)^n = 1 + 8x + 24x^2 + \ldots$, find $a$ and $n$.
\item Find the $7$th term in the expansion of $\left(\frac{4x}{5} - \frac{5}{2x}\right)^9$.
\item Find the value of $(\sqrt{2} + 1)^6 + (\sqrt{2} - 1)^6$.
\item If $A$ be the sum of the odd terms and $B$ be the sum of the even terms in the expansion $(x + a)^n$, show that $4AB = (x +
  a)^{2n} - (x - a)^{2n}$.
\item If $n$ be a positive integer, prove that the integral part of $(5 + 2\sqrt{6})^n$ is an odd integer.
\item If $(3 + \sqrt{8})^n = \alpha + \beta$, where $\alpha, n$ are positive integers and $\beta$ is a proper fraction, then prove
  that $(1 - \beta)(\alpha + \beta) = 1$.
\item Find the coefficient of $x$ in the expansion of $\left(2x - \frac{3}{x}\right)^9$.
\item Find the coefficient of $x^7$ in the expansion of $(3x^2 + (5x)^{-1})^{11}$.
\item Find the coefficient of $x^9$ in the expansion of $(2x^2 - x^{-1})^{20}$.
\item Find the coefficient of $x^{24}$ in the expansion of $(x^2 + 3ax^{-1})^{15}$.
\item Find the coefficient of $x^9$ in the expansion of $(x^2 - (3x)^{-1})^9$.
\item Find the coefficient of $x^{-7}$ in the expansion of $\left(2x - \frac{1}{3x^2}\right)^{11}$.
\item Find the coefficient of $x^7$ in the expansion of $\left(ax^2 + \frac{1}{bx}\right)^{11}$ and the coefficient of $x^{-7}$ in
  the expansion of $\left(ax - \frac{1}{bx}\right)^{11}$. Also, find the relation between $a$ and $b$ so that the coefficients are
  equal.
\item If $x^p$ occurs in the expansion of $\left(x^2 + \frac{1}{x}\right)^{2n}$, show that its coefficient is
  $\frac{2n!}{\left(\frac{4n - p}{3}\right)!\left(\frac{2n + p}{3}\right)!}$.
\item Find the term independent of $x$ in the following binomial expansions:
  \startitemize[i]
  \item $\left(x + \frac{1}{x}\right)^{2n}$,
  \item $\left(2x^2 + \frac{1}{x}\right)^{15}$,
  \item $\left(\sqrt{\frac{x}{3}} + \frac{3}{2x^2}\right)^{10}$,
  \item $\left(2x^2 - \frac{1}{x}\right)^{12}$,
  \item $\left(2x^2 - \frac{3}{x^3}\right)^{25}$,
  \item $\left(x^3 - \frac{3}{x^2}\right)^{15}$,
  \item $\left(x^2 - \frac{3}{x^3}\right)^{10}$, and
  \item $\left(\frac{1}{2}x^{1/3} + x^{-1/3}\right)^8$.
  \stopitemize
\item If there is a term independent of $x$ in $\left(x + \frac{1}{x^2}\right)^n$, show that it is equal to
  $\frac{n!}{\left(\frac{n}{3}\right)!\left(\frac{2n}{3}\right)!}$
\item Prove that in the expansion of $(1 + x)^{m + n}$, coefficients of $x^m$ and $x^n$ are equal, $\forall\;m,n > 0,m,
  n\in\mathbb{N}$.
\item Give that the $4$th term in the expansion of $\left(px + \frac{1}{x}\right)^n$ is $\frac{5}{2}$. Find $n$ and $p$.
\item Find the middle term in the expansion of $\left(x - \frac{1}{2x}\right)^{12}$.
\item Find the middle terms in the expansion of $\left(2x^2 - \frac{1}{x}\right)^7$.
\item Prove that the middle term in the expansion of $\left(x + \frac{1}{x}\right)^{2n}$ is $\frac{1.3.5\ldots(2n - 1)}{n!}2^n$.
\item Show that the coefficient of the middle term in $(1 + x)^{2n}$ is equal to the sum of coefficients of the two middle terms in
  $(1 + x)^{2n - 1}$.
\item Find the middle term in the expansions of;
  \startitemize[i]
  \item $\left(\frac{2x}{3} - \frac{3y}{2}\right)^{20}$,
  \item $\left(\frac{2x}{3} - \frac{3}{2x}\right)^6$,
  \item $\left(\frac{x}{y} - \frac{y}{x}\right)^7$,
  \item $(1 + x)^{2n}$, and
  \item $(1 - 2x + x^2)^n$.
  \stopitemize
\item Find the general and middle term of the expansion $\left(\frac{x}{y} + \frac{y}{x}\right)^{2n + 1}$; $n$ being a positive
  integer show that there is no term free of $x$ and $y$.
\item Show that the middle term in the expansion of $\left(x - \frac{1}{x}\right)^{2n}$ is $\frac{1.3.5\ldots (2n -1)}{n!}.(-2)^n$.
\item If in the expansion of $(1 + x)^{43}$, the coefficient of $(2r + 1)$th term is equal to the coefficient of $(r + 2)$th term,
  find $r$.
\item If the $r$th term in the expansion of $(1 + x)^{20}$ has coefficient equal to that of the $(r + 4)$th term, find $r$.
\item If the coefficient of $(2r + 4)$th term and $(r - 2)$th term in the expansion of $(1 + x)^{18}$ are equal, find $r$.
\item If the coefficient of $(2r + 5)$th term and $(r - 6)$th term in the expansion of $(1 + x)^{39}$ are equal, fin $C_{12}^^r$.
\item Given positive integers $r>1, n>2, n$ being even and the coefficient of $3r$th term and $(r + 2)$th term in the expansion of
  $(1 + x)^{2n}$ are equal, find $r$.
\item If the coefficient of $(p + 1)$th term in the expansion of $(1 + x)^{2n}$ be equal to that of the $(p + 3)$th term, show that
  $p = n - 1$.
\item Show that the coefficient of $(r + 1)$th term in the expansion of $(1 + x)^{n + 1}$ is equal to the sum of the coefficients
  of the $r$th and $(r + 1)$th term in the expansion of $(1 + x)^n$.
\item Find the greatest term in the expansion of $\left(7 - \frac{10}{3}\right)^{11}$.
\item Show that if the greatest term in the expansion of $(1 + x)^{2n}$ has also the greatest coefficient $x$ lies between
  $\frac{n}{n + 1}$ and $\frac{n + 1}{n}$.
\item Find the greatest terms in the expansions of:
  \startitemize[i]
  \item $\left(2 + \frac{9}{5}\right)^{10}$,
  \item $(4 - 2)^7$, and
  \item $(5 + 2)^{13}$.
  \stopitemize
\item Find the limits between which $x$ must lie in order that the greatest term in the expansion of $(1 + x)^{30}$ may have the
  greatest coefficient.
\item If $n\in\mathbb{P}$, then prove that $6^{2n} - 35n - 1$ is divisible by $1225$.
\item Show that $2^{4n} - 2^n(7n + 1)$ is some multuple of the square of $14$, where $n\in\mathbb{P}$.
\item Show that $3^{4n + 1} - 16n - 3$ is divisible by $256$, if $n\in\mathbb{P}$.
\item If $n\in\mathbb{P}$, show that
  \startitemize[i]
  \item $4^n - 3n - 1$ is divisible by $9$,
  \item $2^{5n} - 31n - 1$ is divisible by $961$,
  \item $3^{2n + 2} - 8n  - 9$ is divisible by $64$,
  \item $2^{5n + 5} - 31n - 32$ is divisible by $961$ if $n > 1$, and
  \item $3^{2n} - 32n^2 + 24n - 1$ is divisible by $512$ if $n > 2$.
  \stopitemize
\item If three consecutive coefficients in the expansion of $(1 + x)^n$ be $165, 330$ and $462$, find $n$ and $r$.
\item If $a_1, a_2, a_3$ and $a_4$ be any four consecutive coefficients in the expansion of $(1 + x)^n$, prove that $\frac{a_1}{a_1
  + a_2} + \frac{a_3}{a_3 + a_4} = \frac{2a_2}{a_2 + a_3}$.
\item If $2$nd, $3$rd and $4$th terms in the expansion of $(x + y)^n$ be $240, 720$ and $1080$ respectively, find $x, y$ and $n$.
\item If $a, b, c$ be thre three consecutive terms in the expansion of some power of $(1 + x)$, prove that the exponent is
  $\frac{2ac + ab + bc}{b^2 - ac}$.
\item If $14$the, $15$th and $16$th term in the expansion of $(1 + x)^n$ are in A.P., find $n$.
\item If three consecutive terms in the expansion of $(1 + x)^n$ be $56, 70$ and $56$, find $n$ and the position of the
  coefficients.
\item If three successive coefficients in the expansion of $(1 + x)^n$ be $220, 495$ and $792$, find $n$.
\item If $3$rd, $4$th and $5$th terms in the expansion of $(a + x)^n$ be $84, 280$ and $560$, find $a, x$ and $n$.
\item If $6$th, $7$th and $8$th terms in the expansion of $(x + y)^n$ be $112, 7$ and $\frac{1}{4}$, find $x, y$ and $n$.
\item If $a, b, c$ and $d$ be the $6$th, $7$th, $8$th and $9$th terms respectively in any binomial expansion, prove that $\frac{b^2
  - ac}{c^2 - bd} = \frac{4a}{3c}$.
\item If the four consecutive coefficients in any binomial expansion be $a, b, c,$ and $d$, then prove that (a) $\frac{a + b}{a},
  \frac{b + c}{b}, \frac{c + d}{c}$ are in H.P., and (b) $(bc + ad)(b - c) = 2(ac^2 - b^2d)$.
\item The coefficients of the $5$th, $6$th and $7$th terms in the expansion of $(1 + x)^n$ are in A.P. Find the value of $n$.
\item If the coefficients of the $2$nd, $3$rd and $4$th terms in the expansion of $(1 + x)^{2n}$ are in A.P.,\ show that $2n^2 -9n
  + 7 = 0$.
\item If the coefficients of $r$th, $(r + 1)$th and $(r + 2)$th terms in the expansion of $(1 + x)^n$ are in A.P.\ show that $n^2 -
  n(4r + 1) + 4r^2 - 2 = 0$.
\item If the coefficients of three consecutive terms in the expansion of $(1 + x)^n$ are in the ratio $182:84:30$, prove that $n =
  18$.
\stopitemize

\noindent If $(1 + x)^n = C_0 + C_1x + C_2x^2 + \cdots + C_nx^n$, prove that

\startitemize[n, continue, 1*broad]
\item $C_1 + 2.C_2 + 3.C_3 + \cdots + n.C_n = n.2^{n - 1}$.
\item $C_0 + 2.C_1 + 3.C_2 + \cdots + (n + 1).C_n = (n + 2)2^{n - 1}$.
\item $C_0 + 3.C_1 + 5.C_2 + \cdots + (2n + 1).C_n = (n + 1)2^n$.
\item $C_1 - 2.C_2 + 3.C_3 - 4.C_4 + \cdots + (-1)^{n- 1}n.C_n = 0$.
\item $C_0 + \frac{C_1}{2} + \frac{C_3}{3} + \cdots + \frac{C_n}{n + 1} = \frac{2^{n+ 1} - 1}{n + 1}$.
\item $C_0 - \frac{C_1}{2} + \frac{C_3}{3} - \cdots + (-1)^n\frac{C_n}{n + 1} = \frac{1}{n + 1}$.
\item $\frac{C_1}{2}+ \frac{C_3}{4} + \frac{C_5}{6} + \cdots = \frac{2^n - 1}{n + 1}$.
\item $2.C_0 + 2^2.\frac{C_1}{2} + 2^3.\frac{C_2}{3} + \cdots + 2^{n + 1}.\frac{C_n}{n + 1} = \frac{3^{n + 1} - 1}{n + 1}$.
\item $C_0.C_r + C_1.C_{r + 1} + \cdots + C_{n-r}.C_n = \frac{(2n)!}{(n + r)!(n - r)!}$.
\item $C_0^2 + C_1^2 + C_2^2 + \cdots + C_n^2 = \frac{(2n)!}{n!n!}$.
\item $\frac{C_1}{C_0} + 2.\frac{C_2}{C_1} + 3.\frac{C_3}{C_2} + \cdots + n.\frac{C_n}{C_{n - 1}} = \frac{n(n + 1)}{2}$.
\item $(1 + C_1 + C_2 + \cdots + C_n)^2 = 1 + C_1^^{2n} + C_2^^{2n} + \cdots + C_{2n}^^{2n}$.
\item $(1 + C_1 + C_2 + \cdots + C_n)^5 = 1 + C_1^^{5n} + C_2^^{5n} + \cdots + C_{5n}^^{5n}$.
\item $C_0 + 5.C_1 + 9.C_2 + \cdots + (4n + 1).C_n = (2n + 1)2^n$.
\item $1 - (1 + x)C_1 + (1 + 2x)C_2 - (1 + 3x)C_3 + \cdots = 0$.
\item $3.C_1 + 7.C_2 + 11.C_3 + \cdots + (4n - 1)C_n = (2n - 1)2^{n + 1}$.
\item $C_0 + \frac{C_2}{3} + \frac{C_4}{5} + \cdots = \frac{2^n}{n + 1}$.
\item $C_0^^nC_1^^{n + 1} + C_1^^{n}C_2^^{n + 1} + \cdots + C_n^^{n}C_{n + 1}^^{n + 1} = \frac{(2n + 1)!}{(n
    + 1)!n!}$.
\item $C_0 - 2.C_1 + 3.C_2 - \cdots + (-1)^n(n + 1)C_n = 0$.
\item $C_0 -3.C_1 + 5.C_2 - \cdots + (-1)^n(2n + 1)C_n = 0$.
\item $a - (a - 1)C_1 + (a - 2)C_2 - (a - 3)C_3 + \cdots + (-1)^n(a - n)C_n = 0$.
\item $1^2.C_1 + 2^2.C_2 + 3^2C_3 + \cdots + n^2.C_n = n(n + 1)2^{n - 2}$.
\item If $n>3$ and $n\in\mathbb{N}$, prove that $C_0.abc - C_1(a - 1)(b - 1)(c - 1) + C_2(a - 2)(b - 2)(c -2)- \cdots +
  (-1)^n.C_n(a - n)(b - n)(c - n) = 0$
\item $C_0 - 2^2.C_1 + 3^2.C_2 - \cdots + (-1)^n(n + 1)^2C_n = 0,\;n>2$.
\item Prove that $\displaystyle\sum_{r=0}^nr^2.C_rp^rq^{n- r} = npq + n^2p^2$ if $p + q = 1$.
\item $2.C_0 + \frac{2^2}{2}.C_1 + \frac{2^3}{3}.C_2 + \cdots + \frac{2^{11}}{11}.C_{11} = \frac{3^{11} -
    1}{11}$.
\item $\frac{2^2}{1.2}C_0 + \frac{2^3}{2.3}C_2 + \frac{2^4}{3.4}C_2 + \cdots + \frac{2^{n + 2}}{(n + 1)(n +
    2)}C_n = \frac{3^{n + 2} - 2n -5}{(n + 1)(n + 2)}$.
\item $C_1 - \frac{1}{2}C_2 + \frac{1}{3}C_3 - \cdots + (-1)^n\frac{1}{n}C_n = 1 + \frac{1}{2} + \frac{1}{3} + \cdots +
  \frac{1}{n}$.
\item $\frac{C_0}{1} - \frac{C_1}{5} + \frac{C_2}{9} - \cdots + (-1)^n\frac{C_n}{4n + 1} = \frac{n.4^n}{1.5.9\ldots (4n + 1)}$.
\item $\frac{C_0}{n} - \frac{C_1}{n+ 1} + \frac{C_2}{n + 2} - \cdots + (-1)^n\frac{C_n}{2n} = \frac{n!(n - 1)!}{(2n)!}$.
\item $\frac{C_0}{n(n + 1)} - \frac{C_1}{(n + 1)(n+ 2)} + \frac{C_2}{(n + 2)(n + 3)} - \cdots + (-1)^n\frac{C_n}{2n(2n + 1)} =
  \frac{1}{(2n + 1)}.\frac{1}{{}^{2n}C_{n - 1}}$.
\item $\frac{C_0}{k} - \frac{C_1}{k + 1} + \frac{C_2}{k + 2} - \cdots + (-1)^n\frac{C_n}{k + n} = \frac{n!}{k(k + 1)\ldots(k + n)}$.
\item Show that $C_0^2 - C_1^2 + C_2^2 - \cdots + (-1)^n.C_n^2 = 0$ or $(-1)^{n/2}.\frac{n!}{\left(\frac{n!}{2}\right)^2}$
  according as $n$ is odd or even.
\item Show that $C_r{}^nC_0 + {}^mC_{r-1}.{}^nC_1 + {}^mC_{r-2}.{}^nC_2 + \cdots + {}^mC_0.{}^nC_r = {}^{m + n}C_r$, where $m,
  n, r$ are positive integers and $r<m,\;r<n$.
\item ${}^{2n}C_0^2 - {}^{2n}C_1^2 + {}^{2n}C_2^2 - \cdots + (-1)^{2n}.{}^{2n}C_{2n}^2 = (-1)^n.{}^{2n}C_n$.
\item Show that $C_1^2 + 2.C_2^2 + 3.C_3^2 + \cdots + n.C_n^2 = \frac{(2n - 1)!}{[(n - 1)!]^2}$.
\item Show that $C_0^2 + \frac{C_1^2}{2} + \frac{C_2^2}{3} + \cdots + \frac{C_n^2}{n + 1} = \frac{(2n + 1)!}{[(n + 1)!]^2}$.
\item $C_0 - 2^2C_1 + 3^2C_2 - \cdots + (-1)^n(n + 1)^2C_n = 0,\;n>2$.
\item $\frac{C_0}{1.2} - \frac{C_1}{2.3} + \frac{C_2}{3.4} - \cdots + (-1)^n\frac{C_n}{(n + 1)(n + 2)} = \frac{1}{n + 2}$
\item $\frac{C_0}{2} - \frac{C_1}{3} + \frac{C_2}{4} - \cdots + (-1)^n\frac{C_n}{n + 2} = \frac{1}{(n + 1)(n + 2)}$.
\item $\frac{C_0}{3} - \frac{C_1}{4} + \frac{C_2}{4} - \cdots + (-1)^n\frac{C_n}{n + 3} = \frac{2}{(n + 1)(n + 2)(n + 3)}$.
\item $3.C_0 + 3^2\frac{C_1}{2} + 3^3\frac{C_2}{3} + \cdots + 3^{n + 1}\frac{C_n}{n+ 1} = \frac{4^{n + 1} - 1}{n + 1}$.
\item If $n$ is a positive integer in $(1 + x)^n$, show that $2.\frac{\left(\frac{n!}{2}\right)^2}{n!}[C_0^2 - 2.C_1^2 +
  3.C_2^2 - \cdots + $ $(-1)^n.(n + 1)C_n^2] = (-1)^{n/2}(2 + n)$.
\item Show that $\displaystyle\sum_{0\leq i\leq n}\sum_{i<j\leq n}C_iC_j = 2^{2n - 1} - \frac{(2n)!}{2(n!)^2}$.
\item Show that $\displaystyle_{r=0}^nC_r^3$ is equal to the coefficient of $x^ny^n$ in the expansion of $[(1 + x)(1 + y)(x +
  y)]^n$.
\item Prove that the sum of coefficients in the expansion $(1 + x -3x^2)^{2163}$ is $-1$.
\item If $(1 + x - 2x^2)^6 = 1 + a_1x + a_2x^2 + \cdots + a_{12}x^{12}$ show that $1 + a_3 + a_6 + a_9 +
  \cdots + a_{12} = 31$.
\item Find the sum of the rational terms in the expansion of $(2 + \sqrt[5]{3})^{10}$.
\item Find the fractional pert of $\frac{2^{4n}}{15}$.
\item Show that the integer just above $(\sqrt{3} + 1)^{2n}$ is divisible by $2^{n + 1},\;\forall\;n\in\mathbb{N}$.
\item Let $R = (5\sqrt{5} + 11)^{2n + 1}$ and $f = R - [R]$, where $[\;]$ denotes the greatest integer function. Prove that $Rf =
  4^{2n + 1}$.
\item Show that $(101)^{50} > (100)^{50} + (99)^{50}$.
\item Find the sum of the series $\displaystyle\sum_{r=0}^n(-1)^r.{}^nC_r\left[\frac{1}{2^r} + \frac{3^r}{2^{2r}} +
  \frac{7^r}{2^{3r}} + \cdots \text{~to~}m\text{~terms}\right]$.
\item Find the last digit of the number $(32)^{32}$.
\item Prove that $\displaystyle\sum_{r=0}^k(-3)^{r - 1}.{}^{3n}C_{2n- 1} = 0$, where $k = \frac{3n}{2}$ and $n$ is a positive even
  number.
\item If $t_0, t_1, t_2, t_3, \ldots$ be ther terms of expansion $(a + x)^n$, prove that $(t_0 - t_2 + t_4 - \cdots)^2 + (t_1 - t_3
  + t_5 - \cdots)^2 = (a^2 + x^2)^n$.
\stopitemize

\noindent If $(1 + x + x^2)^n = a_0 + a_1x + a_2x^2 + \cdots + a_{2n}x^{2n}$, show that

\startitemize[n, continue, 1*broad]
\item $a_0 + a_1 + a_2 + \cdots + a_{2n} = 3^n$.
\item $a_0 - a_1 + a_2 - \cdots + a_{2n} = 1$.
\item $a_0 + a_3 + a_6 + \cdots = 3^{2n - 1}$.
\item If $S_n = 1 + q + q^2 + \cdots + q^n$ and $S_n' = 1 + \left(\frac{q + 1}{2}\right) + \left(\frac{q + 1}{2}\right)^2 + \cdots
  + \left(\frac{q + 1}{2}\right)^n,\;q\neq 1$, prove that $C_1^^{n + 1} + C_2^^{n + 1}.S_1 + C_3^^{n + 1}.S_2 + \cdots +
  C_{n + 1}^^{n + 1}.S_n = 2^nS_n'$.
\item Find the number of rational terms in the expansion of $(\sqrt[4]{9} + \sqrt[6]{8})^{1000}$.
\item Find the sum of rational terms in the expansion of $(\sqrt[3]{2} + \sqrt[5]{3})^{15}$.
\item Determine the values of $x$ in the expansion of$(x + x\log_{10}x)^5$ if the third term in that expansion is $1,000,000$.
\item Expand $\left(x + 1 - \frac{1}{x}\right)^3$.
\item Find the value of $x$ for which the sixth term of $\left(\sqrt{2^{\log(10 - 3^x)}} + \sqrt[5]{2^{(x - 2)\log3}}\right)^m$ is
  equal to $21$ and coefficients of second, third and fourth terms are the first, third and fifth terms of an A.P., given base of
  $\log$ is $10$.
\item Find the values of $x$ for which the sixth term of the expansion $\left[2^{\log_2\sqrt{9^{x - 1} + 7}} +
  \frac{1}{2^{\tfrac{1}{5}\log_2(3^{x -1} + 1)}}\right]^7$ is equal to $84$.
\item If $n\in\mathbb{N}$, prove that $\frac{1}{(81)^n} - \frac{10}{(81)^n}.C_1^^{2n} + \frac{10^2}{(81)^n}.C_2^^{2n} -
  \frac{10^3}{(81)^n}.C_3^^{2n} + \cdots + \frac{10^{2n}}{(81)^n} = 1$.
\item Find the value of $\displaystyle\lim_{n\to\infty}S_n = C_n - \frac{2}{3}C_{n- 1} + \left(\frac{2}{3}\right)^2C_{n - 2} -
  \cdots + (-1)^n\left(\frac{2}{3}\right)^nC_0$.
\item If $E = (6\sqrt{6} + 14)^{2n + 1}$ and $F$ be fractional part of $E$, prove that $EF= 20^{2n + 1}$.
\item Find the digits at units, tens and hundreds place in the number $(17)^{256}$.
\item Show that for $n\geq 3,\;n^{n + 1} > (n + 1)^n,\;forall\;n\in\mathbb{P}$.
\item Show that $2<\left(1 + \frac{1}{n}\right)^n < 3\;\forall\;n\in\mathbb{N}$.
\item Show that $1992^{1998} - 1955^{1998} - 1938^{1998} + 1901^{1998}$ is divisible by $1998$.
\item Show that $53^{53} - 33^{33}$ is divisible by $10$.
\item Let $k$ and $n$ be positive integers and $S_k = 1^k + 2^k + \cdots + n^k$, show that $C_1^^{m + 1}S_1
  + C_2^^{m + 1}S_2 + C_m^^{m + 1}S_m = (n + 1)^{m + 1} - n - 1$.
\item Find $\displaystyle\sum_{i=1}^k\sum_{k=1}^nC_k^^nC_i^^k,\;i\leq k$.
\item Prove that $\displaystyle\sum_{r=0}^n(-1)^r.{}^nC_r\frac{1 + r\log_e10}{(1 + \log_e10^n)^r} = 0$.
\item Find the remainder when $32^{32^{32}}$ is divided by $7$.
\item If $\displaystyle\sum_{r=0}^{2n}a_r(x - 2)^r = \sum_{r=0}^{2n}b_r(x - 3)^r$ and $a_r = 1\;\forall\;r\geq n$, then show that
  $b_n = {}^{2n + 1}C_{n + 1}$.
\item Find the coefficient of $x^{50}$ in $(1 + x)^{1000} + 2x(1 + x)^{999} + 3x^2(1 + x)^{998}+ \cdots + 1001x^{1000}$.
\item Show that $C_n^n + C_n^^{n + 1} + C_n^^{n + 2} + \cdots + C_n^^{n + 1} = C_{n + 1}^^{n + k + 1}$.
\item Find the coefficient of $x^n$ in $(1 + x + 2x^2 + 3x^3 + \cdots + nx^n)^2$.
\item Find the coefficient of $x^k,\;0\leq k\leq n$ in the expansion of $1 + (1 + x) + (1 + x)^2 + \cdots + (1 + x)^n$.
\item Find the coefficient of $x^3$ in $(x + 1)^n + (x + 1)^{n- 1}(x + 2) + (x + 1)^{n - 2}(x + 2)^2 + \cdots + (x + 2)^n$.
\item Simplify $\left(\frac{a + 1}{a^{2/3} - a^{1/3} + 1} - \frac{a - 1}{a - a^{1/2}}\right)^{10}$ into a binomial and determine
  the term independent of $a$.
\item Find the coefficient of $x^2$ in $\left(x + \frac{1}{x}\right)^{10}(1 - x + 2x^2)$.
\item Find the coefficient of $x^4$ in the expansion of $(1 + x - 2x^2)^6$.
\item Find the term independent of $x$ in $(1 + x + 2x^3)\left(\frac{3}{2}x^2 - \frac{1}{3x}\right)^9$.
\item Find the term independent of $x$ in $\left(x^2 + \frac{1}{x^3}\right)^7(2 - x)^{10}$.
\item Find the term independent of $x$ in $(1 + x + x^{-2} + x^{-3})^{10}$.
\item Let $\displaystyle(1 + x^2)^2(1 + x)^n = \sum_{k=0}^{n+4}a_kx^k$. If $a_1, a_2$ and $a_3$ are in A.P., find $n$.
\item Show that $C_1^^m + C_2^^{m + 1} + C_3^^{m + 2} + \cdots + C_n^^{m + n - 1} = C_1^^n + C_2^^{n + 1} +
  C_3^^{n + 2} + \cdots + C_n^^{m + n - 1}$.
\item If $\displaystyle n\in\mathbb{N}$ and $\displaystyle(1 + x + x^2)^n = \sum_{r=0}^{2n}a_rx^r$, prove that (a) $a_r = a_{2n - r}$, (b) $a_0
  + a_1 + a_2 + \cdots + a_{n - 1} = \frac{1}{2}(3^n - a_n)$, and (c) $(r+1)a_{r + 1} = (n - r)a_r + (2n - r + 1)a_{r - 1}$, where
  $0<r<2n$.
\item If $\displaystyle(1 - x^3)^n = \sum_{r=0}^na_r.x^r.(1 - x)^{3n- 2r}$, where $n\in\mathbb{N}$, then find $a_r$.
\item Show that the coefficient of middle term in the expansion of $(1 + x)^{2n}$ is double the coefficient of $x^n$ in the
  expansion of $(1 + x)^{2n - 1}$.
\item Find the value of $r$ for which $C_r^^{200}$ is greatest.
\item Committees of how many persons should be made out of $20$ persons so that the number of committees is maximum.
\item Show that the number of permutations which can be formed from $2n$ letters which are either \quote{a} or \quote{b} is greatest when the
  number of a's is equal to the number of b's.
\item Find the consecutive terms in the expansion of $(3 + 2x)^7$ whose coefficients are equal.
\item Find the sum of coefficients in the expansion of $(1 + 5x^2 - 7x^3)^{2000}$.
\item If the sum of the binomial coefficients in the expansion of $\left(3^{-\tfrac{x}{4}} + 3^{\tfrac{5x}{4}}\right)^n$ is $64$ and the term
  with greatest coefficient exceeds the third term by $n- 1$ and $[\alpha] = x$, where $[\alpha]$ denotes the integral part of
  $\alpha$, find the value of $\alpha$.
\item Find the sum of the coefficients in the expansion of $(5p- 4q)^n$, where $n\in\mathbb{P}$.
\item Find the sum of the coefficients in the expansion of the polynomial $(1 - 3x + x^3)^{201}.(1 + 5x - 5x^2)^{503}$.
\item If the sum of the coefficients in the expansion of $(tx^2 - 2x + 1)^n$ is equal to the sum of coefficients in the expansion
  of $(x - ty)^n$, where $n\in\mathbb{N}$, then find the value of $t$.
\item If $a_0, a_1, a_2, \ldots, a_n$ be the successive coefficient of $(1 + x)^n$, show that $(a_0 - a_2 + a_4 - \ldots)^2 + (a_1
  - a_3 + a_5 - \ldots)^2 = a_0 + a_1 + \cdots + a_n = 2^n$.
\item Find the greatest term in the expansion of $\sqrt{3}\left(1 + \frac{1}{\sqrt{3}}\right)^{20}$.
\item In the expansion of $(x + a)^{15}$, if the eleventh term is the G.M. of the eighth and twelfth terms, which term in the
  expression is the greatest?
\item if the greatest term in the expansion of $(1 + x)^{2n}$ has the greatest coefficient if and only if $x\in\left(\frac{10}{11},
  \frac{11}{10}\right)$ and the fourth term in the expansion of $\left(kx + \frac{1}{x}\right)^m$, is $\frac{m}{4}$, then find the
  value of $mk$.
\item Given that the $4$th term in the expansion of $\left(2 + \frac{3}{8}x\right)^{10}$ has the maximum numerical value, find the
  range of values of $x$ for which this would be true.
\item Show that the roots of the equation $ax^2 + 2bx + c = 0$ are real and unequal, where $a, b, c$ are three consecutive binomial
  expansion with positive integral index.
\item If $n\in\mathbb{P}$, show that $9^n + 7$ is divisible by $8$.
\item If $n\in\mathbb{P}$, show that $3^{2n+ 1} + 2^{n + 2}$ is divisible by $7$.
\item Show that no three consecutive binomial coefficients can be in G.P. or H.P.
\item Let $n$ be a positive integer and $(1 + x + x^2)^n = a_0 + a_1x + a_2x^2 + a_3x^3 + \cdots + a_{2n}x^{2n}$, show that $a_0^2
  -a_1^2 + a_2^2 - \cdots + a_{2n}^2 = a_n$.
\item Let $n$ be a positive integer and $(1 + x + x^2)^n = a_0 + a_1x + a_2x^2 + a_3x^3 + \cdots + a_{2n}x^{2n}$, show that $a_0^2
  -a_1^2 + a_2^2 - \cdots + (-1)^na_{n - 1}^2 = \frac{1}{2}a_n[1 - (-1)^na_n]$.
\item Show that $\displaystyle\sum_{0\leq i< j}\sum_{0\leq j\leq n}(C_i + C_j)^2 = (n - 1)^{2n}C_n + 2^{2n},\;(0\leq i\leq j\leq
  n)$.
\item Show that $\displaystyle\sum_{0\leq i<j}\sum_{0\leq j\leq n}(i + j)C_iC_j = n\left(2^{2n - 1} -
  \frac{1}{2}C_n^^{2n}\right)$.
\item Show that $\frac{1}{m!}C_0 + \frac{n}{(m + 1)!}C_1 + \frac{n(n - 1)}{(m + 2)!}C_2 + \cdots + \frac{n(n
  - 1)\cdots 3.2.1}{(m + n)!}C_n = \frac{(m + n + 1)(m + n + 2)\cdots (m + 2n)}{(m + n)!}$.
\item Show that $(C_0 + C_1)(C_1 + C_2)(C_2 + C_3)\cdots (C_{n- 1} + C_n) = \frac{(n + 1)^n}{n!}C_1.C_2.\ldots.C_n$.
\item If $n$ be a positive integer, prove that $\frac{1}{1!(n - 1)!} + \frac{1}{3!(n - 1)!} + \frac{1}{5!(n - 5)!} + \cdots +
  \frac{1}{(n - 1)!1!} = \frac{2^{n - 1}}{n!}$.
\item Prove that $\displaystyle\sum_{r=0}^n(-1)^r.\left(\frac{C_r^^n}{C_r^^{r + 3}}\right) = \frac{3!}{2(n + 3)}$.
\item If $(1 + x)^n = C_0 + C_1x + C_2x^2 + \cdots + C_nx^n$ show that for $m\geq 2, C_0 - C_1 + C_2 - \cdots + (-1)^{m - 1}C_{m -
  1} = (-1)^{m - 1}\frac{(n- 1)(n- 2)\ldots(n - m + 1)}{(m - 1)!}$.
\item Find the G.C.D. of $C_1^^{2n}, C_3^^{2n}, C_5^^{2n}, \ldots, C_{2n - 1}^^{2n}$.
\item Show that $\displaystyle\sum_{r=0}^nC_r^^n.\sin rx\cos(n - r)x = 2^{n- 1}\sin nx$.
\item $a.C_0 + (a - b).C_1 + (a - 2b).C_2 + \cdots + (a - nb).C_n = 2^{n - 1}(2a - nb)$.
\item $a^2.C_0 - (a - 1)^2.C_1 + (a - 2)^2.C_2 - \cdots + (-1)^n(a - n)^2.C_n = 0,\;n>3$.
\item If $a_0, a_1, a_2, \ldots, a_n$ be in an A.P., prove that $a_0 - a_1.C_1 + a_2C_2 - \cdots + (-1)^na_nC_n = 0$.
\item Show that $n>3,\;\displaystyle\sum_{r=0}^n(-1)^r(a - r)(b - r)C_r = 0$.
\item Show that $n>3,\;\displaystyle\sum_{r=0}^n(-1)^r(a - r)(b - r)(c - r)C_r = 0$.
\item Find the value of $n$ for which $\frac{C_0}{2^n} + \frac{2.C_1}{2^n} + \cdots + \frac{(n + 1)C_n}{2^n} = 16$ is true.
\item If $a_1, a_2, \ldots, a_{n + 1}$ be an A.P., prove that $\displaystyle\sum_{k=0}^na_{k + 1}C_k = 2^{n - 1}(a_1 + a_{n +
  1})$.
\item If $s = \frac{n + 1}{2}[2a + nd]$ and $S = a + (a + d)C_1 +(a+ 2d)C_2 + \cdots + (a +nd)C_n$, prove that $(n + 1)S = 2^n.s$.
\item If $(1 + x + x^2 + \cdots + x^p)^n = a_0 + a_1x + a_2x^2 + \cdots + a_{np}x^{np}$, show that $a_1 + 2a_2 + 3a_3 + \cdots +
  np.a_{np} = \frac{1}{2}np(p + 1)^n$.
\item Show that $\displaystyle\sum_{k=0}^{15}\frac{C_k^^{15}}{(k + 1)(k + 2)} = \frac{2^{17} - 18}{16.17}$.
\item Show that $\frac{C_0}{1} - \frac{C_1}{4} + \frac{C_2}{7} - \cdots + (-1)^n\frac{C_n}{3n + 1} = \frac{3^n.n!}{1.4.5\ldots(3n +
  1)}$.
\item Show that $\displaystyle\sum_{r=0}^n\frac{(-1)^rC_r}{(r + 1)(r + 2)} = \frac{1}{n + 2}$.
\item Prove that $\displaystyle\sum_{r=0}^n\frac{C_r.3^{r + 3}}{(r + 1)(r + 2)(r + 3)} = \frac{4^{n + 3} - 1 - \frac{3}{2}(n +
  3)(3n + 8)}{(n + 1)(n + 2)(n + 3)}$.
\item Prove that $\displaystyle\sum_{r=0}^n\frac{r + 2}{r + 1}C_r = \frac{2^n(n + 3) - 1}{n + 1}$.
\item Show that $\displaystyle\sum_{r=0}^n\frac{3^{r + 4}C_r}{(r + 1)(r + 2)(r + 3)(r + 4)} = \frac{1}{(n + 1)(n + 2)(n + 3)(n +
  4)}\left[4^{n + 4} - \sum_{k=0}^n{}^{n + 4}C_k3^k\right]$.
\item Show that $\displaystyle\sum_{r=0}^{n - 3}C_rC_{r + 3} = \frac{(2n)!}{(n + 3)!(n - 3)!}$.
\item Show that the sum of the product taken two at a time from $C_0, C_1, C_2, \ldots$ is $2^{2n - 1} - \frac{(2n - 1)!}{n!(n -
  1)!}$.
\item If $S_n = C_0C_1 + C_1C_2 + \cdots + C_{n - 1}C_n$ and $\frac{S_{n + 1}}{S_n} = \frac{15}{4}$, find $n$.
\item Show that $C_0^2 + 2.C_1^2 + 3.C_2^2 + \cdots + (n + 1)C_n^2 = \frac{(n + 2)(2n - 1)!}{n!(n - 1)!}$.
\item Show that $C_0.C_n^^{2n} - C_1.C_n^^{2n - 2} + C_2.C_n^^{2n - 4} - \cdots = 2^n$.
\item Show that $\displaystyle\sum_{0\leq i\leq j}\sum_{0\leq j\leq n}(i + j)(C_i + C_j + C_iC_j) = n^2.2^n + n\left(2^{2n - 1} -
  \frac{(2n)!}{2(n!)^2}\right)\;[0\leq i\leq j\leq n]$.
\item If $(1 + x + x^2)^n = a_0 + a_1x + a_2x^2 + \cdots + a_{2n}x^{2n}$, show that $a_0a_{2r} - a_1a_{2r + 1} + a_2a_{2r + 2}
  -\cdots + a_{2n - 2r}a_{2n} = a_{n + r}$.
%\item If $a_r = \frac{1.3.5\ldots(2r - 1)}{2.4.6\ldots 2r}$, then show that $a_{2n + 1} + a_1a_{2n} + a_2a_{2n - 1} + \cdots +
%  a_na_{n + 1} = \frac{1}{2}$.
\item If $P_n$ denoted the product of all coefficients in the expansion of $(1 + x)^n$, show that $\frac{P_{n + 1}}{P_n} = \frac{(n
  + 1)^n}{n!}$.
\item Show that $\displaystyle\sum_{r=1}^nr^3\left(\frac{C_r}{C_{r - 1}}\right)^2 = \frac{1}{12}n(n + 1)^2(n + 2)$.
\item Show that $C_3 + C_7 + C_{11} + \ldots = \frac{1}{3}\left[2^{n - 1} - 2^{n/2}\sin\frac{n\pi}{4}\right]$.
\item If $(1 + x + x^2)^{20} = a_0 + a_1x + a_2x^2 + \cdots = a_{40}x^{40}$, then find the value of $a_0 + a_2 + a_4 + \cdots +
  a_{38}$.
\item If $(1 + x + x^2)^{20} = a_0 + a_1x + a_2x^2 + \cdots = a_{40}x^{40}$, then find the value of $a_1 + a_3 + a_5 + \cdots +
  a_{37}$.
\item Show that $C_1 - \frac{C_2}{2} + \frac{C_3}{3} - \cdots + (-1)^n\frac{C_n}{n} + \frac{1}{n(n - 1)} + \frac{2}{(n - 1)(n - 2)}
  + \cdots + \frac{n - 2}{2.3} = \frac{n + 1}{2}$.
\item Show that $\displaystyle\sum_{0\leq i\leq n}\sum_{0\leq j\leq n}\frac{i}{C_i} + \frac{j}{C_j} =
  \frac{n^2}{2}\sum_{r=0}^n\frac{1}{C_r}\;[0\leq i\leq j\leq n]$.
\item Show that $\displaystyle\sum_{0\leq i\leq n}\sum_{0\leq j\leq n}i.j.C_i.C_j = n^2\left[2^{2n- 3} - \frac{1}{2}{}^{2n -
    2}C_{n - 1}\right]\;[0\leq i\leq j\leq n]$.
\item Prove that $C_1 - \left(1 + \frac{1}{2}\right)C_2 + \left(1 + \frac{1}{2} + \frac{1}{3}\right)C_3 - \cdots + (-1)^n\left(1 +
  \frac{1}{2} + \cdots + \frac{1}{n}\right)C_n = \frac{1}{n}$.
\item Find the coefficient of $x^5$ in the expansion of $(1 + 2x + 3x^2)^4$.
\item Find the coefficient of $x^3y^4z^2$ in the expansion of $(2x - 3y + 4x)^9$.
\item Find the number of terms in $(2x - 3y + 4z)^{100}$.
\item Find the coefficient of $x^4$ in the expansion of $(1 + x + x^2)^3$.
\item Find the coefficient of $x^{10}$ in $(1 + x + x^2 + x^3 + x^4 + x^5)^3$.
\item Find the coefficient of $x^7$ in $(1 + 3x - 2x^3)^{10}$.
\item Find the coefficient of $x^3y^4z^5$ in $(xy + yz + zx)^6$.
\item Find the greatest coefficient in $(w + x + y + z)^{15}$.
\item Find the number of terms in $(a + b + c + d + e)^{100}$.
\item If $|x| < 1$, show that $(1 + x)^{-2} = 1 + 2x + 3x^2 + 4x^3 + \cdots$ to $\infty$.
\item Find $a, b$ so that the coefficient of $x^n$ in the expansion of $\frac{(a + bx)}{(1 - x)^2}$ may be $2n + 1$ and hence find
  the sum of the series $1 + 3\left(\frac{1}{2}\right) + 5\left(\frac{1}{2}\right)^2 + \cdots$.
\item Sum the series $1 + \frac{1}{3} + \frac{1.3.5}{3.6.9} + \cdots$ to $\infty$.
\item If $|x| < 1$, show that $(1 - x)^{-1} = 1 + x + x^2 + x^3 + \ldots$ to $\infty$.
\item If $|x| < 1$, show that $(1 + x)^{-1} = 1 - x + x^2 - x^3 + \ldots$ to $\infty$.
\item If $|x| < 1$, show that $(1 + x)^{-2} = 1 - 2x + 3x^2 - 4x^3 + \ldots$ to $\infty$.
\item If $|x| < 1$, show that $(1 - x)^{-3} = 1 + 3x + 6x^2 + 10x^3 + \ldots$ to $\infty$.
\item If $|x| < 1$, show that $(1 + x)^{-3} = 1 - 3x + 6x^2 - 10x^3 + \ldots$ to $\infty$.
\item If $|x| < 1$, show that $(1 + x)^{1/5} = 1 - \frac{x}{5} + \frac{3x^2}{25} - \frac{11x^3}{125} + \ldots$ to $\infty$.
\item Find the first four terms of $\left(\frac{2x}{3} - \frac{3}{2x}\right)^{-3/2}$.
\item Find the first three terms of $\left(1 - \frac{x}{2}\right)^{-2}$.
\item Find the coefficient of $x^6$ in $(1 - 2x)^{-5/2}$.
\item Find the $(r + 1)$th term and the its coefficients in $(1 - 2x)^{-1/2}$.
\item Find the cube root of $1001$ correct to four places of decimal.
\item Show that $(1 + 2x + 3x^2 + 4x^3 + \ldots $ to $\infty)^{3/2} = 1 + 3x + 6x^2 + 10x^3 + \ldots$ to $\infty$, $|x|< 1$.
\item Sum the series $1 + \frac{1}{4} + \frac{1.3}{4.8} + \frac{1.3.5}{4.8.12} + \ldots$ to $\infty$.
\item Sum the series $1 + \frac{2}{6} + \frac{2.5}{6.12} + \frac{2.5.8}{6.12.18} + \ldots$ to $\infty$.
\item If $y = x - x^2 + x^3 - x^4 + \ldots$ to $\infty$, show that $x = y + y^2 + y^3 + \ldots$ to $\infty$.
\item Show that the coefficent of $x^n$ in $(1 + x + x^2)^{-1}$ is $1, 0, -1$ as $n$ is of the form $3m, 3m - 1, 3m + 1$.
\item Show that $\frac{1}{e} = 2\left[\frac{1}{3!} + \frac{2}{5!} + \frac{3}{7!} + \ldots\text{~to~}\infty\right]$.
\item Sum the series $1 + \frac{2^2}{2!} + \frac{3^2}{3!} + \frac{4^2}{4!} + \ldots$ to $\infty$.
\item Show that $\log 2 = \frac{1}{1.2} + \frac{1}{3.4} + \frac{1}{5.6} + \ldots$ to $\infty$.
\item If $y = x - \frac{x^2}{2} + \frac{x^3}{3} - \frac{x^4}{4} + \ldots$ to $\infty$, show that $x = y + \frac{y^2}{2!} +
  \frac{y^3}{3!} + \ldots$ to $\infty$.
\item If $\alpha, \beta$ be the roots of the equation $ax^2 + bx + c = 0$, show that $\log(a - bx + cx^2) = \log a + (\alpha +
  \beta)x - \frac{(\alpha^2 + \beta^2)}{2}x^2 + \ldots$ to $\infty$.
\item Sum the series $\frac{1}{3!} + \frac{2}{5!} + \frac{3}{7!} + \cdots$ to $\infty$.
\item Sum the series $\frac{1}{2!} + \frac{3}{4!} + \frac{5}{6!} + \cdots$ to $\infty$.
\item Sum the series $\frac{1}{2!} + \frac{1 + 2}{3!} + \frac{1 + 2 +3}{4!} + \cdots$ to $\infty$.
\item Sum the series $\frac{1^3}{1!} + \frac{2^3}{2!} + \frac{3^3}{3!} + \cdots$ to $\infty$.
\item Prove that $1 - \log 2 = \frac{1}{2.3} + \frac{1}{4.5} + \frac{1}{6.7} + \cdots$ to $\infty$.
\item Prove that $\log(1 + x) - \log(x - 1) = 2\left[\frac{1}{x} + \frac{1}{3x^3} + \frac{1}{5x^5} + \cdots
  \text{~to~}\infty\right]$.
\item Prove that $\log x - \log(x + 1) - \log(x - 1) = \frac{1}{x^2} + \frac{1}{2x^4} + \frac{1}{3x^5} + \cdots$ to $\infty$.
\item Prove that $\log(1 + x)^{1 + x}\log(1 - x)^{1 - x} = 2\left[\frac{x^2}{1.2} + \frac{x^4}{3.4} + \frac{x^6}{5.6} + \cdots
  \text{~to~}\infty\right]$
\item If $\alpha, \beta$ be the roots of the equation $x^2 - px + q = 0$, show that $\log(1 + px + qx^2) = (\alpha + \beta)x -
  \frac{\alpha^2 + \beta^2}{2}x^2 + \frac{\alpha^3 + \beta^3}{3}x^3 + \ldots$ to $\infty$.
\stopitemize
