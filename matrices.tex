% -*- mode: context; -*-
\chapter{Matrices}
Matrices are an important concept which has numerous real life usage in various mathematical branches. Also, it has huge importance
in modern computer science. It has its applications in computer graphics, artificial intelligence, data structures leading to
various clever algorithms. Thus, it is of paramount importance that the reader understand this particular concept in a sound
manner.

{\bf Definition:} A {\it matrix} is a rectangular array of real or complex numbers. This rectangular array is made up of rows
and columns much like determinants. Let us consider a matrix of $m\times n$ symbols, where $m$ is number of rows and :$n$ is the
number of columns. \startformula A = \startbmatrix\NC a_{11} \NC a_{12} \NC \ldots \NC a_{1n} \NR\NC a_{21} \NC a_{22} \NC \ldots \NC a_{2n} \NR\NC \vdots \NC \vdots \NC
  \vdots \NC \vdots \NR\NC a_{m1} \NC a_{m2} \NC \ldots \NC a_{mn}\NR\stopbmatrix \stopformula
Such a matrix is called $m$ by $n$ matrix or a matrix of order $m\times n$. Sometimes a matrix is shown with parenthese instead of
square brackets as shown in last example.
\startformula A = \startpmatrix\NC a_{11} \NC a_{12} \NC \cdots \NC a_{1n} \NR\NC a_{21} \NC a_{22} \NC
\cdots \NC a_{2n} \NR\NC \vdots \NC \vdots \NC \vdots \NC \vdots \NR\NC a_{m1} \NC a_{m2} \NC
\cdots \NC a_{mn}\NR\stoppmatrix\stopformula
A compact way to write a matrix is $A = [a_{ij}], 1\leq i \leq m; 1\leq j \leq n$ or simply $[a_{ij}]_{m\times n} a_{ij}$ is an
element located at $i^{th}$ row and $j^{th}$ column and is called $(i, j)^{th}$ element of the matrix. A matrix is just a
rectangular array of numbers and unlike determinants it does not have a value.

\section{Classification of Matrices}
\subsection{Equal Matrices}
Two matrices are said to be {\it equal} if they have same order and each corresponding element is equal.

\subsection{Row Matrix}
A matrix having a single row is called a {\it row} matrix. For example, $[1,2, 3, 4]$.

\subsection{Column Matrix}
A matrix having a single column is called a {\it column} matrix. For example, \startformula \startbmatrix\NC 1\NR\NC2\NR\NC3\NR\NC4\NR\stopbmatrix .\stopformula

\subsection{Square Matrix}
If $m = n$ i.e number of rows and columns are equal then the matrix is called a {\it square} matrix. For example,
\startformula \startbmatrix\NC 1 \NC 2 \NC 3 \NR\NC 4 \NC 5 \NC 6 \NR\NC 7 \NC 8 \NC 9\NR\stopbmatrix \stopformula is a $3\times 3$ matrix.

\subsection{Diagonal Matrix}
The diagonal from left-hand side upper corner to right-hand side lower corner is known as leading diagonal or principal diagonal. In the example of square
matrix the elements of diagonal are $1, 5, 9$. When a matrix has all elements as zero except those belonging to its diagonal, then
it is called a {\it diagonal} matrix. Equivalently, We can say that a matrix $[a_{ij}]_{m\times n}$ is a diagonal matrix if $a_{ij} =
0\;\forall\;i\neq j$. For example, the square matrix example can be converted to a diagonal matrix like below:

\startformula \startbmatrix\NC 1 \NC 0 \NC 0 \NR\NC 0 \NC 5 \NC 0 \NR\NC 0 \NC 0 \NC 9\NR\stopbmatrix \stopformula

For an $n\times n$ matrix the diagonal elements are represented as $[d_1, d_2 \ldots, d_n]$ This diagonal is also written with a
{\it diag} prefix like $diag [d_1, d_2 \ldots, d_n]$.

\subsection{Scalar Matrix}
A diagonal matrix whose elements of the diagonal are equal is called {\it scalar} matrix. For example:
\startformula\startbmatrix\NC 5 \NC 0 \NC 0 \NR\NC 0 \NC 5 \NC 0 \NR\NC 0 \NC 0 \NC 5\NR\stopbmatrix \stopformula
For a square matrix $[a_{ij}]_{m\times n}$ to be a scalar matrix:
\startformula a_{ij} = \startmathcases\NC 0, \NC i\neq j \NR\NC m,\NC i = j\NR\stopmathcases\;\forall\;m\neq 0\stopformula

\subsection{Unit Matrix or Identity Matrix}
A diagonal matrix of order $n$, which has all elements of its diagonal as one, is called a {\it unit} or {\it identity}
matrix. It is also denoted by $I_n$. We can rewrite it in concise way like we did for scalar matrix as
\startformula a_{ij} = \startmathcases\NC 0, \NC i\neq j \NR\NC 1,\NC i = j\NR\stopmathcases\stopformula

\subsection{Horizontal Matrix}
An $m\times n$ matrix is called a {\it horizontal} matrix if $m < n$. For example:
\startformula \startbmatrix\NC 1 \NC 2 \NC 3\NR\NC4 \NC 5 \NC 6\NR\stopbmatrix \stopformula

\subsection{Vertical Matrix}
An $m\times n$ matrix is called a {\it vertical} matrix if $m > n$. For example:
\startformula \startbmatrix\NC 1 \NC 2 \NR\NC 3 \NC 4 \NR\NC 5 \NC 6\NR\stopbmatrix \stopformula

\subsection{Triangular Matrix}
A sqaure matrix in which all the elements below the diagonal are zero is called {\it upper triangular} matrix. Conversely, a
sqaure matrix in which all the elements above the diagonal matrix is called {\it lower triangular} matrix. Thus, for a lower
triangular matrix $a_{ij} = 0$ when $i < j$ and for an upper triangular matrix $a_{ij} = 0$ when $i > j$

Clearly, a diagonal matrix is both lower and upper triangular matrix. A triangular matrix is called strictly triangular if $a_{ii}
= 0\;\forall\;1\leq i \leq n$. Example of upper triangular matrix:
\startformula \startbmatrix\NC 1 \NC 2 \NC 3\NR\NC0 \NC 5 \NC 6\NR\NC0 \NC 0 \NC 9\NR\stopbmatrix \stopformula
Example of lower triangular matrix:
\startformula \startbmatrix\NC 1 \NC 0 \NC 0\NR\NC4 \NC 5 \NC 0\NR\NC7 \NC 8 \NC 9\NR\stopbmatrix \stopformula

\subsection{Null or Zero Matrix}
If all elements of a matrix is zero then it is a {\it null} or {\it zero} matrix.

\subsection{Singular and Non-Singular Matrix}
A matrix is said to be {\it non-singular} if $|A|\neq 0$ and s{\it ingular} if $|A| = 0$.

\subsection{Trace of Matrix}
If sum of the elements of a sqaure matrix $A$ lying along the principal diagonal is called the {\it trace} of $A,$
i.e. $tr(A)$. Thus, if $A = [a_{ij}]_{n\times n},$ then $tr(A) = \sum_{i = 1}^n a_{ii}$

\subsection{Properties of Trace of a Matrix}
To prove the second and third properties of a trace of matrix we will have to use properties given further below on algebraic
operations on a matrix. If $A = [a_{ii}]_{n\times n}$ and $B = [b_{ii}]_{n\times n}$ and $\lambda$ is a scalar then

\startitemize[n]
\item $tr(\lambda A) = \lambda tr(A)$
\item $tr(A + B) = tr(A) + tr(B)$
\item $tr(AB) = tr(BA)$
\stopitemize

\subsection{Determinant of a Matrix}
Every square matrix $A$ has a determinant associated with it. This is written as $det(A)$ or $|A|$ or $\Delta$. We observe
following for determinants of matrices:

\startitemize[n]
\item  If $A_1, A_2, \ldots, A_n$ are square matrices of the same order then $|A_1A_2\ldots A_n| = |A_1||A_2|\ldots|A_n|$.
\item If $k$ is a scalar, then $|kA| = k^n|A|$, where $n$ is the   order of matrix.
\item If $A$ and $B$ are two matrices of equal order then $|AB| = |BA|$ even though $AB\neq BA$.
\stopitemize

\section{Algebra of Matrices}

\subsection{Addition of Matrices}
If any two matrices are of same order then addition of those can be performed. The result is a matrix of same order with
corresponding elements added. For example, consider two $3\times 3$ matrices as given below:
\startformula A = \startbmatrix\NC a_1 \NC a_2 \NC a_3\NR\NC a_4 \NC a_5 \NC a_6\NR\NC a_7 \NC a_8 \NC
a_9\NR\stopbmatrix , B = \startbmatrix\NC b_1 \NC b_2 \NC b_3\NR\NC b_4 \NC b_5 \NC b_6\NR\NC b_7 \NC b_8 \NC
b_9\NR\stopbmatrix \stopformula
then,
\startformula A + B = \startbmatrix\NC a_1 + b_1 \NC a_2 + b_2 \NC a_3 + b_3\NR\NC a_4 + b_4 \NC
a_5 + b_5 \NC a_6 + b_6\NR\NC a_7 + b_7 \NC a_8 + b_8 \NC a_9 + b_9\NR\stopbmatrix \stopformula

\subsection{Subtraction of Matrices}
The conditions are same for subtraction to happen i.e.\ order of the matrices must be same. The result is like that of addition
with resulting elements being the difference of original matrices. For example,
\startformula A - B = \startbmatrix\NC a_1 - b_1 \NC a_2 - b_2 \NC a_3 - b_3\NR\NC a_4 - b_4 \NC
a_5 - b_5 \NC a_6 - b_6\NR\NC a_7 - b_7 \NC a_8 - b_8 \NC a_9 - b_9\NR\stopbmatrix \stopformula
where $A$ and $B$ are matrices from previous example. Following is observed for addition and subtraction:

\startitemize[n]
\item Addition of matrices is commutative i.e. $A + B = B + A$ as well as associative i.e. $(A + B) + C = A + (B + C)$.
\item Cancellation laws are true in case of addition.
\item  The equation $A + B = O$ has a unique solution in the set of all $m\times n$ matrices (where $O$ is null matrix).
\stopitemize

\subsection{Scalar Multiplication}
The scalar multiplication of a matrix $A$ with a scalar $\lambda$ is defined as $\lambda A = [\lambda a_{ij}]$.

\subsection{Multiplication of two Matrices}
The prerequisite for matrix multiplication is that number of columns of first matrix must be equal to number of rows of second
matrix. The product is defined as
\startformula A_{m\times n}B_{n\times p} = \sum_{r = 1}^na_{mr}b_{rp}\stopformula
It can be easily verified that the resulting matrix will have $m$ rows and $p$ columns.

\subsubsection{Properties of Matrix Multiplication}
\startitemize[n]
\item Commutative laws does not hold always for matrices.
\item If $AB = BA$, then they are called commutative matrices.
\item If $AB = -BA$, then they are called anti-commutative matrices.
\item Matrix multiplication is associative i.e. $(AB)C = A(BC)$. Proof of this has been left as an exercise.
\item Matrix mulplication is distributive wrt addition and subtraction i.e. $A(B\pm C) = AB \pm AC$.
\stopitemize

\subsection{Transpose of a Matrix}
Let $A$ be any matrix then its {\it transpose} can be obtained by ecxchanging rows and columns. It is denoted by $A'$ or $A^T$
and clearly, if order of $A$ is $m\times n$ then $A'$ will have order of $n\times m$.

\subsubsection{Properties of Transpose Matrices}
\startitemize[n]
\item $(A + B)' = A' + B'$.
\item $(A')' = A$.
\item $(kA)' = kA'$ where $k$ is a constant.
\item $(AB)' = B'A'$.
\stopitemize
Proofs of these properties are simple and have been left as an exercise.

\subsection{Symmetric Matrix}
A sqaure matrix $A = [a_{ij}]$ is called a {\it symmetric} matrix if $a_{ij} = a_{ji}\;\forall\;i,j$. We can also say thet a
matrix is symmetric if and only if $A = A'$.

\subsection{Skew Symmetric Matrix}
A square matrix $A$ is said to be a {\it skew symmetric} matrix if $a_{ij} = -a_{ji}\;\forall i, j$. Clearly, if a matrix is skew
symmetric then elements of its diagonal are all zeros.

\subsection{Orthogonal Matrix}
A matrix is said to be orthogonal if $AA'=1$.
\starttheorem
  If $A$ is a square matrix then $A + A'$ is a symmetric matrix and $A-A'$ is a skew symmetric matrix.
\stoptheorem
\startproof
  $(A + A')' = A' + (A')' = A' + A$. Hence, $A + A'$ is a symmetric matrix. $(A - A')' = A' - A = -(A - A')$. Hence, $A- A'$ is a
  skew symmetric matrix.
\stopproof

\starttheorem
  Every square matrix can be shown as sum of a symetric matrix and a skew   symmetric matrix.
\stoptheorem
\startproof
  Let $A$ be any square matrix. $\frac{1}{2}(A + A') + \frac{1}{2}(A - A') = A$ hus, the matrix $A$ is a sum of symmetrix matrix $A
  + A'$ and a skew symmetric matrix $A - A'$
\stopproof

\subsection{Adjoint of a Matrix}
Let $A= [a_{ij}]$ be a square matrix. Let $B = [A_{ij}]$ where $A_{ij}$ is the cofactor of the element $a_{ij}$ in the
det.\ $A$. The transpose $B'$ of the matrix $B$ is called the adjoint of the matrix $A$ and is written by $adj. A$. For example,

Let $A = \startbmatrix\NC 1 \NC 2\NC 5\NR\NC2 \NC 3 \NC 4\NR\NC2\NC 0 \NC 5\NR\stopbmatrix $, then
$B = \startbmatrix\NC 15 \NC -2 \NC -6\NR\NC-10 \NC -1 \NC 4\NR\NC-1 \NC 2 \NC -1\NR\stopbmatrix $

\startformula adj. A = B' = \startbmatrix\NC 15\NC -10 \NC -1\NR\NC-2 \NC -1 \NC 2\NR\NC-6 \NC -4 \NC -1\NR\stopbmatrix \stopformula
\startformula A.adj(A) = adj(A).A = |A|I_n\stopformula

\subsection{Inverse of a Matrix}
Following from above, inverse of a matrix is $\frac{adj(A)}{|A|}$. Inverse of a matrix $A$ is denoted by $A^{-1}$.

\subsection{Hermitian and Skew Hermitian Matrix}
A sqaure matrix $A = [a_{ij}]$ is said to be a {\it Hermitian} matrix if $a_{ij} = \overline{a_{ij}}\;\forall\;i, j$ i.e. $A =
A^{\theta}$. For example,
\startformula \startbmatrix\NC a \NC b + ic \NR\NC b - ic \NC d\NR\stopbmatrix \stopformula is a Hermitian
matrix.

Similarly, a sqaure matrix $A = [a_{ij}]$ is said to be a {\it skew Hermitian} matrix if $a_{ij} = \overline{a_{ji}}\;\forall\;i,
j$ i.e. $A = -A^{\theta}$. For example,

\startformula \startbmatrix\NC 0 \NC -b + ic \NR\NC b + ic \NC 0\NR\stopbmatrix \stopformula is a skew Hermitian matrix. Following are observed for these types of
matrices:
\startitemize[n]
\item If $A$ is a hermitian matrix, then $a_{ii} = \overline{a_{ii}} \Rightarrow a_{ii}$ is real, $\forall\;i.$ Thus, members of
  diagonal of a Hermitian matrix are all real.
\item A Hermitian matrix over the set of real numbers is actually a real symmetric matrix.
\item If $A$ is a skew Hermitian matrix, then $a_{ii} = -\overline{(a_{ii})} \Rightarrow a_{ii} = 0$ i.e. $a_{ii}$ must be purely
  imaginary or zero.
\item A skew Hermitian matrix over the set of real numbers is acually a real skew-symmetric matrix.
\stopitemize

\subsection{Idempotent Matrix}
A square matrix $A$ is said to be {\it idempotent} if $A^2 = A$ i.e. multiplication of the matrix with itself yields itself.

\subsection{Involuntary Matrix}
A sqaure matrix $A$ is said to be {\it involuntary} if $A^2 = I$ i.e. multiplication of the matrix with itself yields an
indetity matrix.

\subsection{Nilpotent Matrix}
For a positive integer $i$ if a square matrix satisfied the relationship $A^i = O$ then it is called a {\it nilpotent}
matrix. Such smallest integer is called index of the nilpotent matrix.

\section{Properties of adjoint and inverse matrices}
\startitemize[n]
\item If $A$ is a sqaure matrix of order $n,$ then $A(adj(A)) = |A|I_n = (adj(A))A$.

Let $A= [a_{ij}],$ and let $C_{ij}$ be a cofactor of $a_{ij}$ in $A$. Then, $(adj(A)) = C_{ji}\;\forall\;1\leq i, j\leq n$. Now,

\startformula (A\;adj(A)) = \sum_{r = 1}^n (A)_{ir}(adj(A))_{rj}\stopformula
\startformula = \sum_{r = 1}^n a_{ir}C_{rj} = \startmathcases\NC |A|, \NC \text{if}\;i = j \NR\NC 0, \NC \text{if}\;i \neq j\NR\stopmathcases\stopformula
\startformula \Rightarrow = \startbmatrix\NC  |A| \NC 0 \NC 0 \NC \ldots \NC 0 \NR\NC 0 \NC |A| \NC 0
   \ldots \NC 0 \NR\NC \vdots \NC \vdots \NC \vdots \NC \vdots \NC \vdots \NR\NC 0 \NC 0 \NC 0 \NC
   \ldots \NC |A|\NR\stopbmatrix \stopformula
\startformula = |A|I_n\stopformula
Similarly, \startformula (adj(A) A)_{ij} = \sum_{r = 1}^n (adj(A))_{ir}A_{rj}\stopformula
\startformula = \sum_{r = 1}^n C_{ri}a_{rj}= \startmathcases\NC |A|, \NC \text{if}\;i = j
   \NR\NC 0, \NC \text{if}\;i \neq j\NR\stopmathcases\stopformula

\item Every invertible matrix possesses a unique matrix. Let $A$ be a sqaure matrix of order $n\times n$. Let $B$ and $C$ be two
inverses of $A$. Then, $AB = BA = I_n$ and $AC = CA = I_n$
\startformula AB = I_n\Rightarrow C(AB) = CI_n\Rightarrow (CA)B=CI_n\Rightarrow I_nB = CI_n\stopformula
\startformula \Rightarrow B = C\stopformula

\item  Reversal law: If $A$ and $B$ are invertible matrices of same oreder, then $AB$ is invertible and $(AB)^{-1} = B^{-1}A^{-1}.$
In general, if $A,B, C, ...$ are invertible matrices then $(ABC\ldots)^n = \ldots C^{-1}B^{-1}A^{-1}$

If the given matrices are invertible $|A|\neq 0$ and $|B|\neq 0 \Rightarrow |A||B|\neq 0$ Hence, $AB$ is an invertible matrix. Now,

\startformula (AB)(B^{-1}A^{-1}) = A(BB^{-1})A^{-1}\stopformula
\startformula = A(I_n)A^{-1} = AA^{-1} = I_n\stopformula
Similarly, \startformula (B^{-1}A^{-1})(AB) = I_n\stopformula

\item  If $A$ is an invertible matrix, then $A'$ is also invertible and $(A')^{-1} = (A^{-1})'$.

$A$ is an invertible matrix $\therefore |A| \neq 0 \Rightarrow |A'|\neq 0 [\because |A'| = |A|]$. Hence, $A'$ is also invertible. Now,
\startformula AA^{-1} = I_n = A^{-1}A\stopformula
\startformula (AA^{-1})' = (A^{-1}A)'\stopformula
\startformula (A^{-1})'A' = I_n = A'(A^{-1})'\stopformula
\startformula \Rightarrow (A')^{-1} = (A^{-1})'\stopformula

\item  If $A$ is a non-singular square matrix of order $n,$ then $|adj A| = |A|^{n - 1}$.

We have $A(adj(A)) = |A|I_n$

\startformula A(adj(A)) = \startbmatrix\NC |A| \NC 0 \NC 0 \NC \cdots \NC 0 \NR\NC 0 \NC |A| \NC 0 \NC \cdots \NC 0 \NR\NC \vdots \NC \vdots \NC \vdots \NR\NC 0 \NC 0 \NC 0 \NC \ldots
  \NC |A|\NR\stopbmatrix \stopformula
\startformula |A(adj(A))| = |A|^n\stopformula
\startformula |adj(A)| = |A|^{n - 1}\stopformula

\item  Reversal law for adjoint: If $A$ and $B$ are non-singular
   sqaure matrices of the same order, then

   $adj(AB) = adj(B)adj(A)$ using $(AB)^{-1} = B^{-1}A^{-1}$

\item  If $A$ is an invertible square matrix, then $adj(A') =
   (adj(a))'$

\item  If $A$ is a sqaure non-singular matrix, then $adj(adj(A)) = A^{n
   - 2}A$

We know that $B(adj(B)) = |B|I_n$ for every sqaure matrix of order $n.$ Replacing $B$ by $adj(A),$ we get $(adj(A))[adj(adj(A))] =
|adj(A)|I_n = |A|^{n - 1}I_n$. Multiplying both sides by $A$
\startformula (A\;adj(A))[adj(adj(A))] = A\{|A|^{n - 1}I_n\}\stopformula
\startformula |A|I_n (adj(adj(A))) = |A^{n - 1}|(AI_n)\stopformula
\startformula adj(adj(A)) = |A^{n - 2}|A\stopformula

\item  If $A$ is a non-singular matrix then $|A^{-1}| = = |A|^{-1}$ i.e. $|A^{-1}| = \frac{I}{A}$. Since $|A|\neq 0, \therefore
  AA^{-1} = I, |AA^{-1}| = |A|\Rightarrow |A||A^{-1}| = 1$

\item  Inverse of $k^{th}$ power of $A$ is $k^{th}$ power of the inverse of $A$.
\stopitemize

\section{Solution of Simultaneous Linear Equations}
Consider the system of equations given below:

\startformula\startmathcases\NC a_{11}x_1 + a_{12}x_2 + \cdots + a_{1n} = b_1\NR\NC a_{21}x_1 +
a_{22}x_2 + \cdots + a_{2n} = b_2\NR\NC\vdots\NR\NC a_{n1}x_1 + a_{n2}x_2 + \cdots +
a_{nn} = b_n\NR\stopmathcases\stopformula

Let \startformula A = \startbmatrix\NC a_{11} \NC a_{12} \NC \cdots \NC a_{1n}\NR\NC a_{21} \NC a_{22}
\NC \cdots \NC a_{2n}\NR\NC\vdots \NC \vdots \NC \cdots \NC \vdots\NR\NC a_{n1} \NC a_{n2} \NC \cdots
\NC a_{nn}\NR\stopbmatrix , X= \startbmatrix\NC x_1\NR\NC x_2\NR\NC\vdots\NR\NC x_n\NR\stopbmatrix , B =
\startbmatrix\NC b_1 \NR\NC b_2 \NR\NC \cdots\NR\NC b_n\NR\stopbmatrix \stopformula

The system of equations can be written as $AX = B\Rightarrow X=A^{-1}B$. If $|A|\neq 0$, the system of equations has only trivial
solution and the number of solutions is finite. If $|A|=0$, the system of equations has non-trivial solution and the number of
solutions is infintite. If the number of equations is less than the number of unknonwns then it has non-trivial solutions.

\section{Elementary Operations/Transformations of a Matrix}
Following are elementary operations of a matrix:
\startitemize[n]
\item The interchange of any two rows or columns.
\item The multiplication of any row or column with a non-zero number.
\item The addition to the elements of any row or columns the corresponding elements of any other row or column multiplied with any
  non-zero number.
\stopitemize

Elementary operations are also called row or column operation.

\subsection{Equivalent Matrices}
If a matrix $B$ can be obtained from a matrix $A$ by elementary transformations, then they are called equivalent matrices and are
written as $A\text{~} B$.

Every elementary row or column transformation of $m\times n$ matrix (not identiry matrix) can be obtained by pre-multiplcation or
post-multiplication with the corresponding elementary matrix obtained from the identity matrix $I_m(I_n)$ by subjecting it to the
same elementary row or column transformation.

Let $C = AB$ be a product of two matrices. Any elementray row or column transformation of $AB$ can be obained by subjecting the
pre-factor $A$ or post-factor $B$ to the same elementary row or column transformation.

\subsection{Method of Finding Inverse of a Matrix by Elementary Transformation}
Let $A$ be a non-singular matrix of order $n$. Then $A$ can be reduced to the identity matrix $I_n$ by a sequence of elementary
transformations only. As we have discussed every elementary row transformation of a matrix is equivalent top pre-multiplication by
the corresponding elementary matrix. Therefore, there exists elementary matrices $E_1, E_2,\ldots, E_k$ such that
$$(E_1,E_2,\ldots, E_k)A = I_n$$
$$(E_1,E_2,\ldots, E_k)AA^{-1} = I_nA^{-1}$$
$$(E_1,E_2,\ldots, E_k)I_n = A^{-1}$$

\section{Echelon Form of a Matrix}
A matrix is said to be in echelon form if
\startitemize[n]
\item Every row of $A$ which has all its elements $0$, occurs below row which has a non-zero element.
\item The first non-zero element in each non-zero row is $1$.
\item The number of zeros before the first non-zero element in a row is less than the number of such zeros in the next row.
\stopitemize

\section{Rank of a Matrix}
Let $A$ be a matrix of order $m\times n$. If at least one of its minors of order $r$ is different from zero and all minors of order
$r + 1$ are zero, then the number $r$ is called the rank of the matrix $A$ and is denoted by $\rho(A)$.

\startitemize[n]
\item The rank of a zero matrix is zero and rank of an identity matrix of order $n$ is $n$.
\item The rank of a non-singular matrix of order $n$ is $n$.
\item The rank of a matrix in echelon form is equal to the number of non-zero rows of the matrix.
\stopitemize

\section{Application of Matrices to Geometry or Computer Graphics}
As said earlier matrices are very useful to represent many operaion in computer
graphics or geometry. It will require some knowledge of coordinate geometry.

\subsection{Reflection Matrix}
Consider a point $P(x, y)$ and its reflection $Q(x_1, y_1)$ along
x-axis.

\startplacefigure
  \startMPcode
    drawarrow (-0.5cm, 0) -- (3cm, 0);
    drawarrow (0, -0.5cm) -- (0, 3cm);
    draw (2cm, 2cm) -- (2cm, -2cm);
    label("$P(x, y)$", (2cm, 2.2cm));
    label("$Q(x_1, y_1)$", (2cm, -2.2cm));
    label("$x$", (0, 3.2cm));
    label("$y$", (3.2cm, 0));
  \stopMPcode
\stopplacefigure
This may be written as $x_1 = x + 0; y_1 = 0 - y$. This system of equation can be written in matrix form as

\startformula\startbmatrix\NC x_1\NR\NC y_1\NR\stopbmatrix  = \startbmatrix\NC 1 \NC 0 \NR\NC 0 \NC
-1\NR\stopbmatrix  \startbmatrix\NC x \NR\NC y\NR\stopbmatrix\stopformula

Thus the matrix $\startbmatrix\NC 1 \NC 0 \NR\NC 0 \NC -1\NR\stopbmatrix $ is
reflection matrix of a point along x-axis. Similarly, $\startbmatrix\NC -1
\NC 0 \NR\NC 0 \NC 1\NR\stopbmatrix $ is reflection matrix along y-axis.

Similarly, the reflection matrix through origin is $\startbmatrix\NC -1
\NC 0 \NR\NC 0 \NC -1\NR\stopbmatrix $

Similarly, reflection along the line $y = x$ is $\startbmatrix\NC 0 \NC
1 \NR\NC 1 \NC 0\NR\stopbmatrix $

Similarly, reflection along the line $y = x\tan\theta$ is
$\startbmatrix\NC \cos 2\theta \NC \sin 2\theta \NR\NC \sin 2\theta \NC -\cos
2\theta\NR\stopbmatrix$

\subsection{Rotation Through an Angle}
The rotation matrix in such a form would be $\startbmatrix\NC \cos\theta \NC
-\sin\theta \NR\NC \sin\theta \NC \cos\theta\NR\stopbmatrix$ for anti-clockwise
rotation.
\pagebreak
\section{Problems}
\startitemize[n, 1*broad]
\item Find the number of matrices having $12$ elements.
\item Write down the matrix $A = [a_{ij}]_{2\times 3}$ where $a_{ij} = 2i - 3j$.
\item If $A= \startbmatrix\NC a \NC b\NR\NC-b \NC a\NR\stopbmatrix, B=\startbmatrix\NC -a \NC b\NR\NC-b \NC -a\NR\stopbmatrix$, then find $A + B$.
\item If $Y = \startbmatrix\NC 3 \NC 2\NR\NC1 \NC 4\NR\stopbmatrix$ and $2X + Y = \startbmatrix\NC 1 \NC 0\NR\NC-3 \NC 2\NR\stopbmatrix$, find $X$.
\item If $\startbmatrix\NC x^2 - 4x \NC x^2\NR\NC x^2 \NC x^3\NR\stopbmatrix = \startbmatrix\NC -3 \NC 1\NR\NC x- + 2 \NC 1\NR\stopbmatrix$, then find $x$.
\item Find $x, y, z$ and $a$ for which $\startbmatrix\NC x + 3 \NC 2y + x \NR\NC z -1 \NC 4a - 6\NR\stopbmatrix = \startbmatrix\NC 0 \NC -7 \NR\NC 3 \NC
    2a\NR\stopbmatrix$.
\item If $A = \startbmatrix\NC 1 \NC 2 \NC 3\NR\NC-1 \NC 0 \NC 2\NR\NC1 \NC -3 \NC 1\NR\stopbmatrix, B = \startbmatrix\NC 4 \NC 5 \NC 6\NR\NC -1 \NC 0 \NC 1\NR\NC 2 \NC 1 \NC
    2\NR\stopbmatrix$, find $4A - 3B$.
\item If $A = \startbmatrix\NC 1 \NC -2 \NC 3 \NR\NC -4 \NC 2 \NC 5\NR\stopbmatrix, B = \startbmatrix\NC 2 \NC 3 \NR\NC 4 \NC 5 \NR\NC 2 \NC 1\NR\stopbmatrix$, find
    $AB$ and $BA$. Also, show that $AB\neq BA$.
\item If $A, B, C$ are three matrices such that $A = \startbmatrix\NC x \NC y \NC z\NR\stopbmatrix, B = \startbmatrix\NC a \NC h \NC g\NR\NC h \NC b \NC f
    \NR\NC g \NC f \NC c\NR\stopbmatrix, C = \startbmatrix\NC x \NR\NC y \NR\NC z\NR\stopbmatrix$, then find $ABC$.
\item Find the transpose and adjoint of the matrix $A$, where $A =    \startbmatrix\NC 1 \NC 2 \NC 3\NR\NC0 \NC 5 \NC 0\NR\NC2 \NC 4 \NC
  3\NR\stopbmatrix$.
\item Find the inverse of the matrix $A = \startbmatrix\NC 0 \NC 1 \NC 2\NR\NC1 \NC 2 \NC 3\NR\NC3 \NC 1 \NC 1\NR\stopbmatrix$.
\item Find the inverse of the matrix $A= \startbmatrix\NC 1 \NC 2 \NC 5\NR\NC2 \NC 3 \NC 1\NR\NC-1 \NC 1 \NC 1\NR\stopbmatrix$ and verify that $AA^{-1} =
  1$.
\item Let $A = \startbmatrix\NC 1 \NC 2 \NC 2\NR\NC2 \NC 1 \NC 2\NR\NC2 \NC 2 \NC 1\NR\stopbmatrix$, prove that $A^2-4A-5I = 0$, hence obtain $A^{-1}$.
\item Solve the following equations by matrix method: $5x + 3y +z = 16, 2x + y + 3z = 19 \text{~and~}x + 2y + 4z = 25$.
\item Find the product of two matrices $A$ and $B$ where  $A=\startbmatrix\NC -5 \NC 1 \NC 3\NR\NC7 \NC 1 \NC -5\NR\NC1 \NC -1 \NC 1 \NR\stopbmatrix, B
  = \startbmatrix\NC 1 \NC 1 \NC 2\NR\NC 3 \NC 2 \NC 1\NR\NC 2 \NC 1 \NC 3\NR\stopbmatrix$ and use it for solving the equations $x + y + 2z = 1, 3x + 2y +
    z = 7\text{~and~}2x + y + 3z = 2$.
\item If $\startbmatrix\NC x + y \NC 2 \NR\NC 1 \NC x - y\NR\stopbmatrix = \startbmatrix\NC  3 \NC 2 \NR\NC 1 \NC 7\NR\stopbmatrix$, then find $x$ and $y$.
\item If $\startbmatrix\NC x - y \NC 2x + x_1 \NR\NC 2x - y \NC 3x + y_1\NR\stopbmatrix = \startbmatrix\NC -1 \NC 5 \NR\NC 0 \NC 13\NR\stopbmatrix$ and
    co-ordinates of points $P$ and $Q$ be $(x, y)$ and $(x_1, y_1)$, then find $PQ$.
\item Find $X$ and $Y$ if $X + Y = \startbmatrix\NC 7 \NC 0 \NR\NC 2 \NC 5\NR\stopbmatrix$ and $X - Y = \startbmatrix\NC 3 \NC 0 \NR\NC 0 \NC 3\NR\stopbmatrix$.
\item Given $A = \startbmatrix\NC  1 \NC 2 \NC -3 \NR\NC 5 \NC 0 \NC 2 \NR\NC 1 \NC -1 \NC 1\NR\stopbmatrix$ and $B = \startbmatrix\NC 3 \NC -1 \NC 2 \NR\NC 4 \NC 2 \NC
    5 \NR\NC 2 \NC 0 \NC 3\NR\stopbmatrix$, find the matrix $C$ such that $A + C = B$.
\item If $A = \startbmatrix\NC 2 \NC 3 \NC 4 \NR\NC -3 \NC 0 \NC 2\NR\stopbmatrix, B = \startbmatrix\NC 3 \NC -4 \NC -5 \NR\NC 1 \NC 2 \NC 1\NR\stopbmatrix$ and $C
    = \startbmatrix\NC  5 \NC -1 \NC 2 \NR\NC 7 \NC 0 \NC 3\NR\stopbmatrix$, find the matrix $X$ such that $2A + 3B = X + C$.
\item If $A = \startbmatrix\NC  1 \NC 2 \NC 3 \NR\NC -1 \NC 0 \NC 2 \NR\NC 1 \NC -3 \NC 1\NR\stopbmatrix, B = \startbmatrix\NC 4 \NC 5 \NC 6 \NR\NC -1 \NC 0 \NC 1 \NR\NC 2 \NC
    1 \NC 2\NR\stopbmatrix, C = \startbmatrix\NC -1 \NC 2 \NC 1 \NR\NC -1 \NC 2 \NC 3 \NR\NC -1 \NC -2 \NC 2\NR\stopbmatrix$, find $A - 2B + 3C$.
\item If $P(x) = \startbmatrix\NC \cos x \NC \sin x \NR\NC -\sin x \NC \cos x\NR\stopbmatrix,$ then show that $P(x).P(y) = P(x + y) = P(y).P(x)$
\item If $A = \startbmatrix\NC 1 \NC 0 \NC 0 \NR\NC 0 \NC 1 \NC 0 \NR\NC a \NC b \NC -1\NR\stopbmatrix,$ find $A^2$.
\item If $A=\startbmatrix\NC -1 \NC 1 \NC -1 \NR\NC 3 \NC -3 \NC 3 \NR\NC 5 \NC -5 \NC 5 \NR\stopbmatrix, B = \startbmatrix\NC 0 \NC 4 \NC 3 \NR\NC 1 \NC -3 \NC -3 \NR\NC -1
  \NC 4 \NC 4\NR\stopbmatrix,$ then find $A^2B^2$.
\item If $A = \startbmatrix\NC 2 \NC 3 \NC 4 \NR\NC 1\NC 2 \NC 3 \NR\NC -1 \NC 1 \NC 2\NR\stopbmatrix, B = \startbmatrix\NC 1 \NC 3 \NC 0\NR\NC -1 \NC 2 \NC 1 \NR\NC 0 \NC 0 \NC
    2\NR\stopbmatrix$, find $AB$ and $BA$ and show that $AB \neq BA$.
\item Find the product of the following two matrices: $\startbmatrix\NC 0 \NC c \NC -b \NR\NC -c \NC 0 \NC a \NR\NC b \NC -a \NC 0\NR\stopbmatrix$ and
  $\startbmatrix\NC a^2 \NC ab \NC ac \NR\NC ab \NC b^2 \NC bc \NR\NC ac \NC bc \NC c^2\NR\stopbmatrix$.
\item If $A = \startbmatrix\NC 3 \NC -5 \NR\NC -4 \NC 2\NR\stopbmatrix$, find $A^2 - 5A - 14I$, where $I$ is a unit matrix.
\item Verify that $A = \startbmatrix\NC 2 \NC 3\NR\NC 1 \NC 2\NR\stopbmatrix$ satisfies the equation $A^3 - 4A^2 + A = O$.
\item If $A = \startbmatrix\NC 0.8 \NC 0.6 \NR\NC -0.6 \NC 0.8\NR\stopbmatrix,$ find $A^2$.
\item If $A = \startbmatrix\NC 3 \NC 1 \NR\NC -1 \NC 2\NR\stopbmatrix$, find $f(A)$, where $f(x) = x^2 - 5x + 7I$.
\item If $A=\startbmatrix\NC \cos\theta \NC \sin\theta \NR\NC \sin\theta \NC \cos\theta \NR\stopbmatrix, B = \startbmatrix\NC \cos\phi \NC \sin\phi
  \NR\NC\sin\phi \NC \cos\phi \NR\stopbmatrix,$ show that $AB = BA$.
\item Let $f(x) = x^2 - 5x + 6,$ find $f(A),$ if $A = \startbmatrix\NC  2 \NC 0 \NC 1 \NR\NC 2 \NC 1 \NC 3 \NR\NC 1 \NC -1 \NC 0\NR\stopbmatrix$.
\item If the matrix $A = \startbmatrix\NC 5 \NC 3 \NR\NC 12 \NC 7\NR\stopbmatrix$, then verify that $A^2 - 12 A - I = 0,$ where $I$ is a unit matrix.
\item Show that $\startpmatrix\NC\startbmatrix\NC 1 \NC \omega \NC \omega^2 \NR\NC\omega \NC \omega^2 \NC 1 \NR\NC \omega^2 \NC 1 \NC \omega \NR\stopbmatrix
  + \startbmatrix\NC  \omega \NC \omega^2 \NC 1 \NR\NC \omega^2 \NC 1 \NC \omega \NR\NC \omega \NC \omega^2 \NC
    1\NR\stopbmatrix\NR\stoppmatrix \startbmatrix\NC 1 \NR\NC \omega \NR\NC \omega^2 \NR\stopbmatrix = \startbmatrix\NC 0 \NR\NC 0 \NR\NC 0\NR\stopbmatrix$.
\item Let $A = \startbmatrix\NC 0 \NC -\tan\frac{\alpha}{2} \NR\NC\tan\frac{\alpha}{2} \NC 0\NR\stopbmatrix$ and $I$, the identity matrix of
  order $2$. Show that $I+ A = (I - A) \startbmatrix\NC  \cos\alpha \NC -\sin\alpha \NR\NC \sin\alpha \NC \cos\alpha\NR\stopbmatrix$.
\item Without using the concept of inverse of matrix, find the matrix $\startbmatrix\NC  x \NC y \NR\NC z \NC u\NR\stopbmatrix$ such that
  $\startbmatrix\NC  5 \NC -7 \NR\NC -2 \NC 3\NR\stopbmatrix \startbmatrix\NC  x \NC y \NR\NC z \NC u \NR\stopbmatrix = \startbmatrix\NC  -16 \NC -6 \NR\NC 7 \NC
      2\NR\stopbmatrix$.
\item Find $x$ so that $\startbmatrix\NC 1 \NC x \NC 1\NR\stopbmatrix\startbmatrix\NC  1 \NC 3 \NC 2 \NR\NC 0 \NC 5 \NC 1 \NR\NC 0 \NC 3 \NC
    2 \NR\stopbmatrix \startbmatrix\NC 1 \NR\NC 1 \NR\NC x\NR\stopbmatrix = 0$.
\item Prove that the product of two matrices $\startbmatrix\NC  \cos^2\theta \NC\cos\theta\sin\theta \NR\NC \cos\theta\sin\theta \NC
  \sin^2\theta\NR\stopbmatrix$ and $\startbmatrix\NC  \cos^2\phi \NC \cos\phi\sin\phi \NR\NC \cos\phi\sin\phi \NC \sin^2\phi\NR\stopbmatrix$ is a
    zero matrix when $\theta$ and $\phi$ differ by an odd multiple of $\frac{\pi}{2}$.
\item If $A = \startbmatrix\NC \cos\theta \NC -\sin\theta \NR\NC \sin\theta \NC \cos\theta \NR\stopbmatrix$, then show that $A^n
  = \startbmatrix\NC \cos n\theta \NC -\sin n\theta \NR\NC \sin n\theta \NC  \cos n\theta\NR\stopbmatrix$, where $n$ is a positive integer.
\item If $A = \startbmatrix\NC 3 \NC -4 \NR\NC 1 \NC -1\NR\stopbmatrix$, show that $A^n = \startbmatrix\NC 1 + 2n \NC -4n \NR\NC n \NC 1 -
    2n\NR\stopbmatrix$, where $n$ is a positive integer.
\item Let $A = \startbmatrix\NC  0 \NC 1 \NR\NC 0 \NC 0\NR\stopbmatrix$. Show that $(aI + bA)^n = a^nI + na^{n - 1}bA,$ where $I$ is a unit
  matrix of order $2$ and $n$ is a positive integer.
\item Under what condition is the marix equation $A^2 - B^2 = (A + B)(A - B)$ true?
\item A man buys $8$ dozens of mangoes, $10$ dozens of apples and $4$ dozens of bananas.Mangoes cost USD $18$ per dozen, apples $9$
  per dozen and bananas $6$ per dozen. Represent the quntities by a row and a column matrix. Also, find the total cost.
\item A trust fund has USD $30,000$ that is to be invested in two different types of bonds. The first bond pays $5\%$ interest per
  year and second bond pays $7\%$ interest per year. Using matrix multiplication determine how to divide USD $30,000$ among the two
  types of bonds if the turst find must obtain an annual interest of USD $2000$.
\item A store has in stock $20$ dozen shirts, $15$ dozen trousers and $25$ dozen pair of socks. If the selling prices are USD $50$
  per shirt, $90$ per trouser and $12$ per pair of socks, then find the toal amount store owner will get after selling all the
  items in the stock.
\item Co-operative store of a particular school has $10$ dozen physics books, $8$ dozen chemisty books and $5$ dozen mathematics
  books. Their selling prices are USD $8.3, 3.45, 4.5$ each respectively. Find the total amount the store owner will receive after
    selling all the books.
\item If $A = \startbmatrix\NC \cos\alpha \NC \sin\alpha \NR\NC -\sin\alpha \NC \cos \alpha\NR\stopbmatrix$, verify that $AA' = I_2 = A'A$.
\item Express the following matrix as a sum of a symmetric matrix and skew symmetric matrix $\startbmatrix\NC 1 \NC 2 \NC 4 \NR\NC 6 \NC 8 \NC 1
  \NR\NC 3 \NC 5 \NC 7\NR\stopbmatrix$.
\item Show that the following matrix is orthogonal $\startbmatrix\NC  \cos\alpha \NC \sin\alpha \NR\NC -\sin\alpha \NC \cos\alpha\NR\stopbmatrix$.
\item Show that the matrix $\frac{1}{3}\startbmatrix\NC -1 \NC 2 \NC 2 \NR\NC 2 \NC -1 \NC 2 \NR\NC 2 \NC 2 \NC -1\NR\stopbmatrix$ is orthogonal.
\item If $A = \startbmatrix\NC 1 \NC 2 \NC 3 \NR\NC 2 \NC 3 \NC 2 \NR\NC 3 \NC 3 \NC 4\NR\stopbmatrix$, find $adj(A)$.
\item For the matrix $\startbmatrix\NC \cos\alpha \NC -\sin\alpha \NC 0 \NR\NC\sin\alpha \NC \cos\alpha \NC 0 \NR\NC 0 \NC 0 \NC 1\NR\stopbmatrix$ verify
  that $A(adj\;A) = |A|I$.
\item For the matrix $A = \startbmatrix\NC 1 \NC -1 \NC 1 \NR\NC 2 \NC 3 \NC 0 \NR\NC 18 \NC 2 \NC 10\NR\stopbmatrix$, show that $A(adj\;A) = 0$.
\item Find the inverse of $\startbmatrix\NC 1 \NC 3 \NC 3 \NR\NC 1 \NC 4 \NC 3 \NR\NC 1 \NC 3 \NC 4\NR\stopbmatrix$.
\item Find the inverse of $\startbmatrix\NC 2 \NC -3 \NC 3 \NR\NC 2 \NC 2 \NC 3 \NR\NC 3 \NC -2 \NC 2\NR\stopbmatrix$.
\item Find the inverse of $\startbmatrix\NC 1 \NC -2 \NC 3 \NR\NC 0 \NC -1 \NC 4 \NR\NC -2 \NC 2 \NC 1\NR\stopbmatrix$.
\item Find the inverse of $\startbmatrix\NC 1 \NC 2 \NC 3 \NR\NC -3 \NC 5 \NC 0 \NR\NC 0 \NC 1 \NC 1\NR\stopbmatrix$.
\item If $A = \startbmatrix\NC a \NC b \NR\NC c \NC d\NR\stopbmatrix$ such that $ad - bc \neq 0$, then find the inverse of $A$.
\item If $A = \startbmatrix\NC 3 \NC 1 \NR\NC 4 \NC 0\NR\stopbmatrix, B = \startbmatrix\NC  4 \NC 0 \NR\NC 2 \NC 5 \NR\stopbmatrix$, verify thet $(AB)^{-1}
  = B^{-1}A^{-1}$.
\item If $A = \startbmatrix\NC 1 \NC \tan x \NR\NC -\tan x \NC 1\NR\stopbmatrix$, show that $AA^{-1} = \startbmatrix\NC \cos 2x \NC -\sin 2x
    \NR\NC \sin 2x \NC \cos 2x\NR\stopbmatrix$.
\item If $A = \startbmatrix\NC  3 \NC 2 \NR\NC 7 \NC 5 \NR\stopbmatrix$ and $B = \startbmatrix\NC  6 \NC 7 \NR\NC 8 \NC 9 \NR\stopbmatrix$, find $(AB)^{-1}$.
\item Solve the following system of equations by matrix method: $3x - 2y = 7$ and $5x + 3y = 1$.
\item Solve the following system of equations by matrix method: $5x - 7y = 2$ and $7x -5y = 3$.
\item Solve the following system of equations by matrix method: $2x - 3y + 3z = 1,\ 2x + 2y + 3z = 2$ and $3x -2y + 2z = 3$.
\item Solve the following system of equations by matrix method: $x + y + z = 3,\ 2x - y + z = 2$ and $x - 2y + 3z = 2$.
\item Solve the following system of equations by matrix method: $2x - y + 3z = 9,\ x + y + z = 6$ and $x - y + z = 2$.
\item Examine following system of equations for consistency: $2x + 3y = 5$ and $6x + 9y = 10$.
\item Examine following system of equations for consistency: $4x - 2y = 3$ and $6x - 3y = 5$.
\stopitemize
