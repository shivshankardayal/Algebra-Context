% -*- mode: context; -*-
\chapter{Mathematical Induction}
\startitemize[n, 1*broad]
\item Let $P(n) = 1^2 + 2^2 + \cdots + n^2 = \frac{n(n + 1)(2n + 1)}{6}$.

  Putting $n = 1$, we see $P(1) = 1 = \frac{1.2.3}{6} = 1$. So $P(1)$ is true.

  Let it be true for $n = k$. Now for $n = k + 1$,

  $P(k + 1) = 1^2 + 2^2 + \cdots + k^2 + (k + 1)^2 =  \frac{k(k + 1)(2k + 1)}{6} + (k + 1)^2$

  $= \frac{(k +
    1)(2k^2 + k + 6k + 6)}{6} = \frac{(k + 1)(k + 2)(2k + 3)}{6} = P(k + 1)$.

  Thus, by mathematical induction, the result.
\item Let $P(n) = \frac{1}{1.2} + \frac{1}{2.3} + \cdots + \frac{1}{n(n + 1)} = \frac{n}{n + 1}$.

  $P(1) = \frac{1}{1.2} = \frac{1}{1 + 1} = P(1)$, which is true for $n = 1$. Let it be true for $n = k$

  $\Rightarrow P(k) = \frac{1}{1.2} + \frac{1}{2.3} + \cdots + \frac{1}{k(k + 1)} = \frac{k}{k + 1}$.

  Adding $\frac{1}{(k + 1)(k + 2)}$, on both sides, we get

  $P(k + 1) = \frac{1}{1.2} + \frac{1}{2.3} + \cdots + \frac{1}{k(k + 1)} + \frac{1}{(k + 1)(k + 2)}$

  $= \frac{k}{(k + 1)} + \frac{1}{(k + 1)(k + 2)} = \frac{k^2 + 2k + 1}{(k + 1)(k + 2)} = \frac{k + 1}{k + 2} =
  P(k +1)$.

  Hence, by mathematical induction, the result.
\item Let $P(n) = 1^3 + 2^3 + \cdots + n^3 = \left[\frac{n(n + 1)}{2}\right]^2$

  $P(1) = 1^3 = 1 = \left(\frac{1.2}{2}\right)^3 = 1$, which is true for $n = 1$. Let it be true for $n =
  k$.

  $\Rightarrow P(k) = 1^3 + 2^3 + \cdots + k^3 = \left[\frac{k(k + 1)}{2}\right]^2$

  Adding $(k + 1)^3$ to both sides, we get

  $P(k + 1) = 1^3 + 2^3 + \cdots + k^3 + (k + 1)^3= \left[\frac{k(k + 1)}{2}\right]^2 + (k + 1)^3$

  $= \frac{(k
    + 1)^2[k^2 + 4k + 4]}{4} = \left(\frac{(k + 1)(k + 2)}{2}\right)^2 = P(k + 1)$.

  Hence, by mathematical induction, the result.
\item Let $P(n) = \frac{1}{a + d} + \frac{a}{(a + d)(a + 2d)} + \cdots + \frac{a}{[a + (n - 1)d](a + nd)} =
  \frac{n}{a + nd}$

  $P(1) = \frac{1}{a + d} = $R.H.S., which is true for $n = 1$. Let it be true for $n = k$

  $\Rightarrow P(k) = \frac{1}{a + d} + \frac{a}{(a + d)(a + 2d)} + \cdots + \frac{a}{[a + (k - 1)d](a +
    kd)} = \frac{k}{a + kd}$.

  Adding $\frac{a}{[a + kd][a + (k + 1)d]}$ to both sides, we get

  $P(k + 1) = \frac{1}{a + d} + \frac{a}{(a + d)(a + 2d)} + \cdots + \frac{a}{[a + (k - 1)d](a +
    kd)} + \frac{a}{[a + kd][a + (k + 1)d]}$

  $= \frac{k}{a + kd} + \frac{a}{[a + kd][a + (k + 1)d]} = \frac{ka
  + k(k + 1)d + a}{[a + kd][a + (k + 1)d]} = \frac{(k + 1)(a + kd)}{[a + kd][a + (k + 1)d]} = \frac{k + 1}{a
  + (k + 1)d} = P(k + 1)$

  Hence, by mathematical induction, the result.
\item Let $P(n) = \frac{1}{1.2.3} + \frac{1}{2.3.4} + \cdots + \frac{1}{n(n + 1)(n + 2)} = \frac{n(n +
  3)}{4(n + 1)(n + 2)}\forall n\in \mathbb{N}$

  $P(1) = \frac{1}{1.2.3} = \frac{1.4}{4.2.3} = \frac{1}{2.3}$, which is true for $n = 1$. Let it be true
  for $n = k$.

  $\Rightarrow P(k) = \frac{1}{1.2.3} + \frac{1}{2.3.4} + \cdots + \frac{1}{k(k + 1)(k + 2)} = \frac{k(k +
    3)}{4(k + 1)(k + 2)}$

  Adding $\frac{1}{(k + 1)(k + 2)(k + 3)}$ to both sides, we get

  $P(k + 1) = \frac{1}{1.2.3} + \frac{1}{2.3.4} + \cdots + \frac{1}{k(k + 1)(k + 2)} + \frac{1}{(k + 1)(k +
    2)(k + 3)}$

  $= \frac{k(k + 3)}{4(k + 1)(k + 2)} + \frac{1}{(k + 1)(k + 2)(k + 3)}$

  $= \frac{k(k + 3)^2 + 4}{4(k + 1)(k + 2)(k ++ 3)} = \frac{(k + 1)^2(k + 4)}{4(k + 1)(k + 2)(k + 3)}$

  $= \frac{(k + 1)(k + 4)}{4(k + 2)(k ++ 3)} = P(k + 1)$.

  Hence, by mathematical induction, the result.
\item Let $P(n) = 1.3 + 2.3^2 + 3.3^3 + \cdots + n.3^n = \frac{(2n - 1)3^{n + 1} + 3}{4}$

  $P(1) = 1.3 = 3 = \frac{(2 - 1).3^2 + 3}{4} = 3$, which is true for $n = 1$. Let it be true for $n = k$.

  Adding $(k + 1)3^{k + 1}$ to both sides, we get

  $\Rightarrow P(k + 1) = 1.3 + 2.3^2 + 3.3^3 + \cdots + k.3^k + (k + 1).3^{k + 1} = \frac{(2k - 1)3^{k + 1}
    + 3}{4}$\

  $+ (k
  + 1).3^{k + 1} = \frac{(2k - 1).3^{k + 1} + 3 + (4k + 4).3^{k + 1}}{4}$

  $= \frac{(6k + 3).3^{k + 1}}{4} = \frac{[2(k + 2) - 3].3^{k + 1}}{4} = P(k + 1)$.

  Hence, by mathematical induction, the result.
\item Let $P(n) = 1 + 4 + 7 + \cdots + 3n - 2 = \frac{n(3n - 1)}{2}$.

  $P(1) =  1 = \frac{1.(3 - 1)}{2} = 1$, which is true for $n = 1$. Let it be true for $n = k$.

  $P(k) = 1 = 4 + 7 + \cdots + 3k - 2 = \frac{k(3k - 1)}{2}$

  Adding $3k + 1$ to both sides, we get

  $\Rightarrow P(k + 1) = 1 + 4 + 7 + \cdots + 3k - 2 + 3k + 1 = \frac{k(3k - 1)}{2} + 3k + 1$

  $= \frac{3k^2 -
  k + 6k + 2}{2} = \frac{3k^2 + 5k + 2}{2} = \frac{(k + 1)(3k + 2)}{2} = \frac{(k + 1)[3(k + 1) - 1]}{2} =
  P(k + 1)$.

  Hence, by mathematical induction, the result.
\item Let $P(n) = 1^2 + 3^2 + 5^2 + \cdots + (2n - 1)^2 = \frac{n(2n - 1)(2n + 1)}{3}$.

  $P(1) = 1^2 = 1 = \frac{(12 - 1)(2 + 1)}{3} = 1$, which is true for $n = 1$. Let it be true for $n = k$

  $P(k) = 1^2 + 3^2 + 5^2 + \cdots + (2k - 1)^2 = \frac{k(2k - 1)(2k + 1)}{3}$.

  Adding $(2k + 1)^2$ to both sides, we get

  $\Rightarrow P(k + 1) = 1^2 + 3^2 + 5^2 + \cdots + (2k - 1)^2 + (2k + 1)^2 = \frac{k(4k^2 - 1)}{3} + (2k +
  1)^2$

  $= \frac{4k^3 - k + 12k^2 + 12k + 3}{3} = \frac{4k^3 + 122k^2 + 11k + 3}{3} = \frac{(k + 1)(2k + 1)(2k
    + 3)}{3} = P(k + 1)$.

  Hence, by mathematical induction, the result.
\item Let $P(n) = 1 - 3^2 + 5^2 - 7^2 + \cdots + (4n - 3)^2 - (4n - 1)^2 = -8n^2$

  $P(1) = 1 - 3^2 = -8 = -8.1^2$, which is true for $n = 1$. Let it be true for $n = k$.

  $P(k) = 1 - 3^2 + 5^2 - 7^2 + \cdots + (4k - 3)^2 - (4k - 1)^2 = -8k^2$

  Adding $(4k + 1)^2 - (4k + 2)^2$ to both sides, we get

  $P(k + 1) = 1 - 3^2 + 5^2 - 7^2 + \cdots + (4k - 3)^2 - (4k - 1)^2 + (4k + 1)^2 - (4k - 3)^2$

  $= -8k^2 + (4k + 1)^2 - (4k + 3)^2 = -8k^2 -16k - 8 = -8(k + 1)^2$.

  Hence, by mathematical induction, the result.
\item Let $P(n) = 3.6 + 6.9 + 9.12 + \cdots + 3n(3n + 3) = 3n(n + 1)(n + 2)$.

  $P(1) = 3.6 = 3.1(1 + 1)(1 + 2) = 3.6$, which is true for $n = 1$. Let it be true for $n = k$

  $P(k) = 3.6 + 6.9 + 9.12 + \cdots + 3k(3k + 3) = 3k(k + 1)(k + 2)$

  Adding $3(k + 1)[3(k + 1) + 3]$ to both sides, we get

  $P(k + 1) = 3.6 + 6.9 + 9.12 + \cdots + 3k(3k + 3) + 3(k + 1)[3(k + 1) + 3]$

  $= 3k(k + 1)(k + 2) + 3(k + 1).3(k + 2) = 3(k + 1)(k + 2)(k + 3) = P(k + 1)$.

  Hence, by mathematical induction, the result.
\item We have to prove that $1^3 = 1, 2^3 = 3 + 5, 3^3 = 7 + 9 + 11, 4^3 = 13 + 15 + 17 + 19$.

  First term contains one term, second terms contains two terms and so on. Hence, $k$th term will contain
  $k$ terms.

  Sum of no.\ of terms till $k$th term is $1 + 2 + \cdots + k = \frac{k(k + 1)}{2}$.

  And, hence $(k + 1)$th term will begin with $1 + \left(\frac{k(k + 1)}{2}\right)2 = k^2 + k + 1$ and will
  contain $k + 1$ terms with a c.d. of $2$. Let it be true for $P(k)$ i.e. $t_k = k^3$.

  Thus, $t_{k + 1} = \frac{k + 1}{2}[2(k^2 + k + 1) + (k + 1 - 1)2] = \frac{k + 1}{2}[2k^2 + 4k + 2] = (k +
  1)^3$.

  Hence, by mathematical induction, the result.
\item Let $P(n) = \displaystyle\sum_{r = 1}^nr.C_r^^n = n.2^{n - 1}$.

  $P(1) = 1.C_0^^1 = 1 = 1.2^{1 - 1} = 1$. Hence, it is true for $n = 1$. Let it be true for $n = k$.

  $\Rightarrow C_1^^k + 2.C_2^^k + \cdots + k.C_k^^k = k.2^{k - 1}$

  $P(k + 1) = C_1^^{k + 1} + 2.C_2^^{k + 1} + \cdots + (k + 1).C_{k + 1}^^{k + 1}$

  $= (C_0^^k + C_1^^k) + 2(C_1^^k + C_2^^k) + \cdots + (k + 1)(C_k^^k + 0)$

  $= (C_0^^k + 2C_1^^k + \cdots + (k + 1)C_k^^k) + (C_1^^k + 2.C_2^^k + \cdots + k.C_k^^k)$

  $= 2^k + k.2^{k - 1} + k.2^{k - 1} = (k + 1).2^k = P(k + 1)$.

  Hence, by mathematical induction, the result.
\item Let $P(n) = \displaystyle\sum_{r = 1}^nr(2r + 1) = \frac{n(n + 1)(4n + 5)}{6}$.

  $P(1) = 1.(2 + 1) = 3 = \frac{1.2.9}{6} = 3$. Hence, it it true for $n = 1$. Let it be true for $n = k$.

  $\Rightarrow \displaystyle\sum_{r = 1}^kr(2r + 1) = \frac{k(k + 1)(4k + 5)}{6}$.

  $P(k + 1) = \displaystyle\sum_{r = 1}^{k + 1}r(2r + 1) = \frac{k(k + 1)(4k + 5)}{6} + (k + 1)(2k + 3)$

  $= \frac{(k + 1)}{6}\left[\frac{4k^2 + 5k + 12k + 18}{}\right] = \frac{(k + 1)(k + 2)[4(k + 1) + 5]}{6}$.

  Hence, by mathematical induction, the result.
\item Let $P(n) = 1.2.3 + 2.3.4 + 3.4.5 + \cdots + n(n + 1)(n + 2) = \frac{n(n + 1)(n + 2)(n + 3)}{4}$.

  $P(1) = 1.2.3 = 6 = \frac{1.2.3.4}{4} = 6$, which is true for $n = 1$. Let it be true for $n = k$.

  $\Rightarrow 1.2.3 + 2.3.4 + 3.4.5 + \cdots + k(k + 1)(k + 2) = \frac{k(k + 1)(k + 2)(k + 3)}{4}$

  Adding $(k + 1)(k + 2)(k + 3)$ to both sides, we get

  $P(k + 1) = 1.2.3 + 2.3.4 + 3.4.5 + \cdots + k(k + 1)(k + 2) + (k + 1)(k + 2)(k + 3)$

  $= \frac{k(k + 1)(k + 2)(k + 3)}{4} + (k + 1)(k + 2)(k + 3)$

  $= \frac{(k + 1)(k + 2)(k + 3)(k + 4)}{4}$.

  Hence, by mathematical induction, the result.
\item Let $P(n) = \frac{1}{1.4} + \frac{1}{4.7} + \frac{1}{7.10} + \cdots + \frac{1}{(3n - 2)(3n + 1)} =
  \frac{n}{3n + 1}$

  $P(1) = \frac{1}{4} = \frac{1}{3 + 1} = \frac{1}{4}$, which is true for $n = 1$. Let it be true for $n =
  k$

  $\Rightarrow P(k) = \frac{1}{1.4} + \frac{1}{4.7} + \frac{1}{7.10} + \cdots + \frac{1}{(3k - 2)(3k + 1)} =
  \frac{k}{3k + 1}$

  Adding $\frac{1}{(3k + 1)(3k + 4)}$ to both sides, we get

  $P(k + 1) = \frac{1}{1.4} + \frac{1}{4.7} + \frac{1}{7.10} + \cdots + \frac{1}{(3n - 2)(3n + 1)} +
  \frac{1}{(3k + 1)(3k + 4)}$

  $= \frac{k}{3k + 1} + \frac{1}{(3k + 1)(3k + 4)} = \frac{3k^2 + 4k + 1}{(3k + 1)(3k + 4)} =\frac{k + 1}{3k
    + 4} = P(k + 1)$

  Hence, by mathematical induction, the result.
\item Let $P(n) = 7 + 77 + 777 + \cdots + \underbrace{7\ldots77}_{n\text{~digits}} = \frac{7}{81}(10^{n + 1} - 9n -
  10)$

  $P(1) = 7 = \frac{7}{81}(10^2 - 9 - 10) = 7$, which is true for $n = 1$. Let it be true for $n = k$

  $\Rightarrow 7 + 77 + 777 + \cdots + \underbrace{7\ldots77}_{k\text{~digits}} = \frac{7}{81}(10^{k + 1} - 9k -
  10)$

  Adding $\underbrace{7\ldots77}_{k + 1~\text{digits}}$ to both sides, we get

  $P(k + 1) = 7 + 77 + 777 + \cdots + \underbrace{7\ldots77}_{k\text{~digits}} + \underbrace{7\ldots77}_{k +
    1\text{~digits}}$

  $= \frac{7}{81}(10^{k + 1} - 9k - 10) + \underbrace{7\ldots77}_{k + 1\text{~digits}} = \frac{7}{81}(10^{k
    + 1}(10^{k + 1} - 9k - 10)) + \frac{7}{9}(10^{k + 1} - 1)$

  $= \frac{7}{9}\left[\frac{10^{k + 1} - 9k - 10 + 9.10^{k + 1} - 9}{9}\right] = \frac{7}{81}[10^{k + 2} -
    9(k + 1) - 10]$.

  Hence, by mathematical induction, the result.
\item Let $P(n) = 1 + \frac{1}{1 + 2} + \frac{1}{1 + 2 + 3} + \cdots + \frac{1}{1 + 2 + 3 + \cdots + n} =
  \frac{2n}{n + 1}$

  $P(1) = 1 = \frac{2.1}{1 + 1} = 1$, which is true for $n = 1$. Let it be true for $n = k$

  $\Rightarrow P(k) = 1 + \frac{1}{1 + 2} + \frac{1}{1 + 2 + 3} + \cdots + \frac{1}{1 + 2 + 3 + \cdots + k}
  = \frac{2k}{k + 1}$

  Adding $\frac{1}{1 + 2 + 3 + \cdots + (k + 1)}$ to both sides,we get

  $P(k + 1) = 1 + \frac{1}{1 + 2} + \frac{1}{1 + 2 + 3} + \cdots + \frac{1}{1 + 2 + 3 + \cdots + n} +
  \frac{1}{1 + 2 + 3 + \cdots + (k  + 1)} = \frac{2k}{k + 1} + \frac{1}{1 + 2 + 3 \cdots + (k + 1)}$

  $= \frac{2k}{k + 1} + \frac{2}{(k + 1)(k + 2)} = \frac{2}{k + 1}.\frac{k^2 + 2k + 1}{k + 2} = \frac{2(k +
    1)}{k + 2}$.

  Hence, by mathematical induction, the result.
\item Let $P(n) = \left(1 - \frac{1}{2^2}\right)\left(1 - \frac{1}{3^2}\right)\cdots\left(1 - \frac{1}{(n +
  1)^2}\right) = \frac{n + 2}{2n + 2}$

  $P(1) = 1 - \frac{1}{2^2} = \frac{3}{4} = \frac{1 + 2}{2.1 + 2} = \frac{3}{4}$, which is true for $n =
  1$. Let it be true for $n = k$

  $\Rightarrow P(k) = \left(1 - \frac{1}{2^2}\right)\left(1 - \frac{1}{3^2}\right)\cdots\left(1 - \frac{1}{(k +
    1)^2}\right) = \frac{k + 2}{2k + 2}$

  Multiplying both sides with $1 - \frac{1}{(k + 2)^2}$, we get

  $P(k + 1) = \left(1 - \frac{1}{2^2}\right)\left(1 - \frac{1}{3^2}\right)\cdots\left(1 - \frac{1}{(k +
    1)^2}\right)\left(1 - \frac{1}{(k + 2)^2}\right) = \frac{k + 2}{2k + 2}\left(1 - \frac{1}{(k +
    2)^2}\right)$

  $= \frac{k + 2}{2k + 2}.\frac{k^2 + 4k + 3}{(k + 2)^2} = \frac{k + 3}{2k + 4}$.

  Hence, by mathematical induction, the result.
\item Let $P(n) = 1.3 + 2.3^2 + \cdots + n.3^n = \frac{(2n - 1)3^{n + 1} + 3}{4}$.

  $P(1) = 3 = \frac{(2.1 - 1).3^{1 + 1} + 3}{4} = \frac{12}{4} = 3$, which is true for $n = 1$. Let it be
  true for $n = k$

  $\Rightarrow P(k) = 1.3 + 2.3^2 + \cdots + k.3^k = \frac{(2k - 1)3^{k + 1} + 3}{4}$

  Adding $(k + 1).3^{k + 1}$, to both sides, we get

  $P(k + 1) = 1.3 + 2.3^2 + \cdots + k.3^k + (k + 1).3^{k + 1} = \frac{(2k - 1)3^{k + 1} + 3}{4} + (k +
  1).3^{k + 1}$

  $= \frac{(2k - 1).3^{k + 1} + 3 + (4k + 4).3^{k + 1}}{4} = \frac{(6k + 3).3^{k + 1} = 3}{4} = \frac{[2(k +
      1) - 1].3^{k + 2} + 3}{4}$

  Hence, by mathematical induction, the result.
\item Let $P(n) = \cos\alpha + \cos2\alpha + \cdots + \cos n\alpha =
  \sin\frac{n\alpha}{2}\csc\frac{\alpha}{2}\cos\frac{(n + 1)\alpha}{2}$.

  $P(1) = \cos\alpha = \sin\frac{\alpha}{2}\csc\frac{\alpha}{2}\cos\frac{1 + 1}{2}\alpha = \cos\alpha$,
  which is true for $n = 1$. Let it be true for $n = k$.

  $\Rightarrow P(k) = \cos\alpha + \cos2\alpha + \cdots + \cos k\alpha =
  \sin\frac{k\alpha}{2}\csc\frac{\alpha}{2}\cos\frac{(k + 1)\alpha}{2}$

  Adding $\cos(k + 1)\alpha$, to both sides, we get

  $P(k + 1) = = \cos\alpha + \cos2\alpha + \cdots + \cos k\alpha + \cos(k + 1)\alpha =
  \sin\frac{k\alpha}{2}\csc\frac{\alpha}{2}\cos\frac{(k + 1)\alpha}{2} + \cos(k + 1)\alpha$

  $= \frac{1}{2}\csc\frac{\alpha}{2}\left[2\sin\frac{k\alpha}{2}\cos\frac{(k + 1)\alpha}{2} +
    2\sin\frac{\alpha}{2}\cos(k + 1)\alpha\right]$

  $=\frac{1}{2}\csc\frac{\alpha}{2}\left[\sin\frac{(2k + 1)\alpha}{2} - \sin\frac{\alpha}{2} + \sin\frac{(2k
      + 3)\alpha}{2} - \sin\frac{(2k + 1)\alpha}{2}\right]$

  $= \csc\frac{\alpha}{2}\cos\frac{(k + 2)\alpha}{2}\sin\frac{(k + 1)\alpha}{2}$

  Hence, by mathematical induction, the result.
\item Let $P(n) = \tan\alpha + 2\tan2\alpha + 2^2\tan2^2\alpha + \cdots + 2^{n - 1}\tan2^{n - 1}\alpha =
  \cot\alpha - 2^n\cot2^n\alpha$

  $P(1) = \tan\alpha = \cot\alpha - 2\cot2\alpha = \frac{1}{\tan\alpha} - \frac{2}{\tan2\alpha} =
  \frac{1}{\tan\alpha} - \frac{1 - \tan^2\alpha}{\tan\alpha} = \tan\alpha$,

  which is true for $n = 1$. Let it be true for $n = k$

  $\Rightarrow P(k) = \tan\alpha + 2\tan2\alpha + 2^2\tan2^2\alpha + \cdots + 2^{k - 1}\tan2^{k - 1}\alpha =
  \cot\alpha - 2^k\cot2^k\alpha$

  Adding $2^k\tan2^k\alpha$, to both sides, we get

  $P(k + 1) = \tan\alpha + 2\tan2\alpha + 2^2\tan2^2\alpha + \cdots + 2^{k - 1}\tan2^{k - 1}\alpha +
  2^k\tan2^k\alpha$

  $= \cot\alpha - 2^k\cot2^k\alpha + 2^k\tan2^k\alpha = \cot\alpha - 2^k\left(\cot2^k\alpha -
  \tan2^k\alpha\right)$

  $= \cot\alpha - 2^{k+ 1}\left(\frac{1 - \tan2^{k + 1}\alpha}{2\tan2^k\alpha}\right) = \cot\alpha - 2^{k +
    1}\cot2^{k + 1}\alpha$.

  Hence, by mathematical induction, the result.
\item Let $P(n) = \tan^{-1}\frac{1}{3} + \tan^{-1}\frac{1}{7} + \cdots + \tan^{-1}\frac{1}{n^2 + n + 1} =
  \tan^{-1}\frac{n}{n + 2}$

  $P(1) = \tan^{-1}\frac{1}{3} = \tan^{-1}\frac{1}{1 + 2} = \tan^{-1}\frac{1}{3}$, which is true for n =
  1. Let it be true for $n = k$

  $\Rightarrow P(k) = \tan^{-1}\frac{1}{3} + \tan^{-1}\frac{1}{7} + \cdots + \tan^{-1}\frac{1}{k^2 + k + 1} =
  \tan^{-1}\frac{k}{k + 2}$

  Adding $\tan^{-1}\frac{1}{k^2 + 3k + 3}$, to both sides, we get

  $P(k + 1) = \tan^{-1}\frac{1}{3} + \tan^{-1}\frac{1}{7} + \cdots + \tan^{-1}\frac{1}{k^2 + k + 1} +
  \tan^{-1}\frac{1}{k^2 + 3k + 3}$

  $=\tan^{-1}\frac{k}{k + 2} + \tan^{-1}\frac{1}{k^2 + 3k + 3} = \tan^{-1}\frac{\frac{k}{k + 2} +
    \frac{1}{k^2 + 3k + 3}}{1 - \frac{k}{k + 2}.\frac{1}{k^2 + 3k + 3}} = \tan^{-1}\frac{k^3 + 3k^2 + 3k + k
    + 2}{k^3 + 5k^2 + 9k + 6 - k}$

  $=\tan^{-1}\frac{k^3 + 3k^2 + 4k + 2}{k^3 + 5k^2 + 8k + 6} = \tan^{-1}\frac{k + 1}{k + 3}$.

  Hence, by mathematical induction, the result.
\item $u_3 = u_2 + u_1$. Substituting $n = 3$ in the given formula $u_3 =
  \frac{1}{\sqrt{5}}\left[\left(\frac{1 + \sqrt{5}}{2}\right)^3 - \left(\frac{1 -
      \sqrt{5}}{2}\right)^3\right]$

  $= \frac{1}{\sqrt{5}}\left[\frac{1 + 3\sqrt{5} + 15 + 5\sqrt{5}}{8} - \frac{1 - 3\sqrt{5} + 15 -
      5\sqrt{5}}{8}\right] = \frac{1}{\sqrt{5}}\frac{16\sqrt{5}}{8}\right = 2 = u_1 + u_2$.

  Thus, the relation holds for $n = 3$. Similarly, we can prove that it holds for $m = 1, 2$. Let it hold
  for $n = k$ and $k + 1$.

  $\Rightarrow u_k = \frac{1}{\sqrt{5}}\left[\left(\frac{1 + \sqrt{5}}{2}\right)^k - \left(\frac{1 -
      \sqrt{5}}{2}\right)^k\right]$ and $u_{k + 1} = \frac{1}{\sqrt{5}}\left[\left(\frac{1 +
      \sqrt{5}}{2}\right)^{k + 1} - \left(\frac{1 - \sqrt{5}}{2}\right)^{k + 1}\right]$.

  $u_{k + 2} = u_k + u_{k + 1} = \frac{1}{\sqrt{5}}\left[\left(\frac{1 + \sqrt{5}}{2}\right)^k\left(1 +
    \frac{1 + \sqrt{5}}{2}\right) - \left(\frac{1 + \sqrt{5}}{2}\right)^k\left(1 - \frac{1 -
      \sqrt{5}}{2}\right)\right]$

  $= \frac{1}{\sqrt{5}}\left[\left(\frac{1 + \sqrt{5}}{2}\right)^k\left(\frac{1 + \sqrt{5}}{2}\right)^2 -
    \left(\frac{1 - \sqrt{5}}{2}^k\right)\left(\frac{1 - \sqrt{5}}{2}\right)^2\right] = u_{k + 2}$.

  Hence, by mathematical induction, the result.
\item Let $P(n) = p^{n + 1} + (p + 1)^{2n - 1}$.

  $P(1) = p^2 + p + 1$, which is divisible by $p^2 + p + 1$, and hence, our statement is true for $n =
  1$. Let it be true for $n = k$.

  $\Rightarrow P(k) = p^{k + 1} + (p + 1)^{2k - 1}$ is divisible by $p^2 + p + 1$ i.e. $p^{k + 1} + (p + 1)^{2k + 1} =
  (p^2 + p + 1)Q(p)$, where $Q(p)$, is a polynomial of $p$.

  $P(k + 1) = p^{k + 2} + (p + 1)^{2k + 1} = p.p^{k + 1} + (p + 1)^2(p + 1)^{2k - 1}$

  $\therefore P(k) = (p^2 + p + 1)^2Q(k)$, making it divisible by $p^2 + p + 1$.

  Hence, by mathematical induction, the result.
\item Let $P(n) = 2^n > 2n + 1$, where $n > 2$

  $P(3) = 2^3 = 8 > 2.3 + 1 = 7$, hence, it is true for $n = 3$. Let it be true for $n = k$.

  $\Rightarrow P(k) = 2^k > 2k + 1$, where $k> 2$.

  $P(k + 1) = 2^{k + 1} = 2.2^k = 4k + 2 = 2k + 2k + 2 \because k \geq 3\therefore 2k + 2 > 3$, making our
  statement true for $n = k + 1$.

  hence, by mathematical induction, the result.
\item Let $P(n) = 2^n > n^3$, where $n \geq 10$.

  $P(10) = 2^{10} = 1024 > 10^3 = 1000$, hence, it is true for $n = 3$. Let it be true for $n = k$.

  $\Rightarrow P(k) = 2^k > k^3$

  $P(k + 1) = 2^{k + 1} > 2.k^3 >(k + 1)^3 \Rightarrow k^3 - 3k^2 - 3k - 1 > 0 \Rightarrow (k - 1)^3 - 6k > 0$

  Let $k = 10 + a$, where $a\geq 0$, hence, $(9 + a)^3 - 60 - 6a = 669 + 183a + 27a^2 + a^3 > 0$

  hence, by mathematical induction, the result.
\item Given, $n > 1$, so we start with $n = 2$. $\Rightarrow \tan2\alpha = \frac{2\tan\alpha}{1 -
  \tan^2\alpha}> 2\tan\alpha\because1 - \tan^2\alpha < 1$,

  which is true for $n = 2$. Let the statement for $n = k$

  $\Rightarrow \tan k\alpha > k\tan\alpha$. For $n = k + 1$

  $\tan(k + 1)\alpha = \frac{\tan\alpha + \tan k\alpha}{1 - \tan\alpha.\tan k\alpha}> \frac{k\tan\alpha +
    \tan\alpha}{1 - \tan k\alpha\tan\alpha} > (k + 1)\tan\alpha\because 1 - \tan\alpha\tan k\alpha < 1$

  Hence, by mathematical induction, the result.
\item Let $P(n) = n^4 < 10^n\;\forall\;n\geq 2$

  For $n = 1, P(2) = 2^4 < 10^2 \Rightarrow 16 < 100$, which is true for $n = 2$. Let $P(k)$ be true
  i.e. $k^4 < 10^k$.

  We have to prove that $P(k + 1)$ is true i.e. $(k + 1)^4 < 10^{k + 1}$.

  Clearly, $10^{k + 1} > 10k^4$. Now, $\frac{10k^4}{(k + 1)^4} = 10\left(\frac{k}{k + 1}\right)^4$

  $\because k\geq 2\Rightarrow \left(\frac{k}{k + 1}\right)^4\geq \frac{2^4}{3^4}\Rightarrow
  10\left(\frac{k}{k + 1}\right)^4 \geq 10.\frac{16}{81} > 1$.

  Thus, $10^{k + 1} > (k + 1)^4$. Hence, by mathematical induction, the result.
\item Let $P(n) = 1^3 + 3^3 + \cdots + (2n - 1)^3 = n^2(2n^2 - 1)$.

  $P(1) = 1^3 = 1 = 1^2(2.1^1 - 1) = 1$, which is true for $n = 1$. Let it be true for $n = k$.

  $P(k) = 1^3 + 3^3 + \cdots + (2k - 1)^3 = k^2(2k^2 - 1)$

  Adding $(2k + 1)^3$, to both sides, we get

  $P(k + 1) = 1^3 + 3^3 + \cdots + (2k - 1)^3 + (2k + 1)^3 = k^2(2k^2 - 1) + (2k + 1)^3$

  $= 2k^4 - k^2 + 8k^2 + 12k^2 + 6k + 1 = (k + 1)^2[2(k + 1)^2 - 1]$.

  Hence, by mathematical induction, the result.
\item Let $P(n) = 3.2^2 + 3^3.2^3 + \cdots + 3^n.2^{n + 1} = \frac{12}{5}(6^n - 1)$.

  $P(1) = 3.2^2 = 12 = \frac{12}{5}(6^1 - 1) = 12$, which is true for $n = 1$. Let it be true for $n = k$.

  $\Rightarrow P(k) = 3.2^2 + 3^3.2^3 + \cdots + 3^k.2^{k + 1} = \frac{12}{5}(6^k - 1)$.

  Adding $3^{k+ 1}.2^{k + 2}$, to both sides, we get

  $P(k + 1) = 3.2^2 + 3^3.2^3 + \cdots + 3^k.2^{k + 1} + 3^{k + 1}.2^{k + 2} = \frac{12}{5}(6^k - 1) + 3^{k
    + 1}.2^{k + 2}$

  $= \frac{12}{5}(6^k - 1) + 2.6^{k + 1} = \frac{(2.6^{k + 1} - 12 + 10.6^{k + 1})}{5} = \frac{12}{5}(6^{k +
    1} - 6)$.

  Hence, by mathematical induction, the result.
\item Let $P(n) = \frac{1}{1.4} + \frac{1}{4.7} + \cdots + \frac{1}{(3n - 2)(3n + 1)} = \frac{n}{3n+ 1}$.

  $P(1) = \frac{1}{4} = \frac{1}{3.1 + 1} = \frac{1}{4}$, which is true for $n = 1$. Let it be true for $n =
  k$.

  $\Rightarrow P(k) = \frac{1}{1.4} + \frac{1}{4.7} + \cdots + \frac{1}{(3k - 2)(3k + 1)} = \frac{k}{3k+ 1}$

  Adding $\frac{1}{(3k + 1)(3k + 4)}$, to both sides we get

  $P(k + 1) = \frac{1}{1.4} + \frac{1}{4.7} + \cdots + \frac{1}{(3k - 2)(3k + 1)} + \frac{1}{(3k + 1)(3k +
    4)}$

  $= \frac{k}{3k+ 1} + \frac{1}{(3k + 1)(3k + 4)} = \frac{3k^2 + 4k + 1}{(3k + 1)(3k + 4)} = \frac{k +
    1}{3(k + 1) + 1}$.

  Hence, by mathematical induction, the result.
\item Let $P(n) = (\cos\theta + i\sin\theta)^n = \cos n\theta + i\sin n\theta$.

  Clearly, it is true for $n = 1$. Let it be true for $n = k$, i.e.

  $P(k) = (\cos\theta + i\sin\theta)^{k} = \cos k\theta + i\sin k\theta$.

  $P(k + 1) = (\cos\theta + i\sin\theta)^{k + 1} = (\cos k\theta + i\sin k\theta)(\cos\theta + i\sin\theta)$

  $= [\cos k\theta\cos\theta - \sin k\theta\sin\theta + i(\cos\theta\sin k\theta + \cos k\theta\sin\theta)]$

  $= \cos(k + 1)\theta + i\sin(k + 1)\theta$.

  Hence, by mathematical induction, the result.
\item Let $P(n) = \cos\theta.\cos2\theta\ldots\cos2^{n - 1}\theta = \frac{\sin2^n\theta}{2^n\sin\theta}$.

  $P(1) = \cos\theta = \frac{\sin2\theta}{2\sin\theta} = \frac{2\sin\theta\cos\theta}{2\sin\theta} =
  \cos\theta$, which is true for $n = 1$.

  Let it be true for $n = k$.

  $\Rightarrow P(k) = \cos\theta.\cos2\theta\ldots\cos2^{k - 1}\theta = \frac{\sin2^k\theta}{2^k\sin\theta}$

  Multiplying both sides with $cos2^k\theta$, we get

  $P(k + 1) = \cos\theta.\cos2\theta\ldots\cos2^k\theta = \frac{\sin2^k\theta}{2^k\sin\theta}.\cos2^k\theta=
  \frac{2\sin2^k\theta\cos2^k\theta}{2^{k + 1}\sin\theta} = \frac{\sin2^{k + 1}\theta}{2^{k + 1}\sin\theta}$

  Hence, by mathematical induction, the result.
\item Let $P(n) = \sin\alpha + \sin2\alpha + \cdots + \sin n\alpha = \frac{\sin
  \frac{n\alpha}{2}}{\sin\frac{\alpha}{2}}\sin\frac{n + 1}{2}\alpha$

  $P(1) = \sin\alpha = \frac{\sin\frac{\alpha}{2}}{\sin\frac{\alpha}{2}}\sin\frac{1 + 1}{2}\alpha =
  \sin\alpha$, which is true for $n = 1$.

  Let it be true for $n = 1$

  $\Rightarrow P(k) = \sin\alpha + \sin2\alpha + \cdots + \sin k\alpha = \frac{\sin
    \frac{k\alpha}{2}}{\sin\frac{\alpha}{2}}\sin\frac{k + 1}{2}\alpha$

  Adding $\sin(k + 1)\alpha$, to both sides, we get

  $P(k + 1) = \sin\alpha + \sin2\alpha + \cdots + \sin k\alpha + \sin(k + 1)\alpha = \frac{\sin
    \frac{k\alpha}{2}}{\sin\frac{\alpha}{2}}\sin\frac{k + 1}{2}\alpha + \sin(k + 1)\alpha$

  $= \frac{\sin\frac{k\alpha}{2}}{\sin\frac{\alpha}{2}}\sin\frac{k + 1}{2} + 2\sin\frac{k + 1}{2}\alpha\cos\frac{k
    + 1}{2}\alpha$

  $= \sin\frac{k + 1}{2}\alpha\left[\frac{\sin\frac{k\alpha}{2}}{\sin\frac{\alpha}{2}} + 2\cos\frac{k +
      1}{2}\alpha\right] = \sin\frac{k + 1}{2}\alpha\left[\frac{\sin\frac{k\alpha}{2} + 2\cos\frac{k +
        1}{2}\alpha\sin\frac{\alpha}{2}}{\sin\frac{\alpha}{2}}\right]$

  $= \sin\frac{k + 1}{2}\alpha\left[\frac{\sin\frac{k\alpha}{2} + \sin\frac{k +
        2}{2}\alpha - \sin\frac{k\alpha}{2}}{\sin\frac{\alpha}{2}}\right]$

  $= \frac{\sin\frac{(k + 1)\alpha}{2}}{\sin\frac{\alpha}{2}}\sin\frac{k + 1}{2}\alpha$

  Hence, by mathematical induction, the result.
\item Given, $a_1 = 1$ and $a_{n + 1} = \frac{a_n}{n + 1},\;n\geq 1$. Clearly, $a_2 = \frac{a_1}{1 + 1} =
  \frac{1}{2} = \frac{1}{2!}$, which means it holds true for $n = 1$.

  Let it be true for $n = k\Rightarrow a_{k + 1} = \frac{1}{k!}$.

  $P(k + 1) = \frac{a_k}{k + 1} = \frac{1}{k!(k + 1)} = \frac{1}{(k + 1)!}$.

  Hence, by mathematical induction, the result.
\item Given, $a_1 = 1, a_2 = 5$ and $a_{n + 2} = 5a_{n + 1} - 6a_n,\;n\geq 1$

  $\Rightarrow a_3 = 5a_2 - 6a_1 = 25 - 6 = 19 = 3^3 - 2^3 = 19$, which is true for $n = 3$.

  Let it be true for $n = k$ and $n = k + 1$

  $\Rightarrow a_k = 3^k - 2^k$ and $a_{k + 1} = 3^{k + 1} - 2^{k + 1}$

  $a_{k + 2} = 5a_{k + 1} - 6a_k = 5.3^{k + 1} - 5.2^{k + 1} - 6.3^k + 6.2^k = 9.3^k - 4.2^k = 3^{k + 2} -
  2^{k + 2}$.

  Hence, by mathematical induction, the result.
\item Given, $u_0 = 2, u_1 = 3$ and $u_{n + 1} = 3u_n - 2u_{n - 1}$ and $u_n = 2^n + 1$.

  $u_2 = 3u_1 - 2u_0 = 3.3 - 2.2 = 5 = 2^2 + 1$, which is true for $n = 2$.

  Let it be true for $n = k$ and $n = k + 1$

  $\Rightarrow u_k = 2^k - 1$ and $u_{k + 1} = 2^{k + 1} - 1$

  $u_{k + 2} = 3.u_{k + 1} - 2.u_k = 3.2^{k + 1} - 3 - 2.2^k + 2 = 4.2^k + 1 = 2^{k + 2} + 1$.

  Hence, by mathematical induction, the result.
\item Given, $a_0 = 0, a_1 = 1$ and $a_{n + 1} = 3a_n - 2a_{n - 1}$, and $a_n = 2^n - 1$.

  $a_2 = 3a_1 - 2a_0 = 3 = 2^2 - 1$, which is true for $n = 2$.

  Let it be true for $n = k$ and $n = k + 1$

  $\Rightarrow a_k = 2^k - 1$ and $a_{k + 1} = 2^{k + 1} - 1$

  $\Rightarrow a_{k + 2} = 3.2^{k + 1} - 3 - 2.2^k + 2 = 4.2^k - 1 = 2^{k + 2} - 1$.

  Hence, by mathematical induction, the result.
\item Given, $A_1 = \cos\theta, A_2 = \cos2\theta$, and for every natural number $m > 2, A_m =
  2A_{m-1}\cos\theta - A_{m - 2}$.

  $A_3 = 2A_2\cos\theta - A_1 = 2\cos\theta\cos2\theta - \cos\theta = \cos3\theta + \cos\theta - \cos\theta
  = \cos3\theta$, which is true for $n = 3$.

  Let it be true for $n = k$ and $n = k + 1$

  $\Rightarrow A_k = \cos k\theta$ and $A_{k +1} = \cos(k + 1)\theta$

  $A_{k + 2} = 2\cos(k + 1)\theta\cos\theta - \cos k\theta = \cos(k + 2)\theta + \cos k\theta - \cos k\theta
  = \cos(k + 2)\theta$

  Hence, by mathematical induction, the result.
\item Let $P(n) = (2\cos\theta - 1)(2\cos2\theta - 1)\cdots(2\cos2^{n - 1}\theta - 1) =
  \frac{2\cos2^n\theta + 1}{2\cos\theta + 1}$

  $P(1) = 2\cos\theta - 1 = \frac{2\cos2\theta + 1}{2\cos\theta + 1} = \frac{4\cos^2\theta - 1}{2\cos\theta
    + 1} = 2\cos\theta - 1$, which is true for $n = 1$.

  Let it be true for $n = m$

  $\Rightarrow P(m) = (2\cos\theta - 1)(2\cos2\theta - 1)\cdots(2\cos2^{m - 1}\theta - 1) =
  \frac{2\cos2^m\theta + 1}{2\cos\theta + 1}$

  Multiplying both sides by $2\cos2^m\theta - 1$, we get

  $P(m + 1) = (2\cos\theta - 1)(2\cos2\theta - 1)\cdots(2\cos2^{m - 1}\theta - 1)(2\cos2^m\theta - 1) =
  \frac{2\cos2^m\theta + 1}{2\cos\theta + 1}(2\cos2^m\theta - 1)$

  $= \frac{4\cos2^m\theta - 1}{2\cos\theta} = \frac{2\cos2^{m + 1}\theta + 1}{2\cos\theta}$

  Hence, by mathematical induction, the result.
\item Let $P(n) = \tan^{-1}\frac{x}{1.2 + x^2} + \tan^{-1}\frac{x}{2.3 + x^2} + \cdots +
  \tan^{-1}\frac{x}{n(n + 1) + x^2} = \tan^{-1}x - \tan^{-1}\frac{x}{n + 1},\;x\in \mathbb{R}$.

  $P(1) = \tan^{-1}\frac{x}{1.2 + x^2} = \tan^{-1}x - \tan^{-1}\frac{x}{1 + 1} = \tan^{-1}\frac{2x - x}{2 +
    x^2} = \tan^{-1}\frac{x}{1.2 + x^2}$, which is true for $n = 1$.

  Let it be true for $n = m$

  $\Rightarrow P(n) = \tan^{-1}\frac{x}{1.2 + x^2} + \tan^{-1}\frac{x}{2.3 + x^2} + \cdots +
  \tan^{-1}\frac{x}{m(m + 1) + x^2} = \tan^{-1}x - \tan^{-1}\frac{x}{m + 1}$

  Adding $\tan^{-1}\frac{x}{(m + 1)(m + 2) + x^2}$, to both sides, we get

  $P(m + 1) = \tan^{-1}\frac{x}{1.2 + x^2} + \tan^{-1}\frac{x}{2.3 + x^2} + \cdots +
  \tan^{-1}\frac{x}{m(m + 1) + x^2} + \tan^{-1}\frac{x}{(m + 1)(m + 2) + x^2} = \tan^{-1}x -
  \tan^{-1}\frac{x}{m + 1} + \tan^{-1}\frac{x}{(m + 1)(m + 2) + x^2}$

  $= \tan^{-1}x - \tan^{-1}\frac{x}{m + 1} + \tan^{-1}\frac{x}{m +1 } - \tan^{-1}\frac{x}{m + 2} =
  \tan^{-1}x - \tan^{-1}\frac{x}{m + 2}$.

  Hence, by mathematical induction, the result.
\item Let $P(n) = 3 + 33 + \cdots + \underbrace{33\ldots3}_{n\text{~digits}} = \frac{10^{n + 1} - 9n -
  10}{27}$.

  $P(1) = 3 = \frac{10^2 - 9 - 19}{27} = \frac{81}{27} = 3$, which is true for $n = 1$.

  Let it be true for $n = m$

  $P(n) = 3 + 33 + \cdots + \underbrace{33\ldots3}_{m\text{~digits}} = \frac{10^{m + 1} - 9n -
    10}{27}$

  Adding $\underbrace{3\ldots33}_{m + 1\text{~digits}}$, to both sides, we get

  $P(m + 1) = 3 + 33 + \cdots + \underbrace{33\ldots3}_{m\text{~digits}} +
  \underbrace{33\ldots3}_{m + 1\text{~digits}} = \frac{10^{m + 1} - 9n - 10}{27} +
  \underbrace{33\ldots3}_{m + 1\text{~digits}}$

  $= \frac{10^{m + 1} - 9m - 10}{27} + + \frac{3}{9}(10^{m + 1} - 1) = \frac{10^{m + 1} - 9m - 10 +
    9.10^{m + 1} - 9}{27}$

  $= \frac{10^{m + 2} - 9m - 19}{243} = \frac{10^{m + 2} -9(m + 1) - 10}{27}$.

  Hence, by mathematical induction, the result.
\item Let $P(n) = \displaystyle\int_{0}^\pi\frac{\sin(2n + 1)x}{\sin x}dx = \pi$.

  $P(1) = \displaystyle\int_0^\pi \frac{\sin 3x}{\sin x}dx = \int_0^\pi\frac{3\sin x - 4\sin^3x}{\sin x}dx$

  $= \displaystyle3\int_0^\pi dx - 4\int_0^\pi\sin^2x dx = [3x]_0^\pi - 2\int_0^\pi(1 - \cos2x)dx$

  $= 3\pi - [2x]_0^\pi + [\sin2x]_0^\pi = \pi + 0 = \pi$, which is true for $n = 1$.

  Let it be true for $n = m$

  $\Rightarrow P(m) = \displaystyle\int_0^\pi \frac{\sin(2m + 1)x}{\sin x}dx = \pi$

  Now $P(m + 1) - P(m) = \displaystyle\int_0^\pi\frac{\sin(2m + 3)x - \sin(2m + 1)d}{\sin x}dx =
  \int_0^\pi\frac{2\cos(2m + 2)x\sin x}{\sin x}dx$

  $= \displaystyle\int_0^\pi \cos(2m + 2)x dx= 0 \Rightarrow P(m + 1) = P(m) = \pi$.

  Hence, by mathematical induction, the result.
\item Let $P(n) = \displaystyle\int_{0}^\pi\frac{\sin^2nx}{\sin^2x}dx = n\pi$.

  $P(1) = \displaystyle\int_0\frac{\sin^2x}{\sin^2x}dx = [x]_0^\pi = \pi$, which is true for $n = 1$.

  Let it be true for $n = m$

  $\Rightarrow P(m) = \displaystyle\int_0^\pi\frac{\sin^2mx}{\sin^2x}dx = m\pi$

  $P(m + 1) - P(m) = \displaystyle\int_0^\pi \frac{\sin^2(m + 1)x - \sin^2mx}{\sin^2x}dx =
  \int_0^\pi\frac{\cos2mx - \cos(2m + 2)x}{\sin^2x}dx$

  $= \displaystyle\int_0^\pi \frac{2\sin(2m + 1)\sin x}{\sin^2x}dx = \pi$(we have proved this in previous
  problem)

  $\Rightarrow P(m + 1) = (m + 1)\pi$(because $P(m) = m\pi$).

  Hence, by mathematical induction, the result.
\item Let $P(n) = \tan^{-1}\frac{1}{1 + 1 + 1^2} + \tan^{-1}\frac{1}{1 + 2 + 2^2} + \cdots +
  \tan^{-1}\frac{1}{1 + n + n^2} = \tan^{-1}(n + 1) - \frac{\pi}{4}$.

  $P(1) = \tan^{-1}\frac{1}{1 + 2} = \tan^{-1}2 - \tan^{-1}1 = \tan^{-1}(1 + 1) - \frac{\pi}{4}$, which is
  true for $n = 1$.

  Let it be true for $n = m$

  $\Rightarrow P(m) = \tan^{-1}\frac{1}{1 + 1 + 1^2} + \tan^{-1}\frac{1}{1 + 2 + 2^2} + \cdots +
  \tan^{-1}\frac{1}{1 + m + m^2} = \tan^{-1}(m + 1) - \frac{\pi}{4}$.

  Adding $\tan^{-1}\frac{1}{1 + (m + 1) + (m + 1)^2}$, to both sides, we get

  $P(m + 1) = \tan^{-1}\frac{1}{1 + 1 + 1^2} + \tan^{-1}\frac{1}{1 + 2 + 2^2} + \cdots +
  \tan^{-1}\frac{1}{1 + m + m^2} + \tan^{-1}\frac{1}{1 + (m + 1) + (m + 1)^2}= \tan^{-1}(m + 1) -
  \frac{\pi}{4} + \tan^{-1}\frac{1}{1 + (m + 1) + (m + 1)^2}$

  $= \tan^{-1}(m + 1) -\frac{\pi}{4} + \tan^{-1}(m + 2) - \tan^{-1}(m + 1) = \tan^{-1}(m + 1) -
  \frac{\pi}{4}$.

  Hence, by mathematical induction, the result.
\item Let $P(n) = n(n + 1)(n + 5)$. $P(1) = 1.2.6 = 12$, which is divisible by $6$ i.e. the statement is
  true for $n =1$.

  Let it be true for $n = m$.

  $\Rightarrow P(m) = m(m + 1)(m + 5) = m^3 + 6m^2 + 5m = 6k$,where $k\in\mathbb{N}$.

  $P(m + 1) = (m + 1)(m + 2)(m + 6) = m^3 + 9m^2 + 15m + 12 = 6k + 3m^2 + 15m + 12 = 6k + 3(m + 1)(m + 4)$

  Clearly, $3(m + 1)(m + 4) = 6q$, where $q\in\mathbb{N}$.

  Hence, by mathematical induction, the result.
\item Let $P(n) = n^3 + (n + 1)^3 + (n + 2)^3. P(1) = 1^3 + 2^3 + 3^3 = 36$, which is divisible by $9$
  i.e. the statement is true for $n = 1$.

  Let it be true for $n = m$

  $\Rightarrow P(m) = m^3 + (m + 1)^3 + (m + 2)^3 = 9k$, where $k\in\mathbb{N}$.

  $P(m + 1) = (m + 1)^3 + (m + 2)^3 + (m + 3)^3 = 9k + 9m^2 + 27 m + 27 = 9(k + m^2 + 3m + 3)$.

  Hence, by mathematical induction, the result.
\item Let $P(n) = n(n^2 + 20)$, where $n$ is even where $n\in\mathbb{P}$.

  $P(2) = 2.24 = 48$, which is divisible by $48$ i.e. the statement is true for $n = 2$.

  Let it be true for $n = 2m$

  $\Rightarrow P(2m) = 2m(4m^2 + 20) = 8m(m^2 + 5) = 48k$, where $k\in\mathbb{P}$.

  $P(2m + 2) = (2m + 2)\left[4m^2 + 8m + 24\right] = 8(m + 1)(m^2 + 2m + 6) = 8(m^3 + 2m^2 + 6m + m^2 + 2m +
  6)$

  $= 8(m^3 + 3m^2 + 8m + 6) = 8(48k + 3m^2 + 3m + 6) = 24(16k + m^2 + m + 2)$.

  Now, we have to prove that $m^2 + m + 2$ is divisible by $2$.

  Let $Q(m) = m^2 + m + 2$. $Q(1) = 4$, which is divisible by $2$. Let it be true for $m = t$.

  $\Rightarrow Q(t) = t^2 + t + 2 = 2x$, where $x\in\mathbb{x}$.

  $Q(t + 1) = t^2 + 2t + t + 4 = 2x + 2t + 2$, which is divisible by $2$.

  Hence, by mathematical induction, the result.
\item Let $P(n) = 4^n - 3n - 1$. $P(1) = 4 - 3 - 1 = 0$, which is divisible by $9$, so it is true for $n =
  1$.

  Let it be true for $n = m$

  $\Rightarrow P(m) = 4^m - 3m - 1 = 9k$, where $k\in\mathbb{N}$.

  $P(m + 1) = 4^{m + 1} - 3(m + 1) - 1 = 4.4^m - 3m - 4 = 4(9k + 3m + 1) -3m  - 4 = 36k + 9m$, which is
  divisible by $9$.

  Hence, by mathematical induction, the result.
\item Let $P(n) = 3^{2n} - 1$. $P(1) = 3^2 - 1 = 8$, which is divisible by $8$. So the statement is true for
  $n = 1$.

  Let it be true for $n = m$

  $\Rightarrow P(m) = 3^{2m} - 1 = 8k$, where $k\in\mathbb{N}$.

  $P(m + 1) = 3^{2m + 2} - 1 = 9.3^{2m} - 1 = 9(8k + 1) - 1 = 72k + 8$, which is divisible by $8$.

  Hence, by mathematical induction, the result.
\item Let $P(n) = 5.2^{3n - 2} + 3^{3n - 1}$. $P(1) = 5.2 + 3^2 = 19$, which is divisible by $19$, so the
  statement is true for $n = 1$.

  Let it be true for $n = m$

  $\Rightarrow P(m) = 5.2^{3m - 2} + 3^{3m - 1} = 19k$, where $k\in\mathbb{N}$

  $P(m + 1) = 5.2^{3m + 1} + 3^{3m + 2} = 40.2^{3m - 2} + 27.3^{3m - 1} = 8(5.2^{3m - 2} + 3^{3m - 1}) +
  19.3^{3m - 1} = 8.19k + 19.3^{3m - 1}$,

  which is divisible by $19$.

  Hence, by mathematical induction, the result.
\item Let $P(n) = 7^{2n} + 2^{3n - 3}.3^{n - 1}$. $P(1) = 7^2 + 2^03^0 = 50$, which is divisible by $25$, so
  the statement is true for $n = 1$.

  Let it be true for $n = m$

  $\Rightarrow P(m) = 7^{2m} + 2^{3m - 3}.3^{m - 1} = 25k$, where $k\in\mathbb{N}$.

  $P(m + 1) = 7^{2m + 2} + 2^{3m}.3^m = 49.7^{2m} + 24.2^{3m - 3}.3^{m - 1} = 25.7^{2m} + 24.25k$, which is
  divisible by $25$.

  Hence, by mathematical induction, the result.
\item Let $P(n) = 10^n + 3.4^{n + 2} + 5$. $P(1) = 10 + 3.4^3 + 5 = 207$, which is divisible by $9$, so the
  statement is true for $n = 1$.

  Let it be true for $n = m$

  $\Rightarrow P(m) = 10^m  + 3.4^{m + 2} + 5 = 9k$, where $k\in\mathbb{N}$.

  $P(m + 1) = 10^{m + 1} + 3.4^{m + 3} + 5 = 10.10^m + 12.4^{m + 2} + 5 = 10.9k + 9.10^m + 9.4^{m + 2}$,
  which is divisible by $9$.

  Hence, by mathematical induction, the result.
\item Let $P(n) = 3^{4n + 2} + 5^{2n + 1}$. $P(1) = 3^6 + 5^3 = 729 + 125 = 854$, which is divisible by
  $14$, so the statement is true for $n = 1$.

  Let it be true for $n = m$

  $\Rightarrow P(m) = 3^{4m + 2} + 5^{2m + 1} = 14k$, where $k\in\mathbb{N}$.

  $P(m + 1) = 3^{4m + 6} + 5^{2m + 3} = 81.3^{4m + 2} + 25.5^{2m + 1} = 25.14k + 56.3^{4m + 2}$, which is
  divisible by $14$.

  Hence, by mathematical induction, the result.
\item Let $P(n) = 3^{2n + 2} - 8n - 9$. $P(1) = 3^4 - 8.1 - 9 = 64$, which is divisible by $64$, so the
  statement is true for $n = 1$.

  Let it be true for $n = m$

  $\Rightarrow P(m) = 3^{2m + 2} - 8m - 9 = 9^{m + 1} - 8m - 9 = 9.9^m - 8m - 9 = 64k$, where $k\in\mathbb{N}$.

  $P(m + 1) = 3^{2m + 4} - 8(m + 1) - 9 = 9^{m + 2} - 8m - 17 = 81.9^m - 8m - 17 = 64k + 72.9^m - 8 = 64k +
  8(9^{m + 1} - 1)$.

  Now we will prove that $9^{m + 1} - 1$ is divisible by $8$. Let $Q(n) = 9^{n + 1} - 1$. $Q(1) = 80$, which
  is divisible by $8$.

  Let it be true for $n = r$.

  $Q(r) = 9^{r + 1} - 1 = 8s$, where $s\in\mathbb{N}$.

  $Q(r + 1) = 9.9^{r + 1} - 1 = 8.8^{r + 1} + 8s$, which is divisible by $8$.

  Hence, by mathematical induction, the result.
\item Let $P(n) = n^7 - n$. $P(1) = 1^7 - 1 = 0$, which is divisible by $7$, so the statement is true for $n
  = 1$.

  Let it be true for $n = m$

  $\Rightarrow P(m) = m^7 - m = 7k$, where $k\in\mathbb{N}$.

  $P(m + 1) = (m + 1)^7 - (m + 1) = m^7 + C_1^^7m^6 + C_2^^7m^5 + C_3^^7m^4 + C_4^^7m^3 + C_5^^7m^2 +
  C_6^^7m + 1 - m - 1$

  $= m^7 - m + C_1^^7m^6 + C_2^^7m^5 + C_3^^7m^4 + C_4^^7m^3 + C_5^^7m^2 + + C_6^^7m = 7k + 7s$, where
  $s\in\mathbb{N}$, which is divisible by $7$.

  Hence, by mathematical induction, the result.
\item Let $P(n) = 11^{n + 2} + 12^{2n + 1}$. $P(1) = 11^3 + 12^3 = 1331 + 1728 = 3059$, which is divisible
  by $133$, so the statement is true for $n = 1$.

  Let it be true for $n = m$

  $\Rightarrow P(m) = 11^{m + 2} + 12^{2m + 1} = 133k$, where $k\in\mathbb{N}$.

  $P(m + 1) = 11.11^{m + 2} + 144.12^{2m  + 1} = 11.133k + 133.12^{2m + 1}$, which is divisible by $133$.

  Hence, by mathematical induction, the result.
\item Let $P(n) = 10^{2n - 1} + 1$. $P(1) = 10^{2 - 1} + 1 = 11$, which is divisible by $11$, so the
  statement is true for $n = 1$.

  Let it be true for $n = m$

  $\Rightarrow P(m) = 10^{2m - 1} + 1 = 11k$, where $k\in\mathbb{N}$.

  $P(m + 1) = 10^{2m + 1} + 1 = 100.10^{2m - 1} + 1 = 99.10^{2m - 1} + 11k$, which is divisible by $11$.

  Hence, by mathematical induction, the result.
\item Let $P(n) = 7^n - 3^n$. $P(1) = 7 - 3 = 4$, which is divisible by $4$, so the statement is true for $n
  = 1$.

  Let it be true for $n = m$.

  $\Rightarrow P(m) = 7^m - 3^m = 4k$, where $k\in\mathbb{N}$.

  $P(m + 1) = 7^{m + 1} - 3^{m + 1} = 7.7^m - 3.3^m = 4.7^m + 3.4k$, which is divisible by $4$.

  Hence, by mathematical induction, the result.
\item Let $P(n) = 2.7^n + 3.5^n - 5$. $P(1) = 2.7 + 3.5 - 5 = 24$, which is divisible by $24$, so the
  statement is true for $n = 1$.

  Let it be true for $n = m$.

  $P(m) = 2.7^m + 3.5^m - 5 = 24k$, where $k\in\mathbb{N}$.

  $P(m + 1) = 2.7^{m  + 1} + 3.5^{m + 1} - 5 = 14.7^m + 15.5^m - 5 =  4.7^m + 5.24k + 20 = 4(7^m + 5) +
  120k$.

  Now we will prove that $7^m + 5$ is divisible by $6$.

  $Q(n) = 7^n + 5$. $Q(1)= 7 + 5 = 12$, which is divisible by $6$, so the statement is true for $n = 1$.

  Let it be true for $n = m$.

  $\Rightarrow Q(m) = 7^m + 5 = 6s$, where $s\in\mathbb{N}$.

  $Q(m + 1) = 7^{m + 1} + 5 = 6.7^m + 6s$, which is divisible by $6$.

  Hence, by mathematical induction, the result.
\item Let $P(n) = 3^{2n} - 1$. $P(1) = 3^2 - 1 = 8$, which is divisible by $8$, so the statement is true for
  $n = 1$.

  Let it be true for $n = m$.

  $\Rightarrow P(m) = 3^{2m} - 1 = 8k$, where $k\in\mathbb{N}$

  $P(m + 1) = 3^{2m + 2} - 1 = 9.3^{2m} - 1 = 8.3^{2m} + 8k$, which is divisible by $8$.

  Hence, by mathematical induction, the result.
\item Let $P(n) = 10^n + 3.4^{n + 2} + 5$. $P(1) = 10 + 3.4^3 + 5 = 297$, which is divisible by $9$, so the
  statement is true for $n = 1$.

  Let it be true for $n = m$.

  $\Rightarrow P(m) = 10^m + 3.4^{m + 2} + 5 = 9k$, where $k\in\mathbb{N}$.

  $P(m + 1) = 10^{m + 1} + 3.4^{m + 3} + 5 = 9.10^m + 9.4^{m + 3} + 9k$, which is divisible by $9$.

  Hence, by mathematical induction, the result.
\item Let $P(n) = 5^{2n + 1} + 2^{n + 4} + 2^{n + 1}$. $P(1) = 5^3 + 2^5 + 2^2 = 125 + 32 + 4 = 161$, which
  is divisible by $23$, so the statement is true for $n = 1$.

  Let it be true for $n = m$.

  $\Rightarrow P(m) = 5^{2m + 1} + 2^{m + 4} + 2^{m + 1} = 23k$, where $k\in\mathbb{N}$.

  $P(m + 1) = 5^{2m + 3} + 2^{m + 5} + 2^{m + 2} = 25.5^{2m + 1} + 2.2^{m + 4} + 2.2^{m + 1} = 23.5^{2m + 1}
  + 2.23k$, which is divisible by $23$.

  Hence, by mathematical induction, the result.
\item Let $P(n) = 7^{2n} - 1$. $P(1) = 7^2 - 1 = 48$, which is divisible by $8$, so the statement is true
  for $n = 1$.

  Let it be true for $n = m$.

  $\Rightarrow P(m) = 7^{2m} - 1 = 8k$, where $k\in\mathbb{N}$.

  $P(m + 1) = 7^{2m + 2} - 1 = 49.7^{2m} - 1 = 48.7^{2m} + 48k$, which is divisible by $8$.

  Hence, by mathematical induction, the result.
\item Let $P(n) = 3^{2n + 2} - 8n - 9$. $P(1) = 3^4 - 8 - 9 = 64$, which is divisible by $8$, so the
  statement is true for $n = 1$.

  Let it be true for $n = m$.

  $\Rightarrow P(m) = 3^{2m + 2} - 8m - 9 = 8k$, where $k\in\mathbb{N}$.

  $P(m + 1) = 3^{2m + 4} - 8(m + 1) - 9 = 8.3^{2m + 2} + 8k - 8$, which is divisible by $8$.

  Hence, by mathematical induction, the result.
\item Let $P(n) = 41^n - 14^n$. $P(1) = 41 - 14 = 27$, which is divisible by $27$, so the statement is true
  for $n = 1$.

  Let it be true for $n = m$.

  $\Rightarrow P(m) = 41^m - 14^m = 27k$, where $k\in\mathbb{N}$.

  $P(m + 1) = 41^{m + 1} - 14^{m + 1} = 27.41^m + 27k$, which is divisible by $27$.

  Hence, by mathematical induction, the result.
\item Let $P(n) = 15^{2n - 1} + 1$. $P(1) = 15 + 1 = 16$, which is divisible by $16$, so the statement is
  true for $n = 1$.

  Let it be true for $n = m$.

  $\Rightarrow P(m) = 15^{2m - 1} + 1 = 16k$, where $k\in\mathbb{N}$.

  $P(m + 1) = 15^{2m + 1} + 1 = 225.15^{2m - 1} + 1 = 224.15^{2m - 1} + 16$, which is divisible by $16$.

  Hence, by mathematical induction, the result.
\item Let $P(n) = 5^{2n + 1} + 3^{n + 2}.2^{n - 1}$. $P(1) = 5^3 + 3^3.2^0 = 125 + 27 = 152$, which is
  divisible by $19$, so the statement is true for $n = 1$.

  Let it be true for $n = m$.

  $\Rightarrow P(m) = 5^{2m + 1} + 3^{m + 2}.2^{m - 1} = 19k$, where $k\in\mathbb{N}$.

  $P(m + 1) = 5^{2m + 3} + 3^{m + 3}.2^m = 25.5^{2m + 1} + 6.3^{m + 2}.2^{m - 1} = 19.5^{2m - 1} + 6.19k$,
  which is divisible by $19$.

  Hence, by mathematical induction, the result.
\item Let $P(n) = 10^n + 3.4^{n + 2} + 5$. $P(1)= 10 + 3.4^3 + 5 = 10 + 192 + 5 = 207$, which is divisible
  by $9$, so the statement is true for $n = 1$.

  Let it be true for $n = m$.

  $\Rightarrow P(m) = 10^m + 3.4^{m + 2} + 5 = 9k$, where $k\in\mathbb{N}$.

  $P(m  + 1) = 10^{m + 1} + 3.4^^{m + 3} + 5 = 10.10^m + 12.4^{m + 2} + 5 = 6.10^m + 4.9k - 15 = 3(2.10^m -
  5) + 4.9k$.

  Now we will prove that $2.10^m - 5$ is divisible by $3$.

  Let $Q(n) = 2.10^n - 5$. $Q(1) = 20 - 5 = 15$, which is divisible by $3$, so the statement is true for $n
  = 1$.

  Let it be true for $n = m$.

  $\Rightarrow Q(m) = 2.10^m - 5 = 3s$, where $s\in\mathbb{N}$.

  $Q(m  + 1) = 2.10^{m + 1} - 5 = 20.10^m - 5 = 18.10^m + 3s$, which is divisible by $3$.

  Hence, by mathematical induction, the result.
\item Let $P(n) = 9^n - 8n - 1$. $P(1) = 9 - 8 - 1 = 0$, which is divisible by $64$, so the statement is
  true for $n = 1$.

  Let it be true for $n = m$.

  $\Rightarrow P(m) = 9^m - 8m - 1 = 64k$, where $k\in\mathbb{N}$.

  $P(m + 1) = 9^{m + 1} - 8(m + 1) - 1 = 9.9^m - 8m - 9 = 9(9^m - 8m - 1) + 64m = 9.64k + 64m$, which is
  divisible by $64$.

  Hence, by mathematical induction, the result.
\item Let $P(n) = n^3 + 3n^2 + 5n + 3$. $P(1) = 1 + 3 + 5 + 3 = 12$, which is divisible by $3$, so the
  statement is true for $n = 1$.

  Let it be true for $n = m$.

  $\Rightarrow P(m) = m^3 + 3m^2 + 5m + 3 = 3k$, where $k\in\mathbb{N}$.

  $P(m + 1) = (m + 1)^3 + 3(m + 1)^2 + 5(m + 1) + 3 = m^3 + 3m^2 + 5m + 3 + 3m^2 + 3m + 1 + 6m + 3m + 5 = 3k
  + 3m^2 + 9m + 6$, which is divisible by $3$.

  Hence, by mathematical induction, the result.
\item $P(n) = (n + 1)(n + 2)(n + 3)(n + 4)(n + 5)$. $P(1) = 2.3.4.5.6 = 720$, which is divisible by $120$,
  so the statement is true for $n = 1$.

  Let it be true for $n = m$.

  $\Rightarrow P(m) = (m + 1)(m + 2)(m + 3)(m + 4)(m + 5) = 120k$, where $k\in\mathbb{N}$.

  $P(m + 1) - P(m) = 5(m + 2)(m + 3)(m + 4)(m + 5)$

  Among the four consecutive numbers $(m + 2)(m + 3)(m + 4)(m + 5)$, there has to be at least one multiples
  of $2, 3$ and $4$ each. Thus, $P(m + 1)$ is divisible by $120$.

  Hence, by mathematical induction, the result.
\item Let $P(n) = n^5 - n$. $P(1) = 1 - 1 = 0$, which is divisible by $5$, so the statement is true for $n =
  1$.

  Let it be true for $n = m$.

  $\Rightarrow P(m) = m^5 - m = 5k$, where $k\in\mathbb{N}$.

  $P(m  + 1) = (m + 1)^5 - (m + 1) = m^5 - m + C_1^^m^4 + C_2^^5m^3 + C_3^^5m^2 + C_4^^5m$, which is
  divisible by $5$.

  Hence, by mathematical induction, the result.
\item Let $P(n) = (1 + x)^n - nx - 1$. $P(1) = 1 + x - x - 1 = 0$, which is divisible by $x^2$, so the
  statement is true for $n = 1$.

  Let it be true for $n = m$.

  $\Rightarrow P(m) = (1 + x)^m - mx - 1 = Q(x)x^2$, where $Q(x)$ is a polynomial in $x$.

  $P(m + 1) = (1 + x)^{m + 1} - (m + 1)x - 1 = (1 + x)(1 + x)^m - mx - x - 1 = Q(x)x^2 + x(1 + x)^m - x$

  $= Q(x)x^2 + x[C_0^^m + C_1^^mx + \cdots + C_m^^mx^m] - x$, which is divisible by $x^2$.

  Hence, by mathematical induction, the result.
\item Let $P(n) = n(n^2 - 1)$. $P(1) = 0$, which is divisible by $24$, so the statement is true for $n = 1$.

  Llet it be true for $n = m$, where $m = 2k + 1, \forall k\in{N}$.

  $\Rightarrow P(m) = (2k + 1)(4k^2 + 4k) = 4k(2k + 1)(k + 1) = 4k(2k^2 + 3k + 1) = 24s$, where
  $s\in\mathbb{N}$.

  $P(m + 2) = (2k + 3)(4k^2 + 12k + 8) = 8k^3 + 36k^2 + 52k + 24 = 24s + 24k^2 + 48k + 24$, which is
  divisible by $24$.

  Hence, by mathematical induction, the result.
\item Let $P(n) = n(n^2 + 20)$. $P(2) = 2.24 = 48$, which is divisible by $48$, so the statement is true for
  $n = 1$.

  Let it be true for $n = 2m$, where $m\in\mathbb{N}$.

  $\Rightarrow P(2m) = 8m(m^2 + 5) = 48s$, where $s\in\mathbb{N}$.

  $P(2m + 2) = 8(m + 1)(m^2 + 2m + 6) = 8(m^3 + 3m^2 + 8m + 6) = 48s + 24m^2 + 24m + 48$.

  Now we will prove that $Q(n) = n^2 + n = n(n + 1)$ is divisible by $2$. We can prove this by induction or
  by just simple observation product of two consecutive integers is  always divisible by $2$.

  Hence, by mathematical induction, the result.
\item Let $P(n) = 2^{2n} + 1$ and $Q(n) = 2^{2n} - 1$. $P(1) = 2^2 + 1 = 5$ and $Q(2) = 2^4 - 1 = 15$. Both
  are divisible by $5$. So statements are true for $n = 1, 2$ respectively.

  Let they are true for $n = 2m, 2m + 1$ where $\m\in\mathbb{N}$.

  $\Rightarrow P(2m + 1) = 2^{4m + 2} + 1 = 5k$ and $Q(2m) = 2^{4m} - 1 = 5l$, where $l, m\in\mathbb{N}$.

  $P(2m + 3) = 2^{4m + 6} + 1 = 15.2^{4m + 2} + 5k$, which is divisible by $5$.

  $Q(2m + 2) = 2^{4m + 4} - 1 = 15.2^{4m} + 5l$, which is also divisible by $5$.

  Hence, by mathematical induction, the result.
\item Let $P(n) = 5^{2n} + 1$. $P(1) = 5^2 + 1 = 26$, which is divisible by $13$, so the statement is true
  for $n = 1$.

  Let it be true for $n = 2m + 1$, where $m\in\mathbb{N}$.

  $\Rightarrow P(2m + 1) = 5^{4m + 2} + 1 = 13k$, where $k\in\mathbb{N}$.

  $P(2m + 3) = 5^{4m + 6} + 1 = 625.5^{4m + 2} + 1 = 624.5^{4m + 2} + 13k$, which is divisible by $13$.

  Hence, by mathematical induction, the result.

  $5^{99} = 5.5^{98} = 5.5^{2.49} = 5(5^{98} + 1) - 5$. This will leave the remainder $13 - 5 = 8$, when
  divided by $13$.
\item Let $P(n) = 4.6^n + 5^{n + 1}$. $P(1) = 4.6 + 5^2 = 49$, which leaves remainder $9$ when divided by
  $20$, so the statement is true for $n = 1$.

  Let it be true for $n = m$.

  $\Rightarrow P(m) = 4.6^m + 5^{m + 1} = 20k + 9$, where $k\in\mathbb{N}$.

  $P(m + 1) = 4.6^{m + 1} + 5^{m + 2} = 24.6^m + 5.5^{m + 1} = 4.6^m + 5.20k + 45$.

  Now we will prove that $4.6^m + 45$ will leave remainder $9$ when divided by $20$.

  Let $Q(n) = 4.6^n + 45$. $Q(1) = 49$, which leaves remainder $9$ when divided by $20$, so the statement is
  true for $n = 1$.

  Let it be true for $n = r$.

  $Q(r) = 4.6^m + 45 = 20s + 9$, where $s\in\mathbb{N}$.

  $Q(r + 1) = 4.6^{m + 1} + 45 = 24.6^m + 45 = 20.6^m + 20s + 9$, which will leave remainder $9$, when
  divided by $20$.

  Hence, by mathematical induction, the result.
\item Let $P(n) = 3^n + 8^n$. $P(1) = 3 + 8 = 11$, which is not divisiible by $8$, so the statement is true
  for $n = 1$. Quick observation tells us that $3^m$ will be odd, while $8^n$ will be even so sum would be
  odd, which will not be divisible by $8$. We will prove this by induction.

  Let it be true for $n = m$.

  $\Rightarrow P(m) = 3^m + 8^m = 8k + s$, where $k,s\in\mathbb{N}$ such that $s\in\{1,2,3, \ldots, 7\}$.

  $P(m + 1) = 3.3^m + 8.8^m = 8k + s + 5.8^m$, which will leave remainder $s$ when divided by $8$.

  Hence, by mathematical induction, the result.
\item Let $P(n) = 2^{2^n} + 1$. If this has last digit as $7$ then it will leave remainder $7$ when divided
  by $10$. $P(2) = 2^4 + 1 = 17$, which leaves remainder $7$, so the statement is true for $n = 2$.

  Let it be true for $n = m$.

  $\Rightarrow P(m) = 2^{2^m} + 1 = 10k + 7$, where $k\in\mathbb{N}$.

  $P(m + 1) = 2^{2^{m + 1}} + 1 = 2^{2.2^m} + 1 = (10 k + 6)^2 + 1 = 100k^2 + 120k + 37$, which leaves
  remainder $7$, when divided by $10$.

  Hence, by mathematical induction, the result.
\item Let $P(n) = \frac{n^3}{3} + n^2 + \frac{5}{3}n + 1$. $P(1) = \frac{1}{3} + 1 + \frac{5}{3} + 1 = 4$, which
  is a natural number, so the statement is true for $n = 1$.

  Let it be true for $n = m$.

  $\Rightarrow P(m) = \frac{m^3}{3} + m^2 + \frac{5m}{3} + 1 = k$, where $k\in\mathbb{N}$.

  $\Rightarrow P(m + 1) = \frac{(m + 1)^3}{3} + (m + 1)^2 + \frac{5m + 5}{3} + 1 = \frac{m^3}{3} + m^2 +
  \frac{5m}{3} + 1 + m^2 + m + \frac{1}{3} + 2m + 1 + \frac{5}{3} + 1$

  $= k + m^2 + 3m + 4$, which is a ntural number.

  Hence, by mathematical induction, the result.
\item Let $P(n) = x^n + y^n$. $P(1) = x + y$, which is divisible by $x + y$, so the statement is true for $n
  = 1$. Similarly, $x^3 + y^3$ is divisible by $x + y$.

  Let it be true for $n = 2m - 1, 2m + 1$, where $m\in\mathbb{N}$

  $\Rightarrow P(2m - 1) = x^{2m - 1} + y^{2m - 1} = f(x, y)(x + y)$, where $f(x, y)$ is a polynomial in $x,
  y$.

  $\Rightarrow P(2m + 1) = x^{2m + 1} + y^{2m + 1} = g(x, y)(x + y)$, where $g(x, y)$ is a polynomial in $x,
  y$.

  $P(2m + 3) = x^{2m + 3} + y^{2m + 3} = (x^2 + y^2)(x^{2m + 1} + y^{2m + 1}) - x^2y^2(x^{2m - 1} + y^{2m -
    1}) = (x^2 + y^2)(x + y)g(x, y) - x^2y^2(x + y)f(x, y)$, which is divisible by $x + y$.

  Hence, by mathematical induction, the result.
\item Let $P(n) = x^n - y^n$. $P(1) = x - y$, which is divisible by $x - y$, so the statement is true for $n
  = 1$. Similarly, it is true for $n = 2$.

  Let it be true for $n = m, m - 1$.

  $P(m - 1) = x^{m - 1} - y^{m - 1} = f(x, y)(x - y)$, where $f(x, y)$ is a polynomial in $x$ and $y$.

  $\Rightarrow P(m) = x^m - y^m = g(x, y)(x - y)$, where $g(x, y)$ is a polynomial in $x$ and $y$.

  $P(m + 1) = x^{m + 1} - y^{m + 1} = (x^m - y^m)(x + y) - xy(x^{m - 1} - y^{m - 1})$, which is divisible by
  $x - y$.

  Hence, by mathematical induction, the result.
\item Let $P(n) = x(x^{n - 1} - na^{n - 1}) + a^n(n - 1)$. $P(2) = x(x - 2a) + a^2 = (x - a)^2$, which is
  divisible by $(x - a)^2$, so the statement is true for $n = 2$.

  Let it be true for $n = m$

  $\Rightarrow P(m) = x(x^{m - 1} - ma^{m - 1}) + a^m(m - 1) = f(x, y)(x - a)^2$, where $f(x, y)$ is a
  polynomial in $x, y$.

  $P(m + 1) = x(x^m - (m + 1)a^m) + ma^{m + 1} = x.f(x, y)(x - a)^2 + ma^{m - 1}(x - a)^2$, which is
  divisible by $(x - a)^2$.

  Hence, by mathematical induction, the result.
\item Let $P(n) = \frac{n^5}{5} + \frac{n^3}{3} + \frac{7n}{15}$. $P(1) = \frac{1}{5} + \frac{1}{3} +
  \frac{7}{15} = 1$, which is true for $n = 1$.

  Let it be true for $n = m$.

  $\Rightarrow P(m) = \frac{m^5}{5} + \frac{m^3}{3} + \frac{7m}{15} = k$, where $k\in\mathbb{N}$.

  $P(m + 1) = \frac{(m + 1)^5}{5} + \frac{(m + 1)^3}{3} + \frac{7m + 7}{15} = \frac{m^5}{5} + \frac{m^3}{3}
  + \frac{7m}{15} + m^4 + 2m^3 + 2m^2 + m + m^2 + m + \frac{1}{5} + \frac{1}{3} + \frac{1}{15} = k + 1 + m^4
  + 2m^3 + 3m^2 + 2m$, which is a natural number.

  Hence, by mathematical induction, the result.
\item Let $P(n) = \frac{n^7}{7} + \frac{n^5}{5} + \frac{2n^3}{3} - \frac{n}{105}$. $P(1) = \frac{1}{7} +
  \frac{1}{5} + \frac{2}{3} - \frac{1}{105} = 1$, so the statement is true for $n = 1$.

  Let it be true for $n = m$.

  $\Rightarrow P(m) = \frac{m^7}{7} + \frac{m^5}{5} + \frac{2m^3}{3} - \frac{m}{105} = k$, where
  $k\in\mathbb{N}$.

  $P(m + 1) = \frac{(m + 1)^7}{7} + \frac{(m + 1)^5}{5} + \frac{2(m + 1)^3}{3} - \frac{m + 1}{105} =
  \frac{m^7}{7} + \frac{m^5}{5} + \frac{2m^3}{3} - \frac{m}{105} + \frac{1}{7} + \frac{1}{5} + \frac{2}{3} -
  \frac{1}{105} + \frac{C_1^^7m^6 + C_2^^7m^5 + C_3^^7m^4 + C_4^^7m^3 + C_5^^7m^2 + C_6^^7m}{7} + \frac{C_1^^5m^4 +
  C_2^^5m^3 + C_3^^5m^2 + C_4^^5m}{5} + \frac{2C_1^^3m^2 + 2C_2^^3m}{3}$

  $= k + 1  + \frac{C_1^^7m^6 + C_2^^7m^5 + C_3^^7m^4 + C_4^^7m^3 + C_5^^7m^2 + C_6^^7m}{7} + \frac{C_1^^5m^4 +
    C_2^^5m^3 + C_3^^5m^2 + C_4^^5m}{5} + \frac{2C_1^^3m^2 + 2C_2^^3m}{3}$, which is a natural numebr.

  Hence, by mathematical induction, the result.
\item Let $P(n) = 2^n > n^2,\;n\geq 5$. $P(5) = 32 > 25$, so the statement is true for $n = 5$.

  Let it be true for $n = m$, where $m\in\mathbb{N}$, and $m \geq 5$.

  $\Rightarrow P(m) = 2^m > m^2$.

  $P(m + 1) = 2^{m + 1} > (m + 1)^2 \Rightarrow 2.2^m > m^2 + 2m + 1$.

  Now, $2m^2 - (m^2 + 2m + 1) = m^2 - 2m - 1 = (m - 1)^2 - 2$. Let $k = 5 + a$, where $a \geq 0$.

  $(4 + a)^2 - 2 = a^2 + 8a + 14 \geq 0$

  $\Rightarrow 2m^2 > (m + 1)^2 \Rightarrow 2(m + 1)^2 > (m + 1)^2$.

  Hence, by mathematical induction, the result.
\item Let $P(n) = 1 + 2 + \cdots + n\leq \frac{1}{8}(2n + 1)^2$. $P(1) = 1 \leq \frac{9}{8}$,
  so the statement is true for $n = 1$.

  Let it be true for $n = m$.

  $\Rightarrow P(m) = 1 + 2 + \cdots + m\leq \frac{1}{8}(2m + 1)^2$

  $P(m + 1) = 1 + 2 + \cdots + m + (m +1)\leq \frac{1}{8}(2m + 1)^2 + m + 1 = \frac{1}{8}(2m + 3)^2$, which
  is true.

  Hence, by mathematical induction, the result.
\item Let $P(n) = n^n < (n!)^2,\;n>2$. $P(3) = 3^3 < (3!)^2 = 27 < 36$, so the statement is true for $n =
  3$.

  Let it be true for $n = m$, where $m\in\mathbb{N}$, and $m > 2$.

  $\Rightarrow P(m) = m^m < (m!)^2$.

  $P(m + 1) = (m + 1)^{m + 1} < [(m + 1)!]^2$. Dividing $P(m + 1)$ by $P(m)$, we get

  $\left(\frac{(m + 1)^{m + 1}}{m^m}\right) <\left(\frac{(m + 1)!}{m!}\right)^2 = \frac{m}{(m +
    1)^2}\left(\frac{m + 1}{m}\right)^{m + 1} < 1

  \Rightarrow (m + 1)^{m  + 1} < [(m + 1)!]^2$.

  Hence, by mathematical induction, the result.
\item Let $P(n) = n! > 2^n,\;n>3$. $P(4) = 24 > 16$, so the statement is true for $n = 4$.

  Let it be true for $n = m$, sich that $m > 3$, and $m\in\mathbb{N}$.

  $\Rightarrow P(m) = m! > 2^m$.

  $P(m + 1) = (m + 1)! > 2^{m + 1}$. Dividing $P(m + 1)$ by $P(m)$, we get

  $m + 1 > 2$, which is true.

  Hence, by mathematical induction, the result.
\item Let $P(n) = n! < \left(\frac{n + 1}{2}\right)^n, n > 1$. $P(2) = 2! < \left(\frac{3}{2}\right)^2 =
  \frac{9}{4}$, so the statement is true for $n = 2$.

  Let it be true for $n = m$.

  Also, let $F(m) = \left(\frac{m + 1}{2}\right)^2$ and $G(m) = m!$.

  So $F(m) > G(m)$. $\frac{F(m + 1)}{F(m)}.\frac{G(m)}{G(m + 1)} = \frac{1}{2}.\frac{(m + 2)^{m + 1}}{(m +
    1)^m}.\frac{m!}{(m + 1)!}$

  $= \frac{1}{2\left(\frac{n + 2}{n + 1}\right)}^{n + 1}> 1 \Rightarrow F(m + 1) > G(m + 1)$.

  Hence, by mathematical induction, the result.
\item Let $P(n) = \frac{1}{n + 1} + \frac{1}{n + 2} + \cdots + \frac{1}{2n} > \frac{13}{24},\;n>1$. $P(2) =
  \frac{1}{3} + \frac{1}{4} = \frac{7}{12} > \frac{13}{24}$, so the statement is true for $n = 2$.

  Let it be true for $n = m$.

  $\Rightarrow P(m) = \frac{1}{m + 1} +\frac{1}{m + 2} + \frac{1}{m + 3} + \cdots + \frac{1}{2m} >
  \frac{13}{24}$.

  Adding $\frac{1}{2m + 2} + \frac{1}{2m + 1} - \frac{1}{m + 1}$ to both sides

  $P(m + 1) = \frac{1}{m + 2}  + \frac{1}{m + 3} + \cdots + \frac{1}{2m} + \frac{1}{2m + 1} + \frac{1}{2m +
    2} > \frac{13}{24}$.

  Now $\frac{1}{2m + 2} + \frac{1}{2m + 1} - \frac{1}{m + 1} = \frac{2m + 1 + 2m + 2 - 4m - 2}{(2m + 1)(2m + 2)} =
  \frac{1}{(2m + 1)(2m + 2)} > 0$.

  Thus, $P(m + 1)$ is also true.

  Hence, by mathematical induction, the result.
\item Let $P(n) = \frac{1}{n + 1} + \frac{1}{n + 2} + \cdots + \frac{1}{3n + 1} > 1\forall
  n\in\mathbb{N}$. $P(1) = \frac{1}{2} + \frac{1}{3} + \frac{1}{4} = \frac{13}{12} > 2$, so the statement is
  true for $n = 1$.

  Let it be true for $n = m$.

  $\Rightarrow P(m) = \frac{1}{m + 1} + \frac{1}{m + 2} + \cdots + \frac{1}{3m + 1} > 1$.

  Adding $\frac{1}{3m + 2} + \frac{1}{3m + 3} + \frac{1}{3m + 4} - \frac{1}{m + 1}$, to L.H.S., we get

  $P(m + 1) = \frac{1}{m + 1} + \frac{1}{m + 2} + \frac{1}{m + 3} + \frac{1}{3m + 2} + \frac{1}{3m + 3} +
  \frac{1}{3m + 4} - \frac{1}{m + 1}$

  Now, $\frac{1}{3m + 2} + \frac{1}{3m + 3} + \frac{1}{3m + 4} - \frac{1}{m + 1}$

  $= \frac{(3m + 4)(3m + 3) +
    (3m + 2)(2m + 4) + (3m + 2)(3m + 3) - 3(3m + 2)(3m + 4)}{(3m + 2)(3m + 3)(3m + 4)} > 0$

  Thus, $P(m + 1) > 1$.

  Hence, by mathematical induction, the result.
\item Let $P(n) = 1 + \frac{1}{4} + \cdots + \frac{1}{n^2} < 2 - \frac{1}{n}$. $P(2) = \frac{5}{4} < 2 -
  \frac{1}{2} = \frac{3}{2}$, so the statement is true for $n = 2$.

  Let it be true for $n = m$.

  $P(m) = 1 + \frac{1}{4} + \cdots + \frac{1}{m^2} < 2 - \frac{1}{m}$.

  $P(m + 1) = 1 + \frac{1}{4} + \cdots + \frac{1}{(m + 1)^2} < 2 - \frac{1}{m} + \frac{1}{(m + 1)^2}$.

  Now $\frac{1}{(m + 1)^2} - \frac{1}{m} = \frac{m - m^2 - 2m - 1}{m(m + 1)^2} < \frac{1}{m + 1}$.

  Thus, $P(m + 1) < 2 - \frac{1}{m + 1}$.

  Hence, by mathematical induction, the result.
\item Let $P(n) = (2n + 7) < (n + 3)^2$. $P(1) = 8 < 16$, so the statement is true for $n = 1$.

  Let it be true for $n = m$.

  $\Rightarrow P(m) = 2m + 7 < (m + 3)^2$.

  $P(m + 1) = 2m + 9 < (m + 4)^2$. Subtracting $P(m + 1) - P(m)$, we get

  $2 < 2m + 7$, which is true for $m\in\mathbb{N}$.

  Hence, by mathematical induction, the result.
\item Let $P(n) = 2^n > n\forall n\in\mathbb{N}$. $P(1) = 2 > 1$, so the statement is true for $n = 1$.

  Let it be true for $n = m$.

  $\Rightarrow P(m) = 2^m > m$.

  $P(m + 1) = 2^{m + 1} > m + 1$. Dividing $P(m + 1)$ by $P(m)$, we get

  $2 > \frac{m + 1}{m}$, which is true for $m\in\mathbb{N}$.

  Thus, $P(m + 1)$ is true if $P(m)$ is true.

  Hence, by mathematical induction, the result.
\item Let $P(n) = 1 + 2 + 3 + \cdots + n < \frac{(2n + 1)^2}{8} \Rightarrow \frac{n(n + 1)}{2} < \frac{(2n +
  1)^2}{8} \Rightarrow n(n + 1) < \frac{(2n + 1)^2}{4}$. $P(1) = 2 < \frac{9}{4}$, so the statement is true
  for $n = 1$.

  Let it be true for $n = m$.

  $P(m) = m(m + 1) < \frac{(2m + 1)^2}{4}$, where $m\in\mathbb{N}$.

  Adding $m + 1$ to both sides we get

  $P(m + 1) = 1 + 2 + \cdots + m + m + 1 < \frac{(2m + 1)^2}{8} + (m + 1)$

  $= \frac{4m^2 + 4m + 1 + 8m + 8}{8}
  = \frac{(2m + 3)^2}{8}$.

  Hence, by mathematical induction, the result.
\item Let $P(n) = 1^2 + 2^2 + \cdots + n^2 > \frac{n^3}{3}$. $P(1) = 1 > \frac{1}{3}$, so the statement is
  true for $n = 1$.

  Let it be true for $n = m$.

  $\Rightarrow P(m) = 1^2 + 2^2 + \cdots + m^2 > \frac{m^3}{3}$, where $m\in\mathbb{N}$.

  Adding $(m + 1)^2$ to both sides

  $P(m + 1) = 1^2 + 2^2 + \cdots + m^2 + (m + 1)^2 > \frac{m^3}{3} + (m + 1)^2 = \frac{m^3 + 3m^2 + 6m +
    3}{3} > \frac{(m + 1)^3}{3}$.

  Hence, by mathematical induction, the result.
\item Let $P(n) = 2^n > n^2$, where $n \geq 5$. $P(5) = 2^5 > 5^2 = 32 > 25$, so the statement is true for
  $n = 1$.

  Let it be true for $n = m$, where $m\geq 5$.

  $P(m) = 2^m > m^2$.

  $P(m + 1) = 2^{m + 1} > (m + 1)^2 = 2.m^2 > (m + 1)^2 \Rightarrow m^2 -2m - 1 > 0$, which is true for $m >
  5$.

  Hence, by mathematical induction, the result.
\item Let $P(n) = \frac{(2n)!}{(n!)^2} > \frac{4^n}{n + 1}$, where $n > 1$. $P(2) = 12 > \frac{16}{3}$, so
  the statement is true for $n = 2$.

  Let it be true for $n = m$.

  $\Rightarrow P(m) = \frac{2m!}{(m!)^2} > \frac{4^m}{m + 1}$.

  $P(m + 1) = \frac{(2m + 2)!}{[(m+ 1)!]^2} > \frac{4^{m + 1}}{m + 2} > \frac{(2m + 2)(2m + 1)}{(m +
    1)^2}.\frac{4^m}{m + 1} > \frac{4^{m + 1}}{m + 2}$

  $= \frac{2(2m + 1)}{(m + 1)} > \frac{4}{m + 2}$, which is true for $m > 1$.

  Hence, by mathematical induction, the result.
\item Let $P(n) = (1 + x)^n > 1 + nx$. $P(2) = 1 + 2x + x^2 > 1 + 2x$, which is true for $x > -1$.

  Let it be true for $n = m$

  $\Rightarrow P(m) = (1 + x)^m > 1 + mx$

  $P(m + 1) = (1 + x)^{m + 1}> 1 + (m + 1)x = (1 + x)(1 + mx) > 1 + (m + 1)x = mx^2 > 0$, which is true for
  $n > 1$, $x > -1$.

  Hence, by mathematical induction, the result.
\stopitemize
