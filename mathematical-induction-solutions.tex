% -*- mode: context; -*-
\chapter{Mathematical Induction}
\startitemize[n, 1*broad]
\item Let $P(n) = 1^2 + 2^2 + \cdots + n^2 = \frac{n(n + 1)(2n + 1)}{6}$.

  Putting $n = 1$, we see $P(1) = 1 = \frac{1.2.3}{6} = 1$. So $P(1)$ is true.

  Let it be true for $n = k$. Now for $n = k + 1$,

  $P(k + 1) = 1^2 + 2^2 + \cdots + k^2 + (k + 1)^2 =  \frac{k(k + 1)(2k + 1)}{6} + (k + 1)^2$

  $= \frac{(k +
    1)(2k^2 + k + 6k + 6)}{6} = \frac{(k + 1)(k + 2)(2k + 3)}{6} = P(k + 1)$.

  Thus, by mathematical induction, the result.
\item Let $P(n) = \frac{1}{1.2} + \frac{1}{2.3} + \cdots + \frac{1}{n(n + 1)} = \frac{n}{n + 1}$.

  $P(1) = \frac{1}{1.2} = \frac{1}{1 + 1} = P(1)$, which is true for $n = 1$. Let it be true for $n = k$

  $\Rightarrow P(k) = \frac{1}{1.2} + \frac{1}{2.3} + \cdots + \frac{1}{k(k + 1)} = \frac{k}{k + 1}$.

  Adding $\frac{1}{(k + 1)(k + 2)}$, on both sides, we get

  $P(k + 1) = \frac{1}{1.2} + \frac{1}{2.3} + \cdots + \frac{1}{k(k + 1)} + \frac{1}{(k + 1)(k + 2)}$

  $= \frac{k}{(k + 1)} + \frac{1}{(k + 1)(k + 2)} = \frac{k^2 + 2k + 1}{(k + 1)(k + 2)} = \frac{k + 1}{k + 2} =
  P(k +1)$.

  Hence, by mathematical induction, the result.
\item Let $P(n) = 1^3 + 2^3 + \cdots + n^3 = \left[\frac{n(n + 1)}{2}\right]^2$

  $P(1) = 1^3 = 1 = \left(\frac{1.2}{2}\right)^3 = 1$, which is true for $n = 1$. Let it be true for $n =
  k$.

  $\Rightarrow P(k) = 1^3 + 2^3 + \cdots + k^3 = \left[\frac{k(k + 1)}{2}\right]^2$

  Adding $(k + 1)^3$ to both sides, we get

  $P(k + 1) = 1^3 + 2^3 + \cdots + k^3 + (k + 1)^3= \left[\frac{k(k + 1)}{2}\right]^2 + (k + 1)^3$

  $= \frac{(k
    + 1)^2[k^2 + 4k + 4]}{4} = \left(\frac{(k + 1)(k + 2)}{2}\right)^2 = P(k + 1)$.

  Hence, by mathematical induction, the result.
\item Let $P(n) = \frac{1}{a + d} + \frac{a}{(a + d)(a + 2d)} + \cdots + \frac{a}{[a + (n - 1)d](a + nd)} =
  \frac{n}{a + nd}$

  $P(1) = \frac{1}{a + d} = $R.H.S., which is true for $n = 1$. Let it be true for $n = k$

  $\Rightarrow P(k) = \frac{1}{a + d} + \frac{a}{(a + d)(a + 2d)} + \cdots + \frac{a}{[a + (k - 1)d](a +
    kd)} = \frac{k}{a + kd}$.

  Adding $\frac{a}{[a + kd][a + (k + 1)d]}$ to both sides, we get

  $P(k + 1) = \frac{1}{a + d} + \frac{a}{(a + d)(a + 2d)} + \cdots + \frac{a}{[a + (k - 1)d](a +
    kd)} + \frac{a}{[a + kd][a + (k + 1)d]}$

  $= \frac{k}{a + kd} + \frac{a}{[a + kd][a + (k + 1)d]} = \frac{ka
  + k(k + 1)d + a}{[a + kd][a + (k + 1)d]} = \frac{(k + 1)(a + kd)}{[a + kd][a + (k + 1)d]} = \frac{k + 1}{a
  + (k + 1)d} = P(k + 1)$

  Hence, by mathematical induction, the result.
\item Let $P(n) = \frac{1}{1.2.3} + \frac{1}{2.3.4} + \cdots + \frac{1}{n(n + 1)(n + 2)} = \frac{n(n +
  3)}{4(n + 1)(n + 2)}\forall n\in \mathbb{N}$

  $P(1) = \frac{1}{1.2.3} = \frac{1.4}{4.2.3} = \frac{1}{2.3}$, which is true for $n = 1$. Let it be true
  for $n = k$.

  $\Rightarrow P(k) = \frac{1}{1.2.3} + \frac{1}{2.3.4} + \cdots + \frac{1}{k(k + 1)(k + 2)} = \frac{k(k +
    3)}{4(k + 1)(k + 2)}$

  Adding $\frac{1}{(k + 1)(k + 2)(k + 3)}$ to both sides, we get

  $P(k + 1) = \frac{1}{1.2.3} + \frac{1}{2.3.4} + \cdots + \frac{1}{k(k + 1)(k + 2)} + \frac{1}{(k + 1)(k +
    2)(k + 3)}$

  $= \frac{k(k + 3)}{4(k + 1)(k + 2)} + \frac{1}{(k + 1)(k + 2)(k + 3)}$

  $= \frac{k(k + 3)^2 + 4}{4(k + 1)(k + 2)(k ++ 3)} = \frac{(k + 1)^2(k + 4)}{4(k + 1)(k + 2)(k + 3)}$

  $= \frac{(k + 1)(k + 4)}{4(k + 2)(k ++ 3)} = P(k + 1)$.

  Hence, by mathematical induction, the result.
\item Let $P(n) = 1.3 + 2.3^2 + 3.3^3 + \cdots + n.3^n = \frac{(2n - 1)3^{n + 1} + 3}{4}$

  $P(1) = 1.3 = 3 = \frac{(2 - 1).3^2 + 3}{4} = 3$, which is true for $n = 1$. Let it be true for $n = k$.

  Adding $(k + 1)3^{k + 1}$ to both sides, we get

  $\Rightarrow P(k + 1) = 1.3 + 2.3^2 + 3.3^3 + \cdots + k.3^k + (k + 1).3^{k + 1} = \frac{(2k - 1)3^{k + 1}
    + 3}{4}$\

  $+ (k
  + 1).3^{k + 1} = \frac{(2k - 1).3^{k + 1} + 3 + (4k + 4).3^{k + 1}}{4}$

  $= \frac{(6k + 3).3^{k + 1}}{4} = \frac{[2(k + 2) - 3].3^{k + 1}}{4} = P(k + 1)$.

  Hence, by mathematical induction, the result.
\item Let $P(n) = 1 + 4 + 7 + \cdots + 3n - 2 = \frac{n(3n - 1)}{2}$.

  $P(1) =  1 = \frac{1.(3 - 1)}{2} = 1$, which is true for $n = 1$. Let it be true for $n = k$.

  $P(k) = 1 = 4 + 7 + \cdots + 3k - 2 = \frac{k(3k - 1)}{2}$

  Adding $3k + 1$ to both sides, we get

  $\Rightarrow P(k + 1) = 1 + 4 + 7 + \cdots + 3k - 2 + 3k + 1 = \frac{k(3k - 1)}{2} + 3k + 1$

  $= \frac{3k^2 -
  k + 6k + 2}{2} = \frac{3k^2 + 5k + 2}{2} = \frac{(k + 1)(3k + 2)}{2} = \frac{(k + 1)[3(k + 1) - 1]}{2} =
  P(k + 1)$.

  Hence, by mathematical induction, the result.
\item Let $P(n) = 1^2 + 3^2 + 5^2 + \cdots + (2n - 1)^2 = \frac{n(2n - 1)(2n + 1)}{3}$.

  $P(1) = 1^2 = 1 = \frac{(12 - 1)(2 + 1)}{3} = 1$, which is true for $n = 1$. Let it be true for $n = k$

  $P(k) = 1^2 + 3^2 + 5^2 + \cdots + (2k - 1)^2 = \frac{k(2k - 1)(2k + 1)}{3}$.

  Adding $(2k + 1)^2$ to both sides, we get

  $\Rightarrow P(k + 1) = 1^2 + 3^2 + 5^2 + \cdots + (2k - 1)^2 + (2k + 1)^2 = \frac{k(4k^2 - 1)}{3} + (2k +
  1)^2$

  $= \frac{4k^3 - k + 12k^2 + 12k + 3}{3} = \frac{4k^3 + 122k^2 + 11k + 3}{3} = \frac{(k + 1)(2k + 1)(2k
    + 3)}{3} = P(k + 1)$.

  Hence, by mathematical induction, the result.
\item Let $P(n) = 1 - 3^2 + 5^2 - 7^2 + \cdots + (4n - 3)^2 - (4n - 1)^2 = -8n^2$

  $P(1) = 1 - 3^2 = -8 = -8.1^2$, which is true for $n = 1$. Let it be true for $n = k$.

  $P(k) = 1 - 3^2 + 5^2 - 7^2 + \cdots + (4k - 3)^2 - (4k - 1)^2 = -8k^2$

  Adding $(4k + 1)^2 - (4k + 2)^2$ to both sides, we get

  $P(k + 1) = 1 - 3^2 + 5^2 - 7^2 + \cdots + (4k - 3)^2 - (4k - 1)^2 + (4k + 1)^2 - (4k - 3)^2$

  $= -8k^2 + (4k + 1)^2 - (4k + 3)^2 = -8k^2 -16k - 8 = -8(k + 1)^2$.

  Hence, by mathematical induction, the result.
\item Let $P(n) = 3.6 + 6.9 + 9.12 + \cdots + 3n(3n + 3) = 3n(n + 1)(n + 2)$.

  $P(1) = 3.6 = 3.1(1 + 1)(1 + 2) = 3.6$, which is true for $n = 1$. Let it be true for $n = k$

  $P(k) = 3.6 + 6.9 + 9.12 + \cdots + 3k(3k + 3) = 3k(k + 1)(k + 2)$

  Adding $3(k + 1)[3(k + 1) + 3]$ to both sides, we get

  $P(k + 1) = 3.6 + 6.9 + 9.12 + \cdots + 3k(3k + 3) + 3(k + 1)[3(k + 1) + 3]$

  $= 3k(k + 1)(k + 2) + 3(k + 1).3(k + 2) = 3(k + 1)(k + 2)(k + 3) = P(k + 1)$.

  Hence, by mathematical induction, the result.
\item We have to prove that $1^3 = 1, 2^3 = 3 + 5, 3^3 = 7 + 9 + 11, 4^3 = 13 + 15 + 17 + 19$.

  First term contains one term, second terms contains two terms and so on. Hence, $k$th term will contain
  $k$ terms.

  Sum of no.\ of terms till $k$th term is $1 + 2 + \cdots + k = \frac{k(k + 1)}{2}$.

  And, hence $(k + 1)$th term will begin with $1 + \left(\frac{k(k + 1)}{2}\right)2 = k^2 + k + 1$ and will
  contain $k + 1$ terms with a c.d. of $2$. Let it be true for $P(k)$ i.e. $t_k = k^3$.

  Thus, $t_{k + 1} = \frac{k + 1}{2}[2(k^2 + k + 1) + (k + 1 - 1)2] = \frac{k + 1}{2}[2k^2 + 4k + 2] = (k +
  1)^3$.

  Hence, by mathematical induction, the result.
\item Let $P(n) = \displaystyle\sum_{r = 1}^nr.C_r^^n = n.2^{n - 1}$.

  $P(1) = 1.C_0^^1 = 1 = 1.2^{1 - 1} = 1$. Hence, it is true for $n = 1$. Let it be true for $n = k$.

  $\Rightarrow C_1^^k + 2.C_2^^k + \cdots + k.C_k^^k = k.2^{k - 1}$

  $P(k + 1) = C_1^^{k + 1} + 2.C_2^^{k + 1} + \cdots + (k + 1).C_{k + 1}^^{k + 1}$

  $= (C_0^^k + C_1^^k) + 2(C_1^^k + C_2^^k) + \cdots + (k + 1)(C_k^^k + 0)$

  $= (C_0^^k + 2C_1^^k + \cdots + (k + 1)C_k^^k) + (C_1^^k + 2.C_2^^k + \cdots + k.C_k^^k)$

  $= 2^k + k.2^{k - 1} + k.2^{k - 1} = (k + 1).2^k = P(k + 1)$.

  Hence, by mathematical induction, the result.
\item Let $P(n) = \displaystyle\sum_{r = 1}^nr(2r + 1) = \frac{n(n + 1)(4n + 5)}{6}$.

  $P(1) = 1.(2 + 1) = 3 = \frac{1.2.9}{6} = 3$. Hence, it it true for $n = 1$. Let it be true for $n = k$.

  $\Rightarrow \displaystyle\sum_{r = 1}^kr(2r + 1) = \frac{k(k + 1)(4k + 5)}{6}$.

  $P(k + 1) = \displaystyle\sum_{r = 1}^{k + 1}r(2r + 1) = \frac{k(k + 1)(4k + 5)}{6} + (k + 1)(2k + 3)$

  $= \frac{(k + 1)}{6}\left[\frac{4k^2 + 5k + 12k + 18}{}\right] = \frac{(k + 1)(k + 2)[4(k + 1) + 5]}{6}$.

  Hence, by mathematical induction, the result.
\item Let $P(n) = 1.2.3 + 2.3.4 + 3.4.5 + \cdots + n(n + 1)(n + 2) = \frac{n(n + 1)(n + 2)(n + 3)}{4}$.

  $P(1) = 1.2.3 = 6 = \frac{1.2.3.4}{4} = 6$, which is true for $n = 1$. Let it be true for $n = k$.

  $\Rightarrow 1.2.3 + 2.3.4 + 3.4.5 + \cdots + k(k + 1)(k + 2) = \frac{k(k + 1)(k + 2)(k + 3)}{4}$

  Adding $(k + 1)(k + 2)(k + 3)$ to both sides, we get

  $P(k + 1) = 1.2.3 + 2.3.4 + 3.4.5 + \cdots + k(k + 1)(k + 2) + (k + 1)(k + 2)(k + 3)$

  $= \frac{k(k + 1)(k + 2)(k + 3)}{4} + (k + 1)(k + 2)(k + 3)$

  $= \frac{(k + 1)(k + 2)(k + 3)(k + 4)}{4}$.

  Hence, by mathematical induction, the result.
\item Let $P(n) = \frac{1}{1.4} + \frac{1}{4.7} + \frac{1}{7.10} + \cdots + \frac{1}{(3n - 2)(3n + 1)} =
  \frac{n}{3n + 1}$

  $P(1) = \frac{1}{4} = \frac{1}{3 + 1} = \frac{1}{4}$, which is true for $n = 1$. Let it be true for $n =
  k$

  $\Rightarrow P(k) = \frac{1}{1.4} + \frac{1}{4.7} + \frac{1}{7.10} + \cdots + \frac{1}{(3k - 2)(3k + 1)} =
  \frac{k}{3k + 1}$

  Adding $\frac{1}{(3k + 1)(3k + 4)}$ to both sides, we get

  $P(k + 1) = \frac{1}{1.4} + \frac{1}{4.7} + \frac{1}{7.10} + \cdots + \frac{1}{(3n - 2)(3n + 1)} +
  \frac{1}{(3k + 1)(3k + 4)}$

  $= \frac{k}{3k + 1} + \frac{1}{(3k + 1)(3k + 4)} = \frac{3k^2 + 4k + 1}{(3k + 1)(3k + 4)} =\frac{k + 1}{3k
    + 4} = P(k + 1)$

  Hence, by mathematical induction, the result.
\item Let $P(n) = 7 + 77 + 777 + \cdots + \underbrace{7\ldots77}_{n\text{~digits}} = \frac{7}{81}(10^{n + 1} - 9n -
  10)$

  $P(1) = 7 = \frac{7}{81}(10^2 - 9 - 10) = 7$, which is true for $n = 1$. Let it be true for $n = k$

  $\Rightarrow 7 + 77 + 777 + \cdots + \underbrace{7\ldots77}_{k\text{~digits}} = \frac{7}{81}(10^{k + 1} - 9k -
  10)$

  Adding $\underbrace{7\ldots77}_{k + 1~\text{digits}}$ to both sides, we get

  $P(k + 1) = 7 + 77 + 777 + \cdots + \underbrace{7\ldots77}_{k\text{~digits}} + \underbrace{7\ldots77}_{k +
    1\text{~digits}}$

  $= \frac{7}{81}(10^{k + 1} - 9k - 10) + \underbrace{7\ldots77}_{k + 1\text{~digits}} = \frac{7}{81}(10^{k
    + 1}(10^{k + 1} - 9k - 10)) + \frac{7}{9}(10^{k + 1} - 1)$

  $= \frac{7}{9}\left[\frac{10^{k + 1} - 9k - 10 + 9.10^{k + 1} - 9}{9}\right] = \frac{7}{81}[10^{k + 2} -
    9(k + 1) - 10]$.

  Hence, by mathematical induction, the result.
\item Let $P(n) = 1 + \frac{1}{1 + 2} + \frac{1}{1 + 2 + 3} + \cdots + \frac{1}{1 + 2 + 3 + \cdots + n} =
  \frac{2n}{n + 1}$

  $P(1) = 1 = \frac{2.1}{1 + 1} = 1$, which is true for $n = 1$. Let it be true for $n = k$

  $\Rightarrow P(k) = 1 + \frac{1}{1 + 2} + \frac{1}{1 + 2 + 3} + \cdots + \frac{1}{1 + 2 + 3 + \cdots + k}
  = \frac{2k}{k + 1}$

  Adding $\frac{1}{1 + 2 + 3 + \cdots + (k + 1)}$ to both sides,we get

  $P(k + 1) = 1 + \frac{1}{1 + 2} + \frac{1}{1 + 2 + 3} + \cdots + \frac{1}{1 + 2 + 3 + \cdots + n} +
  \frac{1}{1 + 2 + 3 + \cdots + (k  + 1)} = \frac{2k}{k + 1} + \frac{1}{1 + 2 + 3 \cdots + (k + 1)}$

  $= \frac{2k}{k + 1} + \frac{2}{(k + 1)(k + 2)} = \frac{2}{k + 1}.\frac{k^2 + 2k + 1}{k + 2} = \frac{2(k +
    1)}{k + 2}$.

  Hence, by mathematical induction, the result.
\item Let $P(n) = \left(1 - \frac{1}{2^2}\right)\left(1 - \frac{1}{3^2}\right)\cdots\left(1 - \frac{1}{(n +
  1)^2}\right) = \frac{n + 2}{2n + 2}$

  $P(1) = 1 - \frac{1}{2^2} = \frac{3}{4} = \frac{1 + 2}{2.1 + 2} = \frac{3}{4}$, which is true for $n =
  1$. Let it be true for $n = k$

  $\Rightarrow P(k) = \left(1 - \frac{1}{2^2}\right)\left(1 - \frac{1}{3^2}\right)\cdots\left(1 - \frac{1}{(k +
    1)^2}\right) = \frac{k + 2}{2k + 2}$

  Multiplying both sides with $1 - \frac{1}{(k + 2)^2}$, we get

  $P(k + 1) = \left(1 - \frac{1}{2^2}\right)\left(1 - \frac{1}{3^2}\right)\cdots\left(1 - \frac{1}{(k +
    1)^2}\right)\left(1 - \frac{1}{(k + 2)^2}\right) = \frac{k + 2}{2k + 2}\left(1 - \frac{1}{(k +
    2)^2}\right)$

  $= \frac{k + 2}{2k + 2}.\frac{k^2 + 4k + 3}{(k + 2)^2} = \frac{k + 3}{2k + 4}$.

  Hence, by mathematical induction, the result.
\item Let $P(n) = 1.3 + 2.3^2 + \cdots + n.3^n = \frac{(2n - 1)3^{n + 1} + 3}{4}$.

  $P(1) = 3 = \frac{(2.1 - 1).3^{1 + 1} + 3}{4} = \frac{12}{4} = 3$, which is true for $n = 1$. Let it be
  true for $n = k$

  $\Rightarrow P(k) = 1.3 + 2.3^2 + \cdots + k.3^k = \frac{(2k - 1)3^{k + 1} + 3}{4}$

  Adding $(k + 1).3^{k + 1}$, to both sides, we get

  $P(k + 1) = 1.3 + 2.3^2 + \cdots + k.3^k + (k + 1).3^{k + 1} = \frac{(2k - 1)3^{k + 1} + 3}{4} + (k +
  1).3^{k + 1}$

  $= \frac{(2k - 1).3^{k + 1} + 3 + (4k + 4).3^{k + 1}}{4} = \frac{(6k + 3).3^{k + 1} = 3}{4} = \frac{[2(k +
      1) - 1].3^{k + 2} + 3}{4}$

  Hence, by mathematical induction, the result.
\item Let $P(n) = \cos\alpha + \cos2\alpha + \cdots + \cos n\alpha =
  \sin\frac{n\alpha}{2}\csc\frac{\alpha}{2}\cos\frac{(n + 1)\alpha}{2}$.

  $P(1) = \cos\alpha = \sin\frac{\alpha}{2}\csc\frac{\alpha}{2}\cos\frac{1 + 1}{2}\alpha = \cos\alpha$,
  which is true for $n = 1$. Let it be true for $n = k$.

  $\Rightarrow P(k) = \cos\alpha + \cos2\alpha + \cdots + \cos k\alpha =
  \sin\frac{k\alpha}{2}\csc\frac{\alpha}{2}\cos\frac{(k + 1)\alpha}{2}$

  Adding $\cos(k + 1)\alpha$, to both sides, we get

  $P(k + 1) = = \cos\alpha + \cos2\alpha + \cdots + \cos k\alpha + \cos(k + 1)\alpha =
  \sin\frac{k\alpha}{2}\csc\frac{\alpha}{2}\cos\frac{(k + 1)\alpha}{2} + \cos(k + 1)\alpha$

  $= \frac{1}{2}\csc\frac{\alpha}{2}\left[2\sin\frac{k\alpha}{2}\cos\frac{(k + 1)\alpha}{2} +
    2\sin\frac{\alpha}{2}\cos(k + 1)\alpha\right]$

  $=\frac{1}{2}\csc\frac{\alpha}{2}\left[\sin\frac{(2k + 1)\alpha}{2} - \sin\frac{\alpha}{2} + \sin\frac{(2k
      + 3)\alpha}{2} - \sin\frac{(2k + 1)\alpha}{2}\right]$

  $= \csc\frac{\alpha}{2}\cos\frac{(k + 2)\alpha}{2}\sin\frac{(k + 1)\alpha}{2}$

  Hence, by mathematical induction, the result.
\item Let $P(n) = \tan\alpha + 2\tan2\alpha + 2^2\tan2^2\alpha + \cdots + 2^{n - 1}\tan2^{n - 1}\alpha =
  \cot\alpha - 2^n\cot2^n\alpha$

  $P(1) = \tan\alpha = \cot\alpha - 2\cot2\alpha = \frac{1}{\tan\alpha} - \frac{2}{\tan2\alpha} =
  \frac{1}{\tan\alpha} - \frac{1 - \tan^2\alpha}{\tan\alpha} = \tan\alpha$,

  which is true for $n = 1$. Let it be true for $n = k$

  $\Rightarrow P(k) = \tan\alpha + 2\tan2\alpha + 2^2\tan2^2\alpha + \cdots + 2^{k - 1}\tan2^{k - 1}\alpha =
  \cot\alpha - 2^k\cot2^k\alpha$

  Adding $2^k\tan2^k\alpha$, to both sides, we get

  $P(k + 1) = \tan\alpha + 2\tan2\alpha + 2^2\tan2^2\alpha + \cdots + 2^{k - 1}\tan2^{k - 1}\alpha +
  2^k\tan2^k\alpha$

  $= \cot\alpha - 2^k\cot2^k\alpha + 2^k\tan2^k\alpha = \cot\alpha - 2^k\left(\cot2^k\alpha -
  \tan2^k\alpha\right)$

  $= \cot\alpha - 2^{k+ 1}\left(\frac{1 - \tan2^{k + 1}\alpha}{2\tan2^k\alpha}\right) = \cot\alpha - 2^{k +
    1}\cot2^{k + 1}\alpha$.

  Hence, by mathematical induction, the result.
\item Let $P(n) = \tan^{-1}\frac{1}{3} + \tan^{-1}\frac{1}{7} + \cdots + \tan^{-1}\frac{1}{n^2 + n + 1} =
  \tan^{-1}\frac{n}{n + 2}$

  $P(1) = \tan^{-1}\frac{1}{3} = \tan^{-1}\frac{1}{1 + 2} = \tan^{-1}\frac{1}{3}$, which is true for n =
  1. Let it be true for $n = k$

  $\Rightarrow P(k) = \tan^{-1}\frac{1}{3} + \tan^{-1}\frac{1}{7} + \cdots + \tan^{-1}\frac{1}{k^2 + k + 1} =
  \tan^{-1}\frac{k}{k + 2}$

  Adding $\tan^{-1}\frac{1}{k^2 + 3k + 3}$, to both sides, we get

  $P(k + 1) = \tan^{-1}\frac{1}{3} + \tan^{-1}\frac{1}{7} + \cdots + \tan^{-1}\frac{1}{k^2 + k + 1} +
  \tan^{-1}\frac{1}{k^2 + 3k + 3}$

  $=\tan^{-1}\frac{k}{k + 2} + \tan^{-1}\frac{1}{k^2 + 3k + 3} = \tan^{-1}\frac{\frac{k}{k + 2} +
    \frac{1}{k^2 + 3k + 3}}{1 - \frac{k}{k + 2}.\frac{1}{k^2 + 3k + 3}} = \tan^{-1}\frac{k^3 + 3k^2 + 3k + k
    + 2}{k^3 + 5k^2 + 9k + 6 - k}$

  $=\tan^{-1}\frac{k^3 + 3k^2 + 4k + 2}{k^3 + 5k^2 + 8k + 6} = \tan^{-1}\frac{k + 1}{k + 3}$.

  Hence, by mathematical induction, the result.
\item $u_3 = u_2 + u_1$. Substituting $n = 3$ in the given formula $u_3 =
  \frac{1}{\sqrt{5}}\left[\left(\frac{1 + \sqrt{5}}{2}\right)^3 - \left(\frac{1 -
      \sqrt{5}}{2}\right)^3\right]$

  $= \frac{1}{\sqrt{5}}\left[\frac{1 + 3\sqrt{5} + 15 + 5\sqrt{5}}{8} - \frac{1 - 3\sqrt{5} + 15 -
      5\sqrt{5}}{8}\right] = \frac{1}{\sqrt{5}}\frac{16\sqrt{5}}{8}\right = 2 = u_1 + u_2$.

  Thus, the relation holds for $n = 3$. Similarly, we can prove that it holds for $m = 1, 2$. Let it hold
  for $n = k$ and $k + 1$.

  $\Rightarrow u_k = \frac{1}{\sqrt{5}}\left[\left(\frac{1 + \sqrt{5}}{2}\right)^k - \left(\frac{1 -
      \sqrt{5}}{2}\right)^k\right]$ and $u_{k + 1} = \frac{1}{\sqrt{5}}\left[\left(\frac{1 +
      \sqrt{5}}{2}\right)^{k + 1} - \left(\frac{1 - \sqrt{5}}{2}\right)^{k + 1}\right]$.

  $u_{k + 2} = u_k + u_{k + 1} = \frac{1}{\sqrt{5}}\left[\left(\frac{1 + \sqrt{5}}{2}\right)^k\left(1 +
    \frac{1 + \sqrt{5}}{2}\right) - \left(\frac{1 + \sqrt{5}}{2}\right)^k\left(1 - \frac{1 -
      \sqrt{5}}{2}\right)\right]$

  $= \frac{1}{\sqrt{5}}\left[\left(\frac{1 + \sqrt{5}}{2}\right)^k\left(\frac{1 + \sqrt{5}}{2}\right)^2 -
    \left(\frac{1 - \sqrt{5}}{2}^k\right)\left(\frac{1 - \sqrt{5}}{2}\right)^2\right] = u_{k + 2}$.

  Hence, by mathematical induction, the result.
\item Let $P(n) = p^{n + 1} + (p + 1)^{2n - 1}$.

  $P(1) = p^2 + p + 1$, which is divisible by $p^2 + p + 1$, and hence, our statement is true for $n =
  1$. Let it be true for $n = k$.

  $\Rightarrow P(k) = p^{k + 1} + (p + 1)^{2k - 1}$ is divisible by $p^2 + p + 1$ i.e. $p^{k + 1} + (p + 1)^{2k + 1} =
  (p^2 + p + 1)Q(p)$, where $Q(p)$, is a polynomial of $p$.

  $P(k + 1) = p^{k + 2} + (p + 1)^{2k + 1} = p.p^{k + 1} + (p + 1)^2(p + 1)^{2k - 1}$

  $\therefore P(k) = (p^2 + p + 1)^2Q(k)$, making it divisible by $p^2 + p + 1$.

  Hence, by mathematical induction, the result.
\item Let $P(n) = 2^n > 2n + 1$, where $n > 2$

  $P(3) = 2^3 = 8 > 2.3 + 1 = 7$, hence, it is true for $n = 3$. Let it be true for $n = k$.

  $\Rightarrow P(k) = 2^k > 2k + 1$, where $k> 2$.

  $P(k + 1) = 2^{k + 1} = 2.2^k = 4k + 2 = 2k + 2k + 2 \because k \geq 3\therefore 2k + 2 > 3$, making our
  statement true for $n = k + 1$.

  hence, by mathematical induction, the result.
\item Let $P(n) = 2^n > n^3$, where $n \geq 10$.

  $P(10) = 2^{10} = 1024 > 10^3 = 1000$, hence, it is true for $n = 3$. Let it be true for $n = k$.

  $\Rightarrow P(k) = 2^k > k^3$

  $P(k + 1) = 2^{k + 1} > 2.k^3 >(k + 1)^3 \Rightarrow k^3 - 3k^2 - 3k - 1 > 0 \Rightarrow (k - 1)^3 - 6k > 0$

  Let $k = 10 + a$, where $a\geq 0$, hence, $(9 + a)^3 - 60 - 6a = 669 + 183a + 27a^2 + a^3 > 0$

  hence, by mathematical induction, the result.
\item Given, $n > 1$, so we start with $n = 2$. $\Rightarrow \tan2\alpha = \frac{2\tan\alpha}{1 -
  \tan^2\alpha}> 2\tan\alpha\because1 - \tan^2\alpha < 1$,

  which is true for $n = 2$. Let the statement for $n = k$

  $\Rightarrow \tan k\alpha > k\tan\alpha$. For $n = k + 1$

  $\tan(k + 1)\alpha = \frac{\tan\alpha + \tan k\alpha}{1 - \tan\alpha.\tan k\alpha}> \frac{k\tan\alpha +
    \tan\alpha}{1 - \tan k\alpha\tan\alpha} > (k + 1)\tan\alpha\because 1 - \tan\alpha\tan k\alpha < 1$

  Hence, by mathematical induction, the result.
\item Let $P(n) = n^4 < 10^n\;\forall\;n\geq 2$

  For $n = 1, P(2) = 2^4 < 10^2 \Rightarrow 16 < 100$, which is true for $n = 2$. Let $P(k)$ be true
  i.e. $k^4 < 10^k$.

  We have to prove that $P(k + 1)$ is true i.e. $(k + 1)^4 < 10^{k + 1}$.

  Clearly, $10^{k + 1} > 10k^4$. Now, $\frac{10k^4}{(k + 1)^4} = 10\left(\frac{k}{k + 1}\right)^4$

  $\because k\geq 2\Rightarrow \left(\frac{k}{k + 1}\right)^4\geq \frac{2^4}{3^4}\Rightarrow
  10\left(\frac{k}{k + 1}\right)^4 \geq 10.\frac{16}{81} > 1$.

  Thus, $10^{k + 1} > (k + 1)^4$. Hence, by mathematical induction, the result.
\item Let $P(n) = 1^3 + 3^3 + \cdots + (2n - 1)^3 = n^2(2n^2 - 1)$.

  $P(1) = 1^3 = 1 = 1^2(2.1^1 - 1) = 1$, which is true for $n = 1$. Let it be true for $n = k$.

  $P(k) = 1^3 + 3^3 + \cdots + (2k - 1)^3 = k^2(2k^2 - 1)$

  Adding $(2k + 1)^3$, to both sides, we get

  $P(k + 1) = 1^3 + 3^3 + \cdots + (2k - 1)^3 + (2k + 1)^3 = k^2(2k^2 - 1) + (2k + 1)^3$

  $= 2k^4 - k^2 + 8k^2 + 12k^2 + 6k + 1 = (k + 1)^2[2(k + 1)^2 - 1]$.

  Hence, by mathematical induction, the result.
\stopitemize