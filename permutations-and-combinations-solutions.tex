% -*- mode: context; -*-
\chapter{Permutations and Combinations}
\startitemize[n, 1*broad]
\item Given, $P_4^^n = 360 \Rightarrow n(n - 1)(n - 2)(n - 3) = 3\times 4\times 5\times 6\Rightarrow n = 6$.
\item Given, $P_3^^n = 9240 \Rightarrow n(n - 1)(n - 2) = 20\times21\times22 \Rightarrow n = 22$.
\item Given, $P_r^^{10} = 720 = 8\times9\times10 \Rightarrow r = 3$.
\item Given, $P_{n - 1}^^{2n + 1}:P_n^^{2n - 1} = 3: 5 \Rightarrow \frac{(2n + 1)!}{(n + 2)!}.\frac{(2n -
  1)!}{(n -- 1)!} = \frac{3}{5}\Rightarrow \frac{(2n + 1)2n}{n(n + 1)(n + 2)}= \frac{3}{5}$

  $\Rightarrow 3n^2 - 11n - 4 = 0 \Rightarrow n = 4, -\frac{1}{3}$, but $n$ is an integer. Hence, $n = 4$.
\item Given, $P_4^^n = 12\times P_2^^n \Rightarrow n(n - 1)(n - 2)(n - 3) = 12\times n(n - 1)\Rightarrow n^2
  - 5n - 6 = 0 \Rightarrow n = 6, -1$.

  But $n > 0 \Rightarrow n = 6$ is the only solution.
\item Given, $P_5^^n = 20\times P_3^^n \Rightarrow (n -3)(n - 4) = 20  \Rightarrow n^2 - 7n - 8 =
  0\Rightarrow n = 8, -1$.

  But $n > 0 \Rightarrow n = 8$ is the only solution.
\item Given, $P_4^^n:P_4^^{n+1} = 3:4\Rightarrow \frac{n!}{(n - 4)!}.\frac{(n - 3)!}{(n + 1)!} =
  \frac{3}{4}$

  $\Rightarrow \frac{(n - 3)}{n + 1} = \frac{3}{4}\Rightarrow 4n - 12 = 3n + 3 \Rightarrow n = 15$.
\item Given $P_r^^{20} = 6840 = 18\times19\times20 \Rightarrow r = 3$.
\item Given, $P_{k + 1}^^{k + 5} = \frac{11(k - 1)}{2}.P_{k}^^{k + 3} \Rightarrow (k + 5)(k + 4)(k + 3)\cdots
  6.5 = \frac{11(k - 1)}{2}.(k + 3)(k + 2)\cdots 5.4$

  $\Rightarrow (k + 5)(k + 4) = 22k - 22 \Rightarrow k^2 - 13k + 42 = 0 \Rightarrow k = 6, 7$.
\item Given, $P_{r + 1}^^{22}:P_{r + 2}^^{20} = 11:52\Rightarrow \frac{22!}{(21 - r)!}.\frac{(18 - r)!}{20!}
  = \frac{11}{54}$

  $\Rightarrow \frac{22.21}{(21 - r)(20 - r)(19 - r)} = \frac{11}{52} \Rightarrow (21 - r)(20 - r)(19 - r) =
  42.52 = 12. 13.14 \Rightarrow r = 7$.
\item Given, $P_2^^{m + n} = 90 \Rightarrow (m + n)(m + n - 1) = 10.9 \Rightarrow m + n = 10$, and

  $P_2^^{m - n} = 30\Rightarrow (m - n)(m - n - 1) = 6.5\Rightarrow m - n = 6 \Rightarrow m = 8, n = 2$.
\item Given, $P_r^^{12} = 11880 \Rightarrow \frac{12!}{(12 - r)!} = 9\times\10\times11\times 12 \Rightarrow
  r = 4$.
\item Given, $P_{r + 6}^^{56}:P_{r + 3}^^{54} = 30800:1 \Rightarrow \frac{56!}{(50 - r)!}.\frac{(51 -
  r)!}{54!} = 30800$

  $\Rightarrow 56\times55\times(51 - r) = 30800 \Rightarrow 51 - r = 10 \Rightarrow r = 41$.
\item $n.P_n^^n = n.n! = (n + 1 - 1).n! = (n + 1)! - n!$. Similarly, $(n - 1).P_{n - 1}^^{n - 1} = n! - (n -
  1)!, \ldots, 2.P_2^^2 = 3! - 2!, 1.P_1^^1 = 2! - 1!$.

  Adding these, we obtain L.H.S. $= (n + 1)! - 1! = P_{n + 1}^^{n + 1} - 1 =$ R.H.S.
\item Given, $C_{30}^^n = C_4^^n \Rightarrow \frac{n!}{30!(n - 30)!} = \frac{n!}{4!(n - 4)!}$

  Equating $n - 30 = 4$ and $n - 4 = 30$, we obtain $n = 34$ from both.
\item Given, $C_{12}^^n = C_8^^n \Rightarrow \frac{n!}{(n - 12)!12!} = \frac{n!}{(n - 8)!8!} \Rightarrow n -
  12 = 8$ and $n - 8 = 12$. Thus, $n = 20$

  $C_{17}^^{20} = \frac{20!}{17!3!} = \frac{20\times19\times18}{3\times2} = 1140$, and
  $C_{20}^^{22} = \frac{22!}{20!2!} = \frac{22\times21}{2} = 231$.
\item Given, $C_r^^{18} = C_{r + 2}^^18 \Rightarrow \frac{18!}{(18 - r)!r!} = \frac{18!}{(r + 2)!(16 - r)!}
  \Rightarrow 18 - r = r + 2 \Rightarrow r = 8$ and $r = 16 - r \Rightarrow r = 8$.

  $C_6^^r = C_6^^8 = \frac{8!}{6!2!} = 28$.
\item Given, $C_{n- 4}^^n = 15 \Rightarrow \frac{n!}{(n - 4)!4!} = 15 \Rightarrow n(n - 1)(n - 2)(n - 3) =
  3\times 4\times 5\times 6\Rightarrow n = 6$.
\item Given, $C_r^^{15}:C_{r - 1}^^15 = 11:5\Rightarrow \frac{15!}{(15 - r)!r!}.\frac{(r - 1)!(16 -
  r!)}{15!} = \frac{11}{5}\Rightarrow \frac{16 - r}{r} = \frac{11}{5}\Rightarrow r = 5$.
\item Given, $P_r^^n = 2520 \Rightarrow \frac{n!}{(n - r)!} = 2520$ and $C_r^^n = 21 \Rightarrow
  \frac{n!}{(n - r)!r!} = 21$

  $\Rightarrow \frac{2520}{r!} = 21 \Rightarrow r! = 120 \Rightarrow r = 5\Rightarrow n(n - 1)(n - 2)(n -
  3)(n - 4) = 2520 = 7\times 6\times 5\times 4\times 3\Rightarrow n = 7$.
\item We know that $C_r^^n = C_{n - r}^^n\Rightarrow C_{13}^^{20} = C_7^^{20}$ and $C_{14}^^{20} =
  C_6^^{20}$.

  $\therefore C_{13}^^{20} + C_{14}^^{20} - C_6^^{20} - C_7^^{20} = 0$.
\item Given, $C_{r - 1}^^n = 36\Rightarrow \frac{n!}{(n - r + 1)(r - 1)!} = 36, C_r^^n = 84 \Rightarrow
  \frac{n!}{(n - r)!r!} = 84$, and $\frac{n!}{(n - r - 1)!(r + 1)!} = 126$.

  Dividing first two, $\frac{r}{n - r + 1} = \frac{3}{7}\Rightarrow 3n = 10r - 3$, and dividing last two

  $\frac{r + 1}{n - r} = \frac{2}{3}\Rightarrow 2n = 5r + 3$. Solving these two equations, we have $n = 9, r
  = 3$.
\item Thoudand's place can be filled in $5$ ways, hundred's place can be filled in $4$ ways, ten's place can
  be filled in $3$ ways and unit's place can be filled in $2$ ways.

  Thus, total number of $4$ digit numbers is $5\times 4\times 3\times 2 = 120$.

  Alternatively, it is $P_4^^5 = 120$.
\item Hundred's place can be filled in $3$ ways excluding $0, 2, 3$, ten's place can be filled in $5$ ways
  and unit's place can be filled in $4$ ways.

  Thus, no.\ of numbers between $400$ and $1000$ is $5\times 4\times 3 = 60$.
\item {\bf Case I:} When the number is of three digits i.e. between $300$ and $1000$.

  Hundred's place can be filled in $3$ ways using $3, 4$ or $5$, ten's place can be filled in $5$ ways and
  unit's place can be filled in $4$ ways.

  Thus, total no.\ of three digit numbers is $5\times 4\times 3 = 60$.

  {\bf Case II:} When the number is of four digits i.e. between $1000$ and $3000$.

  Thousand's place can be filled in $2$ ways using $1$ or $2$. Three remaining places can be filled in
  $P_3^^5$ ways i.e. $60$ ways.

  Therefore, total no.\ of four digit numbers is $2\times 60 = 120$.

  Thus, total no.\ of numbers between $300$ and $3000$ is $60 + 120 = 180$.
\item {\bf Case I:} When $2$ is at thousands place.

  Hundred's placec can be filled in $4$ ways using $3, 4, 5, 6$. Two remaining places can be filled in
  $P_2^^5$ i.e. $20$ ways. Number of numbers formed in this case is $4\times 20 = 80$.

  {\bf Case II:} When thousands place is occupied by $3, 4, 5$ or $6$.

  We see that there are four ways to fill thousands place. Three remaining placed can be filled in $P_3^^6$
  i.e. $120$ ways. Number of numbers formed in this case is $4\times 120 = 480$.

  Hence, total no.\ of numbers is $80 + 480 = 560$.
\item {\bf Case I:} When the number is of one digit.

  There will be four positive numbers excluding $0$.

  {\bf Case II}: When the number is of two digits.

  Ten's place can be filled in $4$ ways using $1, 2, 3$ or $4$. Unit's place can be filled in $P_1^^4$
  ways. Total no.\ of one digit numbers is $4\times P_1^^4 = 16$.

  {\bf Case III:} When the number is of three digits.

  Hundred's place can be filled in $4$ ways like previous case. Remaining two places can be filled in
  $P_2^^4$ ways. Total no.\ of three digit numbers is $4\times P_2^^4 = 48$.

  {\bf Case IV:} When the number is four digits.

  Thousand's place can be filled in $4$ ways like previous case. Remaining three places can be filled in
  $P_3^^4$ ways. Total no.\ of four digit numbers is $4\times P_3^^4 = 96$.

  {\bf Case V:} When the number is of five digits.

  Ten thousand's place can be filled in $4$ ways. Remaining four places can be filled in $P_4^^4$
  ways. Total no.\ of five digit numbers is $4\times P_4^^4 = 96$.

  Thus, total no.\ of numbers formed is $4 + 16 + 48 + 96 + 96 = 260$.
\item Total no.\ of numbers will be $P_4^^4 = 24$. Now since there are $4$ digits and $24$ numbers each
  no.\ will occur at each place for $6$ times. Thus, sum of digits at each place would be $6(1 + 2 + 3 + 4) =
  60$.

  Therefore, sum of all numbers $60(1 + 10 + 100 + 1000) = 66660$.
\item When any digit except $0$ will occupy unit's place the thousand's place has to be occupied by the
  other two digits. Thus, total no.\ of such numbers is $3\times2\times P_2^^2 = 12$. Thus, $4$ numbers for
  each of positive digits.

  When one of $1, 2, 3$ occupy thousand's place total no.\ of numbers is $3\times P_3^^3 = 18$. Thus, $6$
  numbers for each of the positive digits.

  Sum of digits at units, tens and thousands place will be $4(1 + 2 + 3) = 24$ and sum of digits at
  thousands place will be $6(1 + 2 + 3) = 36$.

  Thus, sum of numbers formed is $24(1 + 10 + 100) + 36\times 1000 = 38,664$.
\item Each of the four digits $1, 2, 2, 3$ occurs at each place $\frac{P_3^^3}{2!}$ i.e. $3$ times. Thus,
  sum of digits at each place is $3(1 + 2 + 2 + 3) = 24$.

  Thus, sum of numbers formed $24(1 + 10 + 100 + 1000) = 26,664$.
\item Each friennd can be sent invitation by one servant. Since there are three servants each friend can
  receive an invitaion in three ways. Thus, total no.\ of ways of sending invitations is $3^6 = 729$.
\item Each prize can be given to any boy. Thus, each prize can be given in $7$ ways, and hence, three
  prizes can be given in $7^3 = 543$ ways.
\item Each arm can occupy four positions, and thus, five arms can have $4^5 = 1024$ ways. But when all arms
  are in rest position no signal can be made. Hence, total no.\ of signal is $1024 -1 = 1023$ ways.
\item Each ring of lock can have one of the ten letters, then three rings can have $10^3$ combinations of
  the letters. However, one of the combinations will be a successful combination.

  Thus, total no.\ of possible unsuccessful attempts that can be made is $1000 - 1 = 999$.
\item We have to find numbers which are greater than $1000$ but not greater than $4000$ i.e. $1000< x\leq
  4000$ which is same as $1000\leq x < 4000$.

  Now thousands place can be filled with $1, 2, 3$ i.e. in $3$ ways. Hundreds, tens and units place can be
  filled in $5$ ways each.

  Thus, total no.\ of numbers which can be formed is $3\times 5^3 = 375$.
\item There are three groups. We can arrange three groups in $3!$ ways. $8$ Indians can be arranged among
  themselves in $8!$ ways, $4$ Ameriacans in $4!$ ways and $4$ Englishmen in $4!$ ways.

  Thus, required answer is $3!8!4!4!$.
\item Total no.\ of volumes is $4 + 1 + 1 + 1 = 7$. We can arrange these volumes in $7!$ ways. $8$ books
  volume can be arranged in $8!$ ways, volume having $5$ books can be arranged in $5!$ ways and volume of
  $3$ books can be arranged in $3!$ ways.

  Thus, required no.\ of arrangements is $7!8!5!3!$.
\item Taking all copies of the same book as one, we have $5$ books, which can be arranged in $5!$ ways.

  All copies being identical can be arranged only in $1$ way. Thus, required no.\ of arrangements is $5! =
  120$.
\item The no.\ of permutations of the $10$ papers without restriction is $10!$.

  We find our no.\ of ways in which the best and worst paper come together then subtract from total no.\ of
  permutations to get the no.\ of permutations in which they never come together.

  Taking the best and the worst paper as one paper we have $9$ papers, which can be arranged in $9!$ ways,
  but the two papers can be arranged among themselves in $2!$ ways. Thus, total no.\ of permutatiosn in which
  both the papers are toegther is $9!2!$.

  Thus, no.\ of permutations in which both are not together is $10! - 9!2! = 8.9!$.
\item Total no.\ ways in which all of them can be seated is $(5 + 3)! = 8!$. Taking all the girls as one
  total no.\ of persons is $6$.

  The no.\ of ways in which these can be seated is $6!$, but the $3$ girls can be arranged in $3!$
  ways. Thus, total no.\ of ways, when all three girls are together can be seate, is $6!3!$.

  Thus, total no.\ of ways in which all girls are not together is $8! - 6!3! = 36,000$.
\item Let us first position I.A. students. $*IA*IA*IA*IA*IA*IA*IA*$. The IA indicated the position where
  I.A. students sit and * indicated the positions where I.Sc. students can sit. We observe that there are
  $8$ open places where I.Sc. students can sit.

  Now, $7$ I.A. students can be seated in $7!$ ways and $8$ I.Sc. students can be seated in $P_5^^8$ ways.

  Thus, no.\ of required arrangements is $7!.\frac{8!}{3!}$.
\item Positioning the boys first, we have $*B*B*B*B*B*B*B*$, where $B$s represents the $7$ boys and $*$s
  represents the open positions for girls.

  $7$ boys can be arranged in $7!$ ways and $3$ girls can be seated in $P_3^^8$ ways. Thus, required no.\ of
  seating arrangememnts is $7!.\frac{8!}{5!} = 42.8!$.
\item {\bf Case I:} When a boy sits at the first place. The possible arrangement in this case is
  $BGBGBGBG$, where $B$ represents a boy and $G$ represents a girl. Now, $4$ boys and $4$ girls can be
  arranged among themselves in $4!$ ways. Thus, no.\ of possible seating arrangement in this case is $4!4!$.

  {\bf Case II:} When a girl sits at the first place. Like previous case the possible no.\ of seating
  arrangements is same i.e. $4!4!$.

  Thus, total no.\ of seating arrangements is $2.4!4! = 1152$.
\item Possible arrangements will have the form $BGBGBGB$, where $B$ represents a boy, and $G$ represents a
  girl. $4$ boys can be seated in $4!$ ways and $3$ girls can be seated in $3!$ ways.

  Thus, total no.\ of seating arrangements is $4!3!$.
\item There are $12$ letters in the word civilization; out of which $4$ are i's and other are different.

  Therefore, total no.\ of permutations is $\frac{12!}{4!}$, which included the word civilization itself.
\item There are $10$ letters in the word university; out of which $4$ are vowels, and {\it i} occurs
  twice. The consonants do not have repetition.

  Treating the $4$ vowels as one letter, because they have to appear together, we have $7$ letters. These
  $7$ letters can be arranged in $7!$ ways. But the four vowels can be arranged among themselves in
  $\frac{4!}{2!}$ ways.

  Thus, total no.of words possible is $7!\frac{4!}{2!}$.
\item There are $8$ letters in the word director; out of which $3$ are vowels, and {\it r} occurs
  twice. Thus, total no.\ of words is $\frac{8!}{2!}$ .

  When the vowels are together, taking them as one letter, we have $6$ letters, which can be arranged in
  $\frac{6!}{2!}$, but the three vowels can be arranged in $3!$ ways among themselves, making the total
  no.\ of words in which vowels are together $3!\frac{6!}{2!}$.

  Thus, no.\ of words in which all three vowels are not together is $\frac{8!}{2!} - 3!\frac{6!}{2!}$.
\item There are $7$ letters in the word welcome; out of which {\it e} occurs twice. Thus, total no.\ of words
  that can be formed is $\frac{7!}{2!}$.

  If \quote{o} comes at end then we will have $6$ letters left giving us total no.\ of words as
  $\frac{6!}{2!}$.
\item There are $10$ letters in the word California; out of which $5$ are consonants without repetition and
  $5$ vowels with {\it a} and {\it i} occurring twice.

  Thus, consonants can be arranged in $5!$ ways and vowels can be arranged in $\frac{5!}{2!2!}$ ways.

  Thus, total no.\ of words possible such that consoanats and vowels occupy their respective places is
  $\frac{5!5!}{2!2!}$.
\item There are $6$ letters in the word pencil with two vowels and three even positions. Thus, vowels can be
  arranged in $P_2^^3 = 6$ ways.

  Rest four positions can be filled in $4! = 24$ ways. Thus, total no.\ of words is $24\times6 = 144$.
\item From $5$ letters $5!= 120$ words can be formed. Consider the form of word when no two vowels are
  together. $VCVCV$, where {\it C} represents consonants and {\it V} represents the vowels.

  Clearly, consonants can be arranged in $2!$ ways and vowels can be arranged in $P_3^^3 = 3! = 6$ ways.

  Thus, no.\ of words where vowels are not together is $2\times6 = 12$.
\item There are seven digits given and we have to form numbers greater than one million, which implies all
  seven digits will have to used. Among the given digits $3$ comes thrice and $2$ comes twice. Thus, total
  no.\ of numbers which can be formed is $\frac{7!}{3!2!} = 420$.

  However, these numbers also contain the numbers where zero is the first digits making them less than one
  million. no.\ of such numbers is $\frac{6!}{3!2!} = 60$.

  Hence, no.\ of numbers greater than one million is $420 - 60 = 360$.
\item
  \startitemize[i]
  \item Total no.\ of persons is $5 + 4 = 9$. With no restirctions they can be seated at a round table in $(9
    - 1)! = 8!$ ways.
  \item Treating all British as a single person because they have to be together we have $6$ persons which
    can be seated in $5!$ ways. But $4$ Britishers can be arranged among themselves in $4!$ ways making the
    total no.\ of ways $5!4!$.
  \item This is equal to $8! - 5!4!$ from previous parts.
  \item First we seat the $5$ Indians in $4!$ ways. Then that will leave $5$ positions open for Britishers
    between Indians to sit, which gives us $P_4^^5$ ways. Thus, total no.\ of ways in which no two Britishers
    are together is $4!5!$.
  \stopitemize
\item $5$ Indians can be seated in a circle in $4!$ ways. We will have $5$ positions between Indians in
  which we can seat $5$ Britishers in $P_5^^5 = 5!$ ways.

  Thus, total no.\ of required ways is $5!4!$.
\item Taking the two delegates who have to always sit together as a single person we have $19$ persons
  which can be seated in $18!$ ways around a round table.

  However, the two delegates themselves can be arranged in $2!$ ways making the required no.\ of ways
  $18!2!$.
\item no.\ of four digit numbers which can be formed with $1, 2, 4, 5, 7$ i.e. $5$ digits is $P_4^^5 = 120$.
\item Units place cannot be filled with $0$ so it can be filled in $4$ ways using one of $1, 2, 3, 4$. Rest
  four positions can be filled in $P_4^^4 = 4! = 24$ ways.

  Thus, no.\ of $5$ digit numbers is $4\times24 = 96$.
\item no.\ of given digits is $7$ and we have to make numbers between $100$ and $1000$ i.e. three digit
  numbers. Since there is no zero in the given digits the required no.\ of numbers is $P_3^^7 = 210$.
\item Units place be filled in $5$ ways excluding $0$ and two remaining places can be filled by remaining
  $5$ digits in $P_2^^5 = 20$ ways.

  Thus, total no.\ of required numbers is $5\times10 = 100$.
\item We have $10$ digits. Units place can be filled in $9$ ways excluding $0$. Rest $8$ places can be
  filled using remaining $9$ digits in $P_8^^9 = 9!$ ways.

  Thus, total no.\ of $9$ digit numbers with no repetition is $9.9!$.
\item Thousannds place can be filled in $5$ ways excluding $0$. Rest three places can be filled using
  remaining $5$ digits in $P_3^^5 = 60$ ways.

  Thus, no.\ of required numbers is $5\times60 = 300$.
\item Thousands place can be filled in $2$ ways using either $5$ or $9$. Rest three places can be filled in
  $3!$ ways using remaining three digits.

  Thus, no.\ of required numbers is $2.3! = 12$.
\item {\bf Case I:} When the number is of three digits.

  Hundreds place can be filled in $3$ ways using $3, 4$ or $5$. Remaining two places can be filled in
  $P_2^^5 = 20$ ways using remaining $5$ digits.

  Thus, no.\ of three digit numbers is $3\times20 = 60$.

  {\bf Case II:} When the number is of four digits.

  Thousands place can be filled in $3$ ways using $1, 2$ or $3$. Remaining three place can be filled
  in $P_3^^5 = 60$ ways using remaining $5$ digits.

  Thus, no.\ of four digit numbers is $3\times60 = 180$.

  Thus, no.\ of required numbers is $60 + 180 = 240$.
\item Since the number has to be divisible by $5$ the units place digit has to be either $0$ or $5$.

  {\bf Case I:} When $0$ is at units place. Rest three places can be filled in $P_3^^4 = 24$ ways using
  remaining $4$ digits.

  Thus, no.\ of four digit numbers in this case is $24$.

  {\bf Case II:} When $5$ is at units place. Thousands place can be filled in $3$ ways using $4, 6$ or
  $7$. Remaining three places can be filled in $P_2^^3 = 6$ ways using remaining $3$ digits.

  Thus, no.\ of four digit numbers in this case is $3\times6 = 18$.

  Hence, total no.\ of required numbers is $18 + 24 = 42$.
\item Since the number has to be even, therefore, units place can be filled by either $2$ or $4$ i.e. in $2$
  ways. Rest four places can be filled in $P_4^^4 = 4! = 24$ ways.

  Thus, total no.\ of $5$ digit numbers is $2\times24 = 48$.
\item Since the no.\ has to be divisible by $5$ units place can be occupied only by $0$ and $5$.

  {\bf Case I:} When the no.\ is of one digit. There are two such numbers $0$ and $5$.

  {\bf Case II:} When the no.\ is of two digits. If $0$ occurs at units place then tens place can be filled
  in $9$ ways giving us $9$ numbers. However, when $5$ occurs at units place then tens place can be filled
  in $8$ ways giving us $8$ numbers. Thus, total no.\ of two digits numbers is $17$.

  {\bf Caae III:} When the no.\ is of three digits. If $0$ occurs at units place then remaining two places
  can be filled in $P_2^^9 = 72$ ways. If $5$ is at units place then hundreds place can be filled in $8$
  ways excluding zero and tens place can be filled in $8$ ways using remaining $8$ digits. Thus, in this
  case otal no.\ of numbers is $72 + 8\times 8 = 136$.

  Thus, total no.\ of numbers is $2 + 17 + 136 = 155$.
\item Hundreds place can be filled in $5$ ways excluding $0$. Rest of two places can be filled in $P_2^^5 =
  20$ ways.

  Thus, total no.\ of numbers is $5\times20 = 100$.

  For odd numbers, units place can be filled in $2$ ways using $5$ or $7$. Hundreds place can be filled in
  $4$ ways excluding $0$ and units place can also be filled in $4$ ways using remaining digits.

  Thus, total no.\ of odd numbers is $2\times4\times4 = 32$.
\item {\bf Case I:} When the no.\ is of one digit. There are three such numbers $0, 2$ and $4$.

  {\bf Case II:} When the no.\ is of two digits. When units place is occupied by $0$, tens place can be filled
  in $4$ ways, making no.\ of such numbers $4$. If units place is occupied by $2$ or $4$ i.e. in two ways then
  tens place can be filled in $3$ ways excluding $0$, making no.\ of such numbers $2\times3 = 6$.

  Thus, no.\ of two digit numbers is $4 + 6 = 10$.

  {\bf Case III:} When the no.\ is of three digits. When units place is occupied by $0$, remaining two
  places can be filled in $P_2^^4 = 12$ ways, making no.\ of such numbers $12$. If units place is occupied by
  $2$ or $4$ i.e. in two ways then hundreds place can be filled in $3$ ways excluding $0$ and tens place can
  be filled in $3$ ways using remaining three digits, making no.\ of such numbers $2\times3\times3 = 18$.

  Thus, no.\ of three digit numbers is $12 + 18 = 30$.

  {\bf Case IV:} When the no.\ if of four digits. When units place is occupied by $0$, remaining three places
  can be filled in $P_3^^4 = 24$ ways, making no.\ of such numbers $24$. Following similarly, when units
  place is occupied by $2$ or $4$, no.\ of such numbers is $2\times3\times3\times2 = 36$.

  Thus, no.\ of four digit numbers is $24 + 36 = 60$.

  {\bf Case V:} When the no.\ is of five digits. In this case, units place must be occupied by $0$ and not by
  $2$ or $4$. Then remaining $4$ places can be filled in $P_4^^4 = 24$ ways.

  Thus, total no.\ of even numbers is $3 + 10 + 30 + 60 + 24 = 127$.
\item Once we fix $5$ at tens place we have $5$ open places and $5$ different digits, which can be arranged
  in $P_5^^5$ ways.

  Thus, no.\ of required numbers is $120$.
\item We have $7$ digits, and have to form four digit numbers. no.\ of such numbers possible is $P_4^^7 =
  840$.

  We have to find numbers greater than $3400$. First we compute numbers between $3400$ and $4000$. The
  thousands place can be filled only by $3$ and hundreds place can be filled by $4, 5, 6$ and $7$ i.e. $4$
  ways. Remaining two positions can be filled in $P_2^^5 = 20$ ways. Thus, no.\ of numbers between $3400$
  and $4000$ is $4\times20 = 80$.

  Now we compute numbers greater than $4000$. Thousands place can be filled by $4, 5, 6$ and $7$ i.e. in
  $4$ ways. Rest three places can be filled in $P_3^^6 = 120$ ways. Thus, no.\ of such numbers is $4\times120
  = 480$.

  Thus, no.\ of numbers greater than $3400$ is $80 + 480 = 560$.
\item Since positions of $3$ and $5$ are fixed rest two positions can be filled with three remaining digits
  in $P_2^^3 = 6$ ways. Thus, no.\ of such numbers is $6$.
\item Thousands place can be filled in $5$ ways excluding $0$. Rremaining three places can be filled in
  $P_3^^5 = 60$ ways using the five remaining digits. Thus, total no.\ of four digit numbers is $5\times60 =
  300$.

  For numbers to be greater than $3000$, thousands place has to be filled by $3, 4$ and $5$ i.e. $3$
  ways. Remaining three places can be filled in $P_3^^5 = 60$. Thus, no.\ of numbers greater than $3000$ is
  $3\times60 = 180$.
\item {\bf Case I:} When the no.\ is of one digit. Total no.\ of numbers possible in this case is $7$
  including $0$.

  {\bf Case II:} When the no.\ is of two digits. Tens place can be filled in $6$ ways excluding $0$ and units
  place can be filled in $6$ ways with remaining digits.

  Thus, no.\ of two digit numbers is $6\times6 = 36$.

  {\bf Case III:} When no.\ is of three digits. Following similarly the no.\ of numbers is $6\times6\times5 =
  180$.

  {\bf Case IV:} When the no.\ is of four digits. Following similarly the no.\ of numbers is
  $6\times6\times5\times4 = 720$.

  {\bf Case V:} When the no.\ is of five digits. Following similarly the no.\ of numbers is
  $6\times6\times5\times4\times3 = 2160$.

  {\bf Case VI:} When the no.\ is of six digits. Following similarly the no.\ of numbers is
  $6\times6\times5\times4\times3\times2 = 4320$.

  {\bf Case VII:} When the no.\ is of seven digits. Following similarly the no.\ of numbers is
  $6\times6\times5\times4\times3\times2\times1 = 4320$.

  Thus, total no.\ of numbers is $7 + 36 + 180 + 720 + 2160 + 4320 + 4320 = 11743$
\item We have $5$ digits so when all of them are taken at a time then no.\ of possible numbers is $P_5^^5 = 120$.

  Each digit will occupy each place for $24$ numbers. Thus, sum of all numbers at any place is $24(1 + 3 + 5
  + 7 + 9) = 600$. Therefore, sum of all such numbers is $600(1 + 10 + 100 + 1000 + 10000) = 6,666,600$.
\item We have $4$ digits with $3$ occurring twice. Thus, total no.\ of numbers is $\frac{P_4^^4}{2!} =
  12$. Now each of the digits will occur at each place $\frac{12}{4} = 3$ times.

  Thus, sum of digits at each place is $3(3 + 2 + 3 + 4) = 36$. Thus, sum of all possible numbers is $36(1 +
  10 + 100 + 1000) = 39,996$.
\item Let us fix $2$ at units place. Then, ten thousands place can be filled in $3$ ways using $4, 6, 8$ and
  remaining two places can be filled in $P_3^^3 = 3!$ ways. Thus, total no.\ of numbers is $3\times6 = 18$.

  Number of numbers when $2$ is at ten throusands place is $P_4^^4 = 24$. Thus, each positive digit will
  occur at units, tens, hundreds and thousands place $18$ times and at thousands place $24$ times.

  Sum of the digits at units, tens, hundreds and thousands place will be each $18(2 + 4 + 6 + 8) = 360$ and
  sum of digits at ten thousands place is $24(2 + 4 + 6 + 8) = 480$.

  Thus, sum of all numbers will be $360(1 + 10 + 100 + 1000) + 480\times10000 = 5,199,960$.
\item Total no.\ of five digit numbers possible is $P_5^^5 = 120$ where each digit will appear at each
  position $\frac{120}{5} = 24$ times.

  Thus, sum of digits at each place is $24(3 + 4 + 5 + 6 + 7) = 600$. Therefore, sum of all such numbers is
  $600(1 + 10 + 100 + 1000 + 10000) = 6,666,600$.
\item Let us fix $2$ at units place. Then, thousands place can be filled in $2$ ways using $3$ or $5$ and
  remaining two places can be filled in $P_2^^2 = 2$ ways. Thus, total no.\ of numbers is $2\times 2 = 4$.

  Number of numbers when $2$ is at thousands place is $P_3^^3 = 6$. Thus, each positive digit will
  occur at units, tens, hundreds and thousands place $4$ times and at thousands place $6$ times.

  Sum of digits at units, tens and hundreds place will eb each $4(2 + 3 + 5) = 40$ and sum of digits at
  thousands place will be $6(2 + 3 + 5) = 60$.

  Thus, sum of all numbers will be $40(1 + 10 + 100) + 60\times1000 = 64,440$.
\item Each letter can be put in any one of the four letter boxes. Thus, $5$ letters can be posted in $4^5$
  ways.
\item Each prize can be given in $5$ ways. So three prizes can be given in $5^3$ ways.
\item Each thing can be given in $p$ ways to $p$ person. Thus, $n$ things can be given in $p^n$ ways.
\item Each monkey can have a master in $m$ ways. Thus, $n$ monkeys can have a master in $m^n$ ways.
\item First prize in mathematics and physics can be given in $10$ ways and second prize in $9$ ways. In
  chemistry, first prize can be given in $10$ ways.

  Thus, total no.\ of ways is $10\times9\times10\times9\times10 = 81,000$.
\item The first animal can be picked in $3$ ways with the possibility of it being a cow, a calf or a
  horse. Similarly, second animal can be picked in $3$ ways. Proceeding this way all $12$ animals for the
  stall can be picked in $3$ ways.

  Thus, total no.\ of making the shipload is $3^{12}$.
\item Each delegate can be put in a hotel in $6$ ways. Therefore, $5$ delegates can be put in $6^5$ ways.
\item Ten thousands place can be filled in $4$ ways exluding $0$. Rest $4$ places can be filled in $5$ ways
  each. Thus, total no.\ of $5$ digits numbers is $4\times5^4 = 2,500$.
\item Each ring can be put in a finger in $4$ ways i.e. by putting it in any finger. Thus, $6$ rings can be
  put in $4$ fingers in $4^6$ ways.
\item Thousands place can be filled in $3$ ways using $3, 4$ or $5$. Remaining places can be filled in $6^3$
  ways using any of the digits. But one of these numbers will be $3000$ itself.

  Thus, no.\ of four digit numbers which can be made is $3\times6^3 - 1$.
\item When the number plate is of three digits, each place can be filled in $9$ ways excluding zero. This
  gives us $9^3$ number plates. Similalry, when the number plate is of four digits the no.\ of possible
  number plates is $9^4$.

  Thus, total no.\ of number plates is $9^3 + 9^4 = 10\times9^3 = 7,290$.
\item Each question can be answered in $4$ ways, therefore, $10$ questions can be answered in $4^{10}$ ways.

  Second part: First question can be answered in $4$ ways. Now this choice won't be available for the second
  answer so there are $3$ ways. Similarly, for third and so on. Thus, total no.\ of ways is $4\times3^9$.
\item Treating all volumes of a book as one book we have four books which can be arranged in $4!$
  ways. However, books having $3$ volumes can be arranged in $3!$ ways among themselves and similarly books
  having $2$ volumes can be arranged in $2!$ ways among themselves.

  Thus, total no.\ of arranging given books is $4!3!3!2!2!$.
\item There are $14$ books having different no.\ of copies. Treating all copies as one book we still have
  $14$ books which can be arranged in $14!$ ways.

  Since copies are identical there is only one way to arrange them among themselves. Thus, total no.\ of
  arranging the given books is $14!$.
\item Treating people of different nationalities as one person we have three persons, which can be arranged
  in $3!$ ways. Now $10$ Indians can be arranged in $10!$ ways among themselves, $5$ Americans can be
  arranged in $5!$ ways among themselves and $5$ Britished can be arranged in $5!$ ways as well.

  Thus, total no.\ of ways of seating them is $3!10!5!5!$.
\item The pattern would be $GBGBGBGBGBGBG$ where {\it B} shows boys position and {\it G} indicates possible
  positions of girls. Boys can be arranged in $6!$ ways. For girls, there are $7$ open positions and $4$
  girls can be seated in $P_4^^7 = \frac{7!}{3!}$ ways.

  Thus, total no.\ of ways of seating them is $6!.\frac{7!}{31}$.
\item $n$ books can be arranged in $n!$ ways. Now we will find the no.\ of arrangements when two given books
  which do not have to be together are together. Treating the two books as one book we have $n - 1$ books
  which can be arraned in $(n - 1)!$ ways. But the two books can be arranged in $2$ ways among themselves,
  making the total no.\ of arrangements is $2.(n - 1)!$.

  Thus, no.\ of arrangements when the two books are not together is $n! - 2.(n - 1)! = (n - 2).(n - 1)!$.
\item From previous problem, we find the answer to be $4.5! = 480$.
\item Following like previous problem, we find theh answer to be $480$.
\item Following like previous problem on boys and girls we first seat the $15$ I.Sc.\ students in $15!$ ways
  which gives us $16$ open positions for $B.Sc.$\ students, which can be seated in $P_{12}^^{16}$.

  Thus, total no.\ of ways of seating the students is $15!.P_{12}^^{16}$.
\item First we arrange black balls which will give us $20$ positions in between them and on the edges for
  white balls. Since the balls are identical we can choose $18$ positions out of $20$ for white balls in
  $C_{18}^^{20} = 190$ ways.
\item First we place $p$ positive signs which will give us $p + 1$ positions for negative signs between them
  and on the edges. Since signs are identical we can choose $n$ positions out of $p + 1$ in $C_n^^{p + 1}$
  ways.
\item $m$ men can be seated in $m!$ ways which will have $m + 1$ positions between them and on the edges for
  women so that no two women sit together. Now $n$ women can be arranged in these $m + 1$ positions in
  $P_n^^{m + 1} = \frac{(m + 1)!}{(m - n + 1)!}$ ways.

  Thus, total no.\ of ways to seat them is $\frac{m!(m + 1)!}{(m - n + 1)!}$.
\item Following like previous problem, we have $m = 5, n = 3$, so the answer woulld be $\frac{5!6!}{3!}$.
\item We have $12$ alphabets excluding c{\symbol[rightquote]}s out of which $5$ are a{\symbol[rightquote]}s,
  $3$ are b{\symbol[rightquote]}s, $1$ d, $2$ e{\symbol[rightquote]}s and $1$ f,
  so these can be arranged in $\frac{12!}{5!3!2!}$ ways. Now these $12$ alphabets will create $13$ positions
  between them and on the edges which are to be filled by $3$ c's in $P_3^^{13}$ ways.

  Thus, total no.\ of arrangements is $\frac{12!}{5!3!2!}\times\frac{13!}{10!}$.
\item The word banana has {\symbol[leftquote]}a{\symbol[rightquote]} repeating $3$ times and
  {\symbol[leftquote]}n{\symbol[rightquote]} repeating twice while total no.\ of alphabets is $6$.

  Hence, to no.\ of different permutations is $\frac{6!}{3!2!}$.
\item There are $13$ alphabets in the word \quotation{circumference}. \quote{c} comes thrice, \quote{r}
  comes twice, \quote{e} comes thrice and rest come once.

  Thus, total no.\ of words that can be made is $\frac{13!}{3!3!2!}$.
\item Three copies of four books means $12$ books with repetition of copies. Thus, total no.\ of arragements
  on the shelf is $\frac{12!}{3!3!3!3!}$.
\item There are $12$ alphabets in the word \quotation{Independence}. \quote{n} comes thrice, \quote{d} comes
  twice, \quote{e} comes four times, and rest come once.

  Thus, total no.\ of words that can be made is $\frac{12!}{4!3!2!}$.
\item There are $8$ alphabets in the word \quotation{Principal}, of which, \quote{p} comes twice, \quote{i}
  comes twice and rest occur once. Treating all vowels as one alphabet we have $6$ alphabets which can be
  arranged in $\frac{6!}{2!}$ ways.

  However, the vowels themselves can be arranged among themselves in $\frac{3!}{2!}$ ways. Thus, total
  no.\ of words is $\frac{6!3!}{2!2!}$.
\item There are $11$ alphabets in the word \quotation{Mathematics}, of  which, \quote{m} comes twice,
  \quote{a} comes twice, \quote{t} comes twice and rest comes once. Thus, no.\ of words that can be formed is
  $\frac{11!}{2!2!2!}$.

  Treating all vowels as one alphabet and all consonants as another we have two alphabets which can be
  arranged in $2!$ ways. But $4$ vowels can be arranged in $\frac{4!}{2!}$ ways and $7$ consonants can be
  arranged in $\frac{7!}{2!2!}$ ways.

  Thus, total no.\ of such words is $\frac{2!7!4!}{2!2!2!}$.
\item There are $8$ alphabets in the word \quotation{Director}, of which, $r$ comes twice and rest come
  once. Since the vowels have to come together, therefore we treat them as one alphabet making a total of
  $6$ alphabets which can be arranged in $\frac{6!}{2!}$ ways.

  However, the three vowels can be arranged in $3!$ ways among themselves making no.\ of such words
  $\frac{6!3!}{2!}$.
\item There are $8$ alphabets in the word \quotation{Plantain}, of which, \quote{a} and \quote{n} come
  twice and rest come once. Since the vowels have to come together, therefore we treat them as one alphabet
  making a total of $6$ alphabets which can be arranged in $\frac{6!}{2!}$ ways.

  However, the three vowels can be arraned in $\frac{3!}{2!}$ ways among themselves making no.\ of such words
  $\frac{6!3!}{2!2!}$.
\item There are $12$ letters in the word \quotation{Intermediate}, of which, \quote{e} comes thrice,
  \quote{i} and \quote{t} comes twice and rest come once.

  We first arrange vowels which can be done in $\frac{6!}{3!2!}$. Now because relative order does not change
  we have six positions for consonants giving us total no.\ of ways of arranging them as $\frac{6!}{2!}$.

  Thus, total no.\ of such words is $\frac{6!6!}{3!2!2!}$.
\item There are $8$ letters in the word \quotation{Parallel}, of which, \quote{a} comes twice, \quote{l}
  comes thrice and rest comes once.

  Total no.\ of arrangements is $\frac{8!}{3!2!}$. Treating all the ls as one letter we have $6$ letters
  which can be arranged in $\frac{6!}{2!}$ ways in which all ls will be together.

  Therefore, no.\ of words in which all ls are not together is $\frac{8!}{3!2!} - \frac{6!}{2!} = 3000$.
\item The parts are solved below:
  \startitemize[i]
  \item Fixing \quote{D} at the first position; rest four positions can be filled in $P_4^^4$ ways. Thus,
    no.\ of such words is $4! = 24$.
  \item Fixing \quote{I} at the end; rest four positions can be filled in $P_4^^4$ ways. Thus,
    no.\ of such words is $4! = 24$.
  \item Fixing \quote{l} in the middle; rest four positions can be filled in $P_4^^4$ ways. Thus,
    no.\ of such words is $4! = 24$.
  \item Fixing \quote{D} and \quote{I}; rest three positions can be fillled in $P_3^^3$ ways. Thus, no.\ of
    such words is $3! = 6$.
  \stopitemize
\item There are $7$ unique letter in the word \quotation{Violent} with $3$ vowels. There are $4$ odd places
  so three vowels can be arranged in $P_3^^4 = 4!$ ways. Rest $4$ consonants can be arrannged in $4! = 24$
  ways. Thus, total no.\ of such words is $24\times 24 = 576$.
\item There are $3$ distinct consonants and $3$ vowels, where \quote{o} repeats once in the word
  \quotation{Saloon}. Since consonants and vowels have to occupy alternate place we will have two
  patterns. $VCVCVC$ and $CVCVCV$, where $C$ represents consonants and $V$ represents vowels.

  Three consonants can be arranged in $3!$ arrangements and $3$ vowels can be arranged in $\frac{3!}{2!}$
  arrangement. Thus, total no.\ of arrangements is $3!3! = 36$.
\item There are $4$ consonants and $3$ vowels in the word \quotation{Article}. Clearly, there are three even
  places which are to be occupied by vowels in $3!$ arrangements and consonants can be arranged in $4!$
  arrangements for remaining $4$ positions.

  Thus, total no.\ of words is $4!3! = 144$.
\item Since the number has to be greater than $4$ million and we are given $7$ digits the ten millions place
  can be occupied by either $4$ or $5$ in $2$ ways.

  Remaning digits can be arranged in $\frac{6!}{2!2!} = 180$ arrangements as $2$ and $3$ repeat once. Thus,
  total no.\ of required numbers is $2\times180 = 360$.
\item In the given digits $2$ comes thrice and $3$ comes twicec so the no.\ of numbers is $\frac{7!}{3!2!} =
  420$.

  For odd numbers units place is to be occupied by $1, 3$ or $5$. When $1$ or $5$ occupy units place
  remaining positions can be filled in $\frac{6!}{3!2!} = 60$ ways making the number $2\times60 = 120$.

  When one of the $3$'s occupy units place rest of the positions can eb filled in $\frac{6!}{3!} = 120$
  ways. Thus, total no.\ of odd numbers is $120 + 120 = 240$.
\item There are four odd digits with both $1$ and $3$ repeating. The even no.\ $2$ repeats once. In a $7$
  digits number there are four odd places which can be filled by odd numbers in $\frac{4!}{2!2!} = 6$ ways.

  Even places can be filled by $2$ and $4$ can be filled in $\frac{3!}{2!} = 3$ ways. Thus, no.\ of required
  numbers is $6\times3 = 18$.
\item {\bf Case I:} When the no.\ if is five digits.

  When ten thousands place is occupied by $2, 3$ or $4$ remaining four places can be filled in
  $\frac{P_4^^5}{2!} = 60$ ways, making such numbers $60\times3 = 180$ in number.

  When ten thousands place is occupied by $1$ remaining four places can be filled in $P_4^^5 = 120$ ways.

  Thus, total no.\ of five digit numbers is $180 + 120 = 300$.

  {\bf Case II:} When the no.\ is of six digits.

  When hundred thousands place is occupied by $2, 3$ or $4$ remaining five places can be filled in
  $\frac{P_5^^5}{2!} = 60$ ways, making such numbers $60\times3 = 180$ in number.

  When hundred thousands place is occupied by $1$ remaining four places can be filled in $P_5^^5 = 120$ ways.

  Thus, total no.\ of six digit numbers is $180 + 120 = 300$.

  Thus, total no.\ of numbers is $300 + 300 = 600$.
\item When the digits are repeated thousands place can be filled in $5$ ways excluding $0$. Remaining $3$
  positions can be filled by $6$ digits in $6^3$ ways.

  Thus, no.\ of such numbers is $5\times6^3 = 1080$.

  To find the no.\ of numbers where at least one digit is repeated we find the no.\ of numbers where no digit
  is repeated and subtract it from previously obtained result.

  For no repetition, thousands placec can be filled in $5$ ways exluding $0$. Remaning $3$ places can be
  filled by $5$ digits in $P_3^^5 = 60$ ways.

  Thus, no.\ of numbers without repetition is $60\times 5 = 300$.

  Thus, no.\ of numbers where at least one digit is repeated is $1080 - 300 = 780$.
\item There are a total of $9$ flags, of which, $2$ are red, $2$ are blue and $5$ are yellow. Thus, total
  no.\ of signals that can be made by using all of them at the same time is $\frac{9!}{2!2!5!}$.
\item When all are of same color $P_1^^6$ signals can be made. When all are of two colors $P_2^^6$ signals
  can be made and so on.

  Thus, total no.\ of signals is $P_1^^6 + P_2^^6 + P_3^^6 + P_4^^6 + P_5^^6 + P_6^^6 = 1956$.
\item {\bf Case I:} When \quote{e} is in first place. Remaining four places can be filled in $4!$ ways.

  {\bf Case II:} When \quote{e} is in second place. First place can be filled in $3$ ways and remaining $3$
  places in $3!$ ways.

  {\bf Case III:} When \quote{e} is in third place. First two places in $3\times2$ ways and remaining two
  places in $2!$ ways.

  {\bf Case IV:} When \quote{e} is in fourth place. First three places in $3!$ ways and last place with
  \quote{i}.

  Thus, total no.\ of words is $4! + 3\times3! + 6\times2! + 3! = 60$.

  {\bf Second method:} Total no.\ of words is $5!$. In half of these \quote{e} will come before \quote{i} and
  in half of them after it. Thus, no.\ of words is $\frac{5!}{2} = 60$.
\item no.\ of ways in which $5$ men can sit around a round table is $(5 - 1)! = 24$ arrangements.
\item When there is no restriction we have $10$ girls and boys. Thus, total no.\ of arrangements would be
  $9!$.

  When no girls are to sit together we first seat the boys in $4!$ arrangements giving us five open
  positions. These can be filled by $5$ girls in $5!$ ways.

  Thus, total no.\ of seating arrangements is $4!5!$.
\item Treating all girls as a single girl we have $7$ boys and girls which can be seated in $6!$ ways. But
  the $4$ girls can be arranged in $4!$ ways among themselves.

  Thus, total no.\ of seating arrangements is $6!4!$.
\item The line can start with boys so we first seat the boys put the boys in $5!$ ways followed by girls in
  between boys in $5!$ ways. This can be repeated starting with girls in same manner.

  Thus, no.\ of lines that can be formed is $2.5!5!$.

  For a round table we have already solved previously giving us $4!5!$ no.\ of arrangements.
\item $6$ boys can be seated first in $5!$ ways giving us $6$ open places in which girls can be seated in
  $P_5^^6$ ways. Thus, total no.\ of seating arrangements is $5!6!$.
\item Since in a necklace clockwise and anticlockwise does not matter, therefore, total no.\ of necklaces
  that can be made using $50$ pearls is $\frac{49!}{2!}$.
\item Treating the two particular delegates as one delegate we have $19$ delegates which can be seated in
  $18!$ ways. But the two delegates can be seated in $2!$ ways among themselves.

  Thus, total no.\ of seating arrangements is $18!2!$.
\item The question effectively asks for alternate seating arrangements among gentlemen and ladies. Thus,
  followin from problem solved previously total no.\ of seating arrangements would be $4!3!$.
\item $7$ Englishmen can be seated in $6!$ ways giving us $7$ open places which can be filled by $6$ Indians
  in $P_6^^7$ ways.

  Thus, total no.\ of seating arrangements is $6!7!$.
\item We know that if $C_x^^n = C_y^^n$ then either $x = y$ or $x + y = n$. Given, $C_{3r}^^{15} = C_{r +
  3}^^{15}$ therefore either $3r = r + 3$ or $3r + r + 3 = 15$.

  However, $3r = r = 3 \Rightarrow r = \frac{3}{2}$, which is not possible, therefore, $3r + r + 3 = 15
  \Rightarrow r = 3$ must be the case.
\item Given, $C_6^^n:C_3^^{n - 3} \Rightarrow \frac{n!}{6!(n - 6!)}.\frac{3!(n - 6)!}{(n - 3)!} =
  \frac{33}{4}$

  $\Rightarrow \frac{n!}{(n - 3)!}.\frac{3!}{6!} = \frac{n(n - 1)(n - 2)}{6.5.4} = \frac{33}{4} \Rightarrow
  n(n - 1)(n - 3) = 11.10.9 \Rightarrow n = 11$.
\item Given, $\displaystyle C_4^^{47} + \sum_{j=1}^5C_3^^{52-j}$

  $= C_4^^{47} + (C_3^^{51} + C_3^^{50} + C_3^^{49} + C_3^^{48} + C_3^^{47}) = (C_4^^{47} + C_3^^{47}) +
  (C_3^^{51} + C_3^^{50} + C_3^^{49} + C_3^^{48})$

  $= C_4^^48 + (C_3^^{51} + C_3^^{50} + C_3^^{49} + C_3^^{48} + C_3^^{47}) [\because C_r^^n + C_{r + 1}^^n =
  C_{r + 1}^^{n + 1}]$

  Repeating this we have the expression equal to $C_4^^{52}$.
\item Let $p$ be the product of $r$ consecutive integers starting from $n$. Then, $p = n(n + 1)(n + 2)\cdots
  (n + r - 1)$

  $\Rightarrow \frac{p}{r!} = \frac{n(n + 1)(n + 2)\cdots(n + r - 1)}{r!} = \frac{1.2.3.\ldots.(n - 1)n(n +
  1)(n + 2)\cdots(n + r - 1)}{1.2.3.\ldots.(n - 1).r!}$

  $= \frac{(n + r - 1)!}{(n - 1)!r!} = C_r^^{n + r - 1}$, which would be an integer, and hence, $p$ is
  divisible by $r!$.
\item A triangle is formed with three vertices so the problem is essentially about choosing $3$ out of $m$
  i.e. $C_3^^m = \frac{m(m - 1)(m - 2)}{6}$.
\item Number of children is $8$. no.\ of children to be taken at a time is $3$. Out of $8$ children $3$ can
  be selected in $C_3^^8$ ways. Hence, the man has to go to zoo $C_3^^8 = 56$ times.

  Number of selection of $3$ children out of $8$ children including a particular child is $1\times C_2^^7 =
  21$. Hence, a particular child will go $21$ times to the zoo.
\item Let there be $n$ students. no.\ of ways in which $2$ students can be selected out of $n$ is $C_2^^n$
  i.e. we have $C_2^^n$ pairs.

  But, for each pair of students no.\ of cards sent is $2$. Thus, total no.\ of cards sent is $2.C_2^^n = n(n
  - 1) = 600 \Rightarrow n = 25$ because $n\neq -24$.

  {\bf Second method:} Each student sends cards to $n - 1$ students. Thus, total no.\ of cards sent is $n(n -
  1) = 600 \Rightarrow n = 25$.
\item A polygon of $m$ sides will have $m$ vertices. When any two vertices of the polygon are joined, either
  a diagonal or a side is formed.

  Total no.\ of selections of $2$ points taken at a time from $m$ points is $C_2^^m$.
\item Total no.\ of persons is $6 + 4 = 10$. Total no.\ of selections of $5$ persons out of $10$ is
  $C_5^^{10}$. Number of selections when no lady is taken is $C_5^^6$.

  Thus, no.\ of selections when at least one lady is present is $C_5^^10 - C_5^^6 = 252 - 6 = 246$.
\item (a) Total no.\ of selections of $3$ points out of $10$ points is $C_3^^{10} = 120$. Number of selections
  of $3$ points out of $4$ collinear points is $C_3^^4 = 4$.

  Thus, no.\ of triangles formed is $120 - 4 = 116$.

  (b) Total no.\ of selections of $2$ points out of $10$ points is $C_2^^{10} = 45$. no.\ of selection of
  points when only one line is formed is $C_2^^4 = 6$

  Therefore, no.\ of straight lines formed is $45 - C_2^^4 + 1 = 40$. (We take $1$ line formed from four
  collinear points)

  (c) Total no.\ of selections of $4$ points out of $10$ points is $C_4^^{10} = 210$. no.\ of selection of
  points when no quadrilateral is formed is $C_3^^4.C_1^^6 + C_4^^4.C_0^^6 = 25$.

  Thus, no.\ of quadrilaterals formed is $210 - 25 = 185$.
\item Zero or more oranges can be selected from $4$ oranges in $5$ ways because oranges are
  identical. Similalry, the no.\ of selection for apples would be $6$ and for mangoes it would be $7$.

  Thus, no.\ of selections when all three types of fruits are selected from is $5\times6\times7 = 210$. But
  one of these selections will contain $0$ fruits.

  Thus, required no.\ of selections is $209$.
\item no.\ of selections by which $1$ or more green dye can be chosen is $C_1^^5 + C_2^^5 + C_3^^5 + C_4^^5 +
  C_5^^5 = 2^5 - 1$. no.\ of selections by which $1$ or more blue dye can be chosen is $C_1^^4 + C_2^^4 +
  C_3^^4 + C_4^^4 = 2^4 - 1$. no.\ of selections by which $0$ or more red dye can be chosen is $C_0^^3 +
  C_1^^3 + C_2^^3 + C_3^^3 = 2^3 = 8$.

  Thus, required no.\ of selections is $21\times15\times8 = 3720$.
\item Factos of $216,\,000$ are $5\ 2$s, $3\ 3$s and $2\ 5$s. Zero or more $2$s can be selected in $5 + 1 =
  6$ ways. Zero or more $3$s can be selected in $3 + 1 = 4$ ways. Zero of more $5$s can be selected in $2 +
  1 = 3$ ways.

  Thus, no.\ of divisors is $6\times4\times3 - 1 = 71$ because one of these would contain no factor. Adding
  $1$ to the no.\ of divisors we have total no.\ of divisors as $72$.
\item A student can fail in one , two, three, four or all of five subjects. Thus, no.\ of ways of failing is
  $C_1^^5 + C_2^^5 + C_3^^5 + C_4^^5 + C_5^^5 = 2^5 - 1 = 31$.
\item Each person can be given $4$ things. no.\ of ways of giving $4$ things out of $12$ to the first person
  is $C_4^^{12}$. Then, $8$ things remain. no.\ of ways of giving $4$ things out of $8$ to the second person
  is $C_4^^8$. Now third person can receive $4$ things out of $4$ in $C_4^^4$ ways.

  Thus, required no.\ of ways is $C_4^^{12}\times C_4^^8\times C_4^^4 = \frac{12!}{(4!)^3}$.

  no.\ of ways in which $12$ things can be divided equally among $3$ sets is $\frac{12!}{(4!)^3.3}$.
\item There are $11$ letters in the word \quotation{Examination} in which three occur in pairs
  i.e. \quote{A}, \quote{N} and \quote{I}. The different letters are $E, X, A, M, I, N, T, O$ i.e. $8$.

  {\bf Case I:} When two pairs of identical letters are chosen.

  The two pairs can be chosen from three in $C_2^^3 = 3$ ways. These letters can be arranged among themselves in
  $\frac{4!}{2!2!} = 6$ ways. Thus, total no.\ of words formed is $3\times6 = 18$.

  {\bf Case II:} When one pair of identical letters is chosen and remaining two letters are different.

  The pair of identical letters can be chosen in $C_1^^3 = 3$ ways. The two different letters can be chosen
  in $C_2^^7 = 21$ ways. These letters can be arranged in $\frac{4!}{2!}$ ways.

  Thus, total no.\ of words formed is $3\times C_2^^7\times\frac{4!}{2!} = 756$.

  {\bf Case III:} When all four letters are different.

  no.\ of words that can be formed is $P_4^^8 = 1680$.

  Thus, total no.\ of words formed is $756 + 18 + 1680 = 2454$.
\item We need to select $4$ vertices out of $n$ of a polygon to form a quadrilateral. no.\ of selections of
  $4$ points is $C_4^^n$.
\item no.\ of ways of selecting $3$ friends out of $7$ is $C_3^^7 = 35$. Thus, no.\ of parties that can be
  given is $35$.

  Suppose a particular friends is mandatory in a party then $2$ other friends can be selected in $C_2^^6$
  ways. Thus, no.\ of parties a particular friend will attend is $C_2^^6 = 15$.
\item If $p$ things always occue then we have to select remaning $r - p$ things out of $n - p$ ways, which
  is $C_{r - p}^^{n - p}$.
\item (a) If a particular member is always added then we have to choose $5$ more from remaining $11$, which
  is $C_5^^{11}$.

  (b) If a particular member is always excluded then we have to chhose $6$ more from remaining $11$, which
  is $C_6^^{11}$.
\item (a) Total no.\ of ways of seating $6$ students is $P_6^^6 = 720$. Now we will put $C$ and $D$ together
  and subtract that from total no.\ of ways to find no.\ of ways of seating them when $C$ and $D$ are not
  together.

  Treating $C$ and $D$ as one student we have $5$ students which can be seated in $P_5^^5 = 120$ ways. But
  these two can be arranged among themseleves in $2!$ ways making total no.\ of ways $120\times2 = 240$.

  Thus, no.\ of ways of seating these $6$ students together when $C$ and $D$ are not together is $720 - 240 =
  480$.

  (b) If $C$ is always included then we need to select $3$ more from remaining $5$, which can be done in
  $C_3^^5 = 10$ ways.

  (c) Since $E$ is always excluded we have only $5$ students left. Thus, following previous part it can be
  done in $C_3^^4 = 4$ ways.
\item Let there be $n$ stations. To print a ticket we need a source station and a desination station. So
  different tickets which can be printed with $n$ stations is $C_2^^n$, which is $105$ in our case.

  $\therefore \frac{n!}{(n - 2)!2!} = 105 \Rightarrow n(n - 1) = 210 = 14.15 \Rightarrow n = 15$.
\item no.\ of ways to select $2$ points to form a straight line out of $15$ points is $C_2^^{15} = 105$.
  This will include $2$ points out of $6$ collinear points which will actually contain only $1$ straight
  line out of it. So no.\ of ways to choose $2$ points out of these $6$ points is $C_2^^6 = 15$. Thus, total
  no.\ of straight lines formed is $105 - 10 + 1 = 91$.

  no.\ of ways of choosing $3$ points out of $15$ is $C_3^^{15} = 455$. We have to not consider cases when
  all three points aree selected from collinear points as those won't form a triangle. no.\ of selections of
  $3$ points out of collinear points is $C_3^^6 = 20$.

  Thus, total no.\ of triangles formed is $455 - 20 = 435$.
\item no.\ of ways of choosing $4$ points out of $10$ is $C_4^^{10} = 210$. When $3$ or $4$ points are chosen
  from $5$ collinear pooints the quadrilateral won't be formed. When we choose $3$ points from collinear
  points we have $C_3^^5 = 10$ ways, and $1$ remaining point from $5$ non-collinear points in $5$ ways. Thus,
  total no.\ of such selections is $10\times5 = 50$.

  When all four points are chosen from collinear points; this can be done in $C_4^^5 = 5$ ways.

  Thus, total no.\ of quadrilaterals formed is $210 - 50 - 5 = 155$.
\item There is a total of $12$ points and we can choose $3$ points from these in $C_3^^{12} = 220$
  ways. However, these points must not come from points of same side.

  Thus, no.\ of triangles formed is $220 - C_3^^3 - C_3^^4 - C_3^^5 = 205$.
\item We need one goalkeeper in the team and two are available so goalkeeper can be chosen in $2$ ways. Rest
  of $10$ players can be chosen from remaining $12$ players in $C_{10}^^{12} = 66$ ways.

  Thus, no.\ of ways in which a team of $11$ out of $14$ can be formed is $2\times66 = 122$.
\item $2$ men can be chosen from $5$ men in $C_2^^5 = 10$ ways. Similarly, $2$ women from $6$ women can be
  chosen in $C_2^^6 = 15$ ways.

  Thus, total no.\ of ways of forming the committee is $10\times 15 = 150$.
\item Since each boy is to receive one article at least one boy will receive $2$ articles. These two
  articles can be given to one of the boys in $C_2^^8$ ways. The second article can be given in $C_1^^7$
  ways and so on.

  Since first article can be given to any of the seven boys the above result if multiplied by $7$ will give
  us total no.\ of ways of distributing the articles.

  Thus, total no.\ of ways is $7(C_2^^8 + C_1^^7 + C_1^^6 + C_1^^5 + C_1^^4 + C_1^^3 + C_1^^2 + C_1^^1)$.
\item {\bf Case I:} When there are $3$ ladies in the committee.

  no.\ of ways of choosing $3$ ways out of $4$ ladies is $C_3^^4$. Remaining $2$ members can be selected out of
  $7$ men is $C_2^^7$ ways. Thus, no.\ of such committees is $C_3^^4\times C_2^^7$.

  {\bf Case II:} When there are $4$ ladies in the committee.

  no.\ of ways of choosing $4$ ways out of $4$ ladies is $C_4^^4$. Remaining $1$ member can be selected out of
  $7$ men is $C_1^^7$ ways. Thus, no.\ of such committees is $C_4^^4\times C_1^^7$.

  Thus, total no.\ of committees is $84 + 7 = 91$.
\item There are three cases. Two questions from first group and four questions from second group, three
  questions from each group, and four questions from first group and two questions from second group.

  This can be done in $C_2^^5\times C_4^^5 + C_3^^5\times C_3^^5 + C_4^^5\times C_2^^5 = 50 + 100 + 50 =
  200$.
\item $3$ students can be chosen from $20$ students in $C_3^^{20}$ ways.

  (a) When a particular professor is included the second professor for the committee out of remaining $9$
  professors can be included in $C_1^^9$ ways.

  Thus, total no.\ of such committees is $C_3^^{20}\times C_1^^9$.

  (b) When a particular profession is always excluded then two professors can be chosen from remaining $9$
  in $C_2^^9$ ways.

  Thus, total no.\ of such committees is $C_3^^{20}\times C_2^^9$.

  Thus, total no.\ of committees is $C_3^^{20}\times C_1^^9 + C_3^^{20}\times C_2^^9$.
\item The committee can comprise of $1, 2, 3, 4$ or $5$ girls, which can be selected out of $7$ girls in
  $C_1^^7, C_2^^7, C_3^^7, C_4^^7$ or $C_5^^7$ ways respectively.

  Remaining $4, 3, 2, 1$ boys can be selected out of $6$ boys in $C_4^^6, C_3^^4, C_2^^4, C_1^^4$ ways
  respectively.

  Thus, no.\ of ways in which committee can be formed is $C_1^^7\times C_4^^6 + C_2^^7\times C_3^^6 +
  C_3^^7\times C_2^^6 + C_4^^7\times C_1^^6 + C_5^^7\times C_0^^6$.
\item (a) When there are no restrictions the committees can be formed by choosing $5$ out of $6 + 4 = 10$
  persons, which is $C_5^^{10} = 252$.

  (b) When no lady is selected no.\ of ways to form committess is $C_5^^6 = 6$. Thus, no.\ of committees when
  at least one lady is selected is $252 - 6 = 246$.
\item Total no.\ of committees would be $C_5^^{12}$. no.\ of committees comprising only of men would be
  $C_5^^8$.

  Thus, no.\ of committees including at least one lady would be $C_5^^{12} - C_5^^{8} = 736$.
\item Out of $6$ hockey players $4, 5, 6$ hockey players can be selected in $C_4^^6, C_5^^6, C_6^^6$ ways
  respectively. Remaining $8, 7, 6$ players can be chosen from remaining $9$ players in $C_8^^9, C_7^^9,
  C_6^^9$ ways respectively.

  Thus, no.\ of ways in which players can be selected is $C_4^^6\times C_8^^9 + C_5^^6\times C_7^^9 +
  C_6^^6\times C_6^^9 = 15\times9 + 6\times36 + 1\times84 = 435$.
\item Total no.\ of selections of $5$ out of $7 + 4 = 11$ persons is $C_5^^{11}$. When no ladies are
  selected, no.\ of ways of forming the boat party is $C_5^^7$.

  Thus, no.\ of ways of forming boat party when at least one lady is selected is $C_5^^{11} - C_5^^7 = 771$.
\item Since girls are not to be outnumbered we have to have $3, 4, 5$ or $6$ girls out of $6$ in the
  committee, which can be done in $C_3^^6, C_4^^6, C_5^^6$ or $C_6^^6$ ways respectively.

  Remaining $3, 2, 1$ positions can be filled from $4$ boys in $C_3^^4, C_2^^4, C_1^^4$ ways respectively.

  Thus, total no.\ of ways in which committee can be formed is $C_3^^6\times C_3^^4 + C_4^^6\times C_2^^4 +
  C_5^^6\times C_1^^4 + C_6^^6 = 20\times4 + 15\times6 + 6\times4 + 1 = 195$.
\item no.\ of relatives which can be invited is $5, 6, 7$ out of $8$ relatives in $C_5^^8, C_6^^8, C_7^^8$
  ways. Remaining $2, 1$ friends can be chosen from remaining $4$ friends which are no relatives in $C_2^^4,
  C_1^^4$ ways.

  Thus, no.\ of ways in which invitations can be made is $C_5^^8\times C_2^^4 + C_6^^8\times C_1^^4 + C_7^^8
  = 56\times6 + 28\times4 + 8 = 336 + 112 + 8 = 456$.
\item The students can choose to answer the question paper in $4$ ways. $5$ questions from first paper and
  $2$ from second paper, $2$ questions from first paper and $5$ questions from second paper, $4$ questions
  from first paper and $3$ from second paper, and $3$ questions from first paper and $3$ questions from
  second paper.

  Because both papers contain $6$ questions each the no.\ of ways for first and second method will be same
  and ways for third and fourth method will be same as well. So we can find no.\ of ways in two cases and
  multiply the sum by $2$ to arrive at the answer.

  {\bf Case I:} When the student chooses first or second method.

  $5$ questions can be chosen out of $6$ in $C_5^^6$ ways and $2$ questions can be chosen out of $6$ in
  $C_2^^6$ ways.

  Thus, no.\ of selections in this case is $C_5^^6 \times C_2^^6 = 6\times15 = 90$.

  {\bf Case II:} When the student chooses third or fourth method.

  Following like previous case, no.\ of selections in this case is $C_4^^6\times C_3^^6 = 15\times20 = 300$.

  Thus, total no.\ of selections of questions is $2(90 + 300) = 780$.
\item We can choose $1$ point out $P$ and $Q$ in $C_1^^2$ and $2$ from remaining other $8$ points in
  $C_2^^8$ ways, making no.\ of triangles $C_1^^2\times C_2^^8 = 56$. Clearly, half of these would include
  $P$ but exclude $Q$. Thus, $28$ triangles will include $P$ and exclude $Q$.

  In second case, both $P$ and $Q$ would be chosen in $1$ way and $1$ point from the other line would be
  chosen in $C_1^^8 = 8$ ways. This gives us $8$ triangles.

  Thus, total no.\ of triangles is $56 + 8 = 64$.
\item There can be two cases. First, when $1$ vote is casted, and second, when $2$ votes are
  casted.

  {\bf Case I:} When $1$ vote is casted.

  We can choose $1$ from men or $1$ from ladies. Thus, total no.\ of choices are $C_1^^7 + C_1^^3 = 7 + 3 =
  10$.

  {\bf Case II:} When $2$ votes are casted.

  The two votes can be casted by making choices in three different manners. We can choose $2$ men or $2$
  ladies or $1$ man and $1$ lady. Thus, total no.\ of choices are $C_2^^7 + C_2^^3 + C_1^^7\times C_1^^3 = 21
  + 3 + 21 = 45$.

  Thus, total no.\ of ways in which votes can be casted are $45 + 10 = 55$.
\item No.\ of ways of choosing boys are $C_3^^{10} = 120$. Let us first choose girls in an unrestricted
  manner. No.\ of ways of choosing $3$ girls out of $7$ are $C_3^^7 = 35$. Now assume that the two particular
  girls, who cannot be together are always there in selection. Treating these two girls as one, and always
  selecting them gives us $C_1^^5 = 5$ choices.

  However, these two girls cannot be in the same group, so total no.\ of choosing girls are $35 - 5 =
  30$. And thus, no.\ of ways of forming the party is $120\times30 = 3600$.
\item Since there are three sets and we have to answer at least two questions from each set for a total of
  seven questions, so we will have to choose three questions from one of the sets.

  Thus, total no.\ of selecting questions is $C_3^^4\times C_2^^5\times C_2^^6 + C_2^^4\times C_3^^5\times
  C_2^^6 + C_2^^4\times C_2^^5\times C_3^^6 = 600 + 900 + 1200 = 2700$.
\item From $5$ apples we can choose one of $0, 1, 2, 3, 4, 5$ apples i.e. $6$ selections. Similarly, for
  oranges no.\ of selections is $5$, and for mangoes it is $4$.

  Thus, total no.\ of selections are $6\times5\times4 = 120$. However, one of these selections will contains
  $0$ fruits. Thus, the answer is $120 - 1 = 119$.
\item No.\ of ways to select red balls $C_1^^4 + C_2^^4 + C_3^^4 + C_4^^4 = 4 + 6 + 4 + 1 = 15$. No.\ of ways
  to select green balls $C_0^^3 + C_1^^3 + C_2^^3 + C_3^^3 = 8$.

  Thus, total no.\ of selections are $15\times8 = 120$.
\item There are three bills. We can choose one, two or all them to form a sum. Thus, total no.\ of sums are
  $C_1^^3 + C_2^^3 + C_3^^3 = 7$.
\item The boy can solve $1 ,2, 3, 4$ or $5$ questions from the paper. Thus, total no.\ of ways are $C_1^^5 +
  C_2^^5 + C_3^^5 + C_4^^5 + C_5^^5 = 5 + 10 + 10 + 5 + 1 = 31$.
\item The voter can vote for one seat in $C_1^^6$ ways, for two seats in $C_2^^6$ ways, and for three seats in
  $C_3^^6$ ways. Thus, total no.\ of ways in which the voter can vote are $C_1^^6 + C_2^^6 + C_3^^6 = 41$.
\item Let there be $n$ candidates out of which $n - 1$ have to be elected. Total no.\ of ways in which this
  can be done are $C_1^^{n} + C_2^^n + \cdots + C_{n - 1}^^n = 30 \Rightarrow 2^n = 32 \Rightarrow n =
  5$(Using binomial theorem).
\item $12$ books are to be distributed equally among $4$ person, so each will get $3$ books. No.\ of ways to
  select $3$ out of $12$ are $C_3^^{12}$, no.\ of selections for $3$ out of remaining $9$ are $C_3^^9$ and
  so on.

  Thus, total no.\ of ways of distributing the books are $C_3^^{12}\times C_3^^9\times C_3^^6\times C_3^^3 =
  \frac{12!}{(3!)^4}$.
\item no.\ of ways distributing $n$ identical things among $r$ people, where any person can get any no.\ of
  things is $C_{r - 1}^^{n + r - 1}$. Therefore, required no.\ of ways are $C_3^^{13}$.
\item $3$ constants can be selected out of $10$ consonants in $C_3^^{10}$ ways. $2$ vowels out of $4$ can be
  selected in $C_2^^4$ ways. Now we have $5$ alphabets which be arranged in $5!$ ways. Thus, total no.\ of
  words formed are $C_3^^{10}\times C_2^^4\times5!$.
\item We seat $X, Y, Z$ on the side facing the window. Now from remaining $4$ one has to sit on this side,
  which is $C_1^^4$ ways of selection. From remaining $3$ all $3$ have to sit on the other side, which is
  $C_3^^3$ ways of selection. Thus, total no.\ of selections are $4$.

  For each selection no.\ of arrnagements are $4!\times3!$. Hence, total no.\ of ways of seating are
  $4!\times3!\times4 = 576$.
\item Six men have preferences. Suppose on one side we seat $4$ men who wish to sit together then to fill
  remaining $4$ positions we have to choose from $10$ i.e. $C_4^^{10}$. Now for the remaining $6$ free seats
  we have $6$ people, which can seated in $1$ way. Both sides can be arranged in $8!$ ways.

  Thus, no.\ of ways of seating them is $C_4^^{10}.8!.8!$.
\item Since two women are to be seated on seats numbered $1$ to $4$, the no.\ of arrangements are $P_2^^4 =
  12$. Now, three men are to be seated on $5$ remaining chairs; the no.\ of arrangements are $P_3^^6 = 120$.

  Thus, total no.\ of arrangements are $12\times120 = 1440$.
\item Consider $\frac{C_r^^{2n}}{C_{r - 1}^^{2n}} = \frac{2n - r + 1}{r}$, which has to be greater than $1$
  if $C_r^^{2n}$ has to be greatest. $\therefore 2n - r + 1 \geq r \Rightarrow r\leq n + \frac{1}{2}$.

  Similarly, considering $\frac{C_r^^{2n}}{C_{r + 1}^^{2n}}$, we find that $r\geq n - \frac{1}{2}$.

  Combining the results $n - \frac{1}{2}\leq r\leq n + \frac{1}{2}$, but $r$ has to be an integer. Thus, $r
  = n$.
\item In a seven digit number there are four odd places. There are two $1$'s, and two $3$'s, which can
  occupy these places. no.\ of such ways are $\frac{4!}{2!2!} = 6$.

  For three even places, we have one $4$, and two $2$'s. no.\ of ways in which three even places can be
  filled are $\frac{3!}{2!} = 3$.

  Thus, total no.\ of required no.\ are $6\times3 = 18$.
\item There are $10$ letters, and the words have five of these. no.\ of words where any letter can be
  repeated are $10^5$. no.\ of letters where none of the letters are repeated are $^{10}P_5$.

  Thus, no.\ of words where at least one letter is repeated are $10^5 - \ ^{10}P_5$.
\item Ternary no.\ include $0, 1, 2$. First we consider the case when the required seuqence begins with
  $210$. We have six vacant places, which can be filled in $3^6 = 729$ ways. Similarly, no.\ of sequences
  which end with $210$ will be $729$.

  However, there will be common sequences between these two which start with $210$ and end with
  $210$. no.\ of such sequences will be $3^3 = 27$ (Hint: we will have only three empty places).

  Thus, no.\ of required numbers are $729 + 729 - 27 = 1431$.
\item A seven digit no.\ will be a no.\ ranging from $1,000,000$ to $9,999,999$. If units place is odd then
  sum of remaining six digits must be odd or if units place is even then sum of remaining six digits
  must be even to satisfy the condition given. Thus, half the no.\ will satisfy the given condition.

  $\therefore $ Required number $= 9\times10\times10\times10\times10\times10\times5 = 45\times10^5$.
\item Treating $10$ Indians as one person, we have $1 + 4 + 5 = 10$ persons. These can be seated in $10!$
  ways. However, $10$ Indians can be seated among themselves in $10!$ ways.

  Thus, total no.\ of seating arrangements are $10!10!$.
\item Total no.\ of letters are $7$; of which $2$ are \quote{A}, and $2$ are \quote{R}. Total no.\ of
  arrangements when there is no restriction $=\frac{7!}{2!2!} = 1260$.

  (a) Treating two \quote{R} as one i.e. we are considering words when both \quote{R}s are together. no.\ of
  such words $=\frac{6!}{2!} = 360$. Thus, no.\ of words where \quote{R} are never together $= 1260 - 360 =
  900$.

  (b) Number of arrangements when two \quote{A}s are together is $360$ like previous case. Treating both
  \quote{A}s and \quote{R}'s as one i.e. when both are together, no.\ of words $=5! = 120$.

  Thus, required no.\ of words $= 360 - 120 = 240$.

  (c) From (a) and (b) it follows that required number $= 900 - 240 = 660$.
\item no.\ of ways of dividing $m + n$ persons into two groups such that one has $m$ persons, and the other
  has $n$ persons is $C_m^^{m + 1}.C_n^^n = \frac{(m + n)!}{m!n!}$.

  Now, $m$ persons can be seated around a round table in $(m - 1)!$ ways; similarly $n$ persons can be
  seated in $(n - 1)!$ ways.

  Thus, total no.\ of ways is $\frac{(m + n)!(m - 1)!(n - 1)!}{m!n!} = \frac{(m + n)!}{mn}$.
\item The signal can be made by using any no.\ of flags. Thus, required no.\ of signals is $P_1^^5 + P_2^^5
  + P_3^^5 + P_4^^5 + P_5^^5 = 325$.
\item There are $5$ letters in the word \quote{Ought}, which are all different. The alphabetical order of
  letters are G, H, O,T, U.

  no.\ of words beginning with G, H, O are $4!\times3 = 24\times3 = 72$. no.\ of words beginning with TG are
  $3! = 6$, and same for $TH$. no.\ of words beginning with TOG and TOH are $2! = 2$. TOUGH is the first word
  beginning with $TOU$. Thus, rank of the word \quote{TOUGH} in the dictionary will be $24\times3 + 6\times2
  + 2\times2 + 1 = 89$.
\item Let the city be represented by a rectangle, whose sides are of length $a$ and $b$ North-South and
  East-West respectively.
  \startplacefigure[location={left,none}]
    \startMPcode
      draw (0, 0) -- (0, 2.5cm);
      draw (0.5cm, 0) -- (0.5cm, 2.5cm);
      draw (1cm, 0) -- (1cm, 2.5cm);
      draw (1.5cm, 0) -- (1.5cm, 2.5cm);
      draw (2cm, 0) -- (2cm, 2.5cm);
      draw (2.5cm, 0) -- (2.5cm, 2.5cm);
      draw (0, 0) -- (2.5cm, 0cm);
      draw (0, 0.5cm) -- (2.5cm, 0.5cm);
      draw (0, 1cm) -- (2.5cm, 1cm);
      draw (0, 1.5cm) -- (2.5cm, 1.5cm);
      draw (0, 2cm) -- (2.5cm, 2cm);
      draw (0, 2.5cm) -- (2.5cm, 2.5cm);
      label.ulft("$P$", (0cm, 2.5cm));
      label.lrt("$Q$", (2.5cm, 0cm));
      label.lft("$W$", (0cm, 1.25cm));
      label.rt("$E$", (2.5cm, 1.25cm));
      label.top("$N$", (1.25cm, 2.5cm));
      label.bot("$S$", (1.25cm, 0cm));
    \stopMPcode
  \stopplacefigure
Man has to go from $P$ to $Q$. For this he has to travel $a$ vertically downward and $b$ horizontally
Eastward. There are $m - 1$ horizontal segments and $n - 1$ vertical segments. Thus, from $P$ to $Q$ there
are $m + n - 2$ segments total. We have to choose $m - 1$, and $n - 1$ segments from these. This can be done
in $\frac{(m + n - 2)!}{(m - 1)!(n - 1)!}$ ways.
\vskip 1.1cm
\item Let the $n$ letters be denoted by $1, 2, 3, \ldots, n$. Let $A_i$ denote the set of distribution of
  letters in envelopes so that only $i$th letter is put in the corresponding envelope. Then $n(A_i) = (n -
  1)!$, because rest $n - 1$ letters can be put in $n - 1$ envelopes in $(n - 1)!$ ways.

  Similarly, $n(A_i\cap A_j)$ i.e. putting two letters in correct envelopes is $(n - 2)!$. Required number
  $= n(A_1'\cap A_2' \cap A_3' \cap\ldots\cap A_n') = n(A_1\cup A_2\cup A_3\cup\ldots \cup A_n)' = n! -
  n(A_1\cup A_2\cup A_3\cup \ldots \cup A_n)'$

  $= n! - [\sum n(A_i) - \sum n(A_i\cap A_j) + \sum n(A_i\cap A_j\cap A_k) - \cdots + (-1)^nn(A_1\cap
    A_2\cap \ldots \cap A_n)]$

  $= n! - \left[C_1^^n (n - 1)! - C_2^^n (n - 2)! \cdots\right] = n!\left[\frac{1}{2!} - \frac{1}{3!} +
    \frac{1}{4!} + \cdots\right]$.
\item Number of non-congruent squares is $8$ as they are of size $1\times 1, 2\times2, 3\times3, \ldots,
  8\times8$. Number of non-congruent rectanges, which are not squares $= C_2^^8 = 28$.

  Thus, required number $= 28 + 8 = 36$.
\item Let the three numbers selected from are $a, b, c$, which have to be in A.P. i.e. $a + c = 2b$. This
  implies that both $a, c$ are either even or odd as sum has to be even.

  {\bf Case I:} When $n$ is even.

  Let $n = 2m$, then no.\ of odd and even numbers are same i.e. $m$. Thus, no.\ of ways in which $a$ and $c$ can
  be selected is $2\times C_2^^m = m(m - 1) = \frac{1}{4}n(n - 2)$.

  {\bf Case II:} When $n$ is odd.

  Let $n = 2m + 1$, then no.\ of odd numbers is $m +1$, and no.\ of even numbers is $m$. Thus, $a$ and $c$
  can be selected in $C_2^^{m + 1} + C_2^^m = \frac{1}{4}(n - 1)^2$ ways.
\item Since there are two packs of $52$ cards, therefore, number of cards from same suit and denomination is
  $2$ for each card.

  no.\ of ways of selecting $26$ cards out of $52$ cards $= C_{26}^^{52}$, however, each card can be selected
  in $2$ ways. $\therefore $ Required numbers $= C_{26}^^{52}.2^{26}$.
\item For $n$ sides there will be $n$ vertices. Selection of any $3$ vertices will give us a
  triangle. no.\ of ways of selecting $3$ vertices out of $n$ vertices i.e. no.\ of triangles $= C_3^^n =
  \frac{n(n - 1)(n - 2)}{6}$.
\item (a) If the $n$ objects are $o_1, o_2, o_3, \ldots, o_n$, then possible solutions will be $o_1o_2o_3,
  o_2o_3o_4, o_3o_4o_5, \ldots, o_{n - 2}o_{n - 1}o_n$.

  $\therefore$ Required number $= n - 2$.

  (b) no.\ of ways to select $3$ objetcs out of $n$ objects without restriction $= C_3^^n$. Thus, following
  from (a) required number $C_3^^n - n + 2 = \frac{(n - 3)(n^2 - 4)}{6}$.
\item Let $a$ be the no.\ of stations before stop $1, b$ be the no.\ of stations before stop $2, c$ be the
  no.\ of stations before stop $3, d$ be the no.\ of stations before stop $4$, and $e$ be the no.\ of stations
  after stop $4$. Then, $a + b + c + d + e = 8$, where $a\geq 0, b, c, d \geq 1, e\geq 0$.

  Let $x = a, y = b - 1, z = c - 1, t = d - 1, w = e$, then $x + y + z + t + w = 5$.

  $\therefore$ Required no.\ = Number of non-negative integral solutions of the above equation $= C_r^^{n + r
    - 1} = C_5^^9 = 126$.
\item no.\ of straight lines formed by given $m$ points $= C_2^^m = n$ (let). Total no.\ of points of
  intersections of these lines $= C_2^^n$ under given conditions.

  Consider a point $A_1$. No.\ of lines passing through $A_1 = m - 1$. No.\ of pair of lines intersecting at
  $A_1 = C_2^^{m - 1}$. Similarly, this will be the case for other points.

  Hence, required no.\ of points of intersections $= C_2^^n - m.C_2^^{m - 1} = \frac{m!}{8(m - 4)!}$.
\item The word BAC cannot be spelled if the $m$ selected coupons do not contain at least one of A, B or C.

  no.\ of ways of selecting m coupon which are $A$ or $B = 2^m$. This also includes when all $m$ coupons are
  all A or all B. Similarly for B or C and for A and C. no.\ of ways of selecting $m$ coupons where all are
  $A = 1$. Similarly for B and C.

  Thus, required no.\ $= 2^m + 2^m + 2^m - 1 - 1 - 1 = 3(2^m - 1)$.
\item The staight cards can be $1-5, 2-6, 3-7, \ldots, 6-10, 7-J, 8-Q, 9-K, 10-A$. Thus, we see there are
  $10$ such straight hands. One card of any denomination can be picked from any of the suits in $4$
  ways. Thus, $5$ cards of five different denominations can be selected from $4$ suits in $4^5$ ways.

  Thus, number of ways of making selections $= 10\times4^5 = 10,240$.

  If all cards are not from same suit then no.\ of ways of making selections $= 10\times4^5 - 10\times4 =
  10,200$ because there aree $4$ suits.
\item Let $A = {a_1, a_2, \ldots, a_n}$. Consider element $a_1$. Either it is in $P_1$ or it is not. So
  total no.\ of ways for $a_1$ and $P_1 = 2$. No.\ of ways in which $a_1$ is in $P_1 = 1$, and same for not
  belonging i.e. in $1$ way.

  Total no.\ of ways for $a_1$, and $m$ subsets $= 2^m$. No.\ of ways in which $a_1$ belongs to $m$ subsets
  $= 1^m = 1$. No.\ of ways in which $a_1$ belongs to none of the subsets $= 1^m = 1$.

  Thus, total no.\ of ways in which $a_1\notin(P_1\cap P_2\cap\ldots\cap P_m) = 2^m - 1,
  a_1\notin(P_1\cup P_2\cup\ldots\cup P_m) = 1^m$, and $a_1\in(P_1\cup P_2\cup\ldots\cup P_m) = 2^m - 1$.

  \startitemize[i]
  \item Here exactly $r$ elements of $A$ belongs to $P_1\cup P_2\cup\ldots\cup P_m$, and $n - r$ elements do
    not belong to $P_1\cup P_2\cup\ldots\cup P_m$.

    $\therefore$ Required number $= C_r^^n(2^m - 1)^r(1)^{n - r}= C_r^^n(2^m - 1)^r$.
  \item Here exactly $r$ elements of $A$ belongs to $P_1\cap P_2\cap\ldots\cap P_m$, and $n - r$ elements do
    not belong to $P_1\cap P_2\cap\ldots\cap P_m$.

    $\therefore$ Required number $= C_r^^n(2^m - 1)^{n - r}(1)^r = C_r^^n(2^m - 1)^{n -r}$.
  \item Let $P_{m + 1} = A - (P_1\cup P_2\cup\ldots\cup P_m)$. Since $P_i\cap P_i = \phi, i\neq j$, where
    $i, j = 1, 2, \ldots, m$.

    Each element of $A$ should belong to exactly one of the $(m + 1)$ subsets $P_1, P_2, \ldots, P_m, P_{m +
      1}$. For one element there are $m + 1$ ways so for $n$ elements there are $(m + 1)^n$ ways.
  \stopitemize
\item Given that number of boxes is $2m$, and number of identical balls is $m$. Number of ways to select $m$
  boxes out of $2m$ is $C_m^^{2m}$. Because $m$ balls are identical they can be arranged in $\frac{m!}{m!} =
  1$ way.

  $\therefore$ Required number $= C_m^^{2m} = \frac{2m!}{m!m!}$.

  We will make use of mathematical induction to show that $\frac{4^m}{2\sqrt{m}}\leq \frac{2m!}{m!m!}\leq
  \frac{4^m}{\sqrt{2m + 1}}$.

  Let $P(m): \frac{4^m}{2\sqrt{m}}\leq \frac{2m!}{(m!)^2}$.

  When $m = 1$, L.H.S. $= 2$, and R.H.S. $= 2$. The equality holds, and hence, $P(1)$ is true.

  Let $P(k)$ be true $\Rightarrow \frac{4^k}{2\sqrt{k}}\leq \frac{2k!}{(k!)^2}$.

  We have to prove that $P(k + 1)$ is true i.e. $\frac{4^{k + 1}}{2\sqrt{k + 1}}\leq\frac{2(k + 1)!}{[(k +
      1)!]^2} = \alpha$ (say).

  Multiplying both sides of $P(k)$ with $\frac{(2k + 1)(2k + 2)}{(k + 1)^2} = \frac{2(2k + 1)}{k + 1}$, we
  have

  $\frac{(2k + 2)!}{[(k + 1)!]^2}\geq \frac{2(2k + 1)4^k}{2\sqrt{k(k + 1)}} = \frac{(2k + 1)4^k}{\sqrt{k(k +
      1)}} = \beta$ (say).

  Now $\frac{\beta}{\alpha} = \frac{(2k + 1)4^k}{\sqrt{k(k + 1)}}.\frac{2\sqrt{k + 1}}{4^{k + 1}} = \frac{2k
  + 1}{2\sqrt{k(k + 1)}} = \frac{2k + 1}{\sqrt{4k^2 + 4k}} = \frac{\sqrt{4k^2 + 4k + 1}}{\sqrt{4k^2 + 4k}} >
  1 \Rightarrow \beta > \alpha$.

  Hence, $P(k + 1)$ is true whenever $P(k)$ is true. Thus, $P(m)$ si true for all natural numbers $m$.

  Let $Q(m): \frac{2m!}{(m!)^2}\leq \frac{4^m}{\sqrt{2m + 1}}$

  When $m = 1$, L.H.S. $= 2$, and R.H.S. $= \frac{4}{\sqrt{3}}$, so L.H.S. $<$ R.H.S. making $Q(1)$ true.

  Let $Q(k)$ be true i.e. $\frac{2k!}{(m!)^2}\leq \frac{4^k}{\sqrt{2k + 1}}$

  We have to prove that $Q(k + 1)$ i.e. $\frac{(2k + 2)!}{[(k + 1)!]^2}\leq\frac{4^{k + 1}}{\sqrt{2k + 3}} =
  x$ (let).

  Multiplying $Q(k)$ with $\frac{(2k + 1)(2k + 2)}{(rk + 1)^2}$, we have $\frac{(2k + 2)!}{[(k + 1)!]^2}\leq
  \frac{4^{k + 1}}{\sqrt{2k + 1}}.\frac{(2k + 1)(2k + 2)}{(k + 1)^2} = \frac{4^k.2\sqrt{2k + 1}}{k + 1} = y$
  (say).

  $\therefore \frac{y}{x} = \frac{4^k.2\sqrt{2k + 1}}{k + 1}.\frac{\sqrt{2k + 3}}{4^{k + 1}} =
  \frac{\sqrt{4k^2 + 8k + 3}}{4k^2 + 8k + 4}\Rightarrow y < x$.

  Thus, $Q(k + 1)$ is true, and hence $Q(m)$ is true for all $m$.

  Hence, we have our proof using induction.
\item To form a parallelogram we need to select $2$ lines from one set, and $2$ from the other set. Thus,
  no.\ of parallelograms formed is $C_2^^m\times C_2^^n = \frac{1}{4}mn(m - 1)(n - 1)$.
\item \starttabulate[|c|c|l|]
  \NC Number of ladies\NC Number of men\NC Number of committees\NC\NR
  \NC $1$\NC $4$\NC $C_1^^4C_4^^6 = 60$\NC\NR
  \NC $2$\NC $3$\NC $(C_2^^4 - C_0^^2).C_3^^6 = 100$\NC\NR
  \NC $3$\NC $1$\NC $(C_3^^4 - C_1^^2).C_2^^6 = 30$\NC\NR
  \NC $4$\NC $1$\NC Not possible\NC\NR
\stoptabulate
\item {\bf Case I:} $3$ men from husband's side, and $3$ ladies from wife's side. no.\ of ways to do this is
  $C_0^^4\times C_3^^3\times C_3^^3\times C_0^^4 = 1$

  {\bf Case II:} $2$ men, and $1$ lady from husband's side, and $1$ man and $2$ ladies from wife's
  side. no.\ of ways to do this is $C_1^^4\times C_2^^3\times C_2^^3\times C_1^^4 = 144$

  {\bf Case III:} $1$ man, and $2$ ladies from husband's side, and $2$ men and $1$ lady from wife's
  side. no.\ of ways to do this is $C_2^^4\times C_1^^3\times C_1^^3\times C_2^^4 = 324$

  {\bf Case IV:} $3$ ladies from husband's side, and $3$ men from wife's side. no.\ of ways to do this is
  $C_3^^4\times C_0^^3\times C_0^^3\times C_3^^4 = 16$

  $\therefore$ Required number $= 1 + 144 + 324 + 16 = 485$.
\item For an intersection we need two lines such that they have one point on each of the given lines. Thus,
  total no.\ of ways to select these four points is $C_2^^m\times C_2^^n = \frac{1}{4}mn(m - 1)(n - 1)$.
\item Let $y$ be the no.\ of children born after John and Mary marry. Then $x + x + 1 + y = 24 \Rightarrow
  2x + y = 23$.

  Let $z$ be the no.\ of fights, then $z = C_1^^x.C_1^^y + C_1^^x.C_1^^{x + 1} + C_1^^y.C_1^^{x + 1}
  \Rightarrow z = xy + x(x + 1) + y(x + 1) = x(23 - 2x) + x^2 + x + (23 - 2x)(x + 1) = -3x^2 + 45x + 23$

  $\Rightarrow 3x^2 - 45x + z - 23 = 0$, now, because $x$ is read $D\geq 0 \Rightarrow 45^2 - 12(z - 23)
  \geq 0 \Rightarrow z \leq \frac{2301}{12} = 191.75$.

  So, greatest value of $z$ is $191$.
\item Doing prime factorization, we have $2520 = 2^3\times3^2\times5\times7$. Each term of the product $(1 +
  2 + 2^2 + 2^3)(1 + 3 + 3^2)(1 + 5)(1 + 7)$ is a divisor of $2520$. Total no.\ of divisors is equal to total
  no.\ of terms in the product $= 48$. Sum of divisors $= (1 + 2 + 2^2 + 2^3)(1 + 3 + 3^2)(1 + 5)(1 + 7) =
  9360$.
\item There can be two sets of three positive integers whose sum is $5$. These sets would be $\{1, 1, 3\}$
  and $\{1, 2, 2\}$. Elements of both the set can be arranged in $\frac{3!}{2!}$ i.e. $3$ ways.

  $\therefore$ Required no.\ $=3\times C_3^^5\times C_1^^2\times C_1^^1 + 3\times C_2^^5\times C_2^^3\times
  C_1^^1 = 150$.
\item Let $m = (n - 1)!$, then $n! = mn$. Now $\frac{(n!)!}{(n!)^{(n - 1)!}} = \frac{(mn)!}{(n!)^m}$, which
  is no.\ of ways of distributing $mn$ things among $m$ persons each having $n$ things.
\item $\frac{{(ab)!}}{a!{(b!)}^{a}}$ is no.\  of ways of distributing $ab$ different things in $a$ sets each
  having $b$ things, which is an integer.
\item Number of ways of distributing $n$ identical objects in $r$ groups, where each group can contain any
  number of objects, and the ordering matters $= P_r^^{n + r - 1} = P_6^^{206}$.
\item {\bf Method I:} Since each person has to get at least $3$ things, if $3$ persons get $3$ things $4$th
  can get at most $7$ things. Thus,

  Required numbers = coeff. of $x^{16}$ in $(x^3 + x^4 + \cdots + x^7)^4 = $ coeff. of $c^{16}$ in $x^{12}(1
  + x + \cdots + x^4)^4 =$ coeff. of $x^4$ in $\left(\frac{1 - x^5}{1 - x}\right)^4 = C_4^^7 = 35$.

  {\bf Method II:} Let the four persons be give $a, b, c, d$ no.\ of things. Then, $a + b + c + d = 16$,
  where $a, b, c, d\geq 3$, then $w + x + y + z = 4, w, x, y, z\geq 0$, and $w = a - 3, x = b - 3, y = c -
  3, z = d - 3$.

  Required no.\ is solution of any of the above equations, which is number of ways of distributing $4$
  identical things among $4$ persons, where each person can get any no.\ of things $= C_r^^{n + r - 1} =
  C_4^^7 = 35$.

  {\bf Method III}: Sets of four posiitve integers each greater than or equal to $3$ whose sum is $16$ are
  $\{7, 3, 3, 3\}, \{6, 4, ,3, 3\}, \{5, 5, 3, 3\}, \{5, 4, 4, 3\}, \{4, 4, 4, 4\}$.

  Elements of first set can be arranged in $\frac{4!}{3!} = 4$ ways. Elements of second set can be arranged
  in $\frac{4!}{2!} = 12$ ways. Elements of third set can be arranged in $\frac{4!}{2!2!} = 6$
  ways. Elements of fourth set can be arranged in $\frac{4!}{2!} = 12$ ways. Elements of fifth set can be
  arranged in $\frac{4!}{4!} = 1$ way.

  Required no.\ $= 4 + 12 + 6 + 12 + 1 = 35$.
\item Let no.\ of red, white, blue, and green balls be $w, x, y$ and $z$ respectively. From question, $w + x
  + y + z = 10$, where $w, x, y, z\geq 0$.

  This is no. of ways of distributing $10$ identical things among four persons where each can get any no.\
  of things $= C_r^^{n + r - 1} = C_{10}^^13 = 286$.

  When the selections contain balls of each color the equation remains same, but $w, x, y, z\geq 1$. So $a +
  b + c + d = 6$, where, $a = w - 1, b = x - 1$ and so on.

  In this case, the method is same but $n = 4, r = 6$, so the answer is $C_6^^9 = 84$.
\item Let the questions contain $x_1, x_2, \ldots, x_8$ marks, then, from question $x_1 + x_2 + \cdots + x_8
  = 30$, where $x-1, x_2, \ldots, x_8\geq 2$.

  $\Rightarrow y_1 + y_2 + \cdots + y_8 = 14$, where $y_1 = x_1 - 2$, and so on.

  Required number is no.\ of solutions of above equations $= C_8^^{n + r - 1} = C_{14}^^{21} = 116,280$.
\item Total marks is $3\times50 + 100 = 250$, and the student must score $60\%$ i.e. $150$ marks.

  Required number = coeff. of $x^{150}$ in $(1 + x + \cdots + x^{50})^3(1 + x + \cdots + x^{100}) =$
  coeff. of $x^{150}$ in $\left(\frac{1 - x^{15}}{1 - x}\right)^3\frac{1 - x^{101}}{1 - x} =$ coeff. of
  $x^{150}$ in $(1 - x^{51})^3(1 - x^{101})(1 - x)^{-4}= $ coeff. of $x^{150}$ in $(1 - 3x^{51} + 3x^{102} -
  x^{101})(1 - x)^{-4}$ (leaving powers greater than $160$)

  $=$ coeff.of $x^{150}$ in $(1 - x)^{-4} - 3.$coeff. of $x^{99}$ in $(1 - x)^{-4} + 3.$coeff. of $x^{48}$
  in $(1 - x)^{-4} -$ coeff. of $x^{49}$ in $(1 - x)^{-4}$

  $= C_{150}^^{153} - 3.C_{90}^^{102} + 3.C_{48}^^{51} - C_{49}^^{52} = 110,556$.
\item Given, $x_1 + x_2 + \cdots + x_k = n$. Let $y_1 = x_1 - 1, y_2 = x_2 - 2, \ldots, y_k = x_k - k$, then
  $y_1 + y_2 + \cdots + y_k = n - (1 + 2 + \cdots + k) = n - \frac{k(k + 1)}{2} = m$.

  Then no. of solutions $\displaystyle = C_m^^{m + k - 1} = \frac{\left(n + k - \frac{k(k + 1)}{2} -
    1\right)!}{\left(n - \frac{k(k + 1)}{2}\right)!(k - 1)!}$.
\item Given $x + y + z + w = 29$, where $x\geq 1, y\geq 2, \geq 3, w\geq 0$. Putting $p = x - 1, q = y - 2,
  r = z - 3$,

  $p + q + r + w = 23$, where $p, q, , r, w\geq 0$.

  Following like previous problems, required no. $= C_{23}^^{26} = 2,600$.
\item Required number $=$ coeff. of $x^{20}$ in $(1 - x)^{-3}(1 - x^4)^{-1} =$ coeff. of $x^{20}$ in $(1 +
  C_1^^3x + C_2^^4x^2 + C_3^^5x^3 + \cdots + C_{20}^^{22}x^{20} + \cdots)(1 + x^4 + x^8 + x^{12} + x^{16} +
  x^{20} + \cdots)$

  $= 1 + C_4^^6 + C_8^^{10} + C_{12}^^{14} + C_{16}^^{18} + C_{20}^^{22} = 536$.
\item From given equations we have $v + w = 15$, and $x + y + z = 3$.

  Number of non-negative integral solution of these equations combined is $= C_5^^7.C_{15}^^{16} = 336$.
\item Given inequality is $3x + y + z\geq 30$. Let $w$ is a non-negative integer such that $3x + y + z + w =
  30$, where $x, y, z, w\geq 1$.

  Let $a = x - 1, b = y - 1, c = z - 1, d = w$, then $3a + b + c + d = 25$, where $a, b, c, d\geq 0$.

  Clearly, $0\leq a\leq 8$. If $a = k$, then $b + c + d = 25 - 3k$.

  No. of non-negative solutions of this equation is $C_{25 - 3k}^^{27 - 3k} = C_2^^{27 - 3k} =
  \frac{3}{2}(3k^2 - 53k + 234)$

  $\therefore$ Required numbers $= \frac{3}{2}\sum_{k = 0}^k(3k^2 - 53k + 234) = 1215$.
\item Given $a + b + c + d = 20,$ where $a, b, c, d \geq 1$. Let us assume that $a < b < c < d$. Also let,
  $x = a, y = b - a, z = c - b, w = d - c \therefore a = x, b = y + x, c = x + y + z, d = x + y + z + w$

  $\therefore 4x + 3y + 2z + w = 20. \therefore $ Sum of minimum values of $4x, 3y, 2z$ and $w = 4 + 3 + 2 +
  1 = 10$.

  Required number $=$ number of positive unequal integral solutions of above equation

  $=$ coeff. of $x^{10}$ in $(1 - x^4)^{-1}(1 - x^3)^{-1}(1 - x^2)^{-1}(1 - x)^{-1} = $ coeff. of $x^{10}$
  in $[(1 + x^4 + x^8)(1 + x^3 + x^6 + x^9)(1 + x^2 + x^4 + \cdots + x^{10})(1 + x + x^2 + \cdots +
    x^{10})]$(leaving terms greater than $x^{10}$)

  $=$ coeff. of $x^{10}$ in $[(1 + x^3 + x^4 + x^6 + x^7 + x^8 + x^9 + x^{10})(1 + x + 2x^2 + 2x^3 + 3x^4 +
    3x^5 + 4x^6 + 4x^7 + 5x^8 + 4x^9 + 6x^{10})] = 23$

  But $a, b, c, d$ can be arranged in $4!$ ways among themselves. Thus, total no. of unique solutions is
  $23\times4! = 552$.
\item Any no. between $1$ and $1,000,000$ must be of less than seven digits. Thus, $a_1 + a_2 + a_3 + a_4 +
  a_5 + a_ 6 = 18$, where $a_1, a_2, \ldots, a_5\in{0, 1, 2, \ldots, 9}$, and the number is of the form
  $a_1a_2a_3a_4a_5a_6$.

  $\therefore$ Required number $=$ coefficient of $x^{18}$ in $(1 + x + x^2 + \cdots + x^9)^6 =$ coeff. of
  $x^{18}$ in $\left(\frac{1 - x^{10}}{1 - x}\right)^6$

  $=$ coeff. of $x^{18}$ in $(1 - x^{10})^6(1 - x)^{-6} =$ coeff. of $x^{18}$ in $(1 - 6x^{10})(1 -
  x)^{-6}$(leaving out powers greater than $x^{18}$)

  $= C_{18}^^{6 + 18 - 1} - 6.C_8^^{6 + 8 - 1} = 25,927$.
\item Required number $=$ coefficient of $x^n$ in $(1 + x + x^2 + \cdots + x^n)^2(1 + x)^n =$ coeff. of
  $x^n$ in $\left(\frac{1 - x^{n + 1}}{1 - x}\right)^2(1 + x)^n =$ coeff. of $x^n$ in $(1 - 2x^{n + 1} +
  x^{2n + 2})(1 - x)^{-2}(1 + x)^n =$ coeff. of $x^n$ in $(1 - x)^{-2}(1 + x)^n$(leaving powers greater than
  $n$)

  $=$ coeff. of $x^n$ in $(1 - x)^{-2}{2 - (1 - x)}^n =$ coeff. of $x^n$ in $(1 - x)^{-2}[2^n - C_1^^n2^{n -
  1}(1 - x) + C_2^^n2^{n - 2}(1 - x)^2 - \cdots + (-1)^nC_n^^n(1 - x)^n] =$ coeff. of $x^n$ in $[2^n(1 -
    x)^{-2} - C_1^^n2^{n - 1}(1 - x)^{-1}]$(other terms will not contain $x^n$)

  $= 2^n.C_n^^{2 + n - 1} - C_1^^n.2^{n - 1}C_n^^{1 + n - 1} = 2^{n - 1}(2n + 2)$.
\item No. of ways in which one crew out of $3$ can be arranged on the steering is $P_1^^3$.

  Since $2$ particular sailors are always to remain on bow side, therefore, $2$ more sailors for bow side
  can be selected out of remaining $6$ sailors in $C_2^^6$, and $4$ sailors for stroke side can be selected
  out of remaining $4$ in $C_4^^4$ ways.

  Now $4$ sailors on bow side can be arranged among themselves in $4!$ ways. Again $4$ sailors on stroke
  side can be arranged among themselves in $4!$ ways.

  $\therefore$ Required no. $= P_1^^3.C_2^^6.4!4! = 25,920$.
\item Total no. of letters is $11$. E and N occurs thrice, D occurs twice, and rest occur once.

  {\bf Case I:} When three letters are identical, and remaining two are identical. We can select three
  E\symbol[rightquote]s and two N\symbol[rightquote]s or three E\symbol[rightquote]s and two
  D\symbol[rightquote]s or three N\symbol[rightquote]s and two E\symbol[rightquote]s or three
  N\symbol[rightquote]s and two D\symbol[rightquote]s. Thus, there are total $4$ ways.

  {\bf Case II:} When three letters are identical, and remaining two are different. Letters selected can be
  three E\symbol[rightquote]s and two out of I, N, D, P, T or three N\symbol[rightquote]s and two out of I,
  E, D, P, T.

  No. of ways to select is $1\times C_2^^5 + 1\times C_2^^5 = 20$.

  {\bf Case III:} When a two letters are identical, and there are two such letters, and the fifth letter is
  different. Letter selected can be two E\symbol[rightquote]s, two N\symbol[rightquote]s, and one out of I,
  D, P, T or two E\symbol[rightquote]s, two D\symbol[rightquote]s, and one out of I, N, P, T or two
  N\symbol[rightquote]s, two D\symbol[rightquote]s, and one out of I, E, P, T.

  No. of ways to make these selections is $3\times C_1^^4 = 12$.

  {\bf Case IV:} When two letters are same, and remaining three are different. Letters selected can be two
  E\symbol[rightquote]s and three out of I, N, P, D, T or two N\symbol[rightquote]s and three out of I, E,
  P, D, T or two D\symbol[rightquote]s and three out of I, E, P, N, T.

  No. of ways to make these selections is $3\times C_3^^5 = 30$.

  {\bf Case V:} When all five letters are different. No. of ways to make these selections is $C_5^^6 = 6$.

  Adding all these we get $72$ as the answer.

  {\bf Second Method:} Previous method is direct, however, for bigger and more complex problems it becomes
  tedious.

  Required number $=$ coeff. of $x^5$ in $(1 + x + x^2 + x^3)^2(1 + x + x^2)(1 + x)^3 =$ coeff. of $x^5$ in
  $(1 + x^2 + x^4 + x^6 + 2x + 2x^3 + 2x^3 + 2x^4 + 2x^5)(1 + x + x^2)(1 + x)^3$

  $=$ coeff. of $x^5$ in $(1 + 2x + 3x^2 + 4x^3 + 3x^4 + 2x^5)(1 + x + x^2)(1 + 3x + 3x^2 + x^3) =$
  coeff. of $x^5$ in $(1 + 2x + 3x^2 + 4x^3 + 3x^4 + 2x^5)(1 + 4x + 7x^2 + 7x^3 + 4x^4 + x^5) = 1 + 8 + 21 +
  28 + 12 + 2 = 72$.
\item Here a occurs twice, l thrice, and p, r, e once.

  No. of combinations $=$ coeff. of $x^4$ in $(1 + x + x^2 + x^3)(1 + x + x^2)(1 + x)^3 =$ coeff. of $x^4$
  in $(1 - x)^{-5}(1 - x^2)^3(1 - x^3)(1 - x^4) = 22$

  No. of permutations $=$ coeff. of $x^4$ in $4!\left(1 + x + \frac{x^2}{2!} + \frac{x^3}{3!}\right)\left(1 + x +
  \frac{x^2}{2!}\right)(1 + x)^3$

  $= 286$.
\item L.H.S. $= \displaystyle\sum_{n = 1}^n(n^2 + 1)n! = (1^2 + 1).1! + (2^2 + 1).2! + \cdots + (n^2 + 1). n!$
\stopitemize
