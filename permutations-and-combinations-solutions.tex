% -*- mode: context; -*-
\chapter{Permutations and Combinations}
\startitemize[n, 1*broad]
\item Given, $P_4^^n = 360 \Rightarrow n(n - 1)(n - 2)(n - 3) = 3\times 4\times 5\times 6\Rightarrow n = 6$.
\item Given, $P_3^^n = 9240 \Rightarrow n(n - 1)(n - 2) = 20\times21\times22 \Rightarrow n = 22$.
\item Given, $P_r^^{10} = 720 = 8\times9\times10 \Rightarrow r = 3$.
\item Given, $P_{n - 1}^^{2n + 1}:P_n^^{2n - 1} = 3: 5 \Rightarrow \frac{(2n + 1)!}{(n + 2)!}.\frac{(2n -
  1)!}{(n -- 1)!} = \frac{3}{5}\Rightarrow \frac{(2n + 1)2n}{n(n + 1)(n + 2)}= \frac{3}{5}$

  $\Rightarrow 3n^2 - 11n - 4 = 0 \Rightarrow n = 4, -\frac{1}{3}$, but $n$ is an integer. Hence, $n = 4$.
\item Given, $P_4^^n = 12\times P_2^^n \Rightarrow n(n - 1)(n - 2)(n - 3) = 12\times n(n - 1)\Rightarrow n^2
  - 5n - 6 = 0 \Rightarrow n = 6, -1$.

  But $n > 0 \Rightarrow n = 6$ is the only solution.
\item Given, $P_5^^n = 20\times P_3^^n \Rightarrow (n -3)(n - 4) = 20  \Rightarrow n^2 - 7n - 8 =
  0\Rightarrow n = 8, -1$.

  But $n > 0 \Rightarrow n = 8$ is the only solution.
\item Given, $P_4^^n:P_4^^{n+1} = 3:4\Rightarrow \frac{n!}{(n - 4)!}.\frac{(n - 3)!}{(n + 1)!} =
  \frac{3}{4}$

  $\Rightarrow \frac{(n - 3)}{n + 1} = \frac{3}{4}\Rightarrow 4n - 12 = 3n + 3 \Rightarrow n = 15$.
\item Given $P_r^^{20} = 6840 = 18\times19\times20 \Rightarrow r = 3$.
\item Given, $P_{k + 1}^^{k + 5} = \frac{11(k - 1)}{2}.P_{k}^^{k + 3} \Rightarrow (k + 5)(k + 4)(k + 3)\cdots
  6.5 = \frac{11(k - 1)}{2}.(k + 3)(k + 2)\cdots 5.4$

  $\Rightarrow (k + 5)(k + 4) = 22k - 22 \Rightarrow k^2 - 13k + 42 = 0 \Rightarrow k = 6, 7$.
\item Given, $P_{r + 1}^^{22}:P_{r + 2}^^{20} = 11:52\Rightarrow \frac{22!}{(21 - r)!}.\frac{(18 - r)!}{20!}
  = \frac{11}{54}$

  $\Rightarrow \frac{22.21}{(21 - r)(20 - r)(19 - r)} = \frac{11}{52} \Rightarrow (21 - r)(20 - r)(19 - r) =
  42.52 = 12. 13.14 \Rightarrow r = 7$.
\item Given, $P_2^^{m + n} = 90 \Rightarrow (m + n)(m + n - 1) = 10.9 \Rightarrow m + n = 10$, and

  $P_2^^{m - n} = 30\Rightarrow (m - n)(m - n - 1) = 6.5\Rightarrow m - n = 6 \Rightarrow m = 8, n = 2$.
\item Given, $P_r^^{12} = 11880 \Rightarrow \frac{12!}{(12 - r)!} = 9\times\10\times11\times 12 \Rightarrow
  r = 4$.
\item Given, $P_{r + 6}^^{56}:P_{r + 3}^^{54} = 30800:1 \Rightarrow \frac{56!}{(50 - r)!}.\frac{(51 -
  r)!}{54!} = 30800$

  $\Rightarrow 56\times55\times(51 - r) = 30800 \Rightarrow 51 - r = 10 \Rightarrow r = 41$.
\item $n.P_n^^n = n.n! = (n + 1 - 1).n! = (n + 1)! - n!$. Similarly, $(n - 1).P_{n - 1}^^{n - 1} = n! - (n -
  1)!, \ldots, 2.P_2^^2 = 3! - 2!, 1.P_1^^1 = 2! - 1!$.

  Adding these, we obtain L.H.S. $= (n + 1)! - 1! = P_{n + 1}^^{n + 1} - 1 =$ R.H.S.
\item Given, $C_{30}^^n = C_4^^n \Rightarrow \frac{n!}{30!(n - 30)!} = \frac{n!}{4!(n - 4)!}$

  Equating $n - 30 = 4$ and $n - 4 = 30$, we obtain $n = 34$ from both.
\item Given, $C_{12}^^n = C_8^^n \Rightarrow \frac{n!}{(n - 12)!12!} = \frac{n!}{(n - 8)!8!} \Rightarrow n -
  12 = 8$ and $n - 8 = 12$. Thus, $n = 20$

  $C_{17}^^{20} = \frac{20!}{17!3!} = \frac{20\times19\times18}{3\times2} = 1140$, and
  $C_{20}^^{22} = \frac{22!}{20!2!} = \frac{22\times21}{2} = 231$.
\item Given, $C_r^^{18} = C_{r + 2}^^18 \Rightarrow \frac{18!}{(18 - r)!r!} = \frac{18!}{(r + 2)!(16 - r)!}
  \Rightarrow 18 - r = r + 2 \Rightarrow r = 8$ and $r = 16 - r \Rightarrow r = 8$.

  $C_6^^r = C_6^^8 = \frac{8!}{6!2!} = 28$.
\item Given, $C_{n- 4}^^n = 15 \Rightarrow \frac{n!}{(n - 4)!4!} = 15 \Rightarrow n(n - 1)(n - 2)(n - 3) =
  3\times 4\times 5\times 6\Rightarrow n = 6$.
\item Given, $C_r^^{15}:C_{r - 1}^^15 = 11:5\Rightarrow \frac{15!}{(15 - r)!r!}.\frac{(r - 1)!(16 -
  r!)}{15!} = \frac{11}{5}\Rightarrow \frac{16 - r}{r} = \frac{11}{5}\Rightarrow r = 5$.
\item Given, $P_r^^n = 2520 \Rightarrow \frac{n!}{(n - r)!} = 2520$ and $C_r^^n = 21 \Rightarrow
  \frac{n!}{(n - r)!r!} = 21$

  $\Rightarrow \frac{2520}{r!} = 21 \Rightarrow r! = 120 \Rightarrow r = 5\Rightarrow n(n - 1)(n - 2)(n -
  3)(n - 4) = 2520 = 7\times 6\times 5\times 4\times 3\Rightarrow n = 7$.
\item We know that $C_r^^n = C_{n - r}^^n\Rightarrow C_{13}^^{20} = C_7^^{20}$ and $C_{14}^^{20} =
  C_6^^{20}$.

  $\therefore C_{13}^^{20} + C_{14}^^{20} - C_6^^{20} - C_7^^{20} = 0$.
\item Given, $C_{r - 1}^^n = 36\Rightarrow \frac{n!}{(n - r + 1)(r - 1)!} = 36, C_r^^n = 84 \Rightarrow
  \frac{n!}{(n - r)!r!} = 84$, and $\frac{n!}{(n - r - 1)!(r + 1)!} = 126$.

  Dividing first two, $\frac{r}{n - r + 1} = \frac{3}{7}\Rightarrow 3n = 10r - 3$, and dividing last two

  $\frac{r + 1}{n - r} = \frac{2}{3}\Rightarrow 2n = 5r + 3$. Solving these two equations, we have $n = 9, r
  = 3$.
\item Thoudand's place can be filled in $5$ ways, hundred's place can be filled in $4$ ways, ten's place can
  be filled in $3$ ways and unit's place can be filled in $2$ ways.

  Thus, total number of $4$ digit numbers is $5\times 4\times 3\times 2 = 120$.

  Alternatively, it is $P_4^^5 = 120$.
\item Hundred's place can be filled in $3$ ways excluding $0, 2, 3$, ten's place can be filled in $5$ ways
  and unit's place can be filled in $4$ ways.

  Thus, no. of numbers between $400$ and $1000$ is $5\times 4\times 3 = 60$.
\item {\bf Case I:} When the number is of three digits i.e. between $300$ and $1000$.

  Hundred's place can be filled in $3$ ways using $3, 4$ or $5$, ten's place can be filled in $5$ ways and
  unit's place can be filled in $4$ ways.

  Thus, total no. of three digit numbers is $5\times 4\times 3 = 60$.

  {\bf Case II:} When the number is of four digits i.e. between $1000$ and $3000$.

  Thousand's place can be filled in $2$ ways using $1$ or $2$. Three remaining places can be filled in
  $P_3^^5$ ways i.e. $60$ ways.

  Therefore, total no. of four digit numbers is $2\times 60 = 120$.

  Thus, total no. of numbers between $300$ and $3000$ is $60 + 120 = 180$.
\item {\bf Case I:} When $2$ is at thousands place.

  Hundred's placec can be filled in $4$ ways using $3, 4, 5, 6$. Two remaining places can be filled in
  $P_2^^5$ i.e. $20$ ways. Number of numbers formed in this case is $4\times 20 = 80$.

  {\bf Case II:} When thousands place is occupied by $3, 4, 5$ or $6$.

  We see that there are four ways to fill thousands place. Three remaining placed can be filled in $P_3^^6$
  i.e. $120$ ways. Number of numbers formed in this case is $4\times 120 = 480$.

  Hence, total no. of numbers is $80 + 480 = 560$.
\item {\bf Case I:} When the number is of one digit.

  There will be four positive numbers excluding $0$.

  {\bf Case II}: When the number is of two digits.

  Ten's place can be filled in $4$ ways using $1, 2, 3$ or $4$. Unit's place can be filled in $P_1^^4$
  ways. Total no. of one digit numbers is $4\times P_1^^4 = 16$.

  {\bf Case III:} When the number is of three digits.

  Hundred's place can be filled in $4$ ways like previous case. Remaining two places can be filled in
  $P_2^^4$ ways. Total no. of three digit numbers is $4\times P_2^^4 = 48$.

  {\bf Case IV:} When the number is four digits.

  Thousand's place can be filled in $4$ ways like previous case. Remaining three places can be filled in
  $P_3^^4$ ways. Total no. of four digit numbers is $4\times P_3^^4 = 96$.

  {\bf Case V:} When the number is of five digits.

  Ten thousand's place can be filled in $4$ ways. Remaining four places can be filled in $P_4^^4$
  ways. Total no. of five digit numbers is $4\times P_4^^4 = 96$.

  Thus, total no. of numbers formed is $4 + 16 + 48 + 96 + 96 = 260$.
\item Total no. of numbers will be $P_4^^4 = 24$. Now since there are $4$ digits and $24$ numbers each
  no. will occur at each place for $6$ times. Thus, sum of digits at each place would be $6(1 + 2 + 3 + 4) =
  60$.

  Therefore, sum of all numbers $60(1 + 10 + 100 + 1000) = 66660$.
\item When any digit except $0$ will occupy unit's place the thousand's place has to be occupied by the
  other two digits. Thus, total no. of such numbers is $3\times2\times P_2^^2 = 12$. Thus, $4$ numbers for
  each of positive digits.

  When one of $1, 2, 3$ occupy thousand's place total no. of numbers is $3\times P_3^^3 = 18$. Thus, $6$
  numbers for each of the positive digits.

  Sum of digits at units, tens and thousands place will be $4(1 + 2 + 3) = 24$ and sum of digits at
  thousands place will be $6(1 + 2 + 3) = 36$.

  Thus, sum of numbers formed is $24(1 + 10 + 100) + 36\times 1000 = 38,664$.
\item Each of the four digits $1, 2, 2, 3$ occurs at each place $\frac{P_3^^3}{2!}$ i.e. $3$ times. Thus,
  sum of digits at each place is $3(1 + 2 + 2 + 3) = 24$.

  Thus, sum of numbers formed $24(1 + 10 + 100 + 1000) = 26,664$.
\item Each friennd can be sent invitation by one servant. Since there are three servants each friend can
  receive an invitaion in three ways. Thus, total no. of ways of sending invitations is $3^6 = 729$.
\stopitemize