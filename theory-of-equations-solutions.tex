% -*- mode: context; -*-
\chapter{Polynomials and Theory of Equations}
\startitemize[n, 1*broad]
\item $x + x^9 + x^{25} + x^{49} + x^{81} = x(1 + x^8 + x^{24} + x^{48} + x^{80}) = x[(x^{80} - 1) + (x^{48}
  - 1) + (x^{24 } - 1) + (x^{8} - 1) + 5]$.

  All terms are divisible by $x(x^2 - 1)$ except last term $5x$, and hence, $5x$ is the remainder.
\item Let $P = x^{9999} + x^{8888} + x^{7777} + \cdots + x^{1111} + 1$ and $Q = x^9 + x^8 + x^7 + \cdots + x
  + 1$, then $P - Q = x^9[(x^{10})^{999} - 1] + x^8[(x^{10})^{888} - 1] + \cdots + x[(x^{10})^{100} - 1]$

  But $(x^{10})^n - 1$ is divisible by $x^{10} - 1\;\forall\;n\geq 1. \therefore P - Q$ is divisible by
  $x^{10} - 1$.

  Because $x^9 + x^8 + x^7 + \cdots + x + 1|x^{10} - 1\Rightarrow x^9 + x^8 + x^7 + \cdots + x + 1|P - Q
  \Rightarrow x^9 + x^8 + x^7 + \cdots + x + 1|P$.
\item We will prove this by contradiction. Suppose that $f(n) = 0$, then $f(x - n)$ divides $f(x)$
  i.e. $f(x) = (x - n)g(x)$, where $g(x)$ is another polynomial with integral coefficients. Now $f(1) = (1 -
  n)g(1)$ and $f(2) = (2 - n)g(2)$. Both of these should be odd numbers but that is not possible as $1 - n$
  and $2 - n$ are consecutive integers. Thus, either $f(1)$ or $f(2)$ should be even, which is a
  contradiction, and hence, the result.
\item Suppose that there exists such an integer $b$, such that $f(b) = 1993$. Let $g(x) = f(x) - 1991$. Now,
  $g$ is a polynomial with integer coefficients and $g(a_i) = 0$ for $i = 1, 2, 3, 4$.

  Thus, $(x - a_1), (x - a_2), (x - a_3)$ and $(x - a_4)$ are all factors of $g(x)$. So $g(x) = (x - a_1)(x
  - a_2)(x - a_3)(x - a_4)h(x)$, where $h(x)$ is a polynomial with integer coefficients. $g(b) = f(b) - 1991
  = 2$ so $g(b) = (b - a_1)(b - a_2)(b - a_3)(b - a_4)h(b) = 2$.

  Thus, $(b - a_1)(b - a_2)(b - a_3)(b - a_4)$ are all divisors of $2$ and distinct. Such values are $1, -1,
  -2, 2$ and $h(b)$ is an integer.

  $\therefore g(b) = 4.h(b) = 2$, which is not possible. Hence, such an integer does not exist.
\item We know that when coefficients of a polynomial are integers then quadratic surds as roots appear in
  pairs. Therefore, the other root would be $-\sqrt{5}$ giving us a second degree polynomial $x^2 -
  5$. Therefore, we can write the polynomial is of the form $ax^2 - 5a$.

  {\bf Second method:} Since the order of the surd $\sqrt{5}$ is $2$, we can expect a polynomial of the
  lowest degree to be a polynomial of degree $2$. Let $f(x) = ax^2 + bx + c, a, b,
  c\in\mathbb{Q}$. $f(\sqrt{5}) = 5a + \sqrt{5}a + c = 0$ But $\sqrt{5}$ is irrational so $5a + c = 0$ and
  $b = 0\Rightarrow c = -5a$ so the polynomial is of the form $ax^2 - 5a$ giving us second root at
  $-\sqrt{5}$.
\item Let $f(x) = x - (\sqrt{5} + \sqrt{2}) = [(x - \sqrt{5}) - \sqrt{2}]$. Using conjugate as the other
  zero, we have $f_1(x) = [(x - \sqrt{5}) - \sqrt{2}][(x - \sqrt{5}) + \sqrt{2}] = (x^2 + 3 - 2\sqrt{5}x)$
  $\Rightarrow f_2(x) = [(x^2 + 3) - 2\sqrt{5}x][(x^2 + 3) + 2\sqrt{5}x] = x^4 - 14x^2 + 9\Rightarrow f(x) =
  ax^4 - 14x^2 + 9a$, where $a\in\mathbb{Z}, a\neq 0$.
\item Putting $x = 0, 0 = -f(0)\Rightarrow f(0) = 0$. Putting $x = 1, f(0) = -3f(1) \Rightarrow f(1) =
  0$. Similalrly, $f(2) = f(3) = 0$. Let is assume $f(x) = x(x - 1)(x - 2)(x - 3)g(x)$, where $g(x)$ is some
  polynomial. Now using the given relation we have $x(x - 1)(x - 2)(x - 3)(x - 4)g(x - 1) = x(x - 1)(x -
  2)(x - 3)(x - 4)g(x)$

  $\Rightarrow g(x - 1) = g(x)\;\forall\;x\in\mathbb{R} - \{0, 1, 2, 3, 4\} \Rightarrow g(x - 1) =
  g(x)\;\forall\;x\in\mathbb{R}$ from identity theorem.

  $\Rightarrow g(x)$ is periodic. $\Rightarrow g(x) = c \Rightarrow f(x) = cx(x - 1)(x - 2)(x - 3)$
\item Because $f(x)$ is a monic coefficient of highest degree will be $1$. Let $g(x) = f(x) - x$, where
  $g(x)$ is also a cubic polynomial.

  $g(1) = 0, g(2) = 0, g(3) = 0 \Rightarrow g(x) = (x - 1)(x - 2)(x - 3) \Rightarrow f(x) = (x - 1)(x - 2)(x
  - 3) + x \Rightarrow f(4) = 10$.
\item Let $f(x) = x - (\sqrt{3} + \sqrt{7}) = [(x - \sqrt{3}) - \sqrt{7}]$. Using conjugate as the other
  zero, we have $f_1(x) = [(x - \sqrt{3}) - \sqrt{7}][(x - \sqrt{3}) + \sqrt{7}] = (x^2 - 4 - 2\sqrt{3}x)$
  $\Rightarrow f_2(x) = [(x^2 - 4) - 2\sqrt{3}x][(x^2 - 4) + 2\sqrt{3}x] = x^4 - 8x^2 + 16 - 12x^2 = x^4 -
  20x^2 + 16 = 0$.
\item Clearly, we will have conjugate roots for the given surds as roots, which would be $2 - \sqrt{3}$ and
  $3 - \sqrt{2}$. Therefore, the polynomial would be

  $f(x) = [(x - 2) - \sqrt{3}][(x - 2) + \sqrt{3}][(x - 3) - \sqrt{2}][(x - 3) + \sqrt{2}] = (x^2 - 4x + 4 -
  3)(x^2 - 6x + 9 - 2) = (x^2 - 4x + 1)(x^2 - 6x + 7) = x^4 - 10x^3 + 32x^2 - 34x + 7 = 0$.
\item Let $y = \sqrt[3]{2}$, then $x = y + 3y^2 = y(3y + 1)$. Cubing both sides $x^3 = y^3(27y^3 + 27y^2 +
  9y + 1) = 2(9x + 55)\Rightarrow x^3 - 18x - 110 = 0$. This is the minimal polynomial as
  $[\mathbb{Q}(\sqrt[3]{2}) : Q] = 3$.
\item $x^n - nx + n - 1 = (x - 1)(x^{n- 1} + x^{n - 2} + \cdots + x + 1) - n(x - 1) = (x - 1)[(x^{n - 1} -
  1) + (x^{n - 2} - 1) + \cdots + (x - 1)]$, which clearly has a factor $(x - 1)^2$.
\item Because $a, b, c, d, e$ are all zeroes of the polynomial $6x^5 + 5x^4 + 4x^3 + 3x^2 + 2x + 1$,
  therefore, $6(x - a)(x - b)(x - c)(x - d)(x - e) = 6x^5 + 5x^4 + 4x^3 + 3x^2 + 2x + 1$.

  Putting $x = 1, -6(1 + a)(1 + b)(1 + c)(1 + d)(1 + e) = -6 + 5 - 4 + 3 - 2 + 1 = -3 \Rightarrow (1 + a)(1
  + b)(1 + c)(1 + d)(1 + e) = \frac{1}{2}$.
\item Because $1, \alpha_1, \alpha_2, \ldots, \alpha_{n - 1}$ are the roots of the equation $x^n - 1 = 0$,
  therefore, $(x - 1)(x - \alpha_1)(x - \alpha_2)\cdots(x - \alpha_{n - 1}) = x^{n} - 1 \Rightarrow (x -
  \alpha_1)(x - \alpha_2)\cdots(x - \alpha_{n - 1}) = x^{n - 1} + x^{n - 2} + \cdots + x + 1$.

  Putting $x = 1$, in the above equation, we deduce the desired result.
\item Consider a function $g(x) = f(x) - 10x$, then $g(1) = g(2) = g(3) = 0$ i.e. $(x - 1)(x - 2)(x - 3)$
  would divide $g(x)$. Since $f(x)$ has a degree of $4$ so $g(x)$ will also have a degree of $4$. Let $g(x)
  = (x - t)(x - 1)(x - 2)(x - 3)$ so $f(x) = 10x + (x - t)(x - 1)(x - 2)(x - 3)$.

  Now for $x = 12, (x - 1)(x - 2)(x - 3) = 990$ and for $x = -8, (x - 1)(x - 2)(x - 3) = -990$.

  $\therefore \frac{f(12) + f(-8)}{10} = \frac{10(12 - 8) + (12 - t)990 + (-8 - t).-990}{10} = 1984$.
\item Roots of $x^2 + x + 1$ are $\omega, \omega^2$. Since given polynomial is not divisible by $x^2 + x +
  1$, so these roots won't satisfy the given polynomial. Thus,

  $\omega^{2k} + 1 + (1 + \omega)^{2k} = \omega^{2k} + 1 + (\omega^2)^{2k} = 1 + \omega^k + \omega^{2k}\neq
  0$. We know that $1 + \omega^k + \omega^{2k} = 3$ when $k = 3n, n\in\mathbb{N}$. Hence, $k = 3, 6, 9,
  \ldots$.
\item Putting $x = 1, -7P(2) = 0 \Rightarrow P(2) = 0$. Putting $x = 8, 0 = 56P(8)\Rightarrow P(8) = 0$.

  $\Rightarrow P(x) = (x - 2)(x - 8)Q(x) \Rightarrow P(2x) = (2x - 2)(2x - 8)Q(2x)$

  $\Rightarrow (x - 8)(2x - 2)(2x - 8)Q(2x) = 8(x - 1)(x - 2)(x - 8)Q(x)\Rightarrow \frac{Q(2x)}{Q(x)} =
  \frac{2x - 4}{x - 4} \Rightarrow Q(x) = x - 4 \Rightarrow P(x) = (x - 2)(x - 4)(x - 8)$.
\item If $(x - 1)^3$ divides $f(x) + 1$, then $(x - 1)^2$ divides $f'(x)$ and if $(x + 1)^3$ divides $f(x) -
  1$ then $(x + 1)^3$ divides $f'(x)$. Since we have to find $f(x)$ of degree $5$, $f'(x)$ will be of degree
  $4$. So $f'(x) = k(x - 1)^2(x + 1)^2 = k(x^4 - 2x^2 + 1)$.

  Integrating both sides, $f(x) = K\left(\frac{x^5}{5} - \frac{2x^3}{3} + x\right) + c$, where
  $c\in\mathbb{R}$. Also, $(x - 1)^3$ divides $f(x) + 1\Rightarrow f(1) + 1 = 0 \Rightarrow f(1) = -1$ and
  $(x + 1)^3$ divides $f(x) - 1\Rightarrow f(-1) - 1= 0 \Rightarrow f(-1) = 1$.

  Putting $x = 1$ in the equation for $f(x), \Rightarrow f(1) = K\left(\frac{1}{5} - \frac{2}{3} + 1\right)
  + c= -1$, and putting $x = -1 \Rightarrow f(-1) = K\left(\frac{-1}{5} + \frac{2}{3} - 1\right) +c = 1$.

  From these two equations we deduce $K = -\frac{15}{8}, c = 0$. Thus, our required polynomial is $f(x) =
  -\frac{3}{8}x^5 + \frac{5}{4}x^3 - \frac{15}{8}x$.
\item Since the polynomial equation has rational coefficients the complex roots must appear in conjugate
  pairs. So we have at least two more roots i.e. $3 - 2i$ and $2 - 3i$ making out polynomial equation of at
  least having a degree of $4$. Let us find out the polynomial equation to test if the coefficients with
  these roots are rational.

  $f(x) = a[(x - 3 - 2i)(x - 3 + 2i)][x - 2 - 3i][x - 2 + 3i] = a(x^4 - 10x^3 + 50x^2 - 130x + 169),
  a\in\mathbb{Q}\setminus\{0\}$.
\item Since all the roots are rational, so they are divisors of $-30$. The divisors or $-30$ are $\pm1,
  \pm2, \pm3, \pm5, \pm6, \pm 10, \pm 15,$ and $\pm30$. By applying remainder theorm, we find the roots as
  $-1, -2, -3$ and $5$.
\item Let the roots be of the form $\frac{p}{q}$, where $(p, q) = 1$ and $q > 0$. Since $q\mid 2, q$ must be
  $1$ or $2$ and $p\mid 6\Rightarrow p = \pm 1, \pm 2, \pm 3, \pm 6$.

  Applying remainder theorem, $f\left(\frac{1}{2}\right) = f\left(\frac{-2}{1}\right) =
  f\left(\frac{3}{1}\right) = 0$. So the three roots of the equation are $\frac{1}{2}, -2,$ and $3$.
\item $x^3 - 3x^3 + 5x - 15 = (x^2 + 5)(x - 3) = 0 \Rightarrow x = 3, \sqrt{5}i, -\sqrt{5}i$.
\item Let the roots be of the form $\frac{p}{q}$, where $(p, q) = 1$ and $q > 0$. Since $q\mid 1\Rightarrow
  q = \pm1$, also $p\mid1 \Rightarrow p = \pm 1 \Rightarrow \frac{p}{q} = \pm1$. But $f(\pm1) \neq 0$.

  Hence, the given equation has no real roots.
\item Let $\alpha$ and $\beta$ be the two roots of the given equation, where $\alpha\in\mathbb{Z}$. Then,

  $\alpha + \beta = -a$ and $\alpha\beta = b + 1\Rightarrow \beta = -a -\alpha$ is an integer. Also, since
  $b + 1\neq 0, \beta\neq 0$. From these equations $a^2 + b^2 = (\alpha + \beta)^2 + (\alpha\beta - 1)^2 =
  (1 + \alpha^2)(1 + \beta)^2$. Hence, $a^2 + b^2$ is a composite number.
\item Let $\alpha$ and $\beta$ bet the roots of the given equation, then $\alpha + \beta = p, \alpha\beta =
  p - 1$.

  $(\alpha^2 + \beta^2) = (\alpha + \beta)^2 - 2\alpha\beta = p^2 - 2p + 2 = (p - 1)^2 + 1$. For the sum to
  be minimum $(p - 1)^2$ has to be minimum, which is minimum at $p = 1$.
\item Let $x^3 + ax^2 + bx + c = 0$ be the polynomial, of which $\alpha,\beta$ and $\alpha\beta$ are the
  roots and $a, b$ and $c$ are all rationals.

  From Vieta's relations $\alpha + \beta + \alpha\beta = -a, \alpha\beta + \alpha^2\beta + \alpha\beta^2 =
  b, \alpha^2\beta^2 = -c.$ $b = \alpha\beta(1 + \alpha + \beta) = \alpha\beta(1 - a - \alpha\beta) = (1 -
  a)\alpha\beta - \alpha^2\beta^2 = (1 - a)\alpha\beta + c$. As $a\neq -1, \alpha\beta = \frac{b - c}{1 -
    a}$ and since $a, b, c$ are rational $\alpha\beta$ is rational.

  Note that $a = 1 \Rightarrow 1 + \alpha + \beta + \alpha\beta = 0 \Rightarrow (1 + \alpha)(1 + \beta) = 0
  \Rightarrow \alpha = -1$ or $\beta = -1$, which is not the case.
\item Let the roots be $\alpha, 2\alpha$ and $\beta$, then from Vieta's relations we have $3\alpha + \beta =
  \frac{27}{9} = 3 \Rightarrow \beta = 3(1 - \alpha), 2\alpha^2 + 3\alpha\beta = \frac{26}{9}$ and
  $2\alpha^2\beta = \frac{8}{9}$.

  From first two equations, we get $2\alpha^2 + 3\alpha.3(1 - \alpha) = \frac{26}{9} \Rightarrow \alpha =
  \frac{13}{21}$ or $\frac{2}{3}$. If $\alpha = \frac{13}{21}$ then $beta = \frac{8}{7}$ but then
  $2\alpha^2\beta = 2\times\frac{169}{144}\times\frac{8}{7}\neq \frac{8}{9}$, which is a contradiction.

  So taking $\alpha = \frac{2}{3}\Rightarrow \beta = 1$. Hence, $\alpha + 2\alpha + \beta =3, 2\alpha^2 +
  3\alpha\beta = \frac{26}{9}$ and $2\alpha^2\beta = \frac{8}{9}$. Hence, the roots are $\frac{2}{3},
  \frac{4}{3}$ and $1$.
\item Suppose the roots are $\alpha, \beta, \gamma, \delta$ and $\alpha\beta = 1$. Now $\alpha + \beta +
  \gamma + \delta = \frac{-24}{6} = -4, (\alpha + \beta)(\gamma + \delta) + \alpha\beta + \gamma\delta =
  \frac{31}{4}\Rightarrow (\alpha + \beta)(\gamma + \delta) + \gamma\delta = \frac{27}{4},
  \gamma\delta(\alpha + \beta) + \alpha\beta(\gamma + \delta) = \frac{-3}{2} \Rightarrow \gamma\delta(\alpha
  + \beta) + \gamma + \delta = \frac{-3}{2}, \alpha\beta\gamma\delta = -2\Rightarrow \gamma\delta = -2$.

  From second and fourth equation, we have $(\alpha + \beta)(\gamma + \delta) = \frac{35}{6}$ from third and
  fourth equation, we have $-2(\alpha + \beta) + \gamma + \delta = \frac{-3}{2} \Rightarrow 3(\alpha +
  \beta) = \frac{15}{2} \Rightarrow \alpha + \frac{1}{\alpha} = \frac{5}{2}\Rightarrow \alpha = 2,
  \frac{1}{2}$. Hence, $\beta = \frac{1}{2}, 2$. Now it is trivial to find $\gamma$ and $\delta$, which can
  be found to be $\frac{-1}{2}$ and $4$.
\item Since the coefficients are rational, where $3 + \sqrt{2}$ is a root, so $3 - \sqrt{2}$ is also a
  root. Thus, if two other roots are $\alpha$ and $\beta$, we  have

  $\sigma_1 = \alpha + \beta + 3 + \sqrt{2} + 3 - \sqrt{2} = -(-5) = 5 \Rightarrow \alpha + \beta = -1$.

  $\sigma_2 = (\alpha + \beta)(3 + \sqrt{2} + 3 - \sqrt{2}) + \alpha\beta + (3 + \sqrt{2})(3 - \sqrt{2}) = a
  \Rightarrow 6(\alpha + \beta) + \alpha + \beta + 7 = a \Rightarrow \alpha\beta = a - 1$.

  $\sigma_3 = \alpha\beta(3 + \sqrt{2} + 3 - \sqrt{2}) + (3 + \sqrt{2})(3 - \sqrt{2})(\alpha + \beta) = -b
  \Rightarrow 6\alpha\beta - 7 = b\Rightarrow \alpha\beta = \frac{7 - b}{6}$

  $\sigma_4 = 7\alpha\beta = c \Rightarrow \alpha\beta = \frac{c}{7}$.

  We take $\alpha + \beta = -1, \alpha\beta = k$. $\alpha$ and $\beta$ are roots of the equation $x^2 + x +
  k= 0$. Since the roots of the given equation are real $\Rightarrow 1 - 4k\geq 0 \Rightarrow k\leq
  \frac{1}{4}$. Now for $a, k = a - 1\Rightarrow a\leq \frac{5}{4}$. So the greatest value of $a$ is
  $\frac{5}{4}$. For $b, k = \frac{7 - b}{6} \Rightarrow b\geq \frac{11}{2}$ so least value of $b$ will be
  $\frac{11}{2}$. For $c, k = \frac{c}{7} \Rightarrow c\leq \frac{7}{4}$ So the maximum value of $c$ will be
  $\frac{7}{4}$.

  The two other roots can be found as $-\frac{1}{2}$, which is a repeated root.
\item Let the rational roots be of the form $\frac{p}{q}$, then $q\mid 1 \Rightarrow q = \pm 1$ and $p\mid
  1\Rightarrow p = \pm 1 \Rightarrow \frac{p}{q} = \pm 1$. But we see that $x = -1$ does not satisfy the
  equation so $x = 1$ is the only root.

  {\bf Second method:} You can observe by looking at the coefficients that it is expansion of $(x - 1)^4$ as
  the coefficients are from binomoal theorem. Hence, the root is $1.$
\item Let $\alpha, \beta, \gamma, \delta$ are the roots of the equation, then from Vieta's relations $\alpha +
  \beta + \gamma + \delta = -10$. From question $\alpha + \beta = \gamma +
\delta \Rightarrow \alpha + \beta =
  \gamma + \delta = -5$.

  Let the roots be of the form $\frac{p}{q}$ then $q\mid 1 \Rightarrow q = \pm 1$ and $p\mid 24 \Rightarrow
  p = \pm1, \pm2, \pm3, \pm4, \pm6, \pm8, \pm12, \pm24$. Clearly, $\pm12$ and $\pm24$ are not possible
  values. Testing with other values we find roots as $-1, -2, -3, -4$.
\item Let the rational roots be of the form $\frac{p}{q}$, then $q\mid 6 \Rightarrow q = \pm1, \pm2, \pm3,
  \pm6$ and $p\mid -4 \Rightarrow p = \pm 1, \pm2, \pm4$.

  We find that $-\frac{1}{2}$ and $\frac{4}{3}$ satisfy the given equation and the given equation becomes
  $(2x + 1)(3x - 4)(x^2 + x + 1) = 0$, which has two more roots $\omega, \omega^2$, which are cube roots of
  unity, and are not rational roots.
\item Let the rational roots be of the form $\frac{p}{q}$, then $q\mid 6 \Rightarrow q = \pm1, \pm2, \pm3,
  \pm6$ and $p\mid 2 \Rightarrow p = \pm1, \pm2$. We see that all coefficients are positive so positive
  values of $\frac{p}{q}$ will not satisfy the given equation.

  From negative values we see that only $x = -1$ satisfies the given equation.
\item Let $\alpha, \beta, \gamma$ are the roots of the given equation, then according to the questions
  $\alpha + \beta = 0 \Rightarrow \alpha = -\beta$.

  From Vieta's relations $\alpha + \beta + \gamma = -\frac{a}{4}\Rightarrow \gamma = -\frac{a}{4}, \alpha \beta
  + \beta\gamma + \alpha\gamma = \alpha\beta = -\beta^2 = -\frac{1}{4}\Rightarrow \beta =
  \pm\frac{1}{2} \Rightarrow \alpha = \mp\frac{1}{2}$ and $\alpha\beta\gamma = -\frac{b}{4}\Rightarrow a +
  4b = 0$, where $b\in\mathbb{Q}$.
\item Let the roots be $\alpha, \alpha.r, \alpha.r^2$ be the roots of the given equation, then from Vieta's
  relations, we have

  $\frac{\alpha}{r} + \alpha + \alpha.r = -a, \frac{\alpha^2}{r} + \alpha^2 + \alpha^2.r = b$ and $\alpha^3
  = 8 \Rightarrow \alpha = 2$.

  From first two equations, $\alpha = -\frac{b}{a} = 2 \Rightarrow b = -2a$. Substituting the value of
  $\alpha$ in the first equation, we have

  $2r^2 + (a + 2)r + 2 = 0$, but $r$ is real so $D\geq 0 \Rightarrow a^2 + 4a - 12 = 0 \Rightarrow
  a\in(-\infty, 6)\cup(2, \infty)$.
\item $2x^6 + 12x^5 + 30x^4 + 60x^3 + 80x^2 + 30x + 45 = 2(x^3 + 3x^2)^2 + 12\left(x^2 +
  \frac{5}{2}x\right)^2 + 5(x + 3)^2 = 0$, but it could be zero only if

  $(x^3 + 3x^2) = \left(x^2 + \frac{5}{2}x\right) = x + 3 = 0$.

  The last and first condition simplifies to $x = -3$, but it contradicts the seccond. Thus, given
  polynomial has no real roots.

  {\bf Second method:} Let the roots be of the form $\frac{p}{q}$ then $q\div 2 \Rightarrow q = \pm 1, \pm
  2$ and $p\div 45 \Rightarrow p = \pm1, \pm3, \pm5, \pm9, \pm15, \pm45$. Clearly, the roots have to be
  negative as all coefficients are positive. But none of the combinations of $\frac{p}{q}$ satisfy the given
  equation, hence, it has no real roots.
\item $\sin30^\circ = 3\sin10^\circ - 4\sin^310^\circ \Rightarrow \sin10^\circ$ is a root of $6x - 8x^3 =
  1$. By the rational root theorem, this equation has no rational roots. Therefore, $\sin10^\circ$ is not
  rational. Since $3$ is prime, this equation is the one with least degree having $\sin10^\circ$ as a root.

  {\bf Second Method}: $\sin10^\circ = \cos80^\circ = \cos\frac{4\pi}{9}$. Let $\omega = e^{2i\pi/9}$, then
  $\omega^6 + \omega^3 + 1 = 0$, from which we can calculate that $\omega + \frac{1}{\omega}, \omega^2 +
  \frac{1}{\omega^2}$ and $\omega^4 + \frac{1}{\omega^4}$ are the roots of $x^3 - 3x + 1 = 0$. Since
  $2\cos80^\circ$ is such a root so $8x^3 - 6x + 1 = 0$ is the equation.
\item Following like previous problem $\sin60^\circ = 3\sin20^\circ - 4\sin^320^\circ$. Putting $x =
  \sin20^\circ$ and squaring, $64x^6 - 96x^4 + 36x^2 - 3 = 0$ is the required equation.
\item Following like previous problem $\cos30^\circ = 4\cos^310^\circ - 3\cos10^\circ \Rightarrow
  \frac{\sqrt{3}}{2} = 4\cos^310^\circ - 3\cos10^\circ \Rightarrow 64x^6 - 96x^4 + 36x^2 - 3 = 0$ is the
  required equation.
\item Following like previous problems $\cos60^\circ = 4\cos20^\circ - 3\cos20^\circ \Rightarrow 8x^3 - 6x -
  1 = 0$.
\item Following like previous problems $\tan30^\circ = \frac{3\tan10^\circ - \tan^310^\circ}{1 -
  3\tan^210^\circ} \Rightarrow \frac{1}{\sqrt{3}} = \frac{3x - x^3}{1 - 3x^2}$. Squaring, we get
  $3x^6 - 27x^4 + 33x^2 - 1 = 0$.
\item Following like previous problems $\tan60^\circ = \frac{3\tan20^\circ - \tan^320^\circ}{1 -
  3\tan^220^\circ} \Rightarrow \sqrt{3} = \frac{3x - x^3}{1 - 3x^2}$. Squaring, we get
  $x^6 - 33x^4 + 27x^2 - 3 = 0$.
\item We have found the equations for $\sin10^\circ$ and $\cos20^\circ$ are $8x^3 - 6x + 1 = 0$ and $8x^3 -
  6x - 1 = 0$. Therefore, the equation having these two as roots must be $(8x^3 - 6x + 1)(8x^3 - 6x - 1) = 0
  \Rightarrow 64x^6 - 96x^4 - 36x^2 - 1 = 0$.
\item From Vieta's relations $p + q + r = 6, pq + qr + rp = 3, pqr = -1\Rightarrow p^2 + q^2 + r^2 = 30, p^3
  + q^3 + r^3 = 159, p^3q^3 + q^3r^3 + r^3p^3 = 84$.

  Let $A = p^2q + q^2r + r^2p$ and $B = p^2r + q^2p + r^2q$, then $A + B = 6(p^2 + q^2 + r^2) - (p^3 + q^3 +
  r^3) = 21$ and $AB = -(p^3 + q^3 + r^3)(p^3q^3 + q^3r^3 + r^3p^3) + 3 = 72$.

  Thus, possible value of $A$ are $24, -3$.
\item Let $\alpha, \beta, \gamma, \delta$ be the roots of the given equation such that $\alpha\beta = -32$,
  then from Vietas relations $\alpha + \beta + \gamma + \delta = 18, \alpha\beta + \beta\gamma +
  \gamma\delta + \alpha\gamma + \alpha\delta + \beta\delta = k, \alpha\beta\gamma + \alpha\beta\delta +
  \alpha\gamma\delta + \beta\gamma\delta = -200$ and $\alpha\beta\gamma\delta = -1984$.

  $\therefore \gamma\delta = \frac{\alpha\beta\gamma\delta}{\alpha\beta} = \frac{-1984}{-32} = 62$.

  $\therefore -32 + \beta\gamma + 62 + \alpha\gamma + \alpha\delta + \beta\delta = k \Rightarrow \beta\gamma
  + \alpha\gamma + \alpha\delta + \beta\delta = k - 30$. Let $p = \alpha + \beta$ and $q = \gamma + \delta$.

  $\therefore -200 = -32q + 62p$ and $p + q = 18 \Rightarrow p = 4, q = 14\Rightarrow \frac{\alpha +
    \beta}{2}\frac{\gamma + \delta}{2} = k - 30 \Rightarrow k = 86$.
\item $x^2 + y^2 = 1 - 2xy \Rightarrow (x^2 + y^2)^2 = (1 - 2xy)^2 \Rightarrow x^4 + y^4 = 2x^2y^2 - 4xy + 1
  \Rightarrow 2x^2y^2 - 4xy + 1 - c = 0 \Rightarrow xy = \frac{4 \pm\sqrt{16 + 8c - 8}}{4} =
  1\pm\sqrt{\frac{1 + c}{2}}$

  Now, $x^2 + y^2 = 1 - 2\left(1\pm\sqrt{\frac{1 + c}{2}}\right) = -1\pm \sqrt{2(1 + c)}$,

  and $x^3 + y^3 = (x + y)^3 - 3xy(x + y) = 2\pm\frac{3}{2}\sqrt{2 + 2c}$.
\item Let $x + y = \alpha$ and $xy = \beta$, then $x^2 + y^2 = \alpha^2 - 2\beta$.

  Now, $x^3 + y^3 = (x + y)(x^2 + y^2 - xy) = \alpha(\alpha^2 - 3\beta) = 7\Rightarrow \alpha^3 -
  3\alpha\beta = 7$,

  and $x^2 + y^2 + x + y + xy = 4 \Rightarrow \alpha^2 - 2\beta + \alpha + \beta = 4 \Rightarrow \beta =
  \alpha^2 + \alpha - 4$.

  From these two equations $\alpha^3 - 3\alpha(\alpha^2 + \alpha - 4) = 7\Rightarrow f(\alpha) = 2\alpha^3 +
  3\alpha^2 - 12\alpha + 7 = 0$.

  Since sum of coefficients is zero, therefore, $\alpha = 1$ must be a solution. $\Rightarrow f(1) = 0
  \Rightarrow f(\alpha) = (\alpha - 1)^2(2\alpha + 7) = 0 \Rightarrow \alpha = 1, -\frac{7}{2}$.

  When $\alpha = 1, \beta = -2$ and when $\alpha = -\frac{7}{2}, \beta = \frac{19}{4}$. Thus, when $\alpha
  =1, \beta = -2$ we find that $(x, y)$ is $(-2, 1)$ or $(1, -2)$. But when $\alpha = -\frac{7}{2}$ and
  $\beta = \frac{19}{4}$, then $x, y$ are roots of $4t^2 + 14t + 19 = 0$, whose discriminant is less than
  $0$ and hence no real roots are possible. Thus, value of $x, y$ is $-2, 1$ or $1, -2$.
\item From Vieta's relations $\alpha + \beta + \gamma = \sum \alpha = 0, \alpha\beta + \beta\gamma +
  \gamma\alpha = \sum \alpha\beta = p, \alpha\beta\gamma = \prod\alpha = q$.

  Since $\alpha, \beta, \gamma$ are roots of $x^3 + px + q = 0 \Rightarrow \alpha^3 + p\alpha + q = 0,
  \beta^3 + p\beta + q = 0, \gamma^3 + p\gamma + q = 0$

  Adding these equations, we have $\sum\alpha^3 + p\sum\alpha + 3q = 0 \Rightarrow \sum\alpha^3 =
  -3q[\because \sum\alpha = 0]$

  $\sum\alpha^2 = (\sum\alpha)^2 - 2\sum\alpha\beta = 0^2 - 2p = -2p$.

  Multiplying the given equation by $x^2$, we get $x^5 + px^3 + qx^2 = 0$. Putting $x = \alpha, \beta,
  \gamma$ and adding, we have

  $\sum\alpha^5 + p\sum\alpha^3 + q\sum\alpha^2 = 0 \Rightarrow \sum\alpha^5 = 5pq\Rightarrow
  \frac{1}{5}\sum\alpha^5 = pq = \frac{1}{3}\sum\alpha^3 .
  \frac{1}{2}\sum\alpha^2$.

  Hence, proved.
\item Following like previous problem and using results from previous problem, multiplying the given
  equation by $x$, we have $x^4 + px^2 + qx = 0 \Rightarrow \sum\alpha^4 + p\sum\alpha^2 + q\sum\alpha =
  0\Rightarrow \sum\alpha^4 = -p\sum\alpha^2$.

  Multiplying the given equation by $x^4$, we get $x^7 + px^5 + qx^4 = 0 \Rightarrow \sum\alpha^7 +
  p\sum\alpha^5 + q\sum\alpha^4 = 0\Rightarrow \sum\alpha^7 = -p\sum\alpha^5 -
  q\sum\alpha^4 = -5p^2q +
  pq\sum\alpha^2 = -7p^2q \Rightarrow \frac{\sum\alpha^7}{7} = pq.(-p) =
  \frac{\sum\alpha^5}{5}.\frac{\sum\alpha^2}{2}$

  $\Rightarrow \frac{\alpha^7 + \beta^7 + \gamma^7}{7} = \frac{\alpha^5 + \beta^5 + \gamma^5}{5}\times
  \frac{\alpha^2 + \beta^2 + \gamma^2}{2}$.
\item Since $\alpha + \beta + \gamma = 0$, therefore, $\alpha, \beta, \gamma$ are the roots of the equation
  $x^3 + px + q = 0.\Rightarrow \sum\alpha\beta = p$ and $\prod\alpha = -q$ as shown in previous problems.

  $\alpha^2 + \beta^2 + \gamma^2 = (\alpha + \beta + \gamma)^2 - 2(\alpha\beta + \beta\gamma + \gamma\alpha)
  = 0^2 - 2p = -2p$ and $\sum\alpha^3 = 3\alpha\beta\gamma = -3q$.

  Multiplying $x^3 + px + q = 0$ with $x$, we have $x^4 + px^2 + qx = 0$. Putting $x = \alpha, \beta,
  \gamma$ and adding, we have

  $\sum\alpha^4 + p\sum\alpha^2 + q\sum\alpha = 0 \Rightarrow \sum\alpha^4 = -p\sum\alpha^2 = 2p^2$.

  Similarly, $x^5 + px^3 + qx^2 = 0 \Rightarrow \sum\alpha^5 = -p\sum\alpha^3 -
  q\sum\alpha^2 = -5pq$.

  $\therefore 3(\alpha^2 + \beta^2 + \gamma^2)(\alpha^5 + \beta^5 + \gamma^5) =
  3\times-2p\times-5pq =
  5\times(-3q)\times-2p^2 = 5(\alpha^3 + \beta^3 + \gamma^3)(\alpha^4 + \beta^4 + \gamma^4)$.

  Hence, proved.
\item Suppose that $a^3 + b^3 = c^3 + d^3$ and $a + b = c + d = m$(say), then $(a + b)^3 = (c + d)^3
  \Rightarrow 3ab(a + b) = 3cd(c + d) \Rightarrow ab = cd = n$ (say).

  If $a, b$ are the roots of a quadratic equation, then the equation is $x^2 - mx + n = 0$. But $a + b = m$
  and $ab = n$. So $a$ and $b$ are roots of this equation, and thus, $c$ and $d$ are also the roots of the
  equation. But a quadratic equation can have at most two distinct roots.

  Hence, our supposition is incorrect. Hence, proved.
\item Let $x, y, z$ be the roots of the cubic equation $t^3 - at^2 + bt - c = 0$, then $x + y + z = a, xy +
  yz + zx = b \Rightarrow 2xy + 2yz + 2zx = 2b = (x + y + z)^2 - (x^2 + y^2 + z^2) = 9 - 3 \Rightarrow b =
  3$.

  Substituting $x, y, z$ in our equation and adding, we get $(x^3 + y^3 + z^3) - a(x^2 + y^2 + z^2) + b(x +
  y + z) -3c = 0 \Rightarrow c = 1$.

  Thus, our equation becomes $t^3 - 3t^2 + 3t - 1 = 0 \Rightarrow (t - 1)^3 = 0$, thus roots are $1, 1,
  1$. And hence, $x = y = z = 1$.
\item $xy + yz + zx = \frac{1}{2}[(x + y + z)^2 - (x^2 + y^2 + z^2)] = 2$.

  We know that $x^3 + y^3 + z^3 - 3xyz = (x + y + z)(x^2 + y^2 + z^2 - xy - yz - zx)\Rightarrow xyz =
  -\frac{2}{3}$.

  $x^4 + y^4 + z^4 = (x^2 + y^2 + z^2)^2 - 2[(xy)^2 + (yz)^2 + (zx)^2] = 25 - 2[(xy + yz + zx)^2 - 2(xy^2z +
    zxy^2 + xyz^2)] = 25 - 2[4 - 2xyz(x + y + z)] = 9$.
\item From question $\alpha + \beta = a + d$ and $\alpha\beta = ad - bc$.

  $\alpha^3 + \beta^3 = (\alpha + \beta)^3 - 3\alpha\beta(\alpha + \beta) = (a + d)^3 - 3(ad - bc)(a + d) =
  a^3 + d^3 + 3a^2d + 3ad^2 - 3a^2d - 3ad^2 + 3abc + 3bcd = a^3 + d^3 + 3abc + 3bcd$ and $\alpha^3\beta^3 =
  (ad - bc)^3$.

  Thus, equation whose roots are $\alpha^3$ and $\beta^3$ is $x^2 - (a^3 + d^3 + 3abc + 3bcd)x + (ad - bc)^3
  = 0$.
\item $a^3 + b^3 + c^3 = a^3 + b^3 + c^3 + 3(a + b)(b + c)(c + a)\Rightarrow (a + b)(b + c)(c + a) = 0$,
  which implies that one of $a + b, b + c, c + a = 0$.

  In any case $a^{2n + 1} + b^{2n + 1} + c^{2n + 1} = (a + b + c)^{2n + 1}\;\forall\;n\in\mathbb{N}$ and for
  $n = 2, a^5 + b^5 + c^5 = (a + b + c)^5$.
\item We know that $p^3 + q^3 + r^3 = (p + q + r)[(p + q + r)^2 - 3(pq + qr + rp)] + 3pqr$.

  From Vieta's relations, we have $p + q + r = 1, pq + qr + rp = 1$ and $pqr = 2$, therefore,

  $p^3 + q^3 + r^3 = 1[1 - 3] + 6 = 4$.
\item Let $a, b, c$ be the roots of the equation, then from Vieta's relations $a + b + c = 0, ab + bc + ca =
  3, abc = -9$.

  Given equation is $x^3 + 3x + 9 = 0$, putting $x = a, b, c$, and adding $a^3 + b^3 + c^3 + 3(a + b + c) +
  27 = 0 \Rightarrow a^3 + b^3 + c^3 = -27$.

  Multiplying given equation with $x^2$, putting $x = a, b, c$, and adding $a^5 + b^5 + c^3 + 3(a^3 + b^3 +
  c^3) + 9(a^2 + b^2 + c^2) = 0$

  $\Rightarrow a^5 + b^5 + c^5 = 81 - 9[(a + b + c)^2 - 2(ab + bc + ca)] = 81 + 18\times 3 = 135$.
\item Let $a, b, c$ are the roots of the equation $x^3 - 7x^2 + 4x - 3 = 0$, then from Vieta's relations $a
  + b + c = 7, ab + bc + ca = 4,$ and $abc = 3$.

  $a^2 + b^2 + c^2 = (a + b + c)^2 - 2(ab + bc + ca) = 7^2 - 2.4 = 41$.

  Putting $x = a, b, c$ and adding, we get $a^3 + b^3 + c^3 = 7(a^2 + b^2 + c^2) - 4(a + b + c) + 9 =
  7\times41 - 4\times7 + 9 = 287 - 28 + 9 = 268$.

  Multiplying given equation with $x$, putting $x = a, b, c$, and adding $a^4 + b^4 + c^4 = 7(a^3 + b^3 +
  c^3) - 4(a^2 + b^2 + c^2) + 3(a + b + c) = 7\times268 - 4\times41 + 3\times7 = 1733$.

  Multiplying given equation with $x^2$, putting $x = a, b, c$, and adding $a^5 + b^5 + c^5 = 7(a^4 + b^4 +
  c^4) - 4(a^3 + b^3 + c^3) + 3(a^2 + b^2 + c^2) = 7\times1733 - 4\times268 + 3\times41 = 11182$.
\item From Vieta's relations we have $\alpha + \beta + \gamma = 0, \alpha\beta + \beta\gamma + \gamma\alpha
  = -9, \alpha\beta\gamma = -9$.

  Putting $x = \frac{1}{\alpha}, \frac{1}{\beta}$ and $\frac{1}{\gamma}$ in the given equation and adding,
  we have $\alpha^{-3} + \beta^{-3} + \gamma^{-3} = 9\left(\frac{1}{\alpha} + \frac{1}{\beta} +
  \frac{1}{\gamma}\right) - 27 = 9\frac{\alpha\beta + \beta\gamma + \gamma\alpha}{\alpha\beta\gamma} - 27 =
  -18$.

  Multiplying the given equation by $x^2$ and putting $x = \frac{1}{\alpha}, \frac{1}{\beta},
  \frac{1}{\gamma}$, and adding

  $\alpha^{-5} + \beta^{-5} + \gamma^{-5} = 9(\alpha^{-3} + \beta^{-3} + \gamma^{-3}) - 9(\alpha^{-2} +
  \beta^{-2} + \gamma^{-2}) = \frac{4}{9}$.
\item Let the cubic equation be $x^3 + ax^2 + bx + c = 0$, then from Vieta's relation $a = -(\alpha + \beta
  + \gamma) = -9$. We also have $\alpha\beta + \beta\gamma + \gamma\alpha = \frac{(\alpha + \beta +
    \gamma)^2 - (\alpha^2 + \beta^2 + \gamma^2)}{2} = \frac{81 - 29}{2} = 26$ and hence $b = \alpha\beta +
  \beta\gamma + \gamma\alpha = 26$.

  Putting $x = \alpha, \beta, \gamma$ in the given equation, and adding

  $(\alpha^3 + \beta^3 + \gamma^3) - 9(\alpha^2 + \beta^2 + \gamma^2) + 26(\alpha + \beta + \gamma) + 3c = 0
  \Rightarrow c = -24$, and hence, our equation is $x^3 - 9x^2 + 26x - 24 = 0$

  Multiplying the given equation with $x$, putting $x = \alpha, \beta, \gamma$ and adding, we have

  $\alpha^4 + \beta^4 + \gamma^4 = 9(\alpha^3 + \beta^3 + \gamma^3) - 26(\alpha^2 + \beta^2 + \gamma^2) +
  9(\alpha + \beta + \gamma) = 353$.
\item Let the cubic equation be $x^3 + ax^2 + bx + c = 0$, then from Vieta's relation $a = -(\alpha + \beta
  + \gamma) = -4$. We also have $\alpha\beta + \beta\gamma + \gamma\alpha = \frac{(\alpha + \beta +
    \gamma)^2 - (\alpha^2 + \beta^2 + \gamma^2)}{2} = \frac{16 - 7}{2} = \frac{9}{2}$ and hence $b =
  \alpha\beta + \beta\gamma + \gamma\alpha = \frac{9}{2}$.

  Putting $x = \alpha, \beta, \gamma$ in the given equation, and adding

  $(\alpha^3 + \beta^3 + \gamma^3) - 4(\alpha^2 + \beta^2 + \gamma^2) + \frac{9}{2}(\alpha + \beta + \gamma)
  + 3c = 0
  \Rightarrow 3c = -28 + 4\times7 -\frac{9}{2}4 = -18\Rightarrow c = -6$.

  Now we can multiply the given equation with $x$ and $x^2$ and put $x = \alpha, \beta, \gamma$ and add to
  find $\alpha^4 + \beta^4 + \gamma^4$ and $\alpha^5 + \beta^5 + \gamma^5$ as $\frac{209}{2}$ and $334$.
\item $(x + y + z)^2 = x^2 + y^2 + z^2 + 2(xy + yz + zx) \Rightarrow a^2 = a^2 + + 2(xy + yz + zx)
  \Rightarrow xy + yz + zx = 0$

  $x^3 + y^3 + z^3 - 3xyz = (x + y + z)(x^2 + y^2 + z^2 - xy - yz - zx) \Rightarrow a^3 - 3xyz = a(a^2 - 0)
  \Rightarrow xyz = 0 \Rightarrow x = 0$ or $y = 0$ or $z = 0$.

  If $x = 0 \Rightarrow y + z = a$ and $y^2 + z^2 = a^2 \Rightarrow (a - z)^2 + z^2 = a^2 \Rightarrow 2z^2 -
  2za = 0 \Rightarrow z = 0, a\Rightarrow y = a, 0$.

  When $y = 0, z = 0$, then $x = a$. Thus the solution is $(x, y, z) = (a, 0, 0), (0, a, 0), (0, 0, a)$.
\item On multiplying given equation with $x - 1$ we have $x^3(x - 1) + (x - 1)(x^3 + x^2 + x + 1) = 0
  \Rightarrow 2x^4 - x^3 - 1 = 0 \Rightarrow \frac{1}{x^3} = 2x - 1$. Thus, required equation becomes

  $E = (2\beta + 2\gamma - 2\alpha - 1)(2\beta + 2\alpha - 2\gamma - 1)(2\alpha + 2\gamma - 2\beta - 1)$.

  From Vieta's relations $\alpha + \beta + \gamma = -\frac{1}{2}$. So the expression becomes

  $E = -8(2\alpha + 1)(2\beta + 1)(2\gamma + 1) = -16$.
\item Observe that $x = 2, y = 3$ or $x = 3, y = 2$ are two possible roots. Then,

  $(x^2 - 5x + 19)(x - 2)(x - 3) = 0$ so the roots of $x^2 - 5x + 19 = 0$ are $\frac{5\pm\sqrt{51}i}{2}$
  complex conjugates.
\item Let $\sqrt[4]{97 - x} = a$ and $\sqrt[4]{x} = b$, then $a + b = 5$ and $a^4 + b^4 = 97$.

  Now, $a^4 + b^4 = (a^2 + b^2)^2 - 2a^2b^2 = [(a + b)^2 - 2ab]^2 - 2a^2b^2 = 625 - 100ab + 2a^2b^2 = 97$

  $\Rightarrow (ab - 25)^2 = 361 \Rightarrow ab = 44$ which is impossible and $ab = 6$ which gives $x$ as
  $16, 81$.
\item The HCF of given polynomials is $x^2 - 2$, and hence, the common roots of the given polynomials are
  the roots of $x^2 - 2 = 0$ i.e. $\pm\sqrt{2}$.
\item The HCF is $4(x^2 - 5x + 6)$ and hence the common roots are $x = 2, 3$. If two other roots of first
  equation are $\alpha$ and $\beta$, then $\alpha + \beta + 5 = -5$ and $6\alpha\beta = 132 \Rightarrow
  \alpha\beta = 22$.

  Therefore, the having $\alpha, \beta$ as roots is $x^^2 + 10x + 22 = 0$, whose roots are $-5\pm\sqrt{3}$.

  Similarly, let $\alpha_1$ and $\beta_1$ be two other roots of the second equation then $\alpha_1 + \beta_1
  + 5 = -1$ and $6\alpha_1\beta_1 = 24 \Rightarrow \alpha_1\beta_1 = 4$. Thus, $\alpha_1$ and $\beta_1$ are
  roots of $x^2 + 6x + 4 = 0$, whose roots are $-3\pm\sqrt{5}$.
\item If $k = 1, p_1(x) = x^9 + x^3 + x^2 + x + 1 = x^9 - x^4 + x^4 + x^3 + x^2 + x + 1 = x^4(x^5 - 1) +
  (x^4 + x^3 + x^2 + x + 1)$

  $= (x^4 + x^3 + x^2 + x + 1)[x^4(x - 1) + 1]$. Thus, $x^4 + x^3 + x^2 + x + 1$ is a non-tirvial polynomial
  divisor of $p_1(x)$.

  $p_k(x) = x^4(x^{5k} - 1) + x^4 + x^3 + x^2 + x + 1$. $x^5 - 1$ divides $x^{5k} - 1, x^4 + x^3 + x^2 + x +
  1$ divides $x^5 - 1$, and hence $x^{5k} - 1$. Therefore, $x^4 + x^3 + x^2 + x + 1$ divides $p_k(x)$ for
  all $k$.
\item The HCF of given equations is $x + 2$, and hence, common root is $-2$.
\item The HCF of given equations is $x^2 - 4$, and hence, common roots are $2, -2$.
\item $\because d, e, f$ are in G.P. $\therefore df = e^2$. Discriminant of second equation is $D = 4e^2 -
  4fd = 4e^2 - 4e^2 = 0$.

  Thus, second equation will have one, repeated root $x = -\frac{2e}{2d} = -\frac{e}{d}$. This is the common
  root with first equation. Thus,

  $\frac{ae^2}{d^2} - \frac{2be}{d} + c = 0 \Rightarrow \frac{adf}{d^2} - \frac{2be}{d} + c = 0 \Rightarrow
  \frac{af}{d} - \frac{2be}{d} + c = 0 \Rightarrow \frac{af}{d} + c = \frac{2be}{d} \Rightarrow \frac{a}{d}
  + \frac{c}{f} = \frac{2be}{df} = \frac{2b}{e}$.

  Thus, $\frac{d}{a}, \frac{e}{b}, \frac{f}{c}$ are in H.P.
\item $x^2 + px + q = (x - \alpha)(x - \beta) \Rightarrow \alpha + \beta = -p$ and $\alpha\beta = q$.

  $x^{2n} + p^nx^n + q^n = (x - \alpha^n)(x - \beta^n) \Rightarrow \alpha^n + \beta^n = -p^n$ and
  $\alpha^n\beta^n = q^n$

  $f\left(\frac{\alpha}{\beta}\right) = \frac{\left(1 + \frac{\alpha^n}{\beta^n}\right)}{1 +
  \frac{\alpha^n}{\beta^n}} = \frac{(\alpha + \beta)^n}{\alpha^n + \beta^n} = \frac{(-p)^n}{-p^n} = -1$.
\item Over $\mathbb{Q}: x^4 + 4 = x^4 + 4x^2 - 4x^2 + 4 = (x^2 + 2)^2 - (2x)^2 = (x^2 + 2x + 2)(x^2 - 2x +
  2)$. Over $\mathbb{R}$ it is same.

  Over $\mathbb{C}$. We need further factorization of $x^2 + 2x + 2$ and $x^2 - 2x + 2. x^2 + 2x + 2 = 0
  \Rightarrow x = -1 \pm i$ and $x^2 - 2x + 2 = 0 \Rightarrow x = 1 \pm i$.

  Thus, $x^4 + 4 = (x + 1 - i)(x + 1 + i)(x - 1 - i)(x - 1 + i)$.
\item $x^4 + x^3 - x - 1 = x^3(x + 1) - (x + 1) = (x^3 - 1)(x + 1) = (x - 1)(x + 1)(x^2 + x + 1)$. Hence, it
  is reducible over $\mathbb{Z}$.
\item As it is a cubic polynomial, if this is reducible then it would have to have a linear factor $x -
  \alpha$, hence a root ($\alpha\in\mathbb{Z}$). But by integer root theorem $\alpha$ would have been an
  integer divisor of constant $3$, hence it would have to be $1, -1, 3$ or $-3$, however, none of these is a
  root, and hence the polynomial is irreducible.
\item Following like previous problem using integer root theorem we have no integral roots, and hence, no
  linear factors. However, it might be a product of two quadratics. Consider:

  $x^4 + x^3 - x + 1 = (x^2 + ax + b)(x^2 + cx + d)$. Now equating coefficients, $a + c = 1, b + ac + d = 0,
  ad + bc = -1, bd = 1$. Since $a, b, c, d$ are all integers, we have either $b = d = 1$ or $b = d = -1$.

  In the first case the other equations become $a + c = 0, ac = -2, a + c = -1$, which is impossible. And in
  the second case we obtain $a + c = 1, ac = 2$ which has no integer solution. Thus, there is no
  factorization, and the poynomial is irreducible.
\item Suppose $f$ can be faactored then $f(x) = (x - n)g(x)$ or $f(x) = (x^2 - bx + c)g(x)$.

  In the first case, $f(n) = n^5 - n + a$. Now $n^5\equiv n(\mod 5)$ by Fermat's little theorem $5\mid(b -
  b^5) = a$, contradiction.

  In the second case, $f(x) = x^5 - x + a$ by $x^2 - bx + c$, we get the remainder $(b^4 + 3b^2c + c^2 -
  1)(b^3c + 2bc^2 + a)$. Since $x^2 - bx + c$ is a factor of $f(x)$, both coefficients of remainder equal to
  $0$. That is $b^4 + 3b^2x + c^2 - 1= 0$ and $b^3c + 2bc^2 + a = 0\Rightarrow b(b^4 + 3b^2x + c^2 - 1) -
  3(b^3c + 2bc^2 + a) = b^5 - b - 5bc^2 - 3a = 0\Rightarrow 3a = b^5 - b - 5bc^2$ is divisible by
  $5\Rightarrow 5\mid a$, which is a contradiction.
\item Let $\alpha$ be any complex zero of $f$.

  {\bf Case I:} Consider $|\alpha|\leq 1$, then $|a_0| = |a_1\alpha + \cdots + a_n\alpha^n|\leq |a_1| +
  \cdots + |a_n|$, which is a contradiction.

  {\bf Case II:} Therefore, all the zeros of $f$ satisfy the condition $|\alpha| > 1$. Let us assume that
  $f(x) = g(x)h(x)$, where $g$ and $h$ are non-constant integer polynomials. Then $a_0 = f(0) =
  g(0)h(0)$. Since $a_0$ is a prime, one of $|g(0), h(0)|$ equals $1.$. Say $|g(0)| = 1$, and let $b$ be the
  leading coefficient of $g$.

  Let $\alpha_1, \alpha_2, \ldots, \alpha_k$ are the roots of $g$, then $|\alpha_1\alpha_2\ldots\alpha_k| =
  \frac{1}{|b|}\leq 1(\because b\in\mathbb{Z}- \{0\}\Rightarrow |b|\geq 1)$.

  But $\alpha_1, \alpha_2, \ldots,\alpha_k$ are also zeroes of $f$, and from case $1$ have magnitude of each
  $\alpha_i\geq 1\Rightarrow |\alpha_1\alpha_2\ldots\alpha_k|\geq 1$, which is a contradiction.

  Hence, $f$ is irreducible.
\item The given polynomial is irreducible by Eisenstein's criterion with $7$ being the prime $p$. $7$ does
  not divide the leading coefficient but it divides all others, and its square $49$, does not divide
  $175$. Note that using prime $5$ is not valid because $25$ divides the constant coefficinet $175$.
\item Let $f(x) = x^3 - 3x^2 + 3x + 22$. Eisenstein's criteria does not apply since there is no suitable
  prime. Substituting $x - 1$ for $x$ gives the polynomial $x^3 - 6x^2 + 6x + 21$ to which we can apply
  Eisenstein's criteria with $p = 3$. Writing $f(x)$ for the original polynomial, we deduce that $f(x - 1)$
  is irreducible. But a factorization of $f(x)$ would give a factorization of $f(x - 1)$, hence $f(x)$ is
  irreducible over $\mathbb{Z}$.
\item $\Phi_p(x) = x^{p - 1} + x^{p - 2} + \cdots + x + 1 = \frac{x^p - 1}{x - 1}$.

  Consider $\Phi_p(x + 1) = \frac{(x + 1)^p - 1}{x + 1 - 1} = \frac{x^p + C_1^^px^{p - 1} + C_2^^px^{p - 1} +
    \cdots + C_{p - 2}^^px^2 + C_{p - 1}^^p{x}}{x} = x^{p - 1} + C_1^^px^{p - 1} + \cdots + C_{p - 2}^^px + C_{p - 1}^^p$.

  As $p\mid C_i^^p\;\forall\; i = 1, 2, 3,\ldots, p - 1$, so all the lower coefficients are divisible by $p$ and
  the constant coefficient is exactly $p$, so it is not divisible by $p^2$. Thus, Eisenstein's criteria apply,
  and $\Phi_p(x + 1)$ is irreducible. Certainly if $\Phi_p(x) = g(x)h(x)$ then $\Phi_p(x + 1) = g(x + 1)h(x
  + 1)$ gives a factorization of $\Phi_p(x + 1)$. Thus, $\Phi_p$ is irreducible.
\item Rewrite the given polynomial as $f(x) = x^n + 5x^{n - 1} + 0.x^{n - 2} + 0.x^{n- 3} + \cdots + 0.x + 3$.
  Taking prime $p = 3$, clearly $3\mid a_i\;\forall\; i = 0, 1, 2, \ldots,  n - 2; 3^2\nmid a_0 = 3,
  3\nmid a_{n - 1} = 5$. Hence, by extended Eisenstein's criterion $f$ has an irreducible factor of degree
  at least $n - 1$. If possible, let us take one factor of degree $n - 1$ then other must be linear and
  monic as $f$ is monic. This implies that $f$ has integral roots. By integer root theorem this root must be
  an integer divisor of $3$, hence would have to be $1, -1, 3$ or $-3$. However, none of these are roots of
  the given equation, and hence, $f$ is irreducible.
\item We treat the polynomial as $f(x) = a_nx^n + a_{n - 1}x^{n - 1} + \cdots + a_1x + a_0$, where $a_n = 0,
  a_i = 0\;\forall\;i\in\{1, 2, \ldots, n - 1\}$ and $a_0 = -p$.

  Clearly if we consider the prime $p$ given in the polynomial then $p\mid a_i$ for $0\leq i\leq n - 1,
  p\nmid 1$ and $p^2\nmid -p$. Thus, $f(x)$ is irreducible over $\mathbb{Z}$.
\item Considering prime $p = 3$, we  have $p\mid a_i$ for $0\leq n - 1, p\nmid 1$ and $p^2\nmid24$. Thus,
  the polynomial is irreducible over $\mathbb{Z}$.
\item Considering prime $p = 2$, we  have $p\mid a_i$ for $0\leq n - 1, p\nmid 1$ and $p^2\nmid2$. Thus,
  the polynomial is irreducible over $\mathbb{Z}$, which implies that it cannot be represented as product of
  two given polynomials.
\item Considerung prime $p = 3$, we  have $p\mid a_i$ for $0\leq n - 1, p\nmid 1$ and $p^2\nmid2$. Thus,
  the polynomial is irreducible over $\mathbb{Z}$.
\item There is no suitable prime for $x^3 + 3x^2 + 3x + 5$. Substituting $x - 3$ for $x$ gives the
  polynomial $x^3 - 6x^2 + 14x - 10$ to which Eisenstein does apply, with $p = 2$. Writing $f(x)$ for the
  original polynomial, we deduce that $f(x - 3)$ is irreducible. But a factorization of $f(x)$ would give a
  factorization of $f(x - 3)$, hence $f(x)$ is irreducible over $\mathbb{Z}$.
\item We see that our prime $p$ will divide the coefficient of $x$ but it won't divide $p - 1$ for $p >
  2$ making the prime $p$ unsuitable for Eisenstein criteria. However, if $p = 2$ the the polynomial is $x^2
  + 2x + 1 = (x + 1)^2$. So if the polynomial has to be
  reducible for some prime then it must be $2$.
\item Given polynomial is irreducible over $\mathbb{Z}$. Substituting $x = \frac{1}{x}$ we obtain the
  desired polynomial and find that it is irreducible over $\mathbb{Z}$.

  If we substitute $x = \frac{1}{x}$ in $21x^5 - 49x^3 + 14x^2 - 4$ then it becomes $4x^5 - 14x^3 + 49x^2 -
  21$ for which Eisenstein's ceriteria is satisfied for $p = 7$.
\item If the polynomial were reducible over $\mathbb{Z}$, then there would exist two monic polynomials
  $P(x)$ and $Q(x)$ such that $P(x)Q(x) = (x - a_1)(x - a_2)\cdots(x - a_n) - 1$. Consequently,
  $P(a_i)Q(a_i) = -1\;\forall i\in\{1, 2, \ldots, n\}$ but $P(a_i)$ and $Q(a_i)$ are integers, so there are
  only two possibilities:

  $P(a_i) = 1, Q(a_i) = -1$ and $P(a_i) = -1, Q(a_i) = 1$. In any case, $P(a_i) + Q(a_i) = 0$, but it is
  impossible because $P(x) + Q(x)\neq 0$ (sum of monic polynomials) has degree less than $n$, so according
  to fundamental theorem of algebra, it cannot have $n$ different roots. Hence, there do not exist such
  polynomials making the given polynomial irreducible.
\item This problem is similar to $81$ and can be solved similarly.
\item $x = \frac{p - 2}{2p(p - 2)}$. If $p = 0$ or $p = 2$ then the equation is undefined. However, if $p =
  0$, then the equation becomes $0 = -2$, which is inconsistent. Hence, no value of $x$ will satisfy it and
  there is no solution for $p = 0$.

  If $p = 2$ then the equation becomes $0 = 0$. Thus, every value from the domain of $x$ will satisfy the
  equation, and hence, there exists infinite number of solutions.

  If $p\neq 0, p\neq 2$, then the equation is well defined and $x = \frac{1}{2p}$.
\item Substituting the roots we have $ax_1^2 + bx_1 + c = 0, -ax_2^2 + bx_2 + c = 0$ and $f(x_1) =
  \frac{a}{2}x_1^2 + bx_1 + c, f(x_2) = \frac{a}{2}x_2^2 + bx_2 + c$.

  $\therefore f(x_1) + \frac{a}{2}x_1^2 = ax_1^2 + bx_1 + c = 0$ and $f(x_2) - \frac{3}{2}ax_2^2 = -ax_2^2
  + bx_2 + c = 0$

  $\because f(x_1)$ and $f(x_2)$ have opposite signs, and hence, $f(x)$ must have a root between $x_1$ and
  $x_2$.
\item Let $P(x) = x^2 + ax + b = (x - \alpha)(x - \beta)$, where $\alpha + \beta = -a$ and $\alpha\beta =
  b$.

  Now $P(n)P(n + 1) = (n - \alpha)(n - \beta)(n + 1 - \alpha)(n + 1 - \beta) = (n - \alpha)(n + 1 - \beta)(n
  - \beta)(n + 1 - \alpha) = [n^2 - (\alpha + \beta - 1)n + \alpha\beta - \alpha][n^2 + (\alpha + \beta -
    1)n + \alpha\beta - b] = [n^2  + (a + 1)n + b - \alpha][n^2 + (a + 1)n + b - \beta] = (M - \alpha)(M -
  \beta)$, where $M = n^2 + (a + 1)n + b$.
\item Let there be a rational root $\frac{p}{q}$, where $(p, q) = 1$. Then, $a\frac{p^2}{q^2} + b\frac{p}{q}
  + c = 0 \Rightarrow ap^2 + bpq + cq^2 = 0$

  Now $p, q$ both may be odd or one of the $p, q$ be even. In both the cases $ap^2 + bpq + cq^2$ cannot be
  equal to zero. Thus, the equation cannot have rational roots.
\item Given, $\frac{1}{a} + \frac{1}{b} + \frac{1}{c} = \frac{1}{a + b + c} \Rightarrow \frac{1}{a} +
  \frac{1}{b} = \frac{1}{a + b + c} - \frac{1}{c}\Rightarrow (a + b)(b + c)(c + a) = 0\Rightarrow a = -b$ or
  $b = - c$ or $c = -a$.

  If $a = -b$, then $a^n = -b^n$ for odd $n \Rightarrow \frac{1}{a^n} = - \frac{1}{b^n}\Rightarrow
  \frac{1}{a^n} + \frac{1}{b^n} + \frac{1}{c^n} = \frac{1}{a^n + b^n + c^n}$. Similarly, it can be proved
  for other two cases as well.
\item We have $\frac{a^3}{(a - b)(a - c)} = -\frac{a^3}{(a - b)(c - a)}, \frac{b^3}{(b - a)(b - c)} =
  -\frac{b^3}{(a - b)(b - c)}$ and $\frac{c^3}{(c - a)(c - b)} = -\frac{c^3}{(c - a)(b - c)}$.

  $\displaystyle\frac{a^3}{(a - b)(a - c)} = \left[\frac{(b - c)a^3 + (c - a)b^3 + (a - b)c^3}{(a - b)(b -
      c)(c - a)}\right]$

  Numerator of RHS is a cyclic symmetric expression in $a, b, c$ in $4$th degree and writting $b = c$, we
  get $(c - a)b^3 + (a - b)c^3 = 0$. So $b - c$ and hence $c - a$ and $a - b$ are factors. Since it is a
  $4$th degree symmetric expression $a + b + c$ is also a factor. Thus, we have

  $k(a + b + c)(a - b)(b - c)(c - a) = (b - c)a^3 + (c - a)b^3 + (a - b)c^3$.

  If $a = 1, b = -1$ and $c = 2$, we get $k = -1$.

  Thus, the expression $\frac{(a + b + c)(a - b)(b - c)(c - a)}{(a - b)(b - c)(c - a)} = a + b + c$.
\item $x^n - a_1x^{n-1} - \cdots - a_{n - 1}x - a_n = 0\Rightarrow -x^n\left[-1 + \frac{a_1}{x} +
  \frac{a_2}{x^2} + \cdots + \frac{a_n}{x^n}\right] = 0$.

  Let $f(x) = \frac{a_1}{x} + \frac{a_2}{x^2} + \cdots + \frac{a_n}{x^n}. f(x)$ is a decreasing function as
  $x$ increases in $(0, \infty), f(x)$ decreases in $(\infty, 0)$. Hence there exists a unique positive real
  number $R$ such that $f(R) = \frac{a_1}{R} + \frac{a_2}{R^2} + \cdots + \frac{a_n}{R^n} = 1$. Thus, for $x
  = R$, we get

  $-R^n\left[-1 + \frac{a_1}{R} + \frac{a_2}{R^2} + \cdots + \frac{a_n}{R^n}\right] = 0$. Therefore, $R$ is
  a root of the given equation.
\item Considering the polynomial $\pm P(\pm x)$ we may assume without loss of generality that $a, b\geq 0$.

  {\bf Case I:} If $c, d\geq 0$, then $|a| + |b| + |c| + |d| = P(1)\leq 1 < 7$

  {\bf Case II:} If $c\geq 0$ and $d\leq 0$, then $|a| + |b| + |c| + |d| = a + b + c - d = (a + b + c + d) -
  2d = P(1) - 2P(0) \leq 1 + 2 = 3 < 7$

  {\bf Case III:} If $c < 0$ and $d \geq 0$, then $|a| + |b| + |c| + |d| = a + b - c + d = \frac{4}{3}P(1) -
  \frac{1}{3}P(-1) - \frac{8}{3}P\left(\frac{1}{2}\right) + \frac{8}{3}P\left(-\frac{1}{2}\right)\leq
  \frac{4}{3} + \frac{1}{3} + \frac{8}{3} + \frac{8}{3} = 7$

  {\bf Case IV:} If $c, d < 0$, then $|a| + |b| + |c| + |d| = a + b - c - d = \frac{5}{3}P(1) -
  4P\left(\frac{1}{2}\right) + \frac{4}{3}P\left(-\frac{1}{2}\right) \leq \frac{5}{3} + 4 + \frac{4}{3} =
  7$.
\item In one hour, the minute hand makes one complete revolution, i.e., it moved through $60$ divisions and
  the hour hand moves through $5$ divisions. Suppose, when the man went out, the hour hand was $x$ divisions
  ahead of the point labeled $12$ of the dial, where $20 < x < 25$ as he went out between $4$ p.m. and $5$
  p.m. Also, suppose, when the man the hour hand was $y$ divisions ahead of $0$ mark and $25 < y < 30$.

  Given that minute hand and hour hand exchanged their places, the minute hand was at $y$ when he went out
  and at $x$ when he returned. Because minute hand moves $12$ times faster than hour hand,

  $y = 12(x - 20)$ and $x = 12(y - 25)\Rightarrow y = 12[12(y - 25) - 20]\Rightarrow y =
  \frac{3840}{143}$. Since the man went out when the hand was at $y$, the man went at $\frac{3840}{143}$
  minutes past $4$ p.m.
\item Sum of roots is $\alpha + \alpha^3 + \alpha^4 + \alpha^{-4} + \alpha^{-3} + \alpha^{-1}  + \alpha^2 +
  \alpha^5 + \alpha^6 + \alpha^{-6} + \alpha^{-5} + \alpha^{-2} = \alpha + \alpha^2 + \alpha^3 + \alpha^4 +
  \alpha^5 + \alpha^6 + \alpha^7 + \alpha^8 + \alpha^9 + \alpha^{10} + \alpha^{11} + \alpha^{12}\;(\because
  \alpha^{13} = 1) = (1 + \alpha + \alpha^2 + \cdots + \alpha^{12}) - 1 = \frac{1 - \alpha^{13}}{1 - \alpha}
  - 1 = -1$.

  Product of roots is $(\alpha + \alpha^3 + \alpha^4 + \alpha^{-4} + \alpha^{-3} + \alpha^{-1})(\alpha^2 +
  \alpha^5 + \alpha^6 + \alpha^{-6} + \alpha^{-5} + \alpha^{-2}) = 3(\alpha + \alpha^2 + \cdots +
  \alpha^{12}) = -3$.

  Thus, the quadratic equation having these roots is $x^2 + x - 3 = 0$.
\item $1 + x^n + x^{2n} + \cdots + x^{mn} = \frac{x^{(m + 1)n} - 1}{x^n - 1}$ and

  $1 + x + x^2 + \cdots + x^m = \frac{x^{m + 1} - 1}{x - 1}$.

  We have to find $(m, n)$ such that $\frac{x^{(m + 1)n} - 1}{x^n - 1}\div \frac{x^{m + 1} - 1}{x - 1} =
  \frac{(x^{(m + 1)n} - 1)(x - 1)}{(x^n - 1)(x^{m + 1} - 1)}$ is a polynomial.

  Now if $k$ and $l$ are relatively prime, then $(x^k - 1)$ and $(x^l - 1)$ have just one common factor $x -
  1$. By De'moivre's theorem, the roots of $x^k - 1 = 0$ are $\cos\frac{2n\pi}{k} + i\sin\frac{2n\pi}{k}$
  for $n = 0, 1, 2, \ldots, k - 1$ and those of $x^l - 1 = 0$ are $\cos\frac{2n\pi}{l} +
  i\sin\frac{2n\pi}{l}$ for $n = 0, 1, 2, \ldots, n - 1$. If $l$ and $k$ are co-prime integer other than
  zero, these roots will be different.

  Since all the factors of $x^{(m + 1)n} - 1$ are distinct, $x^m - 1, x^n - 1$ cannot have any common
  factors other than $x - 1$. Thus, $m + 1$ and $n$ must be relatively prime.

  Again, $x^{(m + 1)n} - 1 = (x^{n})^{m + 1} - 1 = (x^{m + 1})^n - 1$. So $x^{(m + 1)n} - 1$ is divisible by
  both the factors in denominator. Thus, it is sufficient for our result for $m + 1$ and $n$ to be
  relatively prime.
\item Given, $(a - b)^2 + (a - c)^2 = (b - c)^2 \Rightarrow 2a^2 - 2ab - 2ac + 2bc = 0 \Rightarrow a^2 - a(b
  + c) + bc = 0 \Rightarrow (a - b)(a - c) = 0 \Rightarrow a = b$ or $a = c$.

  However, it contradicts with the given fact that $a, b, c$ are all distinct, and hence, has no solution.
\item $2^m = (1 + 1)^m = C_0^^m + C_1^^m + C_2^^m + \cdots + C_m^^m$ for $m = 1, 2, \ldots, n + 1$.

  Now consider the polynomial $f(x) = 2[C_0^^{x - 1} + C_1^^{x - 1} + C_2^^{x - 1} + \cdots + C_n^^{x -
      1}]$, clearly, $f(x)$ is of degree $n$.

  So $f(x) = 2.2^{x - 1}$ for all $x = 1, 2, \ldots, n + 1$. Thus, $f(x + 2) = 2[C_0^^{x + 1} + C_1^^{x + 1}
    + C_2^^{x + 1} + \cdots + C_n^^{x + 1}] = 2[2^{x + 1} - 1] = 2^{x + 2} - 2$.
\item Given, $a^2 + b^2 + c^2 + d^2 = ab + bc + cd + da\Rightarrow (a - b)^2 + (b - c)^2 + (c - d)^2 + (d -
  a)^2 = 0 \Rightarrow a = b, b = c, c = d, d = a\Rightarrow a = b = c = d$.
\item Given, $2x^2 + y^2 + 2z^2 - 8x + 2y - 2xy + 2xz - 16z + 35 = 0 \Rightarrow (x - y)^2 + (x + z)^2 + z^2
  - 16x - 8x + 2y + 35 = 0\Rightarrow (x - y - 1)^2 + (x + z - 3)^2 + z^2 - 10z + 25 = 0\Rightarrow (x - y -
  1)^2 + (x + z - 3)^2 + (z - 5)^2 = 0\Rightarrow x - y = 1, x + z = 3, z = 5\Rightarrow x = -2, y = -3$.
\item We know that $x^8 + y^8 + 6 = 8xy\Rightarrow x$ and $y$ must be of the same sign otherwise LHS $> 0$
  and RHS $< 0$. Moreover $(x, y)$ is a solution then $(-x, -y)$ is also a solution, also WLOG $x, y > 0$.

  Now, $x^8 + y^8 + 1 + 1 + 1 + 1 + 1 + 1 = 8xy$. By AM-GM inequality $x^8 + y^8 + 1 + 1 + 1 + 1 + 1 + 1\geq
  8\sqrt[8]{x^8 + y^8} = 8|xy|$. But, by this hypothesis, equality holds. Hence, all eight terms are
  equal. Therefore, $x^8 = y^8 = 1$. Hence, $(x, y) = (1, 1) = (-1, -1)$ is the solution set.

  By observation you can see that the genral equation is $x^{2n} + y^{2n} = 2nxy + 2n - 2$, where $n = 1, 2,
  \ldots$ which will have the same solution set.
\item Given that, $5x\left(1 + \frac{1}{x^2 + y^2}\right) = 12 \Rightarrow 25x^2 = \frac{144}{\left(1 +
  \frac{1}{x^2 + y^2}\right)^2}$. Similarly, we can find from the second equation $25y^2 = \frac{16}{\left(1
  - \frac{1}{x^2 + y^2}\right)^2}$.

  Adding, $25(x^2 + y^2) = \frac{144}{\left(1 + \frac{1}{x^2 + y^2}\right)^2} + \frac{16}{\left(1 -
    \frac{1}{x^2 + y^2}\right)^2}$. Let $\frac{1}{x^2 + y^2} = t \Rightarrow x^2 + y^2 = \frac{1}{t}$

  $\therefore \frac{25}{t} = \frac{144}{(1 + t)^2} + \frac{16}{(1 - t)^2}\Rightarrow 32t(5t^2 - 8t + 5) =
  25(t^4 - 2t^2 + 1)$. Dividing both sides by $t^2$, we arrive at $32\left[5\left(t + \frac{1}{t}\right) -
    8\right] = 25\left[\left(t + \frac{1}{t}\right)^2 - 4\right]$

  Putting $t + \frac{1}{t} = \alpha$, we have $25\alpha^2 - 160\alpha + 156 = 0\Rightarrow \alpha =
  \frac{6}{5}, \frac{26}{5}\Rightarrow t + \frac{1}{t} = \frac{6}{5}, \frac{26}{5}\Rightarrow 5t^2 - 6t + 5
  = 0$ or $5t^2 - 26t + 5 = 0$. Since the discriminant of $5t^2 -6t + 5 = 0$ is $36 - 100 < 0$, there is no
  real root. $5t^2 - 26t + 5 = 0$, the roots are $5, \frac{1}{5}$.

  If $x^2 + y^2 = 5$ then $5x\left(1 + \frac{1}{5}\right) = 12$ and $5y\left(1 - \frac{1}{5}\right) = 4
  \Rightarrow x = 2, y = 1$. If $x^2 + y^2 = \frac{1}{5}$ then $5x(1 + 5) = 12$ and $5y(1 - 5) = 4$. Thus,
  $x = \frac{2}{5}$ and $y = -\frac{1}{5}$.
\item Adding all $2(x + y + z)^2 = 48 + 2L \Rightarrow x + y + z = \sqrt{24 + L}$. Dividing all the
  equations with $x + y + z = \sqrt{24 + L}$, we get $x + y = \frac{18}{\sqrt{24 + L}}, y + z =
  \frac{30}{\sqrt{24 + L}}, z + x = \frac{2L}{\sqrt{24 + L}}$.

  Solving these, we get $x = \frac{L - 6}{\sqrt{24 + L}}, y = \frac{24 - L}{\sqrt{24 + L}}, z = \frac{L +
    6}{\sqrt{24 + L}}$, where $6 < L < 24$.
\item From eq. $(1), (x - z) = (4 - y)\Rightarrow x^2 - 2zx + z^2 = 16 - 8y + y^2 \Rightarrow (x^2 + z^2 -
  y^2) - 2zx + 8y - 16 = 0 \Rightarrow zx = 2(2y - 5)[\because x^2 + y^2 - z^2 = -4] \cdots (4)$

  From eq. $(3)$ and $(4)$, we get $y\times2(2y - 5) = 6 \Rightarrow y = -\frac{1}{2}, 3$. Putting $y =
  -\frac{1}{2}$ in eq. $(1)$ and $(3)$, we get $x - z = \frac{9}{2}$ and $zx = -12$

  $(x + z)^2 = (x - z)^2 + 4zx = \frac{81}{4} - 48 < 0$. So $y = 3$ is the only valid solution for $y$

  $(x - z) = 1, zx = 2 \Rightarrow x + z = \pm3\Rightarrow x = 2, -1$ and $z = 1, -2$.
\item Multiplying given equations, we get $3xy(x + y - 2)(x + y - 1) = 18xy \Rightarrow 3xy[(x + y - 1)(x +
  y - 2) - 6] = 0\Rightarrow 3xy(x + y - 4)(x + y + 1) = 0$.

  So $x = 0$ or $y = 0$ or $x + y = 4$ or $x + y = -1$. Putting $x + y = 4$ in eq. $(1)$, we get $6x = 2y
  \Rightarrow y = 3x\Rightarrow x = 1, y = 3$.

  Putting $x + y = -1$ in eq. $(1)$, we get $y = -\frac{9x}{2} \Rightarrow -\frac{7}{2}x = -1 \Rightarrow x
  = \frac{2}{7}, y = -\frac{9}{7}$.
\item Adding $1$ to both sides of $(1), xy + x + y + 1 = 24 \Rightarrow (x + 1)(y + 1) = 24$. Similarly, $(y
  + 1)(z + 1) = 32$ and $(z + 1)(x + 1) = 48$.

  $\Rightarrow (x + 1)^2(y + 1)^2(z + 1)^2 = 24 \times32\times48 \Rightarrow (x + 1)(y + 1)(z + 1) =
  \pm(24\times 8) \Rightarrow z + 1 = \pm8, x + 1 = \pm6, y + 1 = \pm4$.

  Thus, $x = 5, y = 3, z = 7$ and $x = -7, y = -5, z = -9$.
\item If $x > 1$, then $y = x^3 + 3x(x^2 - 1) > x^3 > x > 1, z = y^3 + 3y(y^2 - 1) > y^3 > y > 1$ and $x =
  z^3 + 3z(z^2 - 1) > z^3 > z > 1$.

  Thus, $z > y > x > z$, which is impossible. Thus, $x\leq 1$ and again, $x < -1$ leads to $x > y > z > x$
  so $x \geq -1$. So $|x|\leq 1, |y|\leq 1, |z|\leq 1$.

  And hence we can write $x = \cos\theta$, where $0\leq\theta\leq\pi$.

  Now, $y = 4\cos^3\theta - 3\cos\theta = \cos3\theta, z = 4\cos^33\theta - 3\cos3\theta = \cos9\theta$ and
  $x = 4z^3 - 3z = 4\cos^39\theta - 3\cos9\theta = \cos27\theta$.

  Since trigonometric functions are periodic, it is possible. Thus, $\cos\theta = \cos27\theta \Rightarrow
  \cos\theta - \cos27\theta = 0 \Rightarrow 2\sin14\theta\cos13\theta = 0 \Rightarrow \theta =
  \frac{k\pi}{13}$, where $k = 0, 1, 2, \ldots, 13$ or $\theta = \frac{k\pi}{14}$, where $k = 1, 2, \ldots,
  14$. The solution is $(x, y, z) = (\cos\theta, \cos3\theta, \cos9\theta)$ where $\theta$ takes all the
  above values.
\item Consider the equation $p + qt + rt^2 + st^3 = t^4 \Rightarrow t^4 - st^3 - rt^2 - qt - p = 0$. Given
  that $a_1, a_2, a_3, a_4$ are the solutions of this equation, and hence,

  $\sigma_1 = a_1 + a_2 + a_3 + a_4 = s, \sigma_2 = (a_1 + a_2)(a_3 + a_4) + a_1a_2 + a_3a_4 = -r, \sigma_3
  = a_1a_2(a_3 + a_4) + a_3a_4(a_1 + a_2) = q, \sigma_4 = a_1a_2a_3a_4 = -p$.

  The second system of equation is $(t^2)^4 - w(t^2)^3 - z(t^2)^2 - y(t^2) - x = 0$. Putting $t^2 = u$, we
  have $u^4 - wu^3 - zu^2 - yu - x = 0$ and the roots would be $a_1^2, a_2^2, a_3^2, a_4^2$.

  $\sigma_1 = a_1^2 + a_2^2 + a_3^2 + a_4^2 = w \Rightarrow w = \displaystyle\left(\sum a_i\right)^2 -
  2\sum_{i<j}a_ia_j = s^2 + 2r$.

  $\sigma_2 = \displaystyle\sum_{i<j}a_i^2a_j^2 = -z \Rightarrow z = -\left(\sum_{i,j}a_ia_j\right)^2 - 2\sum
  a_i\sum_{i<j<k}a_ia_ja_k - 2a_1a_2a_3a_4 = -r^2 + 2qs + 2p$.

  $\sigma_3 = a_1^2a_2^2a_3^2 + a_1^2a_2^2a_4^2 + a_1^2a_3^2a_4^2 + a_2^2a_3^2a_4^2 = y \Rightarrow y = q^2
  - 2pr$

  $\sigma_4 = -x = -p^2$.
\item We observe that both $(x, y, z)$ and $(-x, -y, z)$ satisfy the given system of equations. Since there
  has to be only one solution, we deduce $x = y = 0$ and so $z^2 = 4 \Rightarrow z = \pm 2$.

  From equations $(1)$ and $(2), z = a, z = b$. So either $a = b = 2$ or $a = b = -2$.

  If $a = b = 2$, we have $xyz + z = 2, xyz^2 + z = 2, x^2 + y^2 + z^2 = 4\Rightarrow xyz(z - 1) = 0$ so
  either $x = 0$ or $y = 0$ or $z = 0$ or $z = 1$. If $z = 0$ the from last equation $0 = 4$, which is not
  possible. If $z = 1$, then $x, y$ are not zero, which gives more than one solution of the equation. Hence,
  $a = b = 2$ does not satisfy the condition.

  If $a = b = -2$, we have $xyz + z = -2, xyz^2 + z = -2, x^2 + y^2 + z^2 = 4$. Following like above when $z
  = 0$, we have $0 = -2$, which is not possible. If $z = 1$, then $xy + 1 = -2\Rightarrow xy = -3$ and $x^2
  + y^2 = 3\Rightarrow (x + y)^2 = -3$, which is not posssible for real $x, y$, and hence, $z\neq 1$.

  Thus, we have a unique solution $(0, 0, -2)$.
\item Clearly, $a^2 + ab + \frac{b^2}{3} = \frac{b^2}{3} + c^2 + c^2 + ca + a^2\Rightarrow 2c^2 + ac - ab =
  0 \Rightarrow a + 2c = \frac{ab}{c}$.

  Also, $25 - 9 + 16 = 32 \Rightarrow 2a + b + c = \frac{32}{a}$.

  $ab + 2bc + 3ca = b(a + 2c) + 3ca = \frac{b\times ab}{c} + 3ca = \frac{3a}{c}\left(\frac{b^2}{3} +
  c\right) = \frac{27a}{c}$.

  Again, $ab + 2bc + 3ca = 2c^2 + ac + 2bc + 3ca = 2c(c + b + 2a)$.

  $\Rightarrow \frac{32}{a} = \frac{27a}{c}.\frac{1}{2c} = \frac{27a}{2c^2} \Rightarrow \frac{a^2}{c^2} =
  \frac{64}{27}\Rightarrow ab + 2bc + 3ca = 27\times\frac{a}{c} = 24\sqrt{3}$.
\item We have $\log_3(\log_2x) + \log_{1/3}(\log_{1/2}y) = 1 \Rightarrow \log_3(\log_2x) -
  \log_3(\log_{1/2}y) = 1 \Rightarrow \frac{\log_2x}{\log_{1/2}y} = 3 \Rightarrow \log_2x = -3\log_2y
  \Rightarrow \log_2xy^3 = 0 \Rightarrow xy^3 = 1 \Rightarrow y = \frac{1}{4}, x = 64$.
\item We know that $\log_ax = \log_{\left(a^n\right)}(x^n)$. So $\log_2x = \log_4x^2, \log_3y = \log_9y^2,
  \log_4z = \log_{16}z^2$.

  Now, $\log_2x + \log_4y + \log_4z = 2 \Rightarrow x^2yz = 16$. Similarly, $y^2zx = 81$ and $z^2xy = 256$.

  $\Rightarrow x^2yz\times y^2zx\times z^2xy = (xyz)^4 = 16\times81\times256 \Rightarrow xyz = 24$.

  $\Rightarrow x = \frac{2}{3}, y = \frac{27}{8}, z = \frac{32}{3}$.
\item By observation one solution is $x = 3, y = 2$. As $\log_33 + \log_22 = 2$ and $3^3 - 2^2 = 23$.

  If $x < 3$, then $\log_3x < 1$. Since, $\log_3x + \log_2y = 2\Rightarrow \log_2y > 2\Rightarrow y >
  2$. Hence, $3^x < 3^3 = 27$ and $2^y > 2^2 = 4\Rightarrow 3^x - 2^y < 27 - 4 = 23$. Hence, $x$ cannot be
  less than $3$.

  Similarly, $x$ cannot be greater than $3$. Thus, $x = 3, y = 2$.
\item Given that $\alpha, \beta, \gamma$ are roots of the equation $x^3 - x^2 - 1 = 0$, so from Vieta's
  relations, we have $\alpha + \beta + \gamma = 1, \alpha\beta + \beta\gamma + \gamma\alpha = 0$ and
  $\alpha\beta\gamma = 1$. Now,

  $\frac{1 + \alpha}{1 - \alpha} + \frac{1 + \beta}{1 - \beta} + \frac{1 + \gamma}{1 - \gamma} = \frac{3 -
    (\alpha + \beta + \gamma) - (\alpha\beta + \beta\gamma + \gamma\alpha) + 3\alpha\beta\gamma}{1 - (\alpha
    + \beta + \gamma) + (\alpha\beta + \beta\gamma + \gamma\alpha) - \alpha\beta\gamma} = -5$.
\item $x^{m + 1} - x^m - x + 1 = (x^m - 1)(x - 1) = (x^{m - 1} + x^{m - 2} + \cdots + x^2 + x + 1)(x - 1)^2$,
  and hence, $(x - 1)^2$ is a factor of $x^{m + 1} - x^m - x + 1$.
\item $x^{10} - x^{8} + 8x^6 - 24x^4 + 32x^2 - 48 = (x^2 - 2)(x^8 + x^6 + 10x^4 - 4x^2 + 24) = 0$.

  If $x^2 - 2 = 0$ then $x = \pm\sqrt{2}$. Now $x^8 + x^6 + 10x^4 - 4x^2 + 24 = x^8 + x^6 + 9x^4 + x^4 -
  4x^2 + 4 + 20 = x^8 + x^6 + 9x^4 + (x^2 - 2)^2 + 20 > 0$.

  Thus, the given equation has two solutions $\pm\sqrt{2}$.
\item Given, $3 + 2x + \cdots + 3x^{96} + 2x^{97} + 3x^{98} + 2x^{99} = 0 \Rightarrow 3\frac{x^{100} -
  1}{x^2 - 1} + 2x\frac{x^{100} - 1}{x^2 - 1} = 0\Rightarrow 3 + 2x = 0$ as $x\neq 1\Rightarrow x =
  -\frac{3}{2}$.
\item $1 + x^{111} + x^{222} + x^{333} + x^{444} = \frac{x^{555} - 1}{x^{111} - 1}$ and $1 + x^{111} +
  x^{222} + x^{333} + \cdots + x^{999} = \frac{x^{1110} - 1}{x^{111} - 1}$.

  Now $x^{1110} = x^{2*555}\Rightarrow x^{1110} - 1 = (x^{555} - 1)(x^{555} + 1)$, and thus, we see that
  required condition is satisfied.
\item Given, $\frac{1}{x} + \frac{1}{y} = \frac{1}{z} \Rightarrow zx + yz - xy = 0$

  $(x + y - z)^2 = x^2 + y^2 + z^2 - zx - yz + xy = x^2 + y^2 + z^2$, and hence, $\sqrt{x^2 + y^2 + z^2} =
  \pm(x + y - z)$, which is rational.
\item Given, $ax^2 = by^2 = cz^2 = k$(say), then also given, $a^2x^3 + b^2y^3 + c^2z^3 = p^5 \Rightarrow
  \frac{k^2}{x} + \frac{k^2}{y} + \frac{k^2}{z} = p^5 \Rightarrow \frac{1}{x} + \frac{1}{y} + \frac{1}{z} =
  \frac{p^5}{k^2} = \frac{1}{p} \Rightarrow k = p^3$.

  Now, $\sqrt{a} + \sqrt{b} + \sqrt{c} = \frac{\sqrt{p^3}}{x} + \frac{\sqrt{p^3}}{y} + \frac{\sqrt{p^3}}{z}
  = \sqrt{p^3}\left(\frac{1}{x} + \frac{1}{y} + \frac{1}{z}\right) = \frac{\sqrt{p^3}}{p} = \sqrt{p}$.
\item Given, $ax^3 = by^3 = cz^3 = k$(say). Then, $\sqrt[3]{ax^2 + by^2 + cz^2} = \sqrt[3]{\frac{k}{x} +
  \frac{k}{y} + \frac{k}{z}} = \sqrt[3]{k} = \sqrt[3]{k}\left(\frac{1}{x} + \frac{1}{y} + \frac{1}{z}\right)
  = \sqrt[3]{a} + \sqrt[3]{b} + \sqrt[3]{c}$.
\item Clearly, $\frac{1}{x} + \frac{1}{y} + \frac{1}{z} = \frac{1}{x + y + z} \Rightarrow (x + y)(y + z)(z +
  x) = 0$. Clearly, at least one of the terms on L.H.S. has to be zero automatically making third as
  \quote{$a$}.
\item Since $a, k\in\mathbb{R}$, the root will have a complex conjugate pair. Thus, second root will be
  $\frac{1}{2}(a - 5i)$. Sum of the roots would be $a = 3$. Product of the roots is $\frac{k}{2} =
  \frac{1}{4}(9 + 25) \Rightarrow k = 17$.
\item Let the given equation have roots $a, a, b$, where $a$ has the multiplicity of $2$. From Vieta's
  relations $2a + b = -p, a^2 + 2ab = 0 \Rightarrow a = -2b$ and $a^2b = -q$.

  We have to prove that $4p^3 + 27q = 0 \Rightarrow -4(2a + b)^3 -27a^2b = 0 \Rightarrow -4(-3b)^3 -
  27(-2b)^2b = 0$. Hence proved.
\item Let the quadratic polynomial be $ax^2 + bx + c = 0$. Given, $f(0) = 6 \Rightarrow c = 6; f(1) = 1
  \Rightarrow a + b + c = 1$, and $f(2) = 0 \Rightarrow 4a + 2b + c = 0$.

  $\Rightarrow 2b + 3c = 4 \Rightarrow b = -7, \Rightarrow a = 2$.

  $\Rightarrow f(3) = 18 - 21 + 6 = 3$.
\item Given, $ac = 2(b + d)$. Discriminant of given equations are $a^2 - 4b$ and $c^2 - 4d$. For roots of
  the equation to be real the discriminant have to be greater or equal to zero. We assume that it is the
  case. Thus,

  $a^2 + c^2\geq 4(b + d)\Rightarrow a^2 + c^2 \geq 2ac \Rightarrow (a - c)^2\geq 0$. Thus, the assumption
  is correct, and hence proven.
\item Suppose we have four real numbers $0 < A < B < C < D$. If $1 < 4BC$, then $1 < 4BC < 4BD < 4CD$; so we
  can choose $B, C, D$. If $1\geq 4BC$, then $1\geq 4BC\geq 4AC\geq 4AB$; so we can choose $A, B,C$. In
  first case all roots are imaginary, and in second case all roots are real.
\item We can rewrite the given equations as $(x^2 - 3)^2 - x^3 - 2x > 0\;\forall\;x < 0$. And thus, the
  given equation has no negative roots.
\item Given equation is $3x^2 - (a + c + 2b + 2d) + (ac + 2bd) = 0$, whose discriminant is $(a + c + 2b +
  2d)^2 - 12(ac + 2db) = [(a + 2d) - (c + 2b)]^2 + 8(c - b)(d - a) > 0$, and hence, the roots will be real
  and distinct.
\item Let $f(x) = x^4 + x^3 + x^2 - x - 1$. We notice that there is one sign change so it can have at most
  one positive root. Now, $f(-x) = x^4 - x^3 + x^2 + x - 1$, and there are three sign changes so it can have
  at most treee negative roots.
\item Dicriminanats are $b^2 - 4ac$ and $b^2 + 4ac$, and thus, one of these have to be positive, making
  either $P(x)$ or $Q(x)$ to have two real roots. And hence, $P(x)Q(x)$ will have two real roots.
\item Let $\alpha, \beta, \gamma$ are the roots of the equation $f(x) = 0$, then roots of $g(x) = 0$ will be
  $\alpha^2, \beta^2, \gamma^2$. It is given that $f(x) = x^3 + x + 1$, and hence, $\alpha + \beta + \gamma
  = 0, \alpha\beta + \beta\gamma + \gamma\alpha = 1, \alpha\beta\gamma = -1$. Also, since $g(0) =
  -1\Rightarrow \alpha^2\beta^2\gamma^2 = -1$.

  Thus, $g(x) = x^3 - (\alpha^2 + \beta^2 + \gamma^2)x^2 + (\alpha^2\beta^2 + \beta^2\gamma^2 +
  \gamma^2\alpha^2)x - \alpha^2\beta^2\gamma^2 \Rightarrow g(9) = 729 - 81[(\alpha + \beta + \gamma)^2 -
    2(\alpha\beta + \beta\gamma + \gamma\alpha)] + 9[(\alpha\beta + \beta\gamma + \gamma\alpha)^2 -
    \alpha\beta\gamma(\alpha + \beta + \gamma)] - \alpha^2\beta^2\gamma^2 = 729 + 162 + 9 - 1 = 899$.
\item The dicriminants of the terms are $p^2 - 12q, r^2 - 4q$ and $s^2 + 8q$. Clearly, $p^2, r^2, s^2$ are
  positive. Now if $q$ is positive then $s^2 + 8q  > 0$ giving us two real roots. However, if $q < 0$, then
  both $p^2 - 12q$ and $r^2 - 4q$ are positive giving us four real roots.
\item Let $a$ be the first term and $d$ be the common difference. Then, $\frac{1}{q} = a + (p - 1)d$ and
  $\frac{1}{p} = a + (q - 1)d\Rightarrow d = \frac{1}{pq} \Rightarrow a = \frac{1}{pq}$.

  Now, $t_{pq} = \frac{1}{pq} + \frac{pq - 1}{pq} = 1$. Clearly, $1$ is a root of the given equation.
\item $D = \sqrt{4p^2 - 8q} = 2\sqrt{p^2 - 2q}$. The term under square root is odd minus even. Let this be
  a perfect square so that we have rational roots. Now since the term under question is odd the square root
  will be odd. Let it be $r$. Now, $p^2 - r^2 = 2q$. Let $p = 2m + 1$ and $r = 2n + 1$, then $(2m + 1)^2 -
  (2n + 1)^2 = 2q \Rightarrow (2m + 2n + 2)(m - n) = q$. L.H.S. is even while R.H.S. is odd and thus our
  supposition is wrong giving us no rational roots.
\item If for some $n$ division is possible then the g.c.d. of $n^3 - n + 3$ and $n^3 + n^2 + n + 2$ must be
  $n^3 - n + 3$. Using Euclidean algorithm, $(n^3 + n^2 + n + 2, n^3 - n + 2) = (n^3 - n + 2, n^2 + 2n - 1)
  = (n^2 + 2n - 1, -2n^2 + 3) = (n^2 + 2n - 1, 4n + 5) = (4n^2 + 8n - 4,  4n + 5) = (4n + 5, 3n - 4) = (3n -
  4, n + 9) = (-31, n + 9)\in\{\pm1, \pm31\}$. So we have four possible values for $n^3 - n + 2$ i.e. $\pm1,
  \pm31$. But for none of these values division is possible.
\item $s_n = \frac{q^{n + 1} - 1}{q - 1}, S_n = \frac{\left(\frac{1 + q}{2}\right)^{n + 1} - 1}{\frac{1 +
    q}{2} - 1} = \frac{(1 + q)^{n + 1} - 2^{n + 1}}{2^n(q - 1)}\Rightarrow 2^nS_n = \frac{(1 + q)^{n + 1} -
  2^{n + 1}}{(q - 1)}$.

  We have to find $\binom{n + 1}{1} + \binom{n + 1}{2}s + \binom{n + 1}{3}s^2 + \cdots + \binom{n + 1}{n +
    1}s^n$. The $r$th term is given by $t_r = C_r^^{n + 1}\frac{q^{r + 1}}{q - 1} - C_r^^{n + 1}\frac{1}{q -
    1}$.

  Therefore, sum is given by $\sum t_r = \frac{1}{q - 1}\left[\sum C_r^^{n + 1}q^{r + 1} - \sum C_r^^{n +
      1}\right] = \frac{1}{q - 1}\left[(1 + q)^{n + 1} - 1 - 2^{n + 1} + 1\right]$. Hence proved.
\item $\frac{1}{a} = \frac{1}{x} + \frac{1}{y}, \frac{1}{b} = \frac{1}{y} + \frac{1}{z}, \frac{1}{c} =
  \frac{1}{x} + \frac{1}{z}\Rightarrow \frac{a - b}{ab} = \frac{z - x}{xz}, \frac{1}{c} = \frac{z +
    x}{zx}\Rightarrow z = \frac{2abc}{ab - bc + ca}, x = \frac{2abc}{ab + bc - ca}, y = \frac{2abc}{bc + ca
    - ab}$.
\item Adding all equations $(x + y + z)^2 = 0\Rightarrow x = \frac{0}{0}$. Thus, $x, y, z$ can assume any
  value as long as $x + y + z = 0$ so if $x = a, y = b$ then $z = -a -b$.
\item $z^2 - x^2 + yz - xy = 12 \Rightarrow (z - x)(x + y + z) = 12, y^2 - z^2 + xy - zx = 4 \Rightarrow (y
  - z)(x + y + z) = 4, y^2 - x^2 + yz - zx = 16 \Rightarrow (y - x)(x + y + z) = 16$.

  $\Rightarrow z - x = 3(y - z), y - x = 4(y - z), 3(y - x) = 4(z - x)\Rightarrow (x, y, z) = (-1, 3, 2),
  (1, -3, -2), \left(-\frac{5}{\sqrt{13}}, \frac{11}{\sqrt{13}}, \frac{7}{\sqrt{13}}\right),
  \left(\frac{5}{\sqrt{13}}, -\frac{11}{\sqrt{13}}, -\frac{7}{\sqrt{13}}\right)$.
\item Let $a = x^2 + 3x - 4, b = 2x^2 - 5x + 3$, the the given equation is of the form $a^3 + b^3 = (a +
  b)^3 \Rightarrow 3ab(a + b) = 0$.

  This implies that $x^2 + 3x - 4 = 0$ or $2x^2 - 5x + 3 = 0$ or $3x^2 - 2x - 1 = 0$. Thus solutions are $x
  = -4, 1, \frac{3}{2}, -\frac{1}{3}$.
\item $n^4 + 2n^3 + 2n^2 + 2n + 1 = (n^2 + 1)(n + 1)^2$ so for the given number to be a perfect square $n^2
  + 1$ will have to be perfect square. Now we see that perfect squares are $1, 4, 9, 16, \ldots$
  i.e. difference between perfect squares is more than $1$ so $n^2 + 1$ cannot a perfect square.

  Another way to solve this is $n^2 + 1$ can only be $(n + 1)^2\Rightarrow n = 0$. Hence, there is no such
  positive number.
\item $a^{x/y} = a^{y/z} = a^{z/x} \Rightarrow x = y = z \Rightarrow a = 3x \Rightarrow (x, y, z) =
  \left(\frac{a}{3}, \frac{a}{3}, \frac{a}{3}\right)$.
\item Suppose the given polynomial can be factored into polynomials with integer coefficients. One such
  factor must be of the form $px + q$ then from rational root theorem $q = \pm1, \pm2$ and $p = \pm1,
  \pm5$. But we see that no such value satisfies the given polynomial and thus $px + q$ cannot be a factor.
\item Let $\alpha, \beta, \gamma, \delta$ be four roots such that $\alpha\beta = -200$. From Vieta's
  relations $\alpha + \beta + \gamma + \delta = -7, \alpha\beta + \beta\gamma + \gamma\delta + \alpha\gamma
  + \alpha\delta + \beta\delta = -240, \alpha\beta(\gamma + \delta) + \gamma\delta(\alpha + \beta) = -k,
  \alpha\beta\gamma\delta = 2000$.

  $\Rightarrow \gamma\delta = -10 \Rightarrow 200(\gamma + \delta) + 10(\alpha + \beta) = k \Rightarrow
  (\alpha + \beta)(\gamma + \delta) = -30 \Rightarrow (\gamma + \delta)^2 + 7(\gamma + \delta) - 30 = 0
  \Rightarrow \gamma + \delta = -10, 3 \Rightarrow \alpha + \beta = 3, -10\Rightarrow k = -1970, 500$.
\item $x^4 - 20x^3 + kx^2 + 590x - 1992 = (x^2 + ax + 24)(x^2 + bx - 83)$. Comparing coefficients
  $a + b = -20, ab - 59 = k, 24b - 83a = 590 \Rightarrow a = -10, b = -10 \Rightarrow k = 41$.
\item Let the roots be $x_1, x_2, x_3, x_4, x_5, x_6$, then from Vieta's relations $\sum x_i = 0$ and $\sum
  x_ix_j = 0 \Rightarrow \sum x_i^2 = 0$. However, if all the roots are real then $\sum x_i^2\neq 0$. Hence,
  all roots cannot be real.
\item $ax^2 + 2bx + c\geq 0 \Rightarrow a > 0$ and $D\leq 0 \Rightarrow b^2\leq ac$. Similarly, $q^2\leq
  pq$. Multiplying $b^2q^2\leq acpr \Rightarrow b^2q^2\leq 4acpr$ which makes $apx^2 + bqx + cr\geq 0$.
\item Let $P(x) = x^n + (2 + x)^n + (2 - x)^n$. First we assume $n$ to be even, then $P(x) = 3 x^n + \cdots
  + 2^{n + 1}$. From rational root theorem if there is a root $\frac{p}{q}$, then $p = \pm1, \pm2$ and $q =
  \pm1, \pm3$. Since we need integral roots, therefore $\frac{p}{q} = \pm1$. Now, $P(\pm1) = 2 + 3^n\neq
  0$. Thus, our assumption is wrong and $n$ is an odd number.

  Also, if $x$ is odd then $P(x)$ is odd, hence, $x$ must be even. Let $x = 2y, Q(y) = \frac{P(2x)}{2^n} =
  y^n + (1 + y)^n + (1 - y)^n = 2 + \cdots + y^n$. Again, from rational root theorem, $y = \pm1, \pm2$. The
  values of $Q(y)$ for these values are $3^n - 2^n - 1, 2^n - 1, 2^n + 1, 3^n + 2^n - 1$. None of these
  vanish unless $n = 1$, which is our first case. When $n = 1, P(x) = x + 4$, and hence, $x = -4$ is the
  only solution.
\item Since the equation is symmetrical so we can assume $\gamma = 90^\circ, \alpha = \theta,$ and $\beta =
  90^\circ - \theta$ because they are angles of a right-angle triangle. Substituting these in the given
  equation

  $\sin\theta\cos\theta\sin(2\theta - 90^\circ) + \cos\theta\sin(-\theta) + \sin\theta\cos\theta +
  \sin(2\theta - 90^\circ)\sin(-\theta)\sin(90^\circ - \theta) = -\sin\theta\cos\theta\cos2\theta -
  \sin\theta\cos\theta + \sin\theta\cos\theta + \cos2\theta\sin\theta\cos\theta = 0$.
\item Since the roots are complex $D < 0 \Rightarrow (a + b + c)^2 - 4(ab + bc + ca) < 0 \Rightarrow a^2 +
  b^2 + c^2 < 2(ab + bc + ca)$. Since $a, b, c$ are real $a^2 + b^2 + c^2 > 0$. $2\alpha = a + b +
  c\Rightarrow a + b + c > 0$. Thus, $a, b, c > 0$.

  We assume that $a\leq b\leq c$, so it is enough to prove that $\sqrt{a} + \sqrt{b} \geq
  \sqrt{c}\Rightarrow a + b + 2\sqrt{ab}\geq c \Rightarrow 4ab\geq a^2 + b^2 + c^2 + 2ab - 2bc - 2ca$, which
  is true.
\item Since $x + 1$ divides $ax^2 + bx + c$, therefore, $x = -1$ must be its root. Thus, $a - b + c = 0
  \Rightarrow b = a + c$.

  Clearly, $b > 2$. So we observe that for $b = 3, 4$ we have $2$ pairs of solutions for $a$ and $c$, for $b
  = 5, 6$, we have $4$ and so on. This becomes an A.P. summing which, we get $498002$.
\item For $x^2 + 2ax + b = 0, D = a^2 - b = (a - k)^2 = a^2 - 2ak + k^2$. Setting $b = k^2 - 2ak$ we have
  discriminant of first equation as $a^2 - 4b = (a - 4k)^2 - 3(2k)^2$. Thus, we can say that if there are
  infinite no. of rational points on $x^2 - 3y^2 = 1$ such that $(a, b)$ is relatively prime then it is
  proved. Now we choose $(2, 1)$ as $(a, b)$, which are relatively prime. The infinte no. of points is given
  by intersection of the curve with $y = m(x - 2) + 1$, where $m\in\mathbb{Q}$.
\item Putting $x = 0, 1, \frac{1}{2}$, we get $|c|\leq 1, |a + b + c|\leq 1, |a + 2b + 4c|\leq 4$.

  Thus, $|b + 3c|\leq 5 \Rightarrow |b|\leq 8 \Rightarrow |a + b + 3c|\leq 3\Rightarrow |a|\leq 8
  \Rightarrow |a| + |b| + |c|\leq 17$.
\item When $x = 0, |c|\leq 1$. Setting $x = \pm 1, |a + b + c|\leq 1$ and $|a - b + c|\leq 1\Rightarrow |a +
  c|\leq 1\Rightarrow |a|\leq 2$.

  Considering $-1\leq x\leq 1$, if $a > 0$ then, $2ax + b \leq 2a + b = a + b + c + a - c\leq 1 + a - c\leq
  1 + 1 + 2 = 4$ and $-4 = -2 - 1 - 1\leq -(a + c) - 1\leq (-a + c) - a + b - c\leq -2a + b\leq 2ax + b$.

  Similarly, it can be proven for $a < 0$.
\item $|a| = max\{-a, a\}$ implies $\frac{|p(1) + p(-1)|}{2} = \frac{|1 + p + q| + |1 - p  q|}{2} =
  \frac{1}{2}\mathrm{max}\{(1 + p + q + 1 - p - q), (1 + p + q - (1 - p + q))\} = \mathrm{max}\{|1 + q|, |p|\}$.

  Clearly, minimum value of $\mathrm{max}\{|1 + q|, |p|\}$ will be when $p = 0$ and $q = -\frac{1}{2}$. Thus, our
  polynomial is $x^2 - \frac{1}{2}$.
\item Clearly, $|a + b + c|\leq 1, |a - b + c|\leq 1, |c|\leq 1 \Rightarrow |a|\leq 2$. Now, $\frac{8}{3}a^2
  + 2b^2 = \frac{2}{3}[4a^2 + 3b^2] = \frac{2}{3}[2(a + b)^2 + 2(a - b)^2 - b^2]$, which will be maximum if
  $b = 0$. Thus, $a = \pm 2, c = \pm 1$.
\item Let the roots are $-p, -q, -r$ so that $p, q, r > 0$. From Vieta's relations, we have $p + q + r = a <
  3, b + c = pq + qr + rp + pqr$. By AM-GM inequality, $pqr\leq\left(\frac{p + q + r}{3}\right)^3 <
  1$. Also, $p^2 + q^2 + r^2\geq pq + qr + rp \Rightarrow (p + q + r)^2\geq 3(pq + qr + rp)\Rightarrow pq +
  qr = rp \leq 3 \Rightarrow pq + qr + rp + pqr < 3 + 1 = 4$.
\item Set $x = 0, p(0) = 1$. Setting $x = 1, p(1) = 0$. We continue till $x = 29$ to find $p(29) = 0$. Let
  $p(x) = x(x - 1)(x - 2)\cdots(x - 29)Q(x)$, where $Q(x)$ is some polynomial. Then,

  $x(x - 1)(x - 2)\cdots(x- 30)Q(x - 1) = x(x - 1)(x - 2)\cdots(x - 30)Q(x)\Rightarrow Q(x - 1) = Q(x)
  \Rightarrow Q(x)$ is periodic. $\Rightarrow Q(x) = c\Rightarrow p(x) = cx(x - 1)(x - 2)\cdots(x - 29)$.
\item We observe that $p(0) = 0, p(1) = 0, p(2) = 0$ and so on. Hence, $p(x)$ has infinite roots. Thus,
  $p(x) = 0$.
\item We will prove this by contradiction. Suppose $f(f(x)) = x$ has a real root $a$ i.e. $f(f(a)) = a$,
  then $f(x) = b\neq a$ since $f(x) = x$ has no real root. WLOG, we may assume that $b > a$, then
  $f(a) = b, f(b) = a$. Let $g(x) = f(x) - x$ for all $x$. It follows that $g(a) = b - a > 0$ and $g(b) = a
  - b < 0$. Since $f$ is a quadratic poplynomial, $f$ is continuous. This means that between $a$ and $b$
  there must exist one $x_0$ such that $f(x_0) = 0$ making it a root, which is a contradiction.
\item We set $x = 0, \pm1, \pm 2$, we have $7|e, 7|a + b + c + d + e, 7|a - b + c - d + e, 7|16a + 8b + 4c +
  2d + e, 7|16a - 8b + 4c - 2d + e$, which simplifies to $7|a + c, 7|32a + 8c$, which implies the given
  result.
\item $(2a + b - 3)^2 + 3(b - 1)^2\geq 0$.
\item $p(x^5) = x^{20} + x^{15} + x^{10} + x^5 + 1 = x^{20} - x^{15} + 2x^{15} - 2x^{10} + 3x^{10} - 3x^5 +
  4x^5 - 4 + 5 = (x^5 - 1)(x^{15} + 2x^{10} + 3x^5 + 4) + 5 = f(x)(x - 1)(x^{15} + 2x^{10} + 3x^5 + 4) +
  5$. Thus, remainder would be $5$.
\item Roots of $x^2 + 1$ are $\pm i$ and roots of $x^2 + x + 1$ are $\omega, \omega^2$. Let
  $x^{2025} = f(x)g(x) + ax^3 + bx^2 + cx + d$, where $g(x)$ is divisible by
  $(x^2 + 1)(x^2 + x + 1)$.

  Setting $x = i, i = -ai - b + ci + d$. Comparing real and imaginary parts, $b = d$ and $c - a = 1$. Setting
  $x = \omega, \omega = a + b\omega^2 + c\omega + d = a + b\left(\frac{-1 + i\sqrt{3}}{2}\right) +
  c\left(\frac{-1 - i\sqrt{3}}{2}\right) + d$. Comparing real and imaginary parts, $2a + 2d - b - c = 2$ and
  $b = c$. Thus, $a = 1, b = 2, c = 2, d = 2$. So the remainder is $x^3 + 2x^2 + 2x + 2$.
\item We may assume that the leading coefficient in $p(x)$ is positive, so that $p(x)\rightarrow \infty$
  when $n\rightarrow \infty$, and $p(n) > 1$ for $n > N$. If $x > N$ and $p(x) = a_nx^n + \cdots = y > 1$
  then $p(ry + x) = a_n(ry + x)^n + \cdots $ is divisible by $y$ for every integral $r$; and $p(ry + x)$
  tends to infinity with $r$. Hence there are infinitely many composite values off($n$).
\item Given, $x + \sqrt{a + \sqrt{x}} = a \Rightarrow (a - x)^2 = a + \sqrt{x} \Rightarrow a^2 - (2x + 1)a +
  x^2 - \sqrt{x} = 0$, which is a quadratic equation in $a$. Thus, $a = \frac{2x + 1 \pm (2\sqrt{x}) +
    1}{2}$. Let $z = \sqrt{x}$, then $a = z^2 + z + 1$ and $a = z^2 - z$. First we consider second root. From
  the given equation $\sqrt{a + \sqrt{x}} = a - x\geq 0 \Rightarrow a \geq x \Rightarrow a \geq
  z^2$. Substituting from first equation $z^2 - z = a \geq z^2 \Rightarrow -z\geq 0 \Rightarrow
  -\sqrt{x}\geq 0$. This is true only if $x = 0 \Rightarrow a = 0$.

  Now we consider the first root. $a = z^2 + z + 1 \Rightarrow z = \frac{-1 \pm \sqrt{4a - 3}}{2}$. Since $z
  > 0$, we discard negative root. $\Rightarrow z = \frac{-1\pm\sqrt{4a - 3}}{2} \Rightarrow 4a - 3\geq 0\cap
  z\geq 0 \Rightarrow a\geq 1$. Thus, $x = z^2 = \frac{2a - 1 - \sqrt{4a - 3}}{2}$.
\item Given, $x^2 - \sqrt{a - x} = a \Rightarrow a^2 - (2x^2 + 1)a + x^4 + x = 0 \Rightarrow a = x^2 + x,
  x^2 - x + 1\Rightarrow x = \frac{-1 \pm \sqrt{1 + 4a}}{2}, \frac{1\pm\sqrt{4a - 3}}{2}$. Clearly,
  $a\geq\frac{3}{4}$ and we also need to discard negative root in the first one.
\item Given, $\sqrt{a - \sqrt{a + x}} = x \Rightarrow a^2 - (2x^2 + 1)a + x^4 - x = 0 \Rightarrow a = x^2 +
  x + 1, x^2 - x \Rightarrow x = \frac{-1 + \sqrt{4a}}{2}, \frac{1\pm\sqrt{1 + 4a}}{2}$. Clearly, if $a\geq
  0$, then $x$ will be real.
\item Let the three rational roots of the equation are $\frac{\alpha_2}{\alpha_1}, \frac{\beta_2}{\beta_1},
  \frac{\gamma_2}{\gamma_1}$, then $ax^3 + bx^2 + cx + d = k(\alpha_1x + \alpha_2) + (\beta_1x + \beta_2) +
  (\gamma_1x + \gamma_2)\Rightarrow a = \alpha_1\beta_1\gamma_1, b = \alpha_1\beta_1\gamma_2 + \alpha_1\beta_2\gamma_1
  + \alpha_2\beta_1\gamma_1, c = \alpha_1\beta_2\gamma_2 + \alpha_2\beta_1\gamma_2 + \alpha_2\beta_2\gamma_1, d =
  \alpha_2\beta_2\gamma_2$. As $ad$ is odd, all of $\alpha_1, \alpha_2, \beta_1, \beta_2, \gamma_1, \gamma_2$ must be
  odd. This implies that $b$ and $c$ must be odd as well, which is a contradiction.
\item We observe that $f(x) = f\left(\frac{1}{x}\right)$. Dividing the given equation by $x^2$, we have $x^2
  + ax + b + \frac{a}{x} + \frac{1}{x^2} = 0 \Rightarrow \left(x + \frac{1}{x}\right)^2 + a\left(x +
  \frac{1}{x}\right) + b - 2 = 0$. Substituting $x + \frac{1}{x} = z\geq 2$, we have

  $z^2 + az + b - 2 = 0 \Rightarrow D = \frac{-a \pm \sqrt{a^2 -4b + 8}}{2}\geq 2 \Rightarrow a^2 - 4b +
  8\geq a^2 + 8a + 16\Rightarrow -b\geq 2a + 2 \Rightarrow a^2 + b^2\geq 5a^2 + 8a + 4$.

  We know that for a quadratic equation minimum value is $-\frac{D^2}{4a}$, so minimum value of $a^2 +
  b^2$ is $\frac{4}{5}$.
\item $(x - 2)^2 = x^2 - 4x + 4\Rightarrow (x - 2)^2 - 2 = x^2 - 4x + 2$ thus we see that the term involving
  power of $x^0$ will always become $4$ as after the expansion and before squaring the constant term will
  always be $2$, and hence, $a_0 = 4$.

  $a_1$ is coefficient of $x$. Again $(x - 2)^2 - 2= x^2 - 4x + 2$. Now if we perform a square then we see
  that coefficient of $x$ will be four times because it will get multiplied with $2$ twice and get
  added. Thus, $a_1 = -4^k$.

  Similarly, we find that on first expansion $a_2 = 1 = \frac{4 - 1}{3}$, on second expansion it is $a_2 =
  4 + 4^2 = \frac{4^3 - 4^1}{3}$, and on third expansion $a_2 = 4^2 + 4^3 + 4^4 = \frac{4^{2k
      - 1} - 4^{k - 1}}{3}$, proceeding we find that $a_k = 4^{k-1} + 4^{k} + \cdots + 4^{2k - 2} =
  \frac{4^{2k - 1} - 4^{k - 1}}{3}$.

  The coefficient of $a_{2^k} = 1$ becauase it is the coefficient of highest power of $x$, which will remain
  constant on any number of squaring.
\item We have $x^2 - 3xy + 2y^2 + x - y = 0 \Rightarrow (x - y)(x - 2y + 1) = 0\Rightarrow x = y$ or $x = 2y
  - 1$. Substituting $x = y$ in the second equation, we have $y = 0$, which satisfy the third
  equation. Setting  $x = 2y - 1$ in the second equation, we have $y^2 - 5y + 6 = 0 \Rightarrow y = 2,3
  \Rightarrow x = 3, 5$. Both these pairs of value satisfy the third equation.
\item From first equation $y = 2x + a$. Setting this in the second equation we have $3x^2 + 3ax + (a^2 - b)
  = 0 \Rightarrow x = \frac{-3a \pm\sqrt{12b - 3a^2}}{3}$.

  Given that roots are rational, $\therefore 12b - 3a^2 = c^2\;c\in\mathbb{I}$. $c$ will be a multiple of
  $3$ becaause L.H.S. is a multiple of $3$. If $a$ is odd, then $c$ is odd and if $a$ is even then $c$ is
  even. Thus, $a$ and $c$ has same parity, and $-3a\pm c$ is always even, and hence is a multiple of
  $6$. This implies that $x$ and $y$ are integers.
\item We can assume that $a\geq b\geq c\geq d\geq e$, then $3a\leq3e \Rightarrow a = e\Rightarrow 3a =
  (3a)^3 \Rightarrow a = 0, \frac{1}{3}, -\frac{1}{3}$. Similarly, it can be proven if $a\leq b\leq c\leq
  d\leq e$.
\item Setting $y = xt$, we have $x^3t^2 = 15x^2 + 17x^2t + 15x^2t^2$ and $x^3t^2 = 20x^2 +
  3x^2t^2$. Multiplying second equation by $t$, and subtracting $3x^2(t - 5)(t^2 + 1) = 0 \Rightarrow x = 0,
  t = 5\Rightarrow (x, y) = (19, 95)$.
\item Subtracting we have $y^2 - x^2 = 7 \Rightarrow (y - x)(y + x) = 7$. Now there are four possibilities
  $y - x = 1, y + x = 7\Rightarrow y = 4, x = 3$, $y - x = 7, y + x = 1\Rightarrow y = 3, -4$ which
  does not satisfy the given equations, $y - x = -1, y + x = -7\Rightarrow y = -4, x = -3$, and $y - x =
  -7, y + x = -1\Rightarrow y = -4, x = 3$, which does not satisfy the given equations.
\item $a_1 + a_2 + a_3 = 40, a_2 + a_3 + a_4 = 40\Rightarrow a_1 = a_4$ or $a_{i+ 1} = a_{i + 4}$. Similarly,
  $a_{3i} = a_{3i + 3}\Rightarrow a_{2013} = a_3 = 10$.
\item $t_1 = 2007, t_2 = 53, t_3 = 34, t_4 = 25, t_5 = 29, t_6 = 85, t_7 = 89, t_8 = 145, t_9 = 42, t_{10} =
  20, t_{11} = 4, t_{12} = 16, t_{13} = 37, t_{14} = 58, t_{15} = 89$, so we see that $89$ has recurred, and
  hence, the sequence will repeat. The sum of first $7$ terms is $2322$. Now we will have this repetition of
  next $8$ terms. No. of such repetitions is $\frac{2013 - 7}{8} = 250 + \frac{6}{8}$. Sum of these $8$
  terms is $411$ and sum of first $6$ terms is $264$. Hence, required sum is $2322 + 411\times250 + 264 =
  105336$.
\item Let the sequence be $a_1, a_2, \ldots, a_{16}$. Then, $a_1 + a_2 + \cdots + a_7 = -1, a_2 + a_3 +
  \cdots + a_8 = -1, \cdots, a_{10} + a_{11} + \cdots + a_{16} = -1$, and $a_1 + a_2 + \cdots + a_{11} = 1,
  a_2 + a_3 + \cdots + a_{12} = 1, \cdots, a_6 + a_7 + \cdots + a_{16} = 1$. Also, $a_1 = a_{16}, a_2 =
  a_{15}, \cdots, a_8 = a_9$. Solving these we get the numbers as $5, 5, -13, 5, 5, 5, -13, 5, 5, -13, 5, 5,
  5, -13, 5, 5$.
\item Because of the inaccuracies of the balance if $x$ is the weight in left pan and $y$ is the weight in
  right pan then there will be two constants $m$ and $n$ such that $y = mx + n$. Thus, $A_2 = mA_1 + n, B_2
  = mB_1 + n \Rightarrow m = \frac{B_2 - A_2}{B_1 - A_1}, n = \frac{A_2B_1 - A_1B_2}{B_1 - A_1}$.

  Thus, $C_2 = mC_1 + n = \frac{C_1(B_2 - A_2) + A_2B_1 - A_1B_2}{B_1 - A_1}$.
\item Let $a, b, c, d$ be the roots of $x^4 + x^3 - 1 = 0$. From Vieta's relations, we have $a + b + c + d =
  -1, ab + bc + cd + ac + ad + bd = 0, abc + abd + acd + bcd = 0$ and $abcd = -1$. Thus, $ab =
  -\frac{1}{cd}, c + d = -1 - a - b$. Also, $ab + (a + b)(c + d) + cd = 0 \Rightarrow ab + (a + b)(-1 - a -
  b) - \frac{1}{ab} = 0$. Let $a + b = m$ and $ab = n$, then $n + m(-1 -m) - \frac{1}{n} = 0$. Also, $abc +
  abd + acd + bcd = 0\Rightarrow n(- 1 - m) - \frac{n}{m} = 0\Rightarrow n = -\frac{m^2}{m^2 + 1}$. Thus,
  $\frac{n^6 + n^4 + n^3 - n^2 - 1}{n(n^2 + 1)^2} = 0$. Thus, $ab$ is a root of $x^6 + x^4 + x^3 - x^2 - 1 =
  0$.
\item Let $l, m, n$ be three distinct fifth roots of unity. Then, $P(1) + lQ(1) + l^2R(1) = \frac{l^5 - 1}{l
- 1}S(l) = 0, P(1) + mQ(1) + m^2R(1) = \frac{m^5 - 1}{m - 1}S(m) = 0 \Rightarrow mQ(1) + m^2R(1) = nQ(1) +
  n^2R(1)\Rightarrow Q(1) = -(l + m)R(1)$. By symmetry, $Q(1) = -(l + n)R(1)$. Since $l\neq n$, it follows
  that $Q(1) = R(1) = 0$. Then as above $P(1) + lQ(1) + l^2R(1) = 0$, so $(x - 1)$ is a factor of $P(x)$ as
  asked for.
\item We have to prove that $x^6 \geq 2a - 1 \Rightarrow x^6 - 2x^5 + 2x^3 - 2x + 1\geq 0 \Rightarrow x^6
  -2x^5 + x^4 - x^4 + 2x^3 - x^2 + x^2 - 2x + 1\geq 0 \Rightarrow x^4(x^2 - 2x + 1) - x^2(x^2 - 2x + 1) +
  x^2 - 2x + 1\geq 0\Rightarrow (x^2 - 2x + 1)(x^4 - x^2 + 1)\geq 0$, which is true.
\item From Vieta's relations, $x_1 + x_2 + x_3 = 0, x_1x_2 + x_2x_3 + x_3x_1 = a, x_1x_2x_3 = -a$. Also,
  because $x_1, x_2, x_3$ are roots of $x^3 + ax + a = 0$, we will have $x_1^3 = -(ax_1 + a), x_2^3 = -(ax_2
  + a), x_3^3 = -(ax_3 + a)$.

  We have to prove that $\frac{x_1^2}{x_2} + \frac{x_2^2}{x_3} + \frac{x_3^2}{x_1} = -8 \Rightarrow
  \frac{x_1^3x_3 + x_2^3x_1 + x_3^3x_2}{x_1x_2x_3} = -8 \Rightarrow \frac{-a(x_1x_2 + x_2x_3 + x_3x_1 + x_1
    + x_2 + x_3)}{-a} = -8 \Rightarrow a = -8 \Rightarrow x^3 - 8x - 8 = 0$. Solving Vieta's relations we
  have roots as $-2, 1\pm\sqrt{5}$.
\item Let $f(x)$ be a function such that $f(x) = 2007 - x$, then $g(x) = p(x) - f(x) = x(x - 1)(x -
  2)\cdots(x - 2007)\Rightarrow p(x) = x(x - 1)(x - 2)\cdots(x - 2007) + (2007 - x)$.
\item $xP(x) - 1 = k(x - 1)(x - 2)\cdots (x - n - 1)$. Setting $x = 0, k = \frac{(-1)^n}{(n +
  1)!}\Rightarrow (n + 2)P(n + 2) - 1 = (-1)^n \Rightarrow P(n + 2) = \frac{1 + (-1)^n}{n + 2}$.
\item $Q(x) = (x + 1)P(x) - x = kx(x - 1)(x - 2)\cdots(x - n)$. Setting $x = -1, k = \frac{(-1)^{n + 1}}{(n
  + 1)!}\Rightarrow Q(n + 1) = (-1)^{n + 1}\Rightarrow P(n + 1) = \frac{n + 1 + (-1)^{n + 1}}{n + 2}$.
\item If $P$ is a polynomial with integral coefficients then $a - b\mid P(a) - P(b)$. Clearly, $a - b\mid
  P(a) - P(b) = b - c\mid P(b) - P(c) = c - a\mid P(c) - P(a)$ so $a - b, b - c, c - a$ must be equal in
  magnitude. Let us say that two of them, $a - b$ and $b - c$, are equal. Then $0 = |a - b + b - c + c - a|
  = |2(a - b) + (c - a)|\geq 2|a - b| - |c - a| = |a - b|$ so $a = b = P(a)$, and $c = P(b) = P(a) = b$, so
  $a, b, c$ are equal.
\item The proof will follow from following Lemma:

  {\bf Lemma:} If $0\leq m\leq 1\leq n$ then $(2 + m)(2 + n)\geq 3(2 + mn)$

  {\bf Proof:} $(2 + m)(2 + n)\geq 3(2 + mn)\Rightarrow 4 + 2m + 2n + mn\geq 6 + 3mn\Rightarrow 0\geq 2 - 2m
  - 2n + 2mn\Rightarrow 0\geq (m - 1)(n - 1)$.

  Now we solve the problem. Let $\alpha_1, \alpha_2, \ldots, \alpha_n$ are $n$ real roots then $P(x) = (x -
  \alpha_1)(x - \alpha_2)\cdots(x - \alpha_n)$

  Let $\beta_i = -\alpha_i$, then without loss of generality we can assume that $\beta_1\leq\beta_2 \leq
  \ldots \leq \beta_n$. As $\beta_1\beta_2\ldots\beta_n = 1$ if $\beta_n\geq 1$, then $\beta_1\beta_2 \ldots
  \beta_{n - 1} \leq 1$.

  $P(2) = (2 + \beta_1)(2 + \beta_2) \cdots (2 + \beta_n)$. Now we repeat the lemma

  $(2 + \beta_1)(2 + \beta_2) \cdots (2 + \beta_n)\geq (3 + \beta_1\beta_2)(2 + \beta_3) \cdots (2 +
  \beta_n) \geq 3^2(2 + \beta_1\beta_2\beta_3) \ldots (2 + \beta_n)\geq 3^{n - 1}(2 + \beta_1\beta_2 \ldots
  \beta_{b})\geq 3^n$.

  {\bf Aliter:} If $\beta_1, \beta_2, \ldots, \beta_n$ are non-negative numbers, then by the AM-GM
  inequality: $2 + \beta_i = 1 + 1 + \beta_i\geq 3\sqrt[3]{\beta_i}$. And thus, $(2 + \beta_1)(2 + \beta_2)
  \cdots (2 + \beta_n)\geq 3^n\sqrt[3]{\beta_1\beta_2\ldots\beta_n} = 3^n$.
\item Let the polynomial be $a_nx^n + a_{n - 1}x^{n - 1} + \cdots + a_0 = 0$ such that $a_i\in\{1,
  -1\}$. Let the roots be $\alpha_i$, where $i = 1, 2, 3, \ldots, n$.

  WLOG, let $a_n = 1$, then from Vieta's relations $\displaystyle\sum_{i=1}^n\alpha_i = -a_{n - 1}, \sum_{1\leq ij\leq
    n}\alpha_i\alpha_j = a_{n - 2}$, and $\displaystyle\prod_{i=1}^n\alpha_i = (-1)^na_0$.

  $\displaystyle\sum_{i=1}^n\alpha_i^2 = \left(\sum_{i = 1}^n\alpha_i\right) - 2\left(\sum_{1\leq ij\leq
    n}\alpha_i\alpha_j\right) = a_{n - 1}^2 - 2a_{n - 2} = 1 - 2a_{n - 2}$. However, all the roots are real,
  therefore $\displaystyle\sum_{i=1}^n\alpha_i^2\geq 0$, and hence, $\displaystyle a_{n - 1} = -1\Rightarrow \sum_{i=1}^n\alpha_i^2 =
  3$.

  We also have that $\displaystyle\sum_{i=1}^n|\alpha_i^2| = \sum_{i=1}^n\alpha_i^2 = 3$, and
  $\displaystyle\prod_{i = 1}^n|\alpha_i|
  = \left|\prod_{i = 1}^n\alpha_i\right| = |(-1)^na_0| = 1$.

  Using RMS-GM inequality, $\displaystyle1 = \left(\prod_{i =
    1}^n|\alpha_i|\right)^{\frac{1}{n}}\leq\left(\frac{1}{n}\sum_{i= 1}^n|\alpha_i|^2\right)^{\frac{1}{2}} =
  \sqrt{\frac{3}{n}}$.

  Thus, maximum value of $n$ is $3$. Now we find out the equations as $x - 1, x + 1, x^2 - x - 1, x^2 + x -
  1, x^3 - x^2 - x + 1, x^3 + x^2 - x - 1$, and their negatives.
\item WLOG we assume that at the three different integer $a, b, c$ the polynomial $p(x)$ assumes the value
  $1$. Let $d$ be one of its roots. Then $p(d) = (d - a)(d - b)(d - c)q(d) + 1$, where $q(x)$ has integer
  coefficients. We know for suree that one
  of $|d - a|, |d - b|, |d - c|$ is greater than $1$. Thus, $p(d) = kq(d) + 1 = 0\Rightarrow q(d) =
  -\frac{1}{k}$, but $d$ is an integer and $q$ has integer coefficients so we have a contradiction, and
  hence, $p(x)$ cannot have integral roots.
\item We have $\alpha + \beta = 6, \alpha\beta = 1$, and $\alpha^n + \beta^n = (\alpha + \beta)(\alpha^{n -
  1} + \beta^{n - 1}) - \alpha\beta(\alpha^{n - 2} + \beta^{n - 2})$, which is a recurrence relation which
  will eventually come down to $\alpha + \beta$ and $\alpha\beta$ both of which are in $\mathbb{Z}$, and
  hence, $\alpha^n + \beta^n\in\mathbb{Z}$. Similarly, second part can be proven.
\item It is given that $P(x)\geq 0$. If it has a root of odd multiplicity, then it changes sign at that
  root. So any real roots must have even multiplicity. If $x = \alpha + i\beta$ is a complex root so is $x =
  \alpha - i\beta$, because the coefficients are real. Let $x_i$ be the even no. of rela roots.

  Therefore, $P(x) = \prod_i(x - x_i)^2\prod_j((x - \alpha_j)^2 + \beta_j^2)$. Now $\prod_j((x - \alpha_j)^2
  + \beta_j^2) = (x - \alpha_j + i\beta_j)(x - \alpha_j - i\beta_j) = R(x)^2 + I(x)^2$.
\item Let $F(x) = f(g(h(x))) = \prod_{i=1}^8(x - i)\Rightarrow F'(x) = 4(2x - 9)(x^6 - 27x^5 + 288x^4
  -1539x^3 + 4299x^2 - 5886x + 3044)$. We see that $6$ degree equation is irreducible. However, $F'(x) =
  f'(g(h(x))).g'(h(x)).h'(x)$, which is not possible. Hence, the required condition is not possible.
\item For $|z| = 1$, we have $\overline{z} = \frac{1}{z} \Rightarrow |P(z)|^2 =
  P(z)\overline{P(z)}$. Expanding, $|P(z)|^2 = 4 + 2\Re(a\overline{b}z + a\overline{z}z^2 + a\overline{c}z^3
  + b\overline{c}z + b\overline{d}z^2 + c\overline{d}z)$.

  Let $\omega = e^{2\pi i/3}$ as third root of unity, we have $|P(z)|^2 + |P(z\omega)|^2 + |P(z\omega^2)|^2
  = 12 + 6\R(a\overline{d}z^3)$. We can choose a $z_0$ such that $a\overline{d}z_0^3 = 1$, then one of the
  terms on left must be greater than or equal to $6$.
\item From the question it is clear that both the functions cross $x$-axis at $a$ and $b$ respectively. This
  means, $a$ and $b$ are roots of $f(x)$ and $g(x)$. Let $a$ and $b$ repeat evenly for $f(x)$, then we have
  to prove that

  $(\alpha - a)^2(\alpha - b)^2 < (\alpha - a)(\alpha - b) \Rightarrow (\alpha - a)(\alpha - b)[(\alpha -
    a)(\alpha - b) - 1] < 0$. Substituting $\alpha = b + c, (2 + c)c[(2 + c)c - 1] < 0 \Rightarrow c(c^3 +
  4c^2 + 3c - 2) < 0$ which is true for very small $c$.

  Let $a$ has a multiplicity of $3$ and $b$ has of $1$, then we have to prove that

  $(x - a)^3(x - b) < (x - a)(x - b) \Rightarrow (x - a)(x - b)[(x - a)^2 - 1] < 0$, which is true between
  $a$ and $b$ because $b - a > 2$.
\item Putting $x = 2\cos t$ we find $P_n(x)= 2\cos2^nt$. Then $P_n(x) = x \Rightarrow \cos2^nt = \cos t$,
  which has $2^n$ solutions giving $2^n$ distinct solutions in $x$.
\item If $\alpha$ is a root then $\alpha^2$ is also a root. Similarly, if $\alpha - 1$ is a root then
  $(\alpha - 1)^2$ is also a root. If $|\alpha| > 1$, then it implies that there are infinitely many
  roots. If $\alpha\in(-1, 0)$ or $\alpha\in(0, 1)$, then it implies that there are infinitely many
  roots. Thus, $f(x)$ can have only $0$ and $1$ as roots. We also see that if we have have root of one kind
  then it must also have root of the other kind.

  Setting $f(x) = kx^a(x - 1)^b$, where $k\neq 0$. Using the identity we get that $-k = x^{a - b}(x +
  1)^{b - a}$, but $k$ is a constant. Hence, $a = b$ and $k = -1$. Thus, $-x^n(x - 1)^n$ is one form of
  function.

  We see that complex numbers, which are root of unity, and satisfy the property $z_1 = z_2^2$ will also be
  roots of this identity. For example, cube roots of unity. Deducing similarly we find that $-(x^2 + x +
  1)^n$ is another form of funciton.

  If $f(x) = c$ then $c = -1$ or $c = 0$.
\item $P(2x^2) = \frac{P(2x^3 + x)}{P(x)}$ is an even function so either both $P(2x^3 + x)$ and $P(x)$ are
  even functions or both are odd functions. We also see that $P(0)P(0) = P(0)\Rightarrow P(0) = 0$ or $1$.

  Consider $P(0) = 0$ and $P(x)$ is even, let $a_{2n}x^{2n}$ be the minimum power of $x$. Then, $(P'(x) +
  a_{2n}x^{2n})(P'(2x^2) + a_{2n}(2x^2)^{2n}) = P'(2x^3 + x) + a_{2n}(2x^3 + x)^{2n}$. Considering only
  lower powers of $x$, we find that $a_{2n} = 0$. Similarly, we find that of $P(0) = 0$ and $P(x)$ odd, we
  can prove that $P(x)$ cannot also be odd.

  So $P(0) = 1$ and $P(x)$ is even then $P(x) = 1 + \sum_{k=1}^na_{2n}x^{2n}$, putting $n = 1, 2, \ldots$
  and so on we find that $P(x) = (1 + x^2)^n$.
\item If $f(x)$ is constant $k$, then $k^2 = k \Rightarrow f(x) = 0$ and $f(x) = 1$. Given, $f(x)f(x + 1) =
  f(x^2 + x +1)$. Setting $x = x - 1$, we have $f(x)f(x - 1) = f(x^2 - x + 1)$. Suppose $f(x)$ is not a
  constant. Assume that it has at least one complex root. Let $z$ be at maximum distance from $O$. From our
  equations $f(z^2 + z + 1) = f(z^2 - z + 1) = 0$. Thus, $z\neq 0$. If also $z^2 + 1 \neq 0$, then $z, z^2 + z
  + 1, z^2 - z + 1, -z$ are vertices of a parallelogram. Thus, either of $z^2 + z + 1$ or $z^2 - z + 1$ is
  greater than $|z|$, which is a contradiction with the choice of $z$. Thus, $z^2 + 1 = 0$ is a factor of
  $f$. Hence, $f(x) = (x^2 + 1)^mg(x),\;m\in\mathbb{N}, x^2 + 1\nmid g(x)$.

  Putting this in our equation we see that it is satisfied. We also see that $g(x)$ satisfies our
  equation. Since it is not divisible by $x^2 + 1$, we must have $g(x) = 1$. Thus the solution is $f(x) =
  (x^2 + 1)^m\;\forall m\in\mathbb{N}$.
\item We see that if $\alpha$ is a root then $\alpha^2$ is also a root. Thus if $|\alpha| > 1$ or $0 < |\alpha|
  < 1$ then there will be infinitely many roots. Thus, all roots must lie on unit circle.

  If $f(x)$ is constant then $c^2 - c^2 = 0 \Rightarrow c = 0$ or $c = 1$. Let us assume that $f(x) = ax +
  b$. then $(ax + b)(b - ax) = ax^2 + b \Rightarrow a = -1, b = 0$ or $b = 1$. Thus, $f(x) = -x$ and $f(x) =
  1 - x$.

  If $f(x) = ax^2 + bx + c$, then $(ax^2 + bx + c)(ax^2 - bx + c) = (ax^4 + bx^2 + c) \Rightarrow a = 1, c^2
  = c \Rightarrow c = 0, 1$, and $2ac - b^2 = b$. If $c = 0\Rightarrow b = 0, -1$. If $c = 1\Rightarrow b =
  1, -2$. So we have $f(x) = x^2, x^2 -x, x^2 - 2x + 1 = (x - 1)^2, x^2 + x + 1$. We can rewrite the second
  and third in the form $f(x) = -x(1 - x)$ and $f(x) = (1 - x)^2$. Now we can write a general solution $f(x)
  = (-x)^p(1 - x)^q(x^2 + x + 1)^r, p, q, r\in\mathbb{Z}$.
\item Suppose $p(x), q(x), r(x)\in\mathbb{Z}[x]$ with $p(x) = q(x)r(x)$ where $0\leq\mathrm{deg}(q)\leq 3 <
  \mathrm{deg}(r)\leq\mathrm{deg}(p) = 7$. Given that for $n_1, n_2, \ldots, n_7$ distinct integers
  $|p(n_i)| = 1$ for $1\leq i\leq 7$. Observe that $q(x)$ and $r(x)$ are both polynomials with integer
  coefficients.

  Then $q(n_i), r(n_i)\in\mathbb{Z}$ for each $i$ and $|q(n_i)r(n_i)| = 1 \Rightarrow q(n_i) = \pm1$. We see
  that the values of $1$ or $-1$ is taken at least $4$ times but the degree of $q(x)$ is at most $3$ thus,
  $q(x) = \pm 1$. Hence, $p(x)$ is irreducible.
\item Let $f(x) = (x - a_1)^2(x - a_2)^2\cdots(x - a_n)^2 + 1$, and suppose that it factors non-trivially as
  $f(x) = g(x)h(x)$ over $\mathbb{Z}$.

  By Gauss's lemma we may assume that $g, h$ are monic polynomials with integeral coefficients. Let $g(x) =
  x^k + b_{k - 1}x^{k - 1} + \cdots + b_0, h(x) = x^l + c_{l - 1}x^{l - 1} + \cdots + c_0\in\mathbb{Z}[x]$

  Observe that the polynomial functions $g, h$ satisfy $g(r), h(r) > 0$ for $r\in\mathbb{r}$ and that
  $g(a_i) = h(a_i) = 1$. Suppose $k < l$ then the polynomial $g$ has the value $1$ then for $n$ distinct
  values $a_1, \ldots, a_n$ and, because it has degree $k < n$, that the polynomial is the constant $1$, and
  the factorization is trivial which is a contradiction. When $k = l = n$, then $f = g^2 \Rightarrow 1 =
  [g(x)] + (x - a_1)\cdots(x - a_n)[g(x)] - (x - a_1)\cdots(x - a_n)$, which is again trivial
  factorization. In both the cases the assumed non-triviality of factorization leads to contradiction, and
  thus $f$ is actullay irreducible.
\item For roots to be equal the discriminant has to be zero.

   $D = 4(1 + 3m)^2 - 4(1 + m)(1 + 8m) = 0\Rightarrow 4(1 + 9m^2 + 6m - 1 - 9m -8m^2) = 0\Rightarrow m^2 -
  3m = 0 \therefore m = 0, 3$
\item Discriminant of the equation is: $D = (c + a - b)^2 - 4(b + c - a)(a + b -c) = 4(b^2 - 4ac)$

  Given $a + b + c = 0 \Rightarrow b = -(a + c).$ Substituting in above equation, $D = 4\{(a + c)^2 - 4ac\}
  = 4(a - c)^2 =$ a perfect square and thus roots are rational.
\item Discriminant of the equation is: $D = 4(ac + bd)^2 - 4(a^2 + b^2)(c^2 + d^2) = -4(ad - bc)^2$. Roots
  are real if $D\geq 0$ i.e. $-4(ad - bc)^2 \geq 0 \Rightarrow (ad - bc)^2 \leq 0$

  But since $(ac - bd)^2 \nless 0 \therefore (ad - bc)^2 = 0$ i.e. $D = 0$ (because roots are real). Thus,
  if roots are real they are equal.
\item Let $A = a(b - c), B = b(c - a)$ and $c = c(a - b)$ Clearly, $A + B + C = 0$. Since roots are equal
  i.e. $D = 0 \therefore B^2 - 4AC = 0$

  Substituting for $B, [-(A + C)^2 - 4AC] = (A - C)^2 = 0 \Rightarrow A = C \Rightarrow 2ac = ab + cb
  \Rightarrow b = \frac{2ac}{a + c}$.

  Thus, $a, b, c$ are in H. P.
\item Given equation is $(b - x)^2 - 4(a - x)(c - x) = 0\Rightarrow -3x^2 + 2(2a + 2c - b)x + b^2 - 4ac = 0$

  Discriminant of the above equation is: $D = 4(2a + 2c - b)^2 + 12(b^2 - 4ac) = 8[(a - b)^2 + (b - c)^2 +
    (c - a)^2]\because a, b, c$ are real $\therefore D > 0$ unless $a = b = c$.

  Hence, roots are real unless $a = b = c$.
\item Discriminant of the equations are $p^2 - 4q$ and $r^2 - 4s$.

  Adding them we have $p^2 + r^2 - 4(q + s) = p^2 + r^2 - 2pr = (p - r)^2 \geq 0$.

  Thus, at least one of the discriminant is greater than zero and that equation has real roots.
\item Since $x^2 - 2px + q = 0$ has equal roots $D = 0 \Rightarrow 4p^2 - 4q = 0 \Rightarrow p^2 = q$.

  Discriminant of the second equation: $D = 4(p + y)^2 - 4(1 + y)(q + y) = 4[p^2 + 2y + y^2 - q -qy -y - y^2]$

  Substituting for $q, D = -4y(p - 1)^2$. Roots of the equation will be real and distinct only if $D \geq
  0$ but $(p - 1) \geq 0$ if $p \neq 1$. Thus, $y$ has to be negative as well.
\item Since roots of equation $ax^2 + 2bx + c = 0$ are equal $\therefore 4b^2 - 4ac \geq 0$. Discriminant of
  the equation $ax^2 + 2mbx + nc = 0$ is $4m^2b^2 - 4anc$.

  Since $m^2 > n > 0$ and $b^2 \geq ac$ $4m^2b^2 - 4anc > 0$. Thus, roots of the second equation are real.
\item Given $ax + by = 1 \Rightarrow y = \frac{1 - ax}{b},$ substituting this in second equation, $cx^2 +
  d\left(\frac{1 - ax}{b}\right)^2 = \frac{b^2cx^2 + d(1 - ax)^2}{b^2} = 1$

  $\Rightarrow (b^2c + da^2)x^2 - 2adx + d - b^2 = 0$. Since first two equations have one solution this
  equation will also have only one solution which means roots will be equal i.e. $D = 0$

  $\Rightarrow 4a^2d^2 - 4(b^2c + a^d)(d - b^2) = 0\Rightarrow b^2(b^2c + a^2d - cd) = 0\because b^2 \ne 0
  \therefore b^2c + a^2d - cd = 0 \Rightarrow b^2c + a^d = cd$

  Dividing both sides by $cd$ we have

  $\frac{b^2}{d} + \frac{a^2}{c} = 1\Rightarrow x = \frac{2ad}{2(b^2c + a^2d)} = \frac{a}{c}$. Substituting
  for $y,$ we get $y = \frac{b}{d}$.
\item Let the roots of the equation be $\alpha$ and $r\alpha$.

  Sum of roots = $\alpha + r\alpha = -\frac{b}{a} \Rightarrow \alpha = -\frac{b}{a(r + 1)}$.

  Product of roots $= r\alpha^2 = \frac{rb^2}{a^2(1 + r)^2} = \frac{c}{a} \Rightarrow \frac{b^2}{ac} =
  \frac{(r + 1)^2}{r}$.
\item Let the roots of the equation be $\alpha$ and $2\alpha.$. Sum of roots $= 3\alpha = -\frac{l}{l - m}
  \Rightarrow \alpha = -\frac{l}{l - m}$.

  Product of roots $= 2\alpha^2 = \frac{1}{l - m}$. Substituting for $\alpha, \frac{2l^2}{9(l - m)^2} =
  \frac{1}{l - m} \Rightarrow 2l^2- 9l + 9m = 0 [\because l\neq m\;\text{else it would not be a quadratic
      equation}]$.

  Since $l$ is real, therefore discriminant of this equation would be $\geq 0, \Rightarrow 81 - 72m \geq 0
  \therefore m \leq \frac{9}{8}$.
\item Let the roots be $\alpha$ and $\alpha^n$, then sum of roots $= \alpha + \alpha^n = -\frac{b}{a}$ and
  product of roots $= \alpha^{n + 1} = \frac{c}{a}$.

  From products, we have $\alpha = \left(\frac{c}{a}\right)^{\frac{1}{n + 1}}$. From sum we have $a\alpha^n + a\alpha + b = 0$.

  Substituting value of $\alpha$ from above $\Rightarrow a\left(\frac{c}{a}\right)^{\frac{n}{n + 1}} +
  a\left(\frac{c}{a}\right)^{\frac{1}{n + 1}} + b = 0$.  From this we arrive at our desired equation.
\item Let the roots be $p\alpha$ and $q\alpha$.

  Sum of roots $= (p + q)\alpha = -\frac{b}{a}$ and product of roots $= pq\alpha^2 = \frac{c}{a}$.

  From equation for product of roots, we have $\alpha^2 = \frac{c}{apq} \therefore \alpha = \sqrt{\frac{c}{apq}}$.

  Substituting this in sum of roots and solving we arrive at desired equation.
\item The questions are solved below:
  \startitemize[i]
  \item $\alpha + \beta = -p$ and $\alpha\beta = q$. Now, $\frac{\alpha^2}{\beta} + \frac{\beta^2}{\alpha} =
    \frac{\alpha^3 + \beta^3}{\alpha\beta}$

    $= \frac{(\alpha + \beta)^3 - 3\alpha\beta(\alpha + \beta)}{\alpha\beta} = \frac{p(3q - p^2)}{q}$.
  \item $(\omega\alpha + \omega^2\beta)(\omega^2\alpha + \omega\beta) = \omega^3\alpha^2 +
    \omega^4\alpha\beta + \omega^2\alpha\beta + \omega^3\beta^2$

    $= \alpha^2 + \omega\alpha\beta + \omega^2\alpha\beta + \beta^2 = \alpha^2 -\alpha\beta + \beta^2 =
    (\alpha + \beta)^2 - 3\alpha\beta = p^2 - 3q$.
  \stopitemize
\item Rewriting the equation we have $(A + cm^2)x^2 + Amx + Am^2 = 0$.

  Sum of roots $= \alpha + \beta = -\frac{Am}{A + cm^2}$ and product of roots $= \alpha\beta = \frac{Am^2}{A + cm^2}$

  The expression to be evaluated is $A(\alpha^2 + \beta^2) + A\alpha\beta + c\alpha^2\beta^2$.

  $= A[(\alpha + \beta)^2 - 2\alpha\beta] + A\alpha\beta + c(\alpha\beta)^2$.

  $= A\left[\frac{A^2m^2}{(A + cm^2)^2} - \frac{2Am^2}{A + cm^2}\right] + \frac{A^2m^2}{A + cm^2} + \frac{cA^2m^4}{(A +
    cm^2)^2} = 0$.
\item Sum of roots $= \alpha + \beta = -\frac{b}{a}$ and product of roots $= \alpha\beta = \frac{c}{a}$.

  Now, $a\left(\frac{\alpha^2}{\beta} + \frac{\beta^2}{\alpha}\right) + b\left(\frac{\alpha}{\beta} +
  \frac{\beta}{\alpha}\right) = \frac{a(\alpha^3 + \beta^3)}{\alpha\beta} + \frac{b(\alpha^2 +
    \beta^2)}{\alpha\beta}$

  $= a\frac{[(\alpha + \beta)^3 - 3\alpha\beta(\alpha + \beta)]}{\alpha\beta} + \frac{b[(\alpha + \beta)^2 -
      2\alpha\beta]}{\alpha\beta}$. Substituting for sum and product of the roots $=
  \frac{a\left[\left(-\frac{b}{a}\right)^3 - 3.\frac{c}{a}\left(-\frac{b}{a}\right)\right]}{\frac{c}{a}} +
  \frac{b\left[\left(-\frac{b}{a}\right)^2 -2 \frac{c}{a}\right]}{\frac{c}{a}}$

  Solving this we get the desired result.
\item Since $a$ and $b$ are the roots of the equation $x^2 + px + 1 = 0$ we have $a + b = -p$ and $ab = 1$.

  Similarly, since $c$ and $d$ are the roots of the equation $x^2 + qx + 1 = 0$ we have $c + d = -p$ and
  $cd = 1$.

  Now $(a - c)(b - c)(a + d)(b + d) = (ab - bc - ac + c^2)(ab + bd + ad + d^2) = [ab - c(a + b) + c^2].[ab +
    d(a + b) + d^2]$

  $= [1 + pc + c^2].[1 - pd + d^2] (\text{putting the values of } a + b\;\text{and}\;ab) = 1 + cp + c^2 - pd -
  cdp^2 - c^2pd + d^2 + cpd^2 + c^2d^2$

  $= 1 + (c^2 + d^2) + c^2d^2 -cdp^2 + p(c - d) + cpd(d - c) = 1 + [(c + d)^2 - 2cd] + c^2d^2 - cdp^2 + p(c
  - d) + cpd(d - c)$.

  Substituting for $c + d$ and $cd, 1 + q^2 - 2 + 1 - p^2 + p(c - d) + p(d - c) = q^2 - p^2$.
\item Let $\alpha$ and $\beta$ be the roots of the equation $x^2 + px + q = 0$ then $\alpha + \beta = -p$ and
  $\alpha\beta = q$.

  Also, let $\gamma$ and $\delta$ be the roots of the equation $x^2 + qx + p = 0$ then $\gamma + \delta =
  -q$ and $\gamma\delta = p$.

  Now, given is that roots differ by the same quantity so we can say that, $\alpha - \beta = \gamma -
  \delta\Rightarrow (\alpha - \beta)^2 = (\gamma - \delta)^2$

  $(\alpha + \beta)^2 - 4\alpha\beta = (\gamma + \delta)^2 - 4\gamma\delta\Rightarrow p^2 - 4q = q^2 - 4p
  \Rightarrow p^2 - q^2 + 4(p - q) = 0 \Rightarrow (p - q)(p + q + 4) = 0$

  Clearly, $p \neq q$ else equations would be same $\therefore p + q + 4 = 0$.
\item Since $\alpha, \beta$ are the roots of the equation $ax^2 + bx + c = 0\therefore a\alpha^2 + b\alpha +
  c = 0$ and $a\beta^2 + b\beta + c = 0$.

  and $\alpha + \beta = -\frac{b}{a}$ and $\alpha\beta = \frac{c}{a}.$ Also, given $S_n = \alpha^n +
  \beta^n$. Now, $aS_{n + 1} + bS_n + cS_{n - 1}$

  $= a(\alpha^{n + 1} + \beta^{n + 1}) + b(\alpha^n + \beta^n) + c(\alpha^{n - 1} + \beta^{n - 1}) =
  \alpha^{n - 1}(a\alpha^2 + b\alpha + c) + \beta^{n - 1}(a\beta^2 + b\beta + c) = \alpha^{n - 1}.0 +
  \beta^{n - 1}.0$

  $\therefore S_{n + 1} = -\frac{b}{a}S_n -\frac{c}{a}S_{n - 1}$

  Substituting $n = 4$ we have

  $S_5 = -\frac{b}{a}S_4 - \frac{c}{a}S_3 = -\frac{b}{a}(-\frac{b}{a}S_3 - \frac{c}{a}S-2) - \frac{c}{a}S_3
  = \left(\frac{b^2}{a^2} - \frac{c}{a}\right)S_3 + \frac{bc}{a^2}S_2$

  Proceeding similarly we have the solution as

  $= -\frac{b}{a^5}(b^2 - 2ac)^2 + \frac{(b^2 - ac)bc}{a^4}$.
\item Let $\alpha$ and $\beta$ be the roots of the equation $ax^2 + bx + c = 0$. Given, $\alpha + \beta =
  \frac{1}{\alpha^2} + \frac{1}{\beta^2}\Rightarrow \alpha + \beta = \frac{(\alpha + \beta)^2 -
    2\alpha\beta}{\alpha^2\beta^2}$

  $-\frac{b}{a} = \frac{\frac{b^2}{a^2} - 2\frac{c}{a}}{\frac{c^2}{a^2}} = \frac{b^2 - 2ac}{c^2}\Rightarrow
  -bc^2 = ab^2 - 2a^2c \Rightarrow ca^2 = \frac{ab^2 + bc^2}{2}$

  Thus, $bc^2, ca^2, ab^2$ are in A. P.
\item Rewriting the equation $m^2x^2 + (2m - m^2)x + 3 = 0$.

  Since $\alpha$ and $\beta$ are the roots of the equation $\alpha + \beta = -\frac{2m - m^2}{m^2} = \frac{m -
    2}{m}$ and $\alpha\beta = \frac{3}{m^2}$

  Given, $\frac{\alpha}{\beta} + \frac{\beta}{\alpha} = \frac{4}{3} \Rightarrow \frac{\alpha^2 + \beta^2}{\alpha\beta} =
  \frac{4}{3}$

  $3(\alpha^2 + \beta^2) = 4\alpha\beta \Rightarrow 3[(\alpha + \beta)^2 - 2\alpha\beta] =
  4\alpha\beta\Rightarrow 3(\alpha + \beta)^2 - 10\alpha\beta = 0 \Rightarrow 3\left[\left(\frac{m -
      2}{m}\right)^2 - \frac{10}{m^2}\right] = 0$

  $\Rightarrow m^2 - 4m - 6 = 0$

  Since $m_1, m_2$ are two values of $m$ we have $m_1 + m_2 = 4$ and $m_1m_2 = -6$. Now, $\frac{m_1^2}{m_2}
  + \frac{m_2^2}{m_1} = \frac{m_1^3 + m_2^3}{m_1m_2} = \frac{(m_1 + m_2)^3 - 3m_1m_2(m_1 + m_2)}{3m_1m_2} =
  -\frac{68}{3}$.
\item Let $\alpha$ and $\beta$ be the roots of the equation $ax^2 + bx + c = 0;$ $\gamma$ and $\delta$
  are the roots of the equation $a_1x^2 + b_1x + c_1 = 0,$ then

  $\alpha + \beta = -\frac{b}{a}, \alpha\beta = \frac{c}{a}$ and $\gamma + \delta = -\frac{b_1}{a_1},
  \gamma\delta = \frac{c_1}{a_1}$

  According to question, $\frac{\alpha}{\beta} = \frac{\gamma}{\delta}$. By componendo and dividendo,

  $\frac{\alpha - \beta}{\alpha + \beta} = \frac{\gamma - \delta}{\gamma + \delta}$. Squaring both sides

  $\Rightarrow \left(\frac{\alpha - \beta}{\alpha + \beta}\right)^2 = \left(\frac{\gamma - \delta}{\gamma +
    \delta}\right)^2$ $\Rightarrow \frac{(\alpha + \beta)^2 - 4\alpha\beta}{(\alpha + \beta)}^2 = \frac{(\gamma + \delta)^2 - 4\gamma\delta}{(\gamma +
      \delta)^2}$

  $\Rightarrow \frac{b^2 - 4ac}{b^2} = \frac{b_1^2 - 4a_1c_1}{b_1^2} \Rightarrow -4acb_1^2 = -4a_1c_1b^2 \Rightarrow
  \left(\frac{b}{b_1}\right)^2 = \frac{ac}{a_1c_1}$.
\item Since irrational roots appear in pairs and are conjugate. Thus, if first root is $\alpha = \frac{1}{2 + \sqrt{5}}$

  $\alpha = \frac{1}{2 + \sqrt{5}}\frac{2 - \sqrt{5}}{2 - \sqrt{5}} = \frac{2 - \sqrt{5}}{4 - 5} = -2 + \sqrt{5}$

  Then second root would be $\beta = -2 + \sqrt{5}$ $\Rightarrow \alpha + \beta = -4$ and $\alpha\beta = -1$

  Therefore, the equation is $x^2 - (\alpha + \beta)x + \alpha\beta = 0 \Rightarrow x^2 + 4x -1 = 0$.
\item Since $\alpha$ and $\beta$ are the roots of the equation $\therefore \alpha + \beta = -\frac{b}{a}$ and
  $\alpha\beta = \frac{c}{a}$. Sum of the roots for which quadratic equation is to be found $=
  \frac{1}{a\alpha + b} + \frac{1}{a\beta + b}$

  $= \frac{a(\alpha + \beta) + 2b}{a^2\alpha\beta + ab(\alpha + \beta) + b^2} = \frac{a\left(-\frac{b}{a}\right) +
    2b}{a^2.\frac{c}{a} + av\left(-\frac{b}{a}\right)} + b^2 = \frac{b}{ac}$

  Product of the roots $= \left(\frac{1}{a\alpha + b}\right)\left(\frac{1}{a\beta + b}\right) =
  \frac{1}{a^2\alpha\beta + ab(\alpha + \beta) + b^2} = \frac{1}{a^2.\frac{c}{a} +
    ab\left(-\frac{c}{a}\right) + b^2} = \frac{1}{ac}$.

  Therefore, the equation is $x^2 - \frac{b}{ac}x + \frac{1}{ac} = 0 \Rightarrow acx^2 - bx + 1 = 0$.
\item Given equation is $(x - a)(x - b) - k = 0 \Rightarrow x^2 - (a + b)x + ab - k = 0$.

  Since $c, d$ are roots of this equation $\Rightarrow c + d = a + b$ and $cd = ab - k$.

  The equation where roots are $a, b$ is $x^2 - (a + b)x + ab = 0 \Rightarrow x^2 - (c + d)x + cd + k = 0$.
\item Correct equation is $x^2 + 13x + q = 0$ and incorrect equation is $x^2 + 17x + q = 0$.

  Roots of correct incorrect equation are $-2$ and $-15.$ Thus $q = 30$.

  Therefore, correct equation is $x^2 + 13x + 30 = 0$ and thus roots are $-3, -10$.
\item Clearly, $\alpha + \beta = -p$ and $\alpha\beta = q$. Substituting $x = \frac{\alpha}{\beta}$ in the
  given equation we have

  $q\frac{\alpha^2}{\beta^2} - (p^2 - 2q)\frac{\alpha}{\beta} + q = 0 \Rightarrow q\alpha^2 - (p^2 -
  2q)\alpha\beta + q\beta^2 = 0$

  $q(\alpha^2 + \beta^2) - (p^2 - 2q)q = 0 \Rightarrow q[(\alpha + \beta)^2 - 2\alpha\beta] - (p^2 - 2q)q = 0$

  $q(p^2 - 2q) - (p^2 - 2q)q = 0 \Rightarrow 0 = 0$. Thus, $\frac{\alpha}{\beta}$ is a root of the given
  equation.
\item Let $\alpha$ and $\beta$ be the roots of $x^2 - ax + b = 0$ and $\alpha$ be the common and equal root
  from the second equation $x^2 - px + q = 0$.

  Thus, $\alpha + \beta = a, \alpha\beta = b$ and $2\alpha = p, \alpha^2 = q\Rightarrow b + q =
  \alpha\beta + \alpha^2 = \alpha(\beta + \alpha) = \frac{p}{2}a = \frac{ap}{2}$.
\item Let $\alpha$ be the common root. Then, we have $a\alpha^2 + 2b\alpha + c = 0$ and $a_1\alpha^2 + 2b_1\alpha + c_1 = 0$.

  Solving equations by cross-multiplication we have $\frac{\alpha^2}{2(bc_1 - b_1c)} = \frac{\alpha}{(ca_1 -
    a_1c)} = \frac{1}{2(ab_1 - a_1b)}$.

  From first two we have $\alpha$ as $\alpha = \frac{2(bc_1 - b_1c)}{ca_1 - a_1c}$ and from last two we have
  $\alpha$ as $\alpha = \frac{ca_1 - ac_1}{2(ab_1 - a_1b)}$

  Equating we get, $\frac{2(bc_1 - b_1c)}{ca_1 - a_1c} = \frac{ca_1 - ac_1}{2(ab_1 - a_1b)}\Rightarrow (ca_1
  - ac_1)^2 = 4(ab_1 - a_1b)(bc_1 - b_1c)$

  Given, $\frac{a}{a_1}, \frac{b}{b_1}, \frac{c}{c_1}$ are in A. P., let $d$ be the common difference.

  $\left(\frac{c}{c_1} - \frac{a}{a_1}\right)^2c_1^2a_1^2 = 4\left(\frac{a}{a_1} -
  \frac{b}{b_1}\right)a_1b_2\left(\frac{b}{b_1} - \frac{c}{c_1}\right)b_1c_1$

  $(2d)^2c_1^2a_2^2 = 4(-d)a_1b_1(-d)b_1c_1\Rightarrow 4d^2c_1^2a_1^2 = 4d^2a_1c_1b_1^2 \Rightarrow c_1a_1 =
  b_1^2$.

  Thus, $a_1, b_1, c_1$ are in G. P.
\item Let $\alpha$ be the common root between first two, $\beta$ be the common root between last two and $\gamma$ be
  the common root between first and last equations.

  Thus, $\alpha$ and $\beta$ are the roots of the first equation. $\Rightarrow \alpha + \gamma = -p_1, \alpha\gamma = q_1$

  Similarly, $\alpha + \beta = -p_2, \alpha\beta = q_2\Rightarrow \beta + \gamma = -p_3, \beta\gamma = q_3$

  L.H.S. $= (p_1 + p_2 + p_3)^2 = 4(\alpha + \beta + \gamma)^2$ and R.H.S. $= 4(p_1p_2 + p_2p_3 + p_1p_3 - q_1 - q_2 - q_3)$

  $= 4[(\alpha + \gamma)(\alpha + \beta) + (\alpha + \beta)(\beta + \gamma) + (\alpha + \gamma)(\beta + \gamma) -
    \alpha\gamma - \alpha\beta - \beta\gamma]$

  $= 4(\alpha^2 + \beta^2 + \gamma^2 + 2\alpha\beta + 2\alpha\gamma + 2\beta\gamma) = 4(\alpha + \beta + \gamma)^2$.

  Hence, proven that L.H.S. = R.H.S.
\item Let $\alpha$ be the common root then we have, $\alpha^2 + c\alpha + ab = 0$ and $\alpha^2 + b\alpha + ca = 0$.

  By cross-multiplication, we get the solution as $\frac{\alpha^2}{ac^2 - ab^2} = \frac{\alpha}{ab - ac} =
  \frac{1}{b - c}$.

  From first two we have $\alpha = \frac{ac^2 - ab^2}{ab - ac} = -(b + c)$. From last two we have $\alpha = a$.

  Equating these two we get $a = -(b + c) \Rightarrow a + b + c = 0$. Let the other root of the equations be
  $\beta$ and $\beta_1$ then we have

  $\alpha\beta = ab$ and $\alpha\beta_1 = ca\therefore \beta = b$ and $\beta_1 = c$. Equation whose roots are
  $\beta$ and $\beta_1$ is

  $x^2 - (\beta + \beta_1)x + \beta\beta_1 = 0 \Rightarrow x^2 -(b + c) + bc = 0 \Rightarrow x^2 + ax + bc =
  0$.
\item Clearly, root of the equation $x^2 + 2x + 9 = 0$ are imaginary and since they appear in pairs both the
  roots will be common and thus the ratio of the coefficients of the terms will be equal. $\Rightarrow a :
  b: c = 1 : 2 : 9$.
\item Let $\alpha$ be a common root. Then, we have $3\alpha^2 -2\alpha + p = 0$ and $6\alpha^2 - 17\alpha + 12 = 0$.

  Solving by cross-multiplication $\frac{\alpha^2}{-24 + 17p} = \frac{\alpha}{6p - 36} = \frac{1}{-39}$.

  From first two we have $\alpha = \frac{17p - 24}{6p - 36}$ and from last two we have $\alpha = \frac{6p -
    36}{-39} = -\frac{2p - 12}{13}$.

  Equating these two and solving for $p$ we get $p = -\frac{15}{4}, -\frac{8}{3}$.
\item When $x = 0, |x|^2 - |x| - 2 = |0|^2 - |0| - 2 = -2 \ne 0$. Since it is not satisfied by
  $x = 0$ it is an equation.
\item When $x = -a$ the equation is satisfied. Similarly, it is satisfied by values of $x$ being $-b$ and
  $-c$. The highest power of $x$ occurring is $2$ and is true for three distinct values of $x$
  therefore it cannot be equation but an identity.
\item Since both the equations have only one common root so the roots must be rational as irrational and
  complex roots appear in pairs. Thus, the roots of these two equations must be rational and therefore the
  discriminants must be perfect squares. Therefore, $b^2 - ac$ and $b_1^2 - a_1c_2$ must be perfect squares.
\item Equating the coefficients for similar powers of $x$, we get, coefficient of $x^2:$ $a^2 - 1 = 0
  \Rightarrow a = \pm1$.

  Coefficient of $x:$ $a - 1 = 0 \Rightarrow a = 1$. Constant term: $a^2 - 4a + 3 = 0 \Rightarrow a = 1, 3$.

  The common value of $a$ is 1 which will make this an identity.
\item Given, $\left(x + \frac{1}{x}\right)^2 = 4 + \frac{3}{2}\left(x - \frac{1}{x}\right)\Rightarrow
  \left(x + \frac{1}{x}\right)^2 - 4 - \frac{3}{2}\left(x - \frac{1}{x}\right) = 0\Rightarrow \left\{\left(x
  - \frac{1}{x}\right)^2 + 4x\frac{1}{x}\right\} - \frac{3}{2}\left(x - \frac{1}{x}\right) - 4 = 0$

  Substituting $a = x - \frac{1}{x}\Rightarrow a^2 - \frac{3}{2}a = 0 \Rightarrow 2a^2 - 3a = 0 \therefore a
  = 0, \frac{3}{2}$

  $x - \frac{1}{x} = 0 \Rightarrow x = \pm1\Rightarrow x - \frac{1}{x} - \frac{3}{2} \Rightarrow x = 2,
  -\frac{1}{2}$.
\item Given equation is $(x + 4)(x + 7)(x + 8)(x + 11) + 20 = 0$.

  Rewriting the equation, $[(x + 4)(x + 11)][(x + 7)(x + 8)] + 20 = 0$

  $\Rightarrow (x^2 + 15x + 44)(x^2 + 15x + 56) + 20 = 0$. Substituting $a = x^2 + 15x,$ we get $(a + 44)(a
  + 56) + 20 = 0\Rightarrow a = -46, -54$

  If $a = -46 \Rightarrow x^2 + 15x + 46 = 0 \Rightarrow x = \frac{-15 \pm \sqrt{41}}{2}$. If $a = -54
  \Rightarrow x^2 + 15x + 54 = 0 \Rightarrow x = - 6, -9$.
\item Given equation is $3^{2x + 1} + 3^2 = 3^{x + 3} + 3^x$. Let $3^x = a,$ then we have $3a^2 + 9 = 28a
  \Rightarrow 3a^2 - 28a + 9 = 0$.

  $\Rightarrow a = \frac{1}{3}, 9$. If $a = \frac{1}{3} \Rightarrow x = -1$. If $a = 9 \Rightarrow x = 2$.
\item Clearly, $(5 + 2\sqrt{6})^{x^2 - 3}(5 - 2\sqrt{6})^{x^2 - 3} = 1$. Let $(5 + 2\sqrt{6})^{x^2 - 3} = 1$
  then $(5 - 2\sqrt{6})^{x^2 - 3} = \frac{1}{y}$.

  The given equation becomes $y + \frac{1}{y} = 10$ where $y = (5 + 2\sqrt{6})^{x^2 - 3}\Rightarrow y^2 -10y
  + 1 = 0$.

  Solving the equation we have roots as $y = 5 \pm 2\sqrt{6}\therefore x^2 - 3 = \pm 1\Rightarrow x = \pm2,
  \pm\sqrt{2}$.
\item Let the speed of the bus $= x$ km/hour $\therefore$ the speed of car $= x + 25$ km/hour.

  Time taken by bus $= \frac{500}{x}$ hours and by car $= \frac{500}{x + 25}$ hours. Given, $\frac{500}{x} =
  \frac{500}{x + 25} + 10\Rightarrow x^2 - 25x + 1250 = 0$.

  $x = -50, 25$ but $x$ cannot be negative as it is a scalar quantity. Thus, speed of car = $50$ km/hour.
\item Given equation is $(a + b)^2x^2 - 2(a^2 - b^2)x + (a - b)^2 = 0$. Discriminant $= 4(a^2 - b^2)^2 - 4(a
  + b)^2(a - b)^2 = 0$. Since discriminant is zero, roots are equal.
\item Given equation is $3x^2 + 7x + 8 = 0$. Discriminant $D = 49 - 96 < 0$.

  Since it is negative roots will be complex and conjugate pair.
\item Given equation is $3x^2 + (7 + a) + 8 - a = 0$. Discriminant $D = (7 + a)^2 + 12a$

  For roots to be equal it has to be zero. $\Rightarrow a^2 + 26a + 49 = 0\Rightarrow a = 13 \pm
  6\sqrt{6}$.
\item It is given that roots are equal i.e. discriminant is zero. $\Rightarrow 4(ac + bd)^2 - 4(a^2 +
  b^2)(c^2 + d^2) = 0\Rightarrow a^2c^2 + b^2d^2 - 2abcd - a^2c^2 - a^2d^2 - b^2c^2 - b^2d^2 = 0$

  $\Rightarrow (ad - bc)^2 = 0\Rightarrow ad = bc \Rightarrow \frac{a}{b} = \frac{c}{d}$.
\item Discriminant is $4(c - a)^2 - 4(b - c)(a - b)$

  $= c^2 + a^2 -2ac - ab + b^2 + ac - bc = a^2 + b^2 + c^2 - ab - bc - ac = \frac{1}{2}[(a - b)^2(b - c)^2(c
  - a)^2]$.

  Clearly the above expression is either greater than zero or equal to zero. Hence, roots are real.
\item Given equation is $x^2 - x + x^2 - (a + 1)x + a + x^2 - ax = 0\Rightarrow 3x^2 - 2(a + 1) + a = 0$.

  Discriminant $D = 4(a + 1)^2 - 12a = a^2 + 2a + 1 - 3a = a^2 - a + 1 = (a - 1)^2 + a$

  which is greater than zero for all $a$ and hence roots are real.
\item Discriminant of the equation $D = b^2 - 4ac$. Given, $a + b + c = 0 \Rightarrow b = -(a + c)$.

  Substituting value of $b, D = (a + c)^2 - 4ac = (a - c)^2$, which is either zero or positive. Hence, roots
  are rational.
\item $D = (c + a - 2b)^2 - 4(b + c - 2a)(a + b - 2c) = c^2 + a^2 + 4b^2 + 2ac - 4bc - 4ab -4ba -4b^2 + 8bc
  - 4ca - 4bc + 8c^2 + 8a^2 + 8ab - 8ca$

  $\Rightarrow 9a^2 + 9c^2 - 18ca = 9(a - c)^2 \geq 0$ which is a perfect square. Hence, roots are rational.
\item Given $r = k + \frac{s}{k} \Rightarrow r^2 = k^2 + \frac{s^2}{k^2} + 2s$

  $\Rightarrow r^2 - 4s = k^2 + \frac{s^2}{k^2} + 2s - 4s\Rightarrow r^ - 4s = k^2 + \frac{s^2}{k^2} - 2s
  = \left(k - \frac{s}{k}\right)^2$

  Clearly, $r^2 - 4s \geq 0$ if $r, s, k$ are rationals which is discriminant of the given equation. Thus, roots will
  be rational provided given condition is met.
\item The given equation is $(x - a)(x - b) + (x - b)(x - c) + (x - c)(x - a) = 0$

  $\Rightarrow 3x^2 -(a + b + b + c + c + a)x + ab + bc + ca = 0\Rightarrow D = 4(a + b + c)^2 - 12(ab + bc + ca)$

  $= 4a^2 + 4b^2 + 4c^2 - 4ab - 4bc - 4ac = 2[(a - b)^2 + (b - c)^2 + (c - a)^2]$.

  This cannot be zero unless $a = b = c$, which is the required condition for the roots to be equal.
\item Given equation is $a^2(b^2 - c^2)x^2 + b^2(c^2 - a^2)x + c^2(a^2 - b^2) = 0$

  $D = b^4(c^2 - a^2)^2 - 4a^2c^2(b^2 - c^2)(a^2 - b^2) = b^4c^4 + b^4a^4 - 2b^4a^2c^2 - 4a^4b^2c^2 +
  4a^2b^4c^2 - 4a^4c^4 + 4a^2b^2c^4$

  $= b^4c^4 + b^4a^4 + 2b^4a^2c^2 - 4a^4b^2c^2 - 4a^4c^4 + 4a^2b^2c^4 = (b^2c^2 + b^2a^2 - 2a^2c^2)^2 \geq
  0$, which is a perfect square, and thus, roots will be rational.
\item $D = 16a^2b^2c^2d^2 - 4(a^4 + b^4)(c^4 + d^4) = 4[4a^2b^2c^22d^2 - a^4c^4 - a^4d^4 - b^4c^4 - b^4d^4]$

  $= -4[(a^2c^2 + b^2d^2)^2(a^2c^2 + b^2d^2)^2]$. Thus, if the roots are real then discriminant has to be
  zero because else it can be only negative and then roots wont remain real.
\item $D = 4q^2 - 4pr = 4(q^2 - pr)$. Since $p, q, r$ are in H. P. $\Rightarrow q = \frac{2pr}{p + r}$

  Substituting for $q$, we get $D = 4\left[\frac{4p^2r^2}{(p + r)^2} - pr\right] = 4\left[\frac{4p^2r^2 -
      p^3r - pr^3 - 2p^2r^2}{(p + r)^2}\right]$

  $= 4\left[\frac{2p^2r^2 - p^3 - r^3}{(p + r)^2}\right] = 4\left[\frac{pr(2pr - p^2 - r^2)}{(p + r)^2}\right]$

  $= 4\left[\frac{-pr(p - r)^2}{(p + r)^2}\right]$. Since $p$ and $r$ have the same sign discriminant is
  bound to be negative and roots will be complex numbers.
\item Discriminant of $bx^2 + (b - c)x + (b - c - a) = 0, D_1 = (b - c)^2 -4b(b - c - a) = b^2 + c^2 -2bc -4b^2 + 4bc + 4ab$

  Discriminant of $ax^2 + 2bc + b = 0, D_2 = 4b^2 - 4ab$. Now, if $D_2 < 0$

  $D_1 = (b + c)^2 - (4b^2 - 4ab) > 0$ and thus roots will be real. However, if $D_1 < 0$ i.e. roots are imaginary then we have

  $D_1 = (b + c)^2 - (4b^2 - 4ab) < 0 \Rightarrow 4b^2 - 4ab > 0 \because [(b + c)^2 > 0]$.

  Then roots of equation $ax^2 + 2bx + b = 0$ will be real.
\item From first equation $x = \sqrt{\frac{1 - by^2}{a}}$ and from second equation $x = \frac{1 - by}{a}$.

  Equating the values obtained $\left(\frac{1 - by}{a}\right)^2 = \frac{1 - by^2}{a}$

  $1 + b^2y^2 - 2by = a - aby^2 \Rightarrow (b^2 + ab)y^2 - 2by + 1 - a = 0$

  Values of $x$ will be equal if values of $y$ are equal i.e. discriminant of above equation is zero.

  $\Rightarrow 42b^2 - 4(b^2 + ab)(1 - a) = 0 \Rightarrow  4b^2 - 4b^2 + 4b^2a - 4ab + 4a^2b = 0$

  $(a^2b + ab^2 - ab) = 0\Rightarrow ab(a + b) = ab\Rightarrow a + b = 1$.
\item Substituting $y = mx + c$ in $x^2 + y^2 = a^2,$ we get $x^2 + m^2x^2 + 2cmx + c^2 - a^2 = 0$

  For roots to be equal, discriminant must be zero. $D = 4c^2m^2 - 4(1 + m^2)(c^2 - a^2) = 0$

  $\Rightarrow c^2m^2 - c^2 + a^2 - c^2m^2 + a^2m^2 = 0\Rightarrow c^2 = a^2(1 + m^2)$.
\item Clearly, roots are $\alpha, \alpha + 1$. Sum of roots $= \alpha + \alpha + 1 = \frac{5a +
  1}{4}\Rightarrow \alpha = \frac{5a - 3}{8}$.

  Product of roots $= \alpha(\alpha + 1) = \frac{5a}{4}$. Substituting value of $\alpha$ from above

  $\left(\frac{5a - 3}{8}\right)^2 + \frac{5a - 3}{8} = \frac{5a}{4}\Rightarrow \frac{25a^2 - 30a + 9 + 40a
    - 24 - 80a}{64} = 0$

  $\Rightarrow 25a^2 - 70a - 15 = 0 \Rightarrow 5a^2 - 14a - 3 = 0\Rightarrow a = 3, -\frac{1}{5}$.

  If $a = 3 \Rightarrow \alpha = \frac{3}{2}$ else if $a = -\frac{1}{5} \Rightarrow \alpha = -\frac{1}{2}$.

  Now it is trivial to calculate the value of $\beta$.
\item Let one of the roots is $\alpha$ then second root is $\frac{1}{\alpha}$.

  Product of roots $= \alpha * \frac{1}{\alpha} = \frac{k}{5} \Rightarrow k = 5$.
\item (a) The equation is :math:$(5 + 4m)x^2 - (4 + 2m)x + 2 - m = 0$

  For roots to be equal discriminant has to be zero.

  $4(2 + m)^2 - 4(5 + 4m)(2 - m) = 0\Rightarrow 4 + 4m + m^2 - 10 - 3m + 4m^2 = 0$

  $5m^2 - m - 6 = 0 \Rightarrow m = 1, -\frac{6}{5}$

  (b) Product of roots $= \frac{2 - m}{5 + 4m} = 2 \Rightarrow 2 - m = 10 + 8m \Rightarrow -\frac{8}{9}$

  (c) Sum of roots $= \frac{4 + 2m}{5 + 4m} = 6 \Rightarrow m = -\frac{13}{11}$
\item Let one root be $\alpha$ then the second root is $n\alpha$.

  Sum of roots $(n + 1)\alpha = -\frac{b}{a} \Rightarrow \alpha = -\frac{b}{(n + 1)a}$

  Product of roots $n\alpha^2 = \frac{c}{a}$

  Substituting value of $\alpha$ from the earlier equation

  $\frac{nb^2}{(n + 1)^2a^2} = \frac{c}{a} \Rightarrow (n + 1)^2 ca = nb^2$.
\item Following from previous problem $n = \frac{3}{4}$ and substituting in final solution

  $\left(\frac{3}{4} + 1\right)^2ca = \frac{3}{4}b^2 \Rightarrow 12b^2 = 49ac$.
\item From earlier problem, we have $a = 4, b = a, c = 3$ and $n = \frac{1}{2}$

  Substituting in the final relation we have, $\frac{9}{4}.3.4 = \frac{1}{2}a^2\Rightarrow a^2 = 54$.

  Discriminant of the second equation, $D = 9 - 4(a^2 - 2a) < 0,$ and thus roots are imaginary.
\item Let $\alpha, \beta$ be the roots of the given equation.

  Sum of roots, $\alpha + \beta = p$ and product of the roots $\alpha\beta = q$

  Given, $\alpha + \beta = m(\alpha - \beta)$. Squaring, $(\alpha + \beta)^2 = m^2(\alpha - \beta)^2$

  $p^2 = m^2(\alpha + \beta)^2 - 4m^2\alpha\beta = m^2p^2 - 4m^2q \Rightarrow p^2(m^2 - 1) = 4m^2q$.
\item Let $\alpha, \beta$ be the roots of the given equation. Sum of roots, $\alpha + \beta = p$ and product
  of the roots $\alpha\beta = q$

  Given, $\alpha - \beta = 1$. Squaring we have,

  $\Rightarrow (\alpha - \beta)^2 = 1 \Rightarrow (\alpha + \beta)^2 - 4\alpha\beta = 1\Rightarrow p^2 - 4q = 1$. Also,
  $[(\alpha - \beta)^2 + 2\alpha\beta]^2 = (1 + 2q)^2$

  $\Rightarrow (\alpha^2 + \beta^2)^2 = \alpha^4 + \beta^4 + 2\alpha^2\beta^2= \alpha^4 + \beta^4 - 2\alpha^2\beta^2 +
  4\alpha^2\beta^2 = (\alpha^2 - \beta^2)^2 + 4q^2$

  $\Rightarrow [(\alpha + \beta)^2(\alpha - \beta)^2] + 4q^2 = p^2 + 4q^2$.
\item The given equation is $a(x - b) + b(x - a) = m(x - a)(x - b)\Rightarrow mx^2 - xm(a + b) - mab - ax +
  ab - bx + ab = 0$

  $\Rightarrow mx^2 - x(m + 1)(a + b) - ab(m - 2) = 0$. If roots are equal in magnitude but opposite in sign
  then sum would be zero.

  $\Rightarrow (m + 1)(a + b) = 0 \Rightarrow m = -1\;\text{or}\;a + b = 0$.
\item Let $\alpha, \beta$ be the roots of the equation.

  Sum of roots, $\alpha + \beta = -\frac{b}{a}$ and product of roots, $\alpha\beta = \frac{c}{a}$.

  Difference of roots, $\alpha - \beta = k$ as given.

  Squaring we get, $(\alpha - \beta)^2 = k^2 \Rightarrow (\alpha + \beta)^2 - 4\alpha\beta = k^2$

  $\frac{b^2}{a^2} - 4\frac{c}{a} = k^2 \Rightarrow b^2 - 4ac = k^2a^2$.
\item Let $\alpha$ be one of the roots of the equation $ax^2 + bx + c = 0$. Clearly, $\alpha^2$ will be the other root.

  Sum of roots, $\alpha + \alpha^2 = -\frac{b}{a}$ and product of the roots $\alpha^3 = \frac{c}{a}$. Cubing sum of roots,

  $\frac{b^3}{a^3} = -\alpha^3(\alpha + 1)^3 = -\frac{c}{a}(\alpha^3 + 3\alpha(\alpha + 1) + 1)$

  $\frac{b^3}{a^3} = -\frac{c}{a}\left(\frac{c}{a} - \frac{3b}{a} + 1\right)$

  Simplifying we get the desired relationship.
\item Let $\alpha$ be one of the roots of the equation $ax^2 + bx + c = 0$. Clearly, $\alpha^2$ will be the other root.

  Sum of roots, $\alpha + \alpha^2 = -p$ and product of roots $\alpha^3 = 1$.

  Thus, $\alpha$ is cube root of unity. If $\alpha = -1$ then $p = -2$

  else if it is one of the complex numbers then we know that $1 + \omega + \omega^2 = 0$ which makes $p =
  1$.
\item Let $\alpha$ be one of the roots of the equation $ax^2 + bx + c = 0$. Clearly, $\alpha^2$ will be the other root.

  Sum of roots, $\alpha + \alpha^2 = -p$ and product of roots $\alpha^3 = q$

  $p^3 = -\alpha^3(\alpha + 1)^3 = -q(\alpha^3 + 3\alpha(\alpha + 1) + 1) = -q(q - 3p + 1)$

  $\Rightarrow p^3 - q(3p - 1) + q^2 = 0$.
\item The solution is given below:
  \startitemize[i]
  \item $\alpha + \beta = -\frac{3}{2}$ and $\alpha\beta = \frac{4}{2} = 2$.

    $\Rightarrow \alpha^2 + \beta^2 = (\alpha + \beta)^2 - 2\alpha\beta = \frac{9}{4} - 4 = -\frac{7}{4}$.
  \item $\frac{\alpha}{\beta} + \frac{\beta}{\alpha} = \frac{\alpha^2 + \beta^2}{\alpha\beta}$

    Substituting for numerator from previous part,

    $\Rightarrow \frac{\alpha}{\beta} + \frac{\beta}{\alpha} = -\frac{7}{8}$.
  \stopitemize
\item Sum of roots, $\alpha + \beta = -\frac{b}{a}$ and product of roots, $\alpha\beta = \frac{c}{a}$

  $\frac{\alpha^2}{\beta} + \frac{\beta^2}{\alpha} = \frac{\alpha^3 + \beta^3}{\alpha\beta} =
  \frac{(\alpha + \beta)^3 - 3\alpha\beta(\alpha + \beta)}{\alpha\beta} = \frac{-\frac{b^3}{c^3} +
    \frac{3c}{a}\frac{b}{a}}{\frac{c}{a}} = \frac{3abc - b^3}{a^2c}$.
\item Sum of roots, $\alpha + \beta = -\frac{b}{a}$ and product of roots, $\alpha\beta = \frac{b}{a}$

  Given expression is, $\sqrt{\frac{\alpha}{\beta}} + \sqrt{\frac{\beta}{\alpha}} + \sqrt{\frac{b}{a}} =
  \frac{\alpha + \beta}{\sqrt{\alpha\beta}} + \sqrt{\frac{b}{a}} =
  \frac{-\frac{b}{a}}{\sqrt{\frac{b}{a}}} + \sqrt{\frac{b}{a}} = 0$.
\item Product of the roots of the first equation is $ b^2$ and sum of roots of the second equation is $2b$.

  Geometric mean of the roots of the first equation $= \text{square root of product of roots} = \sqrt{b^2} =
  b$.

  Arithmetic mean of the roots of the second equation $= \text{half of sum of roots} = \frac{2b}{2} = b$ and
  thus both are equal.
\item Let $\alpha, \beta$ be the roots of the equation.

  Sum of roots, $\alpha + \beta = -\frac{q}{p}$ and product of roots, $\alpha\beta = \frac{r}{p}$.

  Given, sum of roots is equal to sum of square of roots. $\therefore \alpha + \beta = \alpha^2 + \beta^2$

  $-\frac{q}{p} = (\alpha + \beta)^2 - 2\alpha\beta = \frac{q^2}{p^2} - \frac{2r}{p}\Rightarrow 2pr = pq +
  q^2$.
\item Let $\alpha, \beta$ be the roots of the equation. Sum of roots, $\alpha + \beta = p$ and product of
  roots, $\alpha\beta = q$.

  $\frac{\alpha^2}{\beta^2} + \frac{\beta^2}{\alpha^2} = \frac{\alpha^4 + \beta^4}{(\alpha\beta)^2} =
  \frac{(\alpha^2 + \beta^2)^2 - 2\alpha^2\beta^2}{\alpha^2\beta^2} = \frac{[(\alpha + \beta)^2 -
      2\alpha\beta]^2}{\alpha^2\beta^2} - 2$

  $= \frac{(p^2 - 2q)^2}{q^2} - 2 = \frac{p^4}{q^2} - \frac{4p^2}{q} + 2$.
\item Let $\alpha, \beta$ be the roots of the equation. Sum of roots, $\alpha + \beta = -\frac{b}{a}$ and
  product of roots, $\alpha\beta = \frac{c}{a}$

  $\Rightarrow \frac{1}{(a\alpha + b)^2} + \frac{1}{(a\beta + b)^2} = \frac{(a\alpha + b)^2 + (a\beta +
    b)^2}{[(a\alpha + b)(a\beta + b)]^2}$

  $\Rightarrow \frac{a(\alpha^2 + \beta^2) + 2ab(\alpha + \beta) + 2b^2}{(a^2\alpha\beta + 2ab(\alpha +
    \beta) + b^2)^2}$

  Substituting for sum of roots, product of roots and $\alpha^2 + \beta^2 = (\alpha + \beta)^2 -
  2\alpha\beta$ and simplifying

  $= \frac{b^2 - 2ac}{c^2a^2}$.
\item Rewriting the equation we have $\lambda x^2 + x(1 - \lambda) + 5 = 0$.

  Since $\alpha$ and $\beta$ are the roots therefore, we have $\alpha + \beta = \frac{\lambda - 1}{\lambda}$
  and $\alpha\beta = \frac{5}{\lambda}$.

  Given, $\frac{\alpha}{\beta} + \frac{\beta}{\alpha} = \frac{4}{5}$

  $\frac{\alpha^2 + \beta^2}{\alpha\beta} = \frac{(\alpha + \beta)^2 - 2\alpha\beta}{\alpha\beta}\Rightarrow
  \frac{(\lambda - 1)^2 - 10\lambda}{5\lambda} = \frac{4}{5}$

  $\Rightarrow (\lambda - 1)^2 - 10\lambda = 4\lambda \Rightarrow \lambda^2 - 16\lambda + 1 = 0\therefore
  \lambda_1 + \lambda_2 = 16$ and $\lambda_1\lambda_2 = 1$.
  \startitemize[i]
  \item $\frac{\lambda_1}{\lambda_2} + \frac{\lambda_2}{\lambda_1} = \frac{(\lambda_1 + \lambda_2)^2 -
    2\lambda_1\lambda_2}{\lambda_1\lambda_2}$

    Substituting the values for sum and product we have, result as $254$.
  \item $\frac{\lambda_1^2}{\lambda_2} + \frac{\lambda_2^2}{\lambda_1} = \frac{\lambda_1^3 +
    \lambda_2^3}{\lambda_1\lambda_2} = \frac{(\lambda_1 + \lambda_2)^3 - 3\lambda_1\lambda_2(\lambda_1 +
    \lambda_2)}{\lambda_1\lambda_2}$

    $= 4048$.
  \stopitemize
\item For the first equation $\alpha + \beta = -p$ and $\alpha\beta = q$ and similarly for the second
  $\gamma + \delta = -r$ and $\gamma\delta = s$.
  \startitemize[i]
  \item $(\alpha + \gamma)(\alpha + \delta)(\beta + \gamma)(\beta + \delta)$
    $= [\alpha^2 + \alpha(\gamma + \delta) + \gamma\delta][\beta^2 + \beta(\gamma + \delta) + \gamma\delta]$

    $= (\alpha^2 - r\alpha + s)(\beta^2 - r\beta + s)$
    $= (\alpha^2\beta^2 - r\alpha\beta^2 + s\beta^2 - r\alpha^2\beta - r^2\alpha\beta - rs\beta + s\alpha^2 - rs\alpha +
    s^2)$

    $= q^2 - r\alpha\beta(\alpha + \beta) + s(\alpha^2 + \beta^2) + r^2p - rs(\alpha + \beta) + s^2$
    $= q^2 + prs + s(p^2 - 2q) + r^2p - rsq + s^2$
  \item $(\alpha - \gamma)(\beta - \delta) + (\beta - \gamma)(\alpha - \delta) = \alpha\beta -
    \alpha\delta - \beta\gamma + \gamma\delta + \alpha\beta - \beta\delta -\alpha\gamma + \gamma\delta$

    $= 2\alpha\beta + 2\gamma\delta - (\alpha + \beta)(\gamma + \delta) = 2q + 2s - pr$.
  \item $(\alpha - \gamma)^2 + (\beta - \delta)^2 + (\beta - \gamma)^2 + (\alpha - \delta)^2$

    $= 2(\alpha^2 + \beta^2 + \delta^2 + \gamma^2) - 2(\alpha + \beta)(\gamma + \delta) = 2[(\alpha +
    \beta)^2 - 2\alpha\beta + (\gamma + \delta)^2 - 2\gamma\delta] - 2(\alpha + \beta)(\gamma + \delta)$

    $= 2[p^2 + r^2 - 2q - 2s] - 2pr$.
  \stopitemize
\item $\alpha + \beta = p$ and $\alpha\beta = q$

  Now, R.H.S. $= (\alpha + \beta)(\alpha^n + \beta^n) - \alpha\beta(\alpha^{n - 1} + \beta^{n - 1})$ $=
  \alpha^{n + 1} + \beta^{n + 1} =$ L.H.S.
\item $\alpha + \beta = \gamma + \delta = -p, \alpha\beta = -q$ and $\gamma\delta = r$ Also, since $\alpha,
  \beta$ are roots of $x^2 + px + q = 0, \therefore \alpha^2 + p\alpha + q = 0$ and $\beta^2 + p\beta + q =
  0$.

  Now, $(\alpha - \gamma)(\alpha - \delta) = \alpha^2 - \alpha(\gamma + \delta) + \gamma\delta = \alpha^2 +
  p\alpha - r = -q - r = -(q + r)$, and similarly, $(\beta - \gamma)(\beta - \delta) = -(q + r)$.
\item Clearly, $\alpha + \beta = 2p, \alpha\beta = q$ and $\gamma + \delta = 2r, \gamma\delta = s$
  \startitemize[i]
  \item $\frac{\alpha}{\beta} = \frac{\gamma}{\delta}$. By componendo and dividendo

    $\Rightarrow \frac{\alpha + \beta}{\alpha - \beta} = \frac{\gamma + \delta}{\gamma - \delta}$

    Squaring, $\left(\frac{\alpha + \beta}{\alpha - \beta}\right)^2 = \left(\frac{\gamma + \delta}{\gamma
      - \delta}\right)^2$

    $1 - \frac{4\alpha\beta}{(\alpha + \beta)^2} = 1 - \frac{4\gamma\delta}{(\gamma + \delta)^2}\Rightarrow
    \frac{q}{p^2} = \frac{s}{r^2}$.
  \item Since $\alpha, \beta, \gamma, \delta$ are in G. P. Hence, $\frac{\alpha}{\beta} = \frac{\gamma}{\delta}$ and
    then we can proceed like previous part.
  \item Since $\alpha, \beta, \gamma, \delta$ are in A. P. Hence, $\alpha - \beta = \gamma - \delta$

    $\Rightarrow (\alpha + \beta)^2 - 4\alpha\beta = (\gamma + \delta)^2 - 4\gamma\delta\Rightarrow 4p^2 -
    4q = 4r^2 - 4s \Rightarrow s - q = r^2 - p^2$.
  \stopitemize
\item Clearly, $\alpha + \beta = -\frac{2b}{a}$ and $\alpha\beta = \frac{c}{a}$ for $ax^2 + 2bx + c = 0$ and
  $\alpha + \beta + 2k = -\frac{2B}{A}$ and $(\alpha + k)(\beta + k) = \frac{C}{A}$ for $AX^2 + 2Bx + C = 0$.

  Given expression can be rewritten as $\frac{b^2}{a^2} - \frac{c}{a} = \frac{B^2}{A^2} - \frac{C}{A}$

  $\frac{(\alpha + \beta)^2}{4} - \alpha\beta = \frac{(\alpha + \beta + 2k)^2}{4} - (\alpha + k)(\beta +
  k)\Rightarrow (\alpha - \beta)^2 = (\alpha + k - \beta - k)^2$, which is true.
\item Proceeding like previous problem, we have to prove that $\frac{b^2 - 4ac}{B^2 - 4AC} = \frac{a^2}{A^2}
  \Rightarrow \frac{b^2}{a^2} - \frac{4c}{a} = \frac{B^2}{A^2} - \frac{4C}{A} \Rightarrow (\alpha + \beta)^2
  - 4\alpha\beta = (\alpha + \beta + 2k)^2 - 4(\alpha + k)(\beta + k)$

  $\Rightarrow (\alpha - \beta)^2 = (\alpha + k - \beta - k)^2$, which is true.
\item Let $\alpha, \beta$ be the roots of $x^2 + 2px + q = 0$ and $\gamma, \delta$ be the roots of
  $x^2 + 2qx + p = 0$

  $\alpha + \beta = -2p$ and $\gamma + \delta = -2q.$ Also, $\alpha\beta = q$ and
  $\gamma\delta = p$

  Given that roots differ by a constant term say $k$. $\therefore \alpha + k = \gamma$ and $\beta +
  k = \delta$

  Thus, $\alpha + \beta + 2k = -2q \Rightarrow -2p + 2k = -2q \Rightarrow k = p - q\Rightarrow
  \gamma\delta = \alpha\beta + (\alpha + \beta)k + k^2 = p$

  Also, $q - 2pk + k^2 = p \Rightarrow -2p + k = 1 \Rightarrow p + q + 1 = 0$.
\item Clearly, $\alpha + \beta = -\frac{b}{a}$ and $\alpha\beta = \frac{c}{a}$.
  \startitemize[i]
  \item Sum of these roots is $\frac{\alpha}{\beta} + \frac{\beta}{\alpha} = \frac{\alpha^2 +
    \beta^2}{\alpha\beta} = \frac{b^2 - 2ac}{ac}$

    Product of these roots is $1$. Therefore, such an equation is $x^2 -\frac{b^2 - 2ac}{ac}x + 1 = 0$.
  \item Sum of these roots is $\frac{\alpha^3 + \beta^3}{\alpha\beta} = \frac{(\alpha + \beta)^3 -
    3\alpha\beta(\alpha + \beta)}{\alpha\beta} = \frac{3abc - b^3}{a^2c}$.

    Product of these roots is $\alpha\beta = \frac{c}{a}$. Therefore, an equation whose roots were these is
    $x^2 - \frac{3abc - b^3}{a^2c}x + \frac{c}{a} = 0$.
  \item Sum of these roots is $(\alpha + \beta)^2 + (\alpha - \beta)^2 = 2(\alpha + \beta)^2 -
    4\alpha\beta = \frac{2b^2}{a^2} - \frac{4c}{a}$.

    Product of these roots is $(\alpha + \beta)^2(\alpha - \beta)^2 = (\alpha + \beta)^2[(\alpha +
      \beta)^2 - 4\alpha\beta] = \frac{b^2}{a^2}\left(\frac{b^2}{a^2} - \frac{4c}{a}\right)$.

    So the equation is $x^2 -\left(\frac{2b^2}{a^2} - \frac{4c}{a}\right)x +
    \frac{b^2}{a^2}\left(\frac{b^2}{a^2} - \frac{4c}{a}\right) = 0$.
  \item Sum of these roots is $\frac{1 - \alpha}{1 + \alpha} + \frac{1 - \beta}{1 + \beta} = \frac{1 +
    \beta - \alpha -\alpha\beta + 1 + \alpha - \beta -\alpha\beta}{1 + (\alpha + \beta) + \alpha\beta}$

    $= \frac{2 - 2\alpha\beta}{1 + (\alpha + \beta) + \alpha\beta} = \frac{2\left(1 +
    \frac{b}{a}\right)}{1 - \frac{b}{a} + \frac{c}{a}} = \frac{2(a + b)}{a - b + c}$.

    Product of these roots is $\frac{1 - \alpha}{1 + \alpha}.\frac{1 - \beta}{1 + \beta} = \frac{1
      -(\alpha + \beta) + \alpha\beta}{1 + (\alpha + \beta) + \alpha\beta} = \frac{1 + \frac{b}{a} +
      \frac{c}{a}}{1 - \frac{b}{a} + \frac{c}{a}} = \frac{a + b + c}{a - b + c}$.

    Therefore, the equation is $(a - b + x)x^2 - 2(a + b)x + (a + b + c) = 0$.
  \item Sum of these roots is $\frac{1}{(\alpha + \beta)^2} + (\alpha - \beta)^2 = \frac{a^2}{b^2} +
    [(\alpha + \beta)^2 - 4\alpha\beta] = \frac{a^2}{b^2} + \left[\frac{b^2}{a^2} - \frac{4c}{a}\right]$.

    Product of these roots is $\frac{1}{(\alpha + \beta)^2}.(\alpha - \beta)^2 = \frac{1}{(\alpha +
      \beta)^2}.[(\alpha + \beta)^2 - 4\alpha\beta] = \frac{a^2}{b^2}\left[\frac{b^2}{a^2} -
      \frac{4c}{a}\right] = \frac{b^2 - 4ac}{b^2}$.

    Now it is trivial to deduce the equation.
  \stopitemize
\item Let the roots of the equation $ax^2 + bx + c = 0$ are $p$ and $q$, then $p + q = -\frac{b}{a}$ and $pq
  = \frac{c}{a}$.

  (a) The reciprocal of roots are $\frac{1}{p}$ and $\frac{1}{q}$. Sum of these is $\frac{p + q}{pq} =
  -\frac{b}{c}$ and product is $\frac{1}{pq} = \frac{a}{c}$. Therefore, the equation is $cx^2 + bx + a = 0$.

  (b) Let one of the roots is $p$ then the other will be $-p$. Sum will be $0$ and product will be
  $-\frac{c}{a}$. Therefore, the equation is $ax^2 - c = 0$.
\item Clearly, $\alpha + \beta = -p$ and $\alpha\beta = q$.

  (a) $\alpha^4 + \beta^4 = (\alpha^2 + \beta^2) - 2\alpha^2\beta^2 = [(\alpha + \beta)^2 - 2\alpha\beta]^2
  - 2\alpha^2\beta^2 = [p^2 - 2q]^2 - 2q^2 = p^4 - 4p^2q + 2q^2$.

  (b) $\alpha^{-4} + \beta^{-4} = \frac{\alpha^4 + \beta^4}{\alpha^4\beta^4} = \frac{p^4 - 4p^2q +
    2q^2}{q^4}$.
\item Clearly, $\alpha + \beta = p$ and $\alpha\beta = q$.
  \startitemize[i]
  \item Sum of these roots is $\frac{q}{p - \alpha} + \frac{q}{p - \beta} = \frac{2pq - q(\alpha +
    \beta)}{p^2 - p(\alpha + \beta) + \alpha\beta} = \frac{pq}{q} = p$.

    Product of these roots is $\frac{q}{p - \alpha}.\frac{q}{p - \beta} = \frac{q^2}{q} = q$.

    Thus the eqation of these new roots remain same i.e. $x^2 - px + q = 0$.
  \item Sum of these roots is $\alpha + \beta + \frac{1}{\alpha} + \frac{1}{\beta} = \alpha + \beta +
    \frac{\alpha + \beta}{\alpha\beta} = p + \frac{p}{q} = \frac{p(1 + q)}{q}$.

    Product of these roots is $\left(\alpha + \frac{1}{\beta}\right)\left(\beta + \frac{1}{\alpha}\right) =
    \alpha\beta + \frac{\alpha}{\beta} + \frac{\beta}{\alpha} + \frac{1}{\alpha\beta} = q + \frac{1}{q} +
    \frac{\alpha^2 + \beta^2}{\alpha\beta} = \frac{q^2 + 1}{q} + \frac{p^2 - 2q}{q}$.

    Now deducing the equation is trivial.
  \stopitemize
\item Because $5 + 3i$ is a complex root the other root will be complex conjugate i.e. $5 - 3i$. Thus,
  equation having these complex roots will be $x^2 - 10x + 34 = 0$.
\item Because $3 + 4i$ is a complex root the other root will be complex conjugate i.e. $3 - 4i$. Thus,
  equation having these complex roots will be $x^2 - 6x + 25 = 0$.
\item Roots are given by $\frac{-2\pm\sqrt{4 + 16}}{6} = \frac{-1\pm\sqrt{5}}{4}$. Now $\frac{\sqrt{5} -
  1}{4} = \cos72^\circ$ and $-\frac{\sqrt{5} + 1}{4} = -\cos36^\circ = \cos216^\circ = \cos(3.72^\circ)$

  Now, $\cos3x = 4\cos^3x - 3\cos x$, therefore if one root is $\alpha$ then the other would be $4\alpha^3 -
  3\alpha$.
\item Clearly, by observation $\alpha, \beta$ are roots of the eqation $x^2 - 5x + 3 = 0$. $\Rightarrow
  \alpha + \beta = 5$ and $\alpha\beta = 3$.

  Now, $\frac{\alpha}{\beta} + \frac{\beta}{\alpha} = \frac{\alpha^2 + \beta^2}{\alpha\beta} =
  \frac{5(\alpha + \beta) - 6}{3} = \frac{19}{3}$.
\item Correct value of $p = -11$. $q$ is $4\times 6 = 24$. Hence, the correct equation is $x^2 - 11x + 24 =
  0$. Hence roots are $8, 3$.
\item Correct value of $q$ is $2$. $p$ is $-(6 - 1) = 5$. Hence, the correct equation is $x^2 - 5x + 2 = 0$.
\item From first student the correct value of $q = 6\times2 = 12$. From second student the correct value of
  $p = -(2 + -9) = 7$. Hence the correct equation is $x^2 + 7x + 12 = 0$ giving us $3, 4$ as correct roots.
\item We have $\alpha + \beta = -p, \alpha\beta = q, \alpha_1 + \beta_1 = p, \alpha_1\beta_1 = q$.

  Now, $\frac{1}{\alpha_1\beta} + \frac{1}{\alpha\beta_1} + \frac{\alpha\alpha_1}+ {\beta\beta_1} =
  \frac{(\alpha + \beta)(\alpha_1 + \beta_1)}{\alpha\beta\alpha_1\beta_1} = \frac{pq}{qp} = 1$

  and $\left(\frac{1}{\alpha_1\beta} + \frac{1}{\alpha\beta_1}\right)\left(\frac{1}{\alpha\alpha_1} +
  \frac{1}{\beta\beta_1}\right) = \frac{1}{\alpha_1^2\alpha\beta} + \frac{1}{\alpha_1\beta_1\beta^2} +
  \frac{1}{\alpha_1\beta_1\alpha^2} + \frac{1}{\alpha\beta\beta_1^2}$

  $= \frac{1}{\alpha\beta}\left[\frac{1}{\alpha_1^2} + \frac{1}{\beta_1^2}\right] +
  \frac{1}{\alpha_1\beta_1}\left[\frac{1}{\alpha^2} + \frac{1}{\beta^2}\right] =
  \frac{1}{q}\left[\frac{\alpha_1^1 + \beta_1^2}{\alpha_1^2\beta_1^2}\right] +
  \frac{1}{p}\left[\frac{\alpha^2 + \beta^2}{\alpha^2\beta^2}\right]$

  $= \frac{p^3 + q^3 - pq}{p^2q^2}$. Therefore, the equation with these as roots is

  $x^2 - x + \frac{p^3 + q^3 - pq}{p^2q^2} = 0$.
\item We know that complex roots always appear in pair and as $2 + \sqrt{3}i$ is a complex root the other
  root will be its complex conjugate i.e. $2 - \sqrt{3}i$. Hence, $p = -4$ and $q = 13$ makring the equation
  $x^2 - 4x + 13 = 0$.
\item $\frac{1}{2 + \sqrt{3}} = 2 - \sqrt{3}$ which is an irrational root and the other root will be its
  conjugate i.e. $2 + \sqrt{3}$ hence the equation will be $x^2 - 4x + 1 = 0$
\item Since $\alpha, \beta$ are roots of the equation $x^2 - px + q = 0$, $\alpha + \beta = p$ and
  $\alpha\beta = q$.

  Let us assume that $\alpha + \frac{1}{\beta}$ is a root of $qx^2 - p(1 + q)x + (1 + q)^2 = 0$ then
  it must satisfy the equation. Substituting the values we have

  $\alpha\beta\frac{(\alpha\beta + 1)^2}{\beta^2} - \frac{(\alpha + \beta)(1 + \alpha\beta)(\alpha\beta +
    1)}{\beta} + (1 + \alpha\beta)^2 = 0$

  $(\alpha\beta + 1)^2[\alpha\beta - (\alpha + \beta)\beta - \beta^2] = 0$

  $\because$ L.H.S. = R.H.S. it is proven that $\alpha + \frac{1}{\beta}$ is a root of the given
  equation.
\item One of the given equations is $2x^2 + 3x - 2 = 0 \Rightarrow (2x - 1)(x + 2) = 0$ so the roots are $x
  = \frac{1}{2}, -2$. Putting these two in the equation $3x^2 + 4mx + 2 = 0$ we obtain two values
  $-\frac{7}{4}, -\frac{11}{8}$ for $m$.
\item Let $p$ be the common root then it must satisfy both the equations i.e. $p^2 - 11p + a = 0$ and $p^2 -
  14p + 2a = 0$. Equating $a$ from both equations $11p - p^2 = \frac{14p - p^2}{2} \Rightarrow p^2 - 8p = 0
  \Rightarrow p = 0, 8\Rightarrow a = 0, 24$.
\item The condition for having common roots is obtained by cross-multiplication:

  $(ba - c^2)(ca - b^2) = (a^2 - bc)^2\Rightarrow a^2bc - ab^3 - ac^3 + b^2c^2 = a^4 - 2a^2bc +
  b^2c^2\Rightarrow 3a^2bc - ab^3 -ac^3 - a^4 = 0$

  $a(3abc - b^3 - c^3 - a^3) = 0\because a\ne = 0 \Rightarrow a^3 + b^3 + c^3 - 3abc = 0\Rightarrow (a + b +
  c)(a^2 + b^2 + c^2 - ab - bc - ca) = 0$

  $\Rightarrow a + b + c = 0$ or $a = b = c$.
\item Proceeding as in last example, condition for common root is

  $(10m - 189)(9 - 10) = (21 - m)^2\Rightarrow 189 - 10m = 441 - 42m + m^2 \Rightarrow m^2 - 32m + 252 = 0
  \Rightarrow m = 18, 14$.

  Roots of $x^2 + 10x + 21 = 0$ are $-3, -7$. When $m = 18$ roots of $x^2 + 9x + 18 = 0$ are $-3, -6$.

  In that case equation formed with $-7$ and $-6$ is $x^2 + 13x + 42 = 0$ When $m= 14$ roots of $x^2 + 9x
  + 14 = 0$ are $-2, -7$.

  In that case equation formed with $-3$ and $-2$ are $x^2 + 5x + 6 = 0$.
\item Following condition for common roots, we have

  $(-3 + 120)(10 + 3) = (3 + 36)^2\Rightarrow 117 * 13 = 39^2$ which is true and thus equations have a
  common root.

  Roots of $x^2 - x - 12 = 0$ are $4, -3$ and roots of $3x^2 + 10x + 3 = 0$ are $-3, -\frac{1}{3}$ and thus
  common root is $-3$.
\item Condition for common root is given below:

  $(p - q)(3q - 2p) = (3 - 2)^2\Rightarrow (2p - 3q)(p - q) + 1 = 0\Rightarrow 2p^2 + 3q^2 - 5pq + 1 = 0$.
\item The condition for common root is $(b - c)(a - b) =  (a - c)^2$

  $\Rightarrow ab - ac - b^2 + bc = a^2 + c^2 - 2ac\Rightarrow a^2 + b^2 + c^2 - ab - ac - bc = 0$

  $\Rightarrow \frac{1}{2}(a - b)^2(b - c)^2(c - a)^2 = 0\Rightarrow a = b = c$.
\item Let $\alpha$ be the common root then

  $\frac{\alpha^2}{pq_1 - p_1q} = \frac{\alpha}{q - q_1} = \frac{1}{p_1 - p}$. Clearly, the root is either
  $\frac{pq_1 - p_1q}{q - q_1}$ or $\frac{q - q_1}{p_1 - p}$.
\item Condition for having common root is:

  $(-4b + 3c)(-6a - 2b) = (4a - 2c)^2$. Solving this gives us required equation.
\item Condition for having a common root is:

  $[(r - p)(q - r) - (p - q)^2][(p - q)(q - r) - (r - p)^2] = [(q - r)^2 - (p - q)(r - p)]^2$, which is an
  equality and hence the equations have a common root.
\item Let $\alpha$ be a common root then

  $\frac{\alpha^2}{ab^2 - ac^2} = \frac{1}{b - c} = \frac{1}{ac - ab}\Rightarrow \alpha = -a(b + c)$ or
  $\alpha = -\frac{1}{a}$.

  Let $\alpha, \beta$ be roots of first and $\alpha, \gamma$ be roots of the second equation. Then,
  $\alpha + \beta = -ab$ and $\alpha\beta = c$ also, $\alpha + \gamma = -ac$ and
  $\alpha\gamma = b$

  $\Rightarrow 2\alpha + \beta + \gamma = -a(b + c)$ and $\alpha^2\beta\gamma = bc$

  Equation formed by $\beta$ and $\gamma$ would be $x^2 - (\beta + \gamma)x + \beta\gamma = 0$.

  For either values of $\alpha$ equation is $x^2 - a(b + c)x + a^2bc = 0$.
\item Let $\alpha$ is a common root then $x^2 - px + q = 0$ and $x^2 - ax + b = 0$. Let $\beta$ be the
  second root of the first equationa then $\frac{1}{\beta}$ will be the second root of the second equation.

  Clearly, $\alpha + \beta = p, \alpha\beta = q, \alpha + \frac{1}{\beta} = a, \frac{\alpha}{\beta} = b$.

  $\therefore (q - b)^2 = (\alpha\beta - \frac{\alpha}{\beta})^2$,

  $bq(p - a)^2 = \frac{\alpha}{\beta}(\alpha\beta)(\beta - \frac{1}{\beta})^2 = (\alpha\beta -
  \frac{\alpha}{\beta})^2$. Hence, proved.
\item It is a quadratic equation but satisfied by three values of $x = 1, 2, 3$ therefore it is an
  identity.
\item It is a quadratic equation but satisfied by three values of $x = a, b, c$ therefore it is an
  identity.
\item Let $x^5 = y$ then equation becomes $3y^2 - 2y - 8 = 0$.

  Since it is satisfied by two distinct values and it is a quadratic equation therefore it is an
  equation.
\item $\frac{(x + 2)^2 - (x - 2)^2}{x^2 - 4} = \frac{5}{6}$

  $\Rightarrow \frac{8x}{x^2 - 4} = \frac{5}{6}\Rightarrow 5x^2 - 20 - 48x = 0\Rightarrow x = 10 ,
  -\frac{2}{5}$.
\item Let $x = y^2\Rightarrow \frac{2y + 1}{3 - y} = \frac{11 - 3y}{5y - 9}$

  $\Rightarrow 10y^2 - 13y - 9 = 33 - 20y + 3y^2\Rightarrow 7y^2 + 7y - 42 = 0\Rightarrow y = 2, -3$

  $\Rightarrow x = 4, 9$ but $x = 9$ does not apply to the equation and is an impossible solution.
\item $(x + 1)(x - 3)(x + 2)(x - 4) = 336\Rightarrow (x^2 - 2x - 3)(x^2 - 2x - 8) = 336$

  Let $x^2 - 2x - 3 = y\Rightarrow y(y - 5) = 336\Rightarrow ey^2 - 5y - 336 = 0\Rightarrow y = 21, -16$

  $\Rightarrow x = -4, 6, 1 \pm 2\sqrt{3}i$.
\item Squaring $x + 1 + 2x - 5 + 2\sqrt{(x + 1)(2x - 5)} = 9 \Rightarrow 2\sqrt{(x + 1)(2x - 5)} = 13 - 3x$

  Squaring again $4(x + 1)(2x - 5) = 9x^2 - 78x + 169 \Rightarrow x^2 - 66x + 189 = 0\Rightarrow x = 3, 63$.

  We see that $x = 63$ does not satisfy the equation hence the only solution is $x = 3$.
\item We have $2^{2x} + 2^{x + 2} - 32 = 0\Rightarrow (2^x - 4)(2^x + 8) = 0$. However, $2^x \neq
  8\Rightarrow 2^x = 4 \Rightarrow x = 2$.
\item Let the speed be $x$ km/hour. Then, from the statement $\frac{800}{x} = \frac{800}{x + 40} + \frac{2}{3}$

  Solving we get $x = 200$ km/hour.
\item Let width be $w$ meter. Thus, $(w + 8)(w - 2) = 119\Rightarrow w^2 + 6w - 135 = 0\Rightarrow w = 9,
  -15$ but width cannot be negative. Length is $11$ m.
\item Equivalent equation is $-x^2 + 3x + 4 = 0$ and roots are $-1, 4$.

  Since coefficient of $x^2$ is -ve the expression will be +ve if $x$ lies between the root.

  Therefore, for $-x^2 + 3x + 4 > 0$ the range is $]-1, 4[$.
\item $5x - 1 < (x + 1)^2 \Rightarrow x^2 - 3x + 2 > 0$.

  Roots of equivalent equation $x^2 - 3x + 2 = 0$ are $x = 2, 1$.

  Since coefficient of $x^2$ is positive, $x$ must lie outside the range of $[1, 2]$ for the expression
  to be positive.

  Now considering, $(x + 1)^2 < 7x - 3\Rightarrow x^2 - 5x + 4 < 0$

  Roots of the equivalent equation $x^2 - 5x + 4 = 0$ are $x = 1, 4$ and for expression to be negative $x$
  must lie inside the open interval $]1, 4[$.

  Therefore, the only integral value satisfying the original expression is $3$.
\item $\frac{8x^2 + 16x - 51}{(2x - 3)(x + 4)} > 3\Rightarrow \frac{2x^2 + x - 15}{2x^2 + 5x - 12} > 0$

  $2x^2 + x - 15 = 0$ has roots $x = -3 , \frac{5}{2}\Rightarrow 2x^2 + 5x - 12 = 0$ has roots $x = -4,
  \frac{3}{2}$

  Thus, the inequality will hold true for $x < -4$ and $-3 < x < \frac{3}{2}$ and $x > \frac{5}{2}$.
\item Let $y = \frac{x^2 - 3x + 4}{x^2 + 3x + 4}\Rightarrow (y - 1)x^2 + 3(y + 1)x + 4(y - 1) = 0$

  Since $x$ is real, the discriminant will be greater that $0 \Rightarrow 9(y + 1)^2 - 16(y - 1)^2 \ge 0$

  $-7y^2 + 50y - 7 \ge 0$. The roots are $7$ and $\frac{1}{7}$

  Since coefficient of $y^2$ is negative, for the expression to be positive $y$ has to lie between the
  open interval formed by its roots i.e. $]\frac{1}{7}, 7[$
\item Let $y = \frac{x^2 + 34x - 71}{x^2 + 2x - 7}\Rightarrow (y - 1)x^2 + 2(y - 17)x + (71 - y) = 0$

  Since $x$ is real, the discriminant will be greater that $0\Rightarrow 4(y - 17)^2 - 4(y - 1)(71 - 7y) \ge
  0$

  $\Rightarrow y^2 - 14y + 45 \ge 0$. Its roots are $5$ and $9$

  Since coefficient of $y^2$ is positive, therefore for the expression to be positive $y$ has to lie outside
  the open interval formed by its roots. Thus, the expression has no value between $5$ and $9$.
\item Let $y = \frac{4x^2 + 36x + 9}{12x^2 + 8x + 1}\Rightarrow 4(3y - 1)x^2 + 4(2y - 9)x + y - 9 = 0$.

  Since $x$ is real, the discriminant will be greater that $0\Rightarrow 16(2y - 9)^2 - 16(3y - 1)(y - 1)
  \ge 0\Rightarrow y^2 - 8y + 72 \ge 0$

  Corresponding equation is $y^2 - 8y + 72 = 0\Rightarrow D = 64 - 288 = -224 < 0$

  Since coefficient of $y^2$ is positive and discriminant is less than $0$ therefore $y^2 - 8y + 72 \ge 0$
  holds true for all value of $y$. Therefore, the expression can take any value.
\item Let $y = \frac{(x - a)(x - c)}{x -b}\Rightarrow x^2 - (a + c + y)x + ac + yb = 0$

  Since $x$ is real, the discriminant will be greater that $0$

  $\Rightarrow (a + c + y)^2 - 4(ac + yb) \ge 0\Rightarrow y^2 + 2(a + c - 2b)y + (a - c)^2 \ge 0$.

  Corresponding equation is $y^2 + 2(a + c - 2b)y + (a - c)^2 = 0$. Discriminant of above equation is $D
  = -16(a - b)(b - c)$

  If $a > b > c$ then $D < 0$ and if $a < b < c$ then also $D < 0$.

  Since coefficient of $y^2$ is positive and $D < 0$ the expression $y^2 + 2(a + c - 2b)y + (a - c)^2 \ge 0$
  is true for all real values of $y$.

  Therefore, the given expression is capable of holding any value for the given conditions.
\item Given $x + y = k$ (say, a constant). Let $z = xy$, then $z = x(k - x) \Rightarrow x^2 - kx + z = 0$.

  Since $x$ is real, $D \ge 0$ for the above equation.

  $k^2 - 4z \ge 0 \Rightarrow z \le \frac{k^2}{4}$

  Hence, the maximum value of $z = \frac{k^2}{4}$.

  Thus, $x^2 - kx + \frac{k^2}{4} = 0 \Rightarrow \left(x - \frac{k}{2}\right)^2 = 0 \Rightarrow x = \frac{k}{2}$.

  $\therefore y = \frac{k}{2}$ and thus $xy$ is maximum when $x = y$.
\item Let $y = 3 - 6x - 8x^2 \Rightarrow 8x^2 + 6x + y - 3 = 0$. Since $x$ is real, $D \ge 0$ for the this equation.

  $\Rightarrow 36 - 32(y - 3) \ge 0\Rightarrow y \le \frac{33}{8}$. Hence, maximum value of $y = \frac{33}{8}$

  $\Rightarrow 64x^2 + 48x + 9 = 0\Rightarrow (8x + 3)^2 = 0 \Rightarrow x = -\frac{3}{8}$.
\item Let $y = \frac{12x}{4x^2 + 9}\Rightarrow 4yx^2 - 12x + 9y = 0$. Since $x$ is real, $D \ge 0$ for the
  above equation.

  $\Rightarrow 144 - 144y^2 \ge 0\Rightarrow y^2 \le 1\Rightarrow -1 \le y \le 1 \Leftrightarrow |y| \le 1
  \Leftrightarrow\left|\frac{12x}{4x^2 + 9}\right| \le 1$

  Now, $\left|\frac{12x}{4x^2 + 9}\right| = 1 \Leftrightarrow 4|x|^2 - 12|x| + 9 = 0\Rightarrow (2|x| - 3)^2
  = 0 \Rightarrow |x| = \frac{3}{2}$.
\item $x^2 + 9y^2 - 4x + 3 = 0$. Since $x$ is real, $D \ge 0$ for the above equation.

  $\Rightarrow (-4)^2 - 4(9y^2 + 3) \ge 0\Rightarrow 9y^2 - 1 \le 0 \Leftrightarrow y^2 \le
  \frac{1}{9}\Rightarrow -\frac{1}{3} \le y \le \frac{1}{3}$

  The given equation can also be written as $9y^2 + x^2 - 4x + 3 = 0$. Since $y$ is real, $D \ge 0$ for the
  above equation.

  $\Rightarrow -36(x^2 - 4x + 3) \ge 0\Rightarrow x^2 - 4x + 3 \le 0$

  Since coefficient of $x^2$ is positive, it must lie between its root for the above expression to be negative.
  Therefore, $x$ must lie between $1$ and $3$.
\item Given expression is $x^2 - ax + 1 - 2a^2 > 0$

  Since $x$ is real the discriminant of the corresponding equation has to be negative for it to be positive
  for all values of $x$.

  $a^2 - 4(1 - 2a^2) < 0 \Leftrightarrow 9a^2 \le 4\Rightarrow -\frac{2}{3} < a < \frac{2}{3}$.
\item Let $\alpha$ be a common factor, therefore it will satisfy both the equations.

  $\alpha^2 - 11\alpha + a = 0$ and $\alpha^2 - 14\alpha + 2a = 0$. By cross-multiplication

  $\frac{\alpha^2}{-22a + 14x} = \frac{\alpha}{a - 2a} = \frac{1}{-14 + 11}\Rightarrow \frac{\alpha^2}{-8a}
  = \frac{\alpha}{-a} = -\frac{1}{3}$

  From first two we have $\alpha = 8$ and from last two we have $\alpha = \frac{a}{3}\therefore a = 24$.
\item $y = mx$ is a factor of $ax^2 + bxy + cy^2$ means $ax^2 + bxy + cy^2$ will be zero when
  $y = mx$.

  $ax^2 + bx.mx + cm^2x^2 = 0 \Rightarrow cm^2 + bm + a = 0$. Similarly, $a_1m^2 + b_1m + c_1 = 0$ since $my
  - x$ is a factor of $a_1x^2 + b_1xy + c_1y^2$

  Solving these two equations in $m$ by cross-multiplication $\frac{m^2}{bc_1 - ab_1} = \frac{m}{aa_1 -
    cc_1} = \frac{1}{cb_1 - ba_1}$

  From first two we get, $m = \frac{bc_1 - ab_1}{aa_1 - cc_1}$, and from last two we get, $m = \frac{aa_1 -
    cc_1}{cb_1 - ba_1}$

  Equating the two values of $m$ obtained, we get $(bc_1 - ab_1)(cb_1 - ba_1) = (aa_1 - cc_1)^2$.
\item We know that $ax^2 + 2hxy + by^2 + 2gx + 2fy + c$ can be resolved into two linear factors if and only if

  $abc + 2fgh - af^2 - bg^2 - ch^2 = 0$ and $h^2 - ab > 0$. Given expression is $2x^2 + mxy + 3y^2 - 5y - 2$

  Here, $a = 2, h = \frac{m}{2}, b = 3, g = 0, f = \frac{-5}{2}, c = -2\Rightarrow h^2 - ab = \frac{m^2}{4}
  - 6 > 0\Rightarrow m^2 > 24$

  Applying the second condition, $-12 - \frac{25}{2} + \frac{m^2}{2} = 0\Rightarrow m^2 = 49 \therefore m =
  \pm 7$.
\item Given expression is $ax^2 + by^2 + cz^2 + 2ayz + 2bzx + 2cxy$

  $= z^2\left[a\left(\frac{x}{z}\right)^2 + b\left(\frac{y}{z}\right)^2 + c + 2a\frac{y}{z} + 2b\frac{x}{z}
  + 2c\frac{xy}{z^2}\right]$

  $= z^2(aX^2 + bY^2 + c + 2aY + 2bX + 2cXY)$ where $X = \frac{x}{z}, Y = \frac{y}{z}$. Now this will
  resolve in linear factors if

  $abc + 2abc - a.a^2 - b.b^2 -c.c^2 = \Rightarrow a^3 + b^3 + c^3 = 3abc$.
\item Given expression is $2x^2 - y^2 - x + xy + 2y -1$

  Corresponding equation is $2x^2 - y^2 - x + xy + 2y -1 = 0\Rightarrow x = \frac{1 - y \pm \sqrt{(1 - y)^2
      + 8(y^2 - 2y + 1)}}{4}\Rightarrow x = 1 - y, -\frac{1 - y}{2}$.

  Therefore, required linear factors are $x + y - 1$ and $2x - y + 1$.
\item Corresponding quadratic equation is $x^2 + 2(a + b + c)x + 3(ab + bc + ca) = 0$. It will be a perfect
  square if its discriminant is zero.

  $\Rightarrow 4(a + b + c)^2 - 4.3(ab + bc + ca) = 0\Rightarrow a^2 + b^2 + c^2 - ab - bc - ca =
  0$

  $\Rightarrow \frac{1}{2}(a - b)^2(b - c)^2(c - a)^2 = 0\Rightarrow a = b = c$.
\item Discriminant of the given equation is $D = 36 - 72 < 0$.

  Now since coefficient of $x^2$ is less than zero the expression is always positive.
\item $8x - 15 - x^2 > 0\Rightarrow x^2 - 8x + 15 < 0 \Rightarrow (x - 3)(x - 5) < 0$.

  The above is true if $x$ lies in the open interval $]3, 5[$.
\item $-x^2 + 5x - 4 > 0\Rightarrow x^2 - 5x + 4 < 0\Rightarrow (x - 4)(x - 1) < 0$.

  The above is true if $x$ lies in the open interval $]1, 4[$.
\item $x^2 + 6x - 27 > 0\Rightarrow (x + 9)(x - 3) > 0$. This is true if $x < -9$ or $x > 3$.
\item $\frac{4x}{x^2 + 3} \le 1 \Rightarrow x^2 + 3 \le 4x\Rightarrow x^2 - 4x + 3 \le 0$

  $\Rightarrow (x - 3)(x - 1) le 0$, This is true for closed interval $[1, 3]$.
\item $x^2 - 3x + 2 > 0\Rightarrow (x - 2)(x - 1) > 0$. This is true for $x > 2$ or $x < 1$.

  $x^2 - 3x - 4 \le 0\Rightarrow (x - 4)(x + 1) \le 0$. This is true for $-1 \le x \le 4$.

  Thus values of $x$ which satisfy both are $-1 \le x < 1$ and $2 < x \le 4$.
\item Since roots of $ax^2 + bx + c$ are imaginary, therefore discriminant is negative. $\Rightarrow b^2 -
  4ac < 0$.

  Discriminant of $a^2x^2 + abx + ac$ is $D = a^2b^2 - 4a^3c = a^2(b^2 - 4ac) < 0$.

  But coefficient of the expression is positive hence it will be always positive.
\item Let $y = \frac{x^2 - 2x + 4}{x^2 + 2x + 4}\Rightarrow (y - 1)x^2 + 2(y + 1)x + 4(y - 1) = 0$

  Since $x$ is real discriminant will be greater or equal to zero.

  $\Rightarrow 4(y + 1)^2 - 16(y - 1)^2 \ge 0\Rightarrow y^2 + 2y + 1 - 4y^2 + 8y - 4 \ge 0\Rightarrow -3y^2
  + 10y - 3 \ge 0$.

  Roots of corresponding equation are $\frac{1}{3}, 3$. Since coefficient of $y^2$ is negative, for above to be true
  $y$ must lie between $\frac{1}{3}$ and $3$.
\item Let $y = \frac{2x^2 - 3x + 2}{2x^2 + 3x + 2}\Rightarrow 2(y - 1)x^2 + 3(y + 1)x + 2(y - 1) = 0$

  Since $x$ is real discriminant will be greater or equal to zero.

  $\Rightarrow 9(y + 1)^2 - 16(y - 1)^2 \ge 0\Rightarrow 9y^2 + 18y + 9 - 16y^2 + 32y - 16 \ge
  0\Rightarrow -7y^2 + 50y - 7 \ge 0$

  Roots of the corresponding equation are $\frac{1}{7}, 7$. Since coefficients of $y^2$ is negative, for the above to
  be true $y$ must lie between $\frac{1}{7}$ and $7$.
\item Let $y = \frac{x^2 - 2x + p^2}{x^2 + 2x + p^2}\Rightarrow (y - 1)x^2 + 2(y + 1)x + (y - 1)p^2 = 0$.

  Since $x$ is real, discriminant of above equation has to be greater or equal to zero.

  $\Rightarrow 4(y + 1)^2 - 4p^2(y - 1)^2 \ge 0\Rightarrow (1 - p^2)y^2 + 2(1 + p^2)y + 1 - p^2 \ge 0$

  Since $p > 1$ coefficient of $y^2$ is negative and thus $y$ must lie between its roots for
  the above to be true.

  The roots are $y = \frac{-2(1 + p^2) \pm \sqrt{4(1 + p^2)^2 - 4(1 - p^2)^2}}{2(1 - p^2)}$

  $y = \frac{p - 1}{p + 1}, \frac{p + 1}{p - 1}$.
\item Let $y = \frac{(x - 1)(x + 3)}{(x - 2)(x + 4)}\Rightarrow y = \frac{x^2 + 2x - 3}{x^2 + 2x -
  8}\Rightarrow (y - 1)x^2 + 2(y - 1)x^2 + 3 - 8y = 0$.

  Since $x$ is real, discriminant must be greater than or equal to $0$.

  $4(y - 1)^2 + 4(y - 1)(8y - 3) \ge 0\Rightarrow y^2 - 2y + 1 + 8y^2 - 11y + 3 \ge 0\Rightarrow 9y^2 - 13y
  + 4 \ge 0$.

  For above to be true $y$ must not lie between $1$ and $\frac{4}{9}$.
\item Let $y = \frac{x + a}{x^2 + bx + c^2}\Rightarrow yx^2 + (by - 1)x - a + c^2y = 0$.

  Since $x$ is real, discriminant must be greater than or equal to $0$.

  $\Rightarrow (by - 1)^2 - 4y(c^2y - a) \ge 0\Rightarrow b^2y^2 - 2by + 1 + 4ay - 4c^2y^2 \ge 0\Rightarrow
  (b^2 - 4c^2)y^2 + 2(2a - b)y + 1 \ge 0$.

  Discriminant of corresponding equation is $D = 4(2a - b)^2 - 4(b^2 - 4c^2) = 4[4a^2 + b^2 - 4ab - b^2 +
    4c^2] = 16(a^2 + c^2 - ab)$.

  Given $b^2 > 4c^2$ and $a^2 + c^2 > ab$ therefore $D < 0$ and coefficient of $y^2$ is negative.
  Therefore, $y$ is capable of assuming any value.
\item Let $y = \frac{x^2 - bc}{2x - b - c}\Rightarrow x^2 - 2yx + (b + c)y - bc = 0$

  Since $x$ is real, discriminant must be greater than or equal to $0$.

  $\Rightarrow 4y^2 - 4(b + c)y + 4bc \ge 0\Rightarrow y^2 - (b + c)y + bc \ge 0$.

  For above to be true $y$ must not lie between $b$ and $c$.
\item Given $x^2 - xy + y^2 - 4x - 4y + 16 = 0\Rightarrow x^2 - (y + 4)x + y^2 - 4y + 16 = 0$

  Since $x$ is real, discriminant has to be greater than or equal to $0$.

  $\Rightarrow (y + 4)^2 - 4(y^2 - 4y + 16) \ge 0\Rightarrow y^2 + 8y + 16 - 4y^2 + 16y - 64 \ge 0$

  $\Rightarrow -3y^2 + 24y - 48 \ge 0\Rightarrow y^2 - 8y + 16 \le 0\Rightarrow (y - 4)^2 \le 0$

  The above inequality is only satisfied by $y = 4$. However, if $y = 4$ the given equation becomes

  $x^2 - 8x + 16 = 0$ which is again only satisfied by $x = 4$.
\item Given $x^2 + 12xy + 4y^2 + 4x + 8y + 20 = 0\Rightarrow x^2 + 4(1 + 3y)x + 4(y^2 + 2y + 5) = 0$

  Since $x$ is real, discriminant has to be greater than or equal to zero.

  $\Rightarrow 16(1 + 3y)^2 - 16(y^2 + 2y + 5) \ge 0\Rightarrow 1 + 6y + 9y^2 - y^2 - 2y - 5 \ge 0$

  $\Rightarrow 8y^2 + 4y - 4 \ge 0\Rightarrow 2y^2 + y - 1 \ge 0 \Rightarrow (2y - 1)(y + 1) \ge 0$

  Therefore, $y$ cannot lie between $-1$ and $\frac{1}{2}$. Rewriting the equation in terms of $y$

  $4y^2 + 4(3x + 2)y + x^2 + 4x + 20 = 0$.

  Since $x$ is real, discriminant has to be greater than or equal to zero.

  $\Rightarrow (3x + 2)^2 - x^2 - 4x - 20 \ge 0\Rightarrow 8x^2 + 8x - 16 \ge 0 \Rightarrow x^2 + x - 2 \ge
  0$

  Therefore, $x$ cannot lie between $-2$ and $1$.
\item Let $x$ be the length and $y$ be the breadth then $x + 2y = 600$ and we have to maximize $xy$.

  $xy = x\frac{600 - x}{2} = z$ (say) $x^2 - 600x + 2z = 0$.

  Since $x$ is real, discriminant has to be greater than or equal to zero.

  $\Rightarrow 360000 - 8z \ge 0\Rightarrow z \le 45000$. Thus, maximum area is $45000$ mt. sq.

  Substituting, $x^2 - 600x + 90000 = 0\Rightarrow (x - 300)^2 = 0 \Rightarrow x = 300 \Rightarrow y = 150$.
\item If $y - mx$ is a factor then equation reduces to $bm^2 + 2hm + a = 0$ and if $my + x$ is a factor then
  it reduces to $am^2 - 2hm + b = 0$. By cross-multiplication we have

  $\frac{m^2}{-2h(a + b)} = \frac{m}{a^2 - b^2} = \frac{1}{2h(a + b)}$. Thus, condition becomes $a + b = 0$
  or $4h^2 + (a^2 - b^2) = 0$.
\item Roots of equation $P(x)Q(x) = 0$ will be the roots of equation $P(x) = 0$ i.e. $ax^2 + bx + c = 0$
  and $Q(x) = -ax^2 + bx + c = 0$

  Let $D_1$ and $D_2$ be the discriminants of two equations, then $D_1 + D_2 = b^2 - 4ac + b^2 + 4ac = 2b^2
  > 0$.

  Hence, $P(x)Q(x) = 0$ has at least two real roots.
\item Let $D_1$ be the discriminant of $bx^2 + (b - c)x + b - c - a = 0$ and $D_2$ be discriminant of
  $ax^2 + 2bx + b = 0$, then

  $D_1 + D_2 = (b - c)^2 - 4b(b - c - a) + 4b^2 - 4ab = (b + c)^2 \ge 0$. Hence, if $D_2 < 0$, then $D_1 >
  0$.

  Therefore, roots of $bx^2 + (b - c)x + b - c - a = 0$ will be real if roots of $ax^2 + 2bx + b = 0$ are
  imaginary and vice versa.
\item Let $a = 2m + 1, b = 2n + 1, c = 2r + 1$. Now $D = (2n + 1)^2 - 4(2m + 1)(2r + 1)$

  $= (\text{an odd number}) - (\text{an even number}) =$ an odd number.

  If possible, let $D$ be a perfect square then it has to be square of an odd number.

  $\Rightarrow (2k + 1)^2 = (2n + 1)^2 - 4(2m + 1)(2r + 1)\Rightarrow (2m + 1)(2r + 1) = (n + k + 1)(n - k)$.

  If $n$ and $k$ are both odd or even then $n - k$ will be even or zero. However, if one is odd and
  one is even then $(n + k + 1)$ will be even. So, R. H. S. is an even while L. H. S. is an odd number. Thus,
  $D$ cannot be a perfect square. Hence, roots cannot be a rational numbers.
\item Let $D_1$ be discriminant of $ax^2 + 2bx + c = 0$ then $D_1 = 4b^2 - 4ac = 4k,$ where $k = b^2 - ac$.

  Let $D_2$ is discriminant of $(a + c)(ax^2 + 2bx + c) = 2(ac - b^2)(x^2 + 1)$

  $\Rightarrow D_2 = 4(a + c)^2b^2 - 4(a^2 + b^2 + k)(b^2 + c^2 + k) = -D1[4b^2 + (a - c)^2] \Rightarrow D_2
  < 0 \because D_1 > 0$.

  Therefore, roots of second equation are non-real complex numbers.
\item $D = 4[(C_r^^n)^2 - C_{r - 1}^^nC_{r + 1}^^n] = 4(a - b),$ where $a = (C_r^^n)^2, b = C_{r - 1}^^nC_{r
  + 1}^^n$

  $\Rightarrow \frac{a}{b} = \left(1 + \frac{1}{r}\right)\left(1 + \frac{1}{n - r}\right) > 1\Rightarrow a >
  b \Rightarrow D > 0$.

  Thus, roots of given equation are real and distinct.
\item Let $y = e^{\sin x}$ then given equation becomes

  $y - \frac{1}{y} - 4 = 0\Rightarrow y = 2\pm \sqrt{5} \therefore e^{\sin x} = 2 \pm \sqrt{5}$

  $\sin x = \log_e (2 - \sqrt{5})$ is not defined.

  $\sin x = \log_e (2 + \sqrt{5}) > 1$ is not possible. Hence, roots of given equation cannot be real.
\item Given equation is $az^2 + bz + c + i = 0$. $z = \frac{-b \pm \sqrt{b^2 - 4a(c + i)}}{2a} = \frac{-b
  \pm(p + iq)}{2a}$

  where $\sqrt{b^2 - 4a(c + i)} = p + iq$. Now $b^2 - 4ac = p^2 - q^2$ and $-4a = 2qp$

  Since $z$ is purely imaginary $\frac{-b \pm p}{2a} = 0 \Rightarrow \pm p = b\Rightarrow -4a = 2(\pm)q
  \Rightarrow q = \pm \frac{2a}{b}$

  Then, $b^2 - 4ac = b^2 - \frac{4a^2}{b^2}\Rightarrow c = \frac{a}{b^2} \Rightarrow a = b^2c$.
\item $D = a^2 - 4b$. Let $a$ be an odd number then $D$ is an odd number and a perfect square as roots
  are rational. Let $D = (2n + 1)^2,$ and $a = 2m + 1$ where $m, n \in I$.

  Now roots $= \frac{-(2m + 1)\pm (2n + 1)}{2} = \frac{\text{an even no.}}{2} = \text{an integer}$.

  Similarly, it can be proven when $a$ is an even no. then roots are integers.
\item Let $\alpha, \beta$ be integral roots of the given equation. $\alpha + \beta = -7$ and $\alpha\beta =
  14(q^2 + 1)$.

  $\frac{\alpha\beta}{7} = 2(q^2 + 1) = \text{an integer}$.

  $\therefore \alpha\beta$ is divisible by $7$ and $7$ is a prime number.

  $\therefore$ at least one of $\alpha$ and $\beta$ must be a multiple of $7$.

  Let $\alpha = 7k,$ where $k \in I\Rightarrow \beta = -7(k + 1)$

  Thus, $-\frac{2(q^2 + 1)}{7} = -7k(k + 1) = \text{an integer}$

  Let $f(q) = q^2 + 1$ then it can be shown that $f(1), f(2), ..., f(7)$ are not divisible by $7$.

  $f(q + 7) = q^2 + 1 + 14q + 49$ which is not divisible by $7$ as $q^2 + 1$ is not divisible by $7$.

  Hence, $\alpha, \beta$ cannot be integers.
\item Given equation is $[a^3(b - c) + b^3(c - a) + c^3(a - b)]x^2 - [a^3(b^2 - c^2) + b^3(c^2 - a^2) +
  c^3(a^2 - b^2)]x + abc[a^2(b - c) + b^2(c - a) + c^2(a - b)] = 0$

  But $a^3(b - c) + b^3(c - a) + c^3(a - b) = -(a - b)(b - c)(c - a)(a + b + c)$ and $a^3(b^2 - c^2) +
  b^3(c^2 - a^2) + c^3(a^2 - b^2) = -(a - b)(b - c)(c -a)(ab + bc + ca)$ and $a^2(b - c) + b^2(c - a) +
  c^2(a - b) = -(a - b)(b - c)(c - a)$ the above equation becomes

  $(a + b + c)x^2 - (ab + bc + ca)x + abc = 0$.

  Roots are $\frac{(ab + bc + ca \pm \sqrt{(ab + bc + ca)^2 - 4abc(a + b + c)})}{2(a + b + c)}$, which will
  be equal if $D = 0$.

  If $\frac{1}{\sqrt{a}}\pm \frac{1}{\sqrt{b}}\pm \frac{1}{c} = 0\Rightarrow \frac{\sqrt{bc} \pm \sqrt{ca}
    \pm \sqrt{ab}}{\sqrt{abc}} = 0$

  $\Rightarrow \sqrt{bc} \pm \sqrt{ca} \pm \sqrt{ab} = 0$. Squaring

  $bc + ca + ab \pm 2\sqrt{abc}(\sqrt{a}\pm \sqrt{b} \pm \sqrt{c}) = 0\Rightarrow (bc + ca + ab)^2 = 4abc(a
  + b + c + \sqrt{bc} \pm \sqrt{ca} \pm \sqrt{ab})\Rightarrow D = 0$ i.e. roots are equal.
\item Product of roots $= \frac{k + 2}{k} = \frac{c}{a}\Rightarrow k = \frac{2a}{c - a}$

  Sum of roots $= \frac{k + 1}{k} + \frac{k + 2}{k + 1} = -\frac{b}{a}$. Substituting for $k$

  $\frac{c + a}{2a} + \frac{2c}{c + a} = - \frac{b}{a}\Rightarrow \frac{(a + c)^2 + 4ac}{2a(a + c)} = -\frac{b}{a}$

  $\Rightarrow a(a + c)^2 + 4a^2c = -2abc - 2a^2b\Rightarrow (a + c)^2 + 4ac = -2bc - 2ab\Rightarrow (a + b
  + c)^2 = b^2 - 4ac$.
\item Given, $f(x) = ax^2 + bx + c$ and that $\alpha,\beta$ are the roots of the equation $px^2 + qx +
  r = 0$.

  $\Rightarrow \alpha + \beta = -\frac{q}{p}$ and $\alpha\beta = \frac{r}{p}$.

  Now $f(\alpha)f(\beta) = (a\alpha^2 + b\alpha + c)(a\beta^2 + b\beta + c)$

  $= a^2\alpha^2\beta^2 + b^2\alpha\beta + c^2 + ab\alpha\beta(\alpha + \beta) + ac(\alpha^2 + \beta^2) +
  bc(\alpha + \beta)$

  $= a^2\frac{r^2}{p^2} + b^2\frac{r}{p} + c^2 - ab\frac{r}{p}\frac{q}{p} + ac\left(\frac{q^2}{p^2} -
  \frac{2r}{p}\right) - bc\frac{q}{p}$

  $= \frac{1}{p^2}[a^2r^2 + b^2rp + c^2p^2 - abrq + acq^2 - 2acrp - bcqp] = \frac{1}{p^2}[(cp - ar)^2 +
    b^2rp - bcqp - abrq + acq^2]$

  $= \frac{1}{p^2}[(cp - ar)^2 - (bp - aq)(cq - br)]$

  Now since $\alpha, \beta$ are the roots of the equation $px^2 + qx + r = 0$

  Therefore, if $ax^2 + bx + c = 0$ and $px^2 + qx + r = 0$ have to have a common root then it has to be
  either $\alpha$ or $\beta$.

  $f(\alpha) = 0$ or $f(\beta) = 0 \therefore f(\alpha)f(\beta) = 0\Rightarrow (cp - ar)^2 - (bp - aq)(cq -
  br) = 0$

  $\therefore bp - aq, cp - ar, cq - br$ are in G. P.
\item From the given equations it follows that $q$ and $r$ are roots of the equation

  $a(p + x)^2 + 2bpx + c = 0 \Rightarrow ax^2 + 2(a + b)px + c = 0$.

  Product of roots $qr = \frac{ap^2 + c}{a} = p^2 + \frac{c}{a}$
\item Since $\alpha, \beta$ are the roots of the equation $x^2 - px - (p + c) = 0$

  $\alpha + \beta = p$ and $\alpha + \beta = -(p + c)$. Now $(\alpha + 1)(\beta + 1) = -p - c + p + 1 = 1 - c$.

  $\Rightarrow \frac{\alpha^2 + 2\alpha + 1}{\alpha^2 + 2\alpha + c} + \frac{\beta^2 + 2\beta + 1}{\beta^2 +
  2\beta + c} = \frac{(\alpha + 1)^2}{(\alpha + 1)^2 - (1 - c)} + \frac{(\beta + 1)^2}{(\beta + 1)^2 - (1 -
    c)}$

  $= \frac{(\alpha + 1)^2}{(\alpha + 1)^2 - (\alpha + 1)(\beta + 1)} + \frac{(\beta + 1)^2}{(\beta + 1)^2 -
    (\alpha + 1)(\beta + 1)} = \frac{(\alpha + 1)^2}{(\alpha + 1)(\alpha - \beta)} + \frac{(\beta +
  1)^2}{(\beta + 1)(\beta - \alpha)} = 1$. Hence, proved.
\item $\alpha, \beta$ are the roots of the equation $x^2 + px + q = 0$. $\therefore \alpha + \beta = -p$ and
  $\alpha\beta = q$

  Since $\alpha, \beta$ are the roots of the equation $x^2{2n} + p^nx^n + q^n = 0$.

  Substituting it follows that $\alpha^n, \beta^n$ are the roots of the equation $y^2 + p^ny + q^n = 0$

  $\therefore \alpha^n + \beta^n = (-p)^n$ and $\alpha^n\beta^n = q^n\Rightarrow (\alpha + \beta)^n = (-p)^n
  = p^n [\because\;\text{n is even}]$.

  Thus, $\alpha^n + \beta^n + (\alpha + \beta)^n = 0$

  Dividing by $\beta^n$ we have $\left(\frac{\alpha}{\beta}\right)^n + 1 + \left(\frac{\alpha}{\beta} + 1\right)^n = 0$

  Dividing by $\alpha^n$ we have
  $\left(\frac{\beta}{\alpha}\right)^n + 1 + \left(\frac{\beta}{\alpha} + 1\right)^n = 0$

  From last two equations it is evident that $\frac{\alpha}{\beta}$ and $\frac{\beta}{\alpha}$ are roots
  of the equation $x^n + 1 + (x + 1)^n = 0$.
\item Let $\alpha$ and $\beta$ are the roots of the given equation.

  Since roots are real and distinct $D > 0 \Rightarrow a^2 - 4b > 0 \Rightarrow b < \frac{a^2}{4}$

  Again it is given that $|\alpha - \beta| < c \Rightarrow (\alpha - \beta)^2 < c^2$

  $(\alpha + \beta)^2 - 4\alpha\beta < c^2 \Rightarrow a^2 - 4b < c^2 \Rightarrow 4b > a^2 - c^2\Rightarrow
  \frac{a^2 - c^2}{4} < b < \frac{a^2}{4}$.
\item Given, $ax^2 + bx + c - p = 0$ for two integral values of $x$ say $\alpha$ and $\beta$.

  Then, $\alpha + \beta = -\frac{b}{a}$ and $\alpha\beta = \frac{c - p}{a}$

  If possible, let $ax^2 + bx + c - 2p = 0$ for some integer $k$.

  $ak^2 + bk + c - p = p \Rightarrow k^2 - (\alpha + \beta)k + \alpha\beta = \frac{p}{a}\Rightarrow (k -
  \alpha)(k - \beta) =\;\mtext{an integer}\;= \frac{p}{a}$

  But since $p$ is prime this cannot hold true unless $a = p$ or $a = 1$

  $a = p [\because a > 1]\Rightarrow (k - \alpha)(k - \beta) = 1$ which implies that $k - \alpha = k - \beta
  = 1$,
  which is not possible since $\alpha \ne \beta$

  Thus, we have a contradiction. Hence, $ax^2 + bx + c \ne 2p$ for any integral value of $x$.
\item $\alpha + \beta = -p, \alpha\beta = q, \alpha^4 + \beta^4 = r, \alpha^4\beta^4 = s$

  Let $D$ be the discriminant of $x^2 - 4qx + 2q^2 - r = 0$ then

  $D = 16q^2 - 4(2q^2 - r) = 8q^2 + 4r = 8\alpha^2\beta^2 + 4(\alpha^4 + \beta^4) = 4(\alpha^2 + \beta^2)^2$

  $D \ge 0$ hence roots of the third equation are always real.
\item $\alpha + \beta = -\frac{b}{a}$ and $\alpha\beta = \frac{c}{a}\Rightarrow \alpha_1 - \beta =
  -\frac{b_1}{a_1}$ and $-\alpha_1\beta = \frac{c_1}{a_1}$

  $\Rightarrow \alpha + \alpha_1 = -\left(\frac{b}{a} + \frac{b_1}{a_1}\right)$.

  Also, dividing $\alpha + \beta$ by $\alpha\beta, \frac{1}{\beta} + \frac{1}{\alpha} = -\frac{b}{c}$

  Similarly, dividing $\alpha_1 - \beta$ by $-\alpha_1\beta, \frac{1}{\alpha_1} - \frac{1}{\beta} =
  -\frac{b_1}{c_1}$

  Thus, $\frac{1}{\alpha} + \frac{1}{\alpha_1} = -\left(\frac{b}{c} + \frac{b_1}{c_1}\right)$

  Equation whose roots are $\alpha$ and $\alpha_1$ is

  $x^2 - (\alpha + \alpha_1)x + \alpha\alpha_1 = 0\Rightarrow \frac{x^2}{-(\alpha + \alpha_1)} + x -
  \frac{\alpha\alpha_1}{\alpha + \alpha_1} = 0$

  $\frac{x^2}{\frac{b}{a} + \frac{b_1}{a_1}} + x + \frac{1}{\frac{b}{c} + \frac{b_1}{c_1}} = 0$.
\item Let $\alpha$ and $\beta$ be roots of such quadratic equation given by $x^2 + px + q = 0$

  $\Rightarrow \alpha + \beta = -p$ and $\alpha\beta = q$. Now quadratic equation whose roots are $\alpha^2$
  and $\beta^2$ is

  $x^2 - (\alpha^2 + \beta^2)x + \alpha^2\beta^2 = 0\Rightarrow x^2 - (p^2 - 2q)x + q^2 = 0$.

  But the equation remains unchanged, therefore,

  $\frac{1}{1} = \frac{p}{p^2 - 2q} = \frac{q}{q^2}\Rightarrow q = q^2 \Rightarrow q(q - 1) = 0 \Rightarrow
  q = 0, 1$

  If $q = 0 \Rightarrow p = 0, -1$ and if $q = 1 \Rightarrow p = -2, 1$. Thus, four such quadratic equations
  are possible.
\item Given $\frac{d}{a}, \frac{e}{b}, \frac{f}{c}$ are in A. P. and $a, b, c$ are in G. P.

  Equations $ax^2 + 2bx + c = 0$ and $dx^2 + 2ex + f = 0$ will have a common root if

  $\frac{2(bf - ec)}{cd - af} = \frac{cd - af}{2(ae - bd)}\Rightarrow 4(bf - ec)(ae - bd) = (cd - af)^2$

  $4\left[\left(\frac{f}{c} - \frac{e}{b}\right)bc\right]\left[\left(\frac{e}{b} - \frac{d}{a}\right)ab\right]
  = \left(\frac{d}{a} - \frac{a}{f}\right)^2a^2c^2$

  $4k.k.b^2 = 4k^2ac$ where $k$ is the c.d. of the A. P. i.e. $b^2 = ac$ which is true because $a, b, c$ are
  in G. P.
\item Let $\alpha$ be the common root and $\beta_1$ another root of $x^2 + ax + 12 = 0, \beta_2$ be
  another root of $x^2 + bx + 15 = 0$ and $\beta_3$ be a root of $x^2 + (a + b)x + 36 = 0$.

  $\Rightarrow \alpha + \beta_1 = -a$ and $\alpha\beta_1 = 12$, $\alpha + \beta_2 = -b$ and $\alpha\beta_2 =
  15$, and $\alpha + \beta_3 = -(a + b)$ and $\alpha\beta_3 = 36$.

  Thus, $2\alpha + \beta_1 + \beta_2 = \alpha + \beta_3 \Rightarrow \alpha = \beta_3 - \beta_1 - \beta_2$
  and $\alpha(\beta_3 - \beta_1 - \beta_2) = 36 - 12 - 15 = 9$

  $\Rightarrow \alpha^2 = 9 \Rightarrow \alpha = \pm 3$ but $\alpha > 0 \Rightarrow \alpha = 3$
  $\Rightarrow \beta_1 = 4, \beta_2 = 5, \beta_3 = 12$.
\item Given $m(ax^2 + 2bx + c) + px^2 + 1qx + r = n(x + k)^2$. Equating coefficients for powers of $x$, we get

  $ma + p = n, mb + q = nk, mc + r = nk^2\Rightarrow m(ak - b) + pk - q = 0 \Rightarrow m = -\frac{pk - q}{ak - b}$

  $\Rightarrow m(bk - c) + qk - r = 0 \Rightarrow m = -\frac{qk - r}{bk - c}$.

  Equating values for $m, (ak - b)(qk - r) = (pk - q)(bk - c)$.
\item Given equation is $x^3 - x^2 + \beta x + \gamma = 0$. Let it roots $x_1, x_2, x_3$ be $a - d, a, a +
  d$ respectively.

  $\Rightarrow a - d + a + a + d = 1 \Rightarrow a = \frac{1}{3}\Rightarrow (a - d)a + a(a + d) + (a - d)(a +
    d) = \beta \Rightarrow 3a^2 - d^2 = \beta \Rightarrow 1 - 3\beta = 3d^2$

  $(a - d)a(a + d) = \gamma \Rightarrow a(a^2 - d^2) = \gamma \Rightarrow 1 + 27\gamma = 9d^2$

  Since $d$ is real $\therefore 1 - 3\beta \ge 0 \Rightarrow \beta \le \frac{1}{3}$ and $1 + 27\gamma \ge 0
  \Rightarrow \gamma \ge -\frac{1}{27}$.
\item Let $\alpha$ be a common root, then

  $\alpha^3 + 3p\alpha^2 + 3q\alpha + r = 0\;\;\ldots\;\;$  (1) and $\alpha^2 + 2p\alpha + q = 0\;\;\ldots\;\;$ (2)

  $(1) - \alpha (2)$ gives us $\Rightarrow p\alpha^2 + 2q\alpha + r = 0\;\;\ldots\;\;$ (3)

  By cross multiplication between (2) and (3)

  $\frac{\alpha^2}{2(pr - q^2)} = \frac{\alpha}{pq - r} = \frac{1}{2(q - p^2)}$

  Equating for values of $\alpha$ we get the desired condition.
\item Let $\alpha$ be a common root, then

  $\alpha^3 + 2a\alpha^2 + 3b\alpha + c = 0\;\;\ldots\;\;$ (1) and $\alpha^3 + a\alpha^2 + 2b\alpha = 0\;\;\ldots\;\;$ (2)

  Since $c \ne 0,$ therefore $\alpha = 0$ cannot be a common root. Therefore, from (2)

  $\alpha^2 + a\alpha + 2b = 0\;\;\ldots\;\;$ (3)

  $(1) - \alpha (2) \Rightarrow a\alpha^2 + b\alpha + c = 0\;\;\ldots\;\;$ x(4)

  Solving (3) and (4) by cross-multiplication yields the desired result.
\item Given equation is $x^3 + ax + b = 0$ and $\alpha, \beta, \gamma$ be its real roots. Then we have

  $\alpha + \beta + \gamma = 0\;\;\ldots\;\;$ (1) $\alpha\beta + \beta\gamma + \alpha\gamma = a\;\;\ldots\;\;$ (2)
  $\alpha\beta\gamma = -b$

  Let $y = (\alpha - \beta)^2,$ then $y = (\alpha + \beta)^2 - 4\alpha\beta\Rightarrow y = \gamma^2 +
  \frac{4b}{\gamma}$ $\Rightarrow \gamma^3 - y\gamma + 4b = 0$.

  Also, $\gamma$ is a root of the original equation.

  $\gamma^3 + a\gamma + b = 0\Rightarrow (a + y)\gamma - 3b = 0 \Rightarrow \gamma = \frac{3b}{a + y}$

  $\Rightarrow \frac{27b^3}{(a + y)^3} + a\left(\frac{3b}{a + y}\right) + b = 0\Rightarrow y^3 + 6ay^2 +
  9a^2y + 4a^3 + 27b^2 = 0$

  We would have got same equation if we would have chosen $y = (\beta - \alpha)^2$ or $y = (\gamma - \alpha)^2$.

  Hence, product of roots $-(4a^3 + 27b^2) = (\alpha - \beta)^2(\beta - \gamma)^2(\gamma - \alpha)^2 \ge
  0\therefore 4a^3 + 27b^2 \le 0$.
\item $\alpha$ is a root of the equation $ax^2 + bx + c = 0\therefore a\alpha^2 + b\alpha + c = 0$

  Similarly, $-a\beta^2 + b\beta + c = 0$. Let $f(x) = \frac{a}{2}x^2 + bx + c = 0\Rightarrow f(\alpha) =
  -\frac{a}{2}\alpha^2$,

  and $f(\beta) = \frac{3}{2}\beta^2$$\therefore f(\alpha)f(\beta) = -\frac{3}{4}a^2\alpha^2\beta^2 < 0
  [\because \alpha,\beta \ne 0]$

  $\therefore f(\alpha)$ and $f(\beta)$ have opposite signs. Therefore, $f(x)$ will have exactly
  one root between $\alpha$ and $\beta$.
\item Let $f(x) = ax^2 + bx + c = 0$. Since equation $ax^2 + bx + c = 0$ i.e. equation $f(x) = 0$ has no
  real root, therefore, $f(x)$ will have same sign for real values of $x$.

  $\therefore f(1)f(0) > 0 \Rightarrow (a + b + c)c > 0$.
\item Let $f(x) = (x - a)(x - c) + \lambda (x - b)(x - d)$. Given $a > b > c > d$, now $f(b) = (b - a)(b -
  c) < 0$, and $f(d) = (d - a)(d - c) > 0$

  Since $f(b)$ and $f(d)$ have opposite signs, therefore equation $f(x) = 0$ will have one real
  root between $b$ and $d$.

  Since one root is real and $a, b, c, d, \lambda$ are all real the other root will also be real.
\item Let $f'(x) = ax^2 + bx + c,$ then $f(x) = a\frac{x^3}{3} + b\frac{x^2}{2} + cx + k = \frac{2ax^3 +
  3bx^2 + 4cx + 6k}{6}$,

  $\Rightarrow f(1) = \frac{2a + 3b + 6c + 6k}{6} = k$. Again, $f(0) = k$

  Thus, $f(0) = f(1)$ hence equation will have at least one root between $0$ and $1$ which implies
  that it will have a real root between $0$ and $2$.
\item Let $f(x) = \displaystyle\int (1 + \cos^8x)(ax^2 + bx + c)dx$ then $f'(x) = (1 + \cos^8x)(ax^2 + bx + c)$.

  Given, $\displaystyle\int_0^1 (1 + \cos^8 x)(ax^2 + bx + c)dx = \int_0^2 (1 + \cos^8 x)(ax^2 + bx + c)dx$.

  $\Rightarrow f(1) - f(0) = f(2) - f(0) \Rightarrow f(1) = f(2)$.

  Therefore, equation $f(x) = 0$ has at least one root between $1$ and $2$ which implies that
  $ax^2 + bx + c$ has a root between these two limits as $1 + \cos^8x \ne 0$.
\item Given equation $f(x) - x = 0$ has non-real roots where $f(x) = ax^2 + bx + c$ is a continuous function.

  $\therefore f(x) - x$  has same sign for all $x \in R$. Let $f(x) - x > 0\;\forall\;x \in R$

  $\Rightarrow f(f(x)) - f(x) > 0\;\forall\;x\in R\Rightarrow f(f(x)) - x = f(f(x)) - f(x) + f(x) - x > 0\;\forall\;x\in R$

  Hence it has no real roots.
\item Let $f(x) = ax^2 - bx + c = 0$ and that $\alpha, \beta$ be its roots. Then, $f(x) = a(x - \alpha)(x - \beta)$.

  Given $\alpha \ne \beta, 0 < \alpha < 1, 0 < \beta < 1$ and $a, b, c \in N$

  Since quadratic equation has both roots between $0$ and $1$, therefore

  $f(0)f(1) > 0$ but $f(0)f(1) = c(a - b + c) =$ an integer

  Thus, $f(0)f(1) \ge 1 \Rightarrow a\alpha(1 - \alpha)a\beta(1 - \beta) = a^2\alpha\beta(1 - \alpha)(1 -
  \beta)$.

  Let $y = \alpha(1 - \alpha) \Rightarrow \alpha^2 - \alpha + y = 0$.

  Since $\alpha$ is real $\therefore 1 - 4y \ge 0 \Rightarrow y \le \frac{1}{4} \Rightarrow \alpha =
  \frac{1}{2}\;\text{max value}$.

  Similarly, maximum value of $\beta = \frac{1}{2}$.

  Maximum value of $\therefore f(0)f(1) < \frac{a^2}{16} > 1\Rightarrow a > 4 \Rightarrow a = 5$ [least
    integral value]

  Since $ax^2 - bx + c = 0$ has real and distinct roots $\Rightarrow b^2 > 4ac\;[\because a\geq 4, c\geq 1]$

  $\Rightarrow b^2 \geq 20 \Rightarrow b\geq 5$.
\item Proceeding from previous question, $b^2 - 4ac > 0 \Rightarrow b^2 > 4.5.1 [\because c \ge 1]
  \Rightarrow b = 5\Rightarrow \log_5(abc) \ge 2$.
\item Given equation is $ax^2 + bx + 6 = 0$. Let $f(x) = ax^2 + bx + 6$

  Since the equation has imaginary roots or real and equal roots, $f(0) = 6 > 0\therefore f(x) \ge 0$ for
  all real $x$

  $\Rightarrow f(3) \ge 0 \Rightarrow 9a + 3b + 6 \ge 0\Rightarrow 3a + b \ge -2$ and hence least value is
  $-2$.
\item Let $\alpha, \beta, \gamma$ be the roots of the equation. Then,

  $f(x) = 2x^3 - \frac{\alpha + \beta + \gamma}{2}x^2 + \frac{\alpha\beta + \beta\gamma + \gamma\alpha}{2}x
  -\frac{\alpha\beta\gamma}{2} = 0$

  Clearly, all roots have to be negative for signs to be satisfied as $a, b > 0$.

  $f(0) = 4 > 0 \therefore f(1) > 0$ because sign of $f(x)$ will not change for all $x$.

  $2 + a + b + 4 > 0 \Rightarrow a + b > - 6$.
\item $f(x) = x^3 + 2x^2 + x + 5 = 0$ and $f'(x) = 3x^2 + 4x + 1$ which has roots $-1$ and
  $-\frac{1}{3}$.

  $f(0) = 5$ and $f(x)$ is increasing in $(0, \infty)$ therefore it will have no root in $[0,
  \infty[$.

  $f(-2) = 3 > 0$ and $f(-3) = -7 < 0$.

  Since $f(-2)$ and $f(-3)$ are of opposite sign therefore equation $f(x) = 0$ will have one root
  between $-2$ and $-3$ and this will be only one root as $f(x)$ is increasing in $]-\infty,
  -1]\Rightarrow [\alpha] = -3$.
\item Given equation is $(x^2 + 2)^2 + 8x^2 = 6x(x^2 + 2)$. Let $y = x^2 + 2$ then above equation becomes
  $y^2 + 8x^2 = 6xy\Rightarrow y = 4x, 2x$.

  If $y = 4x \Rightarrow x^2 - 4x + 2 = 0 \Rightarrow x = 2 \pm \sqrt{2}$.

  If $y = 2x \Rightarrow x^2 - 2x + 2 = 0 \Rightarrow x = 1 \pm i$.
\item Given equation is $3x^3 = (x^2 + \sqrt{18}x + \sqrt{32})(x^2 - \sqrt{18}x - \sqrt{32}) -
  4x^2\Rightarrow 3x^3 = x^4 - (\sqrt{18}x + \sqrt{32})^2 - 4x^2$

  $\Rightarrow x^2(3x + 4) = x^4 - 2(3x + 4)^2\Rightarrow x^2y = x^4 - 2y^2$ where $y = 3x + 4\Rightarrow y
  = -x^2, \frac{x^2}{2}$.

  If $y = -x^2 \Rightarrow x = \frac{-3 \pm \sqrt{7}i}{2}$ and if $y = \frac{x^2}{2} \Rightarrow x = 3 \pm
  \sqrt{17}$.
\item Clearly, $(15 + 4\sqrt{14})^t(15 - 4\sqrt{14})^t = (225 - 224)^t = 1$. Let $(15 + 4\sqrt{14})^t = y,$
  then $(15 - 4\sqrt{14})^t = \frac{1}{y}$.

  Substituting for the given equation

  $y + \frac{1}{y} = 30 \Rightarrow y^2 - 30y + 1 = 0\Rightarrow y = 15 \pm 4\sqrt{14}$

  If $y = 15 + 4\sqrt{14} \Rightarrow t = 1$, then $x^2 - 2|x| = 1 \Rightarrow |x|^2 - 2|x| - 1 = 0$

  $\Rightarrow |x| = 1 + \sqrt{2} \therefore x = \pm(1 + \sqrt{2})$

  If $y = 15 - 4\sqrt{14} \Rightarrow t = -1\Rightarrow |x|^2 - 2|x| + 1 = 0 \Rightarrow |x| = 1 \Rightarrow
  x = \pm1$.
\item Given equation is $x^2 - 2a|x - a| - 3a^2 = 0$. When $a = 0$ equation becomes $x^2 = 0 \Rightarrow x = 0$

  Let $a < 0$.

  Case I: When $x < a$ then equation becomes

  $x^2 + 2a(x - a) - 3a^2 = 0 \Rightarrow x^2 + 2ax - 5a^2 = 0 \Rightarrow x = -a \pm \sqrt{6}a$

  Since $x < a, x = -a - \sqrt{6}a$  is not acceptable.

  Case II: When $x > a$ the equation becomes

  $x^2 - 2ax - a^2 = 0 \Rightarrow x = a \pm \sqrt{2}a$

  Since $x > a, x = a + \sqrt{2}a$ is not acceptable.

  Clearly, $x = a$ does not satisfy the equation.
\item $x^2 - x - 6 = 0 \Rightarrow x = -2, 3$

  Case I: When $x < -2$ or $x > 3$ then $x^2 - x - 6 > 0$

  Then equation becomes $x^2 - x - 6 = x + 2 \Rightarrow x^2 - 2x - 8 = 0$

  $x = -2, 4$ but $x = -2$ is not acceptable as $x < -2$

  Case II: When $-2 < x < 3$ $x^2 - x - 6 < 0$

  Then equation becomes $-(x^2 - x - 6) = x + 2 \Rightarrow x^2 - 4 = 0 \Rightarrow x = 2$ because $x =
  -2$ is not acceptable.

  Case III: Clearly $x = -2$ satisfies the equation by $x = 3$ does not.
\item $|x + 2| = 0 \Rightarrow x = -2$ and $|2^{x + 1} - 1| = 0 \Rightarrow 2^{x + 1} = 1 \Rightarrow x = -1$

  Case I: When $x < -2$ then $x + 2 < 0$ and $2^{x + 1} - 1 < 0$

  Equation becomes $2^{-(x + 2)} - [-(2^{x + 1} - 1)] = 2^{x + 1} + 1$

  $\Rightarrow x = 3$

  Case II: When $-2 < x < 1$ then $x + 2 > 0$ and $2^{x + 1} - 1 < 0$

  Equation becomes $2^{x + 2} - [-(2^{x + 1} - 1)] = 2^{x + 1} + 1$

  $\Rightarrow x = 1$

  Case III: When $x > -1$ then $x + 2 > 0$ and $2^{x + 1} - 1 > 0$

  Equation becomes $2^{x + 2} - (2^{x + 1} - 1) = 2^{x + 1} + 1$

  $\Rightarrow x + 2 = x + 2$

  which is true for all $x$ but only values for $x > -1$ are acceptable.

  Case IV: Clearly, $x = -2$ does not satisfy the equation but $x = -1$ satisfies it.
\item Given equation is $3^x + 4^x + 5^x = 6^x$. Then,

  $\left(\frac{3}{6}\right)^x + \left(\frac{4}{6}\right)^x + \left(\frac{5}{6}\right)^x = 1$

  Clearly, $x = 3$ satisfies the equation.

  When $x > 3, \left(\frac{3}{6}\right)^x + \left(\frac{4}{6}\right)^x + \left(\frac{5}{6}\right)^x < 1$

  When $x < 3, \left(\frac{3}{6}\right)^x + \left(\frac{4}{6}\right)^x + \left(\frac{5}{6}\right)^x > 1$

  Therefore, $x = 3$ is the only solution.
\item Proceeding as previous problem $x = 2$ is the only solution.
\item $x = [x] + \{x\},$ given equation is $4\{x\} = x + [x] \Rightarrow \{x\} = \frac{2}{3}[x]$

  $\because 0 < \{x\} < 1 \therefore 0 < \frac{2}{3}[x] < 1 \Rightarrow 0 < [x] < \frac{3}{2} \Rightarrow
  [x] = 1$

  $\therefore \{x\} = \frac{2}{3} \Rightarrow x = \frac{5}{3}$.
\item Given, $[x]^2 = x(x - [x])\Rightarrow [x]^2 = ([x] + \{x\})\{x\} [\because x = [x] + \{x\}]$

  $y^2 = (y + z)z,$ where $y = [x]$ and $z = \{x\}\Rightarrow z^2 + yz - y^2 = 0 \Rightarrow z = \frac{-y
  \pm \sqrt{5}y}{2}$

  Since $0 < z < 1$ it implies that

  if $z = -\frac{\sqrt{5} + 1}{2}y$, then

  $0 > y > -\frac{2}{\sqrt{5} + 1}\Rightarrow -\frac{\sqrt{5} - 1}{2} < y < 0$ is not possible as $y$ is an
  integer.

  If $z = \frac{\sqrt{5} - 1}{2}y$ then $0 < y < \frac{2}{\sqrt{5} - 1} \Rightarrow y = 1\Rightarrow z =
  \frac{\sqrt{5} - 1}{2}$ and $x = y + z = \frac{\sqrt{5} + 1}{2}$.
\item Let $y = mx$ the equations become $x^3(1 - m^3) = 127$ and $x^3(m - m^2) = 42$.

  Dividing we get $\frac{1 - m^3}{m - m^2} = \frac{127}{42}\Rightarrow \frac{1 + m + m^2}{m} =
  \frac{127}{42} [\because m = 1]$ does not satisfy the equations.

  $\Rightarrow m = \frac{7}{6}, \frac{6}{7}$. Substituting we get $x = -6, y = -7$ and $x = 7, y = 6$.
\item Solving first two equations by cross-multiplication

  $\frac{x}{7} = \frac{y}{7} = \frac{z}{7}$ or $x = y = z = k$.

  Substituting in third equation $k = \pm \sqrt{7}$.
\item Let $x = u + v$ and $y = u - v$ then first equation becomes $(u + v)^4 + (u - v)^4 = 82$

  $\Rightarrow u^4 + 6u^2v^2 + v^4 = 41$

  Second equation becomes $2u = 4 \Rightarrow u = 2$. Substituting in this equation $v = \pm 5i, \pm 1$

  $\therefore x = 2 \pm 5i, 3, 1$ and $y = 2\mp 5i, 1, 3$.
\item Let $y = 2^x > 0$ then give equation becomes $\sqrt{a(y - 2) + 1} = 1 - y\Rightarrow y^2 - (a + 2)y +
  2a = 0$.

  $y = 2, a$ but $y = 2$ does not satisfy the equation. When $y = a$ then $\sqrt{a(a - 2) + 1} = 1 - a
  \Rightarrow a \le 1$

  $\therefore 0 < a \le 1 [\because y > 0]\Rightarrow y = a \Rightarrow x =\log_2 a,$ where $0 < a \le 1$

  When $a > 1,$ given equation has no solution.
\item Given $(x - 5)(x + m) = -2$.Since $x$ and $m$ are both integers, therefore, $x - 5$ and $x + m$ are
  also integers.

  So we have following combination of solutions:

  $x - 5 = 1$ and $x + m = 2$ then $x = 6, m = -8$

  $x - 5 = 2$ and $x + m = -1$ then $x = 7, m = -8$

  $x - 5 = -1$ and $x + m = 2$ then $x = 4, m = -2$

  $x - 5 = -2$ and $x + m = 1$ then $x = 3, m = -2$

  Thus, $m = -8, -2$.
\item Multiplying the equations we get $(xy)^{x + y} = (xy)^{2n} \therefore x + y = 2n$ where $xy \ne 1$.

  $\Rightarrow x^2 = y$ then $x + x^2 = 2n\Rightarrow x = \frac{-1 \pm \sqrt{1 + 8n}}{2}$

  But $x > 0$ $\therefore x = \frac{-1 + \sqrt{1 + 8n}}{2}\Rightarrow y = x^2 = \frac{1 + 4n - \sqrt{1 +
      8n}}{2}$.
\item Let $y = 12^{|x|},$ then given equation becomes $y^2 - 2y + a = 0\Rightarrow y = 1 \pm \sqrt{1 - a}$

  $|x| = \log_{12}(1 + \sqrt{1 - a})$ as $y = 1 - \sqrt{1 - a}$ has to be rejected as $y > 1$.

  But $\sqrt{1 - a}$ has to be real $1 - a \ge 0 \Rightarrow a \le 1$

  For $\log_{12}(1 + \sqrt{1 - a})$ to be defined $1 + \sqrt{1 - a} > 0\therefore x = \pm \log_{12}(1 +
  \sqrt{1 - a})$.
\item Let $m = 2p + 1$ and $n = 2q + 1$ the $D = 4(2p + 1)^2 - 8(2q + 1) =$ an even no.

  Let $D$ be a perfect square then it has to be perfect square of an even no. Let that no. be $2r$ then

  $4r^2 = 4(2p + 1)^2 - 8(2q + 1) \Rightarrow 2(2q + 1) = (2p + 1 - r)(2p + 1 + r)$.

  Clearly, if $r$ is an even no. then L. H. S. is an even and R. H. S. is even no which is not possible.

  Let $r$ is an odd no. then R. H. S. is product of 2 even numbers. Let $2p + 1 - r = 2k$ and $2p +
  1 + r = 2l$

  $2(2q + 1) = 4kl$ which is an odd no. $2q + 1$ having equality to even no. $2kl$ which is again not
  possible. Thus, under the given conditions equation cannot have rational roots.
\item Equation representing points of local extrema is $f'(x) = 3ax^2 + 2bx + c = 0$.

  Let one of these points is $\alpha$ and then second would be $-\alpha$.

  Sum of these roots $= \alpha - \alpha = -\frac{2b}{3a} \Rightarrow b = 0$.

  Product of roots $= -\alpha^2 = \frac{c}{3a}$ but since roots are opposite in equation it implies that
  $a$ and $c$ have opposite signs.

  $\therefore b^2 - 4ac = -4ac > 0$ therefore roots of $ax^2 + bx + c$ will have real and distinct roots.
\item Given equation is $\frac{(x - a)(ax - 1)}{x^2 - 1} = b$.

  $ax^2 - (1 + a^2)x + a = bx^2 - b \Rightarrow (a - b)x^2 - (1 + a^2)x + a + b = 0$.

  Discriminant $D^2 = (1 + a^2)^2 - a^2 + b^2 = 1 + a^2 + a^4 + b^2 > 0 [\because b \ne 0]$

  Therefore, roots can never be equal.
\item Given equation is $C_r^^nx^2 + 2C_{r + 1}^^nx + C_{r + 2}^^n = 0$. Let $D$ be discriminant, then we
  have to prove that

  $D = 4left(C_{r + 1}^^n\right) - 4\left(C_r^^n.C_{r + 2}^^n\right) > 0$

  $\Rightarrow \left[\frac{n!}{(r + 1)!(n - r - r)!}\right]^2 - \frac{n!}{r!(n - r)!}.\frac{n!}{(r + 2)!(n -
    r - 2)!} = \frac{n!^2}{r!(r + 1)!(n - r - 1)!(n - r - 2)!}\left[\frac{1}{(r + 1)(n - r - 1)} -
    \frac{1}{(n - r)(r + 2)}\right] > 0$

  $\Rightarrow nr + 2n - r^2 - 2r - [nr + n - r^2 - r - r - 1] > 0 \Rightarrow n - 1 > 0$.

  From given conditions minimum value of $n$ is $4$, hence above condition is true proving that roots are
  real.
\item $D = c^2(3a^2 + b^2)^2 + 4abc^2(6a^2 + ab - 2b^2) = c^2(9a^4 + b^4 + 6a^2b^2 + 4a^3b + 4a^2b^2 - 8ab^3)$

  $= c^2(3a^2 - b^2 + 4ab)^2$, which is a perfect square and hence roots are rational.
\item $\sqrt{\frac{m}{n}} + \sqrt{\frac{n}{m}} + \frac{b}{\sqrt{ac}} = 0$

  L.H.S. $= \sqrt{\frac{\alpha}{\beta}} + \sqrt{\frac{\beta}{\alpha}} + \frac{b}{\sqrt{ac}}$

  $= \frac{\alpha + \beta}{\sqrt{\alpha\beta}} + \frac{b}{\sqrt{ac}} =
  \frac{-\frac{b}{a}}{\sqrt{\frac{c}{a}}} + \frac{b}{\sqrt{ac}} = 0$.
\item Let $\alpha$ be the root, then the second root would be $\alpha^3$.

  Product of roots $= \alpha^4 = a \Rightarrow \alpha = a^{\frac{1}{4}}$.

  Sum of roots $= \alpha + \alpha^3 = -f(a)\Rightarrow f(a) = -a^{\frac{1}{4}} - a^{\frac{3}{4}}$.

  Therefore, the general equation in $x$ would be $f(x) = -x^{\frac{1}{4}} - x^{\frac{3}{4}}$.
\item Since $\alpha, \beta$ are roots of the equation $x^2 - px + q = 0$ therefore

  $\alpha + \beta = p$ and $\alpha\beta = q$

  $(\alpha^2 - \beta^2)(\alpha^3 - \beta^3) = (\alpha - \beta)^2(\alpha + \beta)[(\alpha^2 + \beta^2) +
  \alpha\beta] =(p^2 - 4q)p(p^2 + q)$, and

  $\alpha^3\beta^2 + \alpha^2\beta^3 = \alpha^2\beta^2(\alpha + \beta) = pq^2$

  Therefore, the equation would be

  $x^2 - p[(p^2 - 4q)(p^2 + q) + q^2]x + p^2q^2(p^2 - 4q)(p^2 + q) = 0$.
\item $\alpha + \beta = b$ and $\alpha\beta = c$. Then proceeding like previous problem,

  $(\alpha^2 + \beta^2)(\alpha^3 + beta^3) = [(\alpha + \beta)^2 - 2\alpha\beta][(\alpha + \beta)^3 -
  3\alpha\beta(\alpha + \beta)] = (b^2 - 2c)(b^3 - 3bc)$, and

  $\alpha^5\beta^3 + \alpha^3\beta^5 - 2\alpha^4\beta^4 = \alpha^3\beta^3(\alpha^2 + \beta^2 - 2\alpha\beta)
  = c^3(b^2 - 4c)$.

  Therefore, the equation would be

  $x^2 - [(b^2 - 2c)(b^3 - 3bc) + c^3(b^2 - 4c)]x + (b^2 - 2c)(b^3 - 3bc)c^3(b^2 - 4c) = 0$.
\item Let $\alpha, \beta$ be the roots then $\alpha + \beta = -\frac{b}{a}$ and $\alpha\beta = \frac{c}{a}$.

  According to the question $\alpha + \beta = \frac{1}{\alpha^2} + \frac{1}{\beta^2}$

  $\Rightarrow -\frac{b}{a} = \frac{(\alpha + \beta)^2 - 2\alpha\beta}{\alpha^2\beta^2}\Rightarrow
  -\frac{b}{a} = \frac{\frac{b^2}{a^2}}{\frac{c^2}{a^2}} - 2\frac{1}{\frac{c}{a}}$

  $\Rightarrow -\frac{b}{a} = \frac{b^2}{c^2} - 2\frac{a}{c}\Rightarrow \frac{b^2}{ac} + \frac{bc}{a^2} =
  2$.
\item Given, $T = 2\pi \sqrt{\frac{h^2 + k^2}{gh}}$. Squaring, $h^2 + k^2 = \frac{T^2gh}{4\pi^2}$

  $\Rightarrow h^2 - \frac{T^2gh}{4\pi^2} + k^2 = 0$. Clearly, $h_1$ and $h_2$ are two possible roots of
  above equation, where

  $h_1 + h_2 = \frac{T^2g}{4\pi^2}$ and $h_1h_2 = k^2$.
\item Clearly, $\alpha_1 + \alpha_2 = -p$ and $\alpha_1\alpha_2 = q, \beta_1 + \beta2 = -r$ and
  $\beta_1\beta_2 = s$.

  Solving the two equations in $y$ and $z$ by elimination we have

  $\frac{\alpha_1}{\alpha_2} = \frac{\beta_1}{\beta_2} = k\Rightarrow \frac{p^2}{r^2} = \frac{(\alpha_1 +
    \alpha_2)^2}{(\beta_1 + \beta_2)^2}$
  $= \frac{\alpha_1^2(1 + k^2)}{\beta_1(1 + k^2)} =
  \frac{\frac{\alpha_1\alpha_2}{k}}{\frac{\beta_1\beta_2}{k}} = \frac{q}{s}$.
\item $-(1 + \alpha\beta) = -(\frac{a + c}{a})$.

  H. M. of $\alpha$ and $\beta = \frac{2\alpha\beta}{\alpha + \beta} = -\frac{2c}{b}$, but since $a, b, c$
  are in H. P. it becomes

  $= -\frac{2c}{\frac{2ac}{a + c}} = -(\frac{a + c}{a}) = -(1 + \alpha\beta)$.
\item Given equation is $x + 1 = \lambda x - \lambda^2x^2\Rightarrow \lambda^2x^2 + (1 - \lambda)x + 1 = 0$.

  $\Rightarrow \alpha + \beta = \frac{\lambda - 1}{\lambda^2}$ and $\alpha\beta = \frac{1}{\lambda^2}$.

  Also given that, $\frac{\alpha}{\beta} + \frac{\beta}{\alpha} = r - 2$

  $\Rightarrow \alpha^2 + \beta^2 = (r - 2)\alpha\beta\Rightarrow (\alpha + \beta)^2 = r\alpha\beta$

  $\frac{(\lambda - 1)^2}{\lambda^4} = \frac{r}{\lambda^2}\Rightarrow \lambda_1 + \lambda_2 = \frac{2}{1 -
    r}$ and $\lambda_1\lambda_2 = \frac{1}{1 - r}$.

  Now it is trivial to deduce the desired result.
\item Let $\alpha, \beta$ be roots of $ax^2 + bx + c = 0$ then

  $\alpha + \beta = -\frac{b}{a}$ and $\alpha\beta = \frac{c}{a}$.

  According to question, $\frac{1}{\alpha} + \frac{1}{\beta} = -\frac{m}{l}$ and $\frac{1}{\alpha\beta} =
  \frac{n}{l}$.

  From product of roots, $\frac{c}{a} = \frac{l}{n}$ and from sum of roots $\frac{b}{c} = \frac{m}{l}$.
\item Let the roots are $l, lm, lm^2, lm^3$ which is an increasing G. P.

  Sum of roots for first equation $= l(1 + m) = 3$

  Sum of roots for second equation $= lm^2(1 + m) = 12 \Rightarrow m^2 = 4 \Rightarrow m = 2$ because
  G. P. is increasing.

  $\Rightarrow l = 1$.

  $\Rightarrow A = l^2m = 2$ and $B = l^2m^5 = 32$.
\item For first equation, $p + q = 2$ and $pq = A.$ For second equation, $r + s = 18$ and $rs = B$.

  Let $a$ be the first term and $d$ be the common difference, then

  $p = a - 3d, q = a - d, r = a + d, s = a + 3d$.

  Substituting in sums we have $2a - 4d = 2$ and $2a + 4d = 18\therefore a = 5$ and $d = 2$

  $\therefore p = -1, q = 3, r = 7, s = 11\therefore A = -3$ and $B = 77$.
\item $\alpha + \beta = -a$ and $\alpha\beta = -\frac{1}{2a^2}$. Now, $\alpha^4 + \beta^4 = ((\alpha +
  \beta)^2 - 2\alpha\beta)^2 - 2\alpha^2\beta^2$

  $= 2 + a^4 + \frac{1}{2a^4}$.

  Let $a^4 + \frac{1}{2a^4} = y\Rightarrow 2a^8 - 2a^4y - 1 = 0$.

  Since $a$ is real. $\therefore y^2 - 2 \ge 0 \Rightarrow y \ge \sqrt{2} [\because a^4 \ge 0]\Rightarrow
  \alpha^4 + \beta^4 \ge 2 + \sqrt{2}$.
\item $\alpha + \beta = p$ and $\alpha\beta = q$.

  $\alpha^{\frac{1}{4}} + \beta^{\frac{1}{4}} = \sqrt[4]{\left(\alpha^{\frac{1}{4}} +
  \beta^{\frac{1}{4}}\right)^4}$

  $= \sqrt[4]{\alpha + \beta + 6\sqrt{\alpha\beta} + 4\sqrt[4]{\alpha\beta(\alpha^2 + \beta^2)}} =
  \sqrt[4]{p + 6\sqrt{q} + 4\sqrt[4]{q(p^2 - 2q)}}$.
\item Let $\alpha, beta$ be roots of first equation and $\gamma, \delta$ be that of second equation.

  $\alpha + \beta = \frac{b}{a}, \alpha\beta = \frac{c}{a}$ and $\gamma + \delta = \frac{c}{b},
  \gamma\delta = \frac{a}{b}$

  According to question, $\alpha - \beta = \gamma - \delta\Rightarrow (\alpha + \beta)^2 - 4\alpha\beta =
  (\gamma + \delta)^2 - 4\gamma\delta$

  $\Rightarrow \frac{b^2}{a^2} - \frac{4c}{a} = \frac{c^2}{b^2} - \frac{4a}{b}\Rightarrow b^4 - a^2c^2 =
  4ab(bc - a^2)$.
\item A cubic equation whose roots are $\alpha, \beta, \gamma$ is given by $f(x) = (x - \alpha)(x -
  \beta)(x - \gamma)$

  $\therefore f'(x) = (x - \alpha)(x - \beta) + (x - \beta)(x - \gamma) + (x - \alpha)(x - \gamma)$

  Now it is trivial to prove that a sign change occurs for the given limits for $f'(x)$ and thus a root lies
  in these limits.
\item Let $x_1, x_2, \ldots, x_n$ are the $n$ roots of the given polynomial equation. If all the roots are
  equal then we will have the relationship

  $(x_1 - x_2)^2 + (x_1 - x_3)^2 + \cdots + (x_1 - x_n)^2 + (x_2 - x_3)^2 + \cdots + (x_2 - x_n)^2 + \cdots
  + (x_{n - 1} - x_n)^2 > 0$

  $\Rightarrow (n - 1)(x_1^2 + x_2^2 + \cdots + x_n^2) - 2(x_1x_2 + x_1x_3 + \cdots + x_1x_n + x_2x_3 +
  x_2x_4 + \cdots + x_2x_n + \cdots + x_{n - 1}x_n) > 0$

  $\Rightarrow (n - 1)(x_1^2 + x_2^2 + \cdots + x_n^2) + (2n - 2)(x_1x_2 + x_1x_3 + \cdots + x_1x_n + x_2x_3 +
  x_2x_4 + \cdots + x_2x_n + \cdots + x_{n - 1}x_n) - 2n(x_1x_2 + x_1x_3 + \cdots + x_1x_n + x_2x_3 +
  x_2x_4 + \cdots + x_2x_n + \cdots + x_{n - 1}x_n) > 0$

  $\Rightarrow (n - 1)(x_1 + x_2 + \cdots + x_n)^2 - 2n(x_1x_2 + x_1x_3 + \cdots + x_1x_n + x_2x_3 +
  x_2x_4 + \cdots + x_2x_n + \cdots + x_{n - 1}x_n) > 0$

  Now from polynomial $x_1 + x_2 + \cdots + x_n = -a_1$ and $x_1x_2 + x_1x_3 + \cdots + x_1x_n + x_2x_3 +
  x_2x_4 + \cdots + x_2x_n + \cdots + x_{n - 1}x_n = a_1$.

  $\therefore (n - 1)a_1^2 - 2na_2 > 0$. But it is given that $(n - 1)a_1^2 - 2na_2 < 0$, hence all the
  roots cannot be equal.
\item Since $\alpha, \beta, \gamma, \delta$ are in A. P. let $\alpha = l - 3m, \beta = l - m, \gamma = l +
  m, \delta = l + 3m$ where $l$ is the first term and $m$ is the common difference of A. P.

  $\alpha + \beta = -\frac{b}{a}, \alpha\beta = \frac{c}{a}$ and $\gamma + \delta = -\frac{q}{p},
  \gamma\delta = \frac{r}{p}$

  $\frac{D_1}{D_2} = \frac{b^2 - 4ac}{q^2 - 4pr} = \frac{\frac{b^2}{a^2} - \frac{4c}{a}}{\frac{a^2}{p^2} -
    \frac{4r}{p}}\frac{a^2}{p^2}= \frac{(\alpha - \beta)^2}{(\gamma - \delta)^2}\frac{a^2}{p^2} =
  \frac{4d^2}{4d^2}\frac{a^2}{p^2}$.
\item R.H.S. $= \frac{q^2 - 4pr}{p^2} = \frac{q^2}{p^2} - 4\frac{r}{p} = (\alpha + \beta + 2h)^2 - 4(\alpha
  + h)(\beta + h)$

  $= (\alpha + h - \beta - h)^2 = (\alpha - \beta)^2 = (\alpha + \beta)^2 - 4\alpha\beta = \frac{b^2}{a^2} -
  4\frac{c}{a} = \frac{b^2 - 4ac}{a^2} =$ L.H.S.
\item L.H.S. $= 2h = (\alpha + h + \beta + h) - (\alpha + \beta) = -\frac{q}{p} -(\frac{b}{a}) = \frac{b}{a}
  - \frac{q}{p} =$ R.H.S.
\item $\alpha + \beta = -\frac{b}{a}, \alpha\beta = \frac{c}{a}$ and $\alpha^4 + \beta^4 = -\frac{m}{l},
  \alpha^4\beta^4 = \frac{n}{l}$.

  Discriminant of given quadratic equation, $D = 16a^2c^2l^2 - 4a2^l(2c^2l + a^2m) = 8a^2c^2l^2 - 4a^4lm$

  $= 4a^4l^2\left(2\frac{c^2}{a^2} - \frac{m}{l}\right)= 4a^4l^2(2\alpha^2\beta^2 + \alpha^4 + \beta^4) =
  2a^4l^2(\alpha^2 + \beta^2)^2$.

  Therefore, roots of the given equation can be computed which are found to be $(\alpha + \beta)^2, -(\alpha
  + \beta)^2$ which are equal and opposite in sign.
\item $\alpha + \beta = -\frac{b}{a}, \alpha\beta = \frac{c}{a}$ and $\gamma + \delta = -\frac{m}{l},
  \gamma\delta = \frac{n}{l}$

  Equation whose roots are $\alpha\gamma + \beta\delta$ and $\alpha\delta + \beta\gamma$ is

  $x^2 - (\alpha\gamma + \beta\delta + \alpha\delta + \beta\gamma)x + (\alpha\gamma +
  \beta\delta)(\alpha\delta + \beta\gamma) = 0$

  $\Rightarrow x^ - (\alpha + \beta)(\gamma + \delta)x + ((\alpha^2 + \beta^2)\gamma\delta + (\gamma^2 +
  \delta^2)\alpha\beta) = 0$

  $\Rightarrow a^2l^2x^2 - ablmx + (b^2 - 2ac)ln + (m^2 - 2ln)ac = 0$.
\item Since $p$ and $q$ are roots of the equation $x^2 + bx + c = 0$ therefore $p + q = -b$ and
  $pq = c$

  Equation whose roots are $b$ and $c$ is $x^2 - (b + c)x + bc = \Rightarrow x^2 +(p + q - pq)x - pq(p + q)
  = 0$.
\item $p$ and $q$ are roots of the equation $3x^2 - 5x - 2 = 0$.

  $\Rightarrow p + q = \frac{5}{3}$ and $pq = -\frac{2}{3}$.

  Equation whose roots are $3p - 2q$ and $3q - 2p$ is

  $x^2 - (p + q)x - 6p^2 - 6q^2 + 13pq = 0\Rightarrow 3x^2 - 5x - 100 = 0$.
\item Sum of roots $= 2\alpha = -p$ and product of roots $= \alpha^2 - \beta = q \Rightarrow \beta =
  \frac{p^2 - 4q}{4}$.

  Equation whose roots are $\frac{1}{\alpha}\pm \frac{1}{\sqrt{\beta}}$ is $x^2 - \frac{2}{\alpha}x +
  \frac{1}{\alpha^2} - \frac{1}{\beta} = 0$

  $\Rightarrow x^2 + \frac{2}{p}x + \frac{1}{p^2} - \frac{4}{p^2 - 4q} = 0\Rightarrow (p^2 - 4q)(p^2x^2 +
  4px) = 16q$.
\item Sum of roots is $\alpha^2\left(\frac{\alpha^2 - \beta^2}{\beta}\right) + \beta^2\left(\frac{\beta^2 -
  \alpha^2}{\alpha}\right)$

  $= \frac{(\alpha^2 - \beta^2)(\alpha^3 - \beta^3)}{\alpha\beta} = \frac{(\alpha + \beta)(\alpha -
  \beta)^2(\alpha^2 + \beta^2 + \alpha\beta)}{\alpha\beta}= \frac{p}{q}(p^2 - 4q)(p^2 - q)$.

  Product of roots is $-\alpha\beta(\alpha^2 - \beta^2)^2 = -q(\alpha - \beta)^2(\alpha + \beta)^2 =
  -p^2q(p^2 - 4q)$.

  Hence the equation having these as roots is $qx^2 - p(p^2 - q)(p^2 - 4q)x - p^2q^2(p^2 - 4q) = 0$.
\item Solving the system of equations, we have $u = -\frac{1}{3}, v = \frac{2}{3}$ and $w = \frac{5}{3}$.

  Now, $(b - c)^2 + (c - a)^2 + (d - b)^2 = a^2 + 2b^2 + 2c^2 + d^2 - 2bc - 2ca - 2bd$, but because $a, b,
  c, d$ are in G.P. therefore, $ad = bc, ca = b^2$ and $bd = c^2\Rightarrow a^2 + 2b^2 + 2c^2 + d^2 - 2bc -
  2ca - 2bd = (a - d)^2$.

  Rewriting the first quadratic equaiton, $\left(\frac{1}{u} + \frac{1}{v} + \frac{1}{w}\right)x^2 +[(b - c)^2 + (c -
    a)^2 + (d - b)^2]x + u + v + w = 0$ becomes

  $\Rightarrow -\frac{9}{10}x^2 + (a - d)^2x + 2 = 0 \Rightarrow 9x^2 - 10(a - d)^2x - 20 = 0$. Equation
  whose roots will be reciprocal of this equation will be $\frac{9}{x^2} - \frac{10(a - d)^2}{x} - 20 =
  0\Rightarrow 20x^2 + (a - d)^2x - 9 = 0$, which is what we had to prove.
\item Because $\alpha_1, \alpha_2, \ldots, \alpha_n$ are roots of the equation $(\beta_1 - x)(\beta_2 -
  x)\ldots (\beta_n - x) + A = 0$, therefore

  $(\beta_1 - \alpha_1)(\beta_2 - \alpha_2)\ldots (\beta_n - \alpha_n) + A = 0$.

  Therefore, equation having $\beta_1, \beta_2, \ldots, \beta_n$ as roots is

  $(x - \alpha_1)(x - \alpha_2)\ldots(x - \alpha_n) + A = 0$.
\item Given $\alpha_1, \alpha_2, \ldots, \alpha_n$ are roots of the equation $x^n + ax + b = 0$.

  $\Rightarrow (x - \alpha_1)(x - \alpha_2)\cdots(x - \alpha_n) = x^n + nax - b$

  $\displaystyle\Rightarrow \lim_{x\to\alpha_1}(x - \alpha_2)(x - \alpha_3)\cdots(x - \alpha_n) = \frac{x^n
  + nax - b}{x - \alpha_1}$

  Applying L'Hospital's rule, $(\alpha_1 - \alpha_2)(\alpha_1 - \alpha_3)\cdots(\alpha_1 - \alpha_n) = nx^{n
    - 1} + na = n(x^{n - 1} + a)$.
\item We have $1 + \alpha^2 = (\alpha + i)(\alpha - i)$ and so on for other terms of the first given root
  $(1 + \alpha^2)(1 + \beta^2)(1 + \gamma^2)(1 + \delta^2)$.

  Let $P(x) = x^4 + qx^2 + rx + t$ then $(1 + \alpha^2)(1 + \beta^2)(1 + \gamma^2)(1 + \delta^2) = P(i)P(-i)
  = (1 - q + t + ri)(1 - q + t - ri) = (1 - q + t)^2 + r^2$.

  Hence sum of $(1 + \alpha^2)(1 + \beta^2)(1 + \gamma^2)(1 + \delta^2)$ and $1$ is $(1 - q + t)^2 + r^2 +
  1$ and product is $(1 - q + t)^2 + r^2$. Thus, we deduce the equation as

  $x^2 - [(1 - q + t)^2 + r^2 + 1]x + (1 - q + t)^2 + r^2 = 0$.
\item Given $\alpha, \beta, \gamma$ are roots of $x^3 + px + q = 0$, so we have

  $\alpha + \beta + \gamma = 0, \alpha\beta + \beta\gamma + \gamma\alpha = p, \alpha\beta\gamma = -q$.

  Now sum of $\frac{\alpha + 1}{\alpha}, \frac{\beta + 1}{\beta}, \frac{\gamma + 1}{\gamma}$ is
  $\frac{3\alpha\beta\gamma + \alpha\beta + \beta\gamma + \alpha\gamma}{\alpha\beta\gamma} = \frac{3q -
    p}{q}$.

  Product of these roots taken two at a time is $\frac{3\alpha\beta\gamma + 2(\alpha\beta + \beta\gamma +
    \alpha\gamma) + \alpha + \beta + \gamma}{\alpha\beta\gamma} = \frac{3q - 2p}{q}$

  Product of all taken together is $\frac{\alpha\beta\gamma + \alpha\beta + \beta\gamma + \gamma\alpha +
    \alpha + \beta + \gamma + 1}{\alpha\beta\gamma} = \frac{q - p - 1}{q}$.

  Thus the cubic equation having these roots is $x^3 - \frac{3q - p}{q}x^2 + \frac{3q - 2p}{q}x - \frac{q -
    p - 1}{q} = 0 \Rightarrow qx^3 + (p - 3q)x^2 + (3q - 2p)x + 1 + p - q = 0$.
\item Given equations are $ax^2 + bx + c = 0$ and $a_1x^2 + b_1x + c_1 = 0$. Let $\alpha$ be the root which
  satisfies first equation and its reciprocal satisfies the second equation. Then,

  $a\alpha^2 + b\alpha + c = 0$ and $\frac{a_1}{\alpha^2} + \frac{b_1}{\alpha} ++ c_1 = 0 \Rightarrow
  c_1\alpha^2 + b_1\alpha + a_1 = 0$.

  By cross multiplication $\alpha = \frac{cc_1 - aa_1}{ab_1 - bc_1} = \frac{ba_1 - b_1c}{cc_1 - aa_1}
  \Rightarrow (aa_1 - cc_1)^2 = (bc_1 - ab_1)(b_1c - a_1b)$.
\item Let $(\alpha, \beta), (\beta, \gamma), (\gamma, \alpha)$ be three pairs of roots which satisfy the
  given equation. Then, we have

  $\alpha + \beta = -p, \beta + \gamma = -q, \alpha + \gamma = -r$, and hence, sum of all the common roots is
  obtained by adding these three equations

  $\alpha + \beta + \gamma = -\frac{p + q + r}{2}$.
\item The second equation is $(2x\sin\theta - 1)^2 = 0$ i.e. it has only one root, $x =
  \frac{1}{2\sin\theta}$. Since it has a common root with first equation and first equation has equal roots
  then that implies that first equation also has one root which is $\frac{1}{2\sin\theta}$.

  Observing that coefficients in first equation are cyclic we deduce that $x = 1$ will satisfy the
  equation. Hence, $\frac{1}{2\sin\theta} = 1 \Rightarrow \sin\theta = \frac{1}{2}$.

  $\Rightarrow \theta = n\pi + (-1)^n\frac{\pi}{6}$, is the general solution of $\theta$.
\item Let $\alpha$ is a root of $x^2 - x + a = 0$ then $2\alpha$ will be a root of $x^2 - x + 3a= 0$. Thus,

  $\alpha^2 - \alpha + a = 0$ and $4\alpha^2 - 2\alpha + 3a = 0$. By cross-multiplication, we have

  $\frac{\alpha^2}{-3a + 2a} = \frac{\alpha}{3a - 4a} = \frac{1}{-2 + 4} \Rightarrow a^2 = -2a \Rightarrow a
  = 0, -2$.

  However, it is given that $a\neq 0, \therefore a = -2$.
\item If $(x_1, y_1), (x_2, y_2)$ are the two solutions, then $y_1, y_2$ are the two solutions of the
  quadratic in $y$. Then we will have two cases:

  Case I: $x_1 = y_1, x_2 = y_2$. In this case the equation becomes $x^2 + 2lx + m = 0$ therefore $a = 2l, m
  = b$.

  Case II: $x_1 = y_2, x_2 = y_1$. In this case $x_1y_1 + l(x_1 + y_1) + m = 0$. Replacing $y_1$ with $x_2$,
  we get $b - al + m = 0$.
\item Given that roots of the equation $10x^3 - cx^2 - 54x - 27 = 0$ are in H.P. Therefore if we replace $x$
  with $\frac{1}{x}$ then roots will be in A.P.

  $\Rightarrow \frac{10}{x^3} - \frac{c}{x}^2 - \frac{54}{x} - 27 = 0 \Rightarrow 27x^3 + 54x^2 + cx - 10 =
  0$.

  Let the roots are $a - d, a, a + d$, then sum of roots $3a = -\frac{54}{27}\Rightarrow a = -\frac{2}{3}$,
  which is a root of the equation. Substituting this in new equation we find $c = 9$.
\item Given that $a, b, c$ are the roots of the equation $x^3 + px^2 + qx + r = 0$ such that $c^2 = -ab$.

  $\Rightarrow a + b + c = -p, ab + bc + ca = q$ and $abc = -r \Rightarrow c^3 = -abc = r$.

  $pq = -(a + b + c)(ab + bc + ca) = -[a^2b + abc + ca^2 + ab^2 + b^2c + abc + abc + bc^2 + c^2a] = -(a^2b +
  abc + + ca^2 + ab^2 + b^2c + abc + abc - ab^2 -a^2b)$

  $= -(3abc + a^2c + b^2)\therefore pq - 4r = -r - a^2c - b^2c \Rightarrow (pq - 4r)^3 = -c^3(a^2 + b^2 +
  c^2)$.

  L.H.S. $= (p^2 - 2q)^3.r = -[(a + b + c)^2 - 2(ab + bc + ca)].c^3 = -c^3(a^2 + b^2 + c^2) =$ R.H.S.
\item If $\alpha + i\beta$ is one root of $x^3 + qx + r = 0$ then $\alpha - i\beta$ will be another
  root. Let $\gamma$ be the third root.

  Sum of roots $2\alpha + \gamma = 0 \Rightarrow \gamma = -2\alpha$. Since $\gamma$ is a root of given
  equation, therefore

  $(-2\alpha)^3 - 2q\alpha + r = 0$, and hence we have our equation is $x^3 + qx - r = 0$.
\item Clearly, $\alpha + \beta + \gamma = -\frac{1}{2}, \alpha\beta + \beta\gamma + \gamma\alpha = 0,
  \alpha\beta\gamma = 2$.

  We have to find $\displaystyle\sum\left(\frac{\alpha}{\beta} + \frac{\beta}{\alpha}\right) =
  \frac{\alpha}{\beta} + \frac{\beta}{\alpha} + \frac{\beta}{\gamma} + \frac{\gamma}{\beta} +
  \frac{\alpha}{\gamma} + \frac{\gamma}{\alpha}$

  $=\frac{1}{\alpha}(\beta + \gamma) + \frac{1}{\beta}(\gamma + \alpha) + \frac{1}{\gamma}(\alpha + \beta) =
  \frac{1}{\alpha}\left(-\frac{1}{2} - \alpha\right) + \frac{1}{\beta}\left(-\frac{1}{2} - \beta\right) +
  \frac{1}{\gamma}\left(-\frac{1}{2} - \gamma\right)$

  $= -\frac{1}{2}\left(\frac{1}{\alpha} + \frac{1}{\beta} + \frac{1}{\gamma}\right) - 3=
  -\frac{1}{2}\left(\frac{\alpha\beta + \beta\gamma + \gamma\alpha}{\alpha\beta\gamma}\right) - 3 = -3$.
\item Given equations are $x^3 + px^2 + qx + r = 0$ and $x^3 + p'x^2 + q'x + r' = 0$. Let $\alpha, \beta$
  are common roots. Then putting $\alpha$ and $\beta$ in the equations and subtracting

  $(p - p')\alpha^2 + (q - q')\alpha + (r - r') = 0$ and $(p - p')\beta^2 + (q - q')\beta + (r - r') = 0$.

  Thus, the quadratic equation whose roots are $\alpha, \beta$ is $(p - p')x^2 + (q - q')x + (r - r') = 0$.
\item Let $\alpha, \beta, \gamma$ are the roots the given equation and are in G.P. Then, $\beta^2 =
  \alpha\gamma$ and also $\alpha\beta\gamma = -\frac{d}{a} \Rightarrow \beta =
  -\left(\frac{d}{a}\right)^{1/3}$.

  Substituting the value of $\beta$ thus obtained in the given equation

  $a\left(-\frac{d}{a}\right) + 3b\left(-\frac{d}{a}\right)^{2/3} + 3c\left(-\frac{d}{a}\right)^{1/3} + d =
  0\Rightarrow ac^3 = b^3d$, which the needed condition.
\item Let $\alpha, \beta, \gamma$ are the roots of the equation $x^3 - px^2 + qx - r = 0$, then

  $\alpha + \beta + \gamma = p, \alpha\beta + \beta\gamma + \gamma\alpha = q, \alpha\beta\gamma = r$.

  Mean of H.P. $= \beta = \frac{3\alpha\beta\gamma}{\alpha\beta + \beta\gamma + \gamma\alpha} =
  \frac{3r}{q}$. Substituting this in given equation

  $\left(\frac{3r}{q}\right)^3 - p\left(\frac{3r}{q}\right)^2 + q\frac{3r}{q} - r = 0 \Rightarrow 27r^3 -
  9pqr^2 + 2rq^3 = 0 \Rightarrow 27r^2 + 2q^3 = 9pqr$.
\item Let $\alpha, \beta, \gamma$ be the roots of the given equation. Also given that $f(0)$ and $f(-1)$ are
  odd.

  $f(0) = \mtext{odd} \Rightarrow d = \mtext{odd}, f(-1) = -1 + b - c + d = \mtext{odd} \Rightarrow b - c =
  \mtext{odd}$.

  Also, $\alpha\beta\gamma = -d = \mtext{odd}$ which implies $\alpha, \beta, \gamma$ are all odd. However,

  $b - c = -[(\alpha + \beta + \gamma) - (\alpha\beta + \beta\gamma + \gamma\alpha)] = -[\mtext{odd} -
    \mtext{odd}] = \mtext{even}$

  which contradicts the assumption that all roots are integers.
\item Let $\alpha, \beta, \gamma$ are roots of the equation $2x^3 + ax^2 + bx + 4 = 0$, then

  $\alpha + \beta + \gamma = -\frac{a}{2}, \alpha\beta + \beta\gamma + \gamma\alpha = \frac{b}{a}$ and
  $\alpha\beta\gamma = -2$.

  Since all coefficients are positive hence all roots are negative. Let $\alpha = -p, \beta = -q$ and
  $\gamma = -r$, then

  $p + q + r = \frac{a}{2}, pq + qr + rp = \frac{b}{2}$ and $pqr = 2$.

  Now A.M$\geq$ G.M. $\Rightarrow \frac{p + q + r}{3}\geq (pqr)^{\frac{1}{3}}\Rightarrow \frac{a}{6}\geq
  2^{1/3}$

  also, because A.M.$\geq$ G.M $\Rightarrow \frac{pq + qr + rp}{3}\geq (pqr)^{2/3} \Rightarrow b\geq
  6.4^{1/3}$

  Adding we arrive at the required inequality.
\item Given equations are $a_1x^3 + b_1x^2 + c_1x + d_1 = 0$ and $a_2x^3 + b_2x^2 + c_2x + d_2 = 0$. Let
  $\alpha$ be a common repeated root then

  $a_1\alpha^3 + b_1\alpha^2 + c_1\alpha + d_1 = 0$ and $a_2\alpha^3 + b_2\alpha^2 + c_2\alpha + d_2 = 0$

  Multiplying first equation by $a_2$ and second equation by $a_1$ and subtracting, we get

  $(a_2b_2 - a_1b_2)x^2 + (a_2c_1 - a_1c_2)x + (a_2d_1 - a_1d_2) = 0$

  Also, the derivatives will be equal to zero because they have a common root i.e.

  $3a_1x^2 + 2b_1x + c_1 = 0$ and $3a_2x^2 + 2b_2x + c_2 = 0$ and hence the condition is

  $\startdeterminant\NC 3a_1\NC 2b_1\NC c_1\NR\NC 3a_1\NC 2b_1\NC c_1\NR\NC a_2b_1 - a_1b_2\NC a_2c_1 -
  a_1c_2\NC a_2d_1 - a_1d_2\NR\stopdeterminant = 0$
\item Given equations are $a_1x^2 + b_1x + c_1 = 0$ and $a_2x^3 + b_2x^2 + c_2x + d_2 = 0$. Because cubic
  equation has a repeated root therefore its derivative will be equal to $0$, and hence

  $3a_2x^2 + 2b_2x + c_2 = 0$. Multiplying first equation by $a_2x$ and second by $a_1$ and subtracting, we
  get

  $(a_1b_2 - a_2b_1)x^2 + (a_1c_2 - a_2c_1)x + a_1d_2 = 0$ and thus from these three equations we have

  $\startdeterminant\NC a_1 \NC b_1\NC c_1\NR\NC 3a_2 \NC 2b_1\NC c_2\NR\NC a_1b_2 - a_2b_1\NC a_1c_2 -
  a_2c_1\NC a_1d_2\NR\stopdeterminant = 0$
\item Given that $\alpha, \beta, \gamma$ are roots of $x^3 - ax^2 + bx - c = 0$ then we have

  $\alpha + \beta + \gamma = a, \alpha\beta + \beta\gamma + \gamma\alpha = b$ and $\alpha\beta\gamma = c$.

  We know that if $a, b, c$ are sides of a triangle and perimeter is $2s$ then area is given by $\sqrt{s(s -
    a)(s - b)(s - c)}$, therefore area of required triangle is

  $\Delta = \frac{1}{4}\sqrt{(\alpha + \beta + \gamma)(\alpha + \beta - \gamma)(\alpha - \beta + \gamma)(\beta +
    \gamma - \alpha )}$

  $= \frac{1}{4}\sqrt{a(\alpha\beta^2 + \beta\gamma^2 + \gamma\alpha^2 + \alpha^2\beta + \beta^2\gamma +
    \gamma^2\alpha - \alpha^3 - \beta^3 - \gamma^3 - 2\alpha\beta\gamma)}$

  $= \frac{1}{4}\sqrt{a[4(\alpha\beta^2 + \beta\gamma^2 + \gamma\alpha^2 + \alpha^2\beta + \beta^2\gamma +
      \gamma^2\alpha + 3\alpha\beta\gamma) - }$(square root continued)

    $\sqrt{(\alpha^3 + \beta^3 + \gamma^3 + 3\alpha^2\beta + 3\alpha\beta^2
    + 3\beta\gamma^2 + 3\beta^2\gamma + 3\alpha\gamma^2 + 3\alpha^2\gamma + 6\alpha\beta\gamma) -
    8\alpha\beta\gamma]}$

  $= \frac{1}{4}\sqrt{a[4(\alpha + \beta + \gamma)(\alpha\beta + \beta\gamma + \gamma\alpha) - (\alpha +
      \beta + \gamma)^3 - 8\alpha\beta\gamma]}$

  $= \frac{1}{4}\sqrt{a(4ab - a^3 - 8c)}$, hence proved.
\item Given $a < b < c < d$ and $\mu(x - a)(x - c) + \lambda(x - b)(x - d) = 0$. Let $f(x) = \mu(x - a)(x -
  c) + \lambda(x - b)(x - d) = 0$

  $f(a) = \lambda(a - b)(a - d), f(c) = \lambda(c - b)(c - d)\Rightarrow f(a)f(c) < 0$ and similarly
  $f(b)f(d) < 0$. Thus the equation has one root between $a$ and $c$ and second root between $b$ and $d$
  which implies that both the roots are real for real $\mu$ and $\lambda$.
\item Let $f(x) = 3x^5 - 5x^3 + 21x + 3\sin x + 4\cos x + 5 = 0$ then $f(\infty) = -\infty$ and $f(\infty) =
  \infty$.

  $f'(x) = 15x^4 - 15x^2 + 21 + 3\cos x - 4\sin x = 15(x^4 - 2x^2 + 1 + x^2) + 6 + 3\cos x - 4\sin x >
  0\;\forall\;x\in(-\infty, \infty)$ which means $f(x)$ is increasing.

  Thus, we see that $f(x)$ can have only one real root.
\item The plot is given below(not in linear scale):
  \startplacefigure[location={left,none}]
    \startMPcode
      drawarrow (-.15cm, 0) -- (2.7cm, 0);
      drawarrow (0, -2cm) -- (0, 2cm);
      draw function(1,"x","x*x*x - 10*x*x - 11*x - 100",-3, 13,.01) xyscaled(.2cm, .005cm);
    \stopMPcode
  \stopplacefigure
  $f'(x) = 3x^2 - 20x - 11 = 0 \Rightarrow x = \frac{10\pm\sqrt{133}}{3}$ which shows two points in the
  graph where tangent is parallel to $x$-axis. We see that after the higher value of this root the graph is
  increasing and cuts $x$-axis. So we substitute the increasing values of $x$ to obtain the integral part of
  root. $x = \frac{10 + \sqrt{133}}{3}\approx 7.16$. We find that $f(8) < f(9) < f(10) < f(11) < 0$ but
  $f(12) > 0$ . So the root lies between $11$ and $12$, and hence the integral part is $[x] = 11$.
  \vskip 0.9cm
\item $f(x)=(x-m)(b_n x^n+\cdots + b_0)=(x-m)g(x)$ for some $b_0, \ldots, b_m\in\mathbb{Z}$. Then

  $f(0) = -m.g(0)$ and $f(1) = (1 - m).g(1)$ but either $-m$ or $1 - m$ is even. Observe that $f(0) = a_n$
  and $f(1) = \displaystyle\sum_{i=0}^na_i$.
\item Let $g(x) = e^xf(x)$ then $g''(x) = e^x[f(x) + 2f'(x) + f''(x)]. \therefore$ Roots of equation $f(x) +
  2f'(x) + f''(x) = 0$ will be same as those of equation $g''(x) = 0$ as $e^x\neq 0$.

  Also, since $e^x > 0$, therefore roots of the equation $f(x) = 0$ and $g(x) = 0$ will be same.

  Clearly, $g(x) = 0$ will have $\alpha, beta, \gamma$ as roots and hence $g'(x) = 0$ will have roots $a$
  between $\alpha$ and $beta$ and a root $b$ between $\beta$ and $\gamma$. Hence equation $g''(x) = 0$ will
  have a root between $a$ and $b$, which obviously lies between $\alpha$ and $\gamma$.
\item The plot is given below(not in linear scale):
  \startplacefigure[location={middle,none}]
    \startMPcode
      drawarrow (-2.5cm, 0) -- (6cm, 0);
      drawarrow (0, -2cm) -- (0, 2cm);
      draw function(1,"x","x*x*x*x - 4*x*x*x - 8*x*x",-2, 5.7,.01) xyscaled(1cm, .03cm);
      draw (-1cm, 0cm) -- (-1cm, -.09cm);
      draw (4cm, 0cm) -- (4cm, -3.84cm);
      label.bot("$(-1, -3)$", (-1cm, -.09cm));
      label.bot("$(4, 128)$", (4cm, -3.84cm));
    \stopMPcode
  \stopplacefigure
  Let $f(x) = x^4 - 4x^3 - 8x^2 \Rightarrow f'(x) = 4x^3 - 12x^2 - 16x = 4x(x - 4)(x + 1)$ so at $x =
  -1, 0, 4$ there will be tangents and the direction of $f(x)$ will change.

  From the graph it is clear that for $f(x) + a = 0$ to have four real roots $0\leq a\leq 3$.
\item Let $\alpha, \beta$ be two distinct roots of the given equation. Then $\alpha + \beta = -\frac{b}{a},
  \alpha\beta = \frac{c}{a}$. Using A.M $\geq$ G.M. For $0 < \alpha, 1 - \alpha, \beta, 1 - \beta < 1$

  So $\frac{1 - \alpha + \alpha}{2}\geq\sqrt{\alpha(1 - \alpha)} \Rightarrow \alpha(1 - \alpha)\leq
  \frac{1}{4}$

  Similarly $\beta(1 - \beta)\leq \frac{1}{4}\Rightarrow \alpha\beta(1 - \alpha)(1 - \beta)< \frac{1}{16}$

  $\Rightarrow \alpha\beta[1 - (\alpha + \beta) + \alpha\beta] < \frac{1}{16}\Rightarrow 16c(a - b + c) <
  a^2$

  However, $\min[c(a - b + c)] = 1$ so $a^2 > 16$ Thus, $a_{\min} = 5$.

  Now $2 < \alpha + \beta < 4 \Rightarrow 2a < b < 4a \Rightarrow b_{\min} = 11$.
\item Let $f(x) = (x - 1)^5 + (x + 2)^7 + (7x - 5)^9 - 10$ then $f(-\infty) = -\infty$ and $f(\infty) =
  \infty$. $f'(x) = 5(x - 1)^4 + 7(x + 2)6 + 63(7x - 5)^8 > 0$ which makes $f(x)$ and increasing function,
  which means it can cut $x$-axis only once; yielding only one root.
\item Given, $\sqrt{2(x + 3)} - \sqrt{x + 2} = 3$. Squaring $2x + 6 + x + 2 - 2\sqrt{2(x + 3)(x + 2)} = 9$.

  Squaring again, $\Rightarrow 8(x + 2)(x + 3) = (1 - 3x)^2 \Rightarrow x^2 - 46x - 47 = 0 \Rightarrow x =
  47, -1$.

  Substituting these in the original equation, we quickly find that $x = 47$ is the actual root and $x = -1$
  is the extraneous root. Hence, $\tan\theta = 47, \tan\phi = -1$, and hence

  $\tan(\theta + \phi) = \frac{23}{24}$ and $\cot(\theta - \phi) = -\frac{23}{24}$.
\item {\bf Case I:} When $x < -1$ then the equation becomes $-x - 1 + x - 3x + 3 + 2x - 4 = x + 2 \Rightarrow
  2x = -4\Rightarrow x = -2$.

  {\bf Case II:} When $-1 < x < 0$, then $x + 1 + x - 3x + 3 + 2x - 4 = x + 2\Rightarrow x = x + 2$,
  which is not possible.

  {\bf Case III:} When $0 < x < 1$, then $x + 1 - x - 3x + 3 + 2x - 4 = x + 2 \Rightarrow -x = x + 2
  \Rightarrow x = -1$, which is not possible.

  {\bf Case IV:} When $1 < x < 2$, then $x + 1 - x + 3x - 3 + 2x - 4 = x + 2 \Rightarrow 5x - 6 = x +
  2\Rightarrow x = 2$, which is not possible.

  {\bf Case V:} When $x \geq 2$, then $x + 1 - x + 3x - 3 - 2x + 4 = x + 2 \Rightarrow x + 2 = x + 2$, which
  is true.

  Hence, the solution is $x = -2, x\geq 2$.
\item {\bf Case I:} When $x < -1$, then $\frac{1}{2^{x + 1}} - 2^x = -2^x + 1 + 1 \Rightarrow x = -2$.

  {\bf Case II:} When $-1 < x < 0$, then $2^{x + 1} - 2^x = -\frac{1}{2^x} + 1 + 1\Rightarrow 2^{2x + 1} -
  3.2^x + 1 = 0 \Rightarrow 2^x = 0, 2^x = \frac{1}{2}$, which is not possible.

  {\bf Case III:} When $x \geq 0$, $2^{x  + 1} - 2^x = 2^x - 1 + 1 \Rightarrow 0 = 0$.

  Hence, the solution is $x = -2, x \geq 0$.
\item {\bf Case I:} When $x < 0, y < 0$, then $x^2 - 2x + y = 1, x^2 - y = 1 \Rightarrow x = \frac{1 -
  \sqrt{5}}{2}, y = \frac{1 - \sqrt{5}}{2}$

  {\bf Case II:} When $x < 0, y > 0$, then $x^2 - 2x + y = 1, x^2 + y = 1\Rightarrow -2x = 0, y = 1$

  {\bf Case III}: When $0 < x < 2, y < 0$, then $-x^2 + 2x + y = 1, x^2 - y = 1\Rightarrow 2x = 2, y = 0$
  {\bf Case IV}: When $0 < x < 2, y > 0$, then $-x^2 + 2x + y = 1, x^2 + y = 1\Rightarrow -2x^2 + 2x = 0, x
  = 0, 1, y = 1, 0$

  {\bf Case V:} When $x > 2, y < 0$, then $x^2 - 2x + y = 1, x^2 - y = 1\Rightarrow 2x^2 - 2x = 2
  \Rightarrow x = \frac{1 \pm\sqrt{2}}{2} < 2$, which is not possible.

  {\bf Case VI:} When $x > 2, y > 0$, then $x^2 - 2x + y = 1, x^2 + y = 1\Rightarrow x = 0, y = 1$, which is
  not possible.

  Hence, the solution is $x = 0, y = 1, x = y = \frac{1 -\sqrt{5}}{2}, x = 1, y = 0$.
\item Given equation is $|x^2 + 4x + 3| + 2x + 5 = 0\Rightarrow |(x + 1)(x + 3)| + 2x + 5 = 0$.

  {\bf Case I:} When $x < -3$, then $x^2 + 4x + 3 + 2x + 5 = 0 \Rightarrow x^2 + 6x + 8 = 0 \Rightarrow x =
  \frac{-6\pm\sqrt{4}}{2}, \Rightarrow x = -4, -2$. But $x = -2$ is not possible.

  {\bf Case II:} When $-1 < x < -3$, then $-x^2 - 4x - 3 + 2x + 5 = 0 \Rightarrow x^2 + 2x - 2 = 0
  \Rightarrow x = -1\pm\sqrt{3}$. But $x = -1 + \sqrt{3}$ is not possible.

  {\bf Case III:} When $x > -1$, then $x^2 + 4x + 3 + 2x + 5 = 0 \Rightarrow x = -4, -2$, which is not
  possible.

  Hence, the solution is $x = -4, -1 - \sqrt{3}$.
\item Given equation upon simplification is $x^4 + 6x^3 - 9x^2 - 162x - 243 = 0$ and $x\neq -3$.

  Let us assume that $x^4 + 6x^3 - 9x^2 - 162x - 243 = (x^2 + ax + b)(x^2 + cx + d)$. Comparing
  coefficients,

  $a + c = 6, b + d + ad = -9, ad + bc = -162, bd = -243$, which is four equations with four
  unknowns. Solving these, we have $a = -3, b = -9, c = 9, d = 27$, and hence, the solution is

  $x = \frac{3\pm3\sqrt{5}}{2}, \frac{-9\pm3\sqrt{3}i}{2}$.
\item Given equation is $\frac{1}{[x]} + \frac{1}{[2x]} = \{x\} + \frac{1}{3}$. We observe that $[x]$ cannot
  be negative because that will make L.H.S. negative while R.H.S. is positive.

  {\bf Case I:} When $\{x\}\geq\frac{1}{2}$, then $2[x] = 2[x] + 1$. Putting $[x] = n$, where
  $n\in\mathbb{P}$.

  Given equation is $\{x\} = \frac{1}{n} + \frac{1}{2n + 1} - \frac{1}{3}$. Putting $x = 1, 2, 3, \ldots$
  we observe that $\{x\}$ is not satisfied and the function is decreasing in nature.

  {\bf Case II:} When $\{x\} < \frac{1}{2}$, then $\{x\} = \frac{1}{n} + \frac{1}{2n} - \frac{1}{3}$.

  $\Rightarrow \{x\} = \frac{6 + 3 - 2n}{6n}$, now we see that numerator becomes negative once $n \geq 5$,
  thus those values are ruled out. We see that $x = 2, 3, 4$ are the only values which satisfy the given
  conditions.
\item Let $k = \log_ax\log_{10}a\log_a5 = \log_a5^{\log_{10}x}$, then $a^k = 5^{\log_{10}x} = 5^l$(let
  $\log_{10}x = l$).

  Let $m = \log_{10}\left(\frac{x}{10}\right) = \log_{10}x - 1 = l - 1$ and $n = \log_{100}x + \log_42 =
  \frac{1}{2}\log_{10}x + \frac{1}{2}\log_22 = \frac{l + 1}{2}$.

  $\therefore 9^n = 9^{\frac{l + 1}{1}} = 3^{l + 1} = 3.3^l$.

  According to question $\frac{6}{5}.5^l - \frac{3^l}{3} = 3.3^l\Rightarrow 5^{l - 2} = 3^{l - 2}$, which is
  possible only if $l = 2 \Rightarrow x = 100$.
\item $5^{\frac{1}{x}} + 125 = 5^{\log_56 + 1 + \frac{1}{2x}} = 5^{\log_56}.5.5^{\frac{1}{2x}}$

  $\Rightarrow 5^{\frac{1}{x}} + 125 = 6.5.5^{\frac{1}{x}} \Rightarrow k^2 + 125 = 30k$, where $k =
  5^{\frac{1}{2x}}$

  $\Rightarrow k = 5, 25 \Rightarrow x = \frac{1}{2}, \frac{1}{4}$.
\item Taking $\log$ of both sides with base $x$, we have

  $\tfrac{2}{3}\left[(\log_2x)^2 + \log_2x - \tfrac{5}{4}\right] = \frac{1}{2}\log_x2$

  $\Rightarrow \frac{2}{3}\left[(\log_2x)^2 + \log_2x - \frac{5}{4} = \frac{1}{2\log_2x}\right]$ (Putting $\log_2x = y$)

  $\Rightarrow y^2 + y - \frac{5}{4} = \frac{3}{4y} \Rightarrow 4y^3 + 4y^2 - 5y - 3 = 0$.

  Observing that sum of coefficients is zero, we quickly deduce that $y = 1$ is one of the solution. Thus,
  the above equation is reduced to

  $4y^2 + 8y + 3 = 0 \Rightarrow y = -\frac{1}{2}, -\frac{3}{2}$.

  And hence, $x = 2, \frac{1}{\sqrt{2}}, \frac{1}{2\sqrt{2}}$.
\item Given $3x^2 = 8[x] - 1$. Let $[x] = 1$, then $x = \sqrt{\frac{7}{3}}$ and when $[x] = 2 \Rightarrow x
  = \sqrt{5}$. However, when $[x] = 3, x = \sqrt{\frac{23}{3}} < 3$, which is not possible. Further values
  are not possible because if we increase $[x]$ linearly then L.H.S. will increase exponentially.

  Thus, two possible values are $\sqrt{\frac{7}{3}}$ and $\sqrt{5}$.
\item Let $y = t + \sqrt{t^2 - 1}$, then $\frac{1}{y} = t - \sqrt{t^2 - 1}$ and $y + \frac{1}{y} = 2t$

  Thus, the given equation becomes $y^{x^2 - 2x} + \frac{1}{y^{x^2 - 2x}} = y + \frac{1}{y}$

  Let $z = y^{x^2 - 2x}$, then given equation is $z - y + \frac{1}{z} - \frac{1}{y} = 0$

  $\Rightarrow (z - y)\left(1 - \frac{1}{zy}\right) = 0\Rightarrow z = y$ or $z = \frac{1}{y}\Rightarrow x =
  1, 1\pm\sqrt{2}$.
\item Multiplying first equation by $2$ and subtracting, we get

  $5y^2 + 10y -15 = 0 \Rightarrow y^2 + 2y - 3 = 0 \Rightarrow y = -3, 1$. If $y = -3, -3x + 27 - x - 12 - 7
  = 0 \Rightarrow -4x + 8 = 0\Rightarrow x = 2$. If $y = 1, x + 3 - x + 4 - 7 = 0\Rightarrow 0 = 0$ so all
  values of $x\in\mathbb{R}$ will satisfy the equation.
\item We have $2^{x - 1}.27^{\tfrac{x}{x + 2}} = 3$. Taking $\log$ with base $2$, we have

  $x - 1 + \frac{2x - 2}{x + 2}\log_23 = 0 \Rightarrow x - 1 - \frac{2x - 2}{x + 2} + \frac{2x - 2}{x +
  2}(\log_23 + \log_22) = 0$

  $\Rightarrow \frac{x^2 - x}{x + 2} + \frac{2x - 2}{x + 2}\log_26 = 0\Rightarrow \frac{x - 1}{x + 2}(x +
  \log_26) = 0\Rightarrow x = 1, -2\log_26$.
\item We have $4^x - 3^{x - \tfrac{1}{2}} = 3^{x + \tfrac{1}{2}} - 2^{2x - 1} \Rightarrow 2^{2x} + 2^{2x -
  1} = 3^{x + \frac{1}{2}} + 3^{x - \frac{1}{2}}$

  $\Rightarrow 2^{2x - 1}.3 = 3^{x - \frac{1}{2}}.4 \Rightarrow 2^{2x - 3} = 3^{x - \frac{3}{2}}$.

  $x = \frac{3}{2}$ is a solution which satisfies both sides, and is the only solution.
\item We have $\log_{10}[98 + \sqrt{x^3 - x^2 - 12x + 36}] = 2$. Taking antilog,

  $\sqrt{x^3 - x^2 - 12x + 36} = 2 \Rightarrow x^3 - x^2 - 12x + 32 = 0 \Rightarrow (x + 4)(x^2 - 5x + 8) =
  0$.

  We find that the only real solution is $x = -4$.
\item Given, $\log_{2x + 3}(6x^2 + 23x + 21) = 4 - \log_{3x + 7}(4x^2 + 12x + 9) \Rightarrow \log_{2x +
  3}(2x + 3)(3x + 7) = 4 - \log_{3x + 7}(2x + 3)^2$

  $\Rightarrow 1 + \log_{2x + 3}(3x + 7) = 4 - 2\log_{3x + 7}(2x + 3)\Rightarrow \log_{2x + 3}(3x + 7) -
  \log_{3x + 7}(2x + 3) = 3$

  Let $\log_{2x + 3}(3x + 7) = z$ then $\log_{3x + 7}(2x + 3) = \frac{1}{z}$, and given equation becomes

  $z + \frac{2}{z} = 3 \Rightarrow z = 1, 2 \Rightarrow 2x + 3 = 3x + 7 \Rightarrow x = -4$, which is not
  possible as $2x + 3 > 0$ and $3x + 7 = (2x + 3)^2 \Rightarrow 4x^2 + 9x + 2 = 0\Rightarrow x =
  -\frac{1}{4}, -2$, but again $x = -2$ is not possible as it makes $2x + 3 < 0$.

  Hence, the only possible solution is $x = -\frac{1}{4}$
\item Rewriting the given equation $y^4 - 2x^4 = 1402 \Rightarrow (y^2 +\sqrt{2}x^2)(y^2 - \sqrt{2}x^2) =
  701\times2$

  Suppose $x, y$ are integers then $x^2, y^2 > 0$, which implies

  $y^2 + \sqrt{2}x^2 = 701$ and $y^2 - \sqrt{2}x^2 = 2$. Adding, $2y^2 = 703$, which has no integral
  solution.
\item Given equation is $|x - 1|^{\log_3x^2 - 2\log_x9} = (x - 1)^7$. Clearly, $x > 1$ for $\log_x9$ to be
  defined. So the equation becomes

  $(x - 1)^{\log_3x^2 - 2\log_x9} = (x - 1)^7$, taking $\log$ of both sides

  $(2\log_3x - 4\log_x3 - 7)[\log(x - 1)] = 0$. So either

  $2\log_3x - 4\log_x3 - 7 = 0$ or $\log(x - 1) = 0 \Rightarrow x - 1 = 1 \Rightarrow x = 2$.

  Let $\log_3x = z$ then $\log_x3 = \frac{1}{z}$, so we have

  $2z^2 - 7z - 4 = 0 \Rightarrow z = 4, -\frac{1}{2}$ which gives us $x = 81, \frac{1}{\sqrt{3}}$ but $x >
  1$ so $x = 81$ is the second solution.
\item One of the solutions is $\cos x = 1$ which will make exponent $\frac{1}{2}$ equalizing both
  sides. Thus, $x = 2n\pi$ is our first solution.

  The second solution can be obtained by setting exponent to zero i.e. $\sin^2x - \frac{3}{2}\sin x +
  \frac{1}{2} = 0$ giving us $\sin x = 1, \frac{1}{2}$ but if $\sin x = 1$ then $\cos x = 0$, which makes th
  equation invalid.

  Therefore, $\sin x = \frac{1}{2}$ is our second solution. Thus, $x = n\pi + (-1)^n\frac{\pi}{6},
  n\in\mathbb{I}$.
\item We have the equation $(x + a)(x + 1991) + 1 = 0 \Rightarrow (x + a)(x + 1991) = -1$

  Either $x + a = 1$ and $x + 1991 = -1\Rightarrow a = 1993$ or $x + a = -1$ and $x + 1991 = 1\Rightarrow a
  = 1989$.
\item Given equation is $2^{\sin^2x} + 5(2^{\cos^2x}) = 7 \Rightarrow 2^{\sin^2x} + \frac{10}{\sin^2x} = 7$.

  Let $2^{\sin^2x} = y$, then the equation becomes $y + \frac{10}{y} = 7 \Rightarrow y^2 - 7y + 10 = 0
  \Rightarrow y = 2, 5$.

  Now $y = 5$ makes $\sin^2x > 1$, which is not possible. If $y = 2\Rightarrow 2^{\sin^2x} = 2\Rightarrow
  \sin x = \pm1 \Rightarrow x = n\pi + (-1)^n\left(\pm\frac{\pi}{2}\right)$.
\item Given equation is $x + \log_{10}(1 + 2^x) = x\log_{10}5 + \log_{10}6 \Rightarrow x(1 - \log_{10}5) +
  \log_{10}(1 + 2^x) = \log_{10}6$

  $\Rightarrow x(\log_{10} - \log_{10}5) + \log_{10}(1 + 2^x) = \log_{10}6 \Rightarrow x\log_{10}2 +
  \log_{10}(1 + 2^x) = \log_{10}6$

  $\Rightarrow \log_{10}2^x(1 + 2^x) = \log_{10}6\Rightarrow 2^x(1 + 2^x) = 6\Rightarrow 2^x = 2, -3$ but
  for real values of $x, 2^x\neq -3$, thus, $2^x = 2 \Rightarrow x = 1$.
\item Given equation is $\log_a(ax).\log_x(ax) + \log_{a^2}(a) = 0\Rightarrow (1 + \log_ax)(1 + \log_xa) +
  \frac{1}{2} = 0$

  $\Rightarrow 2(\log_ax)^2 + 5\log_ax + 2 = 0 \Rightarrow \log_ax = -2, -\frac{1}{2}\Rightarrow x =
  \frac{1}{a^2}, \frac{1}{\sqrt{a}}$.
\item Given equation is $\sqrt{11x - 6} + \sqrt{x - 1} = \sqrt{4x + 5}$, squaring, we get

  $11x - 6 + 4x + 5 + 2\sqrt{(11x - 6)(x - 1)} = 4x + 5 \Rightarrow \sqrt{(11x - 6)(x - 1)} = -4x + 6$

  Squaring again, $11x^2 - 17x + 6 = 16x^2 - 48x + 36 \Rightarrow 5x^2 - 31x + 30 = 0\Rightarrow x =
  \frac{6}{5}, 5$ but $x = 5$ does not satisfy the given equation, and is result of squaring.
\item Given equation is $\sqrt{3x^2 - 7x - 30} - \sqrt{2x^2 - 7x - 5} = x - 5$. Sqauring,

  $3x^2 - 7x - 30 = (x - 5)^2 + 2x^6 - 7x - 5 + 2(x - 5)\sqrt{2x^2 - 7x - 5}$

  $\Rightarrow (x - 5)(5 - \sqrt{2x^2 - 7x - 5}) = 0$, so $x = 5$ is one of the solutions. The other
  solution will be given by

  $5 = \sqrt{2x^2 - 7x - 5}$, squaring again, $2x^2 - 7x - 30 = 0\Rightarrow x = 6, -\frac{5}{2}$, but $x =
  -\frac{5}{2}$ does not satisfy the equation.

  Hence, $x = 5, 6$ are the solutions.
\item Given euations are $y = 2[x] + 3$ and $y = 3[x - 2] \Rightarrow y = 3[x] - 6$. Solving yields $y = 21,
  [x] = 9$ giving $[x + y] = 30$.
\item $\displaystyle\sum_{i=1}^n(x - a_i)^2 = nx^2 - 2(a_1 + a_2 + \cdots + a_n)x + (a_1^2 + a_2^2 + \cdots
  + a_n^2)$, which is a quadratic equation in $x$ and coefficient of $x^2$ is $n > 0$, therefore, this
  quadratic equation will have least value at $x = \frac{a_1 + a_2 + \cdots + a_n}{n}$.
\item Let the quotient be $\frac{n}{n^2 - 1},\;n\in\mathbb{N}$. According to question,

  $\frac{n + 2}{n^2 - 1 + 2}> \frac{1}{3}\Rightarrow n^2 - 3n - 5 < 0 \Rightarrow \frac{3}{2} -
  \frac{\sqrt{29}}{2} < n < \frac{3}{2} + \frac{\sqrt{29}}{2}$.

  Also, $0<\frac{n - 3}{n^2 - 1 - 3} < \frac{1}{10}\Rightarrow 0 < \frac{n - 3}{n^2 - 4} < \frac{1}{10}$

  Taking the first inequality, $\frac{n - 3}{n^2 - 4} > 0 \Rightarrow -2 < n < 2$ or $3 < n < \infty$.

  Taking the second inequality $\frac{n - 3}{n^2 - 4} < \frac{1}{10} \Rightarrow \frac{n^2 - 10n + 26}{10(4
    - n^2)} < 0 \Rightarrow -n < -2$  or $n > 2$.

  Thus, we have $3 < n < \frac{3}{2} + \frac{\sqrt{29}}{2}\Rightarrow n = 4$ (since $n$ is a natural number)

  Thus, we deduce the quotient to be $\frac{4}{4^2 - 1} = \frac{4}{15}$.
\item Let $f(x) = ax^2 + bx + c$, then $g(x) = f(x) + f'(x) + f''(x) = ax^2 + bx + c + 2ax + b + 2a = ax^2 +
  (b + 2a)x + 2a + b + c$.

  Given $ax^2 + bx + c > 0\;\forall\;x\in\mathbb{R}\therefore b^2 - 4ac < 0$ and $a > 0$.

  Discriminant of $g(x), D = (b + 2a)^2 - 4a(2a + b + c) = (b^2 - 4ac) - 4a^2 < 0$ and $a > 0$.

  Thus, $g(x) > 0\;\forall\;x\in\mathbb{R}$.
\item From given equation it is clear that $f(x)\geq 0\;\forall\;x\in\mathbb{R}$ and

  $f(x) = (a_1^2 + a_2^2 + \cdots + a_n^2)x^2 + 2(a_1b_1 + a_2b_2 + \cdots + a_nb_n)x + (b_1^2 + b_2^2 +
  \cdots + b_n^2)\geq 0\;\forall\;x\in\mathbb{R}$

  $\therefore$ Discriminant of its corresponsing equation $D \leq 0$, because coefficient of $x^2$ is
  positive.

  $\Rightarrow 4(a_1b_1 + a_2b_2 + \cdots + a_nb_n)^2 - 4(a_1^2 + a_2^2 + \cdots + a_n^2)(b_1^2 + b_2^2 +
  \cdots + b_n^2) \leq 0$

  $\Rightarrow (a_1b_1 + a_2b_2 + \cdots + a_nb_n)^2\leq (a_1^2 + a_2^2 + \cdots + a_n^2)(b_1^2 + b_2^2 +
  \cdots + b_n^2)$.
\item Given equation is $x(x + 1)(x + m)(x + m + 1) = m^2 \Rightarrow [x^2 + (m + 1)x + m][x^2 + (m + 1)x] =
  m^2$

  $\Rightarrow y^2 + my - m^2 = 0$, where $y = x^2 + (m + 1)x$. $\therefore y = \frac{-m\pm\sqrt{5}}{2}$

  $\Rightarrow 2x^2 + 2(m + 1)x - (\sqrt{5} - 1)m = 0$ and $2x^2 + 2(m + 1)x + (\sqrt{5} + 1)m = 0$. Thus,
  given equation will have four real roots if these two equations have two real roots each.

  $\therefore 4(m + 1)^2 + 8(\sqrt{5} - 1)m > 0$ and $4(m + 1)^2 - 8(\sqrt{5} + 1)m > 0$

  $\Rightarrow m^2 + 2\sqrt{5}m + 1 > 0$ and $m^2 - 2\sqrt{5}m + 1 > 0$. Thus, $|m|> 2 + \sqrt{5}$ or $|m| <
  \sqrt{5} - 2$.
\item Given equation is $x^4 + (a - 1)x^3 + x^2 + (a - 1)x + 1 = 0 \Rightarrow \left(x +
  \frac{1}{x}\right)^2 - 2.x.\frac{1}{x} + (a - 1)\left(x + \frac{1}{x}\right) + 1 = 0$

  $\Rightarrow y^2 + (a - 1)y - 1 = 0$, where $y = x + \frac{1}{x}$

  $\therefore y = \frac{-(a - 1)\pm\sqrt{(a - )^2 + 4}}{2} = -\frac{(a - 1)\mp\sqrt{(a - 1)^2 - 4}}{2}$

  $\Rightarrow 2x^2 + [(a - 1) - \sqrt{(a - 1)^2 + 4}]x + 2 = 0$ and $2x^2 + [(a - 1) + \sqrt{(a - 1)^2 +
      4}]x + 2 = 0$

  Let $\alpha, \beta$ be roots of first and $\gamma, \delta$ be the roots of second, then

  $\alpha + \beta = -\frac{(a - 1) - \sqrt{(a - 1)^2 + 4}}{2}$ and $\alpha\beta = 1, \gamma + \delta =
  -\frac{(a - 1) + \sqrt{(a - 1)^2 + 4}}{2}$ and $\gamma\delta = 1$

  $\because \sqrt{(a - 1)^2 + 4} > a - 1$, therefore, $\alpha + \beta > 0$ and $\alpha\beta > 0$, which
  means $\alpha, \beta$ are positive. Thus, the equation $2x^2 + [(a - 1) + \sqrt{(a - 1)^2 + 4}]x + 2 = 0$
  must have two negative roots.

  For both roots to be negative $D > 0 \Rightarrow [(a - 1) + \sqrt{(a - 1)^2 + 4}]^2 - 16 > 0$

  $\Rightarrow a - 1 + \sqrt{(a - 1)^2 + 4} - 4 > 0 [\because a - 1 + \sqrt{(a - 1)^2 + 4}] + 4 > 0$

  $\Rightarrow \sqrt{(a - 1)^2 + 4} > 5 - a \Rightarrow a\geq 5$ or $(a - 1)^2 + 4 > (5 - a)^2$ where $a >
  5$.

  $\Rightarrow \frac{5}{2} < a < \infty$.
\item Given equation is $x^4 + 2ax^3 + x^2 + 2ax + 1 = 0 \Rightarrow x^2 + \frac{1}{x^2} + 2a\left(x +
  \frac{1}{x}\right) + 1 = 0$

  $\Rightarrow \left(x + \frac{1}{x}\right)^2 - 2.x.\frac{1}{x} + 2a\left(x + \frac{1}{x}\right) + 1 = 0
  \Rightarrow y^2 + 2ay - 1 = 0$, where $y = x + \frac{1}{x}$

  $\Rightarrow y = -a\pm\sqrt{a^2 + 1}$. When $y = -a + \sqrt{a^2 + 1} = x + \frac{1}{x}\Rightarrow x^2 + (a
  - \sqrt{a^2 + 1})x + 1 = 0$, and, when $y = -a - \sqrt{a^2 + 1} = x + \frac{1}{x}\Rightarrow x^2 + (a +
  \sqrt{a^2 + 1})x + 1 = 0$.

  Let $\alpha,\beta$ be roots of first equation and $\gamma,\delta$ be roots of second equation. Then,

  $\alpha + \beta = \sqrt{a^2 + 1} - a, \alpha\beta = 1$ and $\gamma + \delta = -(a + \sqrt{a^2 + 1}),
  \gamma\delta = 1$.

  Clearly $\alpha, \beta$ are both imaginary or positive so from question $\gamma, \delta$ both must be
  negative. $\Rightarrow D\geq 0$, which leads to

  $(a + \sqrt{a^2 + 1})^2 - 4 > 0 \Rightarrow \sqrt{a^2 + 1} > 2 - a \Rightarrow \frac{3}{4} < a < \infty$.
\item Given system of equations can be written as $ax_1^2 + (b - 1)x_1 + c = x_2 - x_1, ax_2^2 + (b - 1)x_1
  + c = x_3 - x_2, \ldots, ax_{n - 1}^2 + (b - 1)x_{n - 1} + c = x_n - x_{n - 1}, ax_n^2 + (b - 1)x_n + c =
  x_1 - x_n$

  $\therefore f(x_1) + f(x_2) + \cdots + f(x_n) = 0\;\;\;\;(1)$

  {\bf Case I:} When $(b - 1)^2 - 4ac < 0$.

  In this case $f(x_1), f(x_2), \ldots, f(x_n)$ will have same sign as that of $a\therefore f(x_1) + f(x_2)
  + \cdots + f(x_n)\neq 0$.

  Hence, the given system of equations has no solution.

  {\bf Case II:} When $(b - 1)^2 - 4ac = 0$.

  In this case $f(x_1), f(x_2), \ldots, f(x_n) \geq 0$ or $f(x_1), f(x_2), \ldots, f(x_n)\leq 0$, From (1),
  $f(x_1) + f(x_2) + \cdots + f(x_n) = 0 \Rightarrow f(x_1) = f(x_2) = \cdots = f(x_n) = 0$

  But $f(x_i) = 0 \Rightarrow ax_i^2 + (b - 1)x_i + c = 0 \Rightarrow x_i = \frac{1 - b}{2a}$

  $\therefore x_1 = x_2 = \cdots = x_n = \frac{1 - b}{2a}$.

  {\bf Case III:} When $(b - 1)^2 - 4ac > 0$.

  Roots of equation $ax^2 + (b - 1)x + c = 0$ are $\alpha,\beta = \frac{1 - b\pm\sqrt{(1 - b)^2 -
      4ac}}{2a}$.

  If $x_1, x_2, \ldots, x_n$ lie between $\alpha$ and $\beta$, then $f(x_1) + f(x_2) + \cdots + f(x_n)\neq
  0$ (because it is $< 0$ or $> 0$ as $a > 0$ or $a < 0$)

  If $x_1, x_2, \ldots, x_n$ lie in $(-\infty, \alpha)$ or $(\beta, \infty)$ then also $f(x_1) + f(x_2) +
  \cdots + f(x_n)\neq 0$.

  If all roots are either $\alpha$ or $\beta$ then $f(x_1) + f(x_2) + \cdots + f(x_n) = 0$.
\item {\bf Case I:} When $x > 1$. We will have $x^2 - \frac{3}{16} > 0 \Rightarrow x < -\frac{\sqrt{3}}{4}$
  or $x > \frac{\sqrt{3}}{4}$, and $x^2 - \frac{3}{16} > x^4 \Rightarrow \frac{1}{4} < x^2 < \frac{3}{4}$.

  Thus, we see that no value of $x$ satisfies all these inequalities at the same time.

  {\bf Case II:} When $x < 1$. We will have $x^2 - \frac{3}{16} > 0$, which will impose same set of
  inequalities, and $x^2 - \frac{3}{16} < x^4 \Rightarrow x^2 < \frac{1}{4}$ or $x^2 > \frac{3}{4}$.

  Thus, $\left(\frac{\sqrt{3}}{4}, \frac{1}{2}\right)\cup\left(\frac{\sqrt{3}}{2}, 1\right)$ represents the
  set of solution.
\item Given $\log_{\tfrac{1}{2}}x^2 \geq \log_{\tfrac{1}{2}}(x + 2)\Rightarrow x^2\leq x + 2 \Rightarrow x^2
  - x - 2\leq 0 \Rightarrow -1 \leq x\leq 2, x\neq 0$. For lagrithm to be defined $x\neq 0$ and $x > -2$.

  Also, $49x^2 - 4m^4 \leq 0 \Rightarrow -\frac{2}{7}m^2\leq \frac{2}{7}m^2$.

  According to question, $[-1, 2]\subseteq \left[-\frac{2}{7}m^2, \frac{2}{7}m^2\right]$

  $\therefore -\frac{2}{7}m^2\leq -1 \Rightarrow m^2\geq \frac{7}{2}$ and $\frac{2}{7}m^2\geq 2 \Rightarrow
  m^2\geq 7$.

  Thus, $-\infty < m \leq -\sqrt{7}$ or $\sqrt{7}\leq m < \infty$.
\item We have to find $a$ for which $1 + \log_5(x^2 + 1) \geq \log_5(ax^2 + 4x + a)$ is valid
  $\forall\;x\in\mathbb{R}$.

  $\Rightarrow \log_55 + \log_5(x^2 + 1) \geq \log_5(ax^2 + 4x + a) \Rightarrow 5(x^2 + 1)\geq ax^2 + 4x +
  a$

  $\Rightarrow (5 - a)x^2 - 4x + 5 - a \geq 0$

  $D\leq 0 \Rightarrow 16 - 4(5 - a^2)\leq 0\Rightarrow a\leq 3$ or $a\geq 7$ and $5 - a > 0\Rightarrow a <
  5$. Combining $-\infty < a\leq 3$.

  For $\log_5(ax^2 + 4x + a)$ to be defined $ax^2 + 4x + a > 0$ for all real $x$. So $D < 0 \Rightarrow 16 -
  4a^2 < 0 \Rightarrow a < -2$ or $a > 2$ and $a > 0$. Combining $2 < a < \infty$.

  Thus, common values are given by $2 < a \leq 3$.
\item $2x^2 + 2x + \frac{7}{2} > 0\;\forall\;x\in\mathbb{R}$ because discriminant of corresponding equation
  is less than $0$ and coefficient of $x^2$ is greater than $0$.

  Thus, $\log_x\left(2x^2 + 2x + \frac{7}{2}\right)$ is defined $\forall\;x\in\mathbb{R}$.

  For $\log_xa(x^2 + 1)$ to be defined $0 < a < \infty$.

  Given equation is $1 + \log_2\left(2x^2 + 2x + \frac{7}{2}\right)\geq \log_2(ax^2 + a) \Rightarrow \log_22
  + \log_2\left(2x^2 + 2x + \frac{7}{2}\right)\geq \log_2(ax^2 + a)$

  $\Rightarrow \log_22\left(2x^2 + 2x + \frac{7}{2}\right)\geq \log_2(ax^2 + a)\Rightarrow 4x^2 + 4x + 7\geq
  ax^2 + a$

  $\Rightarrow (4 - a)x^2 + 4x + 7 - a\geq 0$. Let $D$ be discriminant of corresponding equation, then

  $D = 16 - 4(4 - a)(7 - a) = 4(4 - a^2 + 11a - 28) = -4(a - 3)(a - 8)$.

  When $D > 0, a\neq 4, 3 < a < 8$

  When $D= 0\Rightarrow a = 3, 8$. When $a = 3$, the equation becomes $x^2+ 4x + 4\geq
  0\;\forall\;x\in\mathbb{R}$.

  When $a = 8$, the equation becomes $-(2x - 1)^2 = 0$, when $x = \frac{1}{2}$.

  When $a = 4$, the equation becomes $4x + 3\geq 0$ for infinitely many real values of $x$.

  The equation will be satisfied for $a < 4$ and $D < 0 \Rightarrow (a - 3)(a - 8) > 0\Rightarrow a < 3$ or
  $a > 8 \therefore -\infty < a < 3$.

  Combining all these we get possible values of $a$ by $-\infty < a \leq 8$.
\item Let $a - c = \alpha, b - c = \beta, c + x = u$, then for $\sqrt{a - c}$ and $\sqrt{b - c}$ to be real
  $\alpha,\beta\geq 0$. Also, as $x > -c\Rightarrow u > 0$.

  Let xm
  $y = \frac{(a + x)(b + x)}{c + x} = \frac{(u + \alpha)(u + \beta)}{u} = \frac{u^2 + (\alpha + \beta)u +
    \alpha\beta}{u} = u + \alpha + \beta + \frac{\alpha\beta}{u}$

  $\Rightarrow u^2 + (\alpha + \beta - y) + \alpha\beta = 0$, and because $u$ is real.

  $\therefore (\alpha + \beta - y)^2 - 4\alpha\beta \geq 0 \Rightarrow y^2 - 2(\alpha + \beta)y + (\alpha -
  \beta)^2\geq 0$

  Corresponding roots are $y = \frac{2(\alpha + \beta)\pm\sqrt{4(\alpha + \beta)^2 - 4(\alpha -
      \beta)^2}}{2} = \alpha + \beta \pm2\sqrt{\alpha\beta}$

  $= (\sqrt{\alpha} + \sqrt{\beta})^2, (\sqrt{\alpha} - \sqrt{\beta})^2$

  But if $y \leq (\sqrt{\alpha} - \sqrt{\beta})^2 \Rightarrow y - (\alpha + \beta) + 2\sqrt{\alpha\beta}
  \leq 0$ is not posssible, because $y - \alpha - \beta = u + \frac{\alpha\beta}{u} > 0$.

  Thus, least values of $y$ is $(\sqrt{\alpha} + \sqrt{\beta})^2 = (\sqrt{a - c} + \sqrt{b - c})^2$.
\item Let $y = 4(a - x)[x - a + \sqrt{a^2 + b^2}] = 4z(-z + k)$, where $z = a - x$ and $k = \sqrt{a^2 +
  b^2}\Rightarrow 4x^2 - 4kz + y = 0$

  Because $z$ is real, therefore, $D\geq 0 \Rightarrow 18k^2 - 16y\geq 0 \Rightarrow y\leq (a^2 + b^2)$

  $\Rightarrow y\ngtr a^2 + b^2$.
\item Let $y = \frac{x^2 + 2x\cos2\alpha + 1}{x^2 + 2x\cos2\beta + 1}\Rightarrow (y - 1)x^2 + 2(y\cos2\beta
  - \cos2\alpha) + y - 1 = 0$

  Because $x$ is real, therefore, $D\geq 0 \Rightarrow 4(y\cos2\beta - \cos2\alpha)^2 - 4(y - 1)^2\geq 0$

  $(1 - \cos^22\beta)y^2 + 2(\cos2\alpha\cos2\beta - 1)y + 1 - \cos^22\alpha \leq 0\Rightarrow \sin^22\beta
  y^2 + 2(\cos2\alpha\cos2\beta - 1)y + \sin^22\alpha \leq 0$

  Roots of corresponding equation are $\frac{2(1 - \cos2\alpha\cos2\beta)\pm4\sin(\alpha - \beta)\sin(\alpha
    + \beta)}{2\sin^22\beta}$

  $= \frac{\sin^2\alpha}{\sin^2\beta}, \frac{\cos^2\alpha}{\cos^2\beta}$, which are real and unequal and
  dicriminant is also greater than zero. Coefficient of $y^2$ is also greater than zero.

  Thus, $y$ does not lie between the roots.
\item Let $y = \frac{2a(x - 1)\sin^2\alpha}{x^2 -\sin^2\alpha} \Rightarrow yx^2 - 2a\sin^2\alpha x + (2a -
  y)\sin^2\alpha = 0$.

  Because $x$ is real, therefore, $D\geq 0 \Rightarrow 4a^2\sin^4\alpha - 4y(2a - y)\sin^2\alpha\geq 0
  \Rightarrow a^2\sin^2\alpha - y(2a - y)\geq 0 \Rightarrow y^2 - 2ay + a^2\sin^2\alpha\geq 0$.

  Roots of the corresponding equation are $y = 2a\sin^2\frac{\alpha}{2}, 2a\cos^2\frac{\alpha}{2}$.

  Hence, $y$ does not lie between these roots.
\item Let $y = \tan(x + \alpha)/\tan(x - \alpha) = \left(\frac{p + q}{1 - pq}\right)\left(\frac{1 + pq}{p -
  q}\right)$, where $p = \tan x$ and $q = \tan\alpha$.

  $\Rightarrow y = \frac{qp^2 + (1 + q^2)p + q}{-qp^2 + (1 + q^2)p - q}$

  $\Rightarrow q(y + 1)p^2 + (1 + q^2)(1 - y)p + q(1 + y) = 0$, but $p$ is real, and hence $D\geq 0$.

  $\Rightarrow (1 + q^2)^2(1 - y)^2 - 4q^2(1 + y)^2\geq 0 \Rightarrow (1 - q^2)^2y^2 - 2[(1 + q^2)^2 +
  4q^2]y + (1 - q^2)^2\geq 0$

  Disrciminant of corresponding equation is $64(1 + q^2)^2q^2$ and roots are $\left(\frac{1 - q}{1 +
    q}\right)^2, \left(\frac{1 + q}{1 - q}\right)^2$

  So roots are $\left(\frac{1 - \tan\alpha}{1 + \tan\alpha}\right)^2, \left(\frac{1 + \tan\alpha}{1 -
    \tan\alpha}\right)^2 = \tan^2\left(\frac{\pi}{4} - \alpha\right), \tan^2\left(\frac{\pi}{4} +
  \alpha\right)$.

  Since roots are real and unequal and the coefficient of $y^2$ is greater than zero, and hence, $y$ cannot
  lie between the given values.
\item Let $y = \frac{ax^2 + 3x - 4}{3x - 4x^2 + a} \Rightarrow (4y + a)x^2 + 3(1 - y)x - (ay + 4) = 0$.

  Since $x$ is real, therefore, $D\geq 0 \Rightarrow 9(1 - y)^2 + 4(4y + a)(ay + 4)\geq 0 \Rightarrow (9 +
  16a)y^2 + 2(2a^2 + 23)y + 9 + 16a\geq 0$

  Discriminant of corresponding equation $D' = 4(2a^2 + 23)^2 - 4(9 + 16a)^2 = 16(a + 4)^2(a - 1)(a - 7)$

  If $1 < a < 7 \Rightarrow D' < 0$ and $9 + 16a > 0$, then $(9 + 16a)y^2 + 2(2a^2 + 23)y + 9 + 16a >
  0\;\forall\;y\in\mathbb{R}$.

  Hence, given expression can assume any value if $1 < a < 7$.
\item Let $y = \frac{(ax - b)(dx - c)}{(bx - a)(cx - d)} = \frac{adx^2 - (bd + ac)x + bc}{bcx^2 - (ac + bd)x
  + ad}$

  $\Rightarrow (bcy - ad)x^2 + (1 - y)(bd + ac) + ady - bc = 0$.

  Because $x$ is real, therefore, $D\geq 0$

  $\Rightarrow (bd + ac)^2(1 - y)^2 - 4(bcy - ad)(ady - bc)\geq 0\Rightarrow (bd - ac)^2y^2 - 2[(bd + ac)^2
    - 2(a^2d^2 + b^2c^2)]y + (bd - ac)^2\geq 0$

  Discriminant of corresponding equation $D' = -16(ad - bc)(a^2 - b^2)(c^2 - d^2)$

  Because $a^2 - b^2$ and $c^2 - d^2$ are having same sign, therefore, $D' \leq 0$.

  Hence, $y$ can have any real value.
\item $D\geq 0 \Rightarrow m^2 - 4\geq 0 \Rightarrow m\leq - 2$ or $m\geq 2$

  Since both the roots are less than unity, therefore $f(1) > 0 \Rightarrow m < 2$

  Since both the roots are less than unity, therefore $1 > \frac{\alpha + \beta}{2}\Rightarrow m < 2$

  Thus, $m\in(-\infty, -2]$.
\item This problem can be solved like previous problem. No value of $m$ satisfies the given condition.
\item Let $\alpha, \beta$ be the roots of the given equation. Since roots are real $\therefore D\geq
  0 \Rightarrow (k - 3)^2 - 9\geq 0 \Rightarrow k\leq 0$ or $k\geq 6$

  $f(-6) > 0 \Rightarrow -12(k - 3) + 45 > 0 \Rightarrow k < \frac{27}{4}$

  $f(1) > 0 \Rightarrow 10 + 2(k - 3) > 0 \Rightarrow k > -2$

  $-6 < \frac{\alpha + \beta}{2}\Rightarrow k < 9$ and $1 > \frac{\alpha + \beta}{2}\Rightarrow k > 2$

  Cobining all the conditions we have $6\leq k\leq 6.75$.
\item Let $\alpha, \beta$ be roots of the equation. When $2m + 1 < 0\Rightarrow m < -\frac{1}{2}$

  Also let $\alpha < 1 < \beta$, then $f(1) > 0\Rightarrow 2m + 1 - m + m - 2 > 0 \Rightarrow m
  > \frac{1}{2}$, which is not acceptable because $m < -\frac{1}{2}$.

  When $ > -\frac{1}{2}, f(1) < 0\Rightarrow m < \frac{1}{2}$

  Thus, we have $-\frac{1}{2} < m < \frac{1}{2}$.
\item Let $\alpha, \beta$ be the real roots of the given equation between $-2$ and $4$ such that
  $\alpha \leq \beta$.

  $D\geq 0 \Rightarrow 4m^2 - 4\left(m^2 - 1\right) > 0 \Rightarrow -\infty\leq m\leq \infty$

  $f(-2) > 0 \Rightarrow 4 + 4m + m^2 - 1 > 0 \Rightarrow m < -3$ or $m > -1$

  $f(4) > 0\Rightarrow 16 - 8m + m^2 - 1 > 0 \Rightarrow m < 3$ or $m > 5$

  $-2<\frac{\alpha + \beta}{2}\Rightarrow m > -2$ and $4 > \frac{\alpha + \beta}{2}\Rightarrow m < 4$

  Combining all equations we have $-1 < m < 3$.
\item Let $D$ be the discriminant of the given equation. Since coefficient of $x^2$ is positive $D\nleq 0$
  for the given equality to hold.

  The given inequation holds only if $D > 0$. The given inequality also holds if roots, let them be $\alpha$
  and $\beta$, lie between $1, 2$.

  $f(1)\leq 0 \Rightarrow a^2 + 7a + 1\leq 0 \Rightarrow \frac{-7 - 3\sqrt{5}}{2}\leq  a\leq \frac{-7 +
    3\sqrt{5}}{2}$

  $f(2) \leq 0\Rightarrow a^2 + 8a + 4\leq 0 \Rightarrow -4 -2\sqrt{3}\leq a\leq -4 + 2\sqrt{3}$.

  Combining the results we have $\frac{-7 - 3\sqrt{5}}{2} \leq a\leq -4 + 2\sqrt{3}$.
\item The given inequality will be satisfied only if $D > 0$. Putting $x = a^2 - \frac{1}{2}$ in the given
  inequation

  $\left(a^2 - \frac{1}{2}\right)^2 + \left(1 - \frac{3}{2}a\right)\left(a^2 - \frac{1}{2}\right)
  + \frac{a^2}{2} - \frac{a}{2} < 0$

  $\Rightarrow \left(2a^2 - 1\right)(a - 1)(2a - 1) + 2a(a - 1) < 0\Rightarrow (a - 1)(2a + 1)\left(2a^2 -
  2a + 1\right) < 0$

  $\Rightarrow (a - 1)(2a + 1) < 0[\because 2a^2 - 2a + 1 > 0\forall a\in(-\infty, \infty)]$

  $\Rightarrow -\frac{1}{2} < a< 1$.
\item Let $y = 2^x$ so the equation becomes $y^2 - ay - (a - 3)\leq 0$ for at least one $y > 0$, because for
  real $x, 2^x > 0 \Rightarrow y > 0$

  $D = a^2 + 4(a - 3) = (a + 6)(a - 2)$. When $D < 0, y^2 - ay - (a - 3) > 0$ for all real $y$

  When $D = 0\Rightarrow y^2 - ay - (a - 3)\nless 0$ for any real $y$

  $y^2 - ay - (a - 3) = 0$ for $y = \frac{a}{2}\Rightarrow$ for $a = 2, y = 1$ is an acceptable solution.

  When $D > 0 \Rightarrow a < -6$ and $a > 2$. Thus, we see for $2\leq a< \infty$ is an acceptable solution.
\item Given inequality is $\frac{x - 2}{x + 2} > \frac{2x - 3}{4x - 1}\Rightarrow \frac{x - 2}{x + 2}
  - \frac{2x - 3}{4x - 1} > 0$

  $\Rightarrow \frac{x^2 - 5x + 4}{4x^2 + 7x - 2}>0 \Rightarrow \frac{(x - 1)(x - 4)}{(x + 2)(4x - 1)} > 0$

  The inequality will hold if both numerator and denominator are positive or negative.

  For both to be negative $1 < x < 4$ and $-2 < x < -\frac{1}{4}$ which gives no solution.

  For both to be positive $x < 1, x > 4$ and $x < -2$ and $x > -\frac{1}{4}$. Combining these  we have $x <
  -2$ and $-\frac{1}{4} < x < 1$.
\item When $x < -4$ the inequality becomes $x^2 - 8x - 32 - 48 > 0 \Rightarrow x^2 - 8x - 80 > 0$

  Roots of the corresponsing equation are $\frac{8\pm \sqrt{384}}{2} = 4\pm\sqrt{96}$; so the valid root
  would be $4 - \sqrt{96}$

  When $x > 4$ the inequality becomes $x^2 + 8x - 16 > 0$

  Roots of the corresponding equation are $\frac{-8\pm\sqrt{128}}{2}= -4\pm\sqrt{32}$; so the valid root
  would be $-4 + \sqrt{32}$.

  Thus complete set of values is $(-\infty, 4 - 4\sqrt{6})\cup (-4 + 4\sqrt{2}, \infty)$.
\item For the logrithm to be valid $x^2 - 2x - 2 \geq 0$. The roots are $\frac{2\pm\sqrt{12}}{2} =
  1\pm\sqrt{3}$. So $x < 1 - \sqrt{3}$ or $x > 1 + \sqrt{3}$

  Given condition is $\log_{10}(x^2 - 2x - 2) \leq 0 \Rightarrow x^2 - 2x - 2\leq 1\Rightarrow x^2 - 2x -
  3\leq 0$

  Roots of corresponding equation are $1\pm2 = -1, 3$ and $x$ must lie between these roots.

  Combining both set of solutions we have $-1\leq x< 1 - \sqrt{3}$ or $1 + \sqrt{3}< x\leq 3$.
\item Discriminant of the denominator is $8^2 - 4.7.4 = 64 - 112 < 0$ so denominator will always be
  positive because coeff.\ of $x^2$ is positive.

  We can write numerator as $(x - 1)(x + 6)(x + 1)(x + 4) + 25 = \left(x^2 + 5x - 6\right)\left(x^2 + 5x +
  4\right)+ 25$

  Let $y = x^2 + 5x$ making numerator $(y - 6)(y + 4) + 25 = y^2 -2y + 1 = (y - 1)^2\geq 0$

  And thus, we have proven both numerator and denominator or the entire expression to be $\geq 0$.
\item Rewriting the given inequality we have $x^2 + (2y - 6)x + 3y^2 - 2y + 11 \geq 0$

  Since the coeff.\ of $x^2$ is positive the discriminant must be less than zero.

  $\Rightarrow (2y - 6)^2 - 4(3y^2 - 2y + 11) \leq 0$

  $\Rightarrow 4y^2 - 24y + 36 - 12y^2 + 8y - 44\leq 0 \Rightarrow -8y^2 - 16y -8 \leq 0$

  $\Rightarrow (y + 1)^2\geq 0$, which is always true.
\item Let $y = \frac{ax^2 - 7x + 5}{5x^2 - 7x + a}\Rightarrow (5y - a)x^2 -7(y - 1)x + ay - 5 = 0$

  Since $x$ is real the discriminant must be greater than zero.

  $\Rightarrow 49(y - 1)^2 - 4(5y - a)(ay - 5)\geq 0\Rightarrow (49 - 20a)y^2 + 2\left(2a^2 + 1\right)y + 49
  - 20a \geq 0$

  For this to hold $49 - 20a\geq 0$ and $D \leq 0$

  $\Rightarrow 4\left(2a^2 + 1\right)^2 - 4(49 - 20a)^2\leq 0 \Rightarrow (a - 5)^2(a + 12)(a - 2)\leq 0$

  $\Rightarrow -12\leq a\leq 2$.
\item Let $\frac{x^2 + 2x + c}{x^2 + 4x + 3c} = y \Rightarrow (y - 1)x^2 + (4y - 2)x + (3y - 1)c = 0$

  Since $x$ is real the discriminant must be greater than or equal to zero.

  $\Rightarrow (4y - 2)^2 - 4(y - 1)(3y - 1)c\geq 0\Rightarrow 4y^2 - 4y + 1 - 3y^2c + 4yc - c\geq 0$

  $(4 - 3c)y^2 -4y(1 - c) + 1 - c\geq 0$

  If $4 - 3c > 0$ then $D < 0$ for the above inequality to hold.

  $\Rightarrow 16(1 - c)^2 - 4(1 - c)(4 - 3c) < 0 \Rightarrow (1 - c)(4 - 4c - 4 + 3c) \leq 0$

  $\Rightarrow -c(1 - c)\leq 0 \Rightarrow 0\leq c\leq 1$.
\item Given that $-2< \frac{y - 3}{y + 1} < 2$

  First we consider $\frac{y - 3}{y + 1} < 2 \Rightarrow \frac{y - 3}{y + 1} - 2 < 0 \Rightarrow \frac{y +
    5}{y + 1} > 0$

  $\Rightarrow y < -5$ or $y > -1$

  Now we consider $-2 < \frac{y - 3}{y + 1}\Rightarrow \frac{3y - 1}{y + 1}\Rightarrow y < -1$ or $y >
  -\frac{1}{3}$

  Combining both parts we have $(-\infty, -5)\cup \left(\frac{1}{3}, \infty\right)$ as the solution set.
\item Let $r$ be the common ratio of the G.P., then given equation becomes $\frac{b}{r} + b + br =
  xb \Rightarrow r^2 + (1 - x)r + 1 = 0$

  Since $r$ is real, and $a b, c$ are distinct,

  therefore $D> 0 \Rightarrow (1 - x)^2 - 4 > 0 \Rightarrow x^2 -2x - 3 > 0$

  $\Rightarrow x < -1$ or $x > 3$.
\item We have been given $\log_x2.\log_{2x}2.\log_24x > 1\Rightarrow \frac{\log 2}{\log x}.\frac{\log
  2}{\log 2x}.\frac{\log 4x}{\log 2} > 1$

  $\Rightarrow \frac{\log 2.\log 4x}{\log x.\log 2x}> 1 \Rightarrow \log 2\left(\log 4 + \log x\right) -
  \log x(\log 2 + \log x) > 0$

  $\Rightarrow -(\log x)^2 - \log 2\log\ 4 > 0 \Rightarrow 2^{-\sqrt{2}} < x < 2^{\sqrt{2}}$

  However, $x\neq 1$ and $x\neq \frac{1}{2}$ for the logrithms to be defined. We also find that given
  inequality is satisfied for $\frac{1}{2} < x < 1$

  Combining these we have $\left(2^{-\sqrt{2}}, \frac{1}{2}\right)\cup \left(\frac{1}{2},
  1\right\cup\left(1, 2^{\sqrt{2}}\right)$.
\item We have been given that $x^{\left(\log_{10}x\right)^2 - 3\log_{10}x + 1} > 1000$

  Taking $\log$ of both sides with base $10$

  $\log_{10}x\left[\left(\log_{10}x\right)^2 - 3\log_{10}x + 1\right] > 3$

  Let $\log_{10}x = y$, which transforms the equation to $y^3 - 3y^2 + y - 3 > 0$

  $\Rightarrow \left(y^2 + 1\right)(y - 3) > 0$

  For real $y$, $\left(y^2 + 1\right)\geq 0$, therefore,  $y > 3 \Rightarrow x > 1000$.
\item We can rewrite the equation as $x^2 + \frac{b}{a}x + \frac{c}{a} = 0$

  Now its $D\geq 0$ because roots are real and coeff.\ of $x^2$ is positive so value of entire expression
  will be negative.

  Thus, $f(\alpha) + f\left(\beta\right) < 0$

  $\Rightarrow f(-1) = 1 -\frac{b}{a} + \frac{c}{a} < 0$ and $f(1) = 1 + \frac{b}{a} + \frac{c}{a} < 0$

  Adding we get the desired inequality.
\item $D = (a + 3)^2 + 4a + 16 = (a + 5)^2\geq 0$

  Roots are $\frac{2}{a + 4}, -4$. So $\frac{2}{a + 4}$ must be positive. This is possible for $a\in(-4,
  \infty)$.
\item Dicrimininant of the given equation is $D = (8 - 2m^2) + 4(8 - 3m)(m - 2) = 8m(m - 3)$, which is
  positive in the range of $(0, 3)$. For this range roots will be real.

  If one of the roots is positive and another is negative then product of the roots will be negative.

  Thus, we consider $\frac{3m - 8}{m - 2} < 0$.

  First we consider numerator positive and denominator negative. Then $m > \frac{8}{3}$ and $m < 2$. Both if
  these cannot be satisfied at the same time.

  Now we consider numerator negative and denominator positive. Then $m < \frac{8}{3}$ and $m > 2$. Combining
  we have the range of $m$ as $2 < m <\frac{8}{3}$.
\item Since both the roots are positive therefore product of the roots is positive. $\Rightarrow 8m - 15 > 0
  \Rightarrow m > \frac{15}{8}$.

  We also have to impose the condition that $D\geq 0 \Rightarrow 4m^2 - 32m + 60 \geq 0 \Rightarrow (m -
  5)(m - 3)\geq 0 \Rightarrow m \leq 3$ or $m \geq 5$

  Combining both the ranges we have $\frac{15}{8} < m \leq 3$ or $5\leq m < \infty$.
\item Let $y = \frac{x^2 + x + 2}{x^2 + x + 1}$ then the equation becomes $y^2 - (a - 3)y + (a - 4) = 0$

  Since $y$ is real, therefore $D\geq 0\Rightarrow (y - 1)x^2 + (y - 1)x + y - 2 = 0$

  $\Rightarrow (y - 1)^2 - 4(y - 1)(y - 2)\geq 0 \Rightarrow 3y^2 - 10y + 7\leq 0 \Rightarrow 1 < y \leq
  \frac{7}{3}$

  Discriminant of the equation $y^2 - (a - 3)y + (a - 4) = 0$ is

  $D' = (a - 3)^2 - 4(a - 4) = (a - 5)^2\geq 0$

  $y = \frac{a - 3 \pm (a - 5)}{2} = a - 4, 1 \Rightarrow 1< a - 4\leq \frac{7}{3}\Rightarrow 5 < a\leq
  \frac{19}{3}$.
\stopitemize
