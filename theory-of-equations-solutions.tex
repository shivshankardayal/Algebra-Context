% -*- mode: context; -*-
\chapter{Theory of Equations}
\startitemize[n, 1*broad]
\item For roots to be equal the discriminant has to be zero.

   $D = 4(1 + 3m)^2 - 4(1 + m)(1 + 8m) = 0\Rightarrow 4(1 + 9m^2 + 6m - 1 - 9m -8m^2) = 0\Rightarrow m^2 -
  3m = 0 \therefore m = 0, 3$
\item Discriminant of the equation is: $D = (c + a - b)^2 - 4(b + c - a)(a + b -c) = 4(b^2 - 4ac)$

  Given $a + b + c = 0 \Rightarrow b = -(a + c).$ Substituting in above equation, $D = 4\{(a + c)^2 - 4ac\}
  = 4(a - c)^2 =$ a perfect square and thus roots are rational.
\item Discriminant of the equation is: $D = 4(ac + bd)^2 - 4(a^2 + b^2)(c^2 + d^2) = -4(ad - bc)^2$. Roots
  are real if $D\geq 0$ i.e. $-4(ad - bc)^2 \geq 0 \Rightarrow (ad - bc)^2 \leq 0$

  But since $(ac - bd)^2 \nless 0 \therefore (ad - bc)^2 = 0$ i.e. $D = 0$ (because roots are real). Thus,
  if roots are real they are equal.
\item Let $A = a(b - c), B = b(c - a)$ and $c = c(a - b)$ Clearly, $A + B + C = 0$. Since roots are equal
  i.e. $D = 0 \therefore B^2 - 4AC = 0$

  Substituting for $B, [-(A + C)^2 - 4AC] = (A - C)^2 = 0 \Rightarrow A = C \Rightarrow 2ac = ab + cb
  \Rightarrow b = \frac{2ac}{a + c}$.

  Thus, $a, b, c$ are in H. P.
\item Given equation is $(b - x)^2 - 4(a - x)(c - x) = 0\Rightarrow -3x^2 + 2(2a + 2c - b)x + b^2 - 4ac = 0$

  Discriminant of the above equation is: $D = 4(2a + 2c - b)^2 + 12(b^2 - 4ac) = 8[(a - b)^2 + (b - c)^2 +
    (c - a)^2]\because a, b, c$ are real $\therefore D > 0$ unless $a = b = c$.

  Hence, roots are real unless $a = b = c$.
\item Discriminant of the equations are $p^2 - 4q$ and $r^2 - 4s$.

  Adding them we have $p^2 + r^2 - 4(q + s) = p^2 + r^2 - 2pr = (p - r)^2 \geq 0$.

  Thus, at least one of the discriminant is greater than zero and that equation has real roots.
\item Since $x^2 - 2px + q = 0$ has equal roots $D = 0 \Rightarrow 4p^2 - 4q = 0 \Rightarrow p^2 = q$.

  Discriminant of the second equation: $D = 4(p + y)^2 - 4(1 + y)(q + y) = 4[p^2 + 2y + y^2 - q -qy -y - y^2]$

  Substituting for $q, D = -4y(p - 1)^2$. Roots of the equation will be real and distinct only if $D \geq
  0$ but $(p - 1) \geq 0$ if $p \neq 1$. Thus, $y$ has to be negative as well.
\item Since roots of equation $ax^2 + 2bx + c = 0$ are equal $\therefore 4b^2 - 4ac \geq 0$. Discriminant of
  the equation $ax^2 + 2mbx + nc = 0$ is $4m^2b^2 - 4anc$.

  Since $m^2 > n > 0$ and $b^2 \geq ac$ $4m^2b^2 - 4anc > 0$. Thus, roots of the second equation are real.
\item Given $ax + by = 1 \Rightarrow y = \frac{1 - ax}{b},$ substituting this in second equation, $cx^2 +
  d\left(\frac{1 - ax}{b}\right)^2 = \frac{b^2cx^2 + d(1 - ax)^2}{b^2} = 1$

  $\Rightarrow (b^2c + da^2)x^2 - 2adx + d - b^2 = 0$. Since first two equations have one solution this
  equation will also have only one solution which means roots will be equal i.e. $D = 0$

  $\Rightarrow 4a^2d^2 - 4(b^2c + a^d)(d - b^2) = 0\Rightarrow b^2(b^2c + a^2d - cd) = 0\because b^2 \ne 0
  \therefore b^2c + a^2d - cd = 0 \Rightarrow b^2c + a^d = cd$

  Dividing both sides by $cd$ we have

  $\frac{b^2}{d} + \frac{a^2}{c} = 1\Rightarrow x = \frac{2ad}{2(b^2c + a^2d)} = \frac{a}{c}$. Substituting
  for $y,$ we get $y = \frac{b}{d}$.
\item Let the roots of the equation be $\alpha$ and $r\alpha$.

  Sum of roots = $\alpha + r\alpha = -\frac{b}{a} \Rightarrow \alpha = -\frac{b}{a(r + 1)}$.

  Product of roots $= r\alpha^2 = \frac{rb^2}{a^2(1 + r)^2} = \frac{c}{a} \Rightarrow \frac{b^2}{ac} =
  \frac{(r + 1)^2}{r}$.
\item Let the roots of the equation be $\alpha$ and $2\alpha.$. Sum of roots $= 3\alpha = -\frac{l}{l - m}
  \Rightarrow \alpha = -\frac{l}{l - m}$.

  Product of roots $= 2\alpha^2 = \frac{1}{l - m}$. Substituting for $\alpha, \frac{2l^2}{9(l - m)^2} =
  \frac{1}{l - m} \Rightarrow 2l^2- 9l + 9m = 0 [\because l\neq m~\text{else it would not be a quadratic
      equation}]$.

  Since $l$ is real, therefore discriminant of this equation would be $\geq 0, \Rightarrow 81 - 72m \geq 0
  \therefore m \leq \frac{9}{8}$.
\stopitemize