% -*- mode: context; -*-
\chapter{Theory of Equations}
\startitemize[n, 1*broad]
\item For roots to be equal the discriminant has to be zero.

   $D = 4(1 + 3m)^2 - 4(1 + m)(1 + 8m) = 0\Rightarrow 4(1 + 9m^2 + 6m - 1 - 9m -8m^2) = 0\Rightarrow m^2 -
  3m = 0 \therefore m = 0, 3$
\item Discriminant of the equation is: $D = (c + a - b)^2 - 4(b + c - a)(a + b -c) = 4(b^2 - 4ac)$

  Given $a + b + c = 0 \Rightarrow b = -(a + c).$ Substituting in above equation, $D = 4\{(a + c)^2 - 4ac\}
  = 4(a - c)^2 =$ a perfect square and thus roots are rational.
\item Discriminant of the equation is: $D = 4(ac + bd)^2 - 4(a^2 + b^2)(c^2 + d^2) = -4(ad - bc)^2$. Roots
  are real if $D\geq 0$ i.e. $-4(ad - bc)^2 \geq 0 \Rightarrow (ad - bc)^2 \leq 0$

  But since $(ac - bd)^2 \nless 0 \therefore (ad - bc)^2 = 0$ i.e. $D = 0$ (because roots are real). Thus,
  if roots are real they are equal.
\item Let $A = a(b - c), B = b(c - a)$ and $c = c(a - b)$ Clearly, $A + B + C = 0$. Since roots are equal
  i.e. $D = 0 \therefore B^2 - 4AC = 0$

  Substituting for $B, [-(A + C)^2 - 4AC] = (A - C)^2 = 0 \Rightarrow A = C \Rightarrow 2ac = ab + cb
  \Rightarrow b = \frac{2ac}{a + c}$.

  Thus, $a, b, c$ are in H. P.
\item Given equation is $(b - x)^2 - 4(a - x)(c - x) = 0\Rightarrow -3x^2 + 2(2a + 2c - b)x + b^2 - 4ac = 0$

  Discriminant of the above equation is: $D = 4(2a + 2c - b)^2 + 12(b^2 - 4ac) = 8[(a - b)^2 + (b - c)^2 +
    (c - a)^2]\because a, b, c$ are real $\therefore D > 0$ unless $a = b = c$.

  Hence, roots are real unless $a = b = c$.
\item Discriminant of the equations are $p^2 - 4q$ and $r^2 - 4s$.

  Adding them we have $p^2 + r^2 - 4(q + s) = p^2 + r^2 - 2pr = (p - r)^2 \geq 0$.

  Thus, at least one of the discriminant is greater than zero and that equation has real roots.
\item Since $x^2 - 2px + q = 0$ has equal roots $D = 0 \Rightarrow 4p^2 - 4q = 0 \Rightarrow p^2 = q$.

  Discriminant of the second equation: $D = 4(p + y)^2 - 4(1 + y)(q + y) = 4[p^2 + 2y + y^2 - q -qy -y - y^2]$

  Substituting for $q, D = -4y(p - 1)^2$. Roots of the equation will be real and distinct only if $D \geq
  0$ but $(p - 1) \geq 0$ if $p \neq 1$. Thus, $y$ has to be negative as well.
\item Since roots of equation $ax^2 + 2bx + c = 0$ are equal $\therefore 4b^2 - 4ac \geq 0$. Discriminant of
  the equation $ax^2 + 2mbx + nc = 0$ is $4m^2b^2 - 4anc$.

  Since $m^2 > n > 0$ and $b^2 \geq ac$ $4m^2b^2 - 4anc > 0$. Thus, roots of the second equation are real.
\item Given $ax + by = 1 \Rightarrow y = \frac{1 - ax}{b},$ substituting this in second equation, $cx^2 +
  d\left(\frac{1 - ax}{b}\right)^2 = \frac{b^2cx^2 + d(1 - ax)^2}{b^2} = 1$

  $\Rightarrow (b^2c + da^2)x^2 - 2adx + d - b^2 = 0$. Since first two equations have one solution this
  equation will also have only one solution which means roots will be equal i.e. $D = 0$

  $\Rightarrow 4a^2d^2 - 4(b^2c + a^d)(d - b^2) = 0\Rightarrow b^2(b^2c + a^2d - cd) = 0\because b^2 \ne 0
  \therefore b^2c + a^2d - cd = 0 \Rightarrow b^2c + a^d = cd$

  Dividing both sides by $cd$ we have

  $\frac{b^2}{d} + \frac{a^2}{c} = 1\Rightarrow x = \frac{2ad}{2(b^2c + a^2d)} = \frac{a}{c}$. Substituting
  for $y,$ we get $y = \frac{b}{d}$.
\item Let the roots of the equation be $\alpha$ and $r\alpha$.

  Sum of roots = $\alpha + r\alpha = -\frac{b}{a} \Rightarrow \alpha = -\frac{b}{a(r + 1)}$.

  Product of roots $= r\alpha^2 = \frac{rb^2}{a^2(1 + r)^2} = \frac{c}{a} \Rightarrow \frac{b^2}{ac} =
  \frac{(r + 1)^2}{r}$.
\item Let the roots of the equation be $\alpha$ and $2\alpha.$. Sum of roots $= 3\alpha = -\frac{l}{l - m}
  \Rightarrow \alpha = -\frac{l}{l - m}$.

  Product of roots $= 2\alpha^2 = \frac{1}{l - m}$. Substituting for $\alpha, \frac{2l^2}{9(l - m)^2} =
  \frac{1}{l - m} \Rightarrow 2l^2- 9l + 9m = 0 [\because l\neq m~\text{else it would not be a quadratic
      equation}]$.

  Since $l$ is real, therefore discriminant of this equation would be $\geq 0, \Rightarrow 81 - 72m \geq 0
  \therefore m \leq \frac{9}{8}$.
\item Let the roots be $\alpha$ and $\alpha^n$, then sum of roots $= \alpha + \alpha^n = -\frac{b}{a}$ and
  product of roots $= \alpha^{n + 1} = \frac{c}{a}$.

  From products, we have $\alpha = \left(\frac{c}{a}\right)^{\frac{1}{n + 1}}$. From sum we have $a\alpha^n + a\alpha + b = 0$.

  Substituting value of $\alpha$ from above $\Rightarrow a\left(\frac{c}{a}\right)^{\frac{n}{n + 1}} +
  a\left(\frac{c}{a}\right)^{\frac{1}{n + 1}} + b = 0$.  From this we arrive at our desired equation.
\item Let the roots be $p\alpha$ and $q\alpha$.

  Sum of roots $= (p + q)\alpha = -\frac{b}{a}$ and product of roots $= pq\alpha^2 = \frac{c}{a}$.

  From equation for product of roots, we have $\alpha^2 = \frac{c}{apq} \therefore \alpha = \sqrt{\frac{c}{apq}}$.

  Substituting this in sum of roots and solving we arrive at desired equation.
\item The questions are solved below:
  \startitemize[i]
  \item $\alpha + \beta = -p$ and $\alpha\beta = q$. Now, $\frac{\alpha^2}{\beta} + \frac{\beta^2}{\alpha} =
    \frac{\alpha^3 + \beta^3}{\alpha\beta}$

    $= \frac{(\alpha + \beta)^3 - 3\alpha\beta(\alpha + \beta)}{\alpha\beta} = \frac{p(3q - p^2)}{q}$.
  \item $(\omega\alpha + \omega^2\beta)(\omega^2\alpha + \omega\beta) = \omega^3\alpha^2 +
    \omega^4\alpha\beta + \omega^2\alpha\beta + \omega^3\beta^2$

    $= \alpha^2 + \omega\alpha\beta + \omega^2\alpha\beta + \beta^2 = \alpha^2 -\alpha\beta + \beta^2 =
    (\alpha + \beta)^2 - 3\alpha\beta = p^2 - 3q$.
  \stopitemize
\item Rewriting the equation we have $(A + cm^2)x^2 + Amx + Am^2 = 0$.

  Sum of roots $= \alpha + \beta = -\frac{Am}{A + cm^2}$ and product of roots $= \alpha\beta = \frac{Am^2}{A + cm^2}$

  The expression to be evaluated is $A(\alpha^2 + \beta^2) + A\alpha\beta + c\alpha^2\beta^2$.

  $= A[(\alpha + \beta)^2 - 2\alpha\beta] + A\alpha\beta + c(\alpha\beta)^2$.

  $= A\left[\frac{A^2m^2}{(A + cm^2)^2} - \frac{2Am^2}{A + cm^2}\right] + \frac{A^2m^2}{A + cm^2} + \frac{cA^2m^4}{(A +
    cm^2)^2} = 0$.
\item Sum of roots $= \alpha + \beta = -\frac{b}{a}$ and product of roots $= \alpha\beta = \frac{c}{a}$.

  Now, $a\left(\frac{\alpha^2}{\beta} + \frac{\beta^2}{\alpha}\right) + b\left(\frac{\alpha}{\beta} +
  \frac{\beta}{\alpha}\right) = \frac{a(\alpha^3 + \beta^3)}{\alpha\beta} + \frac{b(\alpha^2 +
    \beta^2)}{\alpha\beta}$

  $= a\frac{[(\alpha + \beta)^3 - 3\alpha\beta(\alpha + \beta)]}{\alpha\beta} + \frac{b[(\alpha + \beta)^2 -
      2\alpha\beta]}{\alpha\beta}$. Substituting for sum and product of the roots $=
  \frac{a\left[\left(-\frac{b}{a}\right)^3 - 3.\frac{c}{a}\left(-\frac{b}{a}\right)\right]}{\frac{c}{a}} +
  \frac{b\left[\left(-\frac{b}{a}\right)^2 -2 \frac{c}{a}\right]}{\frac{c}{a}}$

  Solving this we get the desired result.
\item Since $a$ and $b$ are the roots of the equation $x^2 + px + 1 = 0$ we have $a + b = -p$ and $ab = 1$.

  Similarly, since $c$ and $d$ are the roots of the equation $x^2 + qx + 1 = 0$ we have $c + d = -p$ and
  $cd = 1$.

  Now $(a - c)(b - c)(a + d)(b + d) = (ab - bc - ac + c^2)(ab + bd + ad + d^2) = [ab - c(a + b) + c^2].[ab +
    d(a + b) + d^2]$

  $= [1 + pc + c^2].[1 - pd + d^2] (\text{putting the values of } a + b~\text{and}~ab) = 1 + cp + c^2 - pd -
  cdp^2 - c^2pd + d^2 + cpd^2 + c^2d^2$

  $= 1 + (c^2 + d^2) + c^2d^2 -cdp^2 + p(c - d) + cpd(d - c) = 1 + [(c + d)^2 - 2cd] + c^2d^2 - cdp^2 + p(c
  - d) + cpd(d - c)$.

  Substituting for $c + d$ and $cd, 1 + q^2 - 2 + 1 - p^2 + p(c - d) + p(d - c) = q^2 - p^2$.
\item Let $\alpha$ and $\beta$ be the roots of the equation $x^2 + px + q = 0$ then $\alpha + \beta = -p$ and
  $\alpha\beta = q$.

  Also, let $\gamma$ and $\delta$ be the roots of the equation $x^2 + qx + p = 0$ then $\gamma + \delta =
  -q$ and $\gamma\delta = p$.

  Now, given is that roots differ by the same quantity so we can say that, $\alpha - \beta = \gamma -
  \delta\Rightarrow (\alpha - \beta)^2 = (\gamma - \delta)^2$

  $(\alpha + \beta)^2 - 4\alpha\beta = (\gamma + \delta)^2 - 4\gamma\delta\Rightarrow p^2 - 4q = q^2 - 4p
  \Rightarrow p^2 - q^2 + 4(p - q) = 0 \Rightarrow (p - q)(p + q + 4) = 0$

  Clearly, $p \neq q$ else equations would be same $\therefore p + q + 4 = 0$.
\item Since $\alpha, \beta$ are the roots of the equation $ax^2 + bx + c = 0\therefore a\alpha^2 + b\alpha +
  c = 0$ and $a\beta^2 + b\beta + c = 0$.

  and $\alpha + \beta = -\frac{b}{a}$ and $\alpha\beta = \frac{c}{a}.$ Also, given $S_n = \alpha^n +
  \beta^n$. Now, $aS_{n + 1} + bS_n + cS_{n - 1}$

  $= a(\alpha^{n + 1} + \beta^{n + 1}) + b(\alpha^n + \beta^n) + c(\alpha^{n - 1} + \beta^{n - 1}) =
  \alpha^{n - 1}(a\alpha^2 + b\alpha + c) + \beta^{n - 1}(a\beta^2 + b\beta + c) = \alpha^{n - 1}.0 +
  \beta^{n - 1}.0$

  $\therefore S_{n + 1} = -\frac{b}{a}S_n -\frac{c}{a}S_{n - 1}$

  Substituting $n = 4$ we have

  $S_5 = -\frac{b}{a}S_4 - \frac{c}{a}S_3 = -\frac{b}{a}(-\frac{b}{a}S_3 - \frac{c}{a}S-2) - \frac{c}{a}S_3
  = \left(\frac{b^2}{a^2} - \frac{c}{a}\right)S_3 + \frac{bc}{a^2}S_2$

  Proceeding similarly we have the solution as

  $= -\frac{b}{a^5}(b^2 - 2ac)^2 + \frac{(b^2 - ac)bc}{a^4}$.
\item Let $\alpha$ and $\beta$ be the roots of the equation $ax^2 + bx + c = 0$. Given, $\alpha + \beta =
  \frac{1}{\alpha^2} + \frac{1}{\beta^2}\Rightarrow \alpha + \beta = \frac{(\alpha + \beta)^2 -
    2\alpha\beta}{\alpha^2\beta^2}$

  $-\frac{b}{a} = \frac{\frac{b^2}{a^2} - 2\frac{c}{a}}{\frac{c^2}{a^2}} = \frac{b^2 - 2ac}{c^2}\Rightarrow
  -bc^2 = ab^2 - 2a^2c \Rightarrow ca^2 = \frac{ab^2 + bc^2}{2}$

  Thus, $bc^2, ca^2, ab^2$ are in A. P.
\item Rewriting the equation $m^2x^2 + (2m - m^2)x + 3 = 0$.

  Since $\alpha$ and $\beta$ are the roots of the equation $\alpha + \beta = -\frac{2m - m^2}{m^2} = \frac{m -
    2}{m}$ and $\alpha\beta = \frac{3}{m^2}$

  Given, $\frac{\alpha}{\beta} + \frac{\beta}{\alpha} = \frac{4}{3} \Rightarrow \frac{\alpha^2 + \beta^2}{\alpha\beta} =
  \frac{4}{3}$

  $3(\alpha^2 + \beta^2) = 4\alpha\beta \Rightarrow 3[(\alpha + \beta)^2 - 2\alpha\beta] =
  4\alpha\beta\Rightarrow 3(\alpha + \beta)^2 - 10\alpha\beta = 0 \Rightarrow 3\left[\left(\frac{m -
      2}{m}\right)^2 - \frac{10}{m^2}\right] = 0$

  $\Rightarrow m^2 - 4m - 6 = 0$

  Since $m_1, m_2$ are two values of $m$ we have $m_1 + m_2 = 4$ and $m_1m_2 = -6$. Now, $\frac{m_1^2}{m_2}
  + \frac{m_2^2}{m_1} = \frac{m_1^3 + m_2^3}{m_1m_2} = \frac{(m_1 + m_2)^3 - 3m_1m_2(m_1 + m_2)}{3m_1m_2} =
  -\frac{68}{3}$.
\item Let $\alpha$ and $\beta$ be the roots of the equation $ax^2 + bx + c = 0;$ $\gamma$ and $\delta$
  are the roots of the equation $a_1x^2 + b_1x + c_1 = 0,$ then

  $\alpha + \beta = -\frac{b}{a}, \alpha\beta = \frac{c}{a}$ and $\gamma + \delta = -\frac{b_1}{a_1},
  \gamma\delta = \frac{c_1}{a_1}$

  According to question, $\frac{\alpha}{\beta} = \frac{\gamma}{\delta}$. By componendo and dividendo,

  $\frac{\alpha - \beta}{\alpha + \beta} = \frac{\gamma - \delta}{\gamma + \delta}$. Squaring both sides

  $\Rightarrow \left(\frac{\alpha - \beta}{\alpha + \beta}\right)^2 = \left(\frac{\gamma - \delta}{\gamma +
    \delta}\right)^2$ $\Rightarrow \frac{(\alpha + \beta)^2 - 4\alpha\beta}{(\alpha + \beta)}^2 = \frac{(\gamma + \delta)^2 - 4\gamma\delta}{(\gamma +
      \delta)^2}$

  $\Rightarrow \frac{b^2 - 4ac}{b^2} = \frac{b_1^2 - 4a_1c_1}{b_1^2} \Rightarrow -4acb_1^2 = -4a_1c_1b^2 \Rightarrow
  \left(\frac{b}{b_1}\right)^2 = \frac{ac}{a_1c_1}$.
\item Since irrational roots appear in pairs and are conjugate. Thus, if first root is $\alpha = \frac{1}{2 + \sqrt{5}}$

  $\alpha = \frac{1}{2 + \sqrt{5}}\frac{2 - \sqrt{5}}{2 - \sqrt{5}} = \frac{2 - \sqrt{5}}{4 - 5} = -2 + \sqrt{5}$

  Then second root would be $\beta = -2 + \sqrt{5}$ $\Rightarrow \alpha + \beta = -4$ and $\alpha\beta = -1$

  Therefore, the equation is $x^2 - (\alpha + \beta)x + \alpha\beta = 0 \Rightarrow x^2 + 4x -1 = 0$.
\item Since $\alpha$ and $\beta$ are the roots of the equation $\therefore \alpha + \beta = -\frac{b}{a}$ and
  $\alpha\beta = \frac{c}{a}$. Sum of the roots for which quadratic equation is to be found $=
  \frac{1}{a\alpha + b} + \frac{1}{a\beta + b}$

  $= \frac{a(\alpha + \beta) + 2b}{a^2\alpha\beta + ab(\alpha + \beta) + b^2} = \frac{a\left(-\frac{b}{a}\right) +
    2b}{a^2.\frac{c}{a} + av\left(-\frac{b}{a}\right)} + b^2 = \frac{b}{ac}$

  Product of the roots $= \left(\frac{1}{a\alpha + b}\right)\left(\frac{1}{a\beta + b}\right) =
  \frac{1}{a^2\alpha\beta + ab(\alpha + \beta) + b^2} = \frac{1}{a^2.\frac{c}{a} +
    ab\left(-\frac{c}{a}\right) + b^2} = \frac{1}{ac}$.

  Therefore, the equation is $x^2 - \frac{b}{ac}x + \frac{1}{ac} = 0 \Rightarrow acx^2 - bx + 1 = 0$.
\item Given equation is $(x - a)(x - b) - k = 0 \Rightarrow x^2 - (a + b)x + ab - k = 0$.

  Since $c, d$ are roots of this equation $\Rightarrow c + d = a + b$ and $cd = ab - k$.

  The equation where roots are $a, b$ is $x^2 - (a + b)x + ab = 0 \Rightarrow x^2 - (c + d)x + cd + k = 0$.
\item Correct equation is $x^2 + 13x + q = 0$ and incorrect equation is $x^2 + 17x + q = 0$.

  Roots of correct incorrect equation are $-2$ and $-15.$ Thus $q = 30$.

  Therefore, correct equation is $x^2 + 13x + 30 = 0$ and thus roots are $-3, -10$.
\item Clearly, $\alpha + \beta = -p$ and $\alpha\beta = q$. Substituting $x = \frac{\alpha}{\beta}$ in the
  given equation we have

  $q\frac{\alpha^2}{\beta^2} - (p^2 - 2q)\frac{\alpha}{\beta} + q = 0 \Rightarrow q\alpha^2 - (p^2 -
  2q)\alpha\beta + q\beta^2 = 0$

  $q(\alpha^2 + \beta^2) - (p^2 - 2q)q = 0 \Rightarrow q[(\alpha + \beta)^2 - 2\alpha\beta] - (p^2 - 2q)q = 0$

  $q(p^2 - 2q) - (p^2 - 2q)q = 0 \Rightarrow 0 = 0$. Thus, $\frac{\alpha}{\beta}$ is a root of the given
  equation.
\item Let $\alpha$ and $\beta$ be the roots of $x^2 - ax + b = 0$ and $\alpha$ be the common and equal root
  from the second equation $x^2 - px + q = 0$.

  Thus, $\alpha + \beta = a, \alpha\beta = b$ and $2\alpha = p, \alpha^2 = q\Rightarrow b + q =
  \alpha\beta + \alpha^2 = \alpha(\beta + \alpha) = \frac{p}{2}a = \frac{ap}{2}$.
\item Let $\alpha$ be the common root. Then, we have $a\alpha^2 + 2b\alpha + c = 0$ and $a_1\alpha^2 + 2b_1\alpha + c_1 = 0$.

  Solving equations by cross-multiplication we have $\frac{\alpha^2}{2(bc_1 - b_1c)} = \frac{\alpha}{(ca_1 -
    a_1c)} = \frac{1}{2(ab_1 - a_1b)}$.

  From first two we have $\alpha$ as $\alpha = \frac{2(bc_1 - b_1c)}{ca_1 - a_1c}$ and from last two we have
  $\alpha$ as $\alpha = \frac{ca_1 - ac_1}{2(ab_1 - a_1b)}$

  Equating we get, $\frac{2(bc_1 - b_1c)}{ca_1 - a_1c} = \frac{ca_1 - ac_1}{2(ab_1 - a_1b)}\Rightarrow (ca_1
  - ac_1)^2 = 4(ab_1 - a_1b)(bc_1 - b_1c)$

  Given, $\frac{a}{a_1}, \frac{b}{b_1}, \frac{c}{c_1}$ are in A. P., let $d$ be the common difference.

  $\left(\frac{c}{c_1} - \frac{a}{a_1}\right)^2c_1^2a_1^2 = 4\left(\frac{a}{a_1} -
  \frac{b}{b_1}\right)a_1b_2\left(\frac{b}{b_1} - \frac{c}{c_1}\right)b_1c_1$

  $(2d)^2c_1^2a_2^2 = 4(-d)a_1b_1(-d)b_1c_1\Rightarrow 4d^2c_1^2a_1^2 = 4d^2a_1c_1b_1^2 \Rightarrow c_1a_1 =
  b_1^2$.

  Thus, $a_1, b_1, c_1$ are in G. P.
\item Let $\alpha$ be the common root between first two, $\beta$ be the common root between last two and $\gamma$ be
  the common root between first and last equations.

  Thus, $\alpha$ and $\beta$ are the roots of the first equation. $\Rightarrow \alpha + \gamma = -p_1, \alpha\gamma = q_1$

  Similarly, $\alpha + \beta = -p_2, \alpha\beta = q_2\Rightarrow \beta + \gamma = -p_3, \beta\gamma = q_3$

  L.H.S. $= (p_1 + p_2 + p_3)^2 = 4(\alpha + \beta + \gamma)^2$ and R.H.S. $= 4(p_1p_2 + p_2p_3 + p_1p_3 - q_1 - q_2 - q_3)$

  $= 4[(\alpha + \gamma)(\alpha + \beta) + (\alpha + \beta)(\beta + \gamma) + (\alpha + \gamma)(\beta + \gamma) -
    \alpha\gamma - \alpha\beta - \beta\gamma]$

  $= 4(\alpha^2 + \beta^2 + \gamma^2 + 2\alpha\beta + 2\alpha\gamma + 2\beta\gamma) = 4(\alpha + \beta + \gamma)^2$.

  Hence, proven that L.H.S. = R.H.S.
\item Let $\alpha$ be the common root then we have, $\alpha^2 + c\alpha + ab = 0$ and $\alpha^2 + b\alpha + ca = 0$.

  By cross-multiplication, we get the solution as $\frac{\alpha^2}{ac^2 - ab^2} = \frac{\alpha}{ab - ac} =
  \frac{1}{b - c}$.

  From first two we have $\alpha = \frac{ac^2 - ab^2}{ab - ac} = -(b + c)$. From last two we have $\alpha = a$.

  Equating these two we get $a = -(b + c) \Rightarrow a + b + c = 0$. Let the other root of the equations be
  $\beta$ and $\beta_1$ then we have

  $\alpha\beta = ab$ and $\alpha\beta_1 = ca\therefore \beta = b$ and $\beta_1 = c$. Equation whose roots are
  $\beta$ and $\beta_1$ is

  $x^2 - (\beta + \beta_1)x + \beta\beta_1 = 0 \Rightarrow x^2 -(b + c) + bc = 0 \Rightarrow x^2 + ax + bc =
  0$.
\item Clearly, root of the equation $x^2 + 2x + 9 = 0$ are imaginary and since they appear in pairs both the
  roots will be common and thus the ratio of the coefficients of the terms will be equal. $\Rightarrow a :
  b: c = 1 : 2 : 9$.
\item Let $\alpha$ be a common root. Then, we have $3\alpha^2 -2\alpha + p = 0$ and $6\alpha^2 - 17\alpha + 12 = 0$.

  Solving by cross-multiplication $\frac{\alpha^2}{-24 + 17p} = \frac{\alpha}{6p - 36} = \frac{1}{-39}$.

  From first two we have $\alpha = \frac{17p - 24}{6p - 36}$ and from last two we have $\alpha = \frac{6p -
    36}{-39} = -\frac{2p - 12}{13}$.

  Equating these two and solving for $p$ we get $p = -\frac{15}{4}, -\frac{8}{3}$.
\item When $x = 0, |x|^2 - |x| - 2 = |0|^2 - |0| - 2 = -2 \ne 0$. Since it is not satisfied by
  $x = 0$ it is an equation.
\item When $x = -a$ the equation is satisfied. Similarly, it is satisfied by values of $x$ being $-b$ and
  $-c$. The highest power of $x$ occurring is $2$ and is true for three distinct values of $x$
  therefore it cannot be equation but an identity.
\item Since both the equations have only one common root so the roots must be rational as irrational and
  complex roots appear in pairs. Thus, the roots of these two equations must be rational and therefore the
  discriminants must be perfect squares. Therefore, $b^2 - ac$ and $b_1^2 - a_1c_2$ must be perfect squares.
\item Equating the coefficients for similar powers of $x$, we get, coefficient of $x^2:$ $a^2 - 1 = 0
  \Rightarrow a = \pm1$.

  Coefficient of $x:$ $a - 1 = 0 \Rightarrow a = 1$. Constant term: $a^2 - 4a + 3 = 0 \Rightarrow a = 1, 3$.

  The common value of $a$ is 1 which will make this an identity.
\item Given, $\left(x + \frac{1}{x}\right)^2 = 4 + \frac{3}{2}\left(x - \frac{1}{x}\right)\Rightarrow
  \left(x + \frac{1}{x}\right)^2 - 4 - \frac{3}{2}\left(x - \frac{1}{x}\right) = 0\Rightarrow \left\{\left(x
  - \frac{1}{x}\right)^2 + 4x\frac{1}{x}\right\} - \frac{3}{2}\left(x - \frac{1}{x}\right) - 4 = 0$

  Substituting $a = x - \frac{1}{x}\Rightarrow a^2 - \frac{3}{2}a = 0 \Rightarrow 2a^2 - 3a = 0 \therefore a
  = 0, \frac{3}{2}$

  $x - \frac{1}{x} = 0 \Rightarrow x = \pm1\Rightarrow x - \frac{1}{x} - \frac{3}{2} \Rightarrow x = 2,
  -\frac{1}{2}$.
\item Given equation is $(x + 4)(x + 7)(x + 8)(x + 11) + 20 = 0$.

  Rewriting the equation, $[(x + 4)(x + 11)][(x + 7)(x + 8)] + 20 = 0$

  $\Rightarrow (x^2 + 15x + 44)(x^2 + 15x + 56) + 20 = 0$. Substituting $a = x^2 + 15x,$ we get $(a + 44)(a
  + 56) + 20 = 0\Rightarrow a = -46, -54$

  If $a = -46 \Rightarrow x^2 + 15x + 46 = 0 \Rightarrow x = \frac{-15 \pm \sqrt{41}}{2}$. If $a = -54
  \Rightarrow x^2 + 15x + 54 = 0 \Rightarrow x = - 6, -9$.
\item Given equation is $3^{2x + 1} + 3^2 = 3^{x + 3} + 3^x$. Let $3^x = a,$ then we have $3a^2 + 9 = 28a
  \Rightarrow 3a^2 - 28a + 9 = 0$.

  $\Rightarrow a = \frac{1}{3}, 9$. If $a = \frac{1}{3} \Rightarrow x = -1$. If $a = 9 \Rightarrow x = 2$.
\item Clearly, $(5 + 2\sqrt{6})^{x^2 - 3}(5 - 2\sqrt{6})^{x^2 - 3} = 1$. Let $(5 + 2\sqrt{6})^{x^2 - 3} = 1$
  then $(5 - 2\sqrt{6})^{x^2 - 3} = \frac{1}{y}$.

  The given equation becomes $y + \frac{1}{y} = 10$ where $y = (5 + 2\sqrt{6})^{x^2 - 3}\Rightarrow y^2 -10y
  + 1 = 0$.

  Solving the equation we have roots as $y = 5 \pm 2\sqrt{6}\therefore x^2 - 3 = \pm 1\Rightarrow x = \pm2,
  \pm\sqrt{2}$.
\item Let the speed of the bus $= x$ km/hour $\therefore$ the speed of car $= x + 25$ km/hour.

  Time taken by bus $= \frac{500}{x}$ hours and by car $= \frac{500}{x + 25}$ hours. Given, $\frac{500}{x} =
  \frac{500}{x + 25} + 10\Rightarrow x^2 - 25x + 1250 = 0$.

  $x = -50, 25$ but $x$ cannot be negative as it is a scalar quantity. Thus, speed of car = $50$ km/hour.
\item Given equation is $(a + b)^2x^2 - 2(a^2 - b^2)x + (a - b)^2 = 0$. Discriminant $= 4(a^2 - b^2)^2 - 4(a
  + b)^2(a - b)^2 = 0$. Since discriminant is zero, roots are equal.
\item Given equation is $3x^2 + 7x + 8 = 0$. Discriminant $D = 49 - 96 < 0$.

  Since it is negative roots will be complex and conjugate pair.
\item Given equation is $3x^2 + (7 + a) + 8 - a = 0$. Discriminant $D = (7 + a)^2 + 12a$

  For roots to be equal it has to be zero. $\Rightarrow a^2 + 26a + 49 = 0\Rightarrow a = 13 \pm
  6\sqrt{6}$.
\item It is given that roots are equal i.e. discriminant is zero. $\Rightarrow 4(ac + bd)^2 - 4(a^2 +
  b^2)(c^2 + d^2) = 0\Rightarrow a^2c^2 + b^2d^2 - 2abcd - a^2c^2 - a^2d^2 - b^2c^2 - b^2d^2 = 0$

  $\Rightarrow (ad - bc)^2 = 0\Rightarrow ad = bc \Rightarrow \frac{a}{b} = \frac{c}{d}$.
\item Discriminant is $4(c - a)^2 - 4(b - c)(a - b)$

  $= c^2 + a^2 -2ac - ab + b^2 + ac - bc = a^2 + b^2 + c^2 - ab - bc - ac = \frac{1}{2}[(a - b)^2(b - c)^2(c
  - a)^2]$.

  Clearly the above expression is either greater than zero or equal to zero. Hence, roots are real.
\item Given equation is $x^2 - x + x^2 - (a + 1)x + a + x^2 - ax = 0\Rightarrow 3x^2 - 2(a + 1) + a = 0$.

  Discriminant $D = 4(a + 1)^2 - 12a = a^2 + 2a + 1 - 3a = a^2 - a + 1 = (a - 1)^2 + a$

  which is greater than zero for all $a$ and hence roots are real.
\item Discriminant of the equation $D = b^2 - 4ac$. Given, $a + b + c = 0 \Rightarrow b = -(a + c)$.

  Substituting value of $b, D = (a + c)^2 - 4ac = (a - c)^2$, which is either zero or positive. Hence, roots
  are rational.
\item $D = (c + a - 2b)^2 - 4(b + c - 2a)(a + b - 2c) = c^2 + a^2 + 4b^2 + 2ac - 4bc - 4ab -4ba -4b^2 + 8bc
  - 4ca - 4bc + 8c^2 + 8a^2 + 8ab - 8ca$

  $\Rightarrow 9a^2 + 9c^2 - 18ca = 9(a - c)^2 \geq 0$ which is a perfect square. Hence, roots are rational.
\item Given $r = k + \frac{s}{k} \Rightarrow r^2 = k^2 + \frac{s^2}{k^2} + 2s$

  $\Rightarrow r^2 - 4s = k^2 + \frac{s^2}{k^2} + 2s - 4s\Rightarrow r^ - 4s = k^2 + \frac{s^2}{k^2} - 2s
  = \left(k - \frac{s}{k}\right)^2$

  Clearly, $r^2 - 4s \geq 0$ if $r, s, k$ are rationals which is discriminant of the given equation. Thus, roots will
  be rational provided given condition is met.
\item The given equation is $(x - a)(x - b) + (x - b)(x - c) + (x - c)(x - a) = 0$

  $\Rightarrow 3x^2 -(a + b + b + c + c + a)x + ab + bc + ca = 0\Rightarrow D = 4(a + b + c)^2 - 12(ab + bc + ca)$

  $= 4a^2 + 4b^2 + 4c^2 - 4ab - 4bc - 4ac = 2[(a - b)^2 + (b - c)^2 + (c - a)^2]$.

  This cannot be zero unless $a = b = c$, which is the required condition for the roots to be equal.
\item Given equation is $a^2(b^2 - c^2)x^2 + b^2(c^2 - a^2)x + c^2(a^2 - b^2) = 0$

  $D = b^4(c^2 - a^2)^2 - 4a^2c^2(b^2 - c^2)(a^2 - b^2) = b^4c^4 + b^4a^4 - 2b^4a^2c^2 - 4a^4b^2c^2 +
  4a^2b^4c^2 - 4a^4c^4 + 4a^2b^2c^4$

  $= b^4c^4 + b^4a^4 + 2b^4a^2c^2 - 4a^4b^2c^2 - 4a^4c^4 + 4a^2b^2c^4 = (b^2c^2 + b^2a^2 - 2a^2c^2)^2 \geq
  0$, which is a perfect square, and thus, roots will be rational.
\item $D = 16a^2b^2c^2d^2 - 4(a^4 + b^4)(c^4 + d^4) = 4[4a^2b^2c^22d^2 - a^4c^4 - a^4d^4 - b^4c^4 - b^4d^4]$

  $= -4[(a^2c^2 + b^2d^2)^2(a^2c^2 + b^2d^2)^2]$. Thus, if the roots are real then discriminant has to be
  zero because else it can be only negative and then roots wont remain real.
\item $D = 4q^2 - 4pr = 4(q^2 - pr)$. Since $p, q, r$ are in H. P. $\Rightarrow q = \frac{2pr}{p + r}$

  Substituting for $q$, we get $D = 4\left[\frac{4p^2r^2}{(p + r)^2} - pr\right] = 4\left[\frac{4p^2r^2 -
      p^3r - pr^3 - 2p^2r^2}{(p + r)^2}\right]$

  $= 4\left[\frac{2p^2r^2 - p^3 - r^3}{(p + r)^2}\right] = 4\left[\frac{pr(2pr - p^2 - r^2)}{(p + r)^2}\right]$

  $= 4\left[\frac{-pr(p - r)^2}{(p + r)^2}\right]$. Since $p$ and $r$ have the same sign discriminant is
  bound to be negative and roots will be complex numbers.
\item Discriminant of $bx^2 + (b - c)x + (b - c - a) = 0, D_1 = (b - c)^2 -4b(b - c - a) = b^2 + c^2 -2bc -4b^2 + 4bc + 4ab$

  Discriminant of $ax^2 + 2bc + b = 0, D_2 = 4b^2 - 4ab$. Now, if $D_2 < 0$

  $D_1 = (b + c)^2 - (4b^2 - 4ab) > 0$ and thus roots will be real. However, if $D_1 < 0$ i.e. roots are imaginary then we have

  $D_1 = (b + c)^2 - (4b^2 - 4ab) < 0 \Rightarrow 4b^2 - 4ab > 0 \because [(b + c)^2 > 0]$.

  Then roots of equation $ax^2 + 2bx + b = 0$ will be real.
\item From first equation $x = \sqrt{\frac{1 - by^2}{a}}$ and from second equation $x = \frac{1 - by}{a}$.

  Equating the values obtained $\left(\frac{1 - by}{a}\right)^2 = \frac{1 - by^2}{a}$

  $1 + b^2y^2 - 2by = a - aby^2 \Rightarrow (b^2 + ab)y^2 - 2by + 1 - a = 0$

  Values of $x$ will be equal if values of $y$ are equal i.e. discriminant of above equation is zero.

  $\Rightarrow 42b^2 - 4(b^2 + ab)(1 - a) = 0 \Rightarrow  4b^2 - 4b^2 + 4b^2a - 4ab + 4a^2b = 0$

  $(a^2b + ab^2 - ab) = 0\Rightarrow ab(a + b) = ab\Rightarrow a + b = 1$.
\item Substituting $y = mx + c$ in $x^2 + y^2 = a^2,$ we get $x^2 + m^2x^2 + 2cmx + c^2 - a^2 = 0$

  For roots to be equal, discriminant must be zero. $D = 4c^2m^2 - 4(1 + m^2)(c^2 - a^2) = 0$

  $\Rightarrow c^2m^2 - c^2 + a^2 - c^2m^2 + a^2m^2 = 0\Rightarrow c^2 = a^2(1 + m^2)$.
\item Clearly, roots are $\alpha, \alpha + 1$. Sum of roots $= \alpha + \alpha + 1 = \frac{5a +
  1}{4}\Rightarrow \alpha = \frac{5a - 3}{8}$.

  Product of roots $= \alpha(\alpha + 1) = \frac{5a}{4}$. Substituting value of $\alpha$ from above

  $\left(\frac{5a - 3}{8}\right)^2 + \frac{5a - 3}{8} = \frac{5a}{4}\Rightarrow \frac{25a^2 - 30a + 9 + 40a
    - 24 - 80a}{64} = 0$

  $\Rightarrow 25a^2 - 70a - 15 = 0 \Rightarrow 5a^2 - 14a - 3 = 0\Rightarrow a = 3, -\frac{1}{5}$.

  If $a = 3 \Rightarrow \alpha = \frac{3}{2}$ else if $a = -\frac{1}{5} \Rightarrow \alpha = -\frac{1}{2}$.

  Now it is trivial to calculate the value of $\beta$.
\item Let one of the roots is $\alpha$ then second root is $\frac{1}{\alpha}$.

  Product of roots $= \alpha * \frac{1}{\alpha} = \frac{k}{5} \Rightarrow k = 5$.
\item (a) The equation is :math:$(5 + 4m)x^2 - (4 + 2m)x + 2 - m = 0$

  For roots to be equal discriminant has to be zero.

  $4(2 + m)^2 - 4(5 + 4m)(2 - m) = 0\Rightarrow 4 + 4m + m^2 - 10 - 3m + 4m^2 = 0$

  $5m^2 - m - 6 = 0 \Rightarrow m = 1, -\frac{6}{5}$

  (b) Product of roots $= \frac{2 - m}{5 + 4m} = 2 \Rightarrow 2 - m = 10 + 8m \Rightarrow -\frac{8}{9}$

  (c) Sum of roots $= \frac{4 + 2m}{5 + 4m} = 6 \Rightarrow m = -\frac{13}{11}$
\item Let one root be $\alpha$ then the second root is $n\alpha$.

  Sum of roots $(n + 1)\alpha = -\frac{b}{a} \Rightarrow \alpha = -\frac{b}{(n + 1)a}$

  Product of roots $n\alpha^2 = \frac{c}{a}$

  Substituting value of $\alpha$ from the earlier equation

  $\frac{nb^2}{(n + 1)^2a^2} = \frac{c}{a} \Rightarrow (n + 1)^2 ca = nb^2$.
\item Following from previous problem $n = \frac{3}{4}$ and substituting in final solution

  $\left(\frac{3}{4} + 1\right)^2ca = \frac{3}{4}b^2 \Rightarrow 12b^2 = 49ac$.
\item From earlier problem, we have $a = 4, b = a, c = 3$ and $n = \frac{1}{2}$

  Substituting in the final relation we have, $\frac{9}{4}.3.4 = \frac{1}{2}a^2\Rightarrow a^2 = 54$.

  Discriminant of the second equation, $D = 9 - 4(a^2 - 2a) < 0,$ and thus roots are imaginary.
\item Let $\alpha, \beta$ be the roots of the given equation.

  Sum of roots, $\alpha + \beta = p$ and product of the roots $\alpha\beta = q$

  Given, $\alpha + \beta = m(\alpha - \beta)$. Squaring, $(\alpha + \beta)^2 = m^2(\alpha - \beta)^2$

  $p^2 = m^2(\alpha + \beta)^2 - 4m^2\alpha\beta = m^2p^2 - 4m^2q \Rightarrow p^2(m^2 - 1) = 4m^2q$.
\item Let $\alpha, \beta$ be the roots of the given equation. Sum of roots, $\alpha + \beta = p$ and product
  of the roots $\alpha\beta = q$

  Given, $\alpha - \beta = 1$. Squaring we have,

  $\Rightarrow (\alpha - \beta)^2 = 1 \Rightarrow (\alpha + \beta)^2 - 4\alpha\beta = 1\Rightarrow p^2 - 4q = 1$. Also,
  $[(\alpha - \beta)^2 + 2\alpha\beta]^2 = (1 + 2q)^2$

  $\Rightarrow (\alpha^2 + \beta^2)^2 = \alpha^4 + \beta^4 + 2\alpha^2\beta^2= \alpha^4 + \beta^4 - 2\alpha^2\beta^2 +
  4\alpha^2\beta^2 = (\alpha^2 - \beta^2)^2 + 4q^2$

  $\Rightarrow [(\alpha + \beta)^2(\alpha - \beta)^2] + 4q^2 = p^2 + 4q^2$.
\item The given equation is $a(x - b) + b(x - a) = m(x - a)(x - b)\Rightarrow mx^2 - xm(a + b) - mab - ax +
  ab - bx + ab = 0$

  $\Rightarrow mx^2 - x(m + 1)(a + b) - ab(m - 2) = 0$. If roots are equal in magnitude but opposite in sign
  then sum would be zero.

  $\Rightarrow (m + 1)(a + b) = 0 \Rightarrow m = -1\;\text{or}\;a + b = 0$.
\item Let $\alpha, \beta$ be the roots of the equation.

  Sum of roots, $\alpha + \beta = -\frac{b}{a}$ and product of roots, $\alpha\beta = \frac{c}{a}$.

  Difference of roots, $\alpha - \beta = k$ as given.

  Squaring we get, $(\alpha - \beta)^2 = k^2 \Rightarrow (\alpha + \beta)^2 - 4\alpha\beta = k^2$

  $\frac{b^2}{a^2} - 4\frac{c}{a} = k^2 \Rightarrow b^2 - 4ac = k^2a^2$.
\item Let $\alpha$ be one of the roots of the equation $ax^2 + bx + c = 0$. Clearly, $\alpha^2$ will be the other root.

  Sum of roots, $\alpha + \alpha^2 = -\frac{b}{a}$ and product of the roots $\alpha^3 = \frac{c}{a}$. Cubing sum of roots,

  $\frac{b^3}{a^3} = -\alpha^3(\alpha + 1)^3 = -\frac{c}{a}(\alpha^3 + 3\alpha(\alpha + 1) + 1)$

  $\frac{b^3}{a^3} = -\frac{c}{a}\left(\frac{c}{a} - \frac{3b}{a} + 1\right)$

  Simplifying we get the desired relationship.
\item Let $\alpha$ be one of the roots of the equation $ax^2 + bx + c = 0$. Clearly, $\alpha^2$ will be the other root.

  Sum of roots, $\alpha + \alpha^2 = -p$ and product of roots $\alpha^3 = 1$.

  Thus, $\alpha$ is cube root of unity. If $\alpha = -1$ then $p = -2$

  else if it is one of the complex numbers then we know that $1 + \omega + \omega^2 = 0$ which makes $p =
  1$.
\item Let $\alpha$ be one of the roots of the equation $ax^2 + bx + c = 0$. Clearly, $\alpha^2$ will be the other root.

  Sum of roots, $\alpha + \alpha^2 = -p$ and product of roots $\alpha^3 = q$

  $p^3 = -\alpha^3(\alpha + 1)^3 = -q(\alpha^3 + 3\alpha(\alpha + 1) + 1) = -q(q - 3p + 1)$

  $\Rightarrow p^3 - q(3p - 1) + q^2 = 0$.
\item The solution is given below:
  \startitemize[i]
  \item $\alpha + \beta = -\frac{3}{2}$ and $\alpha\beta = \frac{4}{2} = 2$.

    $\Rightarrow \alpha^2 + \beta^2 = (\alpha + \beta)^2 - 2\alpha\beta = \frac{9}{4} - 4 = -\frac{7}{4}$.
  \item $\frac{\alpha}{\beta} + \frac{\beta}{\alpha} = \frac{\alpha^2 + \beta^2}{\alpha\beta}$

    Substituting for numerator from previous part,

    $\Rightarrow \frac{\alpha}{\beta} + \frac{\beta}{\alpha} = -\frac{7}{8}$.
  \stopitemize
\item Sum of roots, $\alpha + \beta = -\frac{b}{a}$ and product of roots, $\alpha\beta = \frac{c}{a}$

  $\frac{\alpha^2}{\beta} + \frac{\beta^2}{\alpha} = \frac{\alpha^3 + \beta^3}{\alpha\beta} =
  \frac{(\alpha + \beta)^3 - 3\alpha\beta(\alpha + \beta)}{\alpha\beta} = \frac{-\frac{b^3}{c^3} +
    \frac{3c}{a}\frac{b}{a}}{\frac{c}{a}} = \frac{3abc - b^3}{a^2c}$.
\item Sum of roots, $\alpha + \beta = -\frac{b}{a}$ and product of roots, $\alpha\beta = \frac{b}{a}$

  Given expression is, $\sqrt{\frac{\alpha}{\beta}} + \sqrt{\frac{\beta}{\alpha}} + \sqrt{\frac{b}{a}} =
  \frac{\alpha + \beta}{\sqrt{\alpha\beta}} + \sqrt{\frac{b}{a}} =
  \frac{-\frac{b}{a}}{\sqrt{\frac{b}{a}}} + \sqrt{\frac{b}{a}} = 0$.
\item Product of the roots of the first equation is $ b^2$ and sum of roots of the second equation is $2b$.

  Geometric mean of the roots of the first equation $= \text{square root of product of roots} = \sqrt{b^2} =
  b$.

  Arithmetic mean of the roots of the second equation $= \text{half of sum of roots} = \frac{2b}{2} = b$ and
  thus both are equal.
\item Let $\alpha, \beta$ be the roots of the equation.

  Sum of roots, $\alpha + \beta = -\frac{q}{p}$ and product of roots, $\alpha\beta = \frac{r}{p}$.

  Given, sum of roots is equal to sum of square of roots. $\therefore \alpha + \beta = \alpha^2 + \beta^2$

  $-\frac{q}{p} = (\alpha + \beta)^2 - 2\alpha\beta = \frac{q^2}{p^2} - \frac{2r}{p}\Rightarrow 2pr = pq +
  q^2$.
\item Let $\alpha, \beta$ be the roots of the equation. Sum of roots, $\alpha + \beta = p$ and product of
  roots, $\alpha\beta = q$.

  $\frac{\alpha^2}{\beta^2} + \frac{\beta^2}{\alpha^2} = \frac{\alpha^4 + \beta^4}{(\alpha\beta)^2} =
  \frac{(\alpha^2 + \beta^2)^2 - 2\alpha^2\beta^2}{\alpha^2\beta^2} = \frac{[(\alpha + \beta)^2 -
      2\alpha\beta]^2}{\alpha^2\beta^2} - 2$

  $= \frac{(p^2 - 2q)^2}{q^2} - 2 = \frac{p^4}{q^2} - \frac{4p^2}{q} + 2$.
\item Let $\alpha, \beta$ be the roots of the equation. Sum of roots, $\alpha + \beta = -\frac{b}{a}$ and
  product of roots, $\alpha\beta = \frac{c}{a}$

  $\Rightarrow \frac{1}{(a\alpha + b)^2} + \frac{1}{(a\beta + b)^2} = \frac{(a\alpha + b)^2 + (a\beta +
    b)^2}{[(a\alpha + b)(a\beta + b)]^2}$

  $\Rightarrow \frac{a(\alpha^2 + \beta^2) + 2ab(\alpha + \beta) + 2b^2}{(a^2\alpha\beta + 2ab(\alpha +
    \beta) + b^2)^2}$

  Substituting for sum of roots, product of roots and $\alpha^2 + \beta^2 = (\alpha + \beta)^2 -
  2\alpha\beta$ and simplifying

  $= \frac{b^2 - 2ac}{c^2a^2}$.
\item Rewriting the equation we have $\lambda x^2 + x(1 - \lambda) + 5 = 0$.

  Since $\alpha$ and $\beta$ are the roots therefore, we have $\alpha + \beta = \frac{\lambda - 1}{\lambda}$
  and $\alpha\beta = \frac{5}{\lambda}$.

  Given, $\frac{\alpha}{\beta} + \frac{\beta}{\alpha} = \frac{4}{5}$

  $\frac{\alpha^2 + \beta^2}{\alpha\beta} = \frac{(\alpha + \beta)^2 - 2\alpha\beta}{\alpha\beta}\Rightarrow
  \frac{(\lambda - 1)^2 - 10\lambda}{5\lambda} = \frac{4}{5}$

  $\Rightarrow (\lambda - 1)^2 - 10\lambda = 4\lambda \Rightarrow \lambda^2 - 16\lambda + 1 = 0\therefore
  \lambda_1 + \lambda_2 = 16$ and $\lambda_1\lambda_2 = 1$.
  \startitemize[i]
  \item $\frac{\lambda_1}{\lambda_2} + \frac{\lambda_2}{\lambda_1} = \frac{(\lambda_1 + \lambda_2)^2 -
    2\lambda_1\lambda_2}{\lambda_1\lambda_2}$

    Substituting the values for sum and product we have, result as $254$.
  \item $\frac{\lambda_1^2}{\lambda_2} + \frac{\lambda_2^2}{\lambda_1} = \frac{\lambda_1^3 +
    \lambda_2^3}{\lambda_1\lambda_2} = \frac{(\lambda_1 + \lambda_2)^3 - 3\lambda_1\lambda_2(\lambda_1 +
    \lambda_2)}{\lambda_1\lambda_2}$

    $= 4048$.
  \stopitemize
\item For the first equation $\alpha + \beta = -p$ and $\alpha\beta = q$ and similarly for the second
  $\gamma + \delta = -r$ and $\gamma\delta = s$.
  \startitemize[i]
  \item $(\alpha + \gamma)(\alpha + \delta)(\beta + \gamma)(\beta + \delta)$
    $= [\alpha^2 + \alpha(\gamma + \delta) + \gamma\delta][\beta^2 + \beta(\gamma + \delta) + \gamma\delta]$

    $= (\alpha^2 - r\alpha + s)(\beta^2 - r\beta + s)$
    $= (\alpha^2\beta^2 - r\alpha\beta^2 + s\beta^2 - r\alpha^2\beta - r^2\alpha\beta - rs\beta + s\alpha^2 - rs\alpha +
    s^2)$

    $= q^2 - r\alpha\beta(\alpha + \beta) + s(\alpha^2 + \beta^2) + r^2p - rs(\alpha + \beta) + s^2$
    $= q^2 + prs + s(p^2 - 2q) + r^2p - rsq + s^2$
  \item $(\alpha - \gamma)(\beta - \delta) + (\beta - \gamma)(\alpha - \delta) = \alpha\beta -
    \alpha\delta - \beta\gamma + \gamma\delta + \alpha\beta - \beta\delta -\alpha\gamma + \gamma\delta$

    $= 2\alpha\beta + 2\gamma\delta - (\alpha + \beta)(\gamma + \delta) = 2q + 2s - pr$.
  \item $(\alpha - \gamma)^2 + (\beta - \delta)^2 + (\beta - \gamma)^2 + (\alpha - \delta)^2$

    $= 2(\alpha^2 + \beta^2 + \delta^2 + \gamma^2) - 2(\alpha + \beta)(\gamma + \delta) = 2[(\alpha +
    \beta)^2 - 2\alpha\beta + (\gamma + \delta)^2 - 2\gamma\delta] - 2(\alpha + \beta)(\gamma + \delta)$

    $= 2[p^2 + r^2 - 2q - 2s] - 2pr$.
  \stopitemize
\item $\alpha + \beta = p$ and $\alpha\beta = q$

  Now, R.H.S. $= (\alpha + \beta)(\alpha^n + \beta^n) - \alpha\beta(\alpha^{n - 1} + \beta^{n - 1})$ $=
  \alpha^{n + 1} + \beta^{n + 1} =$ L.H.S.
\item $\alpha + \beta = \gamma + \delta = -p, \alpha\beta = -q$ and $\gamma\delta = r$ Also, since $\alpha,
  \beta$ are roots of $x^2 + px + q = 0, \therefore \alpha^2 + p\alpha + q = 0$ and $\beta^2 + p\beta + q =
  0$.

  Now, $(\alpha - \gamma)(\alpha - \delta) = \alpha^2 - \alpha(\gamma + \delta) + \gamma\delta = \alpha^2 +
  p\alpha - r = -q - r = -(q + r)$, and similarly, $(\beta - \gamma)(\beta - \delta) = -(q + r)$.
\item Clearly, $\alpha + \beta = 2p, \alpha\beta = q$ and $\gamma + \delta = 2r, \gamma\delta = s$
  \startitemize[i]
  \item $\frac{\alpha}{\beta} = \frac{\gamma}{\delta}$. By componendo and dividendo

    $\Rightarrow \frac{\alpha + \beta}{\alpha - \beta} = \frac{\gamma + \delta}{\gamma - \delta}$

    Squaring, $\left(\frac{\alpha + \beta}{\alpha - \beta}\right)^2 = \left(\frac{\gamma + \delta}{\gamma
      - \delta}\right)^2$

    $1 - \frac{4\alpha\beta}{(\alpha + \beta)^2} = 1 - \frac{4\gamma\delta}{(\gamma + \delta)^2}\Rightarrow
    \frac{q}{p^2} = \frac{s}{r^2}$.
  \item Since $\alpha, \beta, \gamma, \delta$ are in G. P. Hence, $\frac{\alpha}{\beta} = \frac{\gamma}{\delta}$ and
    then we can proceed like previous part.
  \item Since $\alpha, \beta, \gamma, \delta$ are in A. P. Hence, $\alpha - \beta = \gamma - \delta$

    $\Rightarrow (\alpha + \beta)^2 - 4\alpha\beta = (\gamma + \delta)^2 - 4\gamma\delta\Rightarrow 4p^2 -
    4q = 4r^2 - 4s \Rightarrow s - q = r^2 - p^2$.
  \stopitemize
\item Clearly, $\alpha + \beta = -\frac{2b}{a}$ and $\alpha\beta = \frac{c}{a}$ for $ax^2 + 2bx + c = 0$ and
  $\alpha + \beta + 2k = -\frac{2B}{A}$ and $(\alpha + k)(\beta + k) = \frac{C}{A}$ for $AX^2 + 2Bx + C = 0$.

  Given expression can be rewritten as $\frac{b^2}{a^2} - \frac{c}{a} = \frac{B^2}{A^2} - \frac{C}{A}$

  $\frac{(\alpha + \beta)^2}{4} - \alpha\beta = \frac{(\alpha + \beta + 2k)^2}{4} - (\alpha + k)(\beta +
  k)\Rightarrow (\alpha - \beta)^2 = (\alpha + k - \beta - k)^2$, which is true.
\item Proceeding like previous problem, we have to prove that $\frac{b^2 - 4ac}{B^2 - 4AC} = \frac{a^2}{A^2}
  \Rightarrow \frac{b^2}{a^2} - \frac{4c}{a} = \frac{B^2}{A^2} - \frac{4C}{A} \Rightarrow (\alpha + \beta)^2
  - 4\alpha\beta = (\alpha + \beta + 2k)^2 - 4(\alpha + k)(\beta + k)$

  $\Rightarrow (\alpha - \beta)^2 = (\alpha + k - \beta - k)^2$, which is true.
\item Let $\alpha, \beta$ be the roots of $x^2 + 2px + q = 0$ and $\gamma, \delta$ be the roots of
  $x^2 + 2qx + p = 0$

  $\alpha + \beta = -2p$ and $\gamma + \delta = -2q.$ Also, $\alpha\beta = q$ and
  $\gamma\delta = p$

  Given that roots differ by a constant term say $k$. $\therefore \alpha + k = \gamma$ and $\beta +
  k = \delta$

  Thus, $\alpha + \beta + 2k = -2q \Rightarrow -2p + 2k = -2q \Rightarrow k = p - q\Rightarrow
  \gamma\delta = \alpha\beta + (\alpha + \beta)k + k^2 = p$

  Also, $q - 2pk + k^2 = p \Rightarrow -2p + k = 1 \Rightarrow p + q + 1 = 0$.
\item Clearly, $\alpha + \beta = -\frac{b}{a}$ and $\alpha\beta = \frac{c}{a}$.
  \startitemize[i]
  \item Sum of these roots is $\frac{\alpha}{\beta} + \frac{\beta}{\alpha} = \frac{\alpha^2 +
    \beta^2}{\alpha\beta} = \frac{b^2 - 2ac}{ac}$

    Product of these roots is $1$. Therefore, such an equation is $x^2 -\frac{b^2 - 2ac}{ac}x + 1 = 0$.
  \item Sum of these roots is $\frac{\alpha^3 + \beta^3}{\alpha\beta} = \frac{(\alpha + \beta)^3 -
    3\alpha\beta(\alpha + \beta)}{\alpha\beta} = \frac{3abc - b^3}{a^2c}$.

    Product of these roots is $\alpha\beta = \frac{c}{a}$. Therefore, an equation whose roots were these is
    $x^2 - \frac{3abc - b^3}{a^2c}x + \frac{c}{a} = 0$.
  \item Sum of these roots is $(\alpha + \beta)^2 + (\alpha - \beta)^2 = 2(\alpha + \beta)^2 -
    4\alpha\beta = \frac{2b^2}{a^2} - \frac{4c}{a}$.

    Product of these roots is $(\alpha + \beta)^2(\alpha - \beta)^2 = (\alpha + \beta)^2[(\alpha +
      \beta)^2 - 4\alpha\beta] = \frac{b^2}{a^2}\left(\frac{b^2}{a^2} - \frac{4c}{a}\right)$.

    So the equation is $x^2 -\left(\frac{2b^2}{a^2} - \frac{4c}{a}\right)x +
    \frac{b^2}{a^2}\left(\frac{b^2}{a^2} - \frac{4c}{a}\right) = 0$.
  \item Sum of these roots is $\frac{1 - \alpha}{1 + \alpha} + \frac{1 - \beta}{1 + \beta} = \frac{1 +
    \beta - \alpha -\alpha\beta + 1 + \alpha - \beta -\alpha\beta}{1 + (\alpha + \beta) + \alpha\beta}$

    $= \frac{2 - 2\alpha\beta}{1 + (\alpha + \beta) + \alpha\beta} = \frac{2\left(1 +
    \frac{b}{a}\right)}{1 - \frac{b}{a} + \frac{c}{a}} = \frac{2(a + b)}{a - b + c}$.

    Product of these roots is $\frac{1 - \alpha}{1 + \alpha}.\frac{1 - \beta}{1 + \beta} = \frac{1
      -(\alpha + \beta) + \alpha\beta}{1 + (\alpha + \beta) + \alpha\beta} = \frac{1 + \frac{b}{a} +
      \frac{c}{a}}{1 - \frac{b}{a} + \frac{c}{a}} = \frac{a + b + c}{a - b + c}$.

    Therefore, the equation is $(a - b + x)x^2 - 2(a + b)x + (a + b + c) = 0$.
  \item Sum of these roots is $\frac{1}{(\alpha + \beta)^2} + (\alpha - \beta)^2 = \frac{a^2}{b^2} +
    [(\alpha + \beta)^2 - 4\alpha\beta] = \frac{a^2}{b^2} + \left[\frac{b^2}{a^2} - \frac{4c}{a}\right]$.

    Product of these roots is $\frac{1}{(\alpha + \beta)^2}.(\alpha - \beta)^2 = \frac{1}{(\alpha +
      \beta)^2}.[(\alpha + \beta)^2 - 4\alpha\beta] = \frac{a^2}{b^2}\left[\frac{b^2}{a^2} -
      \frac{4c}{a}\right] = \frac{b^2 - 4ac}{b^2}$.

    Now it is trivial to deduce the equation.
  \stopitemize
\item Let the roots of the equation $ax^2 + bx + c = 0$ are $p$ and $q$, then $p + q = -\frac{b}{a}$ and $pq
  = \frac{c}{a}$.

  (a) The reciprocal of roots are $\frac{1}{p}$ and $\frac{1}{q}$. Sum of these is $\frac{p + q}{pq} =
  -\frac{b}{c}$ and product is $\frac{1}{pq} = \frac{a}{c}$. Therefore, the equation is $cx^2 + bx + a = 0$.

  (b) Let one of the roots is $p$ then the other will be $-p$. Sum will be $0$ and product will be
  $-\frac{c}{a}$. Therefore, the equation is $ax^2 - c = 0$.
\item Clearly, $\alpha + \beta = -p$ and $\alpha\beta = q$.

  (a) $\alpha^4 + \beta^4 = (\alpha^2 + \beta^2) - 2\alpha^2\beta^2 = [(\alpha + \beta)^2 - 2\alpha\beta]^2
  - 2\alpha^2\beta^2 = [p^2 - 2q]^2 - 2q^2 = p^4 - 4p^2q + 2q^2$.

  (b) $\alpha^{-4} + \beta^{-4} = \frac{\alpha^4 + \beta^4}{\alpha^4\beta^4} = \frac{p^4 - 4p^2q +
    2q^2}{q^4}$.
\item Clearly, $\alpha + \beta = p$ and $\alpha\beta = q$.
  \startitemize[i]
  \item Sum of these roots is $\frac{q}{p - \alpha} + \frac{q}{p - \beta} = \frac{2pq - q(\alpha +
    \beta)}{p^2 - p(\alpha + \beta) + \alpha\beta} = \frac{pq}{q} = p$.

    Product of these roots is $\frac{q}{p - \alpha}.\frac{q}{p - \beta} = \frac{q^2}{q} = q$.

    Thus the eqation of these new roots remain same i.e. $x^2 - px + q = 0$.
  \item Sum of these roots is $\alpha + \beta + \frac{1}{\alpha} + \frac{1}{\beta} = \alpha + \beta +
    \frac{\alpha + \beta}{\alpha\beta} = p + \frac{p}{q} = \frac{p(1 + q)}{q}$.

    Product of these roots is $\left(\alpha + \frac{1}{\beta}\right)\left(\beta + \frac{1}{\alpha}\right) =
    \alpha\beta + \frac{\alpha}{\beta} + \frac{\beta}{\alpha} + \frac{1}{\alpha\beta} = q + \frac{1}{q} +
    \frac{\alpha^2 + \beta^2}{\alpha\beta} = \frac{q^2 + 1}{q} + \frac{p^2 - 2q}{q}$.

    Now deducing the equation is trivial.
  \stopitemize
\item Because $5 + 3i$ is a complex root the other root will be complex conjugate i.e. $5 - 3i$. Thus,
  equation having these complex roots will be $x^2 - 10x + 34 = 0$.
\item Because $3 + 4i$ is a complex root the other root will be complex conjugate i.e. $3 - 4i$. Thus,
  equation having these complex roots will be $x^2 - 6x + 25 = 0$.
\item Roots are given by $\frac{-2\pm\sqrt{4 + 16}}{6} = \frac{-1\pm\sqrt{5}}{4}$. Now $\frac{\sqrt{5} -
  1}{4} = \cos72^\circ$ and $-\frac{\sqrt{5} + 1}{4} = -\cos36^\circ = \cos216^\circ = \cos(3.72^\circ)$

  Now, $\cos3x = 4\cos^3x - 3\cos x$, therefore if one root is $\alpha$ then the other would be $4\alpha^3 -
  3\alpha$.
\item Clearly, by observation $\alpha, \beta$ are roots of the eqation $x^2 - 5x + 3 = 0$. $\Rightarrow
  \alpha + \beta = 5$ and $\alpha\beta = 3$.

  Now, $\frac{\alpha}{\beta} + \frac{\beta}{\alpha} = \frac{\alpha^2 + \beta^2}{\alpha\beta} =
  \frac{5(\alpha + \beta) - 6}{3} = \frac{19}{3}$.
\item Correct value of $p = -11$. $q$ is $4\times 6 = 24$. Hence, the correct equation is $x^2 - 11x + 24 =
  0$. Hence roots are $8, 3$.
\item Correct value of $q$ is $2$. $p$ is $-(6 - 1) = 5$. Hence, the correct equation is $x^2 - 5x + 2 = 0$.
\item From first student the correct value of $q = 6\times2 = 12$. From second student the correct value of
  $p = -(2 + -9) = 7$. Hence the correct equation is $x^2 + 7x + 12 = 0$ giving us $3, 4$ as correct roots.
\item We have $\alpha + \beta = -p, \alpha\beta = q, \alpha_1 + \beta_1 = p, \alpha_1\beta_1 = q$.

  Now, $\frac{1}{\alpha_1\beta} + \frac{1}{\alpha\beta_1} + \frac{\alpha\alpha_1}+ {\beta\beta_1} =
  \frac{(\alpha + \beta)(\alpha_1 + \beta_1)}{\alpha\beta\alpha_1\beta_1} = \frac{pq}{qp} = 1$

  and $\left(\frac{1}{\alpha_1\beta} + \frac{1}{\alpha\beta_1}\right)\left(\frac{1}{\alpha\alpha_1} +
  \frac{1}{\beta\beta_1}\right) = \frac{1}{\alpha_1^2\alpha\beta} + \frac{1}{\alpha_1\beta_1\beta^2} +
  \frac{1}{\alpha_1\beta_1\alpha^2} + \frac{1}{\alpha\beta\beta_1^2}$

  $= \frac{1}{\alpha\beta}\left[\frac{1}{\alpha_1^2} + \frac{1}{\beta_1^2}\right] +
  \frac{1}{\alpha_1\beta_1}\left[\frac{1}{\alpha^2} + \frac{1}{\beta^2}\right] =
  \frac{1}{q}\left[\frac{\alpha_1^1 + \beta_1^2}{\alpha_1^2\beta_1^2}\right] +
  \frac{1}{p}\left[\frac{\alpha^2 + \beta^2}{\alpha^2\beta^2}\right]$

  $= \frac{p^3 + q^3 - pq}{p^2q^2}$. Therefore, the equation with these as roots is

  $x^2 - x + \frac{p^3 + q^3 - pq}{p^2q^2} = 0$.
\item We know that complex roots always appear in pair and as $2 + \sqrt{3}i$ is a complex root the other
  root will be its complex conjugate i.e. $2 - \sqrt{3}i$. Hence, $p = -4$ and $q = 13$ makring the equation
  $x^2 - 4x + 13 = 0$.
\item $\frac{1}{2 + \sqrt{3}} = 2 - \sqrt{3}$ which is an irrational root and the other root will be its
  conjugate i.e. $2 + \sqrt{3}$ hence the equation will be $x^2 - 4x + 1 = 0$
\item Since $\alpha, \beta$ are roots of the equation $x^2 - px + q = 0$, $\alpha + \beta = p$ and
  $\alpha\beta = q$.

  Let us assume that $\alpha + \frac{1}{\beta}$ is a root of $qx^2 - p(1 + q)x + (1 + q)^2 = 0$ then
  it must satisfy the equation. Substituting the values we have

  $\alpha\beta\frac{(\alpha\beta + 1)^2}{\beta^2} - \frac{(\alpha + \beta)(1 + \alpha\beta)(\alpha\beta +
    1)}{\beta} + (1 + \alpha\beta)^2 = 0$

  $(\alpha\beta + 1)^2[\alpha\beta - (\alpha + \beta)\beta - \beta^2] = 0$

  $\because$ L.H.S. = R.H.S. it is proven that $\alpha + \frac{1}{\beta}$ is a root of the given
  equation.
\item One of the given equations is $2x^2 + 3x - 2 = 0 \Rightarrow (2x - 1)(x + 2) = 0$ so the roots are $x
  = \frac{1}{2}, -2$. Putting these two in the equation $3x^2 + 4mx + 2 = 0$ we obtain two values
  $-\frac{7}{4}, -\frac{11}{8}$ for $m$.
\item Let $p$ be the common root then it must satisfy both the equations i.e. $p^2 - 11p + a = 0$ and $p^2 -
  14p + 2a = 0$. Equating $a$ from both equations $11p - p^2 = \frac{14p - p^2}{2} \Rightarrow p^2 - 8p = 0
  \Rightarrow p = 0, 8\Rightarrow a = 0, 24$.
\item The condition for having common roots is obtained by cross-multiplication:

  $(ba - c^2)(ca - b^2) = (a^2 - bc)^2\Rightarrow a^2bc - ab^3 - ac^3 + b^2c^2 = a^4 - 2a^2bc +
  b^2c^2\Rightarrow 3a^2bc - ab^3 -ac^3 - a^4 = 0$

  $a(3abc - b^3 - c^3 - a^3) = 0\because a\ne = 0 \Rightarrow a^3 + b^3 + c^3 - 3abc = 0\Rightarrow (a + b +
  c)(a^2 + b^2 + c^2 - ab - bc - ca) = 0$

  $\Rightarrow a + b + c = 0$ or $a = b = c$.
\item Proceeding as in last example, condition for common root is

  $(10m - 189)(9 - 10) = (21 - m)^2\Rightarrow 189 - 10m = 441 - 42m + m^2 \Rightarrow m^2 - 32m + 252 = 0
  \Rightarrow m = 18, 14$.

  Roots of $x^2 + 10x + 21 = 0$ are $-3, -7$. When $m = 18$ roots of $x^2 + 9x + 18 = 0$ are $-3, -6$.

  In that case equation formed with $-7$ and $-6$ is $x^2 + 13x + 42 = 0$ When $m= 14$ roots of $x^2 + 9x
  + 14 = 0$ are $-2, -7$.

  In that case equation formed with $-3$ and $-2$ are $x^2 + 5x + 6 = 0$.
\item Following condition for common roots, we have

  $(-3 + 120)(10 + 3) = (3 + 36)^2\Rightarrow 117 * 13 = 39^2$ which is true and thus equations have a
  common root.

  Roots of $x^2 - x - 12 = 0$ are $4, -3$ and roots of $3x^2 + 10x + 3 = 0$ are $-3, -\frac{1}{3}$ and thus
  common root is $-3$.
\item Condition for common root is given below:

  $(p - q)(3q - 2p) = (3 - 2)^2\Rightarrow (2p - 3q)(p - q) + 1 = 0\Rightarrow 2p^2 + 3q^2 - 5pq + 1 = 0$.
\item The condition for common root is $(b - c)(a - b) =  (a - c)^2$

  $\Rightarrow ab - ac - b^2 + bc = a^2 + c^2 - 2ac\Rightarrow a^2 + b^2 + c^2 - ab - ac - bc = 0$

  $\Rightarrow \frac{1}{2}(a - b)^2(b - c)^2(c - a)^2 = 0\Rightarrow a = b = c$.
\item Let $\alpha$ be the common root then

  $\frac{\alpha^2}{pq_1 - p_1q} = \frac{\alpha}{q - q_1} = \frac{1}{p_1 - p}$. Clearly, the root is either
  $\frac{pq_1 - p_1q}{q - q_1}$ or $\frac{q - q_1}{p_1 - p}$.
\item Condition for having common root is:

  $(-4b + 3c)(-6a - 2b) = (4a - 2c)^2$. Solving this gives us required equation.
\item Condition for having a common root is:

  $[(r - p)(q - r) - (p - q)^2][(p - q)(q - r) - (r - p)^2] = [(q - r)^2 - (p - q)(r - p)]^2$, which is an
  equality and hence the equations have a common root.
\item Let $\alpha$ be a common root then

  $\frac{\alpha^2}{ab^2 - ac^2} = \frac{1}{b - c} = \frac{1}{ac - ab}\Rightarrow \alpha = -a(b + c)$ or
  $\alpha = -\frac{1}{a}$.

  Let $\alpha, \beta$ be roots of first and $\alpha, \gamma$ be roots of the second equation. Then,
  $\alpha + \beta = -ab$ and $\alpha\beta = c$ also, $\alpha + \gamma = -ac$ and
  $\alpha\gamma = b$

  $\Rightarrow 2\alpha + \beta + \gamma = -a(b + c)$ and $\alpha^2\beta\gamma = bc$

  Equation formed by $\beta$ and $\gamma$ would be $x^2 - (\beta + \gamma)x + \beta\gamma = 0$.

  For either values of $\alpha$ equation is $x^2 - a(b + c)x + a^2bc = 0$.
\item Let $\alpha$ is a common root then $x^2 - px + q = 0$ and $x^2 - ax + b = 0$. Let $\beta$ be the
  second root of the first equationa then $\frac{1}{\beta}$ will be the second root of the second equation.

  Clearly, $\alpha + \beta = p, \alpha\beta = q, \alpha + \frac{1}{\beta} = a, \frac{\alpha}{\beta} = b$.

  $\therefore (q - b)^2 = (\alpha\beta - \frac{\alpha}{\beta})^2$,

  $bq(p - a)^2 = \frac{\alpha}{\beta}(\alpha\beta)(\beta - \frac{1}{\beta})^2 = (\alpha\beta -
  \frac{\alpha}{\beta})^2$. Hence, proved.
\item It is a quadratic equation but satisfied by three values of $x = 1, 2, 3$ therefore it is an
  identity.
\item It is a quadratic equation but satisfied by three values of $x = a, b, c$ therefore it is an
  identity.
\item Let $x^5 = y$ then equation becomes $3y^2 - 2y - 8 = 0$.

  Since it is satisfied by two distinct values and it is a quadratic equation therefore it is an
  equation.
\item $\frac{(x + 2)^2 - (x - 2)^2}{x^2 - 4} = \frac{5}{6}$

  $\Rightarrow \frac{8x}{x^2 - 4} = \frac{5}{6}\Rightarrow 5x^2 - 20 - 48x = 0\Rightarrow x = 10 ,
  -\frac{2}{5}$.
\item Let $x = y^2\Rightarrow \frac{2y + 1}{3 - y} = \frac{11 - 3y}{5y - 9}$

  $\Rightarrow 10y^2 - 13y - 9 = 33 - 20y + 3y^2\Rightarrow 7y^2 + 7y - 42 = 0\Rightarrow y = 2, -3$

  $\Rightarrow x = 4, 9$ but $x = 9$ does not apply to the equation and is an impossible solution.
\item $(x + 1)(x - 3)(x + 2)(x - 4) = 336\Rightarrow (x^2 - 2x - 3)(x^2 - 2x - 8) = 336$

  Let $x^2 - 2x - 3 = y\Rightarrow y(y - 5) = 336\Rightarrow ey^2 - 5y - 336 = 0\Rightarrow y = 21, -16$

  $\Rightarrow x = -4, 6, 1 \pm 2\sqrt{3}i$.
\item Squaring $x + 1 + 2x - 5 + 2\sqrt{(x + 1)(2x - 5)} = 9 \Rightarrow 2\sqrt{(x + 1)(2x - 5)} = 13 - 3x$

  Squaring again $4(x + 1)(2x - 5) = 9x^2 - 78x + 169 \Rightarrow x^2 - 66x + 189 = 0\Rightarrow x = 3, 63$.

  We see that $x = 63$ does not satisfy the equation hence the only solution is $x = 3$.
\item We have $2^{2x} + 2^{x + 2} - 32 = 0\Rightarrow (2^x - 4)(2^x + 8) = 0$. However, $2^x \neq
  8\Rightarrow 2^x = 4 \Rightarrow x = 2$.
\item Let the speed be $x$ km/hour. Then, from the statement $\frac{800}{x} = \frac{800}{x + 40} + \frac{2}{3}$

  Solving we get $x = 200$ km/hour.
\item Let width be $w$ meter. Thus, $(w + 8)(w - 2) = 119\Rightarrow w^2 + 6w - 135 = 0\Rightarrow w = 9,
  -15$ but width cannot be negative. Length is $11$ m.
\item Equivalent equation is $-x^2 + 3x + 4 = 0$ and roots are $-1, 4$.

  Since coefficient of $x^2$ is -ve the expression will be +ve if $x$ lies between the root.

  Therefore, for $-x^2 + 3x + 4 > 0$ the range is $]-1, 4[$.
\item $5x - 1 < (x + 1)^2 \Rightarrow x^2 - 3x + 2 > 0$.

  Roots of equivalent equation $x^2 - 3x + 2 = 0$ are $x = 2, 1$.

  Since coefficient of $x^2$ is positive, $x$ must lie outside the range of $[1, 2]$ for the expression
  to be positive.

  Now considering, $(x + 1)^2 < 7x - 3\Rightarrow x^2 - 5x + 4 < 0$

  Roots of the equivalent equation $x^2 - 5x + 4 = 0$ are $x = 1, 4$ and for expression to be negative $x$
  must lie inside the open interval $]1, 4[$.

  Therefore, the only integral value satisfying the original expression is $3$.
\item $\frac{8x^2 + 16x - 51}{(2x - 3)(x + 4)} > 3\Rightarrow \frac{2x^2 + x - 15}{2x^2 + 5x - 12} > 0$

  $2x^2 + x - 15 = 0$ has roots $x = -3 , \frac{5}{2}\Rightarrow 2x^2 + 5x - 12 = 0$ has roots $x = -4,
  \frac{3}{2}$

  Thus, the inequality will hold true for $x < -4$ and $-3 < x < \frac{3}{2}$ and $x > \frac{5}{2}$.
\item Let $y = \frac{x^2 - 3x + 4}{x^2 + 3x + 4}\Rightarrow (y - 1)x^2 + 3(y + 1)x + 4(y - 1) = 0$

  Since $x$ is real, the discriminant will be greater that $0 \Rightarrow 9(y + 1)^2 - 16(y - 1)^2 \ge 0$

  $-7y^2 + 50y - 7 \ge 0$. The roots are $7$ and $\frac{1}{7}$

  Since coefficient of $y^2$ is negative, for the expression to be positive $y$ has to lie between the
  open interval formed by its roots i.e. $]\frac{1}{7}, 7[$
\item Let $y = \frac{x^2 + 34x - 71}{x^2 + 2x - 7}\Rightarrow (y - 1)x^2 + 2(y - 17)x + (71 - y) = 0$

  Since $x$ is real, the discriminant will be greater that $0\Rightarrow 4(y - 17)^2 - 4(y - 1)(71 - 7y) \ge
  0$

  $\Rightarrow y^2 - 14y + 45 \ge 0$. Its roots are $5$ and $9$

  Since coefficient of $y^2$ is positive, therefore for the expression to be positive $y$ has to lie outside
  the open interval formed by its roots. Thus, the expression has no value between $5$ and $9$.
\item Let $y = \frac{4x^2 + 36x + 9}{12x^2 + 8x + 1}\Rightarrow 4(3y - 1)x^2 + 4(2y - 9)x + y - 9 = 0$.

  Since $x$ is real, the discriminant will be greater that $0\Rightarrow 16(2y - 9)^2 - 16(3y - 1)(y - 1)
  \ge 0\Rightarrow y^2 - 8y + 72 \ge 0$

  Corresponding equation is $y^2 - 8y + 72 = 0\Rightarrow D = 64 - 288 = -224 < 0$

  Since coefficient of $y^2$ is positive and discriminant is less than $0$ therefore $y^2 - 8y + 72 \ge 0$
  holds true for all value of $y$. Therefore, the expression can take any value.
\item Let $y = \frac{(x - a)(x - c)}{x -b}\Rightarrow x^2 - (a + c + y)x + ac + yb = 0$

  Since $x$ is real, the discriminant will be greater that $0$

  $\Rightarrow (a + c + y)^2 - 4(ac + yb) \ge 0\Rightarrow y^2 + 2(a + c - 2b)y + (a - c)^2 \ge 0$.

  Corresponding equation is $y^2 + 2(a + c - 2b)y + (a - c)^2 = 0$. Discriminant of above equation is $D
  = -16(a - b)(b - c)$

  If $a > b > c$ then $D < 0$ and if $a < b < c$ then also $D < 0$.

  Since coefficient of $y^2$ is positive and $D < 0$ the expression $y^2 + 2(a + c - 2b)y + (a - c)^2 \ge 0$
  is true for all real values of $y$.

  Therefore, the given expression is capable of holding any value for the given conditions.
\item Given $x + y = k$ (say, a constant). Let $z = xy$, then $z = x(k - x) \Rightarrow x^2 - kx + z = 0$.

  Since $x$ is real, $D \ge 0$ for the above equation.

  $k^2 - 4z \ge 0 \Rightarrow z \le \frac{k^2}{4}$

  Hence, the maximum value of $z = \frac{k^2}{4}$.

  Thus, $x^2 - kx + \frac{k^2}{4} = 0 \Rightarrow \left(x - \frac{k}{2}\right)^2 = 0 \Rightarrow x = \frac{k}{2}$.

  $\therefore y = \frac{k}{2}$ and thus $xy$ is maximum when $x = y$.
\item Let $y = 3 - 6x - 8x^2 \Rightarrow 8x^2 + 6x + y - 3 = 0$. Since $x$ is real, $D \ge 0$ for the this equation.

  $\Rightarrow 36 - 32(y - 3) \ge 0\Rightarrow y \le \frac{33}{8}$. Hence, maximum value of $y = \frac{33}{8}$

  $\Rightarrow 64x^2 + 48x + 9 = 0\Rightarrow (8x + 3)^2 = 0 \Rightarrow x = -\frac{3}{8}$.
\item Let $y = \frac{12x}{4x^2 + 9}\Rightarrow 4yx^2 - 12x + 9y = 0$. Since $x$ is real, $D \ge 0$ for the
  above equation.

  $\Rightarrow 144 - 144y^2 \ge 0\Rightarrow y^2 \le 1\Rightarrow -1 \le y \le 1 \Leftrightarrow |y| \le 1
  \Leftrightarrow\left|\frac{12x}{4x^2 + 9}\right| \le 1$

  Now, $\left|\frac{12x}{4x^2 + 9}\right| = 1 \Leftrightarrow 4|x|^2 - 12|x| + 9 = 0\Rightarrow (2|x| - 3)^2
  = 0 \Rightarrow |x| = \frac{3}{2}$.
\item $x^2 + 9y^2 - 4x + 3 = 0$. Since $x$ is real, $D \ge 0$ for the above equation.

  $\Rightarrow (-4)^2 - 4(9y^2 + 3) \ge 0\Rightarrow 9y^2 - 1 \le 0 \Leftrightarrow y^2 \le
  \frac{1}{9}\Rightarrow -\frac{1}{3} \le y \le \frac{1}{3}$

  The given equation can also be written as $9y^2 + x^2 - 4x + 3 = 0$. Since $y$ is real, $D \ge 0$ for the
  above equation.

  $\Rightarrow -36(x^2 - 4x + 3) \ge 0\Rightarrow x^2 - 4x + 3 \le 0$

  Since coefficient of $x^2$ is positive, it must lie between its root for the above expression to be negative.
  Therefore, $x$ must lie between $1$ and $3$.
\item Given expression is $x^2 - ax + 1 - 2a^2 > 0$

  Since $x$ is real the discriminant of the corresponding equation has to be negative for it to be positive
  for all values of $x$.

  $a^2 - 4(1 - 2a^2) < 0 \Leftrightarrow 9a^2 \le 4\Rightarrow -\frac{2}{3} < a < \frac{2}{3}$.
\item Let $\alpha$ be a common factor, therefore it will satisfy both the equations.

  $\alpha^2 - 11\alpha + a = 0$ and $\alpha^2 - 14\alpha + 2a = 0$. By cross-multiplication

  $\frac{\alpha^2}{-22a + 14x} = \frac{\alpha}{a - 2a} = \frac{1}{-14 + 11}\Rightarrow \frac{\alpha^2}{-8a}
  = \frac{\alpha}{-a} = -\frac{1}{3}$

  From first two we have $\alpha = 8$ and from last two we have $\alpha = \frac{a}{3}\therefore a = 24$.
\item $y = mx$ is a factor of $ax^2 + bxy + cy^2$ means $ax^2 + bxy + cy^2$ will be zero when
  $y = mx$.

  $ax^2 + bx.mx + cm^2x^2 = 0 \Rightarrow cm^2 + bm + a = 0$. Similarly, $a_1m^2 + b_1m + c_1 = 0$ since $my
  - x$ is a factor of $a_1x^2 + b_1xy + c_1y^2$

  Solving these two equations in $m$ by cross-multiplication $\frac{m^2}{bc_1 - ab_1} = \frac{m}{aa_1 -
    cc_1} = \frac{1}{cb_1 - ba_1}$

  From first two we get, $m = \frac{bc_1 - ab_1}{aa_1 - cc_1}$, and from last two we get, $m = \frac{aa_1 -
    cc_1}{cb_1 - ba_1}$

  Equating the two values of $m$ obtained, we get $(bc_1 - ab_1)(cb_1 - ba_1) = (aa_1 - cc_1)^2$.
\item We know that $ax^2 + 2hxy + by^2 + 2gx + 2fy + c$ can be resolved into two linear factors if and only if

  $abc + 2fgh - af^2 - bg^2 - ch^2 = 0$ and $h^2 - ab > 0$. Given expression is $2x^2 + mxy + 3y^2 - 5y - 2$

  Here, $a = 2, h = \frac{m}{2}, b = 3, g = 0, f = \frac{-5}{2}, c = -2\Rightarrow h^2 - ab = \frac{m^2}{4}
  - 6 > 0\Rightarrow m^2 > 24$

  Applying the second condition, $-12 - \frac{25}{2} + \frac{m^2}{2} = 0\Rightarrow m^2 = 49 \therefore m =
  \pm 7$.
\item Given expression is $ax^2 + by^2 + cz^2 + 2ayz + 2bzx + 2cxy$

  $= z^2\left[a\left(\frac{x}{z}\right)^2 + b\left(\frac{y}{z}\right)^2 + c + 2a\frac{y}{z} + 2b\frac{x}{z}
  + 2c\frac{xy}{z^2}\right]$

  $= z^2(aX^2 + bY^2 + c + 2aY + 2bX + 2cXY)$ where $X = \frac{x}{z}, Y = \frac{y}{z}$. Now this will
  resolve in linear factors if

  $abc + 2abc - a.a^2 - b.b^2 -c.c^2 = \Rightarrow a^3 + b^3 + c^3 = 3abc$.
\item Given expression is $2x^2 - y^2 - x + xy + 2y -1$

  Corresponding equation is $2x^2 - y^2 - x + xy + 2y -1 = 0\Rightarrow x = \frac{1 - y \pm \sqrt{(1 - y)^2
      + 8(y^2 - 2y + 1)}}{4}\Rightarrow x = 1 - y, -\frac{1 - y}{2}$.

  Therefore, required linear factors are $x + y - 1$ and $2x - y + 1$.
\item Corresponding quadratic equation is $x^2 + 2(a + b + c)x + 3(ab + bc + ca) = 0$. It will be a perfect
  square if its discriminant is zero.

  $\Rightarrow 4(a + b + c)^2 - 4.3(ab + bc + ca) = 0\Rightarrow a^2 + b^2 + c^2 - ab - bc - ca =
  0$

  $\Rightarrow \frac{1}{2}(a - b)^2(b - c)^2(c - a)^2 = 0\Rightarrow a = b = c$.
\stopitemize