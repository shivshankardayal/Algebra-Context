% -*- mode: context; -*-
\chapter{Miscellaneous Problems}
\startitemize[n, 1*broad]
\item Two H.M.'s were inserted between two numbers and one of them is equal to the geometric mean between
  the numbers. Prove that one number is equal to four times the other number.
\item The H.M. of two numbers is $4$. The A.M. \quote{A} and G.M. \quote{G} satisfy the relationship $2A +
  G^2 = 27$. Find the numbers.
\item If $n$ is odd find the sum of the series $a - (a + d) + (a + 2d) - (a + 3d) + \cdots$.
\item For what value of $y$, the equation $\cos x = y$ has roots in A.P.
\item Sum of all terms of a G.P.\ is $5$ times the sum of odd terms. Find the common ratio.
\item If the sum of $p$ terms of an A.P.\ be $ap + bp^2$, then find its fourth term.
\item Among $302, 303, 304, 305$, which is a term of the sequence $3, 8, 13, \ldots$?
\item Consider points $(2, 1), (1, -2), (3, 2), (1, 1)$. Now if $A, B, C$ are in A.P. then the straight line
  $Ax + By + C = 0$ will pass through which of the points?
\item If the sum of $n$ terms of an A.P.\ is $\frac{3n^2 + 5n}{36}$, then find the common difference.
\item If $a, b, c$ are three consecutive terms of a sequence such that $\frac{a - b}{b - c} = \frac{a}{c}$,
  then find the type of sequence.
\item Find the values of $x$, for which the sum of the series $\left(\frac{2x}{x + 3}\right)
  + \left(\frac{2x}{x + 3}\right)^2 + \left(\frac{2x}{x + 3}\right)^3 + \cdots$ will be definite.
\item If the product of $n$ positive numbers is $1$, then prove that their sum is never less than $1$.
\item Let $S_n$ denote the sum of first $n$ terms of an A.P. If $S_{2n} = 3S_n$ then find the ratio of
  $S_{3n}/S_n$.
\item Find the sum of $n$ terms of the series $1.3 + 3.5 + 5.7 + \cdots$.
\item Sum the series $1 + \frac{4}{5} + \frac{7}{5^2} + \frac{10}{5^3} + \cdots$ for initite terms.
\item If $a, b, c$ are in A.P.; $a, x, b$ are in G.P.; and $b, y, c$ are in G.P., then find the type of
  sequence $x^2, b^2, y^2$ will form.
\item If $\log_{10}2, \log_{10}(2^x - 1), \log_{10}(2^x + 3)$ are the terms of an A.P., find the value of
  $x$.
\item If the numbers $p, q, r$ are in A.P., then prove that $m^{7p}, m^{7q}, m^{7r}(m > 0)$ are in G.P.x
\item For $0< \phi < \frac{\pi}{2}$, if $x = \displaystyle\sum_{n = 0}^\infty \cos^{2n}\phi, y = \sum_{n =
  0}^\infty\sin^{2n}\phi$ and $z = \displaystyle\sum_{n = 0}^\infty\cos^{2n}\phi\sin^{2n}\phi$, then prove
  that $xyz = xy + z$.
\item Find the sum of the series $\sqrt{2} + \sqrt{8} + \sqrt{18} + \sqrt{32} + \cdots$.
\item In a triangle if lengths of sides $a, b, c$ are in A.P. then prove that
  $\tan\frac{A}{2}, \tan\frac{B}{2}, \tan\frac{C}{2}$ are in H.P.
\item The ex-radii of a triangle are in H.P. Prove that $a, b, c$ are in A.P.
\item If the first and $(2n - 1)$st terms of an A.P., G.P.\ and H.P.\ are equal and their $n$th terms are
  $a, b$ and $c$ then prove that $ac - b^2 = 0$.
\item If the roots of the equation $x^3 - 12x^2 + 39 - 28 = 0$ are in A.P., then find their common
  difference.
\item If $a, b, c, d$ and $p$ are distinct real numbers such that $(a^2 + b^2 + c^2)p^2 - 2(ab + bc + cd)p +
  (b^2 + c^2 + d^2)\leq 0$ then prove that $a, b, c, d$ are in G.P.
\item Four numbers are such that first three are in A.P., while the last three are in G.P. If the first
  number is $6$ and common ratio is of the G.P.\ is $\frac{1}{2}$, then find the numbers.
\item If $\frac{44}{9} = 3 + 5r + 7r^2 + 9r^3 + \cdots$ to $\infty$, then find the value of $r$.
\item If $|x| < 1$, then find the sum of the series $1^2 + 2^2x + 3^2x^2 + 4^2x^3 + \cdots$ to $\infty$.
\item Find the sum of $1^2 + (1^2 + 2^2) + (1^2 + 2^2 + 3^2) + \cdots$ up tp $22$ terms.
\item If the ratio of sum of $m$ and $n$ terms of an A.P.\ is $m^2:n^2$, then find the ratio of its $m$th
  and $n$th term.
\item The sum of $n$ terms of two A.P.\ are in the ratio of $7n + 1:4n + 27$. Find the ratio of their $11$th
  terms.
\item A car travels from station $A$ to station $B$ with a speed of $40$ kmph and returns with a speed of
  $60$ kmph. Find the average speed in kmph.
\item If $a, b, c$ are in A.P.\ and $a^2, b^2, c^2$ are in H.P., then prove that $a = b = c$.
\item Find the sum of series $1 + \frac{1}{1 + 2} + \frac{1}{1 + 2 + 3} + \cdots$ up to $\infty$.
\item If the sum of roots of the quadratic equation $ax^2 + bx + c = 0$ is equal to the sum of the squares
  of the reciprocals, then prove that $ab^2, ca^2, bc^2$ are in A.P.
\item If $p$ times the $p$th term of an A.P.\ is equal to $q$ times the $q$th term of the same A.P., then
  prove that $(p + q)$th terms is zero.
\item If in a G.P.\ the first term is $a$, the second term is $b$ and the last term is $c$, then find the
  sum of the series.
\item In a triangle, the length of the two larger sides are $10$ and $9$ respectively. If the angles are in
  A.P., then find the length of the third side.
\item If one geometric mean $G$ and two arithmetic means $p$ and $q$ are inserted between two quantities,
  then prove that $G^2 = (2p - q)(2q - p)$.
\item If $a, b, c$ are three unequal numbers such that $a, b, c$ are in A.P., and $b - a, c - b, a$ are in
  G.P. then prove that $a:b:c = 1:2:3$.
\item If the first and last terms of an A.P.\ are $a$ and $l$ respectively, and $s$ be the sum of all terms
  in the A.P., then find the common difference.
\item Let $T_r$ be the $r$th term of an A.P. for $r = 1, 2, 3, \ldots$ If for some positive integers $m, n$
  we have $T_m = \frac{1}{n}$ and $T_n = \frac{1}{m}$, then find $T_{mn}$.
\item If $\sin\theta, \cos\theta$ and $\tan\theta$ are in G.P., then find the value of $\cot^6\theta
  - \cot^2\theta$.
\item If the perimeter of a $\triangle ABC$ is $6$ times the arithmetic mean of the sines of its angle and
  the side $a$ is $1$, then find the $\angle A$.
\item Find the sum $1.1! + 2.2! + 3.3! + \cdots + n.n!$.
\item If $x, y, z$ are in A.P., then prove that $\frac{1}{\sqrt{x} + \sqrt{y}}, \frac{1}{\sqrt{z}
  + \sqrt{x}}, \frac{1}{\sqrt{y} + \sqrt{z}}$ are also in A.P.
\item Find the sum of the series $2 + \frac{5}{2!3} + \frac{5.7}{3!3^2} + \frac{5.7.9}{4!3^3} + \cdots$.
\item If $1 + \cos\alpha + \cos^2\alpha + \cdots = 2 - \sqrt{2}$, then find $\alpha(0 < \alpha < \pi)$.
\item Find the H.M.\ of the roots of the equation $(5 + \sqrt{2})x^2 - (4 + \sqrt{5})x + 8 + 2\sqrt{5} = 0$.
\item If $\startdeterminant\NC a\NC b\NC a + b\NR\NC b\NC c\NC b + c\NR\NC a + b\NC b + c\NC
  0\NR\stopdeterminant = 0$, then prove that $a, b, c$ are in G.P.
\item For any three positive real numbers $a, b$ and $c$ if $9(25a^2 + b^2) + 25(c^2 - 3ac) = 15b(3a + c)$,
  then prove that $b, c, a$ are in A.P.
\item Let $AP(a, d)$ denote the set of all terms of an A.P.\ with first term $a$ and common difference $d >
  0$. If $AP(1, 3)\cap AP(2, 5)\cap AP(3, 7) = AP(a, d)$ then find $a$ and $d$.
\item The sides of a right angled triangle are in A.P. If the area of triangle is $24$ then find the length
  of the smallest side.
\item For $x\in\mathbb{R}$, let $[x]$ denote the greatest integer $\leq x$, then find the sum of the series
  $\left[-\frac{1}{3}\right] + \left[-\frac{1}{3} - \frac{1}{100}\right] + \left[-\frac{1}{3}
  - \frac{2}{100}\right] + \cdots + \left[-\frac{1}{3} - \frac{99}{100}\right]$.
\item Let the sun of first $n$ terms of a non-constant A.P.\ be $50n + \frac{n(n - 7)}{2}A$, where $A$ is a
  constant. Find the first term and common difference of this A.P.
\item Find the sum of all two digit positive numbers which when divided by $7$ leaves a remainder of $2$ or
  $5$.
\item Let $a_1, a_2, \ldots, a_{30}$ be in an A.P., $S = \displaystyle\sum_{i = 1}^{30}a_i$ and $T
  = \displaystyle\sum_{i = 1}^{15}a_{2i - 1}$. If $a_5 = 27$ and $S - 2T = 75$, then find $a_{10}$.
\item Let $b_i > 1$ for $i = 1, 2, \ldots, 101$. Suppose $\log_eb_1, \log_eb_2, \ldots, \log_eb_{101}$ are
  in A.P. with the common difference of $\log_e2$. Suppose $a_1, a_2, \ldots, a_{101}$ are in A.P. such that
  $a_1 = b_2$ and $a_{51} = b_{51}$. If $t = b_1 + b_2 + \cdots + b_{51}$ and $s = a_1 + a_2 + \cdots +
  a_{51}$, then prove that $s > t$ and $a_{101} < b_{101}$.
\item If the sum of first $n$ terms of an A.P.\ is $cn^2$, then find the sum of squares of these $n$ terms.
\item Let $V_r$ denote the sum of first $r$ terms of an A.P.\ whose first term is $r$ and the common
  difference is $2r - 1$. Given $T_r = V_{r + 1} - V_r$ and $Q_r = T_{r + 1} - T_r$. Find $V_1 + V_2
  + \cdots + V_n$. Prove that $T_r$ is always a composite number and $Q_1, Q_2, Q_3, \ldots$ form an
  A.P.\ with common difference of $6$.
\item Let $p$ and $q$ be the roots of the equation $x^2 - 2x + A = 0$, and, $r$ and $s$ be the roots of the
  equation $x^2 - 18x + B = 0$. If $p < q < r < s$ form an A.P., then find $A$ and $B$.
\item The sum of the series $1^2 + 2.2^2 + 3^2 + 2.4^2 + 5^2 + 2.6^2 + \cdots$ is $\frac{n(n + 1)^2}{2}$
  when $n$ is even. Find the sum when $n$ is odd.
\item The fourth power of the common difference of an A.P.\ with integer entries is added to the produt of
  four consecutive terms of it. Prove that the resulting sum is the square of an integer.
\item The real numbers $x_1, x_2, x_3$ satisfying the equation $x^3 - x^2 + \beta x + \gamma = 0$ are in
  A.P. Find the intervals in which $\beta$ and $\gamma$ lie.
\item Suppose that all the terms of an A.P.\ are natural numbers. If the ratio of the sum of the first seven
  terms to the sum of the first eleven terms is $6:11$ and the seventh term lies between $130$ and $140$,
  then find the c.d.\ of the A.P.
\item A pack contains $n$ cards numbered from $1$ to $n$. Two consecutive cards are removed from the pack
  and the sum of the remaining cards is $1224$. If the smaller of the numbers on the removed cards is $k$,
  then find $k$.
\item Let $a_1, a_2, a_3, \ldots, a_{100}$ be an A.P.\ with $a_1 = 3$ and $S_p = \displaystyle\sum_{i =
  1}^pa_i, 1\leq p\leq 100$. For any integer $n$ with $1\leq n\leq 20$, let $m = 5n$. If $\frac{S_m}{S_n}$
  does not depend on $n$ then find common difference.
\item Let $a_1, a_2, a_3, \ldots, a_{11}$ be real numbers satisfying $a_1 = 15, 27 - a_2 > 0$ and $a_k =
  2a_{k - 1} - a_{k - 2}$ for $k = 3, 4, \ldots, 11$. If $\frac{a_1^2 + a_2^2 + \cdots + a_{11}^2}{11} =
  90$, then find the value of $\frac{a_1 + a_2 + \cdots + a_{11}}{11}$.
\item Let $a, b$ and $c$ be in G.P. with common ratio $r$, where $a\neq 0$ and $0 < r \leq \frac{1}{2}$. If
  $3a, 7b$ and $15c$ are in A.P., then find the $4$th term of this A.P.
\item If $a, b$ and $c$ are three, distinct real numbers in a G.P.\ such that $a + b + c = xb$, then find
  the value of $x$.
\item Let $f(x) = ax^2 + bx + c, a\neq 0$ and $\Delta = b^2 - 4ac$. If $\alpha + \beta, \alpha^2 + \beta^2$,
  and $\alpha^3 + \beta^3$ are in G.P., ($\alpha, \beta$ are roots of the given quadratic equation) then
  prove that $\Delta = 0$.
\item If $|x| < 1, |y| < 1$ and $x\neq y$, then sum the following series to infinity: $(x + y) + (x^2 + xy +
  y^2) + (x^3 + x^2y + xy^2 + y^2) + \cdots$.
\item Let $S_n = 1 + q + q^2 + \cdots + q^n$ and $T_n = 1 + \left(\frac{q + 1}{2}\right) + \left(\frac{q +
  1}{2}\right)^2 + \cdots + \left(\frac{q + 1}{2}\right)^n$, where $q$ is a real number and $q\neq 1$. If
  $C_1^^{101} + C_2^^{101}S_1 + C_3^^{101}S_2 + \cdots + C_{101}^^{101}S_{100} = \alpha.T_{100}$, then find
  $\alpha$.
\item If $10^9 + 2.11.10^8 + 3.11^2.10^7 + \cdots + 10.11^8 = k.10^9$, then find $k$.
\item Let $\alpha$ and $\beta$ be the roots of $x^2 - x - 1 = 0$, with $\alpha > \beta$. For all positive
  integers $n$ define $a_n = \frac{\alpha^n - \beta^n}{\alpha - \beta}, n\geq 1, b_n = 1$ and $b_n = a_{n -
    1} + a_{n - 2}, n\geq 2$. Prove that
  \startitemize[a]
  \item $b_n = \alpha^n + \beta^n$ for all $n \geq 1$,
  \item $a_1 + a_2 + \cdots + a_n = a_{n + 2} - 1$ for all $n\geq 1$, and
  \item $\displaystyle\sum_{n = 1}^\infty\frac{a_n}{10^n} = \frac{10}{89}$.
  \stopitemize
\item Let $A_n = \frac{3}{4} - \left(\frac{3}{4}\right)^2 + \left(\frac{3}{4}\right)^3 - \cdots + (-1)^{n -
  1}\left(\frac{3}{4}\right)^n, B_n = 1 - A_n$. Find the least odd number $n_0$ such that $B_n >
  A_n, \forall n \geq n_0$.
\item Let $S_k$, where $k = 1, 2, \ldots, 100$, denotes the sum of the infinite geometric series whose first
  terrm is $\frac{k - 1}{k!}$ and common ratio is $\frac{1}{k}$. Find the value of $\frac{100^2}{100!}
  + \displaystyle\sum_{k = 1}^{100}|(k^2 - 3k + 1)S_k|$.
\item If $\cos(x - y), \cos x$ and $\cos(x + y)$ are in H.P., then find the value of $\cos
  x\sec\frac{y}{2}$.
\item Find the sum of the series $1 + \frac{1^3 + 2^3}{1 + 2} + \frac{1^3 + 2^3 + 3^3}{1 + 2 + 3} + \cdots +
  \frac{1^3 + 2^3 + \cdots + 15^3}{1 + 2 + \cdots + 15} - \frac{1}{2}(1 + 2 + \cdots + 15)$.
\item Find the sum of the series $\frac{3\times1^3}{1^2} + \frac{5\times(1^3 + 2^3)}{1^2 + 2^2}
  + \frac{7\times(1^3 + 2^3 + 3^3)}{1^2 + 2^2 + 3^2} + \cdots$ for $10$ terms.
\item If the sum of the series $\left(\frac{3}{4}\right)^3 + \left(1\frac{1}{2}\right)^3
  + \left(2\frac{1}{4}\right)^3 + 3^3 + \left(3\frac{3}{4}\right)^3 + \cdots$ for $15$ terms is $225k$, then
  find $k$.
\item Let $S_k = \frac{1 + 2 + \cdots + k}{k}$. If $S_1^2 + S_2^2 + \cdots + S_{10}^2 = \frac{5}{12}A$, then
  find $A$.
\item Let $x, y$ be positive real numbers and $m, n$ positive integers. Then find the maximum value of the
  expression $\frac{x^my^n}{\left(1 + x^{2m}\right)\left(1 + y^{2n}\right)}$.
\item Find the sum of the series $1 + 6 + \frac{9(1^2 + 2^2 + 3^2)}{7} + \frac{12(1^2 + 2^2 + 3^2 + 4^2)}{9}
  + \frac{15(1^2 + 2^2 + 3^2 + 4^2 + 5^2)}{11} + \cdots$ for $15$ terms.
\item Let $a_1, a_2, \ldots, a_{49}$ be in A.P.\ such that $\displaystyle\sum_{k = 0}^{12}a_{4k + 1} = 416$
  and $a_9 + a_{43} = 66$. If $a_1^2 + a_2^2 + \cdots + a_{17}^2 = 140m$, then find the value of $m$.
\item Let $A$ be the sum of first $20$ terms and $B$ be the sum of $40$ terms of the series $1^2 + 2.2^2 +
  3^2 + 2.4^2 + 5^2 + 2.6^2 + \cdots$. If $B - 2A = 100\lambda$, then find $\lambda$.
\item If the sum of first $10$ terms of the series $\left(1\frac{3}{5}\right)^2
  + \left(2\frac{2}{5}\right)^2 + \left(3\frac{1}{5}\right)^2 + 4^2 + \left(4\frac{4}{5}\right)^2 + \cdots$
  is $\frac{16m}{5}$, then find $m$.
\item If $m$ is the A.M.\ of two distinct real numbers $l$ and $n(l, n > 1)$ and $G_1, G_2$ and $G_3$ are
  three G.M.\ between $l$ and $n$, then find the value of $G_1^4 + 2G_2^4 + G_3^4$ in terms of $l, m$ and
  $n$.
\item Find the sum of first $9$ terms of the series $\frac{1^3}{1} + \frac{1^3 + 2^3}{1 + 3} + \frac{1^3 +
  2^3 + 3^3}{1 + 3 + 5} + \cdots$.
\item For $\alpha\in\left(0, \frac{\pi}{2}\right)$, prove that $\sqrt{x^2 + x}
  + \frac{\tan^2\alpha}{\sqrt{x^2 + x}}$ is always greater than or equal to $2\tan\alpha$.
\item If $a_1, a_2, \ldots, a_n$ are positive real numbers whose product is a fixed number $c$, then find
  the minimum value of $a_1 + a_2 + \cdots + a_{n - 1} + a_{2n}$.
\item If $a, b, c$ are positive real numbers such that $a + b + c + d = 2$, then prove that the relation $M
  = (a + b)(c + d)$ satisfies the relation $0 < M \leq 1$.
\item Find the harmonic mean of the roots of the equation $\left(5 + \sqrt{2}\right)x^2 - \left(4 +
  \sqrt{5}\right)x + 8 + 2\sqrt{5} = 0$.
\item If $a, b, c$ are distinct positive numbers, then prove that the expression $(b + c - a)(c + a - b)(a +
  b - c) - abc$ is negative.
\item If $a, b, c$ are positive real numbers, then prove that $\left[(1 + a)(1 + b)(1 + c)\right]^7 >
  7^7a^4b^4c^4$.
\item Let $a_1, a_2, \ldots$ be positive real numbers in G.P. For each $n$, if $A_n, G_n, H_n$ are
  respectively the A.M., G.M., H.M.\ of $a_1, a_2, \ldots, a_n$. Then find an expression for the G.M.\ of
  $G_1, G_2, \ldots, G_n$ in terms of $A_1, A_2, \ldots, A_n, H_1, H_2, \ldots, H_n$.
\item Let $m$ be the minimum possible value of $\log_3\left(3^{y_1} + 3^{y_2} + 3^{y_3}\right)$, where $y_1,
  y_2, y_3$ are real numbers such that $y_1 + y_2 + y_3 = 9$. Let $M$ be the maximum possible value of
  $\log_3x_1 + \log_3x_2 + \log_3x_3$, where $x_1, x_2, x_3$ are real numbers such that $x_1 + x_2 + x_3 =
  9$. Find the value of $\log_2m^3 + \log_3M^2$.
\item Let $a_1, a_2, a_3, \ldots$ be a sequence of positive integers in A.P.\ with c.d.\ $2$. Also, let
  $b_1, b_2, b_3, \ldots$ be a sequence of positive integers in G.P.\ with c.r.\ $2$. If $a_1 = b_2 = c$,
  then find the number of all possible values of $c$ for which the equality $2(a_1 + a_2 + \cdots + a_n) =
  b_1 + b_2 + \cdots + b_n$ holds.
\item Let $a, b, c$ be positive integers such that $b/a$ is an integer. If $a, b, c$ are in G.P.\ and the
  A.M.\ of $a, b, c$ is $b + 2$, then find the value of $\frac{a^2 + a - 14}{a + 1}$.
\item Find the minimum value of the sum of real numbers $a^{-5}, a^{-4}, 3a^{-3}, 1, a^8$ and $a^{10}$ where
  $a > 0$.
\item Let $a_1 = b_1 = 1; a_n = a_{n - 1} + 2$ and $b_n = a_n + b_{n - 1}$ for every natural number $n\geq
  2$. Find $\displaystyle\sum_{n = 1}^{15}a_n.b_n$.
\item If $\frac{1}{2.3^{10}} + \frac{1}{2^2.3^9} + \cdots + \frac{1}{2^{10}.3} = \frac{K}{2^{10}.3^{10}}$,
  then find the remainder when $K$ is divided by $6$.
\item Find the greatest integer less than or equal to the sum of first $100$ terms of the sequence
  $\frac{1}{3}, \frac{5}{9}, \frac{19}{27}, \frac{65}{81}, \cdots$.
\item Consider two G.P.'s $2, 2^2, 2^3, \ldots$ and $4, 4^2, 4^3, \ldots$ of $60$ and $n$ terms
  respectively. If the G.M.\ of all $60 + n$ terms is $2^{\frac{225}{8}}$, then find $\displaystyle\sum_{k =
    1}^{n}k(n - k)$.
\item If $\displaystyle\sum_{k = 1}^{10}\frac{k}{k^4 + k^2 + 1} = \frac{m}{n}$, where $m$ and $n$ are
  co-prime. Find $m + n$.
\item Different A.P.'s are constructed with the first term $100$ and the last term $199$ with integral
  common difference. Find the sum of common differences of all such A.P.'s having at least $3$ terms and at
  most $33$ terms.
\item If $A = \displaystyle\sum_{n = 1}^\infty\frac{1}{\left(3 + (-1)^n\right)^n}$ and $B =
  \displaystyle\sum_{n = 1}^\infty\frac{(-1)^n}{\left(3 + (-1)^n\right)^n}$, then find $\frac{A}{B}$.
\item If $a_1(> 0), a_2, a_3, a_4, a_5$ are in a G.P., $a_2 + a_4 = 2a_3 + 1$ and $3a_2 + a_3 = 2a_4$, then
  find $a_2 + a_4 + 2a_5$.
\item If the sum of the first ten terms of the series $\frac{1}{5} + \frac{2}{65} + \frac{3}{325} +
  \frac{4}{1025} + \frac{5}{2501} + \cdots$ is $\frac{m}{n}$, where $m, n$ are co-prime, then find $m + n$.
\item Let $S = 2 + \frac{6}{7} + \frac{12}{7^2} + \frac{20}{7^3} + \frac{30}{7^4} + \cdots \infty$, then
  find $4S$.
\item Consider the sequence $a_1, a_2, a_3, \ldots$ such that $a_1 = 1, a_2 = 2$ and $a_{n  + 2} =
  \frac{2}{a_{n + 1}} + a_n$ for $n = 1, 2, 3, \ldots$. If $\left(\frac{a_1 +
    \frac{1}{a_2}}{a_3}\right)\left(\frac{a_2 + \frac{1}{a_3}}{a_4}\right)\left(\frac{a_3 +
    \frac{1}{a_4}}{a_5}\right) \ldots \left(\frac{a_{30} + \frac{1}{a_{31}}}{a_{32}}\right) = 2^\alpha
  C_{31}^^{61}$, then find $\alpha$.
\item Let the coefficients of the middle terms in the expansion of $\left(\frac{1}{\sqrt{6}} + \beta
  x\right)^4, (1 - 3\beta x)^2$ and $\left(1 - \frac{\beta}{2}x\right)^6, \beta > 0$, respectively form the
  first three terms of an A.P. If $d$ is the c.d.\ of this A.P.\ then find the value of $50 -
  \frac{2d}{\beta^2}$.
\item Let $A_1, A_2, A_3, \ldots$ be an increasing G.P.\ of positive real numbers. If $A_1A_3A_5A_7 =
  \frac{1}{1296}$ and $A_2 + A_5 = \frac{7}{36}$, then find the value of $A_6 + A_8 + A_{10}$.
\item If $n$ A.M.\ are inserted between $a$ and $100$ such that the ratio of first mean to the last mean is
  $7$ and $a + n = 33$, then find the value of $n$.
\item Let for $n = 1, 2, 3, \ldots, 50, S_n$ be the sum of the infinite G.P. whose first term is $n^2$ whose
  common ratio is $\frac{1}{(n + 1)^2}$. Find the value of $\frac{1}{26} + \displaystyle\sum_{n =
    1}^{50}\left(S_n + \frac{2}{n + 1} - n - 1\right)$.
\item Let $a_1, a_2, a_3, \ldots$ be an A.P. If $\displaystyle\sum_{r = 1}^\infty\frac{a_r}{2^r} = 4$, then
  find $a_2$.
\item If $\frac{1}{2.3.4} + \frac{1}{3.4.5} + \frac{1}{4.5.6} + \cdots + \frac{1}{100.101.102}
  = \frac{k}{102}$, then find $34k$.
\item Let $\{a_n\}_{n = 0}^\infty$ be a sequence such that $a_0 = a_1 = 0$ and $a_{n + 2} = 3a_{n + 1} -
  2a_n + 1, \forall n \geq 0$. Find $a_{25}a_{23} - 2a_{25}a_{22} - 2a_{23}a_{24} + 4a_{22}a_{24}$.
\item Let $\{a_n\}_{n = 0}^\infty$ be a sequence such that $a_0 = a_1 = 0$ and $a_{n + 2} = 2a_{n + 1} - a_n
  + 1, \forall n\geq 0$. Find $\displaystyle\sum_{n = 2}^\infty\frac{a_n}{7^n}$.
\item Let $l_1, l_2, \ldots, l_{100}$ be consecutive terms of an arithmetic progression with common
  difference $d_1$, and let $w_1, w_2, \ldots, w_{100}$ be consecutive terms of another arithmetic
  progression with common difference $d_2$, where $d_1d_2 = 10$. For each $i = 1, 2, \ldots, 100$. Let $R_i$
  be a rectangle with length $l_i$, width $w_i$ and area $A_i$. If $A_{51} - A_{50} = 1000$, then find the
  value of $A_{100} - A_{90}$.
\item Let $a_1, a_2, a_3, \ldots$ be an A.P.\ with $a_i = 7$ and common difference $8$. Let $T_1, T_2,
  T_3, \ldots$ be such that $T_1 = 3$ and $T_{n + 1} - T_n = a_1\forall n\geq 1$. Find $\displaystyle\sum_{i
    = 1}^{20}T_i$.
\item Find the sum of $10$ terms of the series $\frac{1}{1 + 1^2 + 1^4} + \frac{2}{1 + 2^2 + 2^4}
  + \frac{3}{1 + 3^2 + 3^4} + \cdots$.
\item Find the sum $\displaystyle\sum_{n = 1}^\infty\frac{2n^2 + 3n + 4}{2n!}$.
\item If $20^{19} + 2.21.20^{18} + 3.21^2.20^{17} + \cdots + 20(21)^{19} = k.20^{19}$, then find $k$.
\item Let $a_n$ be the $n$th term of the series $5 + 8 + 14 + 23 + 35 + \cdots$ and $S_n
  = \displaystyle\sum_{i = 1}^na_i$. Then find $S_{30} - a_{40}$.
\item Find the sum of all terms of the A.P.\ $3, 8, 13, 18, \ldots, 373$, which are not divisible by $3$.
\item If $S_n = 4 + 11 + 21 + 34 + 50 + \cdots $ to $n$ terms, then find $\frac{1}{60}(S_{29} - S_9)$.
\item Let $a_1, a_2, 2, a_3, a_4$ be an arithmetico-geometric progression. If the common ratio of the G.P.\
  is $2$ and the sum of all $5$ items of the A.G.P.\ is $\frac{49}{2}$, then find $a_4$.
  %129
\item Let $a, b, c, d$ be positive real numbers such that $a + b + c + d = 11$. Find the maximum value of
  $a^5b^3c^2d$.
  %130
\item For $k\in N$ if the sum of the series $1 + \frac{4}{k} + \frac{8}{k^2} + \frac{13}{k^3}
  + \frac{19}{k^4} + \cdots$ is $10$, then find the value of $k$.
  %131
\item Let $a_1 + a_2 + \cdots + a_n = \frac{n^2 + 3n}{(n + 1)(n + 2)}$. If $\displaystyle28\sum_{i =
  1}^{10}\frac{1}{a_i} = p_1p_2p_3\ldots p_m$, where $p_1, p_2, \ldots, p_m$ are the first $m$ prime
  numbers, then find $m$.
  %132
\item Let $S_1, S_2, S_3, \ldots, S_{10}$ respectively be the sum of $12$ terms of $10$ A.P.'s whose first
  terms are $1, 2, 3, \ldots, 10$ and the common differences are $1, 3, 5, \ldots, 19$ respectively. Find
  $\displaystyle\sum_{i = 1}^{10}S_i$.
  %133
\item Find the sum of series $2.2^2 - 3^2 + 2.4^2 - 5^2 + 2.6^2 - 7^2 + \cdots$ up to $20$ terms.
  %134
\item Let $a_1, a_2, a_3, \ldots$ be a G.P.\ of increasing positive numbers. Let the sum of its $6^{\mathrm{th}}$ and
  $8^{\mathrm{th}}$ terms be $2$ and product of $3^{\mathrm{rd}}$ and $5^{\mathrm{th}}$ terms be
  $\frac{1}{9}$. Then find $6(a_2 + a_4)(a_4 + a_6)$.
  %135
\item Let $A_1$ and $A_2$ be two arithmetic means and $G_1, G_2, G_3$ be three geometric means between two
  positive numbers $a$ and $b$. Prove that $G_1^4 + G_2^4 + G_3^4 + G_1^2G_3^2 = G_1G_3(A_1 + A_2)^2$.
  %136
\item If the sum of the series $\left(\frac{1}{2} - \frac{1}{3}\right) + \left(\frac{1}{2^2}
  - \frac{1}{2.3} + \frac{1}{3^2}\right) + \left(\frac{1}{2^2} - \frac{1}{2^2.3} + \frac{1}{2.3^2}
  - \frac{1}{3^3}\right) + \left(\frac{1}{2^4} - \frac{1}{2^3.3} + \frac{1}{2^2.3^2} - \frac{1}{2.3^3}
  + \frac{1}{3^4}\right) + \cdots$ is $\frac{p}{q}$, where $p$ and $q$ are co-prime, then find $p + 3q$.
  %137
\item If $\frac{1^3 + 2^3 + 3^3 + \cdots \mathrm{\;upto\;} n \mathrm{\;terms}}{1.3 + 2.5 + 3.7
  + \cdots \mathrm{\;upto\;} n \mathrm{\;terms}} = \frac{9}{5}$, then find $n$.
  %138
\item For three integers $p, q, r, x^{pq^2} = y^{qr} = z^{p^2r}$ and $r = pq + 1$ such that $3, 3\log_yx,
  3\log_zy, 7\log_xz$ are in A.P.\ with c.d.\ of $\frac{1}{2}$. Find $r - p - q$.
  %139
\item The $4^{\mathrm{th}}$ term of a G.P.\ is $500$ and its common ratio is $\frac{1}{m},
  m\in\mathbb{N}$. Let $S_n$ denote the sum of the first $n$ terms of the G.P. If $S_6 > S_5 + 1$ and $S_7 <
  S_6 + \frac{1}{2}$ then what is the possible no.\ of values of $m$?
  %140
\item Let $A, A + 1, A + 2$ be the first terms of three A.P.s and $d$ be their c.d. Let $a, b, c$ be the
  $7^{\mathrm{th}}, 9^{\mathrm{th}}, 17^{\mathrm{th}}$ terms of these A.P.s each such that
  $\startdeterminant\NC a\NC 7\NC 1\NR\NC 2b\NC 17\NC 1\NR\NC c\NC 17\NC 1\NR\stopdeterminant + 70 = 0$. If
  $a = 29$ then find the sum of first $20$ terms of the A.P.\ whose first term is $c - a - b$ and c.d.\
  $\frac{d}{12}$.
  %141
\item For two positive numbers $a$ and $b$, if $a, b, \frac{1}{18}$ are in G.P.\ and $\frac{1}{a},
  10, \frac{1}{b}$ are in A.P., then find $16a + 12b$.
  %142
\item Let $a_1 = b_1 = 1$ and $a_n = a_{n - 1} + (n - 1), b_n = b_{n - 1} + a_{n - 1}, \forall n\geq 2$. If
  $S = \displaystyle\sum_{n = 1}^{10}\frac{b_n}{2^n}$ and $T = \displaystyle\sum_{n = 1}^8 \frac{n}{2^{n -
    1}}$, then find $2^7(2S - T)$.
  %143
\item Let $\{a_k\}$ and $\{b_k\}, k\in\mathbb{N}$ be two G.P.s with c.r.\ $r_1$ and $r_2$ respectively such
  that $a_1 = b_1 = 4$ and $r_1 < r_2$. Let $c_k = a_k + b_ k$. If $c_2 = 5$ and $c_3 = \frac{13}{4}$, then
  find $\displaystyle\sum_{k = 1}^\infty c_k - (12a_6 + 8b_4)$.
  %144
\item If $a_n = \frac{-2}{4n^2 - 6n + 15}$, then find $a_1 + a_2 + \cdots + a_{25}$.
  %145
\item Let $a, b, c > 1, a^3, b^3$ and $c^3$ are in A.P., and $\log_ab, \log_ca$ and $\log_bc$ are in G.P. If
  the sum of first $20$ terms of an A.P., whose first term is $\frac{a + 4b + c}{3}$ and the c.d.\ is
  $\frac{a - 8b + c}{10}$ is $-444$, then find $abc$.
  %146
\item If the sum and product of four consecutive terms of a G.P.\ are $126$ and $1296$, respectively, then
  find the sum of common ratios of such G.P.'s.
  %147
\item Let $a_1, a_2, \ldots, a_n$ be in A.P. If $a_5 = 2a_7$ and $a_{11} = 18$, then find
  $12\left(\frac{1}{\sqrt{a_{10}} + \sqrt{a_{11}}} + \frac{1}{\sqrt{a_{11}} + \sqrt{a_{12}}} + \cdots
  + \frac{1}{\sqrt{a_{17}} + \sqrt{a_{18}}}\right)$.
  %148
\item Find the sum $1^2 - 2.3^2 + 3.5^2 - 4.7^2 + 5.9^2 - \cdots + 15.29^2$.
  %149
\item Let $755\ldots 557$ denote the $(r + 2)$ digit number where the first and last digits are $7$ and the
  remaining digits are $5$. Consider the sum $S = 77 + 757 + 7557 + \cdots + 7\underbrace{55\ldots55}_{98
    \;\mathrm{times}}7$. If $S = \frac{7\overbrace{55\ldots55}^{99 \;\mathrm{times}}7 + m}{n}$, where $m$
  and $n$ are natural numbers less than $3000$, then find $m + n$.
  %150
\item If $3, a, b, c$ are in A.P.\ and $3, a - 1, b + 1$ are in G.P., then find the A.M.\ of $a, b$ and $c$.
  %151
\item If $8 = 3 + \frac{1}{4}(3 + p) + \frac{1}{4^2}(3 + 2p) + \frac{1}{4^3}(3 + 3p) + \cdots \infty$, then
  find $p$.
  %152
\item $\log_ea, \log_eb, \log_ec$ are in A.P. and $\log_ea - \log_e2b, \log_e2b - \log_e3c, \log_e3c -
  \log_ea$ are also in A.P. Find $a:b:c$.
  %153
\item Let $\alpha = 1^2 + 4^2 + 8^2 + 13^2 + \cdots$ up to $10$ terms and $\beta = \displaystyle\sum_{n =
  1}^{10}n^4$. If $4\alpha - \beta = 55k + 40$, then find $k$.
  %154
\item Let $a$ and $b$ two distinct positive real numbers. Let $11^{\mathrm{th}}$ term of a G.P., whose first
  term is $a$ and third term is $b$, is equal to $p^{\mathrm{th}}$ term of another G.P., whose first term is
  $a$ and and fifth term is $b$. Find $p$.
  %155
\item Let $S_n$ be the sum of $n$ terms of an A.P.\ $3, 7, 11, \ldots$. If $40 < \frac{6}{n(n +
  1)}\displaystyle\sum_{k = 1}^nS_k < 42$, then find $n$.
  %156
\item Sum the series $\frac{1}{1 - 3.1^2 + 1^4} + \frac{1}{1 - 3.2^2 + 2^4} + \frac{1}{1 - 3.3^3 + 3^4} +
  \cdots$ up to $10$ terms.
  %157
\item If three successive terms of a G.P.\ with common ratio $r(r > 1)$ are the lengths of the sides of a
  triangle and $[r]$ denotes the greatest integer less than or equal to $r$, then find $3[r] + [-r]$.
  %158
\item Find the value of $\frac{1\times2^2 + 2\times3^2 + \cdots + 100\times101^2}{1^2\times2 + 2^2\times3 +
  \cdots + 100^2\times101}$.
  %159
\item Let $a_1, a_2, a_3$ be an A.P.\ of positive numbers. Let $A_k = a_1^2 - a_2^2 + a_3^2 - a_4^2 + \cdots
  + a_{2k - 1}^2 - a_{2k}^2$. If $A_3 = -153, A_5 = -435$ and $a_1^2 + a_2^2 + a_3^2 = 66$ then find $a_{17}
  - A_7$.
  %160
\item For $x\geq 0$, find the least value of $K$, for which $4^{1 + x} + 4^{1 - x}, \frac{K}{2}, 16^x +
  16^{-x}$ are three consecutive terms of an A.P.
  %161
\item Let the first term of a series be $t_1 = 6$ and its $r^{\mathrm{th}}$ term $t_r = 3t_{r - 1} + 6^r, r
  = 1, 2, 3, \ldots, n$. If the sum of the first $n$ terms of this series is $\frac{1}{5}(n^2 - 12n +
  39)(4.6^n - 5.3^n + 1)$, then find $n$.
  %162
\item Let $S_1 = 1, S_2 = 2 + 3, S_3 = 4 + 5 + 6, S_4 = 7 + 8 + 9 + 10; \ldots$. If $S_k$ is sum of exactly
  $k$ numbers for every natural number $k$ then find $k$ for which $S_k$ will contain $5310$.
  %163
\item An A.P.\ is written in following way

  \startformula\startalign[n=7]
  \NC\NC\NC\NC2\NC\NC\NC\NR
  \NC\NC\NC5\NC\NC8\NC\NC\NR
  \NC\NC11\NC\NC14\NC\NC17\NC\NR
  \NC20\NC\NC23\NC\NC26\NC\NC29\NR
  \NC\NC\NC\NC\cdots\NC\NC\NC\NR
  \stopalign\stopformula

  Find the sum of all numbers in $10^{\mathrm{th}}$ row.
  %164
\item If $\left(\frac{1}{\alpha + 1} + \frac{1}{\alpha + 2} + \cdots + \frac{1}{\alpha + 1012}\right)
  - \left(\frac{1}{2.1} + \frac{1}{4.3} + \frac{1}{6.5} + \cdots + \frac{1}{2024.2013}\right)
  = \frac{1}{2024}$, then find $\alpha$.
  %165
\item If $f(x) = \frac{2^x}{2^x + \sqrt{2}}, x\in\mathbb{R}$, then find $\displaystyle\sum_{k =
  1}^{81}f\left(\frac{k}{82}\right)$.
  %166
\item Let $<a_n>$ be a sequence such that $a_0 = 0, a_1 = \frac{1}{2}$ and $2a_{n + 2} = 5a_{n + 1} - 3a_n,
  n = 0, 1, 2, 3, \ldots$. Find $\displaystyle\sum_{k = 1}^{100}a_k$.
  %167
\item Let $T_r$ be the $r^{\mathrm{th}}$ term of an A.P. If for some $m, T_m = \frac{1}{25}, T_{25}
  = \frac{1}{20}$ and $20\displaystyle\sum_{r = 1}^{25}T_r = 13$, then find $4m\displaystyle\sum_{r =
    m}^{2m}T_r$.
  %168
\item For $n\in\mathbb{P}$, if $4a_n = n^2 + 5n + 6$ and $S_n = \displaystyle\sum_{i = 1}^n\frac{1}{a_n}$,
  then find the value of $507S_{2025}$.
  %169
\item Consider an A.P.\ of positive integers, whose sum of first three terms is $54$ and the sum of first
  $20$ terms lies between $1600$ and $1800$. Find its $11^{\mathrm{th}}$ term.
  %170
\item Let $a_1, a_2, \ldots, a_{2024}$ be an A.P.\ such that $a_1 + (a_5 + a_{10} + a_{15} + \cdots +
  a_{2020}) + a_{2024} = 2233$. Find $\displaystyle\sum_{i = 1}^{2024}a_i$.
  %171
\item Find the sum $\displaystyle\sum_{i = 1}^n\frac{4.i}{1 + 4.i^4}$.
  %172
\item Find the sum $\displaystyle\sum_{i = 1}^n\frac{4.i}{4 + 3.i^2 + i^4}$.
  %173
\item If $\frac{1}{1^4} + \frac{1}{2^4} + \frac{1}{3^4} + \frac{1}{4^4} + \cdots \infty
  = \frac{\pi^4}{90}, \frac{1}{1^4} + \frac{1}{3^4} + \frac{1}{5^4} + \cdots \infty = \alpha, \frac{1}{2^4}
  + \frac{1}{4^4} + \frac{1}{6^4} + \cdots \infty = \beta$, then find $\frac{\alpha}{\beta}$.
  %174
\item If $x^2 + 9y^2 + 25z^2 = xyz\left(\frac{15}{x} + \frac{5}{y} + \frac{3}{z}\right)$, then prove that
  $x, y, z$ are in H.P.
  %175
\item The geometric mean of two numbers is $6$. Their arithmetic mean $A$ and harmonic mean $H$ satisfy the
  equation $90A + 5H = 918$. Find $A$.
  %176
\item Prove that the complex numbers $z = x + iy$, which satisfy the equation $\left|\frac{z - 5i}{z +
  5i}\right| = 1$ lie on $x$-axis.
  %177
\item Find the region given by $|z - 4| < |z - 2|$.
  %178
\item Let $z\in\mathbb{C}$ with $\Im(z) = 10$ and it satisfies $\frac{2z - n}{2z + n} = 2i - 1$ for some
  natural number $n$, then find $n$.
  %179
\item Find the locus of set of points $S = \left\{\frac{\alpha + i}{\alpha - i}\right}$, where
  $\alpha\in\mathbb{R}$.
  %180
\item Let $z\in{C}$ be such that $|z| < 1$. If $\omega = \frac{5 + 3z}{5(1 - z)}$, then prove that
  $\Re(\omega) > \frac{1}{5}$.
  %181
\item Let $\left(-2 -\frac{i}{3}\right)^3 = \frac{x + iy}{27}$, where $x$ and $y$ are real numbers, then
  find $y - x$.
  %182
\item Let $S = \left\{\theta\in\left(-\frac{\pi}{2}, \pi\right): \frac{3 + 2i\sin\theta}{1 -
  2i\sin\theta}\right\}$ is a set of purely imaginary numbers, then find the sum of elements of set $S$.
  %183
\item Find principal value of $\theta$ for which $\frac{2 + 3i\sin\theta}{1 - 2i\sin\theta}$ is purely
  imaginary.
  %184
\item If $\startdeterminant\NC6i\NC-3i\NC1\NR\NC 4\NC 3i\NC -1\NR\NC 20\NC 3\NC i\NR\stopdeterminant = x +
  iy$, then find $x$ and $y$.
  %185
\item Find the sum $\displaystyle\sum_{n = 1}^{13}(i^n + i^{n + 1})$.
  %186
\item Find the smallest number $n$ for which $\left(\frac{1 + i}{1 - i}\right)^n = 1$.
  %187
\item Let $a, b, x$ and $y$ be real numbers such that $a - b = 1$ and $y\neq 0$. If the complex number $z =
  x + iy$ satisfies $\Im\left(\frac{az + b}{z + 1}\right) = y$, then find the possible values for $x$.
  %188
\item Find the locus represented by the equation $|z - i| = |z - 1|$.
  %189
\item If $a > 0$ and $z = \frac{(1 + i)^2}{a - i}$ has a magnitude of $\sqrt{\frac{2}{5}}$, then find
  $\overline{z}$.
  %190
\item Let $z_1$ and $z_2$ be two complex numbers satisfying $|z_1| = 9$ and $|z_2 - 3 - 4i| = 4$, then find
  the minimum value of $|z_1 - z_2|$.
  %191
\item If $\frac{z - \alpha}{z + \alpha}(\alpha\in\mathbb{R})$ is purely imaginary number and $|z| = 2$, then
  find the value of $\alpha$.
  %192
\item Let $z$ be a complex number such that $|z| + z = 3 + i$. Find $|z|$.
  %193
\item A complex number $z$ is said to be unimodular, if $|z| = 1$. If $z_1$ and $z_2$ are complex numbers
  such that $\frac{z_1 - 2z_2}{2 - z_1\overline{z_2}}$ is unimodular and $z_2$ is not unimodular. Prove that
  $z_1$ lies on a circle of radius $2$.
  %194
\item If $z$ is a complex number such that $|z|\geq 2$, then find the minimum value of $\left|z +
  \frac{1}{2}\right|$.
  %195
\item Let complex numbers $\alpha$ and $1/\overline{\alpha}$ lies on circles $(x - x_0)^2 + (y - y_0)^2 =
  r^2$ and $(x - x_0)^2 + (y - y_0)^2 = 4r^2$ respectively. If $z_0 = x_0 + iy_0$ satisfies the equation
  $2|z_0|^2 = r^2 + 2$, then find $|\alpha|$.
  %196
\item Let $z$ be a complex number such that it is not purely real and $a = z^2 + z + 1$ is real. Prove that
  $a$ cannot take the value of $\frac{3}{4}$.
  %197
\item Let $z = x + iy$ be a complex number, where $x$ and $y$ are integers. Then, find the area of the
  rectangle whose vertices are the root of the equation $z\overline{z}^3 + z^3\overline{z} = 350$.
  %198
\item If $|z| = 1$ and $z\neq \pm 1$, then find the locus of $\frac{z}{1 - z^2}$.
  %199
\item If $w = \alpha + \i\beta$, where $\beta\neq 0$ and $z\neq 1$, satisfies the condition that $\frac{w -
  \overline{w}z}{1 - z}$ is purely real, then find the set of numbers represented by $z$.
  %200
\item If $|z| = 1$ and $w = \frac{z - 1}{z + 1}$(where, $z\neq -1$), then find $\Re(w)$.
  %201
\item If $z_1, z_2$ and $z_3$ are complex numbers such that $|z_1| = |z_2| = |z_3| = \left|\frac{1}{z_1} +
  \frac{1}{z_2} + \frac{1}{z_3}\right|$, then find $|z_1 + z_2 + z_3|$.
  %202
\item For what positive integers $n_1, n_2$ the expression $(1 + i)^{n_1} + (1 + i^3)^{n_1} + (1 +
  i^5)^{n_2} + (1 + i^7)^{n_2}$ will have a real value?
  %203
\item Find the value of $x$, for which $\sin x + i\cos2x$ and $\cos x -i\sin 2x$ are a conjugate pair.
  %204
\item Let $S$ be the set of complex numbers $z$ satisfying $|z^2 + z + 1| = 1$, then find the constraints on
  $z$ for it to be true.
  %205
\item Let $s, t, r$ be non-zero complex numbers and $L$ be the set of solutions $z = x + iy$ of the equation
  $sz + t\overline{z} + r = 0$, then prove that if $L$ has exactly one element, then $|s|\neq |t|$; the
  number of elements in $L\cap \left\{z: |z - 1 + i| = 5\right\}$ is at most $2$; and if $L$ has more than
  one element, then $L$ has infinitely many elements.
  %206
\item Let $z_1$ and $z_2$ be complex numbers such that $z_1\neq z_2$ and $|z_1| = |z_2|$. If $z_1$ has
  positive real part and $z_2$ has negative imaginary part, then prove that $\frac{z_1 + z_2}{z_1 - z_2}$
  may be zero or purely imaginary.
  %207
\item Let $A, B, C$ be three sets of complex numbers defined as: $A = \left\{z:\Im(z)\geq 1\right\}, B =
  \left\{z: |z - 2 - i|= 3\right\}$ and $C = \left\{z:\Re((1 - i)z) = \sqrt{2}\right\}$.
  \startitemize[i]
  \item Let $z$ be any point in $A\cap B\cap C$ and let $w$ be any point satisfying $|w - 2 - i| < 3$. Find
    the range in which $|z| - |w| + 3$ lies.
  \item Let $z$ be any point in $A\cap B\cap C$. Find the value of  $|z + 1 - i|^2 + |z + 5 - i|^2$
    lies.
  \item Find the number of elements in the set $A\cap B\cap C$.
  \stopitemize
  %208
\item Let $S = S_1\cap s_2\cap S_3$, where $S_1 = \left\{z\in\mathbb{C}: |z| < 4\right\}, S_2 =
  \left\{z\in\mathbb{C}: \Im\left[\frac{z - 1 + \sqrt{3}i}{1 - \sqrt{3}i}\right] > 0\right\}$ and $S_3 =
  \left\{z\in\mathbb{C}: \Re(z) > 0\right\}$.
  \startitemize[i]
  \item Find minimum of $|1 - 3i - z|$.
  \item Find the area of $S$.
  \stopitemize
  %209
\item Find the set of points satisfying $|z - i|z|| = |z + i|z||$.
  %210
\item Find the set of points satisfying $|z + 4| + |z - 4| = 0$.
  %211
\item If $|w| = 2$, then find the locus of set of points $z = w - \frac{1}{w}$.
  %212
\item If $\alpha, \beta, \gamma$ are the cube roots of $p, p < 0$, then for any $x, y, z$ find the value of
  $\frac{x\alpha + y\beta + z\gamma}{x\beta + y\gamma + z\alpha}$.
  %213
\item For any two complex numbers $z_1, z_2$ and any real numbers $a$ and $b$, find $|az_1 - bz_2|^2 + |bz_1
  + az_2|^2$.
  %214
\item Find the center and radius of the circle formed by all the points represented by $z = x + iy$
  satisfying the relation $\left|\frac{z - \alpha}{z - \beta}\right| = k(k\neq 1)$, where $\alpha$ and
  $\beta$ are the constant complex numbers given by $\alpha = \alpha_1 + i\alpha_2, \beta = \beta_1 +
  i\beta_2$.
  %215
\item Prove that there exists no complex number $z$ such that $|z| < 1/3$ and $\displaystyle\sum_{r =
  1}^na_rz^r = 1$ where $|a_r| < 2$.
  %216
\item If $z_1$ and $z_2$ are two complex numbers such that $|z_1| < 1 < |z_2|$, then prove that
  $\left|\frac{1 - z_1\overline{z_2}}{z_1 - z_2}\right| < 1$.
  %217
\item For complex numbers $z$ and $w$ prove that $|z|^2w - |w|^2z = z - w$, if and only if $z = w$ or
  $z\overline{w} = 1$.
  %218
\item Prove that $i^i$ is real and find its value.
  %219
\item Find all non-zero complex numbers such that $\overline{z} = iz^2$.
  %220
\item If $iz^3 + z^2 - z + i = 0$, then show that $|z| = 1$.
  %221
\item A relation $R$ on the set of complex numbers is defined by $z_1Rz_2$ if and only if $\frac{z_1 -
  z_2}{z_1 + z_2}$ is real. Show that $R$ is an equivalence relation.
  %222
\item Find the real values of $x$ and $y$ for which the equation $\frac{(1 + i)x - 2i}{3 + i} + \frac{(2 -
  3i)y + i}{3 - i} = i$ is satisfied.
  %223
\item If $z$ is any complex number satisfying $|z - 3 - 2i| \leq 2$, then find the maximum value of $|2z - 6
  + 5i|$.
  %224
\item If $z_1, z_2$ are two complex numbers such that $\Re(z_1) = |z_1 - 1|, \Re(z_2) = |z_2 - 1|$ and
  $\arg(z_1 - z_2) = \frac{\pi}{6}$, then find $\Im(z_1 + z_2)$.
  %225
\item Let $z_1$ and $z_2$ be any two non-zero complex numbers such that $3|z_1| = 4|z_2|$. If $z =
  \frac{3z_1}{2z_2} + \frac{2z_2}{3z_1}$, then find $|z|$.
  %226
\item If $z$ is a complex number of unit modulus and argument $\theta$, then find $\arg\left(\frac{1 + z}{1
  + \overline{z}}\right)$.
  %227
\item If $\arg(z) < 0$, then find $\arg(-z) - \arg(z)$.
  %228
\item Let $z$ and $w$ be two complex numbers such that $|z|\leq 1, |w|\leq 1$ and $|z + iw| = |z -
  i\overline{w}| = 2$, then find $z$.
  %229
\item Let $z$ and $w$ be two non-zero complex numbers such that $|z| = |w|$ and $\arg(z) + \arg(w) = \pi$,
  then find $z$.
  %230
\item If $z_1$ and $z_2$ are two non-zero complex numbers such that $|z_1 + z_2| = |z_1| + |z_2|$, then find
  $\arg(z_1) - \arg(z_2)$.
  %231
\item Let $z_1$ and $z_2$ be two distinct complex numbers and let $z = (1 - t)z_1 + tz_2$ for some real
  number $t$ with $0 < t < 1$. If $\arg(w)$ denotes the principal argument of a non-zero complex number $w$,
  then prove that $|z - z_1| + |z - z_2| = |z_1 - z_2|, \arg(z - z_1) = \arg(z_2 - z_1)$ and
  $\startdeterminant\NC z - z_1\NC \overline{z} - \overline{z_1}\NR\NC z_2 - z_1\NC \overline{z_2}
  - \overline{z_1}\NR\stopdeterminant = 0$.
  %232
\item If $|z|\leq 1, |w|\leq 1$, then show that $|z - w|^2\leq \left(|z| - |w|\right)^2 + [\arg(z)
  - \arg(w)]^2$.
  %233
\item Let $z_1 = 10 + 6i$ and $z_2 = 4 + 6i$. If $z$ is any complex number such that the argument of
  $\frac{z - z_1}{z - z_2}$ is $\pi/4$, then prove that  $|z - 7 - 9i| = 3\sqrt{2}$.
  %234
\item Let $S$ be the set of all complex numbers $z$ satisfying $|z - 2 + i|\geq \sqrt{5}$. If the complex
  number $z_0$ such that $\frac{1}{|z_0 - 1|}$ is the maximum of the set $\left\{\frac{1}{|z_0 - 1|}: z\in
  S\right\}$, then find the principal argument of $\frac{4 - z_0 - \overline{z_0}}{z_0 - \overline{z_0} +
    2i}$.
  %235
\item A particle $P$ starts from the point $z_0 = 1 + 2i$. It moves first horizontally away from origin by
  $5$ units and then vertically away from origin by $3$ units to reach a point $z_1$. From $z_1$ the
  particle moves $\sqrt{2}$ units in the direction of the vector $\hat{i} + \hat{j}$ and then it moves
  through an angle $\frac{\pi}{2}$ in anti-clockwise direction on a circle with center at origin $O$ to
  reach a point $z_2$. Find the coordinates of $z_2$.
  %236
\item A man walks a distance of $3$ units from the origin towards the North-East(N $45^\circ$ E)
  direction. From there, he walks a distance of $4$ units towards North-West(N $45^\circ$ W) direction to
  reach point $P$. Find the coordinates of point $P$.
  %237
\item Prove that the shaded region given by $P = (-1, 0), Q = (-1 + \sqrt{2}, \sqrt{2}), R = (-1 + \sqrt{2},
  -\sqrt{2}), S = (1, 0)$ is given by $|z + 1| > 2, |\arg(z + 1)| < \frac{\pi}{4}$.

  \startplacefigure[location=force]
    \startMPcode
      numeric u;
      u = 1cm;

      draw (u*(-1 + sqrt(2)), u*sqrt(2)) .. (u, 0) .. (u*(-1 + sqrt(2)), -u*sqrt(2));
      draw ((-u, 0) -- (u*(-1 + sqrt(2)), u*sqrt(2))) shortened -1u;
      draw ((-u, 0) -- (u*(-1 + sqrt(2)), -u*sqrt(2))) shortened -1u;
      fill (u*(-1 + sqrt(2)), u*sqrt(2)) .. (u, 0) .. (u*(-1 + sqrt(2)), -u*sqrt(2)) -- (2u, -3u) .. (3.4u,
      0) .. (2u, 3u) -- cycle withcolor 1/2;
      drawdblarrow (-2u, 0) -- (4u, 0);
      drawdblarrow (0, 2u) -- (0, -2u);
      label.llft("$O$", origin);
      label.lft("$X'$", (-2u, 0));
      label.rt("$X$", (4u, 0));
      label.top("$Y$", (0, 2u));
      label.bot("$Y'$", (0, -2u));
      label.ulft("$P$", (-u, 0));
      label.top("$Q$", (u*(-1 + sqrt(2)), u*sqrt(2)));
      label.bot("$R$", (u*(-1 + sqrt(2)), -u*sqrt(2)));
      label.ulft("$S$", (u, 0));
    \stopMPcode
  \stopplacefigure
  %238
\item Let the complex numbers $z_1, z_2$ and $z_3$ satisfy $\frac{z_1 - z_3}{z_2 - z_3} = \frac{1 -
  i\sqrt{3}}{2}$. Prove that these complex numbers are vertices of an equilateral triangle.
  %239
\item Let $a, b\in\mathbb{R}$ and $a^2 + b^2\neq 0$. Suppose $S = \left\{z\in\mathbb{C}: z = \frac{1}{a +
  ibt}, t\in\mathbb{R}, t\neq 0\right\}$. If $z = x + iy$, then find the locus of $z$.
  %240
\item Let $W = \frac{\sqrt{3} + i}{2}$ and $P = \left\{W^n: n = 1, 2, 3, \ldots\right\}$. Further $H_1
  = \left\{z\in\mathbb{C}: \Re(z) > \frac{1}{2}\right\}$ and $H_2 = \left\{z\in\mathbb{C}: \Re(z) <
  -\frac{1}{2}\right\}$, where $\mathbb{C}$ is the set of all complex numbers. If $z_1\in P\cap H_1, z_2\in
  P\cap H_2$ and $O$ represents the origin, then find $\angle z_1Oz_2$.
  %241
\item Suppose that $z_1, z_2, z_3$ are the vertices of an equilateral triangle inscribed in a circle given
  by $|z| = 2$. If $z_1 = 1 + i\sqrt{3}$, then find $z_2$ and $z_3$.
  %242
\item If one of the vertices of the square circumscribing the circle $|z - 1| = \sqrt{2}$ is $2 +
  \sqrt{3}i$. Find the other vertices of the square.
  %243
\item Let $S$ be the set of all complex numbers $z$ satisfying $z^4 - |z|^3 = 4iz^2$. Find the minimum
  possible value of $|z_1 - z_2|^2$, where $z_1, z_2\in S$ with $\Re(z_1) > 0$ and $\Re(z_2) < 0$.
  %244
\item For any integer $k$, let $\alpha_k = \cos\left(\frac{2\pi}{7}\right) +
  i\sin\left(\frac{2\pi}{7}\right)$. Find the value of the expression $\displaystyle\frac{\sum_{k =
      1}^{12}|\alpha_{k + 1} - \alpha_k|}{\sum_{k = 1}^3|\alpha_{4k - 1}^{4k - 2}|}$.
  %245
\item Find the value of $\displaystyle\left(\frac{1 + \sin\frac{2\pi}{9} + i\cos\frac{2\pi}{9}}{1
  + \sin\frac{2\pi}{9} - i\cos\frac{2\pi}{9}}\right)^3$.
  %246
\item If $z$ and $w$ are two complex numbers such that $|zw| = 1$ and $\arg(z) - \arg(w) = \frac{\pi}{2}$,
  then prove that $\overline{z}w = -i$.
  %247
\item If $z = \frac{\sqrt{3}}{2} + \frac{i}{2}$, then find $\left(1 + iz + z^5 + iz^8\right)^9$.
  %248
\item Let $z_0$ be a root of the equation $x^2 + x + 1 = 0$. If $z = 3 + 6iz_0^{81} - 3iz_0^{93}$, then find
  $\arg(z)$.
  %249
\item Let $z = \cos\theta + i\sin\theta$, then find the value of $\displaystyle\sum_{m =
  1}^{15}\Im\left(z^{2m - 1}\right)$ if $\theta = 2^\circ$.
  %250
\item Find the minimum value of $|a + b\omega + c\omega^2|$, where $a, b$ and $c$ are not all equal integers
  and $\omega(\neq 1)$ is a cube root of unity.
  %251
\item If $\omega(neq 1)$ be a cube root of unity and $(1 + \omega^2)^n = (1 + \omega^4)^n$, then find the
  least possible value of $n$.
  %252
\item If $\omega(neq 1)$ be a cube root of unity and $(1 + \omega)^7 = A + B\omega$, then find $A$ and $B$.
  %253
\item Let $\omega$ be the complex number $\cos\frac{2\pi}{3} + i\sin\frac{2\pi}{3}$, then find the number of
  distinct complex numbers $z$ satisfying $\startdeterminant\NC z +
  1\NC \omega \NC \omega^2\NR\NC \omega \NC z + \omega^2 \NC 1\NR\NC \omega^2\NC 1\NC z
  + \omega\NR\stopdeterminant = 0$.
  %254
\item Find the value of the expression $1(2 - \omega)(2 - \omega^2) + 2(3 - \omega)(3 - \omega^2) + \cdots +
  (n - 1)(n - \omega)(n - \omega^2)$.
  %255
\item Let a complex number $\alpha, \alpha\neq 1$ be a root of the equation $z^{p + q} - z^p - z^q + 1 = 0$,
  where $p, q$ are distinct primes. Show that either $1 + \alpha + \alpha^2 + \cdots + \alpha^{p- 1} = 0$ or
  $1 + \alpha + \alpha^2 + \cdots + \alpha^{q- 1} = 0$ but not both together.
  %256
\item If $1, a_1, a_2, \ldots, a_{n - 1}$ are the $n$ roots of unity, then show that $(1 - a_1)(1 -
  a_2)\cdots(1 - a_{n - 1}) = n$.
  %257
\item Let $\omega\neq 1$ be a cube root of unity. Then find the minimum of $|a + b\omega + c\omega^2|^2$,
  where $a, b, c$ are distinct integers.
  %258
\item Let $\omega = e^{i2\pi/3}$ and $a, b, c, x, y, z$ be non-zero complex numbers such that $a + b + c =
  x, a + b\omega + c\omega^2 = y, a + b\omega^2 + c\omega = z$, then find the value of $\dfrac{|x|^2 + |y|^2
    + |z|^2}{|a|^2 + |b|^2 + |c|^2}$.
  %259
\item If $\alpha, \beta, \gamma, \delta$ are the roots of the equation $x^4 + x^3 + x^2 + x + 1 = 0$, then
  find the value of $\alpha^{2021} + \beta^{2021} + \gamma^{2021} + \delta^{2021}$.
  %260
\item For $n\in\mathbb{N}$, let $S_n = \left\{z\in\mathbb{C}: |z - 3 + 2i| = \frac{n}{4}\right\}$ and $T_n =
  \left\{z\in\mathbb{C}: |z - 2 + 3i| = \frac{1}{n}\right\}$. Find the number of elements in the set
  $\left\{S_n\cap T_n = \phi\right\}$.
  %261
\item For $z\in\mathbb{C}$, if minimum value of $|z - 3\sqrt{2}| + |z - p\sqrt{2}i|$ is $5\sqrt{2}$, then
  find the value of $p$.
  %262
\item Let a circle $C$ in complex plane passes through the points $z_1 = 3 + 4i, z_2 = 4 + 3i$ and $z_3 =
  5i$. If $z\neq z_1$ is a point on $C$ such that the line through $z$ and $z_1$ is perpendicular to the
  line through $z_2$ and $z_3$, then find $\arg(z)$.
  %263
\item Let $z_1$ and $z_2$ be two complex numbers such that $\overline{z_1} = i\overline{z_2}$ and
  $\arg\left(\frac{z_1}{\overline{z_2}}\right) = \pi$, then prove that $\arg(z_1) = \frac{\pi}{4}$.
  %264
\item Let $A = \left\{z\in\mathbb{C}: \left|\frac{z + 1}{z - 1}\right| < 1\right\}$ and $B =
  \left\{z\in\mathbb{C}: \arg\left(\frac{z - 1}{z + 1}\right) = \frac{2\pi}{3}\right\}$. Show $A\cap B$
  graphically.
  %265
\item If $z^2 + z + 1 = 0, z\in\mathbb{C}$, then find $\left|\displaystyle\sum_{n = 1}^{15}\left(z^n +
  (-1)^n\frac{1}{z^n}\right)^2\right|$.
  %266
\item Let the minimum value $\nu_0$ of $\nu = |z|^2 + |z - 3|^2 + |z - 6i|^2, z\in\mathbb{C}$ be attained at
  $z_0$. Find the value of $|2z_0^2 - \overline{z_0}^3 + 3|^2 + \nu_0^2$.
  %267
\item Let $S = \left\{z\in\mathbb{C}: z^2 + \overline{z} = 0\right\}$, then find $\displaystyle\sum_{\mstack z\in
  S}\Re(z) + \Im(z)$.
  %268
\item Let $S$ be the set of all $(\alpha, \beta), \pi < \alpha, \beta < 2\pi$, for which the complex number
  $\frac{1 - i\sin\alpha}{1 + 2i\sin\alpha}$ is prely imaginary and $\frac{1 + i\cos\beta}{1 - 2i\cos\beta}$
  is purely real. Let $z_{\alpha\beta} = \sin2\alpha + i\cos2\beta, (\alpha, \beta)\in S$, then find
  $\displaystyle\sum_{\mstack (\alpha, \beta)\in S}\left(iz_{\alpha\beta} +
  \frac{1}{i\overline{z}_{\alpha\beta}}\right)$.
  %269
\item Let $S_1 = \left\{z\in\mathbb{C}: |z_1 - 3| = \frac{1}{2}\right\}$ and $S_2 = \left\{z\in\mathbb{C}:
  \bigl|z_2 - |z_2 + 1|\bigr| = \bigl|z_2 + |z_2 - 1|\bigr|\right\}$, then find the least value of $|z_2 -
  z_1|$.
  %270
\item Let $z = a + ib, b\neq 0$ be complex numbers satisfying $z^2 = \overline{z}.2^{1 - |z|}$, then find
  the least value of $n\in\mathbb{N}$, such that $z^n = (z + 1)^n$.
  %271
\item Find the number of elements in the set $\left\{z = a + ib\in\mathbb{C}: a, b\in\mathbb{Z} \mathrm{ and
} 1 < |z - 3 + 2i| < 4\right\}$.
  %272
\item Let $A = \startbmatrix\NC1 + i\NC 1\NR -i\NC 0\NR\stopbmatrix$, then find the number of elements in
  the set $\left\{n\in \left\{1, 2, 3, \ldots, 100\right\}: A^n= A\right\}$.
  %273
\item Find the sum of squares of modullii of all the complex numbers $z$ satisfying $\overline{z} = iz^2 +
  z^2 - z$.
  %274
\item If $z\neq 0$ be a complex number such that $\left|z - \frac{1}{z}\right| = 2$, then find the maximum
  value of $|z|$.
  %275
\item Let $S = \left\{z = x + iy: |z - 1 + i|\geq |z|, |z| < 2, |z + i| = |z - 1|\right\}$, then find the
  set of values of $x$ for which $w = 2x + iy\in S$ for some $y\in\mathbb{R}$.
  %276
\item Let $S = \left\{z\in\mathbb{C}: |z - 2|\leq 1, z(1 + i) + \overline{z}(1 - i)\leq 2\right\}$. Let $|z
  - 4i|$ attains minimum and maximum values respectively at $z_1\in S$ and $z_2\in S$. If $5(|z_1|^2 +
  |z_2|^2) = \alpha + \beta\sqrt{5}$, where $\alpha, \beta$ are integers then find $\alpha + \beta$.
  %277
\item Let $\arg(z)$ represent the principal argument of the complex number $z$. Find the no.\ of points of
  intersection of $|z| = 3$ and $\arg(z - 1) - \arg(z + 1) = \frac{\pi}{4}$.
  %278
\item Let $z$ be a complex number with non-zero imaginary part. If $\frac{z + 3z + 4z^2}{2 - 3z + 4z^2}$ is
  a real number then find $|z|^2$.
  %279
\item Find the no.\ of distinct roots of the equation $\overline{z} - z^2 = i(\overline{z} + z^2)$, where
  $z$ is a complex number.
  %280
\item Let $a, b$ be two real numbers such that $ab < 0$. If the complex number $\frac{1 + ai}{b + i}$ is of
  unit modulus and $a + ib$ lies on the circle $|z - 1| = |2z|$, then find $\frac{1 + [a]}{4b}$, where $[t]$
  is greatest integer function.
  %281
\item Let $a\neq b$ be two non-zero real numbers. Find the no.\ of elements in the set $X
  = \left\{z\in\mathbb{C}: \Re(az^2 +bz) = a\mathrm{\;and\;} \Re(bz^2 + az) = b\right\}$.
  %282
\item Let $S = \left\{z\in\mathbb{C} - \left\{i, 2i\right\}: \frac{z^2 + 8iz - 15}{z^2 - 3iz -
  2}\in\mathbb{R}\right\}$. If $\alpha - \frac{13i}{11}\in S, \alpha\in\mathbb{R} - \left\{0\right\}$, then
  find $\alpha^2$.
  %283
\item Let $\omega = z\overline{z} = k_1z + k_2iz + \lambda(1 + i), k_1, k_2\in\mathbb{R}$. Let $\Re(\omega)
  = 0$ be the circle $C$ of radius $1$ in the first quadrant touching the line $y = 1$ and the $y$-axis. If
  the curve $\Im(\omega) = 0$ intersects $C$ at $A$ and $B$, then find $30(AB)^2$.
  %284
\item If the set $\left\{\Re\left(\frac{z - \overline{z} + z\overline{z}}{2 - 3z + 5\overline{z}}\right):
  z\in\mathbb{C}, \Re(z) = 3\right\}$ is equal to the interval$(\alpha, \beta]$, then find $\beta - \alpha$.
  %285
\item Let $z_1 = 2 + 3i$ and $z_2 = 3 + 4i$. Find the locus of the set of points given by $S
  = \left\{z\in\mathbb{C}: |z - z_1|^2 - |z - z_2|^2 = |z_1 - z_2|^2\right\}$.
  %286
\item Let $z$ be a complex number such that $\left|\frac{z - 2i}{z + i}\right| = 2, z\neq -i$, then find the
  locus of $z$.
  %287
\item Let $\alpha = 8 - 14i, A = \left\{z\in\mathbb{C}: \frac{\alpha z - \overline{\alpha}\overline{z}}{z^2
  - (\overline{z})^2 - 112i} = 1\right\}$ and $B = \left\{z\in\mathbb{C}: |z + 3i| = 4\right\}$. Find
  $\displaystyle\sum_{\mstack z\in A\cap B} \Re(z) - \Im(z)$.
  %288
\item Let $S = \left\{z\in\mathbb{C}: |z - 1| = 1 \mathrm{\;and\;} (\sqrt{2} - 1)(z + \overline{z}) - i(z
  - \overline{z}) = 2\sqrt{2}\right\}$. Let $z_1, z_2\in S$ such that $|z_1| = \max_{\mstack z\in S}|z|$ and
  $|z_2| = \min_{\mstack z\in S}|z|$, then find $|\sqrt{2}z_1 - z_2|^2$.
  %289
\item Let $P = \left\{z\in\mathbb{C}: |z + 2 - 3i|\leq 1\right\}$ and $Q = \left\{z\in\mathbb{C}: z(1 + i)
  + \overline{z}(1 - i)\leq -8\right\}$. Let in $P\cap Q, |z -3 + 2i|$ be maximum and minimum at $z_1$ and
  $z_2$ respectively. If $|z_1|^2 + 2|z_2|^2 = \alpha + \beta\sqrt{2}$, where $\alpha, \beta$ are integers,
  then find $\alpha + \beta$.
  %290
\item Find the area enclosed in the region $S = \left\{z\in\mathbb{C}: |z - 1|\leq 2; z + \overline{z} +
  i(z - \overline{z})\leq 2m \Im(z)\geq 0\right\}$.
  %291
\item Let $S_1 = \left\{z\in\mathbb{C}: |z|\leq 5\right\}, S_2 = \left\{z\in\mathbb{C}: \Im\left(\frac{z + 1
  - \sqrt{3}i}{1 - \sqrt{3}i}\right)\geq 0\right\}$ and $S_3 = \left\{z\in\mathbb{C}: \Re(z)\geq
  0\right\}$. Find the area of the region $S_1\cap S_2\cap S_3$.
  %292
\item If $1 + \frac{\sqrt{3} - \sqrt{2}}{2\sqrt{3}} + \frac{5 - 2\sqrt{6}}{18} + \frac{9\sqrt{3} -
  11\sqrt{2}}{36\sqrt{3}} + \frac{49 - 20\sqrt{6}}{180} + \cdots$ up to $\infty = 2\left(\sqrt{\frac{b}{a}}
  + 1\right)\log_e\frac{a}{b}$, where $a$ and $b$ are integers with gcd$(a, b) = 1$ then find $a$ and $b$.
  %293
\item Let $x_1, x_2, x_3, x_4$ be the solution of the equation $4x^4 + 8x^3 - 17x^2 - 12x + 9 = 0$ and $(4x
  + x_1^2)(4 + x_2)^2(4 + x_3^2)(4 + x_4)^2 = \frac{125}{16}m$, then find $m$.
  %294
\item Let $z$ be a complex number such that $|z + 2| = 1$ and $\Im\left(\frac{z + 1}{z + 2}\right)
  = \frac{1}{5}$, then find the value of $|\Re(\overline{z + 2})|$.
  %295
\item If the set $S = \left\{(a, b): a + 5b = 42, a, b\in\mathbb{N}\right\}$ has $n$ elements and
  $\displaystyle\sum_{n = 1}^m(1 + i^{n!}) = x + iy$, then find $m + x + y$.
  %296
\item Find the sum of the squares of modulus of the elements in the set $\left\{z = a + ib: a,
  b\in\mathbb{z}, z\in\mathbb{C}, |z - 1|\leq 1, |z - 5|\leq |z - 5i|\right\}$.
  %297
\item Let $z$ be a complex number such that $\frac{z - 2i}{z + 2i}$ is zero. Find the maximum value of $|z -
  (6 + 8i)|$.
  %298
\item If $S = \left\{z\in\mathbb{C}: |z - i| = |z + i| = |z - 1|\right\}$, then find $n(S)$.
  %299
\item Let the complex numbers $\alpha$ and $\frac{1}{\overline{\alpha}}$ lie on the circles $|z - z_0|^2 =
  4$ and $|z - z_0|^2 = 16$ respectively, where $z_0 = 1 + i$. Find the value of $100|\alpha|^2$.
  %300
\item If $z$ is a complex number then find the common roots of the equations $z^{1985} + z^{100} + 1 = 0$
  and $z^3 + 2z^2 + 2z + 1 = 0$.
  %301
\item If $\alpha$ denotes the number of solutions $|1 - i|^x = 2^x$ and $\beta = \frac{|z|}{\arg(z)}$, where
  $z = \frac{\pi}{4}(1 + i)^4\left(\frac{1 - \sqrt{\pi}i}{\sqrt{\pi} + i} + \frac{\sqrt{\pi} - i}{1
  + \sqrt{\pi}i}\right)$, then find the distance of the point $(\alpha, \beta)$ from the line $4x - 3y = 7$.
  %302
\item Let $z_1$ and $z_2$ be two complex numbers such that $z_1 + z_2 = 5$ and $z_1^3 + z_2^3 = 20 +
  15i$. Find $|z_1^4 + z_2^4|$.
  %303
\item If $z$ be a complex number such that $|z| = 1$. If $\frac{2 + k^2z}{k + \overline{z}} = kz,
  k\in\mathbb{R}$, then find the maximum distance of $k + ik^2$ from the circle $|z - (1 + 2i)| = 1$.
  %304
\item If $z\in\mathbb{C}$ be such that $\frac{z^2 + 3i}{z - 2 + i} = 2 + 3i$, then find the sum of all
  possible values of $z^2$.
  %305
\item If $z_1, z_2, z_3\in\mathbb{C}$ are the vertices of an equilateral triangle, whose centroid is $z_0$,
  then find $\displaystyle\sum_{k = 1}^3\left(z_k - z_0\right)^2$.
  %306
\item Let $A = \left\{z\in\mathbb{C}: |z - 2 - i| = 3\right\}, B = \left\{z\in\mathbb{C}: \Re(z -
  iz)\right\}$ and $S = A\cap B$. Find $\displaystyle\sum_{\mstack z\in S}|z|^2$.
  %307
\item Let the product of $\omega_1 = (8 + i)\sin\theta + (7 + 4i)\cos\theta$ and $\omega_2 = (1 +
  8i)\sin\theta + (4 + 7i)\cos\theta$ be $\alpha + i\beta$. Let $p$ and $q$ be the maximum and minimum
  values of $\alpha + \beta$, then find $p + q$.
  %308
\item If $\alpha$ is a root of the equation $x^2 + x + 1 = 0$ and $\displaystyle\sum{k = 1}^n\left(\alpha^k
  + \frac{1}{\alpha^k}\right)^2 = 20$, then find $n$.
  %309
\item Find the no.\ of elements in the set $S = \bigl\{z\in\mathbb{C} - \left\{-1, -i\right\}: |z| =
  1, \Re\left(\frac{z - 1}{z + 1} = 0\right)\bigr\}$.
  %310
\item If the locus of $z\in\mathbb{C}$ such that $\Re\left(\frac{z - 1}{2z + i}\right)
  + \Re\left(\frac{\overline{z} - 1}{2\overline{z} - i}\right) = 2$ is a circle, then find its center and
  radius.
  %311
\item Let $z_1, z_2$ and $z_3$ be three complex numbers on the circle $|z| = 1$ with $\arg(z_1) =
  -\frac{\pi}{4}, \arg(z_2) = 0$ and $\arg(z_3) = \frac{\pi}{4}$. If $|z_1z_2 + z_2z_3 + z_1z_3|^2 = \alpha
  + \beta\sqrt{2}, \alpha, \beta\in\mathbb{Z}$, then find the value of $\alpha^2 + \beta^2$.
  %312
\item Let the curve $z(1 + i) + \overline{z}(1 - i) = 4, z\in\mathbb{C}$ divide the region $|z - 3|\leq 1$
  into two parts of areas $\alpha$ and $\beta$, then find $|\alpha - \beta|$.
  %313
\item Let $\left|\frac{\overline{z} - i}{2\overline{z} + i}\right| = \frac{1}{3}, z\in\mathbb{C}$, be the
  equation of a circle with center $C$. If the area of the triangle whose vertices are at the points $(0,
  0), C$ and $(\alpha, 0)$ is $11$ sq.\ units, then find $\alpha^2$.
  %314
\item Find the number of complex numbers $z$, satisfying $|z| = 1$ and $\left|\frac{z}{\overline{z}}
  + \frac{\overline{z}}{z}\right| = 1$.
  %315
\item Let $\alpha, \beta$ be the roots of the equation $x^2 - ax - b = 0$ with $\Im(\alpha)
  < \Im(\beta)$. Let $P_n = \alpha^n - \beta^n$. If $P_3 = -5\sqrt{7}i, P_4 = -3\sqrt{7}i, P_5 =
  11\sqrt{7}i$ and $P_6 = 45\sqrt{7}i$, then find $\left|\alpha^4 + \beta^4\right|$.
  %316
\item If $\alpha$ and $\beta$ are the roots of the equation $2z^2 - 3z - 2i = 0$, then find
  $\Re\left(\frac{\alpha^{19} + \beta^{19} + \alpha^{11} + \beta^{11}}{\alpha^{15}
  + \beta^{15}}\right).\Im\left(\frac{\alpha^{19} + \beta^{19} + \alpha^{11} + \beta^{11}}{\alpha^{15}
  + \beta^{15}}\right)$.
  %317
\item Let $O$ be the origin, the point $A$ be $z_1 = \sqrt{3} + 2\sqrt{2}i$, the point $B(z_2)$ be such that
  $\sqrt{3}|z_2| = |z_1|$ and $\arg(z_2) = \arg(z_1) + \frac{\pi}{6}$. What kind of triangle these points
  constitute?
  %318
\item If $\alpha = 1 + \displaystyle\sum_{r = 1}^6(-3)^{r - 1}C_{2r - 1}^^{12}$, then find the distance of
  the point $\left(12, \sqrt{3}\right)$ from the line $\alpha x - \sqrt{3}y + 1 = 0$.
  %319
\item Let $|z_1 - 8 - 2i|\leq 1$ and $|z_2 - 2 + 6i|\leq 2$, then find $|z_1 - z_2|_{\min}$.
  %320
\item Let $a, b\in[-3, 3]$ be integers such that $a + b\neq 0$. Find the number of all ordered pairs $(a,
  b)$ for which $\left|\frac{z - a}{z + b}\right| = 1$ and $\startdeterminant\NC z +
  1\NC \omega\NC \omega^2 \NR\NC \omega\NC z + \omega^2 \NC 1\NR\NC \omega^2\NC 1\NC z
  + \omega\NR\stopdeterminant = 1$, where $\omega$ is a cube root of unity.
  %321
\item Let $\mathbb{R}$ denote the set of all real numbers. Let $z_1 = 1 + 2i$ and $z_2 = 3i$ be two complex
  mnumbers. Let $S = \left\{(x, y)\in\mathbb{R}\times\mathbb{R}: |x + iy - z_1| = 2|x + iy -
  z_2|\right\}$. Find the locus of of points in $S$.
  %322
\item If $l$ and $m$ are real, $l\neq m$, then find the nature of roots of the equation $(l - m)x^2
  + \frac{5}{2}(l + m)x - 2(l - m) = 0$.
  %323
\item Find the number of real solutions of the equation $|x|^2 - 3|x| + 2 = 0$.
  %324
\item Let $a\neq b$ be two non-zero real numbers. Find the no.\ of elements in the set $X =
  \left\{z\in\mathbb{C}: \Re(az^2 + bz) = a \mathrm{\;and\;} \Re(bz^2 + az) = b\right\}$.
  %325
\item If $a^2 + b^2 + c^2 = 1$ then find the range of values for $ab + bc + ca$.
  %326
\item Find the roots of $x^{3/4\left(\log_2x\right)^2 + \log_2x - 5/4} = \sqrt{2}$.
  %327
\item If $p, q, r, s$ are positive numbers then find the minimum values of $\frac{\left(p^2 + p +
  1\right)\left(q^2 + q + 1\right)\left(r^2 + r + 1\right)\left(s^2 + s + 1\right)}{pqrs}$.
  %328
\item If the roots of the equation $x^2 - bx + c = 0$ and $x^2 - cx + b = 0$ differ by same quantity, then
  find $b + c$.
  %329
\item Find the quadratic equation whose roots are the arithmetic and harmonic means between the roots of the
  equation $lx^2 + mx + n = 0$.
  %330
\item Find the values of $a$ for which the equation $2x^2 - \left(a^3 + 8a - 1\right)x + a^2 - 4a = 0$ has
  roots of opposite sign.
  %331
\item If $7 + 10x - 8x^2 = 0$, and $x$ be a real number, what is the maximum value of $x$?
  %332
\item If $\alpha$ and $\beta$ are two real roots of a quadratic equation such that $\alpha + \beta =
  2, \alpha^4 + \beta^4 = 272$, then find the quadratic equation.
  %333
\item Prove that both the roots of the equation $(x - a)(x - b) + (x - b)(x - c) + (x - c)(x - a) = 0$ are
  real.
  %334
\item Find the number of values of $k$ for which the equation $x^2 - 3x + k = 0$ has two distinct roots
  lying in the interval $(0, 1)$.
  %335
\item Suppose $a, b$ denote the distinct real roots of the quadratic polynomial $x^2 + 20x - 2020$ and
  suppose $c, d$ denote the distinct complex roots of the quadratic polynomial $x^@ - 20x + 2020$. Find the
  value of $ac(a - c) + ad(a - d) + bc(b - c) + bd(b - d)$.
  %336
\item Let $\lambda\neq 0, \lambda\in\mathbb{R}$. If $\alpha$ and $\beta$ are the roots of the equation, $x^2
  - x + 2\lambda = 0$, and $\alpha$ and $\gamma$ are the roots of the equation $3x^2 - 10x + 27\lambda = 0$,
  then find $\frac{\beta\gamma}{\lambda}$.
  %337
\item Find the set of all real values of $\lambda$ for which the quadratic equations $\left(\lambda^2 +
  1\right)x^2 - 4\lambda x + 2 = 0$ always have exactly one root in the interval $(0, 1)$.
  %338
\item If $\alpha$ and $\beta$ are the roots of the quadratic equation $x^2 + x\sin\theta - 2\sin\theta = 0,
  \theta\in\left(0, \frac{\pi}{2}\right)$, then find $\dfrac{\left(\alpha^{12} +
    \beta^{12}\right)}{\left(\alpha^{-12} + \beta^{-12}\right)(\alpha - \beta)^{24}}$.
  %339
\item If $m$ is chosen in the quadratic equation $\left(m^2 + 1\right)x^2 - 3x + \left(m^2 + 1\right)^2 = 0$
  such that the sum of its roots is greatest, then find the absolute difference of the cube of its roots.
  %340
\item If $\alpha$ and $\beta$ are the roots of the equation $x^2 - 2x + 2 = 0$, then find the least value of
  $n$ for which $\left(\frac{\alpha}{\beta}\right)^n = 1$.
  %341
\item Find the number of integral values of $m$ for which the equation $\left(1 + m^2\right)x^2 - 2(1 + 3m)x
  + 1 + 8m = 0$ has no real root.
  %342
\item Find the number of integral values of $m$ for which the quadratic expression $(1 + 2m)x^2 - 2(1 + 3m)x
  + 4(1 + m), x\in\mathbb{R}$, is always positive.
  %343
\item If $\lambda$ be the ratio of the roots of the quadratic equation $3m^2x^2 + m(m - 4)x + 2 = 0$, then
  find the least value of $m$ for which $\lambda + \frac{1}{\lambda} = 1$.
  %344
\item Find the value of $\lambda$ such that the sum of the squares of the roots of the quadratic equation
  $x^2 + (3 - \lambda)x + 2 = \lambda$ has the least value.
  %345
\item Find all possible integral values of $k$ for which the roots of the quadratic equation $6x^2 - 11x +
  k= 0$ are rational.
  %346
\item Let $\alpha$ and $\beta$ be two roots of the equation $x^2 + 2x + 2 = 0$, then find $\alpha^{15} +
  \beta^{15}$.
  %347
\item Let $S = \left\{x\in\mathbb{R}: x\geq 0\mathrm{\;and\;}2\left|\sqrt{x} - 3\right| +
  \sqrt{x}\left(\sqrt{x} - 6\right) + 6 = 0\right\}$. Find $S$.
  %348
\item If $\alpha, beta\in\mathbb{C}$ are the distinct roots of the equation $x^2 - x + 1 = 0$, then find
  $\alpha^{101} + \beta^{107}$.
  %349
\item For a positive integer $n$, if the quadratic equation $x(x + 1) + (x + 1)(x + 2) + \cdots + (x + n -
  1)(x + n) = 10n$ has two consecutive integral solutions, then find $n$.
  %350
\item Find the sum of all real values of $x$ saatisfying the equation $\left(x^2 - 5x + 5\right)^{\left(x^2
  + 4x - 60\right)} = 1$.
  %351
\item Let $-\frac{\pi}{6} < \theta < -\frac{\pi}{12}$. Suppose $\alpha_1$ and $\beta_1$ are the roots of the
  equation $x^2 - 2x\sec\theta + 1 = 0$, and $\alpha_2$ and $\beta_2$ are the roots of the equation $x^2 +
  2x\tan\theta - 1= 0$. If $\alpha_1 > \beta_1$ and $\alpha_2 > \beta_2$, find $\alpha_1 + \beta_2$.
  %352
\item Let $\alpha$ and $\beta$ be the roots of the equation $x^2 - 6x - 2 = 0$. If $a_n = \alpha^n
  - \beta^n$ for $n\geq 1$, then find the value of $\frac{a_{10} - 2a_8}{2a_9}$.
  %353
\item In the quadratic equation $p((x) = 0$ with real coefficients has purely imaginary roots. Then prove
  that $p[p(x)]$ has neiter real nor purely imaginary roots.
  %354
\item Let $\alpha$ and $\beta$ be the roots of the equation $x^2 + px + q = 0, p\neq 0$. If $p, q$ and $r$
  are in A.P.\ and $\frac{1}{\alpha} + \frac{1}{\beta} = 4$, then find the value of $|\alpha -\beta|$.
  %355
\item Let $p$ and $q$ be real numbers such that $p\neq 0, p^3\neq \pm q$. If $\alpha$ and $\beta$ are
  non-zero complex numbers satisfying $\alpha + \beta = -p$ and $\alpha^3 + \beta^3 = q$, then find the
  quadratic equation(s) having $\frac{\alpha}{\beta}$ and $\frac{\beta}{\alpha}$ as its roots.
  %356
\item If $a, b, c$ are the sides of a triangle $ABC $such that $x^2 - 2(a + b + c)x + 3\lambda(ab + bc + ca)
  = 0$ has real roots, then find $\lambda$.
  %357
\item If $\alpha$ and $\beta$ are the roots of the equation $x^2 + bx + c = 0$, where $c < 0 < b$, then
  prove that $0 < \alpha < \beta < |\alpha|$.
  %358
\item If $\alpha$ and $\beta$ are the roots of the equation $x^2 + px + q = 0$ and $\alpha^4, \beta^4$ are
  the roots of the equation $x^2 - rx + s = 0$, then find the nature of roots of $x^2 - 4qx + 2q^2 - r = 0$.
  %359
\item Find the no.\ of roots of the equation $x - \frac{2}{x - 1} = 1 - \frac{2}{x - 1}$.
  %360
\item Find the sum of all the roots of the equation $|x - 2|^2 + |x - 2| - 2 = 0$.
  %361
\item If the products of the roots of the equation $x^2 - 3kx + 2e^{2\log k} - 1 = 0$ is $7$, then find the
  values of $k$, for which the roots are real.
  %362
\item Find the coefficient of $x^{99}$ in the polynomial $(x - 1)(x - 2) \cdots (x - 100)$.
  %363
\item If $x^2 - 10cx - 11d = 0$ has roots $a$ and $b$, $x^2 - 10ax - 11b = 0$ has roots $c$ and $d$, then
  find $a + b + c + d$.
  %364
\item If $\alpha, \beta$ are the roots of $ax^2 + bx + c = 0, (a\neq 0)$ and $\alpha + \delta, \beta
  + \delta$ are the roots of $AX^2 + Bx + C = 0, (A\neq 0)$ for some constant $\delta$, then prove that
  $\frac{b^2 - 4ac}{a^2} = \frac{B^2 - 4AC}{A^2}$.
  %365
\item Let $f(x) = Ax^2 + Bx + C$, where $A, B, C$ are real numbers. Prove that if $f(x)$ is an integer
  whenever $x$ is an integer, then the numbers $2A, A + B$ and $C$ are all integers. Conversely, prove that
  if the numbers $2A, A + B, C$ are all integers, then $f(x)$ is an integer whenever $x$ is an integer.
  %366
\item Find the set of all solutions of the equation $2^{|y|} - \left|2^{y - 1} - 1\right| = 2^{y - 1} + 1$.
  %367
\item Find the set of all $x$ for which $\frac{2x}{2x^2 + 5x + 2} > \frac{1}{x + 1}$.
  %368
\item Let $P(4, -4)$ and $Q(9, 6)$ be two points on the parabola $y^2 = 4x$ and let $X$ be any point on the
  arc $POQ$ of this parabola, where $O$ is the vertex of the parabola, such that $\Delta PXQ$ is
  maximum. Find this area.
  %369
\item Consider the quadratic equation $(c - 5)x^2 - 2cx + (c - 4) = 0, \c\neq 5$. Let $S$ be the set of
  integral values of $c$ for which one root of the equation lies in the interval $(0, 2)$ and its other root
  lies in the interval $(2, 3)$, then find the no.\ of elements in the set $S$.
  %370
\item If both the roots of the quadratic equation $x^2 - mx + 4 = 0$ are real and distinct and they lie in
  the interval $[1, 5]$, find the range in which $m$ lies.
  %371
\item If $a\in\mathbb{R}$ and the equation $-3(x - [x])^2 + 2(x - [x]) + a^2 = 0$(where $[x]$ denotes
  greatest integer $\leq x$) has no integral solution, then find all possible values of $a$.
  %372
\item For all \quote{$x$}, $x^2 + 2ax + 10 - 3a > 0$, then find the interval in which $a$ lies.
  %373
\item If $b > 1$, then find the interval in which roots of the quadratic equation $(x - a)(x - b) - 1 = 0$
  lie.
  %374
\item If the roots of the equation $x^2 - 2ax + a^2 + a - 3 = 0$ are real and less than $3$, then find the
  interval in which $a$ must lie.
  %375
\item Let $f(x)$ be a quadratic equation which is positive for all $x\in\mathbb{R}$. If $g(x) = f(x) + f'(x)
  + f''(x)$, then find the interval in which $g(x)$ lies for all $x\in\mathbb{R}$.
  %376
\item If $x^2 + (a - b)x + (1 - a - b) = 0$, where $a, b\in\mathbb{R}$, then find the values of $a$ for
  which equation has real and unequal roots for all values of $b$.
  %377
\item Let $a, b, c$ be real. If $ax^2 + bx + c = 0$ has two real roots $\alpha$ and $\beta$, where $\alpha <
  - 1$ and $\beta > 1$, then show that $1 + \frac{c}{a} + \left|\frac{b}{a}\right| < 0$.
  %378
\item Find the smallest value of $k$, for which both the roots of the equation $x^2 - 8kx + 16\left(k^2 - k
  + 1\right) = 0$ area real, distinct and have a value of at least $4$.
  %379
\item Find the number of real roots of the equation $5 + \left|2^x - 1\right| = 2^x(2^x - 2)$.
  %380
\item Find all the pairs $(x, y)$ that satisfy the inequality $2^{\sqrt{\sin^2x - 2\sin x +
    5}}.\frac{1}{4\sin^2y}\leq 1$ and $\sin x = \left|\sin y\right|$.
  %381
\item Find the soluitions of the equation $\left|\sqrt{x} - 2\right| + \sqrt{x}\left(\sqrt{x} - 4\right) +
  2 = 0(x > 0)$.
  %382
\item Find the real number $k$ for which the equation $2x^3 + 3x + k = 0$ has two distinct real roots in
  $[0, 1]$.
  %383
\item Let $a, b, c$ are real numbers, $a\neq 0$. If $\alpha$ is a root of the equation $a^2x^2 + bx + c = 0,
  \beta$ is the root of $a^2x^2 - bx - c = 0$ and $0 < \alpha <\beta$, then prove that $a^2x^2 + 2bx + 2c =
  0$ has a root $\gamma$ such that $\alpha < \gamma < \beta$.
  %384
\item If $a + b + c = 0$, then prove that the quadratic equation $ax^2 + bx + c = 0$ has at least one root
  in $(0, 1)$.
  %385
\item Find the largest interval for which $x^{12} - x^{9} + x^4 - x + 1 > 0$.
  %386
\item Let $S$ be the set of all non-zero real numbers $\alpha$ such that the quadratic equation $\alpha x^2
  - x + \alpha = 0$ has two distinct real roots $x_1$ and $x_2$ satisfying the inequality $\left|x_1 -
  x_2\right| < 1$. Find the set $S$.
  %387
\item Let $a, b\in\mathbb{R}$ be such that the equation $ax^2 - 2bx + 15 = 0$ has a repeated root
  $\alpha$. If $\alpha$ and $\beta$ are the roots of the equation $x^2 -2bx + 21 = 0$, then find $\alpha^2 +
  \beta^2$.
  %388
\item Find the number of distinct real roots of the equation $x^5\left(x^3 - x^2 - x + 1\right) +
  x\left(3x^3 - 4x^2 - 2x + 4\right) - 1 = 0$.
  %389
\item Find the minimum value of the sum of squares of the roots of $x^2 + (3 - a) + 1 = 2a$.
  %390
\item Find the sum of the cubes of all the roots of the equation $x^4 - 3x^3 - 2x^2 + 3x + 1 = 0$.
  %391
\item If $p$ and $q$ be two real numbers such that $p + q = 3$ and $p^4 + q^4 = 369$, then find
  $\left(\frac{1}{p} + \frac{1}{q}\right)^{-2}$.
  %392
\item Let $f(x) = 2x^2 - x - 1$ and $S = \left\{n\in\mathbb{Z}: |f(n)| \leq 800\right\}$. Find the value of
  $\displaystyle\sum_{\mstack n\in S}f(n)$.
  %393
\item If $\alpha, \beta$ are the roots of the equation $x^2 - \left(5 + 3^{\sqrt{\log_35}} -
  5^{\log_53}\right)x + 3\left(3^{\sqrt[3]{\log_35}} - 5^{\sqrt[3]{\left(\log_53\right)^2}} - 1\right) = 0$,
  then find the equation whose roots are $\alpha + \frac{1}{\beta}$ and $\beta + \frac{1}{\alpha}$.
  %394
\item Find the number of distinct real roots of $x^4 - 4x + 1 = 0$.
  %395
\item if the sum of all the roots of the equation $e^{2x} - 11e^{2x} - 45e^{-x} + \frac{81}{2} = 0$ is
  $\log_eP$, then find $P$.
  %396
\item Find the sum of all real values of $x$ for which $\frac{3x^2 - 9x + 17}{x^2 + 3x + 10} = \frac{5x^2 -
  7x + 19}{3x^2 + 5x + 12}$.
  %397
\item Let $S = \left\{x\in[-6, 3] - \left\{-2, 2\right\}:\frac{|x + 3| - 1}{|x| - 2}\geq 0\right\}$ and $T =
  \left\{x\in\mathbb{Z}: x^2 - 7|x| + 9\leq 0\right\}$. Find the number of elements in $S\cap T$.
  %398
\item Let $\alpha, \beta$ be the roots of the equation $x^2 - \sqrt{2}x + \sqrt{6} = 0$ and
  $\frac{1}{\alpha^2} + 1, \frac{1}{\beta^2} + 1$ be the roots of the equation $x^2 + ax + b = 0$. Find the
  roots of the equation $x^2 - (a + b - 2)x + (a + b + 2) = 0$.
  %399
\item Find the number of real solutions of $x^7 + 5x^3 + 3x + 1 = 0$.
  %400
\item Find the number of real solutions of the equation $e^{4x} + 4e^{3x} - 58e^{2x} + 4e^x + 1 = 0$.
  %401
\item Let $f(x)$ be a quadratic polynomial such that $f(-2) + f(3) = 0$. If one of the roots of $f(x) = 0$
  is $-1$, thne find the sum of the roots of $f(x) = 0$.
  %402
\item Find the no.\ of elements in the set $S = \left\{x\in\mathbb{R}: 2\cos\left(\frac{x^2 + x}{6}\right) =
  4^x + 4^{-x}\right\}$.
  %403
\item Let $S = \left\{(x, y)\in\mathbb{N}\times\mathbb{N}: 9(x - 3)^2 + 16(y - 4)^2 \leq 144\right\}$ and $T
  = \left\{(x, y)\in\mathbb{R}\times\mathbb{R}: (x - 7)^2 + (y - 4)^2\leq 36\right\}$. Find $n(S\cap T)$.
  %404
\item Find the product of all positive real numbers satisfying the equation
  $x^{\left[16\left(\log_5x\right)^3 - 68\log_5x\right]} = 5^{-16}$.
  %405
\item Let $S = \left\{x: x\in\mathbb{R}\;\mathrm{and}\;\left(\sqrt{3} + \sqrt{2}\right)^{x^2 - 4}
  + \left(\sqrt{3} - \sqrt{2}\right)^{x^2 - 4} = 10\right\}$. Find $n(S) = 10$.
  %406
\item Find the integral values of $k$, for which one root of the equation $2x^2 - 8x + k = 0$ lies in the
  interval $(1, 2)$ and its other root lies in the interval $(2, 3)$.
  %407
\item Find the sum of all the roots of the equation $\left|x^2 - 8x + 15\right| = 2x -7$.
  %408
\item Let $\alpha, \beta, \gamma$ be the three roots of the equation $x^3 + bx + c = 0$. If $\beta\gamma = 1
  = -\alpha$, then $b^3 + 2c^3 - 3\alpha^3- 6\beta^3 -8\gamma^3$.
  %409
\item Let $m$ and $n$ be the numbers of real roots of the quadratic equations $x^2 - 12x +[x] + 31 = 0$ and
  $x^2 - 5|x  + 2| - 4 = 0$ respectively, where $[x]$ denotes the greatest integer $\leq x$. Find $m^2 + mn
  n^2$.
  %410
\item Find the number of elements in the set $S = \left\{n\in\mathbb{Z}: \left|n^2 - 10n + 19\right| <
  6\right\}$.
  %411
\item Find the number of integral solutions of $\log_{\left(x + \frac{7}{2}\right)}\left(\frac{x - 7}{2x -
  3}\right)^2\geq 0$.
  %412
\item Find the number of points where the curve $f(x) = e^{8x} - e^{6x}- 3e^{4x} - e^{2x} +
  1,\;x\in\mathbb{R}$ cuts $x$-axis.
  %413
\item Let $\alpha, \beta$ be the roots of the equation $x^2 + \sqrt{6}x + 3 = 0$. Find $\frac{\alpha^{23}
  + \beta^{23} + \alpha^{14} + \beta^{14}}{\alpha^{15} + \beta^{15} + \alpha^{10} + \beta^{10}}$.
  %414
\item Let $\alpha, \beta$ be the roots of equation $x^2 - \sqrt{2}x + 2 = 0$. Find $\alpha^{14}
  + \beta^{14}$.
  %415
\item Find the real roots of $x|x| - 5|x + 2| + 6 = 0$.
  %416
\item Find the number of real solutions of the equation $3\left(x^2 + \frac{1}{x^2}\right) - 2\left(x
  + \frac{1}{x}\right) + 5  = 0$.
  %417
\item Let $p, q\in\mathbb{R}$ and $\left(1 - \sqrt{3}i\right)^{200} = 2^{199}(p + iq)$. Find the equation
  whose roots are $p + q + q^2$ and $p - q + q^2$.
  %418
\item Let $S = \left\{\alpha: \log_2\left(9^{2\alpha - 4} + 13\right) - \log_2\left(\frac{5}{2}3^{2\alpha -
  4} + 1\right) = 2\right\}$. Find the maximum value of $\beta$ for which the equation $\displaystyle x^2 -
  2\left(\sum_{\mstack \alpha\in S}\alpha\right)^2x + \sum_{\mstack \alpha\in S}(\alpha + 1)^2\beta = 0$ has
  real roots.
  %419
\item Let $a\in\mathbb{R}$ and let $\alpha, \beta$ be the roots of the equation $x^2 + \sqrt[4]{60}x + a =
  0$. If $\alpha^4 + \beta^4 = -30$ then find the product of all possible values of $a$.
  %420
\item Let $\lambda\neq 0$ be a real number. Let $\alpha, \beta$ be the roots of the equation $14x^2 - 31x +
  3\lambda = 0$ and $\alpha, \gamma$ be the roots of the equation $35x^2 - 53x + 4\lambda = 0$. Find the
  equation whose roots are $\frac{3\alpha}{\beta}$ and $\frac{4\alpha}{\gamma}$.
  %421
\item Let $\alpha_1, \alpha_2, \ldots, \alpha_7$ be the roots of the equation $x^7 + 3x^5 - 13x^3 - 15x = 0$
  such that $\left|\alpha_1\right|\geq \left|\alpha_2\right|\geq \cdots \geq \left|\alpha_7\right|$. Find
  $\alpha_1\alpha_2 - \alpha_3\alpha_4 + \alpha_5\alpha_6$.
  %422
\item Find the number of real roots of the equation $\sqrt{x^2 - 4x + 3} + \sqrt{x^2 - 9} = \sqrt{4x^2 - 14x
  + 6}$.
  %423
\item Find $x\in\mathbb{R}$ for $e^{4x} + 8e^{3x} + 13e^{2x} - 8e^x + 1 = 0$.
  %424
\item Let $S = \left\{a\in\mathbb{R}: |2a - 1| = 3[a] + 2\left\{a\right\}\right\}$, where $[t]$ denotes the
  greatest integer less than or equal to $t$ and $\left\{t\right\}$ represents the fractional part of
  $t$. Find $\displaystyle\sum_{\mstack a\in S}a$.
  %425
\item Find the number of distinct real roots of the equation $|x||x + 2| - 5|x + 1| - 1 = 0$.
  %426
\item Let the maximum and minimum value of $\left(\sqrt{8x - x^2 - 12} - 4\right)^2 + (x - 7)^2,
  x\in\mathbb{R}$ be $M$ and $m$ respectively. Find $M^2 - m^2$.
  %427
\item Find the number of real solutions of the equation $x|x + 5| + 2|x + 7| - 2 = 0$.
  %428
\item Let $\alpha, \beta$ be roots of $x^2 + \sqrt{2}x - 8 = 0$. If $U_n = \alpha^n + \beta^n$, then find
  $\frac{U_{10} + \sqrt{2}U_9}{2U_8}$.
  %429
\item Find the number of distinct real roots of the equation $|x + 1||x + 3| - 4|x + 2| + 5 = 0$.
  %430
\item Let $\alpha, \beta$ be roots of the equation $x^2 + 2\sqrt{2}x - 1 = 0$. Find the quadratic equation
  whose roots are $\alpha^4 + \beta^4$ and $\frac{1}{10}\left(\alpha^6 + \beta^6\right)$.
  %431
\item Let $\alpha, \beta; \alpha > \beta$, be the roots of the equation $x^2 - \sqrt{2}x - \sqrt{3} =
  0$. Let $P_n = \alpha^n - \beta^n, n\in\mathbb{N}$. Find $\left(11\sqrt{3} - 10\sqrt{2}\right)P_{10}
  + \left(11\sqrt{2} + 10\right)P_{11} - 11P_{12}$.
  %432
\item Let $\alpha, \beta$ be the roots of the equation $x^2 - x + 2 = 0$ with $\Im(\alpha)
  > \Im(\beta)$. Find $\alpha^6 + \alpha^4 + \beta^4 - 5\alpha^2$.
  %433
\item Let $\alpha, \beta$ be the roots of the equation $x^2 - \sqrt{6}x + 3 = 0$ such that $\Im(\alpha)
  > \Im(\beta)$. Let $a, b$ be integers not divisible by $3$ and $n$ be a natural number such that
  $\frac{\alpha^{99}}{\beta} + \alpha^{98} = 3^n(a + ib)$. Find $n + a + b$.
  %434
\item Let $\alpha, \beta\in\mathbb{N}$ be roots of the equation $x^2 - 70x + \lambda = 0$, where
  $\frac{\lambda}{2}, \frac{\lambda}{3}\not\in\mathbb{N}$. If $\lambda$ assumes the minimum possible value
  then find $\frac{\left(\sqrt{\alpha - 1} + \sqrt{\beta -1}\right)(\lambda + 35)}{|\alpha - \beta|}$.
  %435
\item For $0 < c < b < a$, let $(a + b - 2c)x^2 + (b + c - 2a)x + (c + a - 2b) = 0$ and $\alpha\neq 1$ be
  one of its roots. Then prove that if $\alpha\in(-1, 0)$ then $b$ cannot be geomteric mean between $a$ and
  $c$. Also, prove that if $a\in(0, 1)$, then $b$ may be geometric mean between $a$ and $c$.
  %436
\item Let $S$ be the set of positive integral values of $a$ for which $\frac{ax^2 + 2(a + 1)x + 9a + 4}{x^2
  - 8x + 32}<0,\;\forall x\in\mathbb{R}$. Find the number of elements in set $S$.
  %437
\item Let $a = 3\sqrt{2}$ and $b = \frac{1}{\sqrt[6]{5}\sqrt{6}}$. If $x, y\in\mathbb{R}$ are such that $3x
  + 2y = \log_a(18)^{5/4}$ and $2x - y = \log_b\sqrt{1080}$, then find $4x + 5y$.
  %438
\item Let $P_n = \alpha^n + \beta^n, n\in\mathbb{N}$. If $P_{10} = 123, P_9 = 76, P_8 = 47$ and $P_1 = 1$,
  then find the quadratic equation having roots $\frac{1}{\alpha}$ and $\frac{1}{\beta}$.
  %439
\item If the set of all $a\in\mathbb{R} - \{1\}$, for which the roots of the equation $(1 - a)^2 + 2(a - 3)x
  + 9 = 0$ are positive is $(-\infty, -\alpha]\cup [\beta, \gamma)$, then find $2\alpha + \beta + \gamma$.
  %440
\item Let $\alpha$ and $\beta$ be the roots of $x^2 + \sqrt{3}x - 16 = 0$ and $\gamma$ and $\delta$ be the
  roots of $x^2 + 3x - 1 = 0$. If $P_n = \alpha^n + \beta^n$ and $Q_n = \gamma^n + \delta^n$, then find
  $\frac{P_{25} + \sqrt{3}P_{24}}{2P_{23}} + \frac{Q_{25} - Q_{23}}{Q_{24}}$.
  %441
\item Consider the equation $x^2 + 4x- n = 0$, where $n\in[20, 100]$ is a natural number. Find the number of
  all distinct value for which the given equation has integral roots.
  %442
\item Let the set of all values of $p\in\mathbb{R}$, for which both the roots of the equation $x^2 - (p +
  2)x + (2p + 9) = 0$ are negtive real numbers in the interval $(\alpha, \beta]$. Find $\beta - 2\alpha$.
  %443
\item Find the number of real roots of the equation $x|x - 2| + 3|x - 3| + 1 = 0$.
  %444
\item Find the sum of square of the roots of $|x + 2|^2 + |x - 2| - 2 = 0$ and the squares of the roots $x^2
  - 2|x - 3| - 5 = 0$.
  %445
\item Find the product of all solutions of the equation $e^{5\left(\log_ex\right)^2 + 3} = x^8, x > 0$.
  %446
\item If the equation $a(b - c)x^2 + b(c - a)x + c(a - b) = 0$ has equal roots, where $a + c = 15$ and $b
  = \frac{36}{5}$, then find $a^2 + c^2$.
  %447
\item The roots of the quadratic equation $3x^2 - px + q = 0$ are $10$^{th} and $11$^{th} terms of an A.P.\
  with c.d.\ $\frac{3}{2}$. If the sum of first $11$ terms of this A.P.\ is $88$, then find $q - 2p$.
  %448
\item Find the product of all rational roots of $\left(x^2 - 9x + 11\right)^2 - (x - 4)(x - 5) = 3$.
  %449
\item Let $A = \left\{x\in(0, \pi) - \left\{\frac{\pi}{2}\right\}: \log_{(2/\pi)}|\sin x|
  + \log_{(/\pi2)}|\cos x| = 2\right\}$ and $B = \left\{x\geq 0: \sqrt{x}\left(\sqrt{x} - 4\right) -
  3\left|\sqrt{x} - 2\right| + 6 = 0\right\}$. Find $n(A\cup B)$.
  %450
\item Find the sum of squares of all the roots of the equation $x^2 + |2x - 3| - 4 = 0$.
  %451
\item Find the number of solutions of the equation $\left(\frac{9}{x} - \frac{9}{\sqrt{x}} +
  2\right)\left(\frac{2}{x} - \frac{7}{\sqrt{x}} + 3\right) = 0$.
  %452
\item Ten different letters of an alphbet are given. Words with five letters are formed wth these given
  letters. Find the number of words which have at least one letter repeated.
  %453
\item Eight chairs are numbered $1$ to $8$. Two women and three men wish to occupy one chair each. First the
  women choose the chairs from amongst the chairs marked $1$ to $4$; and then the men select the chairs from
  amongst the remaining. Find the total no.\ of possible arrangements.
  %454
\item Find the number of numbers that can be formed by the digits $1, 2, 3, 4, 3, 2, 1$ with odd digits at
  odd places.
  %455
\item How many words can be made out of the word INDEPENDENCE in which vowels always come together?
  %456
\item In a circus there are ten cages for accommodating ten animals. Four of these cages are so small that
  five out of ten animals cannot enter them. How many ways are possible to accommodate the ten animals?
  %457
\item Find the number of triangles that can be formed by choosing the vertices from a set of $12$ points,
  $7$ of which lie on the same straight line.
  %458
\item There are $10$ bulbs in a room. Find the number of ways of illuminating the room.
  %459
\item Find the number of four-digit numbers strictly greater than $4321$ that can be formed using the digits
  $0, 1, 2, 3, 4, 5$(repitition allowed).
  %460
\item How many $3\times3$ matrices $M$ with entries from $\left\{0, 1, 2\right\}$ are there, for which the
  sum of diagonal entries of $M^TM$ is $5$?
  %461
\item Find the no.\ integers greater than $6000$ that can be formed using the digits $3, 5, 6, 7$ and $8$ without
  repetition.
  %462
\item Find the number of seven-digit integers, with sum of digits being equal to $10$ and formed by digits
  $1, 2$ and $3$ only.
  %463
\item How many different nine-digit numbers can be formed from the number $22\,33\,55\,888$ be rearranging
  its digit such that the odd digits occupy the even positions?
  %464
\item Nine hundred distinct $n$-digit numbers are to be formed using only the three digits $2, 5$ and
  $7$. Find the smallest $n$ for which it is possible?
  %465
\item In a college of $300$ students, every student reads $5$ newspapers and every newspaper is read by $60$
  students. Find the number of newspapers.
  %466
\item A five digit number divisible by $3$ is to be formed using the numbers $0, 1, 2, 3, 4$ and $5$ without
  repetition. Find the total number of such numbers.
  %467
\item Eeighteen guests have to be seated half on each side of a table. Four particular guests desire to sit
  on one particular side and three other on other side. Find the number of seating arrangements.
  %468
\item There are $3$ sections in a question paper and each section contains $5$ questions. A candidate has to
  answer a total of $5$ questions, choosing at least one question from each section. Find the number of ways
  in which a student can answer the question paper.
  %469
\item Find the number of ways of choosing $10$ objects out of $31$ objects of which $10$ are identical and
  the remaining $21$ are distinct.
  %470
\item Suppose that $20$ pillars of same height have been erected along the boundary of a circular
  stadium. If the top of each pillar has been connected by beams with the top of its non-adjacent pillars,
  then find the total no.\ of beams.
  %471
\item Some identical balls are arranged in rows to form an equilateral triangle. The first row consists of
  one ball, the second row consists of two balls and so on. If $99$ more identical balls are added to the
  total no.\ of balls used in forming the triangle, then all these balls can be arranged in a square whose
  each side contains $2$ balls less than the number of balls each side of the triangle contains. Find the
  number of balls used to form the equilateral triangle.
  %472
\item There are $m$ men and two women participating in a chess tournament. Each participant plays two games
  with every other participant. If the number of games played by the men between the men exceeds the number
  of games played between the men and women by $84$, then find $m$.
  %473
\item If $\displaystyle\sum_{r = 0}^{25}C_r^^{50}.C_{25 - r}^^{50 - r} = K.C_{25}^^{50}$, then find $K$.
  %474
\item If $\displaystyle\sum_{i = 1}^{20}\left(\frac{C_{i - 1}^^{20}}{C_i^^{20} + C_{i - 1}^^{20}}\right)^3 =
  \frac{k}{21}$, then find $k$.
  %475
\item A man $X$ has $7$ friends, $4$ of them are ladies and $3$ are men. His wife $Y$ also has $7$ friends,
  $3$ of them are ladies and $4$ are men. Assume that $X$ and $Y$ have no common friends. Find the total
  no.\ of ways in which $X$ and $Y$ together can throw a party inviting $3$ ladies and $3$ men, so that $3$
  friends of each of $X$ and $Y$ are invited.
  %476
\item Let $S = \left\{1, 2, 3, \ldots, 9\right\}$. For $k = 1, 2, 3, 4, 5$, let $N_k$ be the number of
  subsets of $S$, each containing five elements out of which exactly $k$ are odd. Find $N_1 + N_2 + N_3 +
  N_4 + N_5$.
  %477
\item A debate club consists of $6$ girls and $4$ boys. A team of $4$ members is to be selected from this
  club including the selection of the captain(from among the $4$ members) for the team. If the team has to
  include at most one boy, find the number of ways of making the team.
  %478
\item Let $T_n$ be the number of all possible triangles formed by joining the vertices of an $n$ sided
  regular polygon.If $T_{n + 1} - T-n = 10$, find $n$.
  %479
\item If $r, s, t$ are prime numbers and $p, q$ are the positive integers such that L.C.M.\ of $p, q$ is
  $r^2s^4t^2$, then find the number of ordered pairs $(p, q)$.
  %480
\item Five persons $A, B, C, D$ and $E$ are seated in a circular arragement. If each of them is given a hat
  of one of the three colors: red, blue and green, then find the number of ways of distributing the hats
  such that the person seated in adjacent seats get hats of different colors.
  %481
\item Let $n > 2$ be an integer. Take $n$ distinct points on a circle and join each pair of points by a line
  segment. Color the line segment joining every pair of adjacent points by the blue and the rest by red. If
  the no.\ of red lines is same as the no.\ of blue lines, find $n$.
  %482
\item In a certain test $a_i$ students gave wrong answers to at least $i$ questions, where $i = 1,
  2, \ldots, k$. No student gave more than $k$ wrong answers. Find the total no.\ of wrong answers.
  %483
\item Find the number of six-digit numbers that can be formed using the digits $0, 1, 2, 5, 7$ and $9$,
  which are divisible by $11$ with no repetition.
  %484
\item A committee of $11$ members is to be formed from $8$ males and $5$ females. If $m$ is number of
  committees can be formed with at least $6$ males and $n$ is the number of committees formed with at least
  $3$ females, then find $m$ and $n$.
  %485
\item Consider three boxes, each containing $10$ balls labelled $1, 2, \ldots, 10$. Suppose one ball is
  randoly drawn  from each of the boxes. Denoted by $n_i$, the label of the ball drawn from the $i$th box($i
  = 1, 2, 3$). Find the number of ways in which the balls can be chosen such that $n_1 < n_2 < n_3$.
  %486
\item Find the number of natural numbers less than $7000$ which can be formed by using the digits $0, 1, 3,
  7, 9$(repitition of digits allowed).
  %487
\item Consider a class of $5$ girls and $7$ boys. Find the number of different teams consisting of $2$ girls
  and $3$ boys that can be formed from this class, if there are two specific boys refuse to be members of
  the same team.
  %488
\item An engineer is required to visit a factory for exactly four days during the first $15$ days of every
  month and it is mandatory that no two visits take place on consecutive days. Find the number of all
  possible ways in which such visits to the factory can be made by the engineer during any month.
  %489
\item Words of length $10$ are formed using the letters $A, B, C, D, E, F, G, H, I$ and $J$. Let $x$ be the
  number of such words where no letter is repeated and $y$ be the number of words where exactly one letter
  is repeated. Find $x$ and $y$.
  %490
\item Let $n$ be the number of ways in which $5$ boys and $5$ girls can stand in a queue in such a way that
  all the girls stand consecutively in the queue. Let $m$ be the number of ways in which $4$ girls stand
  consecutively in the queue. Find $\frac{m}{n}$.
  %491
\item Let $n_1 < n_2 < n_3 < n_4 < n_5$ be positive integers such that $n_1 + n_2 + n_3 + n_4 + n_5 =
  20$. Find the number of distinct arrangements.
  %492
\item Let $n$ and $k$ be positive integers such that $n\geq \frac{k(k + 1)}{2}$. Find the number of
  solutions $(x_1, x_2, \ldots, x_k), x_1\geq 1, x_2\geq 2, \ldots, x_k\geq k$ for all integers satisfying
  $x_1 + x_2 + \cdots = x_k = n$.
  %493
\item Let $S$ be the set of all triangles in the $xy$-plane, each having one vertex at the origin and the
  orher two vertices lie on the coordinate axes. If each triangle in $S$ has area $50$ sq.\ units, then find
  the no.\ of elements of $S$.
  %494
\item From $6$ different novels and $3$ different dictionaries, $4$ novels and $1$ dicitonary are to be
  selected and arranged in a row on the shelf, so that dictionary is always in the middle. Find the number
  of such arrangements.
  %495
\item Using permutation or othereise, prove that $\frac{n^2!}{(n!)^n}$ is an integer, where  $n$ is a
  positive integer.
  %496
\item Find the number of divisors of the form $(4n + 2), n\geq 0$ of the integer $240$.
  %497
\item There are four balls of different colors and four boxes of same colors. Find the number of ways in
  which the balls, one each in the box could be placed such that a ball does not go to a box of its own
  color.
  %498
\item The letters of thw word \quote{MANKIND} are written in all possible orderd and arranged in serial
  order as in an English dictionary. Find the position of the word \quote{MANKIND}.
  %499
\item Find the number of $3$-digit odd numbers, whose sum of digits is a multiple of $7$.
  %500
\item Find the total number of $3$-digit numbers with one digit repeated exactl two times.
  %501
\item Find the number of $5$-digit numbers, product of whose digits is $36$.
  %502
\item Find the number of numbers between $1000$ and $3000$, which are divisible by $4$, using the digits
  $1-6$ without repetition of digits.
  %503
\item There are ten boys and give girls in a class. Find the number of groups of three boys and three girls,
  which can be made if two particular boys are never together in the group.
  %504
\item Find the total number of $3$-digit numbers, whose GCD with $36$ is $2$.
  %505
\item There are $16$ cubes of which $11$ are blue and $5$ are red. Find the number of ways  in which these
  can be arranged in a row such that between any two red cubes there are at least two blue cubes.
  %506
\item Let $S$ be the set of all passwords which are six to eight character long, where each character is
  either an alphabet from $\left\{A, B, C, D, E\right\}$ or a number from $\left\{1, 2, 3, 4, 5\right\}$
  with repetition of character being allowed. Find the number of passwords in $S$ whose at least one
  character is from $\left\{1, 2, 3, 4, 5\right\}$.
  %507
\item Find the number of ways to distribute $30$ identical candies among four children $C_1, C_2, C_3$ and
  $C_4$ such that $C_2$ receives at least $4$ and at most $7$ candies while $C_3$ receives at least $2$ and
  at most $6$ candies.
  %508
\item Find $\displaystyle\sum_{k = 1}^{10}k^2\left(C_k^^{10}\right)^2$.
  %509
\item Let $b_1b_2b_3b_4$ be a $4$-element permutation with $b_i\in\left\{1, 2, 3, \ldots, 100\right\}$ for
  $1\leq i\leq 4$ and $b_i\neq b_j$ for $i\neq j$, such that either $b_1b_2b_3$ are consecutive or
  $b_2b_3b_4$ are consecutive. Find the number of such permutations.
  %510
\item Find the no.\ of four digit numbers such that first three digits are divisible by the fourth digit.
  %511
\item Find the no.\ of four digit numbers in the range $[2022, 4482]$ formed by using the digits $0, 2, 3,
  4, 6, 7$.
  %512
\item Consider $4$ boxes, where each box contains $3$ red balls and $2$ blue balls. Assume that all $20$
  balls are distinct. In how many different ways can $10$ balls be chosen from these $4$ boxes so that from
  each box at least one red ball and one blue ball is chosen?
  %513
\item Find the number of $3$-digit numbers that are divisible by either $2$ or $3$ but not by $7$.
  %514
\item Find the total no.\ of numbers, formed using only the digits $4, 5, 9$, divisible by $6$.
  %515
\item Find the number of ways of giving $20$ distinct oranges to $3$ children such that each child gets one
  orange.
  %516
\item All the letters of the word \quote{PUBLIC} are written in all possible orders and these words are
  written as in a dictionary with serial numbers. Find the serial number of the word \quote{PUBLIC}.
  %517
\item Find the number of ways, in which 5 girls and 7 boys can be seated at a round table so that no two
  girls sit together.
  %518
\item Find the largest natural number $n$ such that $3n$ divides $66!$.
  %519
\item If the number of words, with or without meaning, which can be made using all the letters of the word
  \quote{MATHEMATICS} in which \quote{C} and \quote{S} do not come together.
  %520
\item Some couples participated in a mixed doubles badminton tournament. If the number of matches played, so
  that no couple plays together in a match, is $840$, then find the total numbers of persons, who
  participated in the tournament.
  %521
\item Find the sum of all the four-digit numbers that can be formed using all the digits $2, 1, 2, 3$.
  %522
\item In an examination, $5$ students have been allotted their seats as per their roll numbers. Find the
  number of ways, in which none of the students sits on the allotted seat.
  %523
\item If the letters of the word \quote{MATHS} are permuted and all possible words so formed are arranged as
  in a dictionary with serial number, then find the serial number of the word \quote{THAMS}.
  %524
\item Find the number of five-digit numbers, greater than $40000$ and divisible by $5$, which can be formed
  using the digits $0, 1, 3, 5, 7, 9$ without repetition.
  %525
\item Find the number of seven digit positive integers formed using the digits $1, 2, 3$ and $4$ only and
  sum of the digits equal to $12$.
  %526
\item All words, with or without meaning, are made using all the letters of the word \quote{MONDAY}. These
  words are written as in a dictionary with serial numbers. Find the serial number of the
  word \quote{MONDAY}.
  %527
\item Find the total numbers of $3$-digit numbers that are divisible by $6$ and can be formed by using the
  digits $1, 2, 3, 4, 5$ with repetition allowed.
  %528
\item Find the total number of three-digit numbers, divisible by $3$, which can be formed using the digits
  $1, 3, 5, 8$, if repetition of digits is allowed.
  %529
\item A person forgets his $4$-digit ATM pin code. But he remembers that in the code all the digits are
  different, the greatest digit is $7$ and the sum of the first two digits is equal to the sum of the last
  two digits. Then the maximum number of trials necessary to obtain the correct code.
  %530
\item Find the number of integers, greater than $7000$ that can be formed, using the digits $3, 5, 6, 7, 8$
  without repetition.
  %531
\item The number of $9$ digit numbers, that can be formed using all the digits of the number $123412341$ so
  that the even digits occupy only even place.
  %532
\item A boy needs to select five courses from $12$ available courses, out of which $5$ courses are language
  courses. If he can choose at most two language courses, then find the number of ways he can choose five
  courses.
  %533
\item Let $S = \left\{1, 2, 3, 5, 7, 10, 11\right\}$. The number of non-empty subsets of $S$ that have the
  sum of all elements a multiple of $3$.
  %534
\item Suppose Anil's mother wants to give $5$ whole fruits to Anil from a basket of $7$ red apples, $5$
  white apples and $8$ oranges. If in the selected $5$ fruits, at least $2$ orange, at least one red apple
  and at least one white apple must be given, find the number of ways, Anil's mother can give $5$ fruits to
  Anil.
  %535
\item If all the six digit numbers $x_1, x_2, x_3, x_4, x_5, x_6$ with $0 < x_1 < x_2 < x_3 < x_4 < x_5 <
  x_6$ are arranged in the increasing order, then find the sum of the digits in the $72nd$ number.
  %536
\item Find the number of $3$ digit numbers, that are divisible by either $3$ or $4$ but not divisible by
  $48$.
  %537
\item Find the total number of $4$-digit numbers whose greatest common divisor with $54$ is $2$.
  %538
\item Find the number of $4$-digit numbers (the repetition of digits is allowed) which are made using the
  digits $1, 2, 3$ and $5$, and are divisible by $15$.
  %539
\item Find the number of ways of selecting two numbers $a$ and $b, a\in\left\{2, 4, 6,\ldots,100\right\}$
  and $b\in \left\{1,3,5,\ldots,99\right\}$ such that $2$ is the remainder when $a + b$ is divided by $23$.
  %540
\item The number of seven digits odd numbers, that can be formed using all the seven digits $1, 2, 2, 2, 3,
  3, 5$.
  %541
\item Let $5$-digit numbers be constructed using the digits $0, 2, 3, 4, 7, 9$ with repetition allowed, and
  are arranged in ascending order with serial numbers. Find the serial number of the number $42923$.
  %542
\item Find the number of $4$-digit numbers that are less than or equal to $2800$ and either divisible by $3$
  or by $11$.
  %543
\item Let $A = \left[a_{ij}\right], a_{ij}\in Z\cap[0, 4], 1\leq i, j\leq 2$. The number of matrices $A$
  such that the sum of all entries is a prime number $p\in(2, 13)$.
  %544
\item If $n$ is the number of ways five different employees can sit into four distinguishable offices where
  any office may have any number of persons including $0$, then find $n$.
  %545
\item There are $4$ men and $5$ women in Group A, and $5$ men and $4$ women in Group B. If $4$ persons are
  selected from each group, then find the number of ways of selecting $4$ men and $4$ women.
  %546
\item $60$ words can be made using all the letters of the word \quote{BHBJO}, with or without meaning. If these
  words are written as in a dictionary, then find the $50$-th word.
  %547
\item Let $A = \left\{n \in[100, 700]\cap N: n\;\mathrm{is neither a multiple of 3 nor a multiple of
  4}\right\}$. Then find the number of elements in $A$.
  %548
\item Find the number of triangles whose vertices are at the vertices of a regular octagon but none of whose
  sides is a side of the octagon.
  %549
\item If all the words with or without meaning made using all the letters of the word \quote{NAGPUR} are
  arranged as in a dictionary, then find the word at $315$-th position.
  %550
\item Find the number of $3$-digit numbers, formed using the digits $2, 3, 4, 5$ and $7$, when the
  repetition of digits is not allowed, and which are not divisible by $3$.
  %551
\item Find the number of ways five alphabets can be chosen from the alphabets of the word \quote{MATHEMATICS},
  where the chosen alphabets are not necessarily distinct.
  %552
\item Find the number of integers, between $100$ and $1000$ having the sum of their digits equals to $14$.
  %553
\item Let $\alpha = \frac{(4!)!}{(4!)^{3!}}$ and $\beta = \frac{(5!)!}{(5!)^{5!}}$, then prove that
  $\alpha,\beta\in\mathbb{N}$.
  %554
\item Let $\alpha = \displaystyle\sum_{k = 0}^n\left(\dfrac{\left(C_k^^n\right)^2}{k + 1}\right)$ and $\beta
  = \displaystyle\sum_{k = 0}^n\left(\dfrac{C_k^^n.C_{k + 1}^^n}{k + 2}\right)$. If $5\alpha = 6\beta$, find
  $n$.
  %555
\item Find the total number of words (with or without meaning) that can be formed out of the letters of the
  word \quote{DISTRIBUTION} taken four at a time.
  %556
\item Find the number of ways in which $$21$$ identical apples can be distributed among three children such
  that each child gets at least $2$ apples.
  %557
\item A group of $9$ students $s_1, s_2, \ldots, s_9$ is to be divided to form three teams $X, Y$ and $Z$ of
  sizes $2, 3$ and $4$ respectively. Suppose that $s_1$ cannot be selected for the team $X$ and $s_2$ cannot
  be selected for the team $Y$. Find the number of ways of forming such teams.
  %558
\item There are three rows of boxes containing $8$ boxes. Row $1$ and $2$ contains $3$ boxes, while row $3$
  contains $2$ boxes. Letters $A, B, C, D, E$ are to be placed in these boxes such that no row remains
  empty. One box can hold at most $1$ letter. Find the no.\ of ways.
  %559
\item From a group of $7$ batsmen and $6$ bowlers, $10$ players are to be chosen for a team, which should
  include atleast $4$ batsmen and atleast $4$ bowlers. One batsmen and one bowler who are captain and
  vice-captain respectively of the team should be included. Then find the total number of ways such a
  selection can be made.
  %560
\item From all the English alphabets, five letters are chosen and are arranged in alphabetical order. Find
  the total number of ways, in which the middle letter is \quote{M}.
  %561
\item In a group of $3$ girls and $4$ boys, there are two boys $B1$ and $B2$. Find the number of ways, in
  which these  girls and boys can stand in a queue such that all the girls stand together, all the boys
  stand together, but $B1$ and $B2$ are not adjacent to each other.
  %562
\item The number of words, which can be formed using all the letters of the word \quote{DAUGHTER}, so that
  all the vowels never come together.
  %563
\item Find the number of ways, $5$ boys and $4$ girls can sit in a row so that either all the boys sit
  together or no two boys sit together.
  %564
\item Find the number of $3$-digit numbers, that are divisible by $2$ and $3$, but not divisible by $4$ and
  $9$.
  %565
\item Let $S$ be the set of all seven-digit numbers that can be formed using the digits $0, 1$ and $2$.
  For example, $2210222$ is in $S$, but $0210222$ is NOT in $S$. Then find the number of elements
  $x$ in $S$ such that at least one of the digits $0$ and $1$ appears exactly twice in $x$.
  %566
\item If the constant term in the binomial expansion of $\left(\sqrt{x} - \frac{k}{x^2}\right)^{10}$ is
  $405$, then find $|k|$.
  %567
\item If $\alpha$ and $\beta$ be the coefficients of $x^4$ and $x^2$ respectively in the expansion of
  $\left(x + \sqrt{x^2 - 1}\right)^6 + \left(x - \sqrt{x^2 - 1}\right)^6$, then find $\alpha, \beta$.
  %568
\item Find the coefficient of $x^{18}$ in the product $(1 + x)(1 - x)^{10}\left(1 + x + x^2\right)^9$.
  %569
\item If the coefficients of $x^2$ and $x^3$ are both zero, in the expansion of the expression $\left(1 +
  ax + bx^2\right)(1 - 3x)^{15}$ in the powers of $x$, then find the ordered pairs of $(a, b)$.
  %570
\item Find the term independent of $x$ in $\left(\frac{1}{60} - \frac{x^8}{81}\right)\left(2x^2
  - \frac{3}{x^2}\right)^6$.
  %571
\item Find the smallest natural number $n$, such that the coefficient of $x$ in the expansion of $\left(x^2
  + \frac{1}{x^3}\right)^n$ is $C_{23}^n$.
  %572
\item If some three consecutive coefficients in the binomial expansion of $(x + 1)^n$ in powers of $x$ are
  in the ratio of $2:15:70$, then find the average of these three coefficients.
  %573
\item If the fourth term in the binomial expansion of $\left(\frac{2}{x} + x^{\log_8x}\right)^6\;(x > 0)$ is
  $20\times 8^7$, then find the values of $x$.
  %574
\item If the fourth term in the binomial expansion of $\left(\sqrt{x^{\left(\frac{1}{1
      + \log_{10}x}\right)}} + x^{\frac{1}{12}}\right)^6$ is equal to $200$, and $x > 1$, then find the
  value of $x$.
  %575
\item Find the sum of coefficients of all even degree terms in the expansion of $\left(x + \sqrt{x^3 -
  1}\right)^6 + \left(x - \sqrt{x^3 - 2}\right)^6, (x > 1)$.
  %576
\item Find the total number of irrational terms in the binomial expansion of $\left(7^{1/5} -
  3^{1/10}\right)^{60}$.
  %577
\item Find the ratio of the $5$-th term from the beginning to the $5$-th term from the end in the binomial
  expansion of $\left(2^{\frac{1}{3}} + \frac{1}{2(3)^{\frac{1}{3}}}\right)^{10}$.
  %578
\item Find the sum of real values of $x$ for which the middle term of the expansion $\left(\frac{x^3}{3}
  + \frac{3}{x}\right)^8$ equals $5670$.
  %579
\item Find the positive value of $\lambda$ for which the coefficient of $x^2$ in the expression
  $x^2\left(\sqrt{x} + \frac{\lambda}{x^2}\right)^{10}$ is $720$.
  %580
\item If the third term in the binomial expansion of $\left(1 + x^{\log_2x}\right)^5$ equals 2580, then find
  possible values of $x$.
  %581
\item Find the coefficient of $t^4$ in the expression of $\left(\frac{1 - t^6}{1 - t}\right)^3$.
  %582
\item Find the sum of the coefficients of all odd degree terms in the expansion of $\left(x + \sqrt{x^3 -
  1}\right)^5 + \left(x - \sqrt{x^3 - 1}\right)^5,\;(x > 1)$.
  %583
\item Find the value of $\left(C_1^^{21} - C_1^^{10}\right) + \left(C_2^^{21} - C_2^^{10}\right)
  + \left(C_3^^{21} - C_3^^{10}\right) + \cdots + \left(C_{10}^^{21} - C_{10}^^{10}\right)$.
  %584
\item If the number of terms in the expansion of $\left(1 - \frac{2}{x} + \frac{4}{x^2}\right), x\neq 0$ is
  $28$, then find the sum of the coefficients of all the terms in the expansion.
  %585
\item Find the sum of coefficients of integral powers of $x$ in the binomial expansion of $\left(1 -
  2\sqrt{x}\right)^{50}$.
  %586
\item Find the coefficients of $x^{11}$ in the expansion of $\left(1 + x^2\right)^4\left(1 +
  x^3\right)^7\left(1 + x^4\right)^{12}$.
  %587
\item Find the term, independent of $x$, in the expansion of $\left(\frac{x + 1}{x^{2/3} - x^{1/3} + 1}
  - \frac{x - 1}{x - x^{1/2}}\right)$.
  %588
\item Find the coefficient of $t^{24}$ in $\left(1 + t^2\right)^{12}\left(1 + t^{12}\right)\left(1 +
  t^{24}\right)$.
  %589
\item In the binomial expansion of $(a - b)^n, n\geq 5$ the sum of the $5$-th and $6$-th terms is zero. Find
  $a/b$.
  %590
\item If in the expansion of $(1 + x)^m(1 - x)^n$, the coefficients of $x$ and $x^2$ are $3$ and $-6$
  respectively, then find $m$ and $n$.
  %591
\item Find the degree of the polynomial $\left[x + \sqrt{x^3 - 1}\right]^5 + \left[x - \sqrt{x^3 -
    1}\right]^5$.
  %592
\item Find the coefficient of $x^4$ in $\left(\frac{x}{2} - \frac{3}{x^2}\right)^{10}$.
  %593
\stopitemize
