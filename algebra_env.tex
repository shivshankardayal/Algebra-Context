\startenvironment algebra_env.tex
\setupbackend
  [format=PDF/X-4p,
   intent={ISO Coated v2 (ECI)},
   profile={sRGB.icc,default_gray.icc}]
  %\mainlanguage[en] % Main language = English
  \definefontfamily[mf][rm][modern]
  \definefontfamily[mf][ss][roboto]
  \definefontfamily[mf][tt][modern]
  \definefontfamily[mf][mm][modern]
  \setupbodyfont[9pt,mf] % The document's main (body) font
  \setupmathematics[autopunctuation=yes]

  % Page layout

  \setuplayout
    [
      width=middle,
      height=middle,
      topspace=1.5cm,
      backspace=2.5cm,
      header=1cm,
      headerdistance=.5cm,
      footer=1.7cm
    ]
  \definecolor [maincolor] [r=.0,g=.4,b=.6]
  \definecolor [purple] [r=.0,g=0.4,b=.6]

  \setupitemize[symalign=left, distance=5pt, color=maincolor]

  \setuppapersize[B5]
  \setupexternalfigures [directory=images, location=global]

  \setupwhitespace[big]
  \def\startequation{%
    \placeformula
    \startformula}
  \def\stopequation{%
    \stopformula}

  \expandafter\def\csname startequation*\endcsname{%
    \placeformula[-]
    \startformula}
  \expandafter\def\csname stopequation*\endcsname{%
    \stopformula}

  \setupcaptions [headstyle=\bfx, style=\tfx]



  % Writing commands and code snippets
  \setuptype [color=maincolor]
  \setuptyping
    [color=maincolor,style={\switchtobodyfont[small,tt]}]

  % Labels
  \setuplabeltext
    [en]
    [
      chapter=Chapter~,
      appendix=Appendix~,
      part=Part~
    ]

  % Footnotes
  \setupnotation
    [footnote]
    [way=bypage, numberconversion=n, after={\blank[small]}]

  % Page numbers
  \setuppagenumbering [location={header, right}]
  %\setupuserpagenumber [number=0]

  % Empty initial footers
  \setupfootertexts[]

  \definetext[ChapterFirstPage]
    [footer]
    [pagenumber]

  \setuphead
    [part]
    [
      placehead=yes,
      alternative=middle,
      conversion=I,
      style={\switchtobodyfont[30pt, ss]\bf},
      color=maincolor,
      before={\ \godown[6cm]},
      header=high,
      footer=\empty,
    ]
  \setuphead
    [title, chapter]
    [
      style={\switchtobodyfont[20pt, ss]\bf},
      align=flushright,
      before={\blank[big, force]},
      after={\blank[2*big]},
      color=maincolor,
      sectionsegments=2:2,
      commandbefore={\blank[small]},
      header=high,
      footer=ChapterFirstPage,
      after=\vskip .5cm,
      text=empty
    ]
  \setuphead
    [section, subject]
    [
      style={\switchtobodyfont[14pt, ss]\bf},
      before={\blank[2*big]},
      after=\blank,
      color=maincolor,
      sectionsegments=2:3,
    ]

  \setuphead
    [subsection, subsubject]
    [
      style={\switchtobodyfont[12pt, ss]\bf},
      before=\blank,
      after=\blank,
      color=maincolor,
      sectionsegments=2:4
    ]

  \define[1]\PlacePoint{#1.}

  \setuphead
    [subsubsection,subsubsubject]
    [
      sectionsegments=5:5,
      conversion=A,
      style={\switchtobodyfont[10pt, ss]\bf},
      style=\bfa,
      color=black,
      numbercommand=,
      hidenumber=yes
    ]

%  \setuphead
%    [subsubsection][numbercommand=\PlacePoint]


  % Structures defined by me

  \definedescription
    [description]
    [
      alternative=hanging,
      width=fit,
      distance=0.3cm,
      headstyle=\bf
    ]

  \definedescription
    [semitable]
    [
      alternative=left,
      width=fit,
      distance=1cm,
      headcolor=darkmagenta,
      headstyle=\tt,
    ]

  % ENVIRONMENTS

  % DoubleExample: For two column examples. It gave some errors when
  % defining it with \definestartstop (due to my lack of knowledge) and
  % in the end what I did was to define a TeX style command for opening
  % it, and another command for closing it.
  \def\startDoubleExample
    {
      \startframedtext
      [frame=off]
      \switchtobodyfont[small]
      \setuptyping[style=tt]
      \startcolumns
        [
          n=2,
          tolerance=verytolerant,
          distance=0.5cm,
          separator=rule,
          rulecolor=darkmagenta
        ]
    }

  \def\stopDoubleExample{\stopcolumns\stopframedtext}

  % Small print
  \definestartstop[SmallPrint]
    [
      before={\startnarrower\switchtobodyfont[small,ss]},
      after={\stopnarrower},
    ]

  % Commands and abbreviations

  % Assumption: Print the assumption icon in the margin
  \define\Conjecture
    {\inmargin[line=1]{\externalfigure[miniconjecture.jpg][width=1cm]}}

  % Doubt: Print the doubt icon in the margin
  \define\Doubt
    {\inmargin[line=1]{\externalfigure[minidoubt.jpg][width=1cm]}}

  % example: Show in red and in small print the result of compiling an
  % earlier code snippet.
  \define[1]\example{\color[red]{\tfx#1}}

  % ConTeXt Standalone
  \def\suite-{\quotation{\ConTeXt\ Standalone}}

  % cmd: to be used in place of tex when, in the source file, there is a
  % line break in an argument.
  \define[1]\cmd
   {{\en\tt\color[darkmagenta]{\backslash #1}}}

  % Partial chapter table of contents:
  \def\Separate#1{#1;}
  \def\TocChap
    {
      \startnarrower\switchtobodyfont[small, sl]
      {\bf Table of Contents:}
      \setuplist
        [section,subsection,subsubsection]
        [pagenumber=no, textcommand=\Separate]
      \setuplist[section][style=bold]
      \placecontent[alternative=d]
      \stopnarrower
    }

  % Index
  \defineregister[macro]
  \define[1]\PlaceMacro{\macro[#1]{\backslash #1}}

  % INTERACTIVITY

  \setupinteraction
    [
      state=start,
      color=maincolor,
      contrastcolor=maincolor,
      style=,
    ]

  \setupurl [color=maincolor, style=\tt]

  \setupalign[hz,lesshyphenated,verytolerant,stretch]

  \setuplanguage[en][spacing=packed]% This avoids an extra space after full stops (periods) which is no longer 'de rigeur' in printed material in English.
  \startMPcalculation
    vardef drawarrow_withtext(expr A, B, str) =
        save mylabel, mid;
        pair mid; mid = 0.5[A, B];
        picture mylabel; mylabel = thelabel(str, mid)
            rotatedaround (mid, if xpart(B-A) >= 0: angle(B-A) else: angle(A-B) fi);
        drawarrow A -- B; unfill bbox mylabel; draw mylabel;
    enddef;
    vardef drawdblarrow_withtext(expr A, B, str) =
        save mylabel, mid;
        pair mid; mid = 0.5[A, B];
        picture mylabel; mylabel = thelabel(str, mid)
            rotatedaround (mid, if xpart(B-A) >= 0: angle(B-A) else: angle(A-B) fi);
        drawdblarrow A -- B; unfill bbox mylabel; draw mylabel;
    enddef;
  \stopMPcalculation

  \definesectionblock[roman][romanpages]
\definestructureconversionset [roman:pagenumber] [] [romannumerals]

\startsectionblockenvironment [bodypart]
  \setcounter [userpage] [1]
\stopsectionblockenvironment
\startsectionblockenvironment [bodypart]
  \setcounter [userpage] [1]
\stopsectionblockenvironment
\stopenvironment
