% -*- mode: context; -*-
\chapter{Binomials, Multinomials and Expansions}
\startitemize[n, 1*broad]
\item Using binomial theorem, $\left(x + \frac{1}{x}\right)^5 = C_0^^5x^5 + C_1^^5x^4.\frac{1}{x} +
  C_2^^5x.\frac{1}{x^2} + C_3^^5x^2.\frac{1}{x^3} + C_4^^5x.\frac{1}{x^4} + C_5^^5.\frac{1}{x^5}$

  $= C_0x^5 + C_1^^5x^3 + C_2^^5x + C_3^^5.\frac{1}{x} + C_4^^5.\frac{1}{x^3} + C_5^^5.\frac{1}{x^5}$.
\item $(10.1)^5 = (10 + 0.1)^5$, so we proceed like previous problem to get

  $(10.1)^5 = C_0^^510000 + C_1^^51000 + C_2^^510 + C_3^^5\frac{1}{10} + C_4^^5\frac{1}{1000} +
  C_5^^5\frac{1}{100000}$

  $= 100000 + 5000 + 100 + 1 + .005 + .00001 = 15101.00501$.
\item $(x + \sqrt{x - 1})^6 + (x - \sqrt{x - 1})^6 = C_0^^6x^6 + C_1^^6x^5\sqrt{x - 1} + C_2^^6x^4\sqrt{(x -
    1)^2} + C_3^^6x^3\sqrt{(x - 1)^3} + C_4^^6x^2\sqrt{(x - 1)^4} + C_5^^6x\sqrt{(x - 1)^5} + C_6^^6\sqrt{(x
    - 1)^6} + C_0^^6x^6 - C_1^^6x^5\sqrt{x - 1} + C_2^^6x^4\sqrt{(x - 1)^2} - C_3^^6x^3\sqrt{(x - 1)^3} +
  C_4^^6x^2\sqrt{(x - 1)^4} - C_5^^6x\sqrt{(x - 1)^5} + C_6^^6\sqrt{(x - 1)^6}$

  $= 2x^6 + 30x^4(x - 1) + 30x^2(x - 1)^2 + 2(x - 1)^3$.
\item Consider the expansion of $(x + a)^n$ and $(x - a)^n$. The sum of real terms will be $A$ and the sum
  of imaginary terms will be $B$.

  $(x + a)^n = C_0^^nx^n + C_1^^nx^{n - 1}.a + C_2^^nx^{n - 2}a^2 + \cdots + C_n^^na^n = A + B$, and
  $(x - a)^n = C_0^^nx^n - C_1^^nx^{n - 1}.a + C_2^^nx^{n - 2}a^2 + \cdots + C_n^^n(-a)^n = A - B$

  Multiplying, we get

  $(x^2 - a^2)^n = A^2 - B^2$.
\item Let $(7 + 4\sqrt{3})^n = \alpha + \beta$, where $\alpha$ is a positive integer and $\beta$ is a proper
  fraction.

  Cleaerly, $0 < 7 - 4\sqrt{3} < 1\left[\because 7 - 4\sqrt{3} = \frac{49 - 48}{7 + 4\sqrt{3}} = \frac{1}{7
      + 4\sqrt{3}}\right]$

  $\therefore 0 < (7 - 4\sqrt{3})^n < 1 = \beta_1$(let), then $0 < \beta_1 < 1$.

  $\alpha + \beta + \beta_1 = 2[7^n + C_2^^n7^{n - 2}.48 + \cdots] = $ an even number.

  $\Rightarrow \beta + \beta_1 =$ an even number $- \alpha = $ an integer.

  $\because 0 < \beta < 1$ and $0 < \beta_1 < 1\therefore 0 < \beta + \beta_1 < 2$. Thus, $\beta + \beta_1 =
  1$.

  $\therefore \alpha + 1 =$ an even number $\Rightarrow \alpha = $ an odd number.
\item Proceeding from previous problem, $(\alpha + \beta)(1 - \beta) = (\alpha + \beta)\beta_1 = (7 +
  4\sqrt{3})^n(7 - 4\sqrt{3})^n = 1$.
\item $t_{r} = C_{r}^^{10}y^{10 - r}.\left(\frac{c^3}{y^2}\right)^r$. We have to find coefficient of
  $\frac{1}{y^2}$, hence, $10 - r - 2r = -2 \Rightarrow r = 4$.

  Thus, coefficient is $C_4^^{10}.c^{12}$.
\item We have to find coefficient of $x^9$ in $(1 + 3x + 3x^2 + x^3)^{15} = (1 + x)^{45}$. Therefore,
  coefficient is $C_9^^{45}$.
\item We have to find term independent of $x$ in $\left(\frac{3}{2}x^2 - \frac{1}{3x}\right)^9$. The general term
  will be $t_r = C_{r - 1}^^9.\left(\frac{3}{2}x^2\right)^{9 - r + 1}\left(-\frac{1}{3x}\right)^{r - 1}$.

  $\Rightarrow 21 - 3r = 0 \Rightarrow r = 7$. So coefficient is $(-1)^6.C_6^^9\left(\frac{3}{2}\right)^{10
    - 7}.\frac{1}{3^6} = \frac{7}{18}$.
\item $(1 + x)^m\left(1 + \frac{1}{x}\right)^n = x^{-n}(1 + x)^{m + n}$. We have to find term independent of
  $x$ in the expansion, which is coefficient of $x^n$ in $(1 + x)^{m + n}$.

  Coeff. is $C_n^^{m + n} = \frac{(m + n)!}{m!n!}$.
\item Coeff. of $x^{-1}$ in $(1 + 3x^2 + x^4)\left(1 + \frac{1}{x}\right)^8 =$ coeff. of $x^{-1}$ in
  $\left(1 + \frac{1}{x}\right)^8 +$ coeff. of $x^{-1}$ in $3x^2\left(1 + \frac{1}{x}\right)^8 +$ coeff. of
  $x^{-1}$ in $x^4\left(1 + \frac{1}{x}\right)^8$

  = coeff. of $x^{-1}$ in $\left(1 + \frac{1}{x}\right)^8$ + coeff. of $x^{-3}$ in $3\left(1 +
    \frac{1}{x}\right)^8 +$ coeff. of $x^{-5}$ in $\left(1 + \frac{1}{x}\right)^8$

  General term is given by $t_r = C_{r - 1}^^n\left(\frac{1}{x}\right)^{r - 1} = C_{r - 1}^^nx^{1 - r}$.

  When $r - 1 = 1\Rightarrow r = 2$, we have coeff. as $C_1^^8$. When $r - 1 = 3 \Rightarrow r = 4$, we have
  coeff. as $C_3^^8$, and similarly coeff. of $x^{-5}$ is $C_5^^8$.

  Thus, required coeff. of $x^{-1}$ is $C_1^^8 + 3.C_3^^8 + C_5^^8 = 232$.
\item $r$th term in the expansion of $(1 - x)^n$ is $C_{r - 1}^^{2n - 1}(-1)^{r - 1}x^{r - 1}$, so $(r + 1)$th
  term will have the term $x^r$.

  $\Rightarrow a_{r - 1} = (-1)^{r - 1}C_{r - 1}^^{2n - 1}$ and $a_{2n - r} = (-1)^{2n - r}C_{2n - r}^^{2n -
    1}$.

  We know that $C_r^^n = C_{n - r}^^n$ and $(-1)^{2n} = 1$. Hence, $a_{2n - r} = (-1)^{-r}C_{r - }^^{2n -
    1}$.

  Thus, $a_{r - 1} + a_{2n - r} = 0$.
\item Let the $r$th term be independent of $x$. $t_r = C_{r - 1}^^{10}(\sqrt{x})^{10 - r +
    1}\left(\frac{k}{x^2}\right)^{r - 1} =  C_{r - 1}^^{10}x^{\frac{11 - r}{2} - 2r + 2}k^{r - 1}$.

  Since the term is independent of $x\Rightarrow 15 - 5r = 0 \Rightarrow r = 3$.

  So the term is $C_2^^{10}k^2 = 405 \Rightarrow k = \pm 3$.
\item $k$th term in the expansion is given by $t_k = C_{k - 1}^^{n - 3}x^{n - 3 - k + 1}(x^{-2})^{k - 1}$

  $= C_{k - 1}^^{n - 3}x^{n - 3k}$. Let this term contain $x^{2r} \Rightarrow 2r = n - 3k \Rightarrow k =
  \frac{n - 2r}{3}$.

  Since $n - 2r$ is not a multiple of $3$, $k$ cannot be an integer. So no term will contain $x^{2r}$.
\item Let $r$th term be independent of $x$. $t_r = C_{r - 1}^^n(x^a)^{n - r + 1}(x^{-b})^{r - 1}$.

  This will be independent of $x$ if $an - ar + a - br + b = 0 \Rightarrow an = (a + b)(r - 1) \Rightarrow r
  = 1 + \frac{an}{a + b}$.

  Clearly, $r$ will be an integer only if $an$ is a multiple of $a + b$.
\item $\left(x + \frac{1}{x}\right)^7 = C_0^^7x^7 + C_1^^7x^6.\frac{1}{x} + C_2^^7x^5.\frac{1}{x^2} +
  C_3^^7x^4.\frac{1}{x^3} + C_4^^7x^3.\frac{1}{x^4} + C_5^^7x^2.\frac{1}{x^5} + C_6^^7x.\frac{1}{x^6} +
  C_7^^7.\frac{1}{x^7}$

  $= C_0^^7x^7 + C_1^^7x^5 + C_2^^7x^3 + C_3^^7x + C_4^^7.\frac{1}{x} + C_5^^7.\frac{1}{x^3} +
  C_6^^7.\frac{1}{x^5} + C_7^^7.\frac{1}{x^7}$

  $= x^7 + 7x^5 + 21x^3 + 35x + \frac{35}{x} + \frac{21}{x^3} + \frac{7}{x^5} + \frac{1}{x^7}$.
\item $\left(\frac{2x}{3} - \frac{3}{2x}\right)^6 = C_0^^6\left(\frac{2x}{3}\right)^6 +
  C_1^^6\left(\frac{2x}{3}\right)^5.\left(-\frac{3}{2x}\right) +
  C_2^^6\left(\frac{2x}{3}\right)^4\left(-\frac{3}{2x}\right)^2 +
  C_3^^6\left(\frac{2x}{3}\right)^3\left(-\frac{3}{2x}\right)^3 +
  C_4^^6\left(\frac{2x}{3}\right)^2\left(-\frac{3}{2x}\right)^4 +
  C_5^^6\left(\frac{2x}{3}\right)\left(-\frac{3}{2x}\right)^5 +
  C_6^^6\left(-\frac{3}{2x}\right)^6$

  $= \frac{64}{729}x^6 - \frac{32}{2x}x^4 + \frac{20}{3}x^2 - 20 + \frac{135}{4}x^2 - \frac{243}{8}x^4 +
  \frac{729}{64}x^6$.
\item Given, $(1 + ax)^n = 1 + 8x + 24x^2 + \cdots \Rightarrow 1 + nax + \frac{n(n - 1)}{2}a^2x^2 + \cdots =
  1 + 8x + 24x^2 + \cdots$

  Comparing coefficients of powers of $x$

  $an = 8, \frac{n(n - 1)}{2}a^2 = 24 \Rightarrow 32n^2 - 32n = 24n^2 \Rightarrow n = 4 \Rightarrow a = 2$.
\item $7$th term in the expansion of $\left(\frac{4x}{5} - \frac{5}{2x}\right)^9$ is
  $C_6^^9\left(\frac{4x}{5}\right)^3\left(-\frac{5}{2x}\right)^6 = \frac{10500}{x^3}$.
\item $(\sqrt{2} + 1)^6 + (\sqrt{2} - 1)^6 = C_0^^62^3 + C_1^^64\sqrt{2} + C_2^^62^2 + C_3^^62\sqrt{2} +
  C_4^^62 + C_5^^6\sqrt{2} + C_6^^6 + C_0^^62^3 - C_1^^64\sqrt{2} + C_2^^62^2 - C_3^^62\sqrt{2} +
  C_4^^62 - C_5^^6\sqrt{2} + C_6^^6$

  $=2.C_0^^62^3 + 2.C_2^^62^2 + 2.C_4^^62 + 2.C_6^^6 = 198$.
\item According to questions $(x + a)^n = A + B$, because $A$ is the sum of odd terms and $B$ is the sum of
  even terms. From Binomial theorem $(x - a)^n = A - B$.

  $\Rightarrow (x + a)^{2n} = A^2 + B^2 + 2AB$ and $(x - a)^{2n} = A^2 + B^2 - 2AB$.

  Subtracting, we get $(x + a)^{2n} - (x - a)^{2n} = 4AB$.
\item Let $(5 + 2\sqrt{6})^n = \alpha + \beta$, where $\alpha$ is a positive integer and $\beta$ is a proper
  fraction.

  Cleaerly, $0 < 5 - 2\sqrt{6} < 1\left[\because 5 - 2\sqrt{6} = \frac{25 - 24}{5 + 2\sqrt{6}} = \frac{1}{5
      + 2\sqrt{6}}\right]$

  $\therefore 0 < (5 - 2\sqrt{6})^n < 1 = \beta_1$(let), then $0 < \beta_1 < 1$.

  $\alpha + \beta + \beta_1 = 2[5^n + C_2^^n5^{n - 2}.24 + \cdots] = $ an even number.

  $\Rightarrow \beta + \beta_1 =$ an even number $- \alpha = $ an integer.

  $\because 0 < \beta < 1$ and $0 < \beta_1 < 1\therefore 0 < \beta + \beta_1 < 2$. Thus, $\beta + \beta_1 =
  1$.

  $\therefore \alpha + 1 =$ an even number $\Rightarrow \alpha = $ an odd number.
\item Proceeding from previous problem, $(\alpha + \beta)(1 - \beta) = (\alpha + \beta)\beta_1 = (3 +
  \sqrt{8})^n(3 - \sqrt{8})^n = 1$.
\item Let $r$th term contains the term $x$. Then $t_r = C_{r - 1}^^9(2x)^{10 -
    r}\left(-\frac{3}{x}\right)^{r - 1}$

  Since the term contains $x$, therefore $10 - r - r + 1 = 1 \Rightarrow r = 5$.

  Thus, coefficient is $C_4^^92^5.3^4 = 2592.C_4^^9$.
\item Let $r$th term contain $x^7$ in the expansion of $(3x^2 + (5x)^{-1})^{11}$. $t_r = C_{r -
    1}^^{11}(3x^2)^{12 - r}.(5x^{-1})^{r - 1}$.

  Since the term contains $x^7$, therefore $24 - 2r - r + 1 = 7 \Rightarrow r = 6$.

  Thus, coefficient is $C_5^^{11}.\frac{3^6}{5^5}$.
\item Let $r$th term contain $x^9$ in the expansion of $(2x^2 - x^{-1})^{20}$. Then $t_r = C_{r -
    1}^^{20}(2x^2)^{21 - r}(-x^{-1})^{r - 1}$.

  Since the term contains $x^9$, therefore $42 - 2r - r + 1 = 9$, which does not yield an integral value for
  $r$. Therefore, coefficient is $0$.
\item Let $r$th term contain $x^{24}$ in the expansion of $(x^2 + 3ax^{-1})^{15}$. Then $t_r = C_{r -
    1}^^{15}(x^2)^{16 - r}(3ax^{-1})^{r - 1}$.

  Since the term contains $x^{24}$, therefore $32 - 2r - r + 1 = 24 \Rightarrow r = 3$.

  Therefore, the coefficient is $C_2^^{16}9a^2$.
\item Let $r$th term contain $x^9$ in the expansion of $(x^2 - (3x)^{-1})^9$. Then $t_r = C_{r -
    1}^^9(x^2)^{10 - r}\left(-\frac{1}{3x}\right)^{r - 1}$.

  Since the term contains $x^9$, therefore $20 - 2r - r + 1 = 9 \Rightarrow r = 4$.

  Therefore, the coefficient is $C_3^^9.\frac{-1}{3^3} = -\frac{28}{9}$.
\item Let $r$th term contain $x^{-7}$ in the expansion of $\left(2x - \frac{1}{3x^2}\right)^{11}$.
  Then $t_r = C_{r - 1}^^{11}(2x)^{12 - r}\left(-\frac{1}{3x^2}\right)^{r - 1}$.

  Since the term contains $x^{-7}$, therefore, $12 - r - 2r + 2 = -7 \Rightarrow r = 7$.

  Therefore, coefficient is $C_6^^{11}\frac{2^5}{3^6}$.
\item Let $r$th term contain $x^7$ in the expansion of $\left(ax^2 + \frac{1}{bx}\right)^{11}$. $t_r = C_{r
    - 1}^^{11}(ax^2)^{12 - r}.\left(\frac{1}{bx}\right)^{r - 1}$.

  Since the term contains $x^7$, therefore $24 - 2r - r + 1 = 7 \Rightarrow r = 6$.

  Therefore, coefficient is $C_5^^{11}a^6b^{-5}$.
  Let $s$th term contain $x^{-7}$ in the expansion of $\left(ax - \frac{1}{bx^2}\right)^{11}$. Then $t_r =
  C_{r - 1}^^{11}(ax)^{12 - r}\left(-\frac{1}{bx^2}\right)^{r - 1}$.

  Since the term contains $x^{-7}$, therefore $12 - r - 2r + 2 = -7 \Rightarrow r = 7$.

  Therefore, the coefficient is $C_6^^{11}a^5b^{-6}$.

  Since the coefficients are equal $ab = 1[\because C_5^^{11} = C_6^^{11}]$.
\item Let $r$th term contain $x^p$ in the expansion of $\left(x^2 + \frac{1}{x}\right)^{2n}$. Then $t_r =
  C_{r - 1}^^{2n}(x^2)^{2n + 1 - r}\frac{1}{x^{r - 1}}$.

  Since the term contains $x^p$, therefore $4n + 2 - 2r - r + 1 = p \Rightarrow r = \frac{4n - p}{3} + 1$.

  Therefore, the coefficient is $\frac{2n!}{\left(\frac{4n - p}{3}\right)!\left(\frac{2n + p}{3}\right)!}$.
\item The problems are solved below:
  \startitemize[i]
  \item Let $r$th term be indnependent of $x$ in the expansion of $\left(x + \frac{1}{x}\right)^{2n}$. Then
  $t_r = C_{r - 1}^^{2n}x^{2n + 1 - r}.\frac{1}{x^{r - 1}}$

  Since the term is independent of $x$, therefore $2n + 1 - r - r + 1 = 0 \Rightarrow r = n + 1$.

  Therefore, the coefficient is $C_n^^{2n} = \frac{(2n)!}{n!n!}$.
\item Let $r$th term be independent of $x$ in the expansion of $\left(2x^2 + \frac{1}{x}\right)^{15}$. Then
  $t_r = C_{r - 1}^^{15}(2x^2)^{16 - r}\left(\frac{1}{x}\right)^{r - 1}$.

  Since the term is independent of $x$, therefore $32 - 2r -r + 1 = 0 \Rightarrow r = 11$.

  Therefore, the coefficient is $C_{10}^^{15}.2^5 = 32.C_{10}^^{15}$.
\item Let $r$th term be independent of $x$ in the expansion of $\left(\sqrt{\frac{x}{3}} +
  \frac{3}{2x^2}\right)^{10}$. Then $t_r = C_{r - 1}^^{10}\left(\sqrt{\frac{x}{3}}\right)^{11 -
    r}\left(\frac{3}{2x^2}\right)^{r - 1}$.

  Since the term is independent of $x$, therefore $\frac{11 - r}{2} - 2r + 2 = 0 \Rightarrow r = 3$.

  Therefore, the coefficient is $C_2^^{10}.\left(\frac{1}{3^4}\right).\left(\frac{3^2}{2^2}\right) =
  \frac{5}{4}$.
\item Let $r$th term be independent of $x$ in the expansion of $\left(2x^2 - \frac{1}{x}\right)^{12}$. Then
  $t_r = C_{r - 1}^^{12}(2x^2)^{13 - r}\left(\frac{1}{x}\right)^{r - 1}$.

  Since the term is independent of $x$, therefore $26 - 2r - r + 1 = 0 \Rightarrow r = 9$.

  Therefore, the coefficient is $C_8^^{12}.2^4 = 7920$.
\item Let $r$th term be independent of $x$ in the expansion of $\left(2x^^2 - \frac{3}{x^3}\right)^{25}$. Then
  $t_r = C_{r - 1}^^{25}(2x^2)^{26 - r}\left(-\frac{3}{x^^3}\right)^{r - 1}$.

  Since the term is independent of $x$, therefore $52 - 2r - 3r + 3 = 0 \Rightarrow r = 11$.

  Therefore, the coefficient is $C_{10}^^{25}.2^{15}3^{10}$.
\item Let $r$th term be independent of $x$ in the expansion of $\left(x^3 - \frac{3}{x^2}\right)^{15}$. Then
  $t_r = C_{r - 1}^^{15}(x^3)^{16 - r}\left(-\frac{3}{x^2}\right)^{r - 1}$.

  Since the term is independent of $x$, therefore $48 - 3r - 2r + 2 = 0 \Rightarrow r = 10$.

  Therefore, the coefficient is $C_9^^{15}\left(-3\right)^9 = -3^9.C_{0}^^{15}$.
\item Let $r$th term be independent of $x$ in the expansion of $\left(x^2 - \frac{3}{x^3}\right)^{10}$. Then
  $t_r = C_{r - 1}^^{10}(x^2)^{11 - r}\left(-\frac{3}{x^3}\right)^{r - 1}$.

  Since the term is independent of $x$, therefore $22 - r - 3r + 3 = 0 \Rightarrow r = 5$.

  Therefore, the coefficient is $C_4^^{10}.(-3)^4 = 3^4.C_{4}^^{10}$.
\item Let $r$th term be independent of $x$ in the expansion of $\left(\frac{1}{2}x^{1/3} +
  x^{-1/3}\right)^8$. Then $t_r = C_{r - 1}^^{8}\left(\frac{1}{2}x^{1/3}\right)^{9 - r}(x^{-1/3})^{r - 1}$.

  Since the term is independent of $x$, therefore $9 - r - r + 1= 0 \Rightarrow r = 5$.

  Therefore, the coefficient is $C_4^^8.\frac{1}{2^4} = \frac{35}{8}$.
\stopitemize
\item Let $r$th term be independent of $x$ in the expansion of $\left(x + \frac{1}{x^2}\right)^n$. Then $t_r
  = C_{r - 1}^^nx^{n + 1 - r}\frac{1}{(x^2)^{r - 1}}$.

  Since the term is independent of $x$, therefore $n + 1 - r - 2r + 2 = 0 \Rightarrow r = \frac{n}{3} + 1$.

  Therefore, the coefficient is $C_{\frac{n}{3}}^^{n} =
  \frac{n!}{\left(\frac{n}{3}\right)!\left(\frac{2n}{3}\right)!}$.
\item First we will find coefficient of $x^m$ and then of $x^n$ in the expansion of $(1 + x)^{m + n}$. Let
  $p$th term contain $x^m$.

  Then $t_P = C_{p - 1}^^{m + n}x^{p - 1}$. Since it contains $x^m$, therefore $p = m + 1$.

  Thus coefficient is $C_m^^{m + n}$. Similarly, we find the coefficient of term containing $x^n$ as
  $C_n^^{m + n}$. We know that $C_r^^n = C_{n - r}^^n$.

  Therefore, $C_m^^{m + n} = C_n^^{m + n}$. Hence, proved.
\item $4$th term in the expansion of $\left(px + \frac{1}{x}\right)^n$ is given by $t_4 = C_3^^n(px)^{n + 1
  - 4}\frac{1}{x^{3}}$.

  Since the term is independent of $x$, therefore $n - 3 - 3 = 0 \Rightarrow n = 6$.

  So the term is $C_3^^6p^3 = \frac{5}{2}\Rightarrow p = \frac{1}{2}$.
\item There are $13$ terms in the expansion of $\left(x - \frac{1}{2x}\right)^{12}$. So $7$th term will be
  the middle term.

  $t_7 = C_6^^{12}.x^6\left(-\frac{1}{2x}\right)^6 = C_6^^{12}.\frac{1}{2^6} = \frac{231}{16}$.
\item There are $8$ terms in the expansion of $\left(2x^2 - \frac{1}{x}\right)^7$. These are $4$th and $5$th
  terms.

  $t_4 = C_3^^7(2x^2)^4\left(-\frac{1}{x}\right)^3 = -560x^5,\;t_5 =
  C_4^^7(2x^2)^3\left(-\frac{1}{x}\right)^4 = 280x^2$.
\item There are $2n + 1$ terms in the expansion of $\left(x + \frac{1}{x}\right)^{2n}$. So the middle term
  is $(n + 1)$th term.

  $t_{n + 1} = C_n^^{2n}x^{2n - n}.\frac{1}{x^n} = C_n^^{2n} = \frac{2n!}{n!n!} = \frac{1.3.5\ldots (2n -
    1).2^n}{n!}$.
\item There are $2n + 1$ terms in the expansion of $(1 + x)^{2n}$. So the middle term is $(n + 1)$th term.

  $t_{n + 1} = C_n^^{2n}x^n$. So the ccoefficient is $C_n^^{2n}$.

  There are $2n$ temrs in the expansion of $(1 + x)^{2n - 1}$. So the middle terms are $n$th and $(n + 1)$th
  terms.

  Coefficients are $C_{n - 1}^^{2n - 1}$ and $C_n^^{2n - 1}$.

  Clearly, $C_{n - 1}^^{2n - 1} + C_n^^{2n - 1} = C_n^^{2n}$. Hence, proved.
\item The solutions are given below:
  \startitemize[i]
  \item There will be $21$ terms in the expansions of $\left(\frac{2x}{3} - \frac{3y}{2}\right)^{20}$. So
    $11$th term will be the middle term.

    $t_{11} = C_{10}^^{20}\left(\frac{2x}{3}\right)^{21 - 10}.\left(-\frac{3y}{2}\right)^{10} =
    C_{10}^^{20}x^{10}y^{10}$.
  \item There will be $7$ terms in the expansions of $\left(\frac{2x}{3} - \frac{3}{2x}\right)^6$. So $4$th
    term will be the middle term.

    $t_4 = C_3^^6\left(\frac{2x}{3}\right)^3\left(-\frac{3}{2x}\right)^3 = -20$.
  \item There will be $8$ terms in the expansion of $\left(\frac{x}{y} - \frac{y}{x}\right)^7$. So $4$th and
    $5$th termss will be the middle terms.

    $t_4 = C_3^^7.\left(\frac{x}{y}\right)^4\left(-\frac{y}{x}\right)^3 = -\frac{35x}{y},\;t_5 =
    C_4^^7\left(\frac{x}{y}\right)^3\left(-\frac{y}{x}\right)^3 = \frac{35y}{x}$.
  \item The middle term of the expansion $(1 + x)^{2n}$ will be the $(n + 1)$th term.

    $t_{n + 1} = C_n^^{2n}x^n = \frac{2n!}{n!n!}x^n$.
  \item $(1 - 2x + x^2)^n = (1 - x)^{2n}$ so $(n + 1)$th term will be the middle term.

    $t_{n + 1} = C_n^^{2n}(-x^n) = (-1)^n\frac{2n!}{n!n!}x^n$
  \stopitemize
\item The general $r$th term will be given by $t_r = C_{r - 1}^^{2n + 1}\left(\frac{x}{y}\right)^{2n + 1 + 1
  - r}.\left(\frac{y}{x}\right)^{r - 1}$.

  Since it will $2n + 2$ terms there will be two middle terms. $(n + 1)$th and $(n + 2)$th terms will be
  middle terms. Since the powers of $x$ and $y$ are symmetric if any term has to be free of $x$ and $y$ then
  it has to be middle terms.

  $t_{n + 1} = C_n^^{2n + 1}\left(\frac{x}{y}\right)^{2n + 1 + 1 - n - 1}\left(\frac{y}{x}\right)^n =
  C_n^^{2n + 1}\frac{x}{y}$

  $t_{n + 2} = C_{n + 1}^^{2n + 1}\frac{y}{x}$. Both of these terms are not free of $x$ and $y$.

  We also prove that no term is free of $x$ and $y$ by considering general term. Since the term has to be
  independent of $x$ and $y$, we consider the general term.

  $2n + 2 - r - - r + 1 = = \Rightarrow r = \frac{2n + 1}{2}$, which cannot be an integer. So no terms is
  free of both $x$ and $y$.
\item There will be $2n + 1$ terms in the expansion of $\left(x - \frac{1}{x}\right)^{2n}$. So the middle
  term would be $(n + 1)$th term.

  $t_{n + 1} = C_n^^{2n}x^{2n + 1 - n - 1}\frac{-1^n}{x^{n}} = (-1)^n\frac{2n!}{n!n!} = \frac{1.3.5\ldots
    (2n -1)}{n!}.(-2)^n$.
\item $t_{2r + 1} = C_{2r}^^{43}x^{2r}$ and $t_{r + 2} = C_{r + 1}^^{43}x^{r + 1}$.

  Given that coefficients are equal. $\therefore C_{2r}^^{43} = C_{r + 1}^^43 \Rightarrow 2r + r + 1 = 43
  \Rightarrow r = 14$.
\item Coefficient of $r$th term in the expansion of $(1 + x)^{20}$ is $C_{r - 1}^^20$, and the coefficient
  of $(r + 4)$th term is $C_{r + 3}^^{20}$.

  Clearly, for coefficients to be equal $r - 1 + r + 3 = 20 \Rightarrow r = 9$.
\item Following like previous problem, $r - 3 + 2r + 3 = 18 \Rightarrow r = 6$.
\item Following like previous problem, $2r + 4 + r - 7 = 39 \Rightarrow r = 14$. So $C_{12}^^r = 91$.
\item Following like previous problem, $3r - 1 + r + 1 = 2n \Rightarrow r = \frac{n}{2}$.
\item Following like previous problem, $p + p + 2 = 2n \Rightarrow p = n - 1$.
\item Coefficient of $(r + 1)$th term in the expansion of $(1 + x)^{n + 1}$ is $C_r^^{n + 1}$. Coefficients
  of $r$th and $(r + 1)$th terms in the expansion of $(1 + x)^n$ are $C_{r - 1}^^n$ and $C_r^^n$
  respectively.

  Clearly, $C_{r - 1}^^n + C_r^^{n} = C_r^^{n + 1}$. Hence, proved.
\item Since we have to find numerically greatets term we can replaced $-$ sign with $+$. Let $r$th term be
  the greatest term in the expansion of $\left(7 + \frac{10}{3}\right)^{11}$. $t_r =
  C_{r - 1}^^{11}7^{12 - r}\left(\frac{10}{3}\right)^{r - 1}$. We consider $(r + 1)$th term as well. $t_{r
    + 1} = C_r^^{11}7^{11 - r}\left(\frac{10}{3}\right)^r$

  $\frac{t_r}{t_{r + 1}} = \frac{21r}{(12 - r)10}\geq 1 \Rightarrow r\geq 3\frac{27}{31}$.

  Replacing $r$ with $r - 1$, $\frac{t_{r - 1}}{t_r} = \frac{21r - 21}{130 - 10r}\Rightarrow r\leq
  4\frac{27}{31}\therefore r = 4$.

  So the greatest term will be $C_3^^{11}7^8\frac{10^3}{3^3} = \frac{440}{9}7^85^3$.
\item In any binomial expansion, the middle terms have the greatest coefficient. Therefore, $(n + 1)$th term
  will have greatest coefficient.

  $t_n = C_{n - 1}^^{2n}x^{n - 1},\;t_{n + 1} = C_n^^{2n}x^n,\;t_{n + 2} = C_{n + 1}^^{2n}x^{n + 1}$

  $\therefore \frac{t_{n + 1}}{t_{n + 2}} = \frac{n + 1}{n}.\frac{1}{x}$. Since $t_{n + 1}$ is the greatest
  term $\frac{t_{n + 1}}{t_{n + 2}} > 1 \Rightarrow x < \frac{n + 1}{n}$.

  Similarly, considering $t_n$ and $t_{n + 1},\;x > \frac{n}{n + 1}$.
\item The greatest terms are calculated below:
  \startitemize[i]
  \item $\left(2 + \frac{9}{5}\right)^{10}$ will have $6$th term as the middle term, which will be greatest.

    $t_6 = C_5^^{10}.2^5.\left(\frac{9}{5}\right)^5 = C_5^^{10}\left(\frac{18}{5}\right)^5$.
  \item For $(4 - 2)^7$ let $t_r$ is the greatest term. Then $\frac{t_r}{t_{r + 1}} > 1$ and
    $\frac{t_r}{t_{r - 1}} > 1$. Substituting and evaluating, we find $r = 3$.

    $t_3 = C_2^^7.4^5.2^2 = 86016$.
  \item For $(5 + 2)^{10}$ let $t_r$ is the greatest term. Then $\frac{t_r}{t_{r + 1}} > 1$ and
    $\frac{t_r}{t_{r - 1}} > 1$. Substituting and evaluating, we find $r = 4$.

    $t_4 = C_3^^{13}5^{10}2^3$.
  \stopitemize
\item In any binomial expansion, the middle terms have the greatest coefficient. Therefore, $(15 + 1)$th term
  will have greatest coefficient.

  $t_{15} = C_{14}^^{30}x^{14},\;t_{16} = C_{15}^^{30}x^{15},\;t_{17} = C_{16}^^{30}x^{16}$

  $\therefore \frac{t_{16}}{t_{17}} = \frac{16}{15}.\frac{1}{x}$. Since $t_{16}$ is the greatest
  term $\frac{t_{16}}{t_{17}} > 1 \Rightarrow x < \frac{16}{15}$.

  Similarly, considering $t_{15}$ and $t_{16},\;x > \frac{15}{16}$.
\item Given, $6^{2n} - 35n - 1 = 36^n - 35n - 1 = (1 + 35)^n - 35n - 1 = 35^2[C_2^^n + 35.C_3^^n + \cdots +
  35^{n - 2}]$

  $= 1225[C_2^^n + 35.C_3^^n + \cdots + 35^{n - 2}] = 1225\times$ a positive integer if $n\geq 2$.

  If $n = 1$, given expression becomes $0$. Hence, for all positive integral values of $n$, $6^{2n} - 35n -
  1$ is divisible by $1225$.
\item $2^{4n} - 2^n(7n + 1) = 16^n - 2^n(7n + 1) = (2 + 14)^n - 2^n(7n + 1) = 14^2[C_2^^n.2^{n - 2} +
  C_3^^n.2^{n - 3}.14 + \cdots + 14^{n - 2}]$, which is divisible by $196$ for all positive values of
  $n$. If $n = 1$, given expression becomes $0$, which is also divisible by $196$.
\item $3^{4n + 1} + 16n - 3 = 3(3^{4n} - 1) + 16n = 3[81^n - 1] + 16n = 3[(1 + 80)^n - 1] + 16n$

  $= 3[80n + C_2^^n80^2 + C_3^^n80^3 + \cdots + 80^n] + 16n = 256[n + 75(C_2^^n + C_3^^n.80 + \cdots + 80^{n
    - 2})]$,

  which is divisible by $256$ for all $n\in\mathbb{N}$.
\item The problems are solved below:
  \startitemize[i]
  \item $4^n - 3n - 1 = (1 + 3)^n - 3n - 1 = C_2^^n3^2 + C_3^^n3^3 + \cdots + 3^n$

    $= 9[C_2^^n + C_3^^n.3 + \cdots + 3^{n - 2}]$,

    which is divisible by $9$ for $n\geq 2$. When $n = 1$, the given expression becomes $0$, and hence
    divisible by $9$. Thus, given expression is divisible by $9$ for all $n\in\mathbb{P}$.
  \item $2^{5n} - 31n - 1 = (1 + 31)^n - 31n - 1 = C_2^^n31^2 + C_3^^n31^3 + \cdots + 31^n$

    $= 961[C_2^^n + C_3^^n.31 + \cdots + 31^{n - 2}]$,

    which is divisible by $961$ for $n\geq 2$. When $n = 1$, the given expression becomes $0$, and hence
    divisible by $961$. Thus, given expression is divisible by $961$ for all $n\in\mathbb{P}$.
  \item $3^{2n + 2} - 8n - 9 = 9(1 + 8)^n - 8n - 9 = 9[1 + 8n + C_2^^n.8^2 + C_3^^n.8^3 + \cdots + 8^n] - 8n
    - 9 = 64[n + 9(C_2^^n + C_3^^n8 + \cdots + 8^{n - 2})]$,

    which is divisible by $64$ for $n\geq 2$.
  \item $2^{5n + 5} - 31n - 32 = 32(1 + 31)^n - 31n - 32 = 32[1 + 31n + C_2^^n.31^2 + C_3^^n.31^3 + \cdots +
    31^n] - 31n - 32 = 961[n + 32(C_2^^n + C_3^^n31 + \cdots + 31^{n - 2})]$,

    which is divisible by $961$ for $n > 1$.
  \item $3^{2n} - 1 + 24n - 32n^2 = (1 + 8)^n - 1 + 24n - 32n^2 = 1 + 8n + 32n^2 - 32n + 8^3(C_3^^n +
    C_4^^n.8 + \cdots + 8^{n - 3}) - 1 + 24n - 32n^2 = 8^3(C_3^^n +
    C_4^^n.8 + \cdots + 8^{n - 3})$,

    which is divisible by $512$ for $n > 2$.
  \stopitemize
\item Let the three consecutive coefficients in the expansion of $(1 + x)^n$ be the $r$th, $(r + 1)$th and
  $(r + 2)$th, which are given to be $165, 330$ and $462$ respectively.

  $\therefore C_{r - 1}^^n = 165 \Rightarrow \frac{n!}{(r - 1)!(n - r + 1)!} = 165$

  $C_r^^n = 330 \Rightarrow \frac{n!}{r!(n - r)!} = 330$, and $C_{r + 1}^^n = \frac{n!}{(r + 1)!(n - r -
  1)!} = 462$.

  From first two, we have $\frac{r}{n - r + 1}= \frac{1}{2} \Rightarrow 3r = n + 1$.

  From last two, we have $\frac{r + 1}{n - r} = \frac{5}{7}\Rightarrow 12r = 5n - 7$

  Thus, $n = 11, r = 4$. So positions of coefficients are the $4$th, $5$th and $6$th respectively.
\item Let $a_1, a_2, a_3$ and $a_4$ be the coefficients of the $r$th, $(r + 1)$th, $(r + 2)$th and $(r +
  3)$th terms respectively in the expansion of $(1 + x)^n$.

  $\therefore a_1 = C_{r - 1}^^n, a_2 = C_r^^n, a_3 = C_{r + 1}^^n$, and $a_4 = C_{r + 2}^^n$.

  $\frac{a_2}{a_1} = \frac{n - r + 1}{r}\Rightarrow \frac{a_1 + a_2}{a_1} = \frac{n + 1}{r}\Rightarrow
  \frac{a_1}{a_1 + a_2} = \frac{r}{n + 1}$.

  Similarly, $\frac{a_2}{a_2 + a_3} = \frac{r + 1}{n + 1}$, and $\frac{a_3}{a_3 + a_4} = \frac{r + 2}{n +
    1}$.

  Clearly, $\frac{a_1}{a_1 + a_2} + \frac{a_3}{a_3 + a_4} = \frac{2a_2}{a_2 + a_3}$.
\item $2$nd terms $= C_1^^nx^{n - 1}y = 240$, $3$rd term $= C_2^^nx^{n - 2}y^2 = 720$, and $4$th term $=
  C_3^^nx^{n - 3}y^3 = 1080$.

  From first two, we have $\frac{240}{720} = \frac{2}{n - 1}.\frac{x}{y}$.

  From last two, we have $\frac{720}{1080} = \frac{3}{n - 2}.\frac{x}{y}$

  From these two equations $\frac{1}{2} = \frac{2(n - 2)}{3(n - 1)}\Rightarrow n = 5$.

  $\Rightarrow y = \frac{3x}{2}$.

  $\Rightarrow 240 = C_1^^nx^{n - 1}y \Rightarrow x^5 = 32 \Rightarrow x = 2 \Rightarrow y = 3$.
\item Let the index of the power be $n$. And let $a, b, c$ be the $r$th, $(r + 1)$th, $(r + 2)$th
  coefficients respectively in the expansion of $(1 + x)^n$.

  $a = C_{r - 1}^^n, b = C_r^^n$, and $c = C_{r + 1}^^n$.

  $\frac{a}{b} = \frac{r}{n - r + 1} \Rightarrow an + a = r(a + b)$, $\frac{b}{c} = \frac{r + 1}{n - r}
  \Rightarrow bn - br = cr + c \Rightarrow bn - c = r(b + c)$

  $\Rightarrow n = \frac{2ac + ab + bc}{b^2 - ac}$.
\item The coefficient of $14$th, $15$th and $16$th terms in the expansion of $(1 + x)^n$ will be $C_{13}^^n,
  C_{14}^^n$ and $C_{15}^^n$ respectively. Given that these are in A.P. $\Rightarrow 2C_{14}^^n = C_{13}^^n
  + C_{15}^^n$.

  $\Rightarrow 2.\frac{n!}{14!(n - 14)!} = \frac{n!}{13!(n - 13)!} + \frac{n!}{15!(n - 15)!}$

  $\Rightarrow 2.15(n - 13) = 15.14 + (n - 13)(n - 14)\Rightarrow n = 23, 34$.
\item Let the three consecutive terms are $r$th, $(r + 1)$th and $(r + 2)$th. Then, $C_{r - 1}^^n = 56,
  C_r^^n = 70, C_{r + 1} = 56$.

  From first two, we have $\frac{r}{n - r + 1} = \frac{4}{5}$, and from last two, we have $\frac{r + 1}{n -
    r} = \frac{5}{4}$.

  Solving these gives us $n = 8, r = 4$.
\item Let the three consecutive terms are $r$th, $(r + 1)$th and $(r + 2)$th. Then, $C_{r - 1}^^n = 220,
  C_r^^n = 495, C_{r + 1} = 792$.

  From first two, we have $\frac{r}{n - r + 1} = \frac{4}{9}$, and from last two, we have $\frac{r + 1}{n -
    r} = \frac{5}{8}$.

  Solving these gives us $n = 12$.
\item $t_3 = C_2^^na^{n - 2}x^2 = 84, t_4 = C_3^^na^{n - 3}x^3 = 280$, and $t_5 = C_4^^na^{n - 4}x^4 = 560$.

  From first two, we have $\frac{3}{n - 2}.\frac{a}{x} = \frac{3}{10}$, and from last two we have
  $\frac{4}{n - 3}.\frac{a}{x} = \frac{1}{2}$.

  $\Rightarrow \frac{3(n - 3)}{4(n - 2)} = \frac{3}{5}\Rightarrow n = 7, \Rightarrow x = 2, a = 1$.
\item $t_6 = C_5^^nx^{n - 5}y^5 = 112, t_6 = C_6^^nx^{n - 6}y^6 = 7$, and $t_8 = C_7^^nx^{n - 7}y^7 =
  \frac{1}{4}$.

  From first two, we have $\frac{6}{n - 5}.\frac{x}{y} = 16$, and from last two we have $\frac{7}{n -
    6}.\frac{x}{y} = 28$

  $\Rightarrow \frac{6(n - 6)}{7(n - 5)} = \frac{4}{7}\Rightarrow n = 7, x = 4, y = \frac{1}{2}$.
\item Let the binomial expansion be $(x + y)^n$. $a = C_5^^nx^{n - 5}y^5, b = C_6^^nx^{n - 6}y^6, c =
  C_7^^{n}x^{n - 7}x^7$, and $d = C_8^^nx^{n - 8}y^8$.

  From first two, we have $\frac{b}{a} = \frac{n - 5}{6}.\frac{y}{x}$, from second and third, we have
  $\frac{c}{b} = \frac{n - 6}{7}.\frac{y}{x}$, and from kast two we gace $\frac{d}{c} = \frac{n -
    7}{8}.\frac{y}{x}$.

  Now from first two we have, $\frac{b^2}{ac} = \frac{7(n - 5)}{6(n - 6)}$ and $\frac{c^2}{bd} = \frac{8(n -
    6)}{7(n - 7)}$

  Subtracting $1$ from both of these, we have $\frac{b^2 - ac}{ac} = \frac{7(n - 5) - 6(n - 6)}{6(n - 6)}$,
  and $\frac{c^2 - bd}{bd} = \frac{8(n - 6) - 7(n - 7)}{7(n - 7)}$

  Dividing, we get $\frac{b^2 - ac}{c^2 - bd} = \frac{4a}{3c}$.
\item (a) Let $a, b, c$ and $d$ be the $r$th, $(r + 1)$th, $(r + 2)$th, and $(r + 3)$th term of the binomial
  expansion $(x + y)^n$.

  $a = C_{r - 1}^^n, b = C_r^^n, c = C_{r + 1}^^n$, and $d = C_{r + 2}^^n$.

  $\frac{b}{a} = \frac{n - r + 1}{r}.\Rightarrow \frac{a + b}{a} = \frac{n + 1}{r}$, $\frac{c}{b} = \frac{n
    - r}{r + 1}\Rightarrow \frac{c + b}{b} = \frac{n + 1}{r + 1}$

  $\frac{d}{c} = \frac{n - r - 1}{r + 2}\Rightarrow \frac{c + d}{c} = \frac{n + 1}{r + 2}$. Clearly,
  $\frac{a + b}{a}, \frac{b + c}{b}, \frac{c + d}{c}$ are in H.P.

  (b) $(bc + ad)(b - c) = \frac{(n!)^3}{[(r - 1)!]^3[(n - r - 2)!]^3}\left(\frac{1}{r(n - r)(n - r
    -1)}.\frac{1}{r(r + 1)(n - r - 1)} - \frac{1}{(n - r + 1)(n - r)(n - r - 1)}.\frac{1}{r(r + 1)(r +
      2)}\right)\left(\frac{1}{r(n - r)(n - r - 1)} - \frac{1}{r(r + 1)(n - r - 1)}\right)$

  Now it is trivial to prove that $(bc + ad)(b - c) = 2(ac^2 - b^2d)$.
\item The coefficients of $5$th, $6$th and $7$th terms in the expansion of $(1 + x)^n$ are $C_4^^n, C_5^^n$
  and $C_6^^n$. Given that these are in A.P., therefore

  $2C_5^^n = C_4^^n + C_6^^n \Rightarrow \frac{2}{5(n - 5)} = \frac{1}{(n - 4)(n - 5)} + \frac{1}{5.6}
  \Rightarrow n = 7, 14$.
\item The coefficients of second, third and fourth terms in the expansion of $(1 + x)^{2n}$ are in A.P.

  $\Rightarrow 2C_2^^{2n} = C_1^^{2n} + C_3^^{2n}\Rightarrow \frac{2}{2(2n - 2)} = \frac{1}{(2n - 1)(2n -
  2)} + \frac{1}{2.3}$

  $\Rightarrow 2n^2 - 9n + 7 = 0$.
\item The coefficients of $r$th, $(r + 1)$th and $(r + 2)$th terms in the expansion of $(1 + x)^n$ are in
  A.P.

  $\Rightarrow 2C_r^^n = C_{r - 1}^^n + C_{r + 1}^^n \Rightarrow \frac{2}{r(n - r)} = \frac{1}{(n - r)(n - r
    + 1)} + \frac{1}{r(r + 1)}\Rightarrow n^2 - n(4r + 1) + 4r^2 - 2 = 0$.
\item Let the coefficients of $r$th, $(r + 1)$th and $(r + 2)$th terms in the expansion of $(1 + x)^n$ are
  in the ratio of $182:84:30$.

  $t_r = C_{r - 1}^^n, t_{r + 1} = C_r^^n$ and $t_{r + 2} = C_{r + 1}^^n$

  From first two, we have $\frac{r}{n - r + 1} = \frac{13}{6}$, and from last two we have $\frac{r + 1}{n -
    r} = \frac{14}{5}$

  From these two equations we have $n = 18$.
\item Given series is $C_1 + 2.C_2 + 3.C_3 + \cdots + n.C_n$. Its $r$th term $t_r = r.C_r^^n = n.C_{r -
  1}^^{n - 1}[\because r.C_r^^n = n.C_{r - 1}^^{n - 1}]$

  Now $C_1 + 2.C_2 + 3.C_3 + \cdots + n.C_n = \displaystyle\sum_{r = 1}^nr.C_r^^n = \sum_{r = 1}^nn.C_{r -
    1}^^{n - 1}$

  $= n[C_0^^{n - 1} + C_1^^{n - 1} + C_2^^{n - 1} + \cdots + C_{n - 1}^^{n - 1}] = n(1 + 1)^{n - 1} = n.2^{n
    - 1}$.

  {\bf Calculus Method:} $(1 + x)^n = C_0 + C_1.x + C_2.x^2 + \cdots + C_n.x^n$

  Differentiating w.r.t. $x$, we get $n(1 + x)^{n - 1} = C_1 + 2C_2.x + \cdots + nC_n.x^{n - 1}$

  Putting $x = 1$, we get $n.2^{n - 1} = C_1 + 2.C_2 + 3.C_3 + \cdots + n.C_n$.
\item Given series is $C_0 + 2.C_1 + 3.C_2 + \cdots + (n + 1).C_n$.

  Its $r$th term is $t_r = r.C_{r - 1}^^n = (r - 1).C_{r - 1}^^n + C_{r - 1}^^n = n.C_{r - 2}^^{n - 1} +
  C_{r - 1}^^n[\because (r - 1).C_{r - 1}^^n = n.C_{r - 2}^^{n - 1}]$

  Now, $C_0 + 2.C_1 + 3.C_2 + \cdots + (n + 1).C_n = \displaystyle\sum_{r = 1}^{n + 1}t_r = \sum_{r = 1}^{n
    + 1}n.C_{r - 2}^^{n - 1} + \sum_{r = 1}^^{n + 1}C_{r - 1}^^n$

  $= n[C_0^^{n - 1} + C_1^^{n - 1} + \cdots + C_{n - 1}^^{n - 1}] + (C_0^^n + C_1^^n + \cdots + C_n^^n)$

  $= n.2^{n - 1} + 2^n = 2^{n - 1}(n + 2)$.

  {\bf Calculus Method:} $(1 + x)^n = C_0 + C_1.x + C_2.x^2 + \cdots + C_n.x^n$

  Multiplying with $x$, we get $x(1 + x)^n = C_0.x + C_1.x^2 + C_2.x^3 + \cdots + C_n.x^{n + 1}$

  Differentiating w.r.t. $x$, we get $(1 + x)^n + nx(1 + x)^{n - 1} = C_0 + 2C_1.x + 3C_2.x^2 + \cdots + (n
  + 1)C_n.x^n$

  Putting $x = 1$, we get $C_0 + 2.C_1 + 3.C_2 + \cdots + (n + 1).C_n = 2^{n - 1}(n + 2)$.
\item Given series is $C_0 + 3.C_1 + 5.C_2 + \cdots + (2n + 1).C_n$.

  Its $r$th term is $t_r = (2r - 1)C_{r - 1} = [2(r - 1) + 1]C_{r - 1} = 2(r - 1)C_{r - 1} + C_{r - 1}$

  $= 2.nC_{r - 2}^^{n - 1} + C_{r - 1}[\because (r - 1)C_{r - 2} = n.C_{r - 2}^^{n - 1}]$

  Now, $C_0 + 3.C_1 + 5.C_2 + \cdots + (2n + 1).C_n = \displaystyle\sum_{r = 1}^{n + 1}t_r = 2n\sum_{r = 1}^{n
    + 1}C_{r - 2}^^{n - 1} + \sum_{r = 1}^^{n + 1}C_{r - 1}$

  $= 2n(C_0^^{n - 1} + C_1^^{n - 1} + \cdots + C_{n - 1}^^{n - 1}) + (C_0 + C_1 + \cdots + C_n)$

  $= 2n.2^{n - 1} + 2^n = 2^n(n + 1)$.

  {\bf Calculus Method:} $(1 + x)^n = C_0 + C_1.x + C_2.x^2 + \cdots + C_n.x^n$

  Putting $x = x^2$ and multiplying with $x$, we get

  $x(1 + x^2)^n = C_0.x + C_1.x^3 + C_2.x^5 + \cdots + C_n.x^{2n + 1}$

  Differentiating both sides w.r.t. $x$, we get

  $(1 + x^2)^n + 2x^2.n(1 + x^2)^{n - 1} = C_0 + C_1.3x^2 + C_2.5x^4 + \cdots + C_n.(2n + 1)x^{2n + 1}$

  Putting $x = 1$, we get

  $C_0 + 3.C_1 + 5.C_2 + \cdots + (2n + 1).C_n = 2n.2^{n - 1} + 2^n = 2^n(n + 1)$.
\item We have to prove that $C_1 - 2.C_2 + 3.C_3 - 4.C_4 + \cdots + (-1)^{n - 1}n.C_n = 0$.

  $r$th term $t_r = (-1)^{r - 1}r.C_r^^n = (-1)^{r - 1}.nC_{r - 1}^^{n - 1}$

  $\displaystyle\sum_{r = 1}^nt_r = \sum_{r = 1}^n(-1)^{r - 1}n.C_{r - 1} = n.\sum_{r = 1}^n(-1)^{r - 1}C_{r
  - 1}$

  $= n(C_0^^{n - 1} - C_1^^{n - 1} + C_2^^{n - 1} - C_3^^{n - 1} + \cdots + (-1)^{n - 1}C_{n - 1}^^{n - 1})$

  $= n(1 - 1)^{n - 1} = 0$.

  {\bf Calculus Method:} $(1 + x)^n = C_0 + C_1.x + C_2.x^2 + \cdots + C_n.x^n$.

  Differentiating both sides w.r.t. $x$, we get

  $n(1 + x)^{n - 1} = C_1 + 2x.C_2 + 3x^2C_3 + \cdots + nx^{n - 1}C_n$

  Putting $x = -1$, we get

  $n(1 - 1)^{n - 1} = C_1 - 2C_2 + 3C_3 - \cdots + (-1)^{n - 1}.nC_n = 0$.
\item Given series is $C_0 + \frac{C_1}{2} + \frac{C_3}{3} + \cdots + \frac{C_n}{n + 1}$.

  Its $r$th term is $t_r = \frac{C_{r - 1}^^n}{r}$.

  $C_0 + \frac{C_1}{2} + \frac{C_2}{3} + \cdots + \frac{C_n}{n + 1} = \displaystyle\sum_{r = 1}^{n +
    1}\frac{C_{r - 1}^^n}{r} = \sum_{r = 1}^^{n + 1}\frac{C_r^^{n + 1}}{n + 1}\left[\because \frac{C_{r -
        1}^^n}{r} = \frac{C_r^^{n + 1}}{n + 1}\right]$

  $= \frac{1}{n + 1}(C_1^^{n + 1} + C_2^^{n + 2} + \cdots + C_{n + 1}^^{n + 1}) = \frac{2^{n + 1} - 1}{n +
    1}[\because$ we add and subtract $C_0^^{n + 1}]$.

  {\bf Calculus Method:} $(1 + x)^n = C_0 + C_1x + C_2x^2 + \cdots + C_nx^n$

  Integrating within limits $0$ and $1$, we have

  $\left[\frac{(1 + x)^n}{n}\right]_0^1 = \left[C_0x + C_1\frac{x^2}{2} + C_3\frac{x^3}{3} + \cdots +
    C_n\frac{x^{n + 1}}{n + 1}\right]_0^1$

  $\Rightarrow \frac{2^{n + 1} - 1}{n + 1} = C_0 + \frac{C_1}{2} + \frac{C_2}{3} + \cdots + \frac{C_n}{n + 1}$.
\item Given series is $C_0 - \frac{C_1}{2} + \frac{C_2}{3} - \cdots + (-1)^n\frac{C_n}{n + 1} = \frac{1}{n +
  1}$.

  Its $r$th term $t_r = (-1)^r\frac{C_{r - 1}^^n}{r}$

  Now $C_0 - \frac{C_1}{2} + \frac{C_2}{3} - \cdots + (-1)^n\frac{C_n}{n + 1} = \displaystyle\sum_{r = 1}^{n
    + 1}(-1)^{r - 1}\frac{C_{r - 1}^n}{r}$

  $\displaystyle= \sum_{r = 1}^^{n + 1}(-1)^{r - 1}\frac{C_r^^{n + 1}}{n +
  1}\left[\because \frac{C_{r - 1}^^n}{r} = \frac{C_r^^{n + 1}}{n + 1}\right]$

  $= \frac{1}{n + 1}[C_1^^{n + 1} - C_2^^{n + 1} + C_3^^{n + 1} - \cdots + (-1)^nC_{n + 1}^^{n + 1}]$

  $= \frac{1}{n + 1}[-(1 - 1)^{n + 1} + C_0^^{n + 1}] = \frac{1}{n + 1}$.

  {\bf Calculus Method:} $(1 + x)^n = C_0 + C_1x + C_2x^2 + \cdots + C_nx^n$

  Integrating within limits $0$ and $-1$, we have

  $\left[\frac{(1 + x)^n}{n}\right]_0^{-1} = \left[C_0x + C_1\frac{x^2}{2} + C_3\frac{x^3}{3} + \cdots +
    C_n\frac{x^{n + 1}}{n + 1}\right]_0^{-1}$

  $\Rightarrow -\frac{1}{n + 1} = -\left[C_0 - \frac{C_1}{2} + \frac{C_2}{3} - \cdots + (-1)^n\frac{C_n}{n +
      1}\right]$

  $\Rightarrow C_0 - \frac{C_1}{2} + \frac{C_2}{3} - \cdots + (-1)^n\frac{C_n}{n + 1} = \frac{1}{n + 1}$.
\item Given series is $\frac{C_1}{2} + \frac{C_3}{4} + \frac{C_5}{6} + \cdots$

  $t_r = \frac{C_{2r - 1}^^n}{2r} = \frac{C_{2r}^^{n  + 1}}{n + 1}$

  Now, $\frac{C_1}{2} + \frac{C_3}{4} + \frac{C_5}{6} + \cdots = \frac{1}{n + 1}\displaystyle\sum_{r =
    1}C_{2r}^^{n + 1} = \frac{1}{n + 1}[C_2^^{n + 1} + C_4^^{n + 1} + C_6^^{n + 1} + \cdots]$

  $= \frac{2^n - 1}{n + 1}[\because C_0^^n + C_2^^n + C_4^^n + \cdots = 2^{n - 1}]$.

  {\bf Calculus Method:} Adding the results of last two problems, we have

  $2\left[\frac{C_1}{2} + \frac{C_3}{4} + \frac{C_5}{6} + \cdots\right] = \frac{2^{n + 1} - 1 - 1}{n + 1} =
  \frac{2(2^n - 1)}{n +1}$

  $\Rightarrow \frac{C_1}{2} + \frac{C_3}{4} + \frac{C_5}{6} + \cdots = \frac{2^n - 1}{n + 1}$.
\item Given series is $2.C_0 + 2^2.\frac{C_1}{2} + 2^3.\frac{C_2}{3} + \cdots + 2^{n + 1}.\frac{C_n}{n + 1}$

  $t_r = 2^r.\frac{C_{r - 1}^^n}{r} = 2^r.\frac{C_r^^{n + 1}}{n + 1}$.

  Now, $2.C_0 + 2^2.\frac{C_1}{2} + 2^3.\frac{C_2}{3} + \cdots + 2^{n + 1}.\frac{C_n}{n + 1} =
  \frac{1}{n + 1}\displaystyle\sum_{r = 1}^^{n + 1}C_r^^{n + 1}.2^r$

  $= \frac{1}{n + 1}[C_1^^{n + 1}.2 + C_2^^{n + 1}.2^2 + C_3^^{n + 1}.2^3 + \cdots + C_{n + 1}^^{n + 1}.2^{n
      + 1}]$

  $=\frac{1}{n + 1}[(1 + 2)^{n + 1} - 1] = \frac{3^{n + 1} - 1}{n + 1}$.

  {\bf Calculus Method:} $(1 + x)^n = C_0 + C_1x + C_2x^2 + \cdots + C_nx^n$

  Integrating within the limits of $0$ and $2$, we have

  $\left[C_0x + C_1.\frac{x^2}{2} + C_2.\frac{x^3}{3} + \cdots + C_n..\frac{x^{n + 1}}{n + 1}\right]_0^2 =
  \left[\frac{(1 + x)^{n + 1}}{n + 1}\right]_0^2$

  $\Rightarrow C_0.2 + \frac{C_1}{2}.2^2 + \frac{C_2}{3}.2^3 + \cdots + \frac{C_n}{n + 1}.2^{n + 1} =
  \frac{3^{n + 1} - 1}{n + 1}$.
\item $(1 + x)^n = C_0 + C_1x + C_2x^2 + \cdots + C_nx^n$, and $(x + 1)^n = C_0x^n + C_1x^{n - 1} + C_2x^{n
  - 2} + \cdots + C_n$

  Multiplying these two, we have $(1 + x)^{2n} = (C_0 + C_1x + C_2x^2 + \cdots + C_nx^n)(C_0x^n + C_1x^{n -
    1} + C_2x^{n - 2} + \cdots + C_n)$

  Coeff.\ of $x^{n + r}$ on R.H.S.\ $= C_0c_r + C_1C_{r + 1} + \cdots + C_{n - r}C_n$, and on L.H.S.\ $=
  C_{n + r}^^{2n} = \frac{(2n)!}{(n + r)!(n - r)!}$.
\item $(1 + x)^{2n} = (C_0 + C_1x + C_2x^2 + \cdots + C_nx^n)(C_0x^n + C_1x^{n - 1} + C_2x^{n - 2} + \cdots
  + C_n)$

  Equating the coeff. of $x^n$, we have

  $C_0^2 + C_1^2 + C_2^2 + \cdots + C_n^2 = C_n^^{2n} = \frac{2n!}{n!n!}$.
\item $t_r = r.\frac{C_r^^n}{C_{r - 1}^^n} = n - r + 1$

  L.H.S.\ $= t_1 + t_2 + t_3 + \cdots + t_n = n + (n - 1) + (n - 2) + \cdots + 1 = \frac{n(n + 1)}{2}$.
\item $[(1 + x)^n]^2 = (C_0 + C_1.x + C_2.x^2 + \cdots + C_n.x^n)^2$, and $(1 + x)^{2n} = (C_0^^{2n} +
  C_1^^{2n}.x + C_2^^{2n}.x^2 + \cdots + C_{2n}^^{2n}.x^{2n})$

  Putting $x = 1$, it is trivial to see that

  $(1 + C_1 + C_2 + \cdots + C_n)^2 = 1 + C_1^^{2n} + C_2^^{2n} + \cdots + C_{2n}^^{2n}$.
\item $[(1 + x)^n]^5 = (C_0 + C_1.x + C_2.x^2 + \cdots + C_n.x^n)^5$, and $(1 + x)^{5n} = (C_0^^{5n} +
  C_1^^{5n}.x + C_2^^{5n}.x^2 + \cdots + C_{5n}^^{5n}.x^{5n})$

  Putting $x = 1$, it is trivial to see that

  $(1 + C_1 + C_2 + \cdots + C_n)^5 = 1 + C_1^^{5n} + C_2^^{5n} + \cdots + C_{5n}^^{5n}$.
\item We have to prove that $C_0 + 5.C_1 + 9.C_2 + \cdots + (4n + 1).C_n = (2n + 1)2^n$.

  L.H.S.\ $= 4.0.C_0 + 4.1.C_1 + 4.2.C_2 + \cdots + 4nC_n + C_0 + C_1 + C_2 + \cdots + C_n$

  We know that $C_0 + C_1 + C_2 + \cdots + C_n = 2^n$.

  $C_1 + 2.C_2 + 3.C_3 + \cdots + n.C_n$. Its $r$th term $t_r = r.C_r^^n = n.C_{r -
  1}^^{n - 1}[\because r.C_r^^n = n.C_{r - 1}^^{n - 1}]$

  Now $C_1 + 2.C_2 + 3.C_3 + \cdots + n.C_n = \displaystyle\sum_{r = 1}^nr.C_r^^n = \sum_{r = 1}^nn.C_{r -
    1}^^{n - 1}$

  $= n[C_0^^{n - 1} + C_1^^{n - 1} + C_2^^{n - 1} + \cdots + C_{n - 1}^^{n - 1}] = n(1 + 1)^{n - 1} = n.2^{n
    - 1}$

  Multiplying with $4$, we have

  $4.0.C_0 + 4.1.C_1 + 4.2.C_2 + \cdots + 4nC_n = 2n.2^n$.

  Thus, $C_0 + 5.C_1 + 9.C_2 + \cdots + (4n + 1).C_n = (2n + 1)2^n$.
\item We have to prove that $1 - (1 + x)C_1 + (1 + 2x)C_2 - (1 + 3x)C_3 + \cdots = 0$.

  $\Rightarrow 1 - 1 + 1 - 1 + \cdots - x[C_1 - 2.C_2 + 3.C_3 - \cdots] = 0$

  That is we have to prove that $C_1 - 2.C_2 + 3.C_3 - \cdots = 0$, which has been proven earlier.
\item We have to prove that $3.C_1 + 7.C_2 + \cdots + (4n - 1).C_n = (2n - 1)2^n$.

  L.H.S.\ $= 4.0.C_0 + 4.1.C_1 + 4.2.C_2 + \cdots + 4nC_n - [C_0 + C_1 + C_2 + \cdots + C_n]$

  We know that $C_0 + C_1 + C_2 + \cdots + C_n = 2^n$.

  $C_1 + 2.C_2 + 3.C_3 + \cdots + n.C_n$. Its $r$th term $t_r = r.C_r^^n = n.C_{r -
  1}^^{n - 1}[\because r.C_r^^n = n.C_{r - 1}^^{n - 1}]$

  Now $C_1 + 2.C_2 + 3.C_3 + \cdots + n.C_n = \displaystyle\sum_{r = 1}^nr.C_r^^n = \sum_{r = 1}^nn.C_{r -
    1}^^{n - 1}$

  $= n[C_0^^{n - 1} + C_1^^{n - 1} + C_2^^{n - 1} + \cdots + C_{n - 1}^^{n - 1}] = n(1 + 1)^{n - 1} = n.2^{n
    - 1}$

  Multiplying with $4$, we have

  $4.0.C_0 + 4.1.C_1 + 4.2.C_2 + \cdots + 4nC_n = 2n.2^n$.

  Thus, $3.C_1 + 7.C_2 + \cdots + (4n - 1).C_n = (2n - 1)2^n$.
\item We have to prove that $C_0 + \frac{C_2}{3} + \frac{C_4}{5} + \cdots = \frac{2^n}{n + 1}$.

  $t_r = \frac{C_{2r - 2}^^n}{2r - 1} = \frac{C_{2r - 1}^^{n + 1}}{n + 1}$

  Now $C_0 + \frac{C_2}{3} + \frac{C_4}{5} + \cdots = \frac{2^n}{n + 1} = \frac{1}{n + 1}[C_0^^{n + 1} +
    C_3^^{n + 1} + C_{5}^^{n + 1} + \cdots]$

  $= \frac{2^n}{n + 1}$.
\item We know that $(1 + x)^n = C_0^^n + C_1^^nx + C_2^^nx^2 + \cdots + C_n^^nx^n$ and $(1 + x)^{n + 1} =
  C_0^^{n + 1} + C_1^^{n + 1}x + C_2^^{n + 2}x^2 + \cdots + C_{n + 1}^^{n + 1}x^{n + 1}$.

  Multiplying equations above and equating coeff.\ of $x^{n + 1}$, we get

  $C_0^^nC_1^^{n + 1} + C_1^^{n}C_2^^{n + 1} + \cdots + C_n^^{n}C_{n + 1}^^{n + 1} = C_{n + 1}^{2n + 1} =
  \frac{(2n + 1)!}{(n + 1)!n!}$.
\item Given series is $C_0 - 2.C_1 + 3.C_2 - \cdots + (-1)^n(n + 1)C_n = 0$.

  $t_r = (-1)^{r - 1}.rC_{r - 1}^^n = (-1)^r(r - 1)C_{r - 1}^^n + (-1)^{r - 1}C_{r - 1}^^n = (-1)^{r -
  1}.nC_{r - 2}^^{n - 1} + (-1)^{r - 1}C_{r - 1}^^n$.

  $C_0 - 2.C_1 + 3.C_2 - \cdots + (-1)^n(n + 1)C_n = \displaystyle\sum_{r = 1}^^{n + 1}(-1)^{r - 1}.nC_{r -
  2}^^{n - 1} + \sum_{r = 1}^^{n + 1}(-1)^{r - 1}C_{r - 1}^^n$

  $= -n(C_0^^{n - 1} - C_1^^{n - 1} + \cdots + (-1)^{n - 1}C_{n - }^^{n - 1}) + (C_0^^n - C_1^^n + C_2^^n +
  \cdots + (-1)^{n - 1}C_n^^n)$

  $= -n(1 - 1)^{n - 1} + (1 - 1)^n = 0$.

  {\bf Calculus Method:} Given $(1 + x)^n = C_0 + C_1x + C_2x^2 + \cdots + C_nx^n$

  Multiplying with $x$, and differentiating w.r.t. $x$, we get

  $1.(1 + x)^n + x.n(1 + x)^{n - 1} = C_0 + C_1.2x + C_2.3x^2 + \cdots + (n + 1)C_n.x^n$

  Putting $x = -1$, we get

  $C_0 - 2.C_1 + 3.C_2 - \cdots + (-1)^n(n + 1)C_n = 0$.
\item Given series is $C_0 -3.C_1 + 5.C_2 - \cdots + (-1)^n(2n + 1)C_n = 0$.

  $t_r = (-1)^{r - 1}(2r - 1)C_{r - 1} = (-1)^{r - 1}[2(r - 1) + 1]C_{r - 1} = 2(-1)^{r - 1}C_{r - 1}^^n +
  (-1)^{r - 1}C_{r - 1}^^n = 2(-1)^{r - 1}.nC_{r - 2}^^{n - 1} + (-1)^{r - 1}.C_{r - 1}^^n$

  $C_0 -3.C_1 + 5.C_2 - \cdots + (-1)^n(2n + 1)C_n = \displaystyle 2n\sum_{r = 1}^^{n + 1}(-1)^{r - 1}C_{r -
  2}^^{n - 1} + \sum_{r = 1}^^{n + 1}C_{r - 1}^^n$

  $= -2n[C_0^^{n - 1} - C_1^^{n - 1} + \cdots + (-1)^nC_{n - 1}^^{n - 1}] + [C_0 - C_1 + C_2 - \cdots +
  (-1)^nC_n]$

  $= -2n(1 - 1)^{n - } + (1 - 1)^n = 0$.

  {\bf Calculus Method:} We know that $(1 + x^2)^n = C_0 + C_1x^2 + C_2x^4 + \cdots + C_nx^{2n}$

  Multiplying both sides with $x$, and differentiating w.r.t. $x$, we get

  $C_0 + 3C_1x^2 + 5.C_2x^4 + \cdots + (2n + 1)C_nx^{2n} = (1 + x^2)^n + nx.2x(1 + x^2)^n$

  Putting $x = -1$, we have

  $C_0 -3.C_1 + 5.C_2 - \cdots + (-1)^n(2n + 1)C_n = 0$.
\item Given series is $a - (a - 1)C_1 + (a - 2)C_2 - (a - 3)C_3 + \cdots + (-1)^n(a - n)C_n = 0$.

  $t_r = (-1)^{r - 1}[a + 1 - r]C_{r - 1}^^n = (-1)^{r - 1}[a - (r - 1)]C_{r - 1} = a(-1)^{r - 1}C_{r - 1} -
  (-1)^{r - 1}n.C_{r - 2}^^{n - 1}$.

  $a - (a - 1)C_1 + (a - 2)C_2 - (a - 3)C_3 + \cdots + (-1)^n(a - n)C_n = a\displaystyle\sum_{r = 1}^{n +
  1}(-1)^{r - 1}C_{r - 1} - n\sum_{r = 1}^{n + 1}.C_{r - 2}^^{n + 1}$

  $= a[C_0 - C_1 + C_2 - \cdots + (-1)^nC_n] - n[C_0^^{n - 1} - C_1^^{n - 1} + C_2^^{n - 1} - \cdots +
  (-1)^nC_{n - 1}^^{n - 1}]$

  $= a(1 - 1)^n - n(1 - 1)^{n - 1} = 0$.

  {\bf Calculus Method:} L.H.S.\ $= a - (a - 1)C_1 + (a - 2)C_2 - (a - 3)C_3 + \cdots + (-1)^n(a - n)C_n$

  $= a[C_0 - C_1 + C_2 - \cdots + (-1)^{n}C_n] + 1[C_1 - 2.C_2 + 3.C_3 - \cdots + (-1)^n(-n)C_n]$

  We have shown both the series in brackets equal to zero earlier.
\item Given series is $1^2.C_1 + 2^2.C_2 + 3^2C_3 + \cdots + n^2.C_n = n(n + 1)2^{n - 2}$.

  $t_r = r^2.C_r = r.n.C_{r - 1}^^{n - 1} = n.[r - 1 + 1]C_{r - 1}^^{n - 1} = n(n - 1).C_{r - 2}^{n - 2} +
  n.C_{r - 2}^^{n - 2}$.

  $1^2.C_1 + 2^2.C_2 + 3^2C_3 + \cdots + n^2.C_n = n(n - 1)[C_0^^{n - 2} + C_1^^{n - 2} + \cdots + C_{n -
      2}^^{n - 2}] + n[C_0^^{n - 1} + C_1^^{n - 1} + C_2^^{n - 2} + \cdots + C_{n - 1}^^{n - 1}]$

  $= n(n - 1).2^{n - 2} + n.2^{n - 1} = 2^{n - 2}[n(n - 1) + 2n] = n(n + 1).2^{n - 2}$.

  {\bf Calculus Method:} $(1 + x)^n = C_0 + C_1x + C_2x^2 + \cdots + C_nx^n$

  Differentiating w.r.t.\ $x$, we have

  $n(1 + x)^{n - 1} = C_1 + 2.C_1x + 3.C_2x^2 + \cdots + n.C_nx^{n - 1}$

  Multiplying both sides w.r.t.\ $x$, and differentiating again, we have

  $n[(1 + x)^{n - 1} + (n - 1)x(1 + x)^{n - 2}] = 1^2.C_1 + 2^2.C_1x + 3^2C_2x^2 + \cdots + n^2.C_nx^{n -
    1}$

  Putting $x = 1$, we arrive at

  $1^2.C_1 + 2^2.C_2 + 3^2C_3 + \cdots + n^2.C_n = n[2^{n - 1} + (n - 1)2^{n - 2}] = n(n + 1)2^{n - 2}$.
\item Given series is $C_0 - 2^2.C_1 + 3^2.C_2 - \cdots + (-1)^n(n + 1)^2C_n = 0,\;n>2$.

  $t_{r + 1} = (-1)^{r}r^2C_r = (-1)^{r}[r^2 + 2r + 1]C_r$

  $\displaystyle\sum_{r = 0}^n(-1)^r.C_r^^n = C_0 - C_1 + C_2 - \cdots + (-1)^n.C_n = 0$

  $\displaystyle\sum_{r = 0}^n(-1)^r.rC_{r} = \sum_{r = 0}^n(-1)^r.nC_{r - 1}^^{n - 1} = -n[C_0^^{n - 1} -
  C_1^^{n - 1} + C-2^^{n - 1} - \cdots + (-1)^{n - 1}C_{n - 1}^^{n - 1}]$

  $= -n(1 - 1)^{n - 1} = 0$.

  $\displaystyle\sum_{r = 0}^{n}(-1)^r.r^2C_r^^n = \sum_{r = 0}^n(-1)^r.r.nC_{r - 1}^^{n - 1} = n\sum_{r =
  0}^n(-1)^r(r - 1)C_{r - 1}^^{n - 1} + n\sum_{r = 0}^nC_{r - 1}^^{n - 1}$

  $= n(n - 1)\displaystyle\sum_{r = 0}^^n(-1)^{r}.C_{r - 2}^^{n - 2} + n.0 = 0$.

  Thus, $\displaystyle\sum_{r = 0}^nt_r = 0$.
\item Given series is $C_0.abc - C_1(a - 1)(b - 1)(c - 1) + C_2(a - 2)(b - 2)(c -2)- \cdots +
  (-1)^n.C_n(a - n)(b - n)(c - n) = 0$.

  $t_{r + 1} = (-1)^r[abc + (a + b + c)r^2 - (ab + bc + ca)r - r^3]C_r^^n$.

  $\displaystyle\sum_{r = 0}^n(-1)^r.C_r^^n = C_0 - C_1 + C_2 - \cdots + (-1)^n.C_n = 0$

  $\displaystyle\sum_{r = 0}^n(-1)^r.rC_{r} = \sum_{r = 0}^n(-1)^r.nC_{r - 1}^^{n - 1} = -n[C_0^^{n - 1} -
  C_1^^{n - 1} + C-2^^{n - 1} - \cdots + (-1)^{n - 1}C_{n - 1}^^{n - 1}]$

  $= -n(1 - 1)^{n - 1} = 0$.

  $\displaystyle\sum_{r = 0}^{n}(-1)^r.r^2C_r^^n = \sum_{r = 0}^n(-1)^r.r.nC_{r - 1}^^{n - 1} = n\sum_{r =
  0}^n(-1)^r(r - 1)C_{r - 1}^^{n - 1} + n\sum_{r = 0}^nC_{r - 1}^^{n - 1}$

  $= n(n - 1)\displaystyle\sum_{r = 0}^^n(-1)^{r}.C_{r - 2}^^{n - 2} + n.0 = 0$.

  Similarly, $\displaystyle\sum_{r = 0}^n(-1)^r.r^3.C_r^^n = 0$. And thus $\displaystyle\sum_{r= 0}^n t_r =
  0$.
\item $t_{r + 1} = r^2.C_rp^rq^{n - r} = r.n.C_{r - 1}^^{n - 1}p^rq^{n - r} = n.(r - 1 + 1)C_{r - 1}^^{n -
  1}p^rq^{n - r}$

  $= n[(n - 1)C_{r - 2}^^{n - 2} + C_{r - 1}^{n - 1}]p^rq^{n - r} = n(n - 1)p^2C_{r - 2}^^{n - 2}p^{r -
  2}q^{n - 2 - (r - 2)} + np^{n - 1}.C_{r - 1}^^{n - 1}p^{r - 1}q^{n - 1 - (r - 1)}$

  $\therefore \displaystyle\sum_{r = 0}^nt_{r + 1} = n(n - 1)p^2(p + q)^{n- 2} + np(p + q)^{n - 1} = n(n -
  1)p^2 + np = n^2p^2 + npq$.
\item $(1 + x)^{10} = C_0 + C_1x + C_2x^2 + \cdots + C_{10}x^{10}$

  Integrating between limits of $0$ and $2$, we have

  $\left[C_0x +C_1\frac{x^2}{2} + C_2\frac{x^3}{3} + \cdots + C_{10}\frac{x^{11}}{11}\right]_0^2 =
  \left[\frac{(1 + x)^{11}}{11}\right]_0^2$

  $\Rightarrow 2.C_0 + \frac{2^2}{2}.C_1 + \frac{2^3}{3}.C_2 + \cdots + \frac{2^{11}}{11}.C_{11} =
  \frac{3^{11} - 1}{11}$.
\item Given series is $\frac{2^2}{1.2}C_0 + \frac{2^3}{2.3}C_2 + \frac{2^4}{3.4}C_2 + \cdots + \frac{2^{n +
    2}}{(n + 1)(n + 2)}C_n = \frac{3^{n + 2} - 2n -5}{(n + 1)(n + 2)}$.

  $t_r = \frac{2^{r + 1}}{r(r + 1)}C_{r - 1}^^n = \frac{2^{r + 1}}{r + 1}.\frac{C_r^^{n + 1}}{n + 1} =
  \frac{2^{r + 1}}{r(r + 1)}.\frac{C_r^^{n + 1}}{r + 1} = \frac{2^{r + 1}}{n + 1}\frac{C_{r + 1}^^{n + 2}}{n
    + 2}$

  $\Rightarrow \frac{2^2}{1.2}C_0 + \frac{2^3}{2.3}C_2 + \frac{2^4}{3.4}C_2 + \cdots + \frac{2^{n +
    2}}{(n + 1)(n + 2)}C_n = \frac{1}{(n + 1)(n + 2)}\displaystyle\sum_{r = 1}^{n + 1}2^{r + 1}.C_{r +
    1}^^{n + 2}$

  $= \frac{1}{(n + 1)(n + 2)}[2^2.C_2^^{n + 2} + 2^3.C_3^^{n + 2} + \cdots + 2^{n + 2}.C_{n + 2}^^{n + 2}]$

  $\frac{1}{(n + 1)(n + 2)}[(1 + 2)^{n + 2} - C_0^^{n + 2} - C_1^^{n + 2}]$

  $= \frac{3^{n + 2} - 2n -5}{(n + 1)(n + 2)}$.

  {\bf Calculus Method:} $(1 + x)^n = C_0 + C_1x + C_2x^2 + \cdots + C_nx^n$.

  Integrating within the limits of $0$ and $x$, we have

  $\left[C_0x + C_1\frac{x^2}{2} + C_2\frac{x^3}{3} + \cdots + C_n\frac{x^{n + 1}}{n + 1}\right]_0^x=
  \left[\frac{(1 + x)^{n + 1}}{n + 1}\right]_0^x$

  $\Rightarrow \left[C_0x + C_1\frac{x^2}{2} + C_2\frac{x^3}{3} + \cdots + C_n\frac{x^{n + 1}}{n +
      1}\right]_0^x= \frac{(1 + x)^{n + 1}}{n + 1} - \frac{1}{n + 1}$

  Integrating again within the limits of $0$ and $2$, we arrive at

  $\frac{2^2}{1.2}C_0 + \frac{2^3}{2.3}C_2 + \frac{2^4}{3.4}C_2 + \cdots + \frac{2^{n +
      2}}{(n + 1)(n + 2)}C_n = \left[\frac{(1 + x)^{n + 2}}{(n + 1)(n + 2)} - \frac{x}{n + 1}\right]_0^2$

  $= \frac{3^{n + 2}}{(n + 1)(n + 2)} - \frac{2}{n + 1} - \frac{1}{(n + 2)(n + 2)} = \frac{3^{n + 2} - 2n
    -5}{(n + 1)(n + 2)}$.
\item Let $S_n = C_1 - \frac{C_2}{2} + \frac{C_3}{3} - \cdots + \frac{(-1)^nC_n}{n} = n -
  \frac{1}{2}.\frac{n(n - 1)}{2!} + \frac{1}{3}.\frac{n(n - 1)(n - 2)}{3!} - \cdots$

  $= (n - 1 + 1) - \frac{1}{2}.\frac{(n - 1)(n - 2 + 2)}{2!} + \frac{1}{3}.\frac{(n - 1)(n - 2)(n - 3 +
    3)}{3!} - \cdots$

  $= \left[(n - 1) - \frac{1}{2}.\frac{(n - 1)(n - 2)}{2!} + \frac{1}{3}.\frac{(n - 1)(n - 2)(n - 3)}{3!} -
    \cdots\right] + \left[1 - \frac{n - 1}{2!} + \frac{(n - 1)(n - 2)}{3!} - \cdots\right]$

  $= S_{n - 1} + \frac{1}{n}[C_1 - C_2 + C_3 - \cdots] = S_{n - 1} + \frac{1}{n}\Rightarrow S_n - S_{n - 1}
  = \frac{1}{n}$

  And thus, $S_{n - 1} - S_{n - 2} = \frac{1}{n - 1}, S_{n - 2} - S_{n - 3} = \frac{1}{n - 2}, \ldots, S_2 -
  S_1 = \frac{1}{2}, S_1 = 1$.

  Adding, we get $S_n = 1 + \frac{1}{2} + \frac{1}{3} + \cdots + \frac{1}{n}$.

  {\bf Calculus Method:} $1 - C_1x + C_2x^2 - C_3x^3 + \cdots + (-1)^nC_nx^n = (1 - x)^n$

  $\Rightarrow C_1 - C_2x + C_3x^2 - \cdots + (-1)^{n - 1}C_nx^{n - 1} = \frac{1 - (1 - x)^n}{x}$

  Integrating between the limits of $0$ and $1$, we arrive at

  $\left[C_1x - C_2.\frac{x^2}{2} + C_3.\frac{x^3}{3} - \cdots + (-1)^{n - 1}C_n.\frac{x^n}{n}\right]_0^1 =
  \displaystyle\int_0^1\frac{1 - (1 - x)^n}{x}dx$

  Now $\displaystyle\int_0^1\frac{1 - (1 - x)^n}{x}dx = \int_0^1\frac{1 - z^n}{1 - z}dz$, where $z = 1 - x$

  $= \displaystyle\int_0^1(1 + z + z^2 + \cdots + z^n)dz = 1 + \frac{1}{2} + \frac{1}{3} + \cdots +
  \frac{1}{n}$.
\item $(1 - x)^n = C_0 - C_1x + C_2x^2 - \cdots + (-1)^nC_nx^n$

  Substituting $x = x^4$, we have

  $(1 - x^4)^n = C_0 - C_1x^4 + C_2x^8 - \cdots + (-1)^nC_nx^{4n}$

  Integrating within the limits of $0$ and $1$, we deduce

  $\left[C_0x - C_1\frac{x^5}{5} + C_2.\frac{x^9}{9} - \cdots + (-1)^nC_n\frac{x^{4n + 1}}{4n + 1}\right] =
  \displaystyle\int_0^1(1 - x^4)^ndx$

  Now we will evaluate the R.H.S. Let $I_n = \displaystyle\int_0^n(1 - x^4)^ndx = [x(1 - x^4)^n]_0^1 -
  \int_0^1x.n(1 - x^4)^{n - 1}.(-4x^3)dx$

  $= -4n\displaystyle\int_0^1(1 - x^4)^{n - 1}(1 - x^4 - 1)dx = -4nI_n + 4nI_{n - 1} \Rightarrow
  \frac{I_n}{I_{n - 1}} = \frac{4n}{4n + 1}$

  Now, $\frac{I_n}{I_0} = \frac{I_n}{I_{n - 1}}.\frac{I_{n - 1}}{I_{n -
      1}}\cdots\frac{I_3}{I_2}.\frac{I_2}{I_1}.\frac{I_1}{I_0}$

  $= \frac{4n}{4n + 1}.\frac{4n - 4}{4n - 3}.\frac{4n - 8}{4n - 7}\cdots \frac{4}{5} =
  \frac{4^n.n!}{5.9.\cdots(4n + 1)}$.

  {\bf Aliter:} Putting $x^2 = \sin\theta, 2xdx = \cos\theta d\theta \therefore dx = \frac{\cos\theta
    d\theta}{\sqrt{\sin\theta}}$

  When $x = 0, \theta = 0$ and when $x = 1, \theta = \frac{\pi}{2}$.

  $I_n = \displaystyle\int_0^{\frac{\pi}{2}}\frac{\cos^{2n + 1}\theta}{2\sqrt{\sin\theta}}d\theta =
  \int_0^{\frac{\pi}{2}}\cos^{2n}\theta.\frac{\cos\theta}{2\sqrt{\sin\theta}}d\theta$

  $= [\cos^{2n}\theta.\sqrt{\sin\theta}]_0^{\frac{\pi}{2}} - 2n\displaystyle\int_0^{\frac{\pi}{2}}\cos^{2n -
    1}\theta(-\sin\theta)\sqrt{\sin\theta}d\theta$

  $= 0 + 2n\displaystyle\int_0^{\frac{\pi}{2}}\frac{\cos^{2n -
      1}\theta.\sin^2\theta}{\sqrt{\sin\theta}}d\theta =
  4n\int_0^{\frac{\pi}{2}}\cos^{2n- 2}\theta(1 - \cos^2\theta).\frac{\cos\theta}{2\sqrt{\sin\theta}}d\theta$

  $= \displaystyle 4n\int_0^{\frac{\pi}{2}}\cos^{2n - 2}\theta.\frac{\cos\theta}{2\sqrt{\sin\theta}}d\theta
  - 4n\int_0^{\frac{\pi}{2}}\cos^{2n}\theta.\frac{\cos\theta}{2\sqrt{\sin\theta}}d\theta$

  $= 4I_{n - 1} - 4nI_n \Rightarrow \frac{I_n}{I_{n - 1}} = \frac{4n}{4n + 1}$

  And now we can proceed like earlier.
\item $(1 - x)^n = C_0 - C_1x + C_2x^2 - \cdots + C_nx^n$

  Multiply both sides with $x^{n - 1}$ to get

  $x^{n - 1}(1 - x)^n = C_0x^{n - 1} - C_1x^n + C_2x^{n + 1} - \cdots + (-1)^nC_nx^{2n - 1}$

  Integrate both sides between limits $0$ and $1$ to get

  $\displaystyle\int_0^1x^{n - 1}(1 - x)^ndx = \left[C_0.\frac{x^n}{n} - C_1.\frac{x^{n + 1}}{n + 1} +
    C_2.\frac{x^{n + 2}}{n + 2} - \cdots + C_n\frac{x^{2n}}{2n}\right]_0^1$

  $\Rightarrow \frac{C_0}{n} - \frac{C_1}{n+ 1} + \frac{C_2}{n + 2} - \cdots + (-1)^n\frac{C_n}{2n} =
  \displaystyle\int_0^1x^{n - 1}(1 - x)^ndx$

  Now we evaluate $\displaystyle\int_0^1x^{n - 1}(1 - x)^ndx$.

  Let $I_{n - 1, n} = \displaystyle\int_0^1x^{n - 1}(1 - x)^ndx = \left[x^{n - 1}.\frac{(1 - x)^{n + 1}}{-(n
      + 1)}\right]_0^1 - \int_0^1(n - 1)x^{n - 2}\frac{(1 - x)^{n + 1}}{-((n + 1)}dx$

  $= 0 + \frac{n - 1}{n + 1}\displaystyle\int_0^1x^{n - 1}(1 - x)^{n + 1}dx = \frac{n - 1}{n + 1}I_{n - 2, n
    + 1}$

  $= \frac{n - 1}{n + 1}.\frac{n - 2}{n + 2}I_{n - 3, n + 2} = \frac{n - 1}{n + 1}.\frac{n - 2}{n +
    2}.\frac{n - 3}{n + 3}I_{n - 4, n + 3} = \cdots = \frac{n - 1}{n + 1}.\frac{n - 2}{n + 2}.\frac{n - 3}{n
    + 3}\cdots \frac{1}{2n - 1}I_{0, 2n - 1}$

  $=\frac{n - 1}{n + 1}.\frac{n - 2}{n + 2}.\frac{n - 3}{n + 3}\cdots \frac{1}{2n -
    1}\displaystyle_int_0^1x^0(1 - x)^{2n - 1}dx = \frac{n - 1}{n + 1}.\frac{n - 2}{n + 2}.\frac{n - 3}{n
    + 3}\cdots \frac{1}{2n - 1}.\frac{1}{2n}$

  $= \frac{n!(n - 1)!}{2n!}$.
\item L.H.S.\ $= \left(\frac{C_0}{n} - \frac{C_0}{n + 1}\right) - \left(\frac{C_1}{n + 1} - \frac{C_1}{n +
  2}\right) + \cdots + (-1)^n\left(\frac{C_n}{2n} - \frac{C_n}{2n + 1}\right)$

  $= \frac{C_0}{n} - \frac{C_1}{n + 1} + \frac{C_2}{n + 2} - \cdots + (-1)^n\frac{C_n}{2n} -
  \left[\frac{C_0}{n + 1} - \frac{C_1}{n + 2} + \cdots + (-1)^n\frac{C_n}{2n + 2}\right]$

  Now the problem is similar to previous one and we proceed similarly to arrive at the answer.
\item $(1 - x)^n = C_0 - C_1x + C_2x^2 - \cdots + (-1)^nC_nx^n$

  Multiply both sides by $x^{jk - 1}$ to get

  $C_0x^{k - 1} - C-_1x^k + C_2x^{k + 1} - \cdots + (-1)^nC_nx^{n + k - 1} = x^{k - 1}(1 - x)^n$

  Integrate both sides with limts $0$ and $1$ to get

  $\left[C_0\frac{x^k}{k} - C_1\frac{x^{k + 1}}{k + 1} + C_2\frac{x^{k + 2}}{k + 2} - \cdots +
    (-1)^nC_n\frac{x^{n + k}}{n + k}\right]_0^1 = \displaystyle\int_0^1x^{k - 1}(1 - x)^ndx$

  $\Rightarrow \frac{C_0}{k} - \frac{C_1}{k + 1} + \frac{C_2}{k + 2} - \cdots + (-1)^n\frac{C_n}{k + n} =
  \displaystyle\int_0^1x^{k - 1}(1 - x)^ndx$.

  Now we evaluate $\displaystyle\int_0^1x^{k - 1}(1 - x)^ndx$.

  $I_{k - 1, n} = \displaystyle\int_0^1x^{k - 1}(1 - x)^ndx = \left[x^{k - 1}.\frac{(1- x)^{n + 1}}{-(n +
      1)}\right]_0^1 + \frac{k - 1}{n + 1}\int_0^1x^{k - 2}(1 - x)^{n + 1}dx$

  $= 0 + \frac{k - 1}{n + 1}I_{k - 2, n + 1}$.

  Now we proceed like previous problem we obtain the answer as $\frac{n!}{k(k + 1)(k + 2)\cdots(n + k)}$.
\item $(1 - x)^n = C_0 - C_1x + C_2x^2 - \cdots + (-1)^nC_nx^n$ and $(x + 1)^n = C_0x^n + C_1x^{n - 1} +
  C_2x^{n - 2} + \cdots + C_n$

  Multiply both of above equation we arrive at

  $(C_0 - C_1x + C_2x^2 - \cdots + (-1)^nC_nx^n)(C_0x^n + C_1x^{n - 1} + C_2x^{n - 2} + \cdots + C_n) = (1 -
  x^2)^n$

  Coeff.\ of $x^n$ on L.H.S.\ $= C_0^2 - C_1^2 + C_2^2 - \cdots + (-1)^nC_n^2$

  R.H.S.\ $= C_0 - C_1x^2 + C_2x^4 + \cdots$

  We observe that power of $x$ are even on right hand side. So if $n$ is odd then coeff.\ is $0$.

  If $n$ is even then coeff.\ of $x^n$ on R.H.S.\ $= (-1)^{\frac{n}{2}}C_{n/2} =
  (-1)^{\frac{n}{2}}\frac{n!}{\left(\frac{n!}{2}\right)^2}$.
\item $(1 + x)^n = C_0^^n + C_1^^n + C_2^^nx^2 + \cdots + C_{r - 1}^^nx^{r - 1} + C_r^^nx^r + \cdots +
  C_n^^nx^n$ and $(1 + x)^m = C_0^^m + C_1^^m + C_2^^mx^2 + \cdots + C_{r - 1}^^mx^{r - 1} + C_r^^mx^r +
  \cdots + C_m^^nx^m$.

  Multiply these two and equating the coeff.\ of $x^r$ we get

  $C_r^^mC_0^^n + C_{r - 1}^^mC_1^^n + C_{r - 2}^^mC_2^^n + \cdots + C_0^^mC_r^^n = C_r^^{m + n}$.
\item $(1 - x)^{2n} = C_0^^{2n} - C_1^^{2n}x + C_2^^{2n}x^2 - \cdots + (-1)^{2n}C_{2n}^^2n$ and $(x +
  1)^{2n} = C_0^^{2n}x^{2n} + C_1^^{2n}x^{2n - 1} + \cdots + C_{2n}^^{2n}$.

  We multiply these two and equate the coeff.\ of $x^{2n}$ to arrive at

  $(C_0^^{2n})^2 - (C_1^^{2n})^2 + \cdots + (-1)^{2n}(C_{2n}^^{2n})^2 = (-1)^nC_n^^{2n}$.
\item $(1 + x)^n = C_0 + C_1x + C_2x^2 + \cdots + C_nx^n$

  Differentiating both sides w.r.t.\ $x$ we deduce

  $n(1 + x)^{n - 1} = C_1 + 2.C_2x + 3.C_3x^2 + \cdots + n.C_nx^{n - 1}$

  Also, $(x + 1)^n = C_0x^n + C_1x^{n - 1} + C_2x^{n - 2} + \cdots + C_n$

  Multiply last two equations to arrive at

  $n(1 + x)^{2n - 1} = (C_1 + 2.C_2x + 3.C_3x^2 + \cdots + n.C_nx^{n - 1})(C_0x^n + C_1x^{n - 1} + C_2x^{n -
    2} + \cdots + C_n)$

  Equating coeff.\ of $x^{n - 1}$, we get

  $C_1^2 + 2.C_2^2 + 3.C_3^2 + \cdots + n.C_n^2 = n.C_{n - 1}^^{2n - 1} = \frac{(2n - 1)!}{[(n - 1)!]^2}$.
\item $(1 + x)^n = C_0 + C_1x + C_2x^2 + \cdots + C_nx^n$

  Integrating within limits $0$ and $x$, we get

  $\left[\frac{(1 + x)^{n + 1}}{n + 1}\right]_0^x = \left[C_0x + C_1.\frac{x^2}{2} + C_2.\frac{x^3}{3} +
    \cdots + C_n.\frac{x^{n + 1}}{n + 1}\right]_0^x$

  $\Rightarrow \frac{(1 + x)^{n + 1} - 1}{n + 1} = C_0x + C_1.\frac{x^2}{2} + C_2.\frac{x^3}{3} +
  \cdots + C_n.\frac{x^{n + 1}}{n + 1}$

  Also, $(x + 1)^n = C_0x^n + C_1x^{n - 1} + C_2x^{n - 2} + \cdots + C_n$

  Multiplying last two equations and equating coeff.\ of $x^{n + 1}$ we get desired reqult.
\item $(1 - x)^n = C_0 - C_1x + C_2x^2 - \cdots + (-1)^nC_nx^n$

  Multiply with $x$ to get

  $x(1 - x)^n = C_0x - C_1x^2 + C_2x^3 - \cdots + (-1)^nC_nx^{n + 1}$

  Differentiating w.r.t.\ $x$ gives us

  $(1 - x)^n - nx(1 - x)^{n - 1} = C_0 - 2.C_1x + 3.C_2x^2 - \cdots + (-1)^n.(n + 1).C_nx^n$

  We multiply again with $x$ and differentiate again w.r.t.\ $x$ to get

  $(1 - x)^n - nx(1 - x)^{n - 1} - 2nx(1 - x)^{n - 1} + n(n - 1)x^2(1 - x)^{n - 2} = C_0 - 2^2.C_1x +
  3^2.C_2x^2 - \cdots + (-1)^n.(n + 1)^2.C_nx^n$

  Putting $x = 1$ gives us

  $C_0 - 2^2C_1 + 3^2C_2 - \cdots + (-1)^n(n + 1)^2C_n = 0$.
\item $(1 - x)^n = C_0 - C_1x + C_2x^2 - \cdots + (-1)^nC_nx^n$

  Integrating within the limits $0$ and $x$ gives us

  $-\frac{(1 - x)^{n + 1}}{n + 1} + \frac{1}{n + 1} = C_0x - C_1.\frac{x^2}{2} + C_2.\frac{x^3}{3} - \cdots
  + (-1)^nC_n.\frac{x^{n + 1}}{n + 1}$

  Integrating again within the limits $0$ and $1$ we arrive at

  $\left[\frac{(1 - x)^{n + 2}}{(n + 1)(n + 1)} + \frac{x}{n + 1}\right]_0^1 = \frac{C_0}{1.2} -
  \frac{C_1}{2.3} + \frac{C_2}{3.4} - \cdots + (-1)^n\frac{C_n}{(n + 1)(n + 2)}$

  $\frac{C_0}{1.2} - \frac{C_1}{2.3} + \frac{C_2}{3.4} - \cdots + (-1)^n\frac{C_n}{(n + 1)(n + 2)} =
  \frac{1}{n + 2}$.
\item $(1 - x)^n = C_0 - C_1x + C_2x^2 - \cdots + (-1)^nC_nx^n$

  Multiplying with $x$ gives us

  $x(1 - x)^n = C_0x - C_1x^2 + C_2x^3 - \cdots + (-1)^nC_nx^{n + 1}$

  Integrating within limits $0$ and $1$ yields

  $\left[-\frac{x(1 - x)^{n + 1}}{n + 1}\right]_0^1 + \displaystyle\int_0^1\frac{(1 - x)^{n + 1}}{n + 1}dx
    = \frac{C_0}{1.2} - \frac{C_1}{2.3} + \frac{C_2}{3.4} - \cdots + (-1)^n\frac{C_n}{(n + 1)(n + 2)}$

  $\Rightarrow \left[-\frac{(1 - x)^{n + 2}}{(n + 1)(n + 2)}\right]_0^1 = \frac{C_0}{1.2} -
  \frac{C_1}{2.3} + \frac{C_2}{3.4} - \cdots + (-1)^n\frac{C_n}{(n + 1)(n + 2)}$

  $\Rightarrow \frac{C_0}{1.2} - \frac{C_1}{2.3} + \frac{C_2}{3.4} - \cdots + (-1)^n\frac{C_n}{(n + 1)(n +
    2)} = \frac{1}{(n + 1)(n + 2)}$.
\item $(1 - x)^n = C_0 - C_1x + C_2x^2 - \cdots + (-1)^nC_nx^n$

  Multiplying with $x^2$ gives us

  $x^2(1 - x)^n = C_0x^2 - C_1x^3 + C_2x^4 - \cdots + (-1)^nC_nx^{n + 2}$

  Integrating within limits $0$ and $1$ yields

  $\left[C_0.\frac{x^3}{3} - C_1.\frac{x^4}{4} + C_2.\frac{x^5}{5} - \cdots + (-1)^nC_n.\frac{x^{n + 3}}{n +
      3}\right]_0^1 = \displaystyle\int_0^1x^2(1 - x)^ndx$

  Now we evaluate $\displaystyle\int_0^1x^2(1 - x)^ndx$

  $\displaystyle\int_0^1x^2(1 - x)^ndx = \left[-\frac{x^2(1 - x)^{n + 1}}{n + 1}\right]_0^1 +
  \int_0^1\frac{2x(1 - x)^{n + 1}}{(n + 1)}dx$

  $= 0 - \left[\frac{2x(1 - x)^{n + 2}}{(n + 1)(n + 2)}\right]_0^1 + \frac{2}{(n + 1)(n +
    2)}\displaystyle\int_0^1(1 - x)^{n + 2}dx$

  $= -0 - \frac{2}{(n + 1)(n + 2)}\displaystyle\left[\frac{(1 - x)^{n + 3}}{n + 3}\right]_0^1 = \frac{2}{(n + 1)(n + 2)(n
    + 3)}$.
\item $(1 + x)^n = C_0 + C_1x + C_2x^2 + \cdots + C_nx^n$

  Integrating within limits $0$ and $3$ we get the desired result.
\item $(1 - x)^n = C_0 - C_1x + C_2x^2 - \cdots + (-1)^nC_nx^n$

  Multiply with $x$ to get

  $x(1 - x)^n = C_0x - C_1x^2 + C_2x^3 - \cdots + (-1)^nC_nx^{n + 1}$

  Differentiating w.r.t. $x$ leads us to

  $[(1 - x)^n - nx(1 - x)^{n - 1}] = C_0 - 2.C_1x + 3.C_2x^2 - \cdots + (-1)^n.(n + 1).C_nx^n$

  Also, $(x + 1)^n = C_0x^n + C_1x^{n - 1} + C_2x^{n - 2} + \cdots + C_n$

  Multiplying last two equations gives us

  $(1 - x^2)^n - nx(1 + x)(1 - x^2)^{n - 1} = (C_0 - 2.C_1x + 3.C_2x^2 - \cdots + (-1)^n.(n +
  1).C_nx^n)(C_0x^n + C_1x^{n - 1} + C_2x^{n - 2} + \cdots + C_n)$

  Equating the coeff.\ of $x^n$ gives us

  $C_0^2 - 2.C_1^2 + 3.C_2^2 - \cdots + (-1)^n(n + 1)C_n^2 =
  (-1)^{\frac{1}{2}}C_{\frac{n}{2}}.nC_{\frac{n}{2} - 1}^^{n - 1}$

  $= (-1)^{\frac{n}{2}}\left[C_{\frac{n}{2}} + n.C_{\frac{n}{2} - 1}^^{n - 1}\right]$

  $\Rightarrow 2.\frac{\left(\frac{n!}{2}\right)^2}{n!}[C_0^2 - 2.C_1^2 +
    3.C_2^2 - \cdots + $ $(-1)^n.(n + 1)C_n^2] = (-1)^{n/2}(2 + n)$.
\item $2\displaystyle\sum_{0\leq i\leq n}\sum_{i<j\leq n}C_iC_j = (C_0 + C_1 + C_2 + \cdots + C_n)^2 - (C_0^2
  + C_1^2 + C_2^2 + \cdots + C_n^2) = (2^n)^2 - C_n^^{2n} = 2^{2n} - \frac{2n!}{(n!)^2}$

  $\Rightarrow \displaystyle\sum_{0\leq i\leq n}\sum_{i<j\leq n}C_iC_j = 2^{2n - 1} - \frac{2n!}{2(n!)^2}$.
\item $(1 + x)^n = C_0 + C_1x + C_2x^2 + \cdots + C_nx^n$, $(1 + y)^n = C_0 + C_1y + C_2y^2 + \cdots +
  C_ny^n$, and $(x + y)^n = C_0x^n + C_1x^{n - 1}y + C_2x^{n - 2}y^2 + \cdots + C_ny^n$

  We multiply all three and equate the coeff.\ of $x^ny^n$ which is equal to $C_0^3 + C_1^3 + C_2^3 + \cdots
  + C_n^3$.
\item Let $(1 + x - 3x^2)^{2163} = a_0 + a_1x + a_2x^2 + \cdots + a_{6489}x^{6489}$.

  Putting $x = 1$, we get sum of coefficients as

  $a_0 + a_1 + a_2 + \cdots + a_{6489} = (1 + 1 - 3)^{2163} = (-1)^{2163} = -1$.
\item Putting $x = 1, \omega, \omega^2$ in the given equation, and adding we get the desired result.
\item $(r + 1)$th term in the given expression is given by $t_{r + 1} = C_r^^{10}2^{\frac{10 -
    r}{2}}3^{\frac{r}{5}}$.

  For rational terms $r$ has to a multiple of $5$ for $3$ and $\frac{10 - r}{2}$ has to be a multiple of
  $2$. In the given series for the first condition $r = 0, 5, 10$, and for the second condition $r = 0, 2,
  4, 6, 8, 10$.

  So common values of $r$ are $0$ and $10$.

  $\therefore$ Sum of rational terms $= t_1 + t_{11} = C_0(\sqrt{2})^{10} + C_{10}^^{10}(3^{1/5})^{10} =
  41$.
\item $\frac{2^{4n}}{15} = \frac{16^n}{15} = \frac{(1 + 15)^n}{15} = \frac{1 + C_1.15 + C_2.15^2 + \cdots +
  C_n.15^n}{15}$.

  It is clear from above that the fractional part would be $\frac{1}{15}$.
\item Let $(\sqrt{3} + 1)^{2n} = p + f$, where $p$ is the integral part and $0 < f < 1$.

  $(\sqrt{3} + 1)^{2n} = [(\sqrt{3} + 1)^2]^n = (4 + 2\sqrt{3})^n = 2^n(2 + \sqrt{3})^n$

  Thus, $p + f = 2^n(2 + \sqrt{3})^n$. Also, $0 < \sqrt{3 - 1} < 1 \therefore 0 < (\sqrt{3} - 1)^{2n} < 1$

  Let $f_1 = (\sqrt{3} - 1)^{2n} = 2^n(2 - \sqrt{3})^n$

  $p + f + f_1 = 2^n.2[2^n + C_2.2^n(\sqrt{3})^2 + \cdots] = 2^{n + 1}.$ an integer $=$ an integer.

  $f + f_1 = $ even number $- p =$ an integer. Also, $0 < f + f_1 < 2$ and thus, $f + f_1 = 1$.

  Hence, integer just above $(\sqrt{3} + 1)^{2n}$ i.e. $p + 1$ is divisible by $2^{n + 1}$.
\item Clearly $0 < f < 1$. $R = (5\sqrt{5} + 11)^{2n + 1}$

  Let $f' = (5\sqrt{5} - 11)^{2n + 1}$. $\because 0 < 5\sqrt{5} - 11 < 1 \therefore 0 < (5\sqrt{5} - 11)^{2n
    + 1} <1$.

  $R - f' = 2[C_1^^{2n + 1}.(5\sqrt{5})^{2n}.11 + C_3^^{2n + 1}(5\sqrt{5})^{2n - 2}.11^3 + \cdots] =$ an
  even number

  $f - f' =$ an even number $- [R] =$ an integer. But $-1 < f - f' < 1$.

  Thus, $f - f' = 0 \Rightarrow f = f'$. $\Rightarrow Rf' = ((5\sqrt{5} + 11)^{2n + 1})((5\sqrt{5} - 11)^{2n
    + 1}) = 4^{2n + 1} = Rf[\because f = f']$.
\item $101^{50} - 99^{50} = (100 + 1)^{50} - (100 - 1)^{50}$

  $= 2[C_1^^{50}.100^{49} + C_3^^{50}100^{47} + \cdots + C_{49}^^{50}100]$

  $= 100^{50} + 2[C_3^^{50}100^{47} + \cdots + C_{49}^^{50}100]$

  $\therefore 101^{50} - 99^{50} > 100^{50} \Rightarrow 101^{50} > 100^{50} + 99^{50}$.
\item $t_1 = \displaystyle\sum_{r = 0}^nC_r\left(\frac{1}{2}\right)^r = \left(1 - \frac{1}{2}\right)^r =
  \frac{1}{2^n}$, $t_2 = \displaystyle\sum_{r = 0}^nC_r\left(\frac{3}{4}\right)^r = \left(1 -
  \frac{3}{4}\right)^r = \frac{1}{2^{2n}}, \ldots$ and so on.

  Therefore, required sum $= \frac{1}{2^n} + \frac{1}{2^{2n}} + \cdots + \frac{1}{2^{mn}} =
  \frac{1}{2^n}\left[\frac{1 - \left(\frac{1}{2^n}\right)^m}{1 - \frac{1}{2^n}}\right] = \frac{1 -
    \frac{1}{2^{mn}}}{2^n - 1}$.
\item $32^{32} = (2 + 3\times10)^{32} = 2^{32} + 10k$, where $k\in\mathbb{N}$. Therefore last digit in
  $32^{32}$ is same as last digit in $2^{32}$.

  $2^{32} = (2^5)^6.2^2 = 32^6.4 = (2 + 10)^6.4 = 4.(2^6 + 10r)$, where $r\in\mathbb{N}$.

  Therefore, last digit in $2^{32}$ is same as last digit in $4.2^6 =$ last digit in $16 = 6$.
\item Let $n = 2m$, where $m\in\mathbb{P}$, then $k = 3n$.

  Now, L.H.S.\ $= \displaystyle\sum_{r = 1}^^{3m}(-3)^{r - 1}C_{2r - 1}^^{6m} = C_1^^{6m} - C_3^^{6m}.3 +
  C_5^^{6m}.3^2 - \cdots + (-1)^{3m - 1}C_{6m - 1}^^{6m}3^{3m - 1}$

  $= \frac{1}{\sqrt{3}}[C_1^^{6m}\sqrt{3} - C_3^^{6m}(\sqrt{3})^3 + C_5^^{6m}(\sqrt{3})^5 - \cdots +
    (-1)^{3m - 1}C_{6m - 1}^^{6m}(\sqrt{3})^{6m - 1}]$

  We observe that $(-1)^{3m - 1} = i^{6m - 2} = -i.i^{6m - 1}$.

  Now $(1 + \sqrt{3}i)^{6m} = 1 + C_1^^{6m}\sqrt{3}i + C_2^^{6m}(\sqrt{3}i)^2 + C_3^^{6m}(\sqrt{3}i)^3 +
  \cdots + C_{6m - 1}^^{6m}(\sqrt{3}i)^{6m - 1} + C_{6m}^^{6m}(\sqrt{3}i)^{6m}$

  $= [1 - C_2^^{6m}.3 + C_4^^{6m}.3^2 - \cdots] + i[C_1^^{6m}\sqrt{3} - C_3^^{6m}(\sqrt{3})^3 + \cdots +
    C_{6m - 1}^^{6m}.i^{6m - 2}(\sqrt{3})^{6m - 1}]$

  However, $(1 + \sqrt{3}i)^{6m} = \left[2\left(\cos\frac{\pi}{3} + i\sin\frac{\pi}{3}\right)\right]^{6m} =
  2^{6m}$.

  Equating coeff.\ of imaginary parts yields

  $C_1^^{6m}\sqrt{3} - C_3^^{6m}(\sqrt{3})^3 + \cdots + C_{6m - 1}^^{6m}.i^{6m - 2}(\sqrt{3})^{6m - 1} = 0$.
\item $(a + x)^n = a^n + C_1^^na^{n - 1}x + C_2^^na^{n - 2}x^2 + \cdots + x^n$

  Thus, $t_0 = a^n, t_1 = C_1^^na^{n - 1}x, t_2 = C_2^^na^{n - 2}x^2, \ldots$

  Now we replace $x$ with $ix$ to get

  $(a + ix)^n = a^n + C_1^^na^{n - 1}ix - C_2^^na^{n - 2}x^2 + \cdots + (ix)^n$

  $= (a^n - C_2^^na^{n - 2}x^2 + \cdots) + i(C_1^^na^{n - 1}x - C_3^^na^{n - 3}x^3 + \cdots)$

  $= (t_0 - t_2 + t_4 - \cdots) + i(t_1 - t_3 + t_5 - \cdots)$

  Taking modulus and squaring yields

  $(a^2 + x^2)^n = (t_0 - t_2 + t_4 - \cdots)^2 + i(t_1 - t_3 + t_5 - \cdots)^2$.
\item Given, $(1 + x + x^2)^n = a_0 + a_1x + a_2x^2 + \cdots + a_{2n}x^{2n}$.

  Putting $x = 1$ yields

  $a_0 + a_1 + a_2 + \cdots + a_{2n} = 3^n$.

\item Putting $x = -1$ yields

  $a_0 - a_1 + a_2 - \cdots + a_{2n} = 1$.

\item Putting $x = 1, \omega, \omega^2$ yields

  $a_0 + a_1 + a_2 + \cdots + a_{2n} = 3^n$,

  $a_0 + a_1\omega + a_2\omega^2 + \cdots + a_{2n}\omega^{2n} = 0$, and

  $a_0 + a_1\omega^2 + a_2\omega^4 + \cdots + a_{2n}\omega^{4n} = 0$.

  Adding we get $3(a_0 + a_3 + a_6 + \cdots) = 3^n$

  $\Rightarrow a_0 + a_3 + a_6 + \cdots = 3^{n - 1}$.

\item $S_n = 1 + q + q^2 + \cdots + q^n = \frac{1 - q^{n + 1}}{1 - q}$ and $S_n' = 1 + \left(\frac{q +
  1}{2}\right) + \left(\frac{q + 1}{2}\right)^2 + \cdots + \left(\frac{q + 1}{2}\right)^n = \frac{1 -
  \left(\frac{q + 1}{2}\right)^{n + 1}}{1 - \frac{q + 1}{2}} = \frac{2^{n + 1} - (q + 1)^{n + 1}}{(1 -
  q).2^n}$.

  Now $C_1^^{n + 1} + C_2^^{n + 1}.S_1 + C_3^^{n + 1}.S_2 + \cdots + C_{n + 1}^^{n + 1}.S_n = C_1^^{n +
    1}\left(\frac{1 - q}{1 - q}\right) + C_2^^{n + 1}\left(\frac{1 - q^2}{1 - q}\right) + C_3^^{n +
    1}\left(\frac{1 - q^3}{1 - q}\right) + \cdots + C_{n + 1}^^{n + 1}\left(\frac{1 - q^{n + 1}}{1 -
    q}\right)$

  $= \frac{1}{1 - q}[(C_1^^{n + 1} + C_2^^{n + 1} + C_3^^{n + 1} + \cdots + C_{n + 1}^^{n + 1}) - (C_1^^{n +
      1}q + C_2^^{n + 1}q^2 + C_3^^{n + 1}q^3 + \cdots + C_{n + 1}^^{n + 1}q^{n + 1})]$

  $= \frac{1}{1 - q}[2^{n + 1} - 1 - \{(1 + q)^{n + 1} - 1\}] = \frac{1}{1 - q}[2^{n + 1} - (1 + q)^{n + 1}]
  = 2^nS_n'$.

\item $(r + 1)$th term in the given expansion is given by $t_{r + 1} = C_r^{1000}9^{\frac{1000 -
    r}{4}}.8^{\frac{r}{6}}$, where $r = 0, 1, 2, \ldots, 1000$.

  For rational terms $r$ has to be a multile of $6 = 0, 6, 12, 18, \ldots, 996$ and $1000 - r =$ a multiple
  of $4 = 0, 4, 8, 12, \ldots, 1000$.

  From both of these the common values are multiple of $12$, which is L.C.M.\ of $4$ and $6$. Thus, sum
  of rational terms would be $84$.

\item $(r + 1)$th term in the given expansion is given by $t_{r + 1} = C_r^^{15}2^{\frac{15 -
    r}{3}}3^{\frac{r}{5}}$, where $r = 0, 1, 2, \ldots, 15$.

  For rational terms $r$ has to be a multiple of $5 = 0, 5, 10, 15$ and $15 - r =$ a multiple of $3 = 0, 3,
  6, 9, 12, 15$.

  The common values will depend on the L.C.M.\ of $3$ and $5$ which is $15$. So there are two terms which
  will satisfy the criteria; first terms and last term.

  Sum would be $= C_0^^{15}2^5 + C_{15}^^{15}3^3 = 32 + 27 = 59$.

\item $t_3$ in the expansion of $(x + x\log_{10}x)^5$ is $C_2^^5x^3.x^2(\log_{10}x)^2 = 1,000,000$

  $\Rightarrow x^5(\log_{10}x)^2 = 100,000 \Rightarrow x = 10$.
\item Replacing $x - \frac{1}{x} = y$ we have $(1 + y)^3 = 1 + 3y + 3y^2 + y^3$. Substituting back we get

  $\left(x + 1 - \frac{1}{x}\right)^3 = 1 + 3\left(x - \frac{1}{x}\right) + 3\left(x^2 + \frac{1}{x^2} -
  2\right) + \left(x - \frac{1}{x}\right)^3 = 1 + 3x - \frac{3}{x} + 3x^2 + \frac{3}{x^2} - 6 + x^3 -3x +
  \frac{3}{x} - \frac{1}{x^3}$

  $= x^3 + 3x^2 - 5 + \frac{3}{x^2} - \frac{1}{x^3}$.
\item It is given that coefficients of second, third and fourth terms are the first, third and fifth terms
  of an A.P. i.e. they are in A.P.

  $\therefore 2.C_2^^m = C_1^^m + C_3^^m\Rightarrow 2.\frac{m(m - 1)}{2} = m + \frac{m(m - 1)(m - 2)}{6}$

  $\Rightarrow m - 1 = 1 + \frac{(m - 1)(m - 2)}{6} \Rightarrow m^2 - 9m + 14 = 0$

  $\Rightarrow m = 2, 7$ but it cannot be $2$ as we have more than $3$ terms, so $m = 7$.

  $t_6 = C_5^^7[2^{\log(10-3^x) + (x - 2)\log3}] = 21 \Rightarrow 2^{\log(10-3^x) + \log3^{x - 2}} = 1 =
  2^0$

  $\Rightarrow \log(10 - 3^x) + \log3^{x - 2} = 0 \Rightarrow (10 - 3^x)3^{x - 2} = 1$

  $\Rightarrow 3^{2x} - 10.3^x + 9 = 0 \Rightarrow 3^x = 9, 1 \Rightarrow x = 2, 0$.
\item Sixth term of the expansion $\left[2^{\log_2\sqrt{9^{x - 1} + 7}} +
  \frac{1}{2^{\tfrac{1}{5}\log_2(3^{x -1} + 1)}}\right]^7$ is given by $t_6 = C_5^^7(\sqrt{9^{x -
      1} + 7})^2.\frac{1}{(3^{x - 1} + 1)} = 84$

  $\Rightarrow \frac{9^{x - 1} + 7}{3^{x - 1} + 1} = 4\Rightarrow x = 1, 2$.
\item We have to prove that $\frac{1}{(81)^n} - \frac{10}{(81)^n}.C_1^^{2n} +
  \frac{10^2}{(81)^n}.C_2^^{2n} - \frac{10^3}{(81)^n}.C_3^^{2n} + \cdots + \frac{10^{2n}}{(81)^n} = 1$.

  $\Rightarrow \frac{1}{81^n}[C_0^^{2n} - C_1^^{2n}10 + C_2^^{2n}10^2 - C_3^^{2n}10^3 + \cdots +
    C_{2n}^^{2n}10^{2n}] = 1$

  $\Rightarrow \frac{1}{81^n}(1 - 10)^{2n} = \frac{1}{81^n}.(-9)^{2n} = 1$.
\item We know that $C_r = C_{n - r}$. Thus we can rewrite the given series as

  $\displaystyle\lim_{n\to\infty} = C_0 - C_1.\frac{2}{3} + C_2.\left(\frac{2}{3}\right)^2 -
  C_3.\left(\frac{2}{3}\right)^3 + \cdots + (-1)^nC_n.\left(\frac{2}{3}\right)^n$

  $= \left(1 - \frac{2}{3}\right)^n = \frac{1}{3^n} = 0$.
\item Given, $E = (6\sqrt{6} + 14)^{2n + 1} = [E] + F$. Let $F' = (6\sqrt{6} - 14)^{2n + 1} =
  \frac{20^{2n + 1}}{(6\sqrt{6} + 14)^{2n + 1}}$

  $E - F' = C_1^^{2n + 1}(6\sqrt{6})^{2n}.14 + C_3^^{2n + 1}(6\sqrt{6})^{2n - 2}.14^3 + ... =$ an even
  number.

  $F - F' =$ an even number $- [E] =$ an integer. But $0 < F < 1$ and $0 < F' < 1\Rightarrow -1 < F - F' <
  1$

  $\Rightarrow F - F' = 0 \Rightarrow F = F'$.

  $\therefore EF = EF' = 20^{2n + 1}$.
\item $(17)^{256} = (289)^{128} = (290 - 1)^{128} = C_0.290^{128} - C_1.290^{127} + \cdots +
  (-1)^{128}C_{128}$

  Clearly, all the terms except last term would be multiple of $10$. So at unit's place last term will occur
  which is $1$.

  Similarly for ten's place only second last term will matter which is $-C_{127}.290 = -128\times290 =
  -37120$. This is a negative term and the highest term is positive and multiple of $100$ so the tens place
  unit will be $10 - 2 = 8$.

  Similarly for hundred's place third last term will matter which is $C_{126}.290^2 = 683564800$ and thus
  the digit will be $6$ after adjusting with second last term.
\item We have to prove that for $n\geq 3,\;n^{n + 1} > (n + 1)^n,\;forall\;n\in\mathbb{P}$.

  Rewriting $\Rightarrow n > \left(1 + \frac{1}{n}\right)^n$.

  Now $\left(1 + \frac{1}{n}\right)^n = 1 + n.\frac{1}{n} + \frac{n(n - 1)}{2!}.\frac{1}{n^2} + \frac{n(n -
    1)(n - 2)}{3!}.\frac{1}{n^3} + \frac{n(n - 1)(n - 2)\cdots [n - (n - 1)]}{n!}.\frac{1}{n^n}$

  $= 1 + 1 + \frac{1}{2!}\left(1 - \frac{1}{n}\right) + \frac{1}{3!}\left(1 - \frac{1}{n}\right)\left(1 -
  \frac{2}{n}\right) + \cdots + \frac{1}{n!}\left(1 - \frac{1}{n}\right)\left(1 -
  \frac{2}{n}\right)\cdots\left(1 - \frac{n - 1}{n}\right)$

  $< 1 + 1 + \frac{1}{2!} + \frac{1}{3!} + \cdots + \frac{1}{n!} < 1 + 1 + \frac{1}{2} + \frac{1}{2^2} +
  \cdots + \frac{1}{2^{n - 1}}$

  $= 1 + 2\left[1 - \frac{1}{2^n}\right] = 3 - \frac{1}{2^{n - 1}} < 3$.

  Now we have been given that $n\geq 3$ and hence $n^{n + 1} > (n + 1)^n$.
\item We have to prove that $2<\left(1 + \frac{1}{n}\right)^n < 3\;\forall\;n\in\mathbb{N}$.

  Proceeding like previous problem we obtain $2 \leq \left(1 + \frac{1}{n}\right)^n < 3$.
\item We have to prove that $1992^{1998} - 1955^{1998} - 1938^{1998} + 1901^{1998}$ is divisible by $1998$.

  Rewriting as $(1992^{1998} - 1955^{1998}) - (1938^{1998} - 1901^{1998})$. We know that $a^n - b^n$ is
  divisible by $a - b$. Thus given expression is divisible by $37$.

  Rewriting again as $(1992^{1998} - 1938^{1998}) - (1955^{1998} - 1901^{1998})$, which is divisible by
  $54$.

  Since $37$ is a prime number there will be no common factor with $54$. Hence, given expression is
  divisible by $37\times 54 = 1998$.
\item $53^{53} = (50 + 3)^{53} = 50k + 3^{53}$ and $33^{33} = (30 + 3)^{33} = 30k + 3^{33}$.

  So if the difference $3^{53} - 3^{33}$ is divisible by $10$ then our proof will be complete.

  We observe the powers of $3$. $3^1 = 3, 3^2 = 9, 3^3 = 27, 3^4 = 81, 3^5 = 243, \ldots$. Thus we see that
  it repeats in the fashion of $3, 9, 7, 1, 3, 9, 7, 1, \ldots$ as far as last digits are concerned.

  $53 = 4*13 + 1$ and $33 = 4*8 + 1$ so the last digits of both will be same i.e. $9$. Hence, the given
  difference is divisible by $10$.
\item $(1 + x)^{m + 1} = C_0^^{m + 1} + C_1^^{m + 1}x + C_2^^{m + 1}x^2 + \cdots + C_{m + 1}^^{m + 1}x^{m +
  1}$

  $\Rightarrow[(1 + x)^{m + 1} - 1 - x^{m + 1}] = C_1^^{m + 1}x + C_2^^{m + 1}x^2 + \cdots + C_m^^{m +
  1}x^{m + 1}$.

  Putting $x = 1, 2, 3, \ldots, n$ and adding we get the desired result.
\item $\displaystyle\sum_{i=1}^k\sum_{k=1}^nC_k^^nC_i^^k = \sum_{k = 1}^n(C_k^^nC_1^^k + C_k^^nC_2^^k +
  C_k^^nC_3^^k + \cdots + C_k^^nC_k^^k)$

  $= (C_1^^nC_1^^1 + C_2^^nC_1^^2 + C_3^^nC_1^^3 + \cdots + C_n^^nC_1^^n) + (C_2^^nC_2^^2 + C_3^^nC_2^^3 +
  \cdots + C_n^^nC_2^^n) + (C_3^^nC_3^^3 + \cdots + C_n^^nC_n^^nC_3^^n) + \cdots + C_n^^nC_n^^n$

  $= C_1^^n(C_1^^1 + C_0^^1) - C_1^^nC_0^^1 + C_2^^n(C_0^^2 + C_1^^2 + C_2^^2) - C_2^^nC_0^^2 +
  C_3^^n(C_0^^3 + C_1^^3 + C_2^^3 + C_3^^3) - C_3^^nC_0^^3 + \cdots$

  $= C_1^^n2 + C_2^^n2^2 + C_3^^n2^3 + \cdots + C_n^^n2^n - 1 - [C_0^^n + C_1^^n + C_2^^n + \cdots + C_n^^n
    - 1] = (1 + 2)^n - (1 + 1)^n = 3^n - 2^n$.
\item We have to prove that $\displaystyle\sum_{r=0}^n(-1)^r.{}^nC_r\frac{1 + r\log_e10}{(1 + \log_e10^n)^r}
  = 0$. Note that this equality will hold for positive $n$ but not for $n = 0$ for which L.H.S.\ is equal to
  $1$.

  $\displaystyle\sum_{r=0}^n(-1)^r.{}^nC_r\frac{1 + r\log_e10}{(1 + \log_e10^n)^r} =
  \sum_{r=0}^n(-1)^rC_r^^n\frac{1}{(1 + \log10^n)^r} + \sum_{r=0}^n(-1)^rC_{r - 1}^{n - 1}\frac{n\log 10}{(1
    + \log 10^n)^r}$

  $= 1 + \displaystyle\sum_{r = 1}^{n - 1}(C_r^^{n - 1} + C_{r - 1}^^{n - 1})\frac{1}{(1 + \log10^n)^r} +
  (-1)^n\frac{1}{(1 + \log10^n)^r} - \sum_{r = 0}^{n - 1}(-1)^rC_{r}^^{n - 1}\frac{\log10^n}{(1 +
    \log10^n)^{r + 1}}$

  $= 1+\displaystyle\sum_{r=1}^{n-1}(-1)^rC_r^^{n - 1}\frac{1}{\left(1+\log 10^n\right)^r}
  -\sum_{r=0}^{n-2}(-1)^rC_r^^{n - 1}\frac{1}{\left(1+\log 10^n\right)^{r+1}}\\
  +(-1)^n\frac{1}{(1+\log 10^n)^r}-\sum_{r=0}^{n-1}(-1)^rC_r^^{n - 1}\frac{\log 10^n}{\left(1+\log
    10^n\right)^{r+1}}$

  $=1+\displaystyle\sum_{r=1}^{n-1}(-1)^rC_r^^{n - 1}\frac{1}{\left(1+\log 10^n\right)^r}+(-1)^n\frac{1}{(1+\log 10^n)^r}\\
  -\sum_{r=0}^{n-2}(-1)^rC_r^^{n - 1}\frac{1}{\left(1+\log 10^n\right)^{r}}+(-1)^n\frac{\log
    10^n}{(1+\log 10^n)^n}$

  $=1+(-1)^{n-1}\frac{1}{(1+\log 10^n)^{n-1}}+(-1)^n\frac{1}{(1+\log 10^n)^{n-1}}\\
  -1+(-1)^n\frac{\log 10}{(1+\log 10^n)^n} = 0$.
\item $32^{32} = 2^{160} = (3 - 1)^{160} = 3m + 1$, where $m\in\mathbb{N}$.

  $32^{32^{32}} = 32^{3m + 1} = 2^{3(5m + 1)}.2^2 = 23(8^{5m}) = 32(1 + 7)^{5m}$

  $= 32(1 + 7k), k\in\mathbb{N} = 4 + 28 + 7(32k) = 4 + 7r$, where $r\in\mathbb{N}$.

  Thus, remainder would be $4$.
\item Let $t = x - 3$, then $x - 2 = 1 + t$. Now $\displaystyle\sum_{r = 0}^{2n}a_r(x - 2)^r = \sum_{r =
  0}^{2n}b_r(x - 3)^r$

  $\Rightarrow \displaystyle\sum_{r = 0}^{2n}a_r(1 + t)^r = \sum_{r=0}^{2n}b_rt^r$

  $\Rightarrow a_0 + a_1(1 + t) + \cdots + a_{n - 1}(1 + t)^{n - 1} + 1.(1 + t)^n + 1.(1 + t)^{n + 1} +
  \cdots + 1.(1 + t)^{2n} = b_0 + b_1t + b_2t^2 + \cdots + b_{2n}t^{2n}$

  Equating the coefficients of $t^n$ gives us

  $b_n = C_n^^n + C_n^^{n + 1} + C_n^^{n + 2} + \cdots + C_n^^{2n} = C_{n + 1}^^{n + 2} + C_n^^{n + 2} +
  \cdots + C_n^^{2n}[\because C_n^^n = C_{n + 1}^^{n + 1} = 1]$

  $\cdots$

  $= C_{n + 1}^^{2n + 1}$.
\item Given, $x^{50}$ in $(1 + x)^{1000} + 2x(1 + x)^{999} + 3x^2(1 + x)^{998}+ \cdots + 1001x^{1000}$.

  Rewriting, $(1 + x)^{1000}\left[1 + 2\frac{x}{1 + x} + 3\left(\frac{x}{1 + x}\right)^2 + \cdots +
    1001\left(\frac{x}{1 + x}\right)^{1000}\right]$

  $= (1 + x)^{1000}[1 + 2k + 3k^2 + \cdots + 1001k^{1000}]$, where $k = \frac{x}{1 + x}$

  \startformula\startalign[n=3]
    \NC S \NC = 1 + \NC 2k + 3k^2 + \cdots + 1001k^{1000}\NR
    \NC kS\NC =  \NC\;\; k + 2k^2 + \cdots + 1000k^{1000} + 1001k^{1001}\NR
  \stopalign\stopformula

  Subtracting $(1 - k)S = 1 + k + k^2 + \cdots + k^{1000} - 1001k^{1001} = \frac{1 - k^{1001}}{1 - k} -
  1001k^{1001}$

  $S = (1 + x)^2\left[1 - \frac{x^{1001}}{(1 + x)^{1001}}\right] - 1001.\frac{x^{1001}}{(1 + x)^{1000}}$

  So given expression becomes $(1 + x)^{1002} - x^{1001}(1 + x) - 1001x^{1001}$, and hence, coefficient of
  $x^{50}$ is $C_{50}^^{1002}$.
\item L.H.S.\ $=$ coeff.\ of $x^n$ in $(1 + x)^n + (1 + x)^{n + 1} + (1 + x)^{n + 2} + \cdots + (1 + x)^{n +
  k}$

  Now, $(1 + x)^n + (1 + x)^{n + 1} + (1 + x)^{n + 2} + \cdots + (1 + x)^{n + k} = (1 + x)^n\left[\frac{(1 +
      x)^{k + 1} - 1}{x}\right]$

  $= \frac{1}{x}(1 + x)^{n + k + 1} - \frac{1}{x}(1 + x)^n$

  Equating coefficient of $x^n$ gives us

  $C_n^n + C_n^^{n + 1} + C_n^^{n + 2} + \cdots + C_n^^{n + 1} = C_{n + 1}^^{n + k + 1}$(there is no power
  of $x^n$ in second term).
\item Let $S = x + 2x^2 + 3x^3 + \cdots + nx^n$ then $xS = x^2 + 2x^3 + \cdots + (n - 1)x^n + nx^{n
  + 1}$

  Subtracting yields $(1 - x)S = x + x^2 + x^3 + \cdots + x^n - nx^{n + 1} \Rightarrow S = x\frac{(1 -
    x^n)}{(1 - x)^2} - \frac{nx^{n + 1}}{1 - x}$

  $\Rightarrow (1 + x + 2x^2 + 3x^3 + \cdots + nx^n)^2 = \left[1 + x\frac{(1 - x^n)}{(1 - x)^2} -
    \frac{nx^{n + 1}}{1 - x}\right]^2$

  Coefficient of $x^n =$ coefficient of $x^n$ in $\left[1 + \frac{x}{(1 - x)^2}\right]^2$(leaving terms
  containing powers of $x$ greater than $n$)

  $=$ coefficient of $x^n$ in $2x(1 - x)^{-2} + x^2(1 - x)^{-4} = 2.C_{n - 1}^^n + C_{n - 2}^^{n + 1} =
  2.C_1^^n + C_3^^{n + 1}$

  $= \frac{n(n^2 + 11)}{6}$.
\item $1 + (1 + x) + (1 + x)^2 + \cdots + (1 + x)^n = \frac{(1 + x)^{n + 1} - 1}{x}$.

  Clearly, coefficient of $x^k$ in $\frac{(1 + x)^{n + 1} - 1}{x}$ is equal to the coefficient of $x^{k +
    1}$ in $(1 + x)^{n + 1} = C_{k + 1}^^{n + 1}$.
\item $(x + 1)^n + (x + 1)^{n- 1}(x + 2) + (x + 1)^{n - 2}(x + 2)^2 + \cdots + (x + 2)^n$ is a G.P.\ with
  first term $(x + 1)^n$ and common ratio $\frac{x + 2}{x + 1}$.

  Thus, sum of the series is $\frac{(x + 1)^n\left[\frac{(x + 2)^n}{(x + 1)^n} - 1\right]}{\frac{x + 2}{x +
      1} - 1} = \frac{(x + 2)^n - (x + 1)^n}{\frac{1}{x + 1}} = (x + 1)[(x + 2)^n - (x + 1)^n]$.

  Thus, coefficient of $x^3$ in $(x + 1)[(x + 2)^n - (x + 1)^n]$ is equal to coefficient of $x^2$ in $(x +
  2)^n - (x + 1)^n + $ coefficient of $x^3$ in $[(x + 2)^n - (x + 1)^n]$

  $= C_{n - 2}^^n.2^{n - 2} - C_{n - 2}^^n + C_{n - 3}^^n.2^{n - 3} - C_{n - 3}^^n = C_3^^{n + 1}(2^{n - 2}
  - 1)$.
\item $\left(\frac{a + 1}{a^{2/3} - a^{1/3} + 1} - \frac{a - 1}{a - a^{1/2}}\right)^{10} = \left(a^{1/3} + 1
  - \frac{\sqrt{a} + 1}{\sqrt{a}}\right)^{10} =(a^{1/3} - a^{-1/2})^{10}$.

  Let $(r + 1)$th term be independent of $a$, then $t_{r + 1} = C_r^^{10}a^{(10 - r)/3}(-a)^{-r/2}$.

  For this term to be independent of $a, \frac{10 - r}{3} - \frac{r}{2} = 0 \Rightarrow 20 - 5r = 0
  \Rightarrow r = 4$.

  So $5$th term is independent of $a$. $t_5 = C_4^^{10} = 210$.
\item Coefficient of $x^2$ in $\left(x + \frac{1}{x}\right)^{10}(1 - x + 2x^2) =$ coefficient of $x^2$ in
  $\left(x + \frac{1}{x}\right)^{10} -$ coefficient of $x$ in $\left(x + \frac{1}{x}\right)^{10} + 2.$
  coefficient of term independeng of $x$ in $\left(x + \frac{1}{x}\right)^{10}$.

  Consider $(r + 1)$th term in the expansion of $\left(x + \frac{1}{x}\right)^{10}$.

  $t_{r + 1} = C_r^^{10}.x^{10 - r}.x^{-r}$ so power of $x$ is $10 - 2r$.

  If $10 - 2r = 2 \Rightarrow r = 4$; for $10 - 2r = 1$ we do not have an integral $r$; and for $10 - 2r = 0
  \Rightarrow r = 5$.

  So final answer is $C_4^^{10} + 2.C_5^^{10} = 714$.
\item $(1 + x - 2x)^6 = (1 - x)^6(1 + 2x)^6 = (1 - 6x + 15x^2 - 20x^3 + 15x^4 - 6x^5 + x^6)(1 + 12x + 60x^2
  + 160x^3 + 140x^4 + 192x^5 + 64x^6)$

  Hence, coefficient of $x^4$ is $240 - 6\times160 + 15\times60 - 20\times12 + 15 = -45$.
\item We have to find the term independent of $x$ in $(1 + x + 2x^3)\left(\frac{3}{2}x^2 -
  \frac{1}{3x}\right)^9$.

  We consider $(r + 1)$th term in the expansion of $\left(\frac{3}{2}x^2 - \frac{1}{3x}\right)^9$.

  $t_{r + 1} = C_r^^9\left(\frac{3}{2}x^2\right)^{9 - r}\left(-\frac{1}{3x}\right)^r$

  So the power of $x$ would be $18 - 3r$. This term is multiplied with $1 + x + 2x^3$ so for $1$ the value
  of $18 - 3r = 0$ for the term to be independent of $x$ i.e. $r = 6$. For $x$ there will be no such term
  because $18 - 3r = -1$ does not give an integral $r$. For $2x^3$ the term would be $18 - 3r = -3
  \Rightarrow r = 7$.

  So the final term would be $C_6^^9\left(\frac{3}{2}x^2\right)^3\left(-\frac{1}{3x}\right)^6 +
  2x^3.C_7^^9\left(\frac{3}{2}x^2\right)^2\left(-\frac{1}{3x}\right)^7$

  $= \frac{17}{54}$.
\item We have to find the term independent of $x$ in $\left(x^2 + \frac{1}{x^3}\right)^7(2 - x)^{10}$.

  The terms in $(2 - x)^{10}$ will have terms in which powers of $x$ will vary from $0$ to $10$ increasing
  by $1$.

  We consider $(r + 1)$th term in $\left(x^2 + \frac{1}{x^3}\right)^7$.

  $t_{r + 1} = C_r^^7(x^2)^{7 - r}.\left(\frac{1}{x^3}\right)^r$. So the power of $x$ would be $14 - 5r$. We
  vary $r$ which gives us $14, 9, 4, -1, -6, -11$ for $r = 0, 1, 2, 3, 4, 5$ and so on. Out of these only
  powers of $-1$ and $-6$ can be neutralized by the second expansion. Thus, $r = 3, 4$.

  Corresponding terms in the expansion of $(2 - x)^{10}$ are $-C_1^^{10}2^9x$ and $C_6^^{10}2^4x^6$.

  Corresponding terms in the expansion of $\left(x^2 + \frac{1}{x^3}\right)^{7}$ are $C_3^^7x^{-1}$ and
  $C_4^^7x^{-6}$. Now it is trivial to obtain the final answer.
\item We have to find the term independent of $x$ in $(1 + x + x^{-2} + x^{-3})^{10}$. Rewriting

  $(1 + x + x^{-2} + x^{-3})^{10} = \frac{1}{x^{30}}(x^3 + x^4 + x + 1) = \frac{1}{x^{30}}[(1 + x)(1 +
  x^3)]^{10}$.

  Powers of $x$ in first expansion are $0, 1, 2, 3, \ldots, 10$ and in the second expansion are $0, 3, 6, 9,
  \ldots, 30$. We need to add powers such that the sum is $30$. Such powers are $(0, 30), (3, 27), (6, 24),
  (9, 21)$.

  Thus, the term is $C_0^^{10}C_{10}^^{10} + C_3^^{10}C_9^^{10} + C_6^^{10}C_8^^{10} + C_9^^{10}C_7^^{10}$.
\item Powers of $x$ would be $1, 2, 3$ for $a_1, a_2, a_3$ respectively as we observe from the given series.

  Now we consider $(r + 1)$th term of $(1 + x^2)^2(1 + x)^n = (1 + 2x^2 + x^4)(1 + x)^n$.

  Coeff.\ of term containing $x$ would be $c_1 = C_1^^n$. Coeff.\ of term containing $x^2$ would be $c_2 =
  C_2^^n + 2.C_0^^n$ and coeff.\ of term containing $x^3$ would ve $c_3 = C_3^^n + 2.C_1^^n$.

  Given that $c_1, c_2, c_3$ are in A.P. $\Rightarrow C_3^^n + 3C_1^^n = 2C_2^^n + 4$

  $\Rightarrow \frac{n(n - 1)(n - 2)}{3!} + 3n = n(n - 1) + 4\Rightarrow n = 2, 3, 4$.
\item We have to show that $C_1^^m + C_2^^{m + 1} + C_3^^{m + 2} + \cdots + C_n^^{m + n - 1} = C_1^^n +
  C_2^^{n + 1} + C_3^^{n + 2} + \cdots + C_n^^{m + n - 1}$.

  L.H.S.\ $= C_1^^m + C_2^^{m + 1} + C_3^^{m + 2} + \cdots + C_n^^{m + n - 1}$

  Adding and subtracting $C_0^^m$ and applying $C_r^^n + C_{r + 1}^^n = C_{r + 1}^^{n + 1}$ repeatedly

  $\Rightarrow C_0^^m + C_1^^m + C_2^^{m + 1} + C_3^^{m + 2} + \cdots + C_n^^{m + n - 1}$

  $= C_1^^{m + 1} + C_2^^{m + 1} + C_3^^{m + 2} + \cdots + C_n^^{m + n - 1} = C_2^^{m + 2} + C_3^^{m + 2} +
  \cdots + C_n^^{m + n - 1}$

  $\ldots$

  $= C_{n - 1}^^{m + n - 1} + C_n^^{m + n - 1} = C_n^^{m + n}$

  Doing similar steps, R.H.S.\ $= C_m^^{m + n}$ and thus, L.H.S.\ = R.H.S.
\item We have $\displaystyle(1 + x + x^2)^n = \sum_{r=0}^{2n}a_rx^r$.

  (a) Putting $x = \frac{1}{x}$, we have $\left(1 + \frac{1}{x} + \frac{1}{x^2}\right)^n = \displaystyle\sum_{r
    = 0}^na_r\frac{1}{x^r}$

  $\Rightarrow (1 + x + x^2)^n = \displaystyle\sum_{r = 0}^na_rx^{2n - r}$.

  Putting $r = 2n - r$ we see that $a_r = a_{2n - r}$.

  (b) Putting $x = 1$ gives us $3^n = a_0 + a_1 + a_2 + a_3 + \cdots + a_{n - 1} + a_n + a_{n + 1} + \cdots
  + a_{2n}$

  Using the result obtained in first section we see that $a_0 = a_{2n}, a_1 = a_{2n - 1}, \ldots, a_{n - 1}
  = a_{n + 1}$, and thus,

  $2(a_0 + a_1 + \cdots + a_{n _ 1}) + a_n = 3^n\Rightarrow a_0 + a_1 + \cdots + a_{n _ 1} = \frac{1}{2}(3^n
  - a_n)$

  (c) Given, $(1 + x + x^2)^n = \displaystyle\sum_{r=0}^{2n}a_rx^r$

  Difference w.r.t.\ $x$ yields $n(1 + 2x)(1 + x + x^2)^{n - 1} = \displaystyle\sum_{r=0}^{2n}ra_rx^{r - 1}$

  Multiplying both sides by $(1 + x + x^2)$ yields

  $n(1 + 2x)(1 + x + x^2)^n  = n(1 + 2x)\displaystyle\sum_{r=0}^{2n}a_rx^r = (1 + x + x^2)\sum_{r = 0}^{2n}ra_rx^{r - 1}$

  Equating coefficients of $x^r(0 < r < 2n)$ gives us

  $na_r + 2na_{r - 1} = (r + 1)a_{r + 1} + ra_r + (r - 1)a_{r - 1}\Rightarrow (r+1)a_{r + 1} = (n - r)a_r +
  (2n - r + 1)a_{r - 1}$.
\item Given, $\displaystyle(1 - x^3)^n = \sum_{r=0}^na_r.x^r.(1 - x)^{3n- 2r}$.

  Rewriting $\frac{(1 - x^3)^n}{(1 - x)^{3n}} = \displaystyle\sum_{r = 0}^na_r.\frac{x^r}{(1 - x)^{2r}}$

  $\Rightarrow \displaystyle\left(\frac{1 + x + x^2}{(1 - x)^2}\right)^n = \sum_{r = 0}^{n}a_r\alpha^2$,
  where $\alpha = \frac{x}{(1 - x)^2}$

  $\Rightarrow (1 + 3\alpha)^n = \displaystyle\sum_{r=0}^na_r\alpha^r$

  Equating the coefficient of $\alpha^r$ yields $a_r = C_r^^n3^r$.
\item No.\ of terms in the expansion of $(1 + x)^{2n}$ is $2n + 1$, so middle term would be $n +
  1$. Coefficient of this term would be $C_n^^{2n}$.

  Coefficient of $x^n$ in the expansion of $(1 + x)^{2n - 1}$ is $C_{n}^^{2n - 1}$.

  $2.C_n^^{2n - 1} = 2\frac{(2n - 1)!}{n!(n - 1)!} = \frac{2n.(2n - 1)!}{n!n!} = \frac{(2n)!}{n!n!} =
  C_n^^{2n}$.
\item The middle term has the greatest coefficient. In $C_r^^{200}$ the middle term would be $101$st
  term. Coefficient of $101$st term is $C_{100}^^{200}$.
\item Let $r$ be the no.\ of people needed for making maximum no.\ of committees. So no.\ of committees is
  $C_r^^{20}$. Since the middle term has the largest coefficient so $11$th term willl have largest no.\ of
  committees. Thus, no.\ persons chosen should be $10$.
\item Let there be $a$ and $b$ permutations then $a + b = 2n$. Thus, no.\ of permutations is $C_a^^{2n}$ and
  $C_{2n - a}^^{2n}$ which are greatest when $a = b$ because those will be middle terms.
\item We consider the general $(r + 1)$th term's coefficient, which is, $C_r^^3^{7 - r}2^{r}$.

  This ciefficient shoul be equal to $(r + 2)$th term's coefficient, which is, $C_{r + 1}^^73^{6 - r}2^{r +
    1}$.

  Equating and solving gives us $r = 3$, so $4$th and $5$th terms have equal coefficients.
\item Let $(1 + 5x^2 - 7x^3)^{2000} = a_0 + a_1x + a_2x^2 + \cdots + a_{6000}^{6000}$.

  Putting $x = -1$, we get

  $a_0 + a_1 + a_2 + \cdots + a_{6000} = (1 + 5 - 7)^{2000} = 1$.
\item Given that sum of the binomial coefficients of the expansion $\left(3^{-\tfrac{x}{4}} +
  3^{\tfrac{5x}{4}}\right)^n$ is $64$.

  $(1 + 1)^n = 64 \Rightarrow n = 6$[We put $3^{-\frac{x}{4}} = 3^{\frac{5x}{4}} = 1$]

  Here middle term will be greatest term, which is $4$th term.

  According to question $t_4 = (n - 1) + t_3 = 5 + t_3$.

  $\Rightarrow C_3^^6\left(3^{-\frac{x}{4}}\right)^3\left(3^{\frac{5x}{4}}\right)^3 = 5 +
  C_2^^6\left(3^{-\frac{x}{4}}\right)^4\left(3^{\frac{5x}{4}}\right)^2$

  $\Rightarrow 20.3^{3x} = 5 + 15.3^{\frac{3x}{2}}\Rightarrow 20y^2 - 15y - 5 = 0$, where $y = 3^{\frac{3x}{2}}$

  $4y^2 - 3y - 1= 0 \Rightarrow y = 1, -\frac{1}{4}\Rightarrow y = 1[\because y = 3^{\frac{3x}{2}} > 0]$

  $\Rightarrow x = 0 \Rightarrow [\alpha] = 0\Rightarrow 0\leq\alpha< 1$.
\item Let $(5p - 4q)^n = a_0p^n + a_1p^{n - 1}q + a_2p^{n - 2}q^2 + \cdots + a_nq^n$

  Putting $p = 1, q = 1$ gives us

  $a_0 + a_1 + a_2 + \cdots + a_n = (5 - 4)^n = 1$.
\item Let $(1 - 3x + x^3)^{201}.(1 + 5x - 5x^2)^{503} = a_0 + a_1x + a_2x^2 + \cdots + a_{2112}x^{3520}$

  Putting $x = 1$ gives us

  $a_0 + a_1 + a_2 + \cdots + a_{2112} = (-1)^{201}.1^{503} = -1$.
\item Putting $x = 1$ in $(tx^2 - 2x + 1)^n$ we have sum of coefficients as $(t - 1)^n$. Similarly putting
  $x = 1, y = 1$ in $(x - ty)^n$ we have sum of coefficients as $(1 - t)^n$.

  Given that sum of coefficients is equal so $(t - 1)^n = (-1)^n(t - 1)^n$.

  Consider $n$ to be odd; then $(-1)^n = -1$ so there is no way these will be equal if $t - 1$ is not
  zero. So the only possible value of $(t - 1)^n = 0$, which gives $t = 1$.

  Similarly, if $n$ is even; then $(t - 1)^n = (1 - t)^n$ for all values of $n$.
\item Given, $(1 + x)^n = a_0 + a_1x + a_2x^2 + a_3x^3 + \cdots + a_nx^n$

  Putting $x = 1$ gives us $a_0 + a_1 + a_2 + a_3 + \cdots + a_n = 2^n$

  Putting $x = i$ gives us $(1 + i)^n = a_0 + a_1i - a_2 - a_3i + \cdots + a_ni^n$

  Taking modulus and squaring yields

  $2^n = (a_0 - a_2 + a_4 - \ldots)^2 + (a_1 - a_3 + a_5 - \ldots)^2$.

  Q.E.D.
\item Let $r$th term be the greatest term, which is given by $t_r = \sqrt{3}\left[C_{r -
    1}^^{20}\left(\frac{1}{\sqrt{3}}\right)^{r - 1}\right]$. Similarly $t_{r + 1} =
  \sqrt{3}\left[C_r^^{20}\left(\frac{1}{\sqrt{3}}\right)^r\right]$

  $\frac{t_r}{t_{r + 1}} = \frac{C_{r - 1}^^20}{C_r^^{20}}\sqrt{3} = \frac{r}{21 - r}\sqrt{3} \geq 1$

  $\Rightarrow \sqrt{3}r geq 21 - r \Rightarrow r\geq = 7.69$(approximately).

  Similarly $\frac{t_r}{t_{r - 1}}\geq 1 \Rightarrow r\leq 8.5$(approximately).

  The only integer between these limits is $8$. Hence, $t_8 = \frac{25840}{9}$ is the greatest term.
\item Let $t_r$ represent the $r$th term of the expansion of $(x + a)^{15}$.

  Now, $t_{11} = C_{10}^^{15}x^5a^{10}, t_8 = C_7^^{15}x^8a^7$, and $t_{12} = C_{11}^^{15}x^4a^{11}$

  Given, that $t_{11}$ is G.M.\ of $t_8$ and $t_{12}$, thus

  $\left(C_{10}^^{15}x^5a^{10}\right)^2 = C_7^^{15}x^8a^7.C_{11}^^{15}x^4a^{11}$

  Solving this gives us, $\frac{x}{a} = \sqrt{\frac{77}{75}}$.

  Let $t_r$ be the greatest term. Then $\frac{t_r}{t_{r + 1}} = \frac{r}{16 - r}.\frac{x}{a} \geq 1$

  $\Rightarrow r\geq 7.947 \Rightarrow r = 8$.

  $\Rightarrow t_8 = \frac{15!}{7!8!}\left(\frac{77}{75}\right)^4a^{15}$.
\item $t_{n + 1} = C_n^^{2n}x^n, t_{n + 2} = C_{n + 1}^^{2n}x^{n + 1}$, and $t_n = C_{n - 1}^^{2n}x^{n - 1}$.

  $\frac{t_{n + 1}}{t_{n + 2}} = \frac{n + 1}{n}.\frac{1}{x}$, and $\frac{n + 1}{n}x$.

  Since $t_{n + 1}$ is the only greatest term popssible here.

  $\Rightarrow \frac{t_{n + 1}}{t_{n + 2}} > 1\Rightarrow x < \frac{n + 1}{n}$, and similarly $\frac{t_{n +
      1}}{t_n} > 1 \Rightarrow x > \frac{n}{n + 1}$.

  Thus, $x\in\left(\frac{n}{n + 1}, \frac{n + 1}{n}\right)$, which is given as $x\in\left(\frac{10}{11},
  \frac{11}{10}\right)$, and thus $n = 10$.

  Given that $t_4$ in the expansion of $\left(kx + \frac{1}{x}\right)^m$ is $\frac{n}{4}$.

  $t_4 = \frac{n}{4} = \frac{5}{2} =  C_3^^m(kx)^{m - 3}.\frac{1}{x^3} \Rightarrow C_3^^mk^{m - 3}x^{m - 6} =
  \frac{5}{2}$

  Since the term is independent of $x \Rightarrow m = 6 \Rightarrow C_3^^6k^3 = \frac{5}{2}\Rightarrow k =
  \frac{1}{2}\Rightarrow mk = 3$.
\item $t_4 = C_3^^{10}2^7\left(\frac{3}{8}x\right)^3, t_3 = C_2^^{10}2^8\left(\frac{3}{8}x\right)^2$, and
  $t_5 = C_4^^{10}2^6\left(\frac{3}{8}x\right)^4$.

  Given that $4$th term has greatest numerical value, so $\frac{t_4}{t_3} > 1 \Rightarrow
  \frac{2!8!}{3!7!}\frac{3}{16}x > 1 \Rightarrow \frac{x}{2} > 1$, and

  $\frac{t_4}{t_5} > 1 \Rightarrow \frac{4!6!}{3!7!}\frac{16}{3x} > 1 \Rightarrow x < \frac{64}{21}$

  $\Rightarrow 2 < x < \frac{64}{21}$.
\item Let the binomial expansion be $(x + y)^n$ and $a, b,$ and $c$ be the coefficients of $r$th, $(r +
  1)$th, and $(r + 2)$th terms respectively. Then,

  $a = C_{r - 1}^^n, b = C_r^^n, c = C_{r + 1}^^n$.

  Discriminant of the quadratic equation $ax^2 + 2bx + c = 0$ is $D = 4b^2 - 4ac = 4b^2\left(1 -
  \frac{a}{b}.\frac{c}{b}\right)$

  $= 4(C_r^^n)^2\left(1 - \frac{C_{r - 1}^^n}{C_r^^n}.\frac{C_{r + 1}^^n}{C_r^^n}\right) =
  4(C_r^^n)^2\left(1 - \frac{r}{n - r + 1}.\frac{n - r}{r + 1}\right)$

  $= 4(C_r^^n)^2.\frac{n + 1}{(n - r + 1)(r + 1)} > 0$.

  Hence, roots of the quadratic equation are real and unequal.
\item $9^n + 7 = (1 + 8)^n + 7 = C_0^^n + C_1^^n.8 + C_2^^n.8^2 + \cdots + C_n^^n.8^n + 7$

  $= 8\left(1 + C_1^^n + C_2^^n.8 + C_3^^n.8^2 + \cdots + C_n^^n.8^{n - 1}\right)$

  $= 8.$an integer.

  Thus, $9^n + 7$ is divisible by $8$.
\item For $n = 1, 3^{2n+ 1} + 2^{n + 2} = 3^3 + 2^3 = 27 + 8 = 35$, which is divisible by $7$.

  Let it be true for $n = m$ i.e. $3^{2m + 1} + 2^{m + 2} = 7k$, where $k\in\mathbb{N}$.

  For $n = m + 1, 9.3^{2m + 1} + 2.2^{m + 2} = 7.3^{2m + 1} + 2(3^{2m + 1} + 2^{m + 2}) = 7.3^{2m + 1} +
  7k$, which is divisible by $7$.

  Q.E.D.
\item Let $C_{r - 1}^^n, C_r^^n, C_{r + 1}^^n$ be in G.P. $\therefore \frac{C_r^^n}{C_{r - 1}^^n} =
  \frac{C_{r + 1}^^n}{C_r^^n}$

  $\Rightarrow \frac{n - r + 1}{r} = \frac{n - (r + 1) + 1}{r + 1} \Rightarrow n = -1$, which is not
  possible.

  Let $C_{r - 1}^^n, C_r^^n, C_{r + 1}^^n$ be in H.P. $\Rightarrow \frac{2}{C_r^^n} = \frac{1}{C_{r - 1}^^n}
  + \frac{1}{C_{r + 1}^^n}$

  $\Rightarrow 2 = \frac{n - r + 1}{r} + \frac{r + 1}{n - r}\Rightarrow (n - 2r)^2 + n = 0$, whihc is not
  possible.
\item $a_n =$ coefficient of $x^n$ in $(1 + x + x^2)^n =$ coefficient of $y^n$ in $(1 + y + y^2)^n$.

  Given, $(1 + x + x^2)^n = a_0 + a_1x + a_2x^2 + a_3x^3 + \cdots + a_{2n}x^{2n}$.

  Putting $x = -\frac{1}{x}$, we get $(x^2 - x + 1)^n = a_0x^{2n} - a_1x^{2n - 1} + a_2x^{2n - 2} - \cdots +
  a_{2n}$

  $\therefore a_0^2 -a_1^2 + a_2^2 - \cdots + a_{2n}^2 =$ coefficient of $x^{2n}$ in $(1 + x + x^2)^n(x^2 -
  x + 1) = (x^4 + x^2 + 1)$

  $=$ coefficient of $y^n$ in $(1 + y + y^2)^n$, where $y = x^2$

  $= a_n$.
\item We have $(1 + x + x^2)^n = a_0 + a_1x + a_2x^2 + a_3x^3 + \cdots + a_{2n}x^{2n}$ and we have proved
  that

  $a_0^2 -a_1^2 + a_2^2 - \cdots + a_{2n}^2 = a_n$.

  Putting $x = \frac{1}{x}$, we have

  $(x^2 + x + 1) = a_0x^{2n} + a_1x^{2n - 1} + \cdots + a_{2n}$.

  Equating the coefficients of same powers of $x$ gives us $a_0 = a_{2n}, a_1 = a_{2n - 1}, \cdots$. This
  also follows from earlier result where we proved $a_r = a_{2n - r}$.

  Thus, $a_0^2 - a_1^2 + a_2^2 - \cdots + (-1)^{n - 1}a_{n - 1}^2 + (-1)^na_n^2 + (-1)^{n + 1}a_{n + 1}^2 +
  \cdots + a_{2n}^2 = 0$

  $\Rightarrow a_0^2 -a_1^2 + a_2^2 - \cdots + (-1)^na_{n - 1}^2 = \frac{1}{2}a_n[1 - (-1)^na_n]$.

  Q.E.D.
\item Let $E = \displaystyle\sum_{0\leq i< j}\sum_{0\leq j\leq n}(C_i + C_j)^2$, where $i = 0, 1, 2, \ldots,
  (n - 1)$, and $j = 1, 2, 3, \ldots, n$ and $i < j$.

  $E = \displaystyle n(C_0^2 + C_1^2 + \cdots + C_n^2) + 2\sum_{0\leq i< j}\sum_{0\leq j\leq n}C_iC_j$

  $= n.C_n^^{2n} + [(C_0 + C_1 + C_2 + \ldots + C_n)^2 - (C_0^2 + C_1^2 + \cdots + C_n^2)]$

  $= n.C_n^^{2n} + (2^n)^2 - C_n^^{2n} = (n - 1)C_n^^{2n} + 2^{2n}$.
\item Let $E = \displaystyle\sum_{0\leq i<j}\sum_{0\leq j\leq n}(i + j)C_iC_j$, where $i = 0, 1, 2, \ldots,
  (n - 1)$, and $j = 1, 2, 3, \ldots, n$ and $i < j$. Clearly, $n - i = n, (n - 1), (n - 2), \ldots, 3, 2,
  1$ and $n - j = n - 1, n - 2, n - 3, \ldots, 2, 1, 0$.

  Thus, we see that $n - j$ behaves as $i$ and $n - i$ behaves as $j$.

  $\Rightarrow E = \displaystyle\sum_{0\leq i<j}\sum_{0\leq j\leq n} (n - j + n - i)C_{n - j}C_{n - i} =
  \displaystyle\sum_{0\leq i<j}\sum_{0\leq j\leq n}[2n - (i + j)]C_iC_j[\because C_r = C_{n - r}]$

  $= 2n.\displaystyle\sum_{0\leq i<j}\sum_{0\leq j\leq n}C_iC_j - E \Rightarrow 2E =
  2n\displaystyle\sum_{0\leq i<j}\sum_{0\leq j\leq n}C_iC_j$

  $E = \frac{n}{2}[(C_0 + C_1 + \cdots + C_n)^2 - (C_0^2 + C_1^2 + \cdots + C_n^2)] = \frac{n}{2}(2^{2n} -
  C_n^^{2n})$.
\item L.H.S.\ $= \frac{n!}{(m + n)!}\left[\frac{(m + n)!}{m!n!}C_0 + \frac{n(m + n)!}{(m + 1)!n!}C_1 +
  \cdots + \frac{n!(m + n)!}{(m + n)!n!}C_n\right]$

  $= \frac{n!}{(m + n)!}[C_n^^{m + n}C_0^^n + C_{n - 1}^^{m + n}C_1^^n + \cdots + C_0^^{m + n}C_n^^n]$

  We know that $(1 + x)^{m + n} = C_0^^{m + n} + C_1^^{m + n}x + C_2^^{m + n}x^2 + \cdots + C_{m + n}^^{m +
    n}x^{m + n}$

  and $(1 + x)^n = C_0^^n + C_1^^nx + C_2^^nx^2 + \cdots + C_n^^nx^n$

  Coefficient of $x^n$ in $(1 + x)^{m + n}(1 + x)^n = C_n^^{m + n}C_0^^n + C_{n - 1}^^{m + n}C_1^^n + \cdots
  + C_0^^{m + n}C_n^^n = C_n^^{m + 2n}$

  $\therefore$ L.H.S.\ $= \frac{n!}{(m + n)!}C_n^^{m + 2n} = \frac{(m + 2n)!}{(m + n)!(m + n)!} =$ R.H.S.
\item $r$th factor of $(C_0 + C_1)(C_1 + C_2)(C_2 + C_3)\cdots (C_{n- 1} + C_n)$ is given by

  $t_r = C_{r - 1} + C_r = C_r^^{n + 1} = \frac{n + 1}{r}.C_{r - 1}$.

  Now, $(C_0 + C_1)(C_1 + C_2)(C_2 + C_3)\cdots (C_{n- 1} + C_n) = t_1.t_2.t_3\ldots t_n$

  $= \left(\frac{n + 1}{1}C_0\right)\left(\frac{n + 1}{2}C_1\right)\left(\frac{n +
    1}{3}C_2\right)\cdots\left(\frac{n + 1}{n}C_{n - 1}\right)$

  $= \frac{(n + 1)^n}{n!}C_1C_2\ldots C_n$.
\item L.H.S.\ $= \frac{1}{n!}\left[\frac{n!}{1!(n - 1)!} + \frac{n!}{3!(n - 3)!} + \frac{n!}{5!(n - 5)!} +
  \cdots + \frac{n!}{(n - 1)!1!}\right]$

  $= \frac{1}{n!}(C_1 + C_3 + C_5 + \cdots + C_{n - 1}) = \frac{2^{n - 1}}{n!}$.
\item R.H.S.\ $= \displaystyle\sum_{r = 0}^n(-1)^r\frac{C_r^^n}{C_r^^{r + 3}} = \sum_{r =
  0}^n(-1)^r\frac{n!}{r!(n - r)!}.\frac{r!3!}{(r + 3)!}$

  $= 3!\displaystyle\sum_{r = 0}^n(-1)^r\frac{n!}{(n - r)!(r + 3)!} = \frac{3!}{(n + 1)(n + 2)(n +
  3)}\sum_{r = 0}^n(-1)^r\frac{(n + 3)!}{(r + 3)!(n - r)!}$

  $= \frac{3!}{(n + 1)(n + 2)(n + 3)}\displaystyle\sum_{r = 0}^n(-1)^rC_{r + 3}^^{n + 3}$

  $= \frac{3!}{(n + 1)(n + 2)(n + 3)}[C_3^^{n + 3} - C_4^^{n + 3} + C_5^^{n + 3} - \cdots + (-1)^nC_{r +
    3}^^{n + 3}]$

  $\because C_0^^{n + 3} - C_1^^{n + 3} + C_2^^{n + 3} - C_3^^{n + 3} + \cdots + (-1)^{n + 3}C_{n + 3}^^{n +
  3} = (1 - 1)^{n + 3} = 0$

  $\Rightarrow C_3^^{n + 3} - C_4^^{n + 3} + C_5^^{n + 3} - \cdots + (-1)^nC_{r + 3}^^{n + 3} = C_0^^{n + 3}
  - C_1^^{n + 3} + C_2^^{n + 3}$.

  $\Rightarrow \frac{3!}{(n + 1)(n + 2)(n + 3)}[C_3^^{n + 3} - C_4^^{n + 3} + C_5^^{n + 3} - \cdots + (-1)^nC_{r +
    3}^^{n + 3}] = \frac{3!}{(n + 1)(n + 2)(n + 3)}\left[1 - (n + 3) + \frac{(n + 3)(n + 2)}{2}\right] =
  \frac{3!}{2(n + 3)}$.
\item $C_0^^n = C_0^^{n - 1}, -C_1^^n = -C_0^^{n - 1} - C_1^^{n - 1}, C_2^^n = C_1^^{n - 1} + C_2^^{n - 1},
  \ldots, (-1)^{m - 1}C_{m - 1}^^n = (-1)^{m - 1}C_{m - 2}^^{n - 1} + (-1)^{m - 1}C_{m - 1}^^{n - 1}$

  Adding gives us

  $C_0 - C_1 + C_2 - \cdots + (-1)^{m - 1}C_{m - 1} = (-1)^{m - 1}C_{m - 1}^^{n - 1} = (-1)^{m - 1}\frac{(n
    - 1)!}{(m - 1)!(n - m)!} = (-1)^{m - 1}\frac{(n - 1)(n - 2)\cdots (n - m + 1)}{(m - 1)!}$.
\item Let $d$ be the common divisor of $C_1^^{2n}, C_3^^{2n}, C_5^^{2n}, \ldots, C_{2n - 1}^^{2n}$.

  We know that $C_1^^{2n} + C_3^^{2n} + C_5^^{2n} + \cdots + C_{2n - 1}^^{2n} = 2^{2n - 1}$.

  Thus, because $d$ is a common divisor so it will have a form of $2^k$ because it has to divide $2^{2n -
    1}$. Thus, $0 < k\leq 2n - 1$.

  Let $n = 2^m.r$, where $r$ is an odd positive integer. $\Rightarrow 2n = 2^{m + 1}.r\Rightarrow C_1^^{2n}
  = 2n = 2^{m + 1}.r$

  Thus, common divisor $\leq 2^{m + 1}$. We claim that $2^{m + 1}$ divides all of $C_1^^{2n}, C_3^^{2n},
  C_5^^{2n}, \ldots, C_{2n - 1}^^{2n}$.

  For odd positive integer $p, C_p^^{2n} = \frac{2n}{p}C_{p - 1}^^{2n - 1} = \frac{2^{m + 1}r}{p}C_{p -
    1}^^{2n - 1} = 2^{m + 1}\left(\frac{r.C_{p - 1}^^{2n - 1}}{p}\right)$.
\item $2\displaystyle\sum_{r=0}^nC_r^^n.\sin rx\cos(n - r)x = (C_0^^n\sin0x \cos nx + C_n^^n\sin nx \cos0x)
  + (C_1^^n\sin x\cos(n - 1)x + C_{n - 1}^^n\sin(n - 1)x\cos x) + \cdots + (C_n^^n\sin nx\cos0x +
  C_0^^n\sin0x\cos nx)$

  $=(C_0 + C_1 + C_2 + \cdots + C_n)\sin nx = 2^n\sin nx$

  $\Rightarrow \displaystyle\sum_{r=0}^nC_r^^n.\sin rx\cos(n - r)x = 2^{n- 1}\sin nx$.
\item We have proven earlier that $C_1 + 2C_2 + 3C_3 + \cdots + nC_n = n.2^{n - 1}$.

  Rewriting $a.C_0 + (a - b).C_1 + (a - 2b).C_2 + \cdots + (a - nb).C_n = a(C_0 + C_1 + C_2 + \cdots + C_n)
  - b(C_1 + 2C_2 + 3C_3 + \cdots + nC_n)$

  $= a.2^n - bn2^{n - 1} = 2^{n - 1}(2a - nb)$.
\item Given, $a^2.C_0 - (a - 1)^2.C_1 + (a - 2)^2.C_2 - \cdots + (-1)^n(a - n)^2.C_n = 0,\;n>3$.

  $= a^2[C_0 - C_1 + C_2 - \cdots + (-1)^nC_n] + 2a[C_1 - 2C_2 + 3C_3 - \cdots + (-1)^nC_n] - [C_1 - 2^2C_2
  + 3^2C_3 - \cdots]$

  All three series' have been proven equal to zero earlier, and thus, sum is zero.
\item Given that $a_0, a_1, a_2, \ldots, a_n$ form an A.P. Let $d$ be the common difference of this
  A.P. Then

  $a_1 = a_0 + d, a_2 = a_0 + 2d, \ldots, a_n = a_0 + nd$.

  We have to prove that $a_0 - a_1.C_1 + a_2C_2 - \cdots + (-1)^na_nC_n = 0$

  L.H.S.\ $= a_0 - (a_0 + d)C_1 + (a_0 + 2d)C_2 - \cdots + (-1)^n(a_0 + nd)C_n$

  $= a_0(C_0 - C_1 + C_2 - \cdots + (-1)^nC_n) - d(C_1 - 2C_2 + \cdots - (-1)^nnC_n)$

  We have proven the two series in question equal to be zero. Q.E.D.
\item We have to prove that $\displaystyle\sum_{r=0}^n(-1)^r(a - r)(b - r)C_r = 0$.

  $\displaystyle\sum_{r=0}^n(-1)^r(a - r)(b - r)C_r = \sum_{r = 0}^n(-1)^r[ab - (a + b)r + r^2]C_r$

  This will lead to three series $[C_0 - C_1 + C_2 - \cdots + (-1)^nC_n], [C_1 - 2C_2 - \cdots +
    (-1)^nnC_n]$ and $[C_1 - 2^2C_2 + 3^2C_3 - \cdots + (-1)^nn^2C_n]$.

  We have proven the three series in question equal to be zero. Q.E.D.
\item We have to prove that $\displaystyle\sum_{r=0}^n(-1)^r(a - r)(b - r)(c - r)C_r = 0$.

  $\Rightarrow \displaystyle\sum_{r=0}^n(-1)^r(a - r)(b - r)(c - r)C_r = 0 = \sum_{r = 0}^n(-1)^r[abc -(ab +
  bc + ca)r + (a^2 + b^2 + c^2)r^2 - r^3]C_r = 0$.

  Out of these four series first three have been proven to be equal to zero. Now we will prove that
  $\displaystyle\sum_{r = 0}^n(-1)^rr^3C_r = 0$.

  Consider $(1 - x)^n = C_0 - C_1x + C_2x^2 - \cdots + (-1)^nC_nx^n$

  Differentiating w.r.t. $x$ gives us

  $-n(1- x)^{n - 1} = -C_1 + 2C_2x - 3C_3x^2 + \cdots + (-1)^nnC_nx^{n - 1}$

  Now we multiply with $x$ and differentiate again to get

  $-n(1 - x)^{n - 1} + n(n - 1)x(1 - x)^{n - 2} = -C_1 + 2^2C_2x - 3^2C_3x^2 + \cdots + (-1)^nn^2C_nx^{n -
    1}$

  Repeating previous step and putting $x = 1$ gives us

  $-C_1 + 2^3C_2 - 3^3C_3 + \cdots + (-1)^nn^3C_n= 0$. Q.E.D.
\item We have to prove that $\frac{C_0}{2^n} + \frac{2.C_1}{2^n} + \cdots + \frac{(n + 1)C_n}{2^n} = 16$.

  Earlier we have proven that $C_0 + 2C_1 + 3C_2 + \cdots + (n + 1)C_n = (n + 2)2^{n - 1}$

  Substituting this result gives us

  $\frac{n + 2}{2} = 16 \Rightarrow n = 30$.
\item Clearly, $a_2 = a_1 + d, a_3 = a_1 + 2d, \ldots, a_{n + 1} = a_1 + nd$, where $d$ is the common difference
  of the A.P.

  We have to prove that $\displaystyle\sum_{k=0}^na_{k + 1}C_k = 2^{n - 1}(a_1 + a_{n +1})$.

  L.H.S.\ $= a_1C_0 + a_2C_1 + a_3C_2 + \cdots + a_{n + 1}C_n = a_1(C_0 + C_1 + C_2 + \cdots + C_n) + d(C_1
  + 2C_2 + \cdots + nC_n) = a_1.2^n + d.n.2^{n - 1}$

  $= 2^{n - 1}(2a_1 + nd) = 2^{n - 1}(a_1 + a_{n + 1}) =$ R.H.S.
\item $S = a + (a + d)C_1 +(a+ 2d)C_2 + \cdots + (a +nd)C_n = a(C_0 + C_1 + C_2 + \cdots + C_n) + d(C_1
  + 2C_2 + \cdots + nC_n) = a.2^n + d.n.2^{n - 1} = 2^{n - 1}[2a + nd] = 2^n\frac{2a + nd}{2} =
  2^n.\frac{s}{n + 1}$

  $\Rightarrow (n + 1)S = 2^n.s$
\item Given that $(1 + x + x^2 + \cdots + x^p)^n = a_0 + a_1x + a_2x^2 + \cdots + a_{np}x^{np}$

  Differentiating w.r.t.\ $x$ and putting $x = 1$ gives us

  $n(1 + 2 + 3 + \cdots + p)(p + 1)^{n - 1} = a_1 + 2a_2 + 3a_3 + \cdots + npa_{np}$

  $\Rightarrow \frac{1}{2}np(p + 1)^n = a_1 + 2a_2 + 3a_3 + \cdots + npa_{np}$.
\item We have to prove that $\displaystyle\sum_{k=0}^{15}\frac{C_k^^{15}}{(k + 1)(k + 2)} = \frac{2^{17} -
  18}{16.17}$.

  Consider $(1 + x)^{15} = C_0^^{15} + C_1^^{15}x + C_2^^{15}x^2 + \cdots + C_{15}^^{15}x^{15}$

  Integrating w.r.t.\ $x$ with limits $0$ and $x$ gives us

  $\left[\frac{(1 + x)^{16}}{16}\right]_0^x = \left[C_0^^{15} + \frac{C_{1}^^{15}}{2}x^2 + \cdots +
    \frac{C_{15}^^{15}x^{16}}{16}\right]_0^x$

  $\Rightarrow \frac{(1 + x)^{16}}{16} - \frac{1}{16} = C_0^^{15} + \frac{C_{1}^^{15}}{2}x^2 + \cdots +
  \frac{C_{15}^^{15}x^{16}}{16}$

  Integrating again w.r.t.\ $x$ with limits $0$ and $1$ we get

  $\displaystyle\sum_{k=0}^{15}\frac{C_k^^{15}}{(k + 1)(k + 2)} = \left[\frac{(1 + x)^{17}}{16.17} -
    \frac{x}{16}\right]_0^1$

  $= \frac{2^{17}}{16.17} - \frac{1}{16.17} - \frac{1}{16} = \frac{2^{17} - 18}{16.17}$.

  Q.E.D.
\item $(1 - x)^n = C_0 - C_1x + C_2x^2 - \cdots + (-1)^nC_nx^n$

  Substituting $x = x^3$, we have

  $(1 - x^3)^n = C_0 - C_1x^3 + C_2x^6 - \cdots + (-1)^nC_nx^{3n}$

  Integrating within the limits of $0$ and $1$, we deduce

  $\left[C_0x - C_1\frac{x^4}{4} + C_2.\frac{x^7}{7} - \cdots + (-1)^nC_n\frac{x^{3n + 1}}{3n + 1}\right] =
  \displaystyle\int_0^1(1 - x^3)^ndx$

  Now we will evaluate the R.H.S. Let $I_n = \displaystyle\int_0^n(1 - x^3)^ndx = [x(1 - x^3)^n]_0^1 -
  \int_0^1x.n(1 - x^3)^{n - 1}.(-3x^2)dx$

  $= -3n\displaystyle\int_0^1(1 - x^3)^{n - 1}(1 - x^3 - 1)dx = -3nI_n + 3nI_{n - 1} \Rightarrow
  \frac{I_n}{I_{n - 1}} = \frac{3n}{3n + 1}$

  Now, $\frac{I_n}{I_0} = \frac{I_n}{I_{n - 1}}.\frac{I_{n - 1}}{I_{n -
      1}}\cdots\frac{I_3}{I_2}.\frac{I_2}{I_1}.\frac{I_1}{I_0}$

  $= \frac{3n}{3n + 1}.\frac{3n - 3}{3n - 2}.\frac{3n - 6}{3n - 5}\cdots \frac{3}{4} =
  \frac{3^n.n!}{4.7.\cdots(3n + 1)}$.
\item We have proven earlier that $\frac{C_0}{1.2} - \frac{C_1}{2.3} + \frac{C_2}{3.4} - \cdots +
  (-1)^n\frac{C_n}{(n + 1)(n + 2)} = \frac{1}{n + 2}$.

  Thus, $\displaystyle\sum_{r=0}^n\frac{(-1)^rC_r}{(r + 1)(r + 2)} = \frac{1}{n + 2}$.
\item We have to prove that $\displaystyle\sum_{r=0}^n\frac{C_r.3^{r + 3}}{(r + 1)(r + 2)(r + 3)} =
  \frac{4^{n + 3} - 1 - \frac{3}{2}(n + 3)(3n + 8)}{(n + 1)(n + 2)(n + 3)}$.

  $(1 + x)^n = C_0 + C_1x + C_2x^2 + \cdots + C_nx^n$

  Integrating within limits $0$ and $x$ gives us

  $\frac{(1 + x)^{n + 1}}{n + 1} - \frac{1}{n + 1} = C_0x + \frac{C_1}{2}x^2 + \frac{C_2}{3}x^3 + \cdots +
  \frac{C_n}{n + 1}x^{n + 1}$

  Integrating again with limits $0$ and $x$ gives us

  $\frac{(1 + x)^{n + 2}}{(n + 1)(n + 2)} - \frac{1}{(n + 1)(n + 2)} - \frac{x}{n + 1} = \frac{C_0}{2}x^2 +
  \frac{C_1}{2.3}x^3 + \frac{C_2}{3.4}x^4 + \cdots + \frac{C_n}{(n + 1)(n + 2)}x^{n + 2}$

  Integrating again with limits $0$ and $3$ gives us

  $\frac{4^{n + 3}}{(n + 1)(n + 2)(n + 3)} - \frac{1}{(n + 1)(n + 2)(n + 3)} - \frac{3}{(n + 1)(n + 2)} -
  \frac{9}{2(n + 1)} = \displaystyle\sum_{r=0}^n\frac{C_r.3^{r + 3}}{(r + 1)(r + 2)(r + 3)}$

  $\displaystyle\sum_{r=0}^n\frac{C_r.3^{r + 3}}{(r + 1)(r + 2)(r + 3)} = \frac{4^{n + 3} - 1 -
    \frac{3}{2}(n + 3)(3n + 8)}{(n + 1)(n + 2)(n + 3)}$.
\item $(1 + x)^n = C_0 + C_1x + C_2x^2 + \cdots + C_nx^n$

  Integrating within limits $0$ and $x$ gives us

  $\frac{(1 + x)^{n+ 1}}{n + 1} - \frac{1}{n + 1} = C_0x + \frac{C_1}{2}x^2 + \frac{C_2}{3}x^3 + \cdots +
  \frac{C_n}{n + 1}x^{n + 1}$

  Multiplying with $x$ and differentiang w.r.t.\ $x$ gives us

  $\frac{(1 + x)^{n + 1} + (n + 1)x(1 + x)^{n}}{n + 1} - \frac{1}{n + 1} = 2C_0x + \frac{3}{2}C_1x^2 +
  \frac{4}{3}C_2x^3 + \cdots + \frac{n + 2}{n + 1}C_nx^{n + 1}$

  Putting $x = 1$ gives us

  $\displaystyle\sum_{r=0}^n\frac{r + 2}{r + 1}C_r = \frac{2^n(n + 3) - 1}{n + 1}$.
\item Proceeding like previous to previous problem and adjusting last step with integration between $0$ and
  $x$ and integrating once more for the fourth time and putting $x = 3$ we get the desired result.
\item $(1 + x)^n = C_0 + C_1x + C_2x^2 + \cdots + C_nx^n$ and $(x + 1)^n = C_0x^{n} + C_1x^{n - 1} + C_2x^{n
  - 2} + \cdots + C_n$

  Multiplying and equating the power of $x^{n - 3}$ gives us

  $C_0C_3 + C_1C_4 + C_2C_5 + \cdots + C_{n - 3}C_n =$ coefficient of $x^{n - 3}$ in $(1 + x)^{2n}$

  $\Rightarrow \displaystyle\sum_{r=0}^{n - 3}C_rC_{r + 3} = \frac{(2n)!}{(n + 3)!(n - 3)!}$.
\item We have to find $\displaystyle\sum_{0\leq i< j}\sum_{0\leq j\leq n}C_iC_j$.

  $\displaystyle\left(\sum C_k\right)^2 = \sum C_k^2 + 2\sum_{0\leq i< j}\sum_{0\leq j\leq n}C_iC_j$.

  We know that $\left(\sum C_k\right)^2 = (2^n)^2 = 2^{2n}$ and $\sum C_k^2 = \frac{2n!}{n!n!}$

  $\therefore \displaystyle\sum_{0\leq i< j}\sum_{0\leq j\leq n}C_iC_j = 2^{2n - 1} - \frac{(2n - 1)!}{n!(n -
    1)!}$.
\item Given that $S_n = C_0C_1 + C_1C_2 + \cdots + C_{n - 1}C_n$.

  $(1 + x)^n = C_0 + C_1x + C_2x^2 + \cdots + C_nx^n$ and $(x + 1)^n = C_0x^{n} + C_1x^{n - 1} + C_2x^{n
  - 2} + \cdots + C_n$

  Multiplying and equating power of $x^{n - 1}$ gives us

  $S_n = C_0C_1 + C_1C_2 + \cdots + C_{n - 1}C_n =$ coefficient of $x^{n - 1}$ in $(1 + x)^{2n} = C_{n -
    1}^^{2n}$

  $\Rightarrow \frac{S_{n + 1}}{S_n} = \frac{(2n + 2)!}{n!(n + 2)!}.\frac{(n - 1)!(n + 1)!}{(2n)!} =
  \frac{(2n + 2)(2n + 1)}{n(n +2)} = \frac{15}{4}\Rightarrow n = 2, 4$.
\item $(1 + x)^n = C_0 + C_1x + C_2x^2 + \cdots + C_nx^n$ and $(x + 1)^n = C_0x^{n} + C_1x^{n - 1} + C_2x^{n
  - 2} + \cdots + C_n$.

  Multiplying first equation with $x$ and differentiating first equation w.r.t.\ $x$ gives us

  $(1 + x)^n + nx(1 + x)^{n - 1} = C_0 + 2C_1x + 3C_2x^2 + \cdots + (n + 1)C_nx^n$.

  Multiplying this with second equation and then equating coefficient of $x^n$ gives us

  $C_0^2 + 2.C_1^2 + 3.C_2^2 + \cdots + (n + 1)C_n^2 =$ coefficient of $x^n$ in $(1 + x)^{2n} + nx(1 +
  x)^{2n - 1}$

  $= C_n^^{2n} + n.C_{n - 1}^^{2n - 1} = \frac{(n + 2)(2n - 1)!}{n!(n - 1)!}$.
\item We have $x^n(2 + x)^n = (2x + x^2)^n = [(x + 1)^2 - 1]^n$

  $\Rightarrow x^n(2^n + C_1^^n.2^{n - 1}+ \cdots + C_n^^nx^n) = C_0^^n(x + 1)^{2n} - C_1^^n(x + 1)^{2n - 2}
  + C_2^^n(x + 1)^{2n - 4} - \cdots$

  Equating coefficients of $x^n$ gives us

  $C_0.C_n^^{2n} - C_1.C_n^^{2n - 2} + C_2.C_n^^{2n - 4} - \cdots = 2^n$.
\item We have to find $\displaystyle\sum_{0\leq i\leq j}\sum_{0\leq j\leq n}(i + j)(C_i + C_j + C_iC_j)$.

  Earlier we have proven that $\displaystyle\sum_{0\leq i\leq j}\sum_{0\leq j\leq n}(i + j)(C_iC_j) =
  \frac{n}{2}\left(2^{2n1} - \frac{2n!}{(n!)^2}\right)$

  Now $\displaystyle\sum_{0\leq i\leq j}\sum_{0\leq j\leq n}(i + j)(C_i + C_j)$ we proceed similarly to get

  $\displaystyle\sum_{0\leq i\leq j}\sum_{0\leq j\leq n}(i + j)(C_i + C_j) = \sum_{0\leq i\leq j}\sum_{0\leq
    j\leq n}(2n - i - j)(C_i + C_j) = E$

  $\Rightarrow 2E = \sum_{0\leq i\leq j}\sum_{0\leq j\leq n}2n(C_i + C_j) = 2n\sum_{0\leq j\leq n}[(C_0+
    C_j) + (C_1 + C_j) + \cdots + (C_{j - 1} + C_j)]$

  $= 2n[(2C_0 + C_1 + C_2 + \cdots + C_n) + (C_0+ 2C_1 + C_2 + \cdots + C_n) + \cdots + (C_0 + C_1 + \cdots +
  2C_{n - 1}) + C_n]$

  $= 2n^2[C_0 + C_1 + C_2 + \cdots + C_n] \Rightarrow E = n^2.2^n$
\item Given that $(1 + x + x^2)^n = a_0 + a_1x + a_2x^2 + \cdots + a_{2n}x^{2n}$.

  Putting $x = -x$ gives us $(1 - x + x^2)^n = a_0x^{2n} - a_1x^{2n - 1} + a_2x^{2n - 2} - \cdots + a_{2n}$

  Multiplying and equating coefficients of $x^{2n - 2r}$ gives us

  $a_0a_{2r} - a_1a_{2r + 1} + a_2a_{2r + 2} - \cdots + a_{2n - 2r}a_{2n} =$ coefficient of $x^{2n - 2r}$
  in $(x^4 + x^2 + 1)^{n}$

  Putting $x^2 = y$ it is coefficient of $y^{n - 2}$ in $(1 + y + y^2) = a_{n - r}$.

  Earlier we have proven that for the given series $a_r = a_{2n - r}\Rightarrow a_{n - r} = a_{n + r}$.
\stopitemize