% -*- mode: context; -*-
\chapter{Binomials, Multinomials and Expansions}
\startitemize[n, 1*broad]
\item Using binomial theorem, $\left(x + \frac{1}{x}\right)^5 = C_0^^5x^5 + C_1^^5x^4.\frac{1}{x} +
  C_2^^5x.\frac{1}{x^2} + C_3^^5x^2.\frac{1}{x^3} + C_4^^5x.\frac{1}{x^4} + C_5^^5.\frac{1}{x^5}$

  $= C_0x^5 + C_1^^5x^3 + C_2^^5x + C_3^^5.\frac{1}{x} + C_4^^5.\frac{1}{x^3} + C_5^^5.\frac{1}{x^5}$.
\item $(10.1)^5 = (10 + 0.1)^5$, so we proceed like previous problem to get

  $(10.1)^5 = C_0^^510000 + C_1^^51000 + C_2^^510 + C_3^^5\frac{1}{10} + C_4^^5\frac{1}{1000} +
  C_5^^5\frac{1}{100000}$

  $= 100000 + 5000 + 100 + 1 + .005 + .00001 = 15101.00501$.
\item $(x + \sqrt{x - 1})^6 + (x - \sqrt{x - 1})^6 = C_0^^6x^6 + C_1^^6x^5\sqrt{x - 1} + C_2^^6x^4\sqrt{(x -
    1)^2} + C_3^^6x^3\sqrt{(x - 1)^3} + C_4^^6x^2\sqrt{(x - 1)^4} + C_5^^6x\sqrt{(x - 1)^5} + C_6^^6\sqrt{(x
    - 1)^6} + C_0^^6x^6 - C_1^^6x^5\sqrt{x - 1} + C_2^^6x^4\sqrt{(x - 1)^2} - C_3^^6x^3\sqrt{(x - 1)^3} +
  C_4^^6x^2\sqrt{(x - 1)^4} - C_5^^6x\sqrt{(x - 1)^5} + C_6^^6\sqrt{(x - 1)^6}$

  $= 2x^6 + 30x^4(x - 1) + 30x^2(x - 1)^2 + 2(x - 1)^3$.
\item Consider the expansion of $(x + a)^n$ and $(x - a)^n$. The sum of real terms will be $A$ and the sum
  of imaginary terms will be $B$.

  $(x + a)^n = C_0^^nx^n + C_1^^nx^{n - 1}.a + C_2^^nx^{n - 2}a^2 + \cdots + C_n^^na^n = A + B$, and
  $(x - a)^n = C_0^^nx^n - C_1^^nx^{n - 1}.a + C_2^^nx^{n - 2}a^2 + \cdots + C_n^^n(-a)^n = A - B$

  Multiplying, we get

  $(x^2 - a^2)^n = A^2 - B^2$.
\item Let $(7 + 4\sqrt{3})^n = \alpha + \beta$, where $\alpha$ is a positive integer and $\beta$ is a proper
  fraction.

  Cleaerly, $0 < 7 - 4\sqrt{3} < 1\left[\because 7 - 4\sqrt{3} = \frac{49 - 48}{7 + 4\sqrt{3}} = \frac{1}{7
      + 4\sqrt{3}}\right]$

  $\therefore 0 < (7 - 4\sqrt{3})^n < 1 = \beta_1$(let), then $0 < \beta_1 < 1$.

  $\alpha + \beta + \beta_1 = 2[7^n + C_2^^n7^{n - 2}.48 + \cdots] = $ an even number.

  $\Rightarrow \beta + \beta_1 =$ an even number $- \alpha = $ an integer.

  $\because 0 < \beta < 1$ and $0 < \beta_1 < 1\therefore 0 < \beta + \beta_1 < 2$. Thus, $\beta + \beta_1 =
  1$.

  $\therefore \alpha + 1 =$ an even number $\Rightarrow \alpha = $ an odd number.
\item Proceeding from previous problem, $(\alpha + \beta)(1 - \beta) = (\alpha + \beta)\beta_1 = (7 +
  4\sqrt{3})^n(7 - 4\sqrt{3})^n = 1$.
\item $t_{r} = C_{r}^^{10}y^{10 - r}.\left(\frac{c^3}{y^2}\right)^r$. We have to find coefficient of
  $\frac{1}{y^2}$, hence, $10 - r - 2r = -2 \Rightarrow r = 4$.

  Thus, coefficient is $C_4^^{10}.c^{12}$.
\item We have to find coefficient of $x^9$ in $(1 + 3x + 3x^2 + x^3)^{15} = (1 + x)^{45}$. Therefore,
  coefficient is $C_9^^{45}$.
\item We have to find term independent of $x$ in $\left(\frac{3}{2}x^2 - \frac{1}{3x}\right)^9$. The general term
  will be $t_r = C_{r - 1}^^9.\left(\frac{3}{2}x^2\right)^{9 - r + 1}\left(-\frac{1}{3x}\right)^{r - 1}$.

  $\Rightarrow 21 - 3r = 0 \Rightarrow r = 7$. So coefficient is $(-1)^6.C_6^^9\left(\frac{3}{2}\right)^{10
    - 7}.\frac{1}{3^6} = \frac{7}{18}$.
\item $(1 + x)^m\left(1 + \frac{1}{x}\right)^n = x^{-n}(1 + x)^{m + n}$. We have to find term independent of
  $x$ in the expansion, which is coefficient of $x^n$ in $(1 + x)^{m + n}$.

  Coeff. is $C_n^^{m + n} = \frac{(m + n)!}{m!n!}$.
\item Coeff. of $x^{-1}$ in $(1 + 3x^2 + x^4)\left(1 + \frac{1}{x}\right)^8 =$ coeff. of $x^{-1}$ in
  $\left(1 + \frac{1}{x}\right)^8 +$ coeff. of $x^{-1}$ in $3x^2\left(1 + \frac{1}{x}\right)^8 +$ coeff. of
  $x^{-1}$ in $x^4\left(1 + \frac{1}{x}\right)^8$

  = coeff. of $x^{-1}$ in $\left(1 + \frac{1}{x}\right)^8$ + coeff. of $x^{-3}$ in $3\left(1 +
    \frac{1}{x}\right)^8 +$ coeff. of $x^{-5}$ in $\left(1 + \frac{1}{x}\right)^8$

  General term is given by $t_r = C_{r - 1}^^n\left(\frac{1}{x}\right)^{r - 1} = C_{r - 1}^^nx^{1 - r}$.

  When $r - 1 = 1\Rightarrow r = 2$, we have coeff. as $C_1^^8$. When $r - 1 = 3 \Rightarrow r = 4$, we have
  coeff. as $C_3^^8$, and similarly coeff. of $x^{-5}$ is $C_5^^8$.

  Thus, required coeff. of $x^{-1}$ is $C_1^^8 + 3.C_3^^8 + C_5^^8 = 232$.
\item $r$th term in the expansion of $(1 - x)^n$ is $C_{r - 1}^^{2n - 1}(-1)^{r - 1}x^{r - 1}$, so $(r + 1)$th
  term will have the term $x^r$.

  $\Rightarrow a_{r - 1} = (-1)^{r - 1}C_{r - 1}^^{2n - 1}$ and $a_{2n - r} = (-1)^{2n - r}C_{2n - r}^^{2n -
    1}$.

  We know that $C_r^^n = C_{n - r}^^n$ and $(-1)^{2n} = 1$. Hence, $a_{2n - r} = (-1)^{-r}C_{r - }^^{2n -
    1}$.

  Thus, $a_{r - 1} + a_{2n - r} = 0$.
\item Let the $r$th term be independent of $x$. $t_r = C_{r - 1}^^{10}(\sqrt{x})^{10 - r +
    1}\left(\frac{k}{x^2}\right)^{r - 1} =  C_{r - 1}^^{10}x^{\frac{11 - r}{2} - 2r + 2}k^{r - 1}$.

  Since the term is independent of $x\Rightarrow 15 - 5r = 0 \Rightarrow r = 3$.

  So the term is $C_2^^{10}k^2 = 405 \Rightarrow k = \pm 3$.
\item $k$th term in the expansion is given by $t_k = C_{k - 1}^^{n - 3}x^{n - 3 - k + 1}(x^{-2})^{k - 1}$

  $= C_{k - 1}^^{n - 3}x^{n - 3k}$. Let this term contain $x^{2r} \Rightarrow 2r = n - 3k \Rightarrow k =
  \frac{n - 2r}{3}$.

  Since $n - 2r$ is not a multiple of $3$, $k$ cannot be an integer. So no term will contain $x^{2r}$.
\item Let $r$th term be independent of $x$. $t_r = C_{r - 1}^^n(x^a)^{n - r + 1}(x^{-b})^{r - 1}$.

  This will be independent of $x$ if $an - ar + a - br + b = 0 \Rightarrow an = (a + b)(r - 1) \Rightarrow r
  = 1 + \frac{an}{a + b}$.

  Clearly, $r$ will be an integer only if $an$ is a multiple of $a + b$.
\item $\left(x + \frac{1}{x}\right)^7 = C_0^^7x^7 + C_1^^7x^6.\frac{1}{x} + C_2^^7x^5.\frac{1}{x^2} +
  C_3^^7x^4.\frac{1}{x^3} + C_4^^7x^3.\frac{1}{x^4} + C_5^^7x^2.\frac{1}{x^5} + C_6^^7x.\frac{1}{x^6} +
  C_7^^7.\frac{1}{x^7}$

  $= C_0^^7x^7 + C_1^^7x^5 + C_2^^7x^3 + C_3^^7x + C_4^^7.\frac{1}{x} + C_5^^7.\frac{1}{x^3} +
  C_6^^7.\frac{1}{x^5} + C_7^^7.\frac{1}{x^7}$

  $= x^7 + 7x^5 + 21x^3 + 35x + \frac{35}{x} + \frac{21}{x^3} + \frac{7}{x^5} + \frac{1}{x^7}$.
\item $\left(\frac{2x}{3} - \frac{3}{2x}\right)^6 = C_0^^6\left(\frac{2x}{3}\right)^6 +
  C_1^^6\left(\frac{2x}{3}\right)^5.\left(-\frac{3}{2x}\right) +
  C_2^^6\left(\frac{2x}{3}\right)^4\left(-\frac{3}{2x}\right)^2 +
  C_3^^6\left(\frac{2x}{3}\right)^3\left(-\frac{3}{2x}\right)^3 +
  C_4^^6\left(\frac{2x}{3}\right)^2\left(-\frac{3}{2x}\right)^4 +
  C_5^^6\left(\frac{2x}{3}\right)\left(-\frac{3}{2x}\right)^5 +
  C_6^^6\left(-\frac{3}{2x}\right)^6$

  $= \frac{64}{729}x^6 - \frac{32}{2x}x^4 + \frac{20}{3}x^2 - 20 + \frac{135}{4}x^2 - \frac{243}{8}x^4 +
  \frac{729}{64}x^6$.
\item Given, $(1 + ax)^n = 1 + 8x + 24x^2 + \cdots \Rightarrow 1 + nax + \frac{n(n - 1)}{2}a^2x^2 + \cdots =
  1 + 8x + 24x^2 + \cdots$

  Comparing coefficients of powers of $x$

  $an = 8, \frac{n(n - 1)}{2}a^2 = 24 \Rightarrow 32n^2 - 32n = 24n^2 \Rightarrow n = 4 \Rightarrow a = 2$.
\item $7$th term in the expansion of $\left(\frac{4x}{5} - \frac{5}{2x}\right)^9$ is
  $C_6^^9\left(\frac{4x}{5}\right)^3\left(-\frac{5}{2x}\right)^6 = \frac{10500}{x^3}$.
\item $(\sqrt{2} + 1)^6 + (\sqrt{2} - 1)^6 = C_0^^62^3 + C_1^^64\sqrt{2} + C_2^^62^2 + C_3^^62\sqrt{2} +
  C_4^^62 + C_5^^6\sqrt{2} + C_6^^6 + C_0^^62^3 - C_1^^64\sqrt{2} + C_2^^62^2 - C_3^^62\sqrt{2} +
  C_4^^62 - C_5^^6\sqrt{2} + C_6^^6$

  $=2.C_0^^62^3 + 2.C_2^^62^2 + 2.C_4^^62 + 2.C_6^^6 = 198$.
\item According to questions $(x + a)^n = A + B$, because $A$ is the sum of odd terms and $B$ is the sum of
  even terms. From Binomial theorem $(x - a)^n = A - B$.

  $\Rightarrow (x + a)^{2n} = A^2 + B^2 + 2AB$ and $(x - a)^{2n} = A^2 + B^2 - 2AB$.

  Subtracting, we get $(x + a)^{2n} - (x - a)^{2n} = 4AB$.
\item Let $(5 + 2\sqrt{6})^n = \alpha + \beta$, where $\alpha$ is a positive integer and $\beta$ is a proper
  fraction.

  Cleaerly, $0 < 5 - 2\sqrt{6} < 1\left[\because 5 - 2\sqrt{6} = \frac{25 - 24}{5 + 2\sqrt{6}} = \frac{1}{5
      + 2\sqrt{6}}\right]$

  $\therefore 0 < (5 - 2\sqrt{6})^n < 1 = \beta_1$(let), then $0 < \beta_1 < 1$.

  $\alpha + \beta + \beta_1 = 2[5^n + C_2^^n5^{n - 2}.24 + \cdots] = $ an even number.

  $\Rightarrow \beta + \beta_1 =$ an even number $- \alpha = $ an integer.

  $\because 0 < \beta < 1$ and $0 < \beta_1 < 1\therefore 0 < \beta + \beta_1 < 2$. Thus, $\beta + \beta_1 =
  1$.

  $\therefore \alpha + 1 =$ an even number $\Rightarrow \alpha = $ an odd number.
\item Proceeding from previous problem, $(\alpha + \beta)(1 - \beta) = (\alpha + \beta)\beta_1 = (3 +
  \sqrt{8})^n(3 - \sqrt{8})^n = 1$.
\item Let $r$th term contains the term $x$. Then $t_r = C_{r - 1}^^9(2x)^{10 -
    r}\left(-\frac{3}{x}\right)^{r - 1}$

  Since the term contains $x$, therefore $10 - r - r + 1 = 1 \Rightarrow r = 5$.

  Thus, coefficient is $C_4^^92^5.3^4 = 2592.C_4^^9$.
\item Let $r$th term contain $x^7$ in the expansion of $(3x^2 + (5x)^{-1})^{11}$. $t_r = C_{r -
    1}^^{11}(3x^2)^{12 - r}.(5x^{-1})^{r - 1}$.

  Since the term contains $x^7$, therefore $24 - 2r - r + 1 = 7 \Rightarrow r = 6$.

  Thus, coefficient is $C_5^^{11}.\frac{3^6}{5^5}$.
\item Let $r$th term contain $x^9$ in the expansion of $(2x^2 - x^{-1})^{20}$. Then $t_r = C_{r -
    1}^^{20}(2x^2)^{21 - r}(-x^{-1})^{r - 1}$.

  Since the term contains $x^9$, therefore $42 - 2r - r + 1 = 9$, which does not yield an integral value for
  $r$. Therefore, coefficient is $0$.
\item Let $r$th term contain $x^{24}$ in the expansion of $(x^2 + 3ax^{-1})^{15}$. Then $t_r = C_{r -
    1}^^{15}(x^2)^{16 - r}(3ax^{-1})^{r - 1}$.

  Since the term contains $x^{24}$, therefore $32 - 2r - r + 1 = 24 \Rightarrow r = 3$.

  Therefore, the coefficient is $C_2^^{16}9a^2$.
\item Let $r$th term contain $x^9$ in the expansion of $(x^2 - (3x)^{-1})^9$. Then $t_r = C_{r -
    1}^^9(x^2)^{10 - r}\left(-\frac{1}{3x}\right)^{r - 1}$.

  Since the term contains $x^9$, therefore $20 - 2r - r + 1 = 9 \Rightarrow r = 4$.

  Therefore, the coefficient is $C_3^^9.\frac{-1}{3^3} = -\frac{28}{9}$.
\item Let $r$th term contain $x^{-7}$ in the expansion of $\left(2x - \frac{1}{3x^2}\right)^{11}$.
  Then $t_r = C_{r - 1}^^{11}(2x)^{12 - r}\left(-\frac{1}{3x^2}\right)^{r - 1}$.

  Since the term contains $x^{-7}$, therefore, $12 - r - 2r + 2 = -7 \Rightarrow r = 7$.

  Therefore, coefficient is $C_6^^{11}\frac{2^5}{3^6}$.
\item Let $r$th term contain $x^7$ in the expansion of $\left(ax^2 + \frac{1}{bx}\right)^{11}$. $t_r = C_{r
    - 1}^^{11}(ax^2)^{12 - r}.\left(\frac{1}{bx}\right)^{r - 1}$.

  Since the term contains $x^7$, therefore $24 - 2r - r + 1 = 7 \Rightarrow r = 6$.

  Therefore, coefficient is $C_5^^{11}a^6b^{-5}$.
  Let $s$th term contain $x^{-7}$ in the expansion of $\left(ax - \frac{1}{bx^2}\right)^{11}$. Then $t_r =
  C_{r - 1}^^{11}(ax)^{12 - r}\left(-\frac{1}{bx^2}\right)^{r - 1}$.

  Since the term contains $x^{-7}$, therefore $12 - r - 2r + 2 = -7 \Rightarrow r = 7$.

  Therefore, the coefficient is $C_6^^{11}a^5b^{-6}$.

  Since the coefficients are equal $ab = 1[\because C_5^^{11} = C_6^^{11}]$.
\item Let $r$th term contain $x^p$ in the expansion of $\left(x^2 + \frac{1}{x}\right)^{2n}$. Then $t_r =
  C_{r - 1}^^{2n}(x^2)^{2n + 1 - r}\frac{1}{x^{r - 1}}$.

  Since the term contains $x^p$, therefore $4n + 2 - 2r - r + 1 = p \Rightarrow r = \frac{4n - p}{3} + 1$.

  Therefore, the coefficient is $\frac{2n!}{\left(\frac{4n - p}{3}\right)!\left(\frac{2n + p}{3}\right)!}$.
\item The problems are solved below:
  \startitemize[i]
  \item Let $r$th term be indnependent of $x$ in the expansion of $\left(x + \frac{1}{x}\right)^{2n}$. Then
  $t_r = C_{r - 1}^^{2n}x^{2n + 1 - r}.\frac{1}{x^{r - 1}}$

  Since the term is independent of $x$, therefore $2n + 1 - r - r + 1 = 0 \Rightarrow r = n + 1$.

  Therefore, the coefficient is $C_n^^{2n} = \frac{(2n)!}{n!n!}$.
\item Let $r$th term be independent of $x$ in the expansion of $\left(2x^2 + \frac{1}{x}\right)^{15}$. Then
  $t_r = C_{r - 1}^^{15}(2x^2)^{16 - r}\left(\frac{1}{x}\right)^{r - 1}$.

  Since the term is independent of $x$, therefore $32 - 2r -r + 1 = 0 \Rightarrow r = 11$.

  Therefore, the coefficient is $C_{10}^^{15}.2^5 = 32.C_{10}^^{15}$.
\item Let $r$th term be independent of $x$ in the expansion of $\left(\sqrt{\frac{x}{3}} +
  \frac{3}{2x^2}\right)^{10}$. Then $t_r = C_{r - 1}^^{10}\left(\sqrt{\frac{x}{3}}\right)^{11 -
    r}\left(\frac{3}{2x^2}\right)^{r - 1}$.

  Since the term is independent of $x$, therefore $\frac{11 - r}{2} - 2r + 2 = 0 \Rightarrow r = 3$.

  Therefore, the coefficient is $C_2^^{10}.\left(\frac{1}{3^4}\right).\left(\frac{3^2}{2^2}\right) =
  \frac{5}{4}$.
\item Let $r$th term be independent of $x$ in the expansion of $\left(2x^2 - \frac{1}{x}\right)^{12}$. Then
  $t_r = C_{r - 1}^^{12}(2x^2)^{13 - r}\left(\frac{1}{x}\right)^{r - 1}$.

  Since the term is independent of $x$, therefore $26 - 2r - r + 1 = 0 \Rightarrow r = 9$.

  Therefore, the coefficient is $C_8^^{12}.2^4 = 7920$.
\item Let $r$th term be independent of $x$ in the expansion of $\left(2x^^2 - \frac{3}{x^3}\right)^{25}$. Then
  $t_r = C_{r - 1}^^{25}(2x^2)^{26 - r}\left(-\frac{3}{x^^3}\right)^{r - 1}$.

  Since the term is independent of $x$, therefore $52 - 2r - 3r + 3 = 0 \Rightarrow r = 11$.

  Therefore, the coefficient is $C_{10}^^{25}.2^{15}3^{10}$.
\item Let $r$th term be independent of $x$ in the expansion of $\left(x^3 - \frac{3}{x^2}\right)^{15}$. Then
  $t_r = C_{r - 1}^^{15}(x^3)^{16 - r}\left(-\frac{3}{x^2}\right)^{r - 1}$.

  Since the term is independent of $x$, therefore $48 - 3r - 2r + 2 = 0 \Rightarrow r = 10$.

  Therefore, the coefficient is $C_9^^{15}\left(-3\right)^9 = -3^9.C_{0}^^{15}$.
\item Let $r$th term be independent of $x$ in the expansion of $\left(x^2 - \frac{3}{x^3}\right)^{10}$. Then
  $t_r = C_{r - 1}^^{10}(x^2)^{11 - r}\left(-\frac{3}{x^3}\right)^{r - 1}$.

  Since the term is independent of $x$, therefore $22 - r - 3r + 3 = 0 \Rightarrow r = 5$.

  Therefore, the coefficient is $C_4^^{10}.(-3)^4 = 3^4.C_{4}^^{10}$.
\item Let $r$th term be independent of $x$ in the expansion of $\left(\frac{1}{2}x^{1/3} +
  x^{-1/3}\right)^8$. Then $t_r = C_{r - 1}^^{8}\left(\frac{1}{2}x^{1/3}\right)^{9 - r}(x^{-1/3})^{r - 1}$.

  Since the term is independent of $x$, therefore $9 - r - r + 1= 0 \Rightarrow r = 5$.

  Therefore, the coefficient is $C_4^^8.\frac{1}{2^4} = \frac{35}{8}$.
\stopitemize
\item Let $r$th term be independent of $x$ in the expansion of $\left(x + \frac{1}{x^2}\right)^n$. Then $t_r
  = C_{r - 1}^^nx^{n + 1 - r}\frac{1}{(x^2)^{r - 1}}$.

  Since the term is independent of $x$, therefore $n + 1 - r - 2r + 2 = 0 \Rightarrow r = \frac{n}{3} + 1$.

  Therefore, the coefficient is $C_{\frac{n}{3}}^^{n} =
  \frac{n!}{\left(\frac{n}{3}\right)!\left(\frac{2n}{3}\right)!}$.
\item First we will find coefficient of $x^m$ and then of $x^n$ in the expansion of $(1 + x)^{m + n}$. Let
  $p$th term contain $x^m$.

  Then $t_P = C_{p - 1}^^{m + n}x^{p - 1}$. Since it contains $x^m$, therefore $p = m + 1$.

  Thus coefficient is $C_m^^{m + n}$. Similarly, we find the coefficient of term containing $x^n$ as
  $C_n^^{m + n}$. We know that $C_r^^n = C_{n - r}^^n$.

  Therefore, $C_m^^{m + n} = C_n^^{m + n}$. Hence, proved.
\item $4$th term in the expansion of $\left(px + \frac{1}{x}\right)^n$ is given by $t_4 = C_3^^n(px)^{n + 1
  - 4}\frac{1}{x^{3}}$.

  Since the term is independent of $x$, therefore $n - 3 - 3 = 0 \Rightarrow n = 6$.

  So the term is $C_3^^6p^3 = \frac{5}{2}\Rightarrow p = \frac{1}{2}$.
\item There are $13$ terms in the expansion of $\left(x - \frac{1}{2x}\right)^{12}$. So $7$th term will be
  the middle term.

  $t_7 = C_6^^{12}.x^6\left(-\frac{1}{2x}\right)^6 = C_6^^{12}.\frac{1}{2^6} = \frac{231}{16}$.
\item There are $8$ terms in the expansion of $\left(2x^2 - \frac{1}{x}\right)^7$. These are $4$th and $5$th
  terms.

  $t_4 = C_3^^7(2x^2)^4\left(-\frac{1}{x}\right)^3 = -560x^5,\;t_5 =
  C_4^^7(2x^2)^3\left(-\frac{1}{x}\right)^4 = 280x^2$.
\item There are $2n + 1$ terms in the expansion of $\left(x + \frac{1}{x}\right)^{2n}$. So the middle term
  is $(n + 1)$th term.

  $t_{n + 1} = C_n^^{2n}x^{2n - n}.\frac{1}{x^n} = C_n^^{2n} = \frac{2n!}{n!n!} = \frac{1.3.5\ldots (2n -
    1).2^n}{n!}$.
\item There are $2n + 1$ terms in the expansion of $(1 + x)^{2n}$. So the middle term is $(n + 1)$th term.

  $t_{n + 1} = C_n^^{2n}x^n$. So the ccoefficient is $C_n^^{2n}$.

  There are $2n$ temrs in the expansion of $(1 + x)^{2n - 1}$. So the middle terms are $n$th and $(n + 1)$th
  terms.

  Coefficients are $C_{n - 1}^^{2n - 1}$ and $C_n^^{2n - 1}$.

  Clearly, $C_{n - 1}^^{2n - 1} + C_n^^{2n - 1} = C_n^^{2n}$. Hence, proved.
\item The solutions are given below:
  \startitemize[i]
  \item There will be $21$ terms in the expansions of $\left(\frac{2x}{3} - \frac{3y}{2}\right)^{20}$. So
    $11$th term will be the middle term.

    $t_{11} = C_{10}^^{20}\left(\frac{2x}{3}\right)^{21 - 10}.\left(-\frac{3y}{2}\right)^{10} =
    C_{10}^^{20}x^{10}y^{10}$.
  \item There will be $7$ terms in the expansions of $\left(\frac{2x}{3} - \frac{3}{2x}\right)^6$. So $4$th
    term will be the middle term.

    $t_4 = C_3^^6\left(\frac{2x}{3}\right)^3\left(-\frac{3}{2x}\right)^3 = -20$.
  \item There will be $8$ terms in the expansion of $\left(\frac{x}{y} - \frac{y}{x}\right)^7$. So $4$th and
    $5$th termss will be the middle terms.

    $t_4 = C_3^^7.\left(\frac{x}{y}\right)^4\left(-\frac{y}{x}\right)^3 = -\frac{35x}{y},\;t_5 =
    C_4^^7\left(\frac{x}{y}\right)^3\left(-\frac{y}{x}\right)^3 = \frac{35y}{x}$.
  \item The middle term of the expansion $(1 + x)^{2n}$ will be the $(n + 1)$th term.

    $t_{n + 1} = C_n^^{2n}x^n = \frac{2n!}{n!n!}x^n$.
  \item $(1 - 2x + x^2)^n = (1 - x)^{2n}$ so $(n + 1)$th term will be the middle term.

    $t_{n + 1} = C_n^^{2n}(-x^n) = (-1)^n\frac{2n!}{n!n!}x^n$
  \stopitemize
\item The general $r$th term will be given by $t_r = C_{r - 1}^^{2n + 1}\left(\frac{x}{y}\right)^{2n + 1 + 1
  - r}.\left(\frac{y}{x}\right)^{r - 1}$.

  Since it will $2n + 2$ terms there will be two middle terms. $(n + 1)$th and $(n + 2)$th terms will be
  middle terms. Since the powers of $x$ and $y$ are symmetric if any term has to be free of $x$ and $y$ then
  it has to be middle terms.

  $t_{n + 1} = C_n^^{2n + 1}\left(\frac{x}{y}\right)^{2n + 1 + 1 - n - 1}\left(\frac{y}{x}\right)^n =
  C_n^^{2n + 1}\frac{x}{y}$

  $t_{n + 2} = C_{n + 1}^^{2n + 1}\frac{y}{x}$. Both of these terms are not free of $x$ and $y$.

  We also prove that no term is free of $x$ and $y$ by considering general term. Since the term has to be
  independent of $x$ and $y$, we consider the general term.

  $2n + 2 - r - - r + 1 = = \Rightarrow r = \frac{2n + 1}{2}$, which cannot be an integer. So no terms is
  free of both $x$ and $y$.
\item There will be $2n + 1$ terms in the expansion of $\left(x - \frac{1}{x}\right)^{2n}$. So the middle
  term would be $(n + 1)$th term.

  $t_{n + 1} = C_n^^{2n}x^{2n + 1 - n - 1}\frac{-1^n}{x^{n}} = (-1)^n\frac{2n!}{n!n!} = \frac{1.3.5\ldots
    (2n -1)}{n!}.(-2)^n$.
\item $t_{2r + 1} = C_{2r}^^{43}x^{2r}$ and $t_{r + 2} = C_{r + 1}^^{43}x^{r + 1}$.

  Given that coefficients are equal. $\therefore C_{2r}^^{43} = C_{r + 1}^^43 \Rightarrow 2r + r + 1 = 43
  \Rightarrow r = 14$.
\item Coefficient of $r$th term in the expansion of $(1 + x)^{20}$ is $C_{r - 1}^^20$, and the coefficient
  of $(r + 4)$th term is $C_{r + 3}^^{20}$.

  Clearly, for coefficients to be equal $r - 1 + r + 3 = 20 \Rightarrow r = 9$.
\item Following like previous problem, $r - 3 + 2r + 3 = 18 \Rightarrow r = 6$.
\item Following like previous problem, $2r + 4 + r - 7 = 39 \Rightarrow r = 14$. So $C_{12}^^r = 91$.
\item Following like previous problem, $3r - 1 + r + 1 = 2n \Rightarrow r = \frac{n}{2}$.
\item Following like previous problem, $p + p + 2 = 2n \Rightarrow p = n - 1$.
\item Coefficient of $(r + 1)$th term in the expansion of $(1 + x)^{n + 1}$ is $C_r^^{n + 1}$. Coefficients
  of $r$th and $(r + 1)$th terms in the expansion of $(1 + x)^n$ are $C_{r - 1}^^n$ and $C_r^^n$
  respectively.

  Clearly, $C_{r - 1}^^n + C_r^^{n} = C_r^^{n + 1}$. Hence, proved.
\item Since we have to find numerically greatets term we can replaced $-$ sign with $+$. Let $r$th term be
  the greatest term in the expansion of $\left(7 + \frac{10}{3}\right)^{11}$. $t_r =
  C_{r - 1}^^{11}7^{12 - r}\left(\frac{10}{3}\right)^{r - 1}$. We consider $(r + 1)$th term as well. $t_{r
    + 1} = C_r^^{11}7^{11 - r}\left(\frac{10}{3}\right)^r$

  $\frac{t_r}{t_{r + 1}} = \frac{21r}{(12 - r)10}\geq 1 \Rightarrow r\geq 3\frac{27}{31}$.

  Replacing $r$ with $r - 1$, $\frac{t_{r - 1}}{t_r} = \frac{21r - 21}{130 - 10r}\Rightarrow r\leq
  4\frac{27}{31}\therefore r = 4$.

  So the greatest term will be $C_3^^{11}7^8\frac{10^3}{3^3} = \frac{440}{9}7^85^3$.
\item In any binomial expansion, the middle terms have the greatest coefficient. Therefore, $(n + 1)$th term
  will have greatest coefficient.

  $t_n = C_{n - 1}^^{2n}x^{n - 1},\;t_{n + 1} = C_n^^{2n}x^n,\;t_{n + 2} = C_{n + 1}^^{2n}x^{n + 1}$

  $\therefore \frac{t_{n + 1}}{t_{n + 2}} = \frac{n + 1}{n}.\frac{1}{x}$. Since $t_{n + 1}$ is the greatest
  term $\frac{t_{n + 1}}{t_{n + 2}} > 1 \Rightarrow x < \frac{n + 1}{n}$.

  Similarly, considering $t_n$ and $t_{n + 1},\;x > \frac{n}{n + 1}$.
\item The greatest terms are calculated below:
  \startitemize[i]
  \item $\left(2 + \frac{9}{5}\right)^{10}$ will have $6$th term as the middle term, which will be greatest.

    $t_6 = C_5^^{10}.2^5.\left(\frac{9}{5}\right)^5 = C_5^^{10}\left(\frac{18}{5}\right)^5$.
  \item For $(4 - 2)^7$ let $t_r$ is the greatest term. Then $\frac{t_r}{t_{r + 1}} > 1$ and
    $\frac{t_r}{t_{r - 1}} > 1$. Substituting and evaluating, we find $r = 3$.

    $t_3 = C_2^^7.4^5.2^2 = 86016$.
  \item For $(5 + 2)^{10}$ let $t_r$ is the greatest term. Then $\frac{t_r}{t_{r + 1}} > 1$ and
    $\frac{t_r}{t_{r - 1}} > 1$. Substituting and evaluating, we find $r = 4$.

    $t_4 = C_3^^{13}5^{10}2^3$.
  \stopitemize
\item In any binomial expansion, the middle terms have the greatest coefficient. Therefore, $(15 + 1)$th term
  will have greatest coefficient.

  $t_{15} = C_{14}^^{30}x^{14},\;t_{16} = C_{15}^^{30}x^{15},\;t_{17} = C_{16}^^{30}x^{16}$

  $\therefore \frac{t_{16}}{t_{17}} = \frac{16}{15}.\frac{1}{x}$. Since $t_{16}$ is the greatest
  term $\frac{t_{16}}{t_{17}} > 1 \Rightarrow x < \frac{16}{15}$.

  Similarly, considering $t_{15}$ and $t_{16},\;x > \frac{15}{16}$.
\item Given, $6^{2n} - 35n - 1 = 36^n - 35n - 1 = (1 + 35)^n - 35n - 1 = 35^2[C_2^^n + 35.C_3^^n + \cdots +
  35^{n - 2}]$

  $= 1225[C_2^^n + 35.C_3^^n + \cdots + 35^{n - 2}] = 1225\times$ a positive integer if $n\geq 2$.

  If $n = 1$, given expression becomes $0$. Hence, for all positive integral values of $n$, $6^{2n} - 35n -
  1$ is divisible by $1225$.
\item $2^{4n} - 2^n(7n + 1) = 16^n - 2^n(7n + 1) = (2 + 14)^n - 2^n(7n + 1) = 14^2[C_2^^n.2^{n - 2} +
  C_3^^n.2^{n - 3}.14 + \cdots + 14^{n - 2}]$, which is divisible by $196$ for all positive values of
  $n$. If $n = 1$, given expression becomes $0$, which is also divisible by $196$.
\item $3^{4n + 1} + 16n - 3 = 3(3^{4n} - 1) + 16n = 3[81^n - 1] + 16n = 3[(1 + 80)^n - 1] + 16n$

  $= 3[80n + C_2^^n80^2 + C_3^^n80^3 + \cdots + 80^n] + 16n = 256[n + 75(C_2^^n + C_3^^n.80 + \cdots + 80^{n
    - 2})]$,

  which is divisible by $256$ for all $n\in\mathbb{N}$.
\item The problems are solved below:
  \startitemize[i]
  \item $4^n - 3n - 1 = (1 + 3)^n - 3n - 1 = C_2^^n3^2 + C_3^^n3^3 + \cdots + 3^n$

    $= 9[C_2^^n + C_3^^n.3 + \cdots + 3^{n - 2}]$,

    which is divisible by $9$ for $n\geq 2$. When $n = 1$, the given expression becomes $0$, and hence
    divisible by $9$. Thus, given expression is divisible by $9$ for all $n\in\mathbb{P}$.
  \item $2^{5n} - 31n - 1 = (1 + 31)^n - 31n - 1 = C_2^^n31^2 + C_3^^n31^3 + \cdots + 31^n$

    $= 961[C_2^^n + C_3^^n.31 + \cdots + 31^{n - 2}]$,

    which is divisible by $961$ for $n\geq 2$. When $n = 1$, the given expression becomes $0$, and hence
    divisible by $961$. Thus, given expression is divisible by $961$ for all $n\in\mathbb{P}$.
  \item $3^{2n + 2} - 8n - 9 = 9(1 + 8)^n - 8n - 9 = 9[1 + 8n + C_2^^n.8^2 + C_3^^n.8^3 + \cdots + 8^n] - 8n
    - 9 = 64[n + 9(C_2^^n + C_3^^n8 + \cdots + 8^{n - 2})]$,

    which is divisible by $64$ for $n\geq 2$.
  \item $2^{5n + 5} - 31n - 32 = 32(1 + 31)^n - 31n - 32 = 32[1 + 31n + C_2^^n.31^2 + C_3^^n.31^3 + \cdots +
    31^n] - 31n - 32 = 961[n + 32(C_2^^n + C_3^^n31 + \cdots + 31^{n - 2})]$,

    which is divisible by $961$ for $n > 1$.
  \item $3^{2n} - 1 + 24n - 32n^2 = (1 + 8)^n - 1 + 24n - 32n^2 = 1 + 8n + 32n^2 - 32n + 8^3(C_3^^n +
    C_4^^n.8 + \cdots + 8^{n - 3}) - 1 + 24n - 32n^2 = 8^3(C_3^^n +
    C_4^^n.8 + \cdots + 8^{n - 3})$,

    which is divisible by $512$ for $n > 2$.
  \stopitemize
\item Let the three consecutive coefficients in the expansion of $(1 + x)^n$ be the $r$th, $(r + 1)$th and
  $(r + 2)$th, which are given to be $165, 330$ and $462$ respectively.

  $\therefore C_{r - 1}^^n = 165 \Rightarrow \frac{n!}{(r - 1)!(n - r + 1)!} = 165$

  $C_r^^n = 330 \Rightarrow \frac{n!}{r!(n - r)!} = 330$, and $C_{r + 1}^^n = \frac{n!}{(r + 1)!(n - r -
  1)!} = 462$.

  From first two, we have $\frac{r}{n - r + 1}= \frac{1}{2} \Rightarrow 3r = n + 1$.

  From last two, we have $\frac{r + 1}{n - r} = \frac{5}{7}\Rightarrow 12r = 5n - 7$

  Thus, $n = 11, r = 4$. So positions of coefficients are the $4$th, $5$th and $6$th respectively.
\item Let $a_1, a_2, a_3$ and $a_4$ be the coefficients of the $r$th, $(r + 1)$th, $(r + 2)$th and $(r +
  3)$th terms respectively in the expansion of $(1 + x)^n$.

  $\therefore a_1 = C_{r - 1}^^n, a_2 = C_r^^n, a_3 = C_{r + 1}^^n$, and $a_4 = C_{r + 2}^^n$.

  $\frac{a_2}{a_1} = \frac{n - r + 1}{r}\Rightarrow \frac{a_1 + a_2}{a_1} = \frac{n + 1}{r}\Rightarrow
  \frac{a_1}{a_1 + a_2} = \frac{r}{n + 1}$.

  Similarly, $\frac{a_2}{a_2 + a_3} = \frac{r + 1}{n + 1}$, and $\frac{a_3}{a_3 + a_4} = \frac{r + 2}{n +
    1}$.

  Clearly, $\frac{a_1}{a_1 + a_2} + \frac{a_3}{a_3 + a_4} = \frac{2a_2}{a_2 + a_3}$.
\item $2$nd terms $= C_1^^nx^{n - 1}y = 240$, $3$rd term $= C_2^^nx^{n - 2}y^2 = 720$, and $4$th term $=
  C_3^^nx^{n - 3}y^3 = 1080$.

  From first two, we have $\frac{240}{720} = \frac{2}{n - 1}.\frac{x}{y}$.

  From last two, we have $\frac{720}{1080} = \frac{3}{n - 2}.\frac{x}{y}$

  From these two equations $\frac{1}{2} = \frac{2(n - 2)}{3(n - 1)}\Rightarrow n = 5$.

  $\Rightarrow y = \frac{3x}{2}$.

  $\Rightarrow 240 = C_1^^nx^{n - 1}y \Rightarrow x^5 = 32 \Rightarrow x = 2 \Rightarrow y = 3$.
\item Let the index of the power be $n$. And let $a, b, c$ be the $r$th, $(r + 1)$th, $(r + 2)$th
  coefficients respectively in the expansion of $(1 + x)^n$.

  $a = C_{r - 1}^^n, b = C_r^^n$, and $c = C_{r + 1}^^n$.

  $\frac{a}{b} = \frac{r}{n - r + 1} \Rightarrow an + a = r(a + b)$, $\frac{b}{c} = \frac{r + 1}{n - r}
  \Rightarrow bn - br = cr + c \Rightarrow bn - c = r(b + c)$

  $\Rightarrow n = \frac{2ac + ab + bc}{b^2 - ac}$.
\item The coefficient of $14$th, $15$th and $16$th terms in the expansion of $(1 + x)^n$ will be $C_{13}^^n,
  C_{14}^^n$ and $C_{15}^^n$ respectively. Given that these are in A.P. $\Rightarrow 2C_{14}^^n = C_{13}^^n
  + C_{15}^^n$.

  $\Rightarrow 2.\frac{n!}{14!(n - 14)!} = \frac{n!}{13!(n - 13)!} + \frac{n!}{15!(n - 15)!}$

  $\Rightarrow 2.15(n - 13) = 15.14 + (n - 13)(n - 14)\Rightarrow n = 23, 34$.
\item Let the three consecutive terms are $r$th, $(r + 1)$th and $(r + 2)$th. Then, $C_{r - 1}^^n = 56,
  C_r^^n = 70, C_{r + 1} = 56$.

  From first two, we have $\frac{r}{n - r + 1} = \frac{4}{5}$, and from last two, we have $\frac{r + 1}{n -
    r} = \frac{5}{4}$.

  Solving these gives us $n = 8, r = 4$.
\item Let the three consecutive terms are $r$th, $(r + 1)$th and $(r + 2)$th. Then, $C_{r - 1}^^n = 220,
  C_r^^n = 495, C_{r + 1} = 792$.

  From first two, we have $\frac{r}{n - r + 1} = \frac{4}{9}$, and from last two, we have $\frac{r + 1}{n -
    r} = \frac{5}{8}$.

  Solving these gives us $n = 12$.
\item $t_3 = C_2^^na^{n - 2}x^2 = 84, t_4 = C_3^^na^{n - 3}x^3 = 280$, and $t_5 = C_4^^na^{n - 4}x^4 = 560$.

  From first two, we have $\frac{3}{n - 2}.\frac{a}{x} = \frac{3}{10}$, and from last two we have
  $\frac{4}{n - 3}.\frac{a}{x} = \frac{1}{2}$.

  $\Rightarrow \frac{3(n - 3)}{4(n - 2)} = \frac{3}{5}\Rightarrow n = 7, \Rightarrow x = 2, a = 1$.
\item $t_6 = C_5^^nx^{n - 5}y^5 = 112, t_6 = C_6^^nx^{n - 6}y^6 = 7$, and $t_8 = C_7^^nx^{n - 7}y^7 =
  \frac{1}{4}$.

  From first two, we have $\frac{6}{n - 5}.\frac{x}{y} = 16$, and from last two we have $\frac{7}{n -
    6}.\frac{x}{y} = 28$

  $\Rightarrow \frac{6(n - 6)}{7(n - 5)} = \frac{4}{7}\Rightarrow n = 7, x = 4, y = \frac{1}{2}$.
\item Let the binomial expansion be $(x + y)^n$. $a = C_5^^nx^{n - 5}y^5, b = C_6^^nx^{n - 6}y^6, c =
  C_7^^{n}x^{n - 7}x^7$, and $d = C_8^^nx^{n - 8}y^8$.

  From first two, we have $\frac{b}{a} = \frac{n - 5}{6}.\frac{y}{x}$, from second and third, we have
  $\frac{c}{b} = \frac{n - 6}{7}.\frac{y}{x}$, and from kast two we gace $\frac{d}{c} = \frac{n -
    7}{8}.\frac{y}{x}$.

  Now from first two we have, $\frac{b^2}{ac} = \frac{7(n - 5)}{6(n - 6)}$ and $\frac{c^2}{bd} = \frac{8(n -
    6)}{7(n - 7)}$

  Subtracting $1$ from both of these, we have $\frac{b^2 - ac}{ac} = \frac{7(n - 5) - 6(n - 6)}{6(n - 6)}$,
  and $\frac{c^2 - bd}{bd} = \frac{8(n - 6) - 7(n - 7)}{7(n - 7)}$

  Dividing, we get $\frac{b^2 - ac}{c^2 - bd} = \frac{4a}{3c}$.
\item (a) Let $a, b, c$ and $d$ be the $r$th, $(r + 1)$th, $(r + 2)$th, and $(r + 3)$th term of the binomial
  expansion $(x + y)^n$.

  $a = C_{r - 1}^^n, b = C_r^^n, c = C_{r + 1}^^n$, and $d = C_{r + 2}^^n$.

  $\frac{b}{a} = \frac{n - r + 1}{r}.\Rightarrow \frac{a + b}{a} = \frac{n + 1}{r}$, $\frac{c}{b} = \frac{n
    - r}{r + 1}\Rightarrow \frac{c + b}{b} = \frac{n + 1}{r + 1}$

  $\frac{d}{c} = \frac{n - r - 1}{r + 2}\Rightarrow \frac{c + d}{c} = \frac{n + 1}{r + 2}$. Clearly,
  $\frac{a + b}{a}, \frac{b + c}{b}, \frac{c + d}{c}$ are in H.P.

  (b) $(bc + ad)(b - c) = \frac{(n!)^3}{[(r - 1)!]^3[(n - r - 2)!]^3}\left(\frac{1}{r(n - r)(n - r
    -1)}.\frac{1}{r(r + 1)(n - r - 1)} - \frac{1}{(n - r + 1)(n - r)(n - r - 1)}.\frac{1}{r(r + 1)(r +
      2)}\right)\left(\frac{1}{r(n - r)(n - r - 1)} - \frac{1}{r(r + 1)(n - r - 1)}\right)$

  Now it is trivial to prove that $(bc + ad)(b - c) = 2(ac^2 - b^2d)$.
\item The coefficients of $5$th, $6$th and $7$th terms in the expansion of $(1 + x)^n$ are $C_4^^n, C_5^^n$
  and $C_6^^n$. Given that these are in A.P., therefore

  $2C_5^^n = C_4^^n + C_6^^n \Rightarrow \frac{2}{5(n - 5)} = \frac{1}{(n - 4)(n - 5)} + \frac{1}{5.6}
  \Rightarrow n = 7, 14$.
\item The coefficients of second, third and fourth terms in the expansion of $(1 + x)^{2n}$ are in A.P.

  $\Rightarrow 2C_2^^{2n} = C_1^^{2n} + C_3^^{2n}\Rightarrow \frac{2}{2(2n - 2)} = \frac{1}{(2n - 1)(2n -
  2)} + \frac{1}{2.3}$

  $\Rightarrow 2n^2 - 9n + 7 = 0$.
\item The coefficients of $r$th, $(r + 1)$th and $(r + 2)$th terms in the expansion of $(1 + x)^n$ are in
  A.P.

  $\Rightarrow 2C_r^^n = C_{r - 1}^^n + C_{r + 1}^^n \Rightarrow \frac{2}{r(n - r)} = \frac{1}{(n - r)(n - r
    + 1)} + \frac{1}{r(r + 1)}\Rightarrow n^2 - n(4r + 1) + 4r^2 - 2 = 0$.
\item Let the coefficients of $r$th, $(r + 1)$th and $(r + 2)$th terms in the expansion of $(1 + x)^n$ are
  in the ratio of $182:84:30$.

  $t_r = C_{r - 1}^^n, t_{r + 1} = C_r^^n$ and $t_{r + 2} = C_{r + 1}^^n$

  From first two, we have $\frac{r}{n - r + 1} = \frac{13}{6}$, and from last two we have $\frac{r + 1}{n -
    r} = \frac{14}{5}$

  From these two equations we have $n = 18$.
\item Given series is $C_1 + 2.C_2 + 3.C_3 + \cdots + n.C_n$. Its $r$th term $t_r = r.C_r^^n = n.C_{r -
  1}^^{n - 1}[\because r.C_r^^n = n.C_{r - 1}^^{n - 1}]$

  Now $C_1 + 2.C_2 + 3.C_3 + \cdots + n.C_n = \displaystyle\sum_{r = 1}^nr.C_r^^n = sum_{r = 1}^nn.C_{r -
    1}^^{n - 1}$

  $= n[C_0^^{n - 1} + C_1^^{n - 1} + C_2^^{n - 1} + \cdots + C_{n - 1}^^{n - 1}] = n(1 + 1)^{n - 1} = n.2^{n
    - 1}$.

  {\bf Calculus Method:} $(1 + x)^n = C_0 + C_1.x + C_2.x^2 + \cdots + C_n.x^n$

  Differentiating w.r.t. $x$, we get $n(1 + x)^{n - 1} = C_1 + 2C_2.x + \cdots + nC_n.x^{n - 1}$

  Putting $x = 1$, we get $n.2^{n - 1} = C_1 + 2.C_2 + 3.C_3 + \cdots + n.C_n$.
\item Given series is $C_0 + 2.C_1 + 3.C_2 + \cdots + (n + 1).C_n$.

  Its $r$th term is $t_r = r.C_{r - 1}^^n = (r - 1).C_{r - 1}^^n + C_{r - 1}^^n = n.C_{r - 2}^^{n - 1} +
  C_{r - 1}^^n[\because (r - 1).C_{r - 1}^^n = n.C_{r - 2}^^{n - 1}]$

  Now, $C_0 + 2.C_1 + 3.C_2 + \cdots + (n + 1).C_n = \displaystyle\sum_{r = 1}^{n + 1}t_r = \sum_{r = 1}^{n
    + 1}n.C_{r - 2}^^{n - 1} + \sum_{r = 1}^^{n + 1}C_{r - 1}^^n$

  $= n[C_0^^{n - 1} + C_1^^{n - 1} + \cdots + C_{n - 1}^^{n - 1}] + (C_0^^n + C_1^^n + \cdots + C_n^^n)$

  $= n.2^{n - 1} + 2^n = 2^{n - 1}(n + 2)$.

  {\bf Calculus Method:} $(1 + x)^n = C_0 + C_1.x + C_2.x^2 + \cdots + C_n.x^n$

  Multiplying with $x$, we get $x(1 + x)^n = C_0.x + C_1.x^2 + C_2.x^3 + \cdots + C_n.x^{n + 1}$

  Differentiating w.r.t. $x$, we get $(1 + x)^n + nx(1 + x)^{n - 1} = C_0 + 2C_1.x + 3C_2.x^2 + \cdots + (n
  + 1)C_n.x^n$

  Putting $x = 1$, we get $C_0 + 2.C_1 + 3.C_2 + \cdots + (n + 1).C_n = 2^{n - 1}(n + 2)$.
\item Given series is $C_0 + 3.C_1 + 5.C_2 + \cdots + (2n + 1).C_n$.

  Its $r$th term is $t_r = (2r - 1)C_{r - 1} = [2(r - 1) + 1]C_{r - 1} = 2(r - 1)C_{r - 1} + C_{r - 1}$

  $= 2.nC_{r - 2}^^{n - 1} + C_{r - 1}[\because (r - 1)C_{r - 2} = n.C_{r - 2}^^{n - 1}]$

  Now, $C_0 + 3.C_1 + 5.C_2 + \cdots + (2n + 1).C_n = \displaystyle\sum_{r = 1}^{n + 1}t_r = 2n\sum_{r = 1}^{n
    + 1}C_{r - 2}^^{n - 1} + \sum_{r = 1}^^{n + 1}C_{r - 1}$

  $= 2n(C_0^^{n - 1} + C_1^^{n - 1} + \cdots + C_{n - 1}^^{n - 1}) + (C_0 + C_1 + \cdots + C_n)$

  $= 2n.2^{n - 1} + 2^n = 2^n(n + 1)$.

  {\bf Calculus Method:} $(1 + x)^n = C_0 + C_1.x + C_2.x^2 + \cdots + C_n.x^n$

  Putting $x = x^2$ and multiplying with $x$, we get

  $x(1 + x^2)^n = C_0.x + C_1.x^3 + C_2.x^5 + \cdots + C_n.x^{2n + 1}$

  Differentiating both sides w.r.t. $x$, we get

  $(1 + x^2)^n + 2x^2.n(1 + x^2)^{n - 1} = C_0 + C_1.3x^2 + C_2.5x^4 + \cdots + C_n.(2n + 1)x^{2n + 1}$

  Putting $x = 1$, we get

  $C_0 + 3.C_1 + 5.C_2 + \cdots + (2n + 1).C_n = 2n.2^{n - 1} + 2^n = 2^n(n + 1)$.
\item We have to prove that $C_1 - 2.C_2 + 3.C_3 - 4.C_4 + \cdots + (-1)^{n - 1}n.C_n = 0$.

  $r$th term $t_r = (-1)^{r - 1}r.C_r^^n = (-1)^{r - 1}.nC_{r - 1}^^{n - 1}$

  $\displaystyle\sum_{r = 1}^nt_r = \sum_{r = 1}^n(-1)^{r - 1}n.C_{r - 1} = n.\sum_{r = 1}^n(-1)^{r - 1}C_{r
  - 1}$

  $= n(C_0^^{n - 1} - C_1^^{n - 1} + C_2^^{n - 1} - C_3^^{n - 1} + \cdots + (-1)^{n - 1}C_{n - 1}^^{n - 1})$

  $= n(1 - 1)^{n - 1} = 0$.

  {\bf Calculus Method:} $(1 + x)^n = C_0 + C_1.x + C_2.x^2 + \cdots + C_n.x^n$.

  Differentiating both sides w.r.t. $x$, we get

  $n(1 + x)^{n - 1} = C_1 + 2x.C_2 + 3x^2C_3 + \cdots + nx^{n - 1}C_n$

  Putting $x = -1$, we get

  $n(1 - 1)^{n - 1} = C_1 - 2C_2 + 3C_3 - \cdots + (-1)^{n - 1}.nC_n = 0$.
\item Given series is $C_0 + \frac{C_1}{2} + \frac{C_3}{3} + \cdots + \frac{C_n}{n + 1}$.

  Its $r$th term is $t_r = \frac{C_{r - 1}^^n}{r}$.
\stopitemize