% -*- mode: context; -*-
\chapter{Matrices}
\startitemize[n, 1*broad]
\item The matrix will be any one of the following type $1\times12, 12\times1, 2\times6, 6\times2, 3\times4,
  4\times3$. So the answer is $6$.
\item $a_{11} = 2.1 - 3.1 = -1, a_{12} = 2.1 - 3.2 = -4, a_{13} = 2.1 - 3.3 = -7$

  $a_{21} = 2.2 - 3.1 = 1, a_{22} = 2.2 - 3.2 = -2, a_{23} = 2.2 - 3.3 = -5$

  $\therefore A = \startbmatrix\NC-1\NC-4\NC-7\NR\NC1\NC -2\NC -5\NR\stopbmatrix$.
\item $A + B = \startbmatrix\NC a - a\NC b + b\NR\NC -b - b \NC a - a\NR\stopbmatrix = \startbmatrix\NC0\NC
  2b\NR\NC -2b \NC0\NR\stopbmatrix$.
\item $2X = (2X + Y) - Y = \startbmatrix\NC 1\NC 0\NR\NC -3 \NC 2\NR\stopbmatrix - \startbmatrix\NC 3\NC
  2\NR\NC 1\NC 4\NR\stopbmatrix = \startbmatrix\NC1 - 3\NC 0 - 2\NR\NC -3 -1\NC 2 - 4\NR\stopbmatrix
  = \startbmatrix\NC -2\NC -2\NR\NC -4\NC -2\NR\stopbmatrix$

  $\Rightarrow X = \startbmatrix\NC -1\NC -1\NR\NC -2\NC -1\NR\stopbmatrix$.
\item $x^2 - 4x = -3 \Rightarrow x = 1, 3$. $x^2 = 1 \Rightarrow x = \pm 1$. $x^2 = -x + 2 \Rightarrow x =
  -2, 1$. $x^3 = 1 \Rightarrow x = 1, \omega, \omega^2$.

  Common value of $x$ is $1$.
\item $x + 3 = 0 \Rightarrow x = -3$. $2y + x = -7 \Rightarrow 2y = -4 \Rightarrow y = -2$. $z - 1 =
  3\Rightarrow z = 4$. $4a - 6 = 2a \Rightarrow 2a = 6\Rightarrow a = 3$.
\item $4A - 3B = 4\startbmatrix\NC 1 \NC 2 \NC 3\NR\NC-1 \NC 0 \NC 2\NR\NC1 \NC -3 \NC 1\NR\stopbmatrix -
  3\startbmatrix\NC 4 \NC 5 \NC 6\NR\NC -1 \NC 0 \NC 1\NR\NC 2 \NC 1 \NC 2\NR\stopbmatrix = \startbmatrix\NC
  4\NC 8\NC 12\NR\NC -4\NC 0\NC 8\NR\NC 4\NC -12\NC 4\NR\stopbmatrix - \startbmatrix\NC
  12\NC 15\NC 18\NR\NC -3\NC 0\NC 3\NR\NC 6\NC 3\NC 6\NR\stopbmatrix$

  $=\startbmatrix\NC-8\NC -7\NC -6\NR\NC -1\NC 0\NC 5\NR\NC -2\NC -15\NC -2\NR\stopbmatrix$.
\item $A$ is a $2\times3$ matrix and $B$ is a $3\times2$ matrix. $AB$ is defined and will be a $2\times2$
  matrix.

  $AB = \startbmatrix\NC 1 \NC -2 \NC 3 \NR\NC -4 \NC 2 \NC 5\NR\stopbmatrix\startbmatrix\NC 2 \NC 3 \NR\NC
  4 \NC 5 \NR\NC 2 \NC 1\NR\stopbmatrix = \startbmatrix\NC 2 - 8 + 6\NC 3 - 10 + 3\NR\NC -8 + 8 + 10\NC
  -12 + 10 + 5\NR\stopbmatrix$

  $= \startbmatrix\NC 0\NC -4\NR\NC 10\NC 3\NR\stopbmatrix$.

  $BA$ is also defined and will be a $3\times3$ matrix.

  $BA = \startbmatrix\NC 2 \NC 3 \NR\NC 4 \NC 5 \NR\NC 2 \NC 1\NR\stopbmatrix\startbmatrix\NC 1 \NC -2 \NC 3
  \NR\NC -4 \NC 2 \NC 5\NR\stopbmatrix$

  $= \startbmatrix\NC 2 - 12\NC -4 + 6\NC 6 + 15\NR\NC 4 - 20\NC -8 + 10\NC 12 + 25\NR\NC 2 - 4\NC -4 + 2\NC
  6 + 5\NR\stopbmatrix = \startbmatrix\NC -10\NC 2\NC 21\NR\NC -16\NC 2\NC 37\NR\NC -2\NC -2\NC
  11\NR\stopbmatrix$.

  Clearly, $AB\neq BA$.
\item Because associative law holds for matrix multiplication, therefore, $A(BC) = (AB)C$.

  $BC = \startbmatrix\NC a \NC h \NC g\NR\NC h \NC b \NC f \NR\NC g \NC f \NC
  c\NR\stopbmatrix\startbmatrix\NC x \NR\NC y \NR\NC z\NR\stopbmatrix = \startbmatrix\NC ax + by + gz\NR\NC
  hx + by + fz\NC gx + fy + zc\NR\stopbmatrix$

  $ABC = [x y z]\startbmatrix\NC ax + by + gz\NR\NC hx + by + fz\NR\NC gx + fy + zc\NR\stopbmatrix$

  $= x(ax + by + gz) + y(hx + by + fz) + z(gx + fy + zc) = ax^2 + by^2 + cz^2 + 2hxy + 2gzx + 2fyz$.
\item $A' = \startbmatrix\NC 1\NC 0\NC 2\NR\NC 2\NC 5\NC 4\NR\NC 3\NC 0\NC 3\NR\stopbmatrix$

  Let $B$ be the matrix whose elements are cofactors of the corresponding elements of the matrix $A$. Then
  $B = \startbmatrix\NC 15\NC 0\NC -10\NR\NC 6\NC -3\NC 0\NR\NC -15\NC 0\NC 5\NR\stopbmatrix$

  $\therefore \adj(A) = B' = \startbmatrix\NC 15\NC 6\NC -15\NR\NC 0\NC -3\NC 0\NR\NC -10\NC 0\NC
  5\NR\stopbmatrix$.
\item Let $B$ be the matrix whose elements are cofactors of the corresponding elements of $A$. Then

  $B = \startbmatrix\NC -1\NC 8\NC 5\NR\NC 1\NC -6\NC 3\NR\NC -1\NC 2\NC -1\NR\stopbmatrix\therefore \adj(A)
  = B' = \startbmatrix\NC -1 \NC 1\NC -1\NR\NC 8\NC -6\NC 2\NR\NC -5\NC 3\NC
  -1\NR\stopbmatrix$

  $|A| = \startdeterminant\NC 0 \NC 1 \NC 2\NR\NC1 \NC 2 \NC 3\NR\NC3 \NC 1 \NC 1\NR\stopdeterminant =
  -2\therefore A^{-1} = \frac{\adj(A)}{|A|} = \startbmatrix\NC\frac{1}{2}\NC-\frac{1}{2}\NC\frac{1}{2}\NR\NC
  -4\NC 3\NC -1\NR\NC\frac{5}{2}\NC-\frac{3}{2}\NC\frac{1}{2}\NR\stopbmatrix$.
\item Let $B$ be the matrix whose elements are cofactors of the corresponding elements of $A$. Then

  $B = \startbmatrix\NC 2\NC -3\NC 5\NR\NC 3\NC 6\NC -3\NR\NC -13\NC 9\NC -1\NR\stopbmatrix\therefore
  \adj(A) = B' = \startbmatrix\NC 2\NC 3\NC -13\NR\NC -3\NC 6\NC 9\NR\NC 5\NC -3\NC -1\NR\stopbmatrix$

  $|A| = \startdeterminant\NC 1 \NC 2 \NC 5\NR\NC2 \NC 3 \NC 1\NR\NC-1 \NC 1 \NC 1\NR\stopdeterminant =
  21\therefore A^{-1} = \frac{\adj(A)}{|A|} = \frac{1}{21}\startbmatrix\NC 2\NC 3\NC -13\NR\NC -3\NC 6\NC
  9\NR\NC 5\NC -3\NC -1\NR\stopbmatrix$.

  $A^{-1}A = \frac{1}{21}\startbmatrix\NC 2 + 6 + 13\NC 4 + 9 - 13\NC 10 + 3 - 13\NR\NC -3 + 12 - 9\NC -6 +
  18 + 9\NC -15 + 6 + 9\NR\NC 5 - 6 + 1\NC 10 -9 - 1\NC 25 - 3 - 1\NR\stopbmatrix = \startbmatrix\NC 1\NC
  0\NC 0\NR\NC 0\NC 1\NC 0\NR\NC 0\NC 0\NC 1\NR\stopbmatrix = I$
\item $A^2 = A.A = \startbmatrix\NC 1 \NC 2 \NC 2\NR\NC2 \NC 1 \NC 2\NR\NC2 \NC 2 \NC
  1\NR\stopbmatrix \startbmatrix\NC 1 \NC 2 \NC 2\NR\NC2 \NC 1 \NC 2\NR\NC2 \NC 2 \NC 1\NR\stopbmatrix$

  $= \startbmatrix\NC 1 + 4 + 4\NC 2 + 2 + 4\NC 2 + 4 + 2\NR\NC 2 + 2 + 4\NC 4 + 1 + 4\NC 4 + 2 + 2\NR\NC 2
  + 4 + 2\NC 4 + 2 + 2\NC 4 + 4 + 1\NR\stopbmatrix = \startbmatrix\NC 9\NC 8\NC 8\NR\NC 8\NC 9\NC 8\NR\NC
  8\NC 8\NC 9\NR\stopbmatrix$

  $A^2 - 4A - 5I = \startbmatrix\NC 9\NC 8\NC 8\NR\NC 8\NC 9\NC 8\NR\NC 8\NC 8\NC 9\NR\stopbmatrix
  - \startbmatrix\NC 4\NC 8\NC 8\NR\NC 8\NC 4\NC 8\NR\NC 8\NC 8\NC 4\NR\stopbmatrix - \startbmatrix\NC 5\NC
  0\NC 0\NR\NC 0\NC 5\NC 0\NR\NC 0\NC 0\NC 5\NR\stopbmatrix$

  $=\startbmatrix\NC 0\NC 0\NC 0\NR\NC 0\NC 0\NC 0\NR\NC 0\NC 0\NC 0\NR\stopbmatrix = O$

  $A^2 - 4A - 5I = O\Rightarrow A^{-1}A^2 - 4A^{-1}A - 5A^{-1}I = A^{-1}O = O$

  $(A^{-1}A)A - 4(A^{-1}A) - 5A^{-1}I = O \Rightarrow IA - 4I - 5A^{-1} = O$

  $\Rightarrow 5A^{-1} = A - 4I = \startbmatrix\NC 1 \NC 2 \NC 2\NR\NC2 \NC 1 \NC 2\NR\NC2 \NC 2 \NC
  1\NR\stopbmatrix - \startbmatrix\NC 4\NC 0\NC 0\NR\NC 0\NC 4\NC 0\NR\NC 0\NC 0\NC 4\NR\stopbmatrix$

  $= \startbmatrix\NC -3\NC 2\NC 2\NR\NC 2\NC -3\NC 2\NR\NC 2\NC 2\NC -3\NR\stopbmatrix\Rightarrow A^{-1} =
  \frac{1}{5}\startbmatrix\NC -3\NC 2\NC 2\NR\NC 2\NC -3\NC 2\NR\NC 2\NC 2\NC -3\NR\stopbmatrix$.
\item Let $A = \startbmatrix\NC 5\NC 3\NC 1\NR\NC 2\NC 1\NC 3\NR\NC 1\NC 2\NC 4\NR\stopbmatrix, X
  = \startbmatrix\NC x\NR\NC y\NR\NC z\NR\stopbmatrix$ and $B = \startbmatrix\NC 16\NR\NC 19\NR\NC
  25\NR\stopbmatrix$.

  Then the matrix equation of the gives system of equation becomes $AX = B$.

  Now $A = \startdeterminant\NC 1 \NC 2 \NC 2\NR\NC2 \NC 1 \NC 2\NR\NC2 \NC 2 \NC 1\NR\stopdeterminant =
  -22\neq 0$

  Hence, $A$ is non-singular. Therefore, the given system of equations will have the unique solution given
  by $X = A^{-1}B$.

  Let $C$ be the matrix whose elements are cofactors of the corresponding elements of $A$, then

  $C = \startbmatrix\NC -2\NC -5\NC 3\NR\NC -10\NC 19\NC -7\NR\NC 8\NC -13\NC -1\NR\stopbmatrix\therefore
  \adj(A) = C' = \startbmatrix\NC -2\NC -10\NC 8\NR\NC -5\NC 19\NC -13\NR\NC 3\NC -7\NC -1\NR\stopbmatrix$

  $A^{1-} = \frac{\adj(A)}{|A|} = -\frac{1}{22}\startbmatrix\NC -2\NC -10\NC 8\NR\NC -5\NC 19\NC -13\NR\NC
  3\NC -7\NC -1\NR\stopbmatrix$

  $\Rightarrow X = A^{-1}B = \startbmatrix\NC 1\NR\NC 2\NR\NC 5\NR\stopbmatrix\therefore x = 1, y = 2, z =
  5$.
\item $AB = \startbmatrix\NC -5 \NC 1 \NC 3\NR\NC7 \NC 1 \NC -5\NR\NC1 \NC -1 \NC 1
  \NR\stopbmatrix \startbmatrix\NC 1 \NC 1 \NC 2\NR\NC 3 \NC 2 \NC 1\NR\NC 2 \NC 1 \NC 3\NR\stopbmatrix$

  $=\startbmatrix\NC -5 + 3 + 6\NC -5 + 2 + 3\NC -10 + 1 + 9\NR\NC 7 + 3 - 10\NC 7 + 2 - 5\NC 14 + 1 - 15\NR
  \NC 1 - 3 + 2\NC 1 - 2 + 1\NC 2 - 1 + 3\NR\stopbmatrix = \startbmatrix\NC 4\NC 0\NC 0\NR\NC 0\NC 4\NC
  0\NR\NC 0\NC 0\NC 4\NR\stopbmatrix = 4I_3$

  Given system of equations in matrix form is $BX = C$, where $x = \startbmatrix\NC x\NR\NC y\NR\NC
  z\NR\stopbmatrix$ and $C = \startbmatrix\NC 1\NR\NC 7\NR\NC 2\NR\stopbmatrix$

  We have $BX = C$. Multiplying both sides with $B^{-1}$, $B^{-1}BX = B^{-1}C \Rightarrow IX = X = B^{-1}C$.

  However, $AB = 4I_3 \Rightarrow \frac{A}{4}B = I_3\Rightarrow B^{-1} = \frac{A}{4}\Rightarrow X
  = \startbmatrix\NC 2\NR\NC 1\NR\NC -1\NR\stopbmatrix$

  $\therefore x = 2, y = 1, z = -1$.
\item $x + y = 3, x - y = 7 \Rightarrow x = 5, y = -2$.
\item $x - y = -1, 2x - y = 0 \Rightarrow x = 1, y = 2$. $2x + x_1 = 5 \Rightarrow x_1 = 3, 3x + y_1 = 13
  \Rightarrow y_1 = 10$

  So $P$ is $(1, 2)$ and $Q$ is $(3, 10)$. $PQ = \sqrt{(1 - 3)^2 + (2 - 10)^2} = \sqrt{68} = 2\sqrt{17}$.
\item $2X = \startbmatrix\NC 10\NC 0\NR\NC 2\NC 8\NR\stopbmatrix \Rightarrow X = \startbmatrix\NC 5\NC
  0\NR\NC 1\NC 4\NR\stopbmatrix$

  $2Y = \startbmatrix\NC 4\NC 0\NR\NC 2\NC 2\NR\stopbmatrix\Rightarrow Y = \startbmatrix\NC 2\NC 0\NR\NC
  1\NC 1\NR\stopbmatrix$.
\item $C = B - A = \startbmatrix\NC 3 \NC -1 \NC 2 \NR\NC 4 \NC 2 \NC 5 \NR\NC 2 \NC 0 \NC 3\NR\stopbmatrix -
  \startbmatrix\NC  1 \NC 2 \NC -3 \NR\NC 5 \NC 0 \NC 2 \NR\NC 1 \NC -1 \NC 1\NR\stopbmatrix
    = \startbmatrix\NC 2\NC -3\NC 5\NR\NC -1\NC 2\NC 3\NR\NC 1\NC 1\NC 2\NR\stopbmatrix$.
\item $X = 2A + 3B - C = \startbmatrix\NC 4\NC 6\NC 8\NR\NC -6\NC 0\NC 4\NR\stopbmatrix + \startbmatrix\NC 9
  \NC -12\NC -15\NR\NC 3\NC 6\NC 3\NR\stopbmatrix - \startbmatrix\NC  5 \NC -1 \NC 2 \NR\NC 7 \NC 0 \NC
  3\NR\stopbmatrix$

  $= \startbmatrix\NC 8\NC -5\NC -9\NR\NC -10\NC 6\NC 4\NR\stopbmatrix$.
\item $A - 2B + 3C = \startbmatrix\NC  1 \NC 2 \NC 3 \NR\NC -1 \NC 0 \NC 2 \NR\NC 1 \NC -3 \NC
  1\NR\stopbmatrix - \startbmatrix\NC 8 \NC 10 \NC 12 \NR\NC -2 \NC 0 \NC 2 \NR\NC 4 \NC 2 \NC
  4\NR\stopbmatrix + \startbmatrix\NC -3 \NC 6 \NC 3 \NR\NC -3 \NC 6 \NC 9 \NR\NC -3 \NC -6 \NC
  6\NR\stopbmatrix$

  $= \startbmatrix\NC -10\NC -14\NC -6\NR\NC -2\NC 6\NC 9\NR\NC -6 \NC -11\NC 3\NR\stopbmatrix.$
\item $P(x).P(y) = \startbmatrix\NC \cos x \NC \sin x \NR\NC -\sin x \NC \cos x\NR\stopbmatrix
  . \startbmatrix\NC \cos y \NC \sin y \NR\NC -\sin y \NC \cos y\NR\stopbmatrix$

  $= \startbmatrix\NC \cos x\cos y - \sin x\sin y\NC \cos x\sin y + \sin x\cos y\NR\NC -\sin x\cos y - cos
  x\sin y\NC -\sin x\sin y + \cos x\cos y\NR\stopbmatrix = \startbmatrix\NC \cos (x + y) \NC \sin (x + y)
  \NR\NC -\sin (x + y) \NC \cos (x + y)\NR\stopbmatrix$.

  Similarly, it can be proven to be equal to $P(y).P(x)$.
\item $A^2 = \startbmatrix\NC 1 \NC 0 \NC 0 \NR\NC 0 \NC 1 \NC 0 \NR\NC a \NC b \NC -1\NR\stopbmatrix
  . \startbmatrix\NC 1 \NC 0 \NC 0 \NR\NC 0 \NC 1 \NC 0 \NR\NC a \NC b \NC -1\NR\stopbmatrix$

  $=\startbmatrix\NC 1*1 + 0*0 + 0*a\NC 1*0 + 0*1 + 0*b\NC 1*0 + 0*0 + 0*-1\NR\NC 0*1 + 1*0 + 0*a\NC 0*0 +
  1*1 + 0*b\NC 0*0 + 1*0 + 0*-1\NR\NC a*1 + b*0 + -1*a\NC a*0 + b*1 + -1*b\NC a*0 + b*0 +
  -1*-1\NR\stopbmatrix$

  $=\startbmatrix\NC 1\NC 0\NC 0\NR\NC 0\NC 1\NC 0\NR\NC 0\NC 0\NC 1\NR\stopbmatrix = I_3$.
\item $A^2 = \startbmatrix\NC -1 \NC 1 \NC -1 \NR\NC 3 \NC -3 \NC 3 \NR\NC 5 \NC -5 \NC 5 \NR\stopbmatrix
  \startbmatrix\NC -1 \NC 1 \NC -1 \NR\NC 3 \NC -3 \NC 3 \NR\NC 5 \NC -5 \NC 5 \NR\stopbmatrix$

  $= \startbmatrix\NC -1*-1 + 1*3 + -1*5\NC -1*1 + 1*-3 + -1*-5\NC -1*-1 + 1*3 + -1*5\NR\NC 3*-1 + -3*3 +
    3*5\NC 3*1 + -3*-3 + 3*-5\NC 3*-1 + -3*3 + 3*5\NR\NC 5*-1 + -5*3 + 5*5\NC 5*1 + -5*-3 + 5*-5\NC 5*-1 +
    -5*3 + 5*5\NR\stopbmatrix$

  $= \startbmatrix\NC -1\NC 1\NC -1\NR\NC 3\NC -3\NC 3\NR\NC 5\NC -5\NC 5\NR\stopbmatrix = A$

  $B^2 = \startbmatrix\NC 0 \NC 4 \NC 3 \NR\NC 1 \NC -3 \NC -3 \NR\NC -1\NC 4 \NC
    4\NR\stopbmatrix \startbmatrix\NC 0 \NC 4 \NC 3 \NR\NC 1 \NC -3 \NC -3 \NR\NC -1\NC 4 \NC
    4\NR\stopbmatrix$

  $= \startbmatrix\NC 0*0 + 4*1 + 3*-1\NC 0*4 + 4*-3 + 3*4\NC 0*3 + 4*-3 + 3*4\NR\NC 1*0 + -3*1 + -3*-1\NC
    1*4 + -3*-3 + -3*4\NC 1*3 + -3*-3 + -3*4\NR\NC -1*0 + 4*1 + 4*-1\NC -1*4 + 4*-3 + 4*4\NC -1*3 + 4*-3 +
    4*4\NR\stopbmatrix$

  $= \startbmatrix\NC 1\NC 0\NC 0\NR\NC 0\NC 1\NC 0\NR\NC 0\NC 0\NC 1\NR\stopbmatrix = I$

  $\therefore A^2B^2 = A$
\item Given, $A = \startbmatrix\NC 2 \NC 3 \NC 4 \NR\NC 1\NC 2 \NC 3 \NR\NC -1 \NC 1 \NC 2\NR\stopbmatrix ,
  B = \startbmatrix\NC 1 \NC 3 \NC 0\NR\NC -1 \NC 2 \NC 1 \NR\NC 0 \NC 0 \NC 2\NR\stopbmatrix$.

  $AB = \startbmatrix\NC 2*1 + 3*-1 + 4*0\NC 2*3 + 3*2 + 4*0\NC 2*0 + 3*1 + 4*2\NR\NC 1*1 + 2*-1 + 3*0\NC
  1*3 + 2*2 + 3*0\NC 1*0 + 2*1 + 3*2\NR\NC -1*1 + 1*-1 + 2*0\NC -1*3 + 1*2 + 2*0\NC -1*0 + 1*1 +
  2*2\NR\stopbmatrix$

  $= \startbmatrix\NC -1\NC 12\NC 11\NR\NC -1\NC 7\NC 8\NR\NC -2\NC -1\NC 5\NR\stopbmatrix$

  $\BA = \startbmatrix\NC 1*2 + 3*1 + 0*-1\NC 1*3 + 3*2 + 0*1\NC 1*4 + 3*3 + 0*2\NR\NC -1*2 + 2*1 + 1*-1\NC
  -1*3 + 2*2 + 1*1\NC -1*4 + 2*3 + 1*2\NR\NC 0*2 + 0*1 + 2*-1\NC 0*3 + 0*2 + 2*1\NC 0*4 + 0*3 +
  2*2\NR\stopbmatrix$


  $= \startbmatrix\NC 5\NC 9\NC 13\NR\NC -1\NC 2\NC 4\NR\NC -2\NC 2\NC 4\NR\stopbmatrix$.


  Clearly, $AB\neq BA$.
\item Let $A = \startbmatrix\NC 0 \NC c \NC -b \NR\NC -c \NC 0 \NC a \NR\NC b \NC -a \NC 0\NR\stopbmatrix$
  and $B = \startbmatrix\NC a^2 \NC ab \NC ac \NR\NC ab \NC b^2 \NC bc \NR\NC ac \NC bc \NC
  c^2\NR\stopbmatrix$

  $AB = \startbmatrix\NC 0*a^2 + c*ab + -b*ac\NC  0*ab + c*b^2 + -b*bc\NC  0*ac + c*bc + -b*c^2\NR\NC
  -c*a^2 + 0*ab + a*ac\NC  -c*ab + 0*b^2 + a*bc\NC  -c*ac + 0*bc + a*c^2\NR\NC  b*a^2 + -a*ab + 0*ac\NC
  b*ab + -a*b^2 + 0*bc\NC  b*ac + -a*bc + 0*c^2\NR\stopbmatrix$

  $=$ a zero matrix.
\item Given $A = \startbmatrix\NC 3 \NC -5 \NR\NC -4 \NC 2\NR\stopbmatrix\therefore A^2 = \startbmatrix\NC
  3*3 + -5*-4\NC 3*-5 + -5*2\NR\NC -4*3 + 2*-4\NC -4*-5 + 2*2\NR\stopbmatrix$

  $= \startbmatrix\NC 29\NC -25\NR\NC -20\NC 24\NR\stopbmatrix$

  $\therefore A^2 - 5A - 14I = \startbmatrix\NC 0\NC 0\NR\NC 0\NC 0\NR\stopbmatrix$
\item Given $A = \startbmatrix\NC 2 \NC 3\NR\NC 1 \NC 2\NR\stopbmatrix\therefore A^2 = \startbmatrix\NC 2*2
  + 3*1\NC 2*3 + 3*2\NR\NC 1*2 + 2*1\NC 1*3 + 2*2\NR\stopbmatrix$

  $= \startbmatrix\NC 7\NC 12\NR\NC 4\NC 7\NR\stopbmatrix$

  $\therefore A^3 = \startbmatrix\NC 7*7 + 12*4\NC 7*12 + 12*7\NR\NC 4*7 + 7*4\NC 4*12 + 7*7\NR\stopbmatrix
  = \startbmatrix\NC 97\NC 168\NR\NC 56\NC 97\NR\stopbmatrix$

  Clearly, $A^3 - 4A^2 + A = \startbmatrix\NC 71\NC 123\NR\NC 41\NC 71\NR\stopbmatrix$, which can be easily
  shown to be an orthogonal matrix.
\item $A^2 = \startbmatrix\NC 0.8*0.8 + 0.6*-0.6\NC 0.8*0.6 + 0.6*0.8\NR\NC -0.6*0.8 + 0.8*-0.6\NC -0.6*0.6
  + 0.8*0.8\NR\stopbmatrix$

  Similarly we proceed for $A^3$ which turns out to be $\startbmatrix\NC -0.352\NC 0.936\NR\NC -0.936\NC
  -0.352\NR\stopbmatrix$.
\item $f(A) = A^2 - 5A + 7I$, where $A = \startbmatrix\NC 3 \NC 1 \NR\NC -1 \NC 2\NR\stopbmatrix$

  $A^2 = \startbmatrix\NC 3*3 + 1*-1\NC 3*1 + 1*2\NR\NC -1*3 + 2*-1\NC -1*1 + 2*2\NR\stopbmatrix
  = \startbmatrix\NC 8\NC 5\NR\NC -5\NC 3\NR\stopbmatrix$

  Thus, $A^2 - 5A + 7I$ is a $2\times2$ zero matrix, which is trivial to prove.
\item $AB = \startbmatrix\NC \cos\theta*\cos\phi + \sin\theta*\sin\phi\NC \cos\theta*\sin\phi +
  \sin\theta*\cos\phi\NR\NC \sin\theta*\cos\phi + \cos\theta*\sin\phi\NC \sin\theta*\sin\phi +
  \cos\theta*\cos\phi\NR\stopbmatrix$

  $= \startbmatrix\NC \cos(\theta - \phi)\NC \sin(\theta + \phi)\NR\NC \sin(\theta + \phi)\NC \cos(\theta -
  \phi)\NR\stopbmatrix$

  Similarly $BA = \startbmatrix\NC \cos(\theta - \phi)\NC \sin(\theta + \phi)\NR\NC \sin(\theta + \phi)\NC
  \cos(\theta - \phi)\NR\stopbmatrix$

  Thus, $AB = BA$.
\item $f(A) = A^2 - 5A + 6,\;A^2 = \startbmatrix\NC 2*2 + 0*2 + 1*1\NC 2*0 + 0*1 + 1*-1\NC 2*1 + 0*3 +
  1*0\NR\NC 2*2 + 1*2 + 3*1\NC 2*0 + 1*1 + 3*-1\NC 2*1 + 1*3 + 3*0\NR\NC 1*2 + -1*2 + 0*1\NC 1*0 + -1*1 +
  0*-1\NC 1*1 + -1*3 + 0*0\NR\stopbmatrix$

  $= \startbmatrix\NC 5\NC -1\NC 2\NR\NC 9\NC -2\NC 5\NR\NC 0\NC -1\NC -2\NR\stopbmatrix$

  Thus, $A^2 - 5A + 6 = \startbmatrix\NC 1\NC -1\NC -3\NR\NC -1\NC -1\NC -10\NR\NC -5\NC 4\NC
  4\NR\stopbmatrix$.
\item Given $A = \startbmatrix\NC 5 \NC 3 \NR\NC 12 \NC 7\NR\stopbmatrix\;\therefore A^2 = \startbmatrix 5*5
  + 3*12\NC 5*3 + 3*7\NR\NC 12*5 + 7*12\NC 12*3 + 7*7\NR\stopbmatrix$

  $= \startbmatrix\NC 61\NC 36\NR\NC 144\NC 85\NR\stopbmatrix$

  $\therefore A^2 - 12 A - I = 0$
\item We have $\startpmatrix\NC\startbmatrix\NC 1 \NC \omega \NC \omega^2 \NR\NC\omega \NC \omega^2 \NC 1 \NR\NC
  \omega^2 \NC 1 \NC \omega \NR\stopbmatrix + \startbmatrix\NC  \omega \NC \omega^2 \NC 1 \NR\NC \omega^2
  \NC 1 \NC \omega \NR\NC \omega \NC \omega^2 \NC 1\NR\stopbmatrix\NR\stoppmatrix \startbmatrix\NC 1 \NR\NC
  \omega \NR\NC \omega^2 \NR\stopbmatrix = \startbmatrix\NC 1 + \omega\NC \omega + \omega^2\NC \omega^2 +
  1\NR\NC \omega + \omega^2\NC \omega^2 + 1\NC 1 + \omega\NR\NC \omega^2 + \omega\NC 1 + \omega^2\NC\omega +
  1\NR\stopbmatrix\startbmatrix\NC 1\NR\NC \omega \NR\NC \omega^2 \NR\stopbmatrix$

  $= \startbmatrix1 + \omega + \omega^2 + \omega^3 + \omega^4 + \omega^2\NR\NC \omega + \omega^2 + \omega^3
  + \omega + \omega^4 + \omega^3\NR\NC \omega^2 + \omega + \omega + \omega^3 + \omega^3 +
  \omega^2\NR\stopbmatrix = 2\startbmatrix\NC 1 + \omega + \omega^2\NR\NC 1 + \omega + \omega^2\NR\NC 1 +
  \omega + \omega^2\NR\stopbmatrix = \startbmatrix\NC 0\NR\NC 0\NR\NC 0\NR\stopbmatrix$.
\item $I + A = \startbmatrix\NC 1\NC -\tan\frac{\alpha}{2}\NR\NC \tan\frac{\alpha}{2}\NC 1\NR\stopbmatrix$

  $(I - A)\startbmatrix\NC\cos\alpha\NC -\sin\alpha\NR\NC \sin\alpha\NC
  \cos\alpha\NR\stopbmatrix= \startbmatrix\NC 1\NC \tan\frac{\alpha}{2}\NR\NC -\tan\frac{\alpha}{2}\NC
  1\NR\stopbmatrix\startbmatrix\NC\cos\alpha\NC -\sin\alpha\NR\NC \sin\alpha\NC \cos\alpha\NR\stopbmatrix$

  $= \startbmatrix\NC\cos\alpha + \tan\frac{\alpha}{2}.\sin\alpha\NC \tan\frac{\alpha}{2}\cos\alpha -
  \sin\alpha\NR\NC \sin\alpha - \tan\frac{\alpha}{2}\cos\alpha\NC \tan\frac{\alpha}{2}\sin\alpha +
  \cos\alpha\NR\stopbmatrix$

  Substituting $\cos\alpha = \cos^2\frac{\alpha}{2} - \sin^2\frac{\alpha}{2}$ and $\sin\alpha =
  2\sin\frac{\alpha}{2}\cos\frac{\alpha}{2}$ we get the desired result.
\item If we multiply two matrices on left-hand side and compare the terms with right-hand side then we will
  get four equations in $x, y, z$ and $u$. We also have four unknowns, which is a solvable system of linear
  equations. The solution is left as an exercise.
\item We have $\startbmatrix\NC 1 \NC x \NC 1\NR\stopbmatrix\startbmatrix\NC  1 \NC 3 \NC 2 \NR\NC 0 \NC 5
  \NC 1 \NR\NC 0 \NC 3 \NC 2 \NR\stopbmatrix \startbmatrix\NC 1 \NR\NC 1 \NR\NC x\NR\stopbmatrix = 0$

  $\Rightarrow \startbmatrix\NC 1\NC 3 + 5x + 3\NC 2 + x + 2\NR\stopbmatrix\startbmatrix\NC 1 \NR\NC 1
  \NR\NC x\NR\stopbmatrix = 0$

  $1 + 6 + 5x + 4x + x^2 = 0 \Rightarrow x^2 + 9x + 7 = 0 \Rightarrow x = \frac{-9 \pm\sqrt{53}}{2}$.
\item Product is $\startbmatrix\NC \cos^2\theta\cos^2\phi + \sin\theta\cos\theta\cos\phi\sin\phi\NC
  \cos^2\theta\cos\phi\sin\phi + \sin\theta\cos\theta\sin^2\phi\NR\NC \cos^2\theta\sin\theta\cos\theta +
  \sin^2\theta\cos\phi\sin\phi\NC \cos\theta\sin\theta\cos\phi\sin\phi +
  \sin^2\theta\sin^2\phi\NR\stopbmatrix$

  $= \startbmatrix\NC\cos\theta\cos\phi\cos(\theta - \phi)\NC \sin\phi\sin\theta\cos(\theta - \phi)\NR\NC
  \cos\phi\sin\theta\cos(\theta - \phi)\NC \sin\theta\sin\phi\cos(\theta - \phi)\NR\stopbmatrix$

  Clearly the above matrix is a zero matrix of the difference of angles is an odd multiple of
  $\frac{\pi}{2}$.
\item This we will prove by mathematical induction. We have $A = \startbmatrix\NC \cos\theta \NC -\sin\theta
  \NR\NC \sin\theta \NC \cos\theta \NR\stopbmatrix$.

  $A^2 = \startbmatrix\NC \cos\theta \NC -\sin\theta\NR\NC \sin\theta \NC \cos\theta
  \NR\stopbmatrix.\startbmatrix\NC \cos\theta \NC -\sin\theta \NR\NC \sin\theta \NC \cos\theta
  \NR\stopbmatrix$

  $= \startbmatrix\NC \cos^2\theta - \sin^2\theta\NC -\cos\theta\sin\theta - \sin\theta\cos\theta\NR\NC
  \sin\theta\cos\theta + \cos\theta\sin\theta\NC -\sin^2\theta + \cos^2\theta\NR\stopbmatrix =
  \startbmatrix\NC\cos2\theta\NC-\sin2\theta\NR\NC \sin2\theta\NC \cos2\theta\NR\stopbmatrix$

  Thus, the result is true for $n = 2$. Let it be true for $n = k$ i.e. $A^k = \startbmatrix\NC \cos k\theta
  \NC -\sin k\theta \NR\NC \sin k\theta \NC  \cos k\theta\NR\stopbmatrix$.

  $A^{k + 1} = \startbmatrix\NC\cos k\theta\cos\theta -\sin k\theta\sin\theta\NC -\cos k\theta\sin\theta -
  \sin k\theta\cos\theta\NR\NC \sin k\theta\cos\theta + \cos k\theta\sin\theta\NC -\sin k\theta\sin\theta +
  \cos k\theta\cos\theta\NR\stopbmatrix$

  $= \startbmatrix\NC\cos(k + 1)\theta\NC -\sin(k + 1)\theta\NR\NC \sin(k + 1)\theta\NC \cos(k +
  1)\theta\NR\stopbmatrix$, which is true for $n = k + 1$.

  Thus, we have proven the required result by mathematical induction.
\item We have $A = \startbmatrix\NC  0 \NC 1 \NR\NC 0 \NC 0\NR\stopbmatrix$. We will prove this by
  mathematical induction like last problem.

  $A^2 = \startbmatrix\NC 3*3 - 4*1\NC 3*-4 - 4*-1\NR\NC 1*3 - 1*1\NC 1*-4 - 1*-1\NR\stopbmatrix =
  \startbmatrix\NC 1 + 2*2\NC -4*2\NR\NC 2\NC 1 - 2*2\NR\stopbmatrix$, which is true for $n = 2$. Let it be
  true for $n = k$ i.e. $A^k = \startbmatrix\NC 1 + 2k \NC -4k \NR\NC k \NC 1 -
  2k\NR\stopbmatrix$

  $A^{k + 1} =\startbmatrix\NC(1 + 2k)*3 - 4k*1\NC (1 + 2k)*-4 - 4k*-1\NR\NC k*3 + (1 - 2k)*1\NC k*-4 + (1 -
  2k)*-1\NR\stopbmatrix = \startbmatrix\NC 1 + 2(k + 1)\NC -4(k + 1)\NR\NC k + 1\NC 1 - 2(k + 1)\NR\stopbmatrix$,
  which is true for $n = k + 1$.

  Thus, we have proven the required result by mathematical induction.
\item $(aI + bA)^1 = aI + 1a^{1 - 1}bA = aI + bA$, thus the statement holds true for $n = 1$.

  Let it be true for $n = k$ i.e. $(aI + bA)^k = a^kI + ka^{k - 1}bA$

  For $n = k + 1, (aI + bA)^{k + 1} = (a^kI + ka^{k - 1}bA)(aI + bA) = a^{k + 1}I + ka^{k - 1}abA + a^kbA +
  ka^{k - 1}bA*bA$

  However, $A^2 = 0$, so$a^{k + 1}I + ka^kbA + a^kbA = a^{k + 1}I + (k + 1)a^kbA$.

  Thus, the statement is true for $n = k + 1$, and, hence proved.
\item We know that for matrix multiplication it is not neccessary that $AB = BA$. However, $(A + B)(A - B) =
  A^2 - AB + BA - B^2$, which will be equal to $A^2 - B^2$ if $AB = BA$.
\item We can represent quanitity bought using a row matrix, for example, $Q = \startbmatrix\NC 8\NC 10\NC
  4\NR\stopbmatrix$ and rate as $R = \startbmatrix\NC18\NR\NC 9\NR\NC 6\NR\stopbmatrix$.

  Total cost would be product of these two matrices i.e. $8*18 + 10*9 + 4*6 = 144 + 90 + 24 = 258$.
\item Let the amount invested in first fund is USD $x$, and in second fund USD $30000 - x$. Then we can
  represent interest as

  $\startbmatrix\NC x\NC 30000 - x\NR\stopbmatrix.\startbmatrix\NC 0.05\NR\NC 0.07\NR\stopbmatrix = 2000$

  $\Rightarrow 0.05x + 2100 - 0.07x = 2000 \Rightarrow 0.02x = 100 \Rightarrow x = 5000$. Thus, amount to be
  invested in first fund is USD $5000$, and in second fund USD $25000$ should be invested.
\item The store owner has $240$ shirts, $180$ trousers and $300$ pair of socks, which can be represented by
  a row matrix, $I = \startbmatrix\NC 240\NC 180\NC 300\NR\stopbmatrix$ for example. The respective costs
  can be represented by a column matrix, $R = \startbmatrix\NC 50\NR\NC 90\NR\NC 12\NR\stopbmatrix$.

  Thus, according to question, total amount receieved would be $IR = 240*50 + 180*90 + 300*12 = \$24600$.
\item There are $120$ physics books, $96$ chemistry books, and $60$ mathematics books. These can be
  represented by a row matric,$B = \startbmatrix\NC 120\NC 96\NC 60\NR\stopbmatrix$, for example. The
  respective costs can be represented by a column matrix, $R = \startbmatrix\NC 8.3\NR\NC 3.45\NR\NC
  4.5\NR\stopbmatrix$.

  Thus, total amount received by the store owner upon sell of all the books will be $BR = 120*8.3 + 96*3.45
  + 60*4.5 = 1597.2$
\item Given, $A = \startbmatrix\NC \cos\alpha \NC \sin\alpha \NR\NC -\sin\alpha \NC \cos
  \alpha\NR\stopbmatrix$. Therefore, $A' = \startbmatrix\NC\cos\alpha\NC -\sin\alpha\NR\NC \sin\alpha\NC
  \cos\alpha\NR\stopbmatrix$

  $AA' = \startbmatrix\NC\cos^2\alpha + \sin^2\alpha\NC \cos\alpha.-\sin\alpha + \sin\alpha.\cos\alpha\NR\NC
  -\sin\alpha.\cos\alpha + \cos\alpha.\sin\alpha\NC -\sin\alpha.-\sin\alpha +
  \cos\alpha.\cos\alpha\NR\stopbmatrix = \startbmatrix\NC 1\NC 0\NR\NC 0\NC 1\NR\stopbmatrix = I_2$

  $A'A = \startbmatrix\NC -\sin\alpha.-\sin\alpha + \cos\alpha.\cos\alpha\NC \cos\alpha\sin\alpha
  -\sin\alpha\cos\alpha\NR\NC \sin\alpha\cos\alpha - \cos\alpha\sin\alpha\NC \sin^2\alpha +
  \cos^2\alpha\NR\stopbmatrix = \startbmatrix\NC 1\NC 0\NR\NC 0\NC 1\NR\stopbmatrix = I_2$

  Thus, $AA' = A'A = I_2$.
\item We know that for a symmetric matrix its trnaspose is equal to original matrix i.e. if $A$ is the
  matrix then $A = A^T$, while for a skew-symmetric matrix its transpose is equal to neation of original
  matrix i.e. $A = -A^T$.

  Thus, any square matrix can be represented as sum of its symmetric matrix(say $S$) and skew-symmetric
  matrix(say $K$) as $S = \frac{1}{2}(A + A^T)$ and $K = \frac{1}{2}(A - A^T)$.

  Thus, $S = \frac{1}{2}\left(\startbmatrix\NC 1\NC 2\NC 4\NR\NC 6\NC 8\NC 1\NR\NC 3\NC 5\NC
  7\NR\stopbmatrix + \startbmatrix\NC 1\NC 6\NC 3\NR\NC 2\NC 8\NC 5\NR\NC 4\NC 1\NC 7\NR\stopbmatrix\right)
  = \startbmatrix\NC 1\NC 4\NC \frac{7}{2}\NR\NC 4\NC 8\NC 3\NR\NC \frac{7}{2}\NC 3\NC 7\NR\stopbmatrix$

  and $K = \frac{1}{2}\left(\startbmatrix\NC 1\NC 2\NC 4\NR\NC 6\NC 8\NC 1\NR\NC 3\NC 5\NC
  7\NR\stopbmatrix - \startbmatrix\NC 1\NC 6\NC 3\NR\NC 2\NC 8\NC 5\NR\NC 4\NC 1\NC 7\NR\stopbmatrix\right)
  = \startbmatrix\NC 0\NC -2\NC \frac{1}{2}\NR\NC 2\NC 0\NC -2\NR\NC -\frac{1}{2}\NC 2\NC 0\NR\stopbmatrix$.
\item We know that a matrix is orthogonal if its product with its transpose is yields an identity matrix. We
  have already shown this in second last problem for the given matrix.
\item The transpose of given matrix is equal to the original i.e. $A = A'$.

  $AA' = \frac{1}{9}\startbmatrix\NC -1 \NC 2 \NC 2 \NR\NC 2 \NC -1 \NC 2 \NR\NC 2 \NC 2 \NC
  -1\NR\stopbmatrix. \startbmatrix\NC -1 \NC 2 \NC 2 \NR\NC 2 \NC -1 \NC 2 \NR\NC 2 \NC 2 \NC
  -1\NR\stopbmatrix$

  $ = \startbmatrix\NC -1*-1 + 2*2 + 2*2\NC -1*2 + 2*-1 + 2*2\NC -1*2 + 2*2 + 2*-1\NR\NC 2*-1 - 1*2 + 2*2\NC
  2*2 - 1*-1 + 2*2\NC 2*2 - 1*2 + 2*-1\NR\NC 2*-1 + 2*2 - 1*2\NC 2*2 + 2*-1 + -1*2\NC 2*2 + 2*2 - 1 *
  -1\NR\stopbmatrix= I_3$.

  Thus, given matrix is orthogonal.
\item Given $A = \startbmatrix\NC 1 \NC 2 \NC 3 \NR\NC 2 \NC 3 \NC 2 \NR\NC 3 \NC 3 \NC
  4\NR\stopbmatrix$. Let $B$ represent the matrix of cofactors of $A$, then

  $B = \startbmatrix\NC 3*4- 3*2\NC 3*2 - 4*2\NC 3*2 - 3*3\NR\NC 3*3 - 4*2\NC 4*1 - 3*3\NC 3*2 - 3*1\NR\NC
  2*2 - 3*3\NC 3*2 - 2*1\NC 3*1 - 2*2\NR\stopbmatrix = \startbmatrix\NC 6\NC -2\NC -3\NR\NC 1\NC -5\NC
  3\NR\NC -5\NC 4\NC 1\NR\stopbmatrix$.

  Thus, $adj(A) = B' = \startbmatrix\NC 6\NC 1\NC -5\NR\NC -2\NC -5\NC 4\NR\NC -3\NC 3\NC
  -1\NR\stopbmatrix$.
\item First we find the adjoint of the matrix for which we need to find cofactors. Let $C_{ij}$ represent
  the cofactors, then,

  $C_{11} = \cos\alpha*1 - 0*0 = \cos\alpha, C_{12} = -(\sin\alpha*1 - 0*0) = -\sin\alpha, C_{13} =
  \sin\alpha*0 - \cos\alpha*0 = 0, C_{21} = -(\sin\alpha*1 - 0*0), C_{22} = \cos\alpha*1 - 0*0 =
  \cos\alpha, C_{23} = -(\cos\alpha*0 - 0*0) = 0, C_{31} = \cos\alpha*0 - 0*0 = 0, C_{32} = -\cos\alpha*0 -
  0*0 = 0, C_{33} = \cos\alpha*1 - 0*0 = \cos\alpha$

  Hence, $adj(A) = \startbmatrix\NC \cos\alpha\NC -\sin\alpha\NC 0\NR\NC -\sin\alpha\NC \cos\alpha\NC
  0\NR\NC 0\NC 0\NC \cos\alpha\NR\stopbmatrix$

  $\therefore A*adj(A) = \startbmatrix\NC\cos^2\alpha + \sin^2\alpha\NC 0\NC 0\NR\NC 0\NC \cos^2\alpha +
  \sin^2\alpha\NC 0\NR\NC 0\NC 0\NC 1\NR\stopbmatrix = I$

  Now, $|A| = \cos\alpha(\cos\alpha*1 - 0*0) -(-\sin\alpha)(\sin\alpha*1 - 0*0) + 0 = 1$

  $|A|I = I$. Hence proven.
\item We know that $A(adj\;A) = |A|I$. Here $|A| = 1(3*10 - 2*0) - (-1)(2*10 - 18*0) + 1(2*2 - 18*3) =
  0$. Hence, $A(adj\;A) = 0$.
\item Let $A = \startbmatrix\NC 1 \NC 3 \NC 3 \NR\NC 1 \NC 4 \NC 3 \NR\NC 1 \NC 3 \NC 4\NR\stopbmatrix$

  $|A| = 1(4*4 - 3*3) - 3(1.*4 - 4*1) + 3(1*3 - 4*1) = 1$

  Matrix of cofactos is $\startbmatrix\NC (4*4 - 3*3)\NC (1*4 - 3*1)\NC (1*3 - 4*1)\NR\NC (3*4 - 3*3)\NC
  (1*4 - 3*1)\NC (1*3 - 3*1)\NR\NC (3*3 - 4*3)\NC (1*3 - 3*1)\NC (1*4 - 3*1)\NR\stopbmatrix$

  Hence, $adj(A) = \startbmatrix\NC 7\NC -3\NC -3\NR\NC -1\NC 1\NC 0\NR\NC -1\NC 0\NC 1\NR\stopbmatrix$

  Because $|A| = 1$, therefore, $A^{-1} = adj(A) = \startbmatrix\NC 7\NC -3\NC -3\NR\NC -1\NC 1\NC 0\NR\NC
  -1\NC 0\NC 1\NR\stopbmatrix$.
\item $|A| = 2(2*2 - 3*-2) - (-3)(2*2 - 3*3) + 3(2*-2 - 2*3) = -25$

  Matrix of cofactors is $\startbmatrix\NC(2*2 - 3*-2)\NC -(2*2 - 3*3)\NC (2*-2 -2*3)\NR\NC -(-3*2 -
  3*-2)\NC (2*2 - 3*3)\NC -(2*-2 - (-3)*3)\NR\NC (-3*3 - 2*3)\NC -(2*3 - 2*3)\NC (2*2 -
  (-3)*2)\NR\stopbmatrix = \startbmatrix\NC 10\NC 5\NC -10\NR\NC 0\NC -5\NC -5\NR\NC -15\NC 0\NC
  10\NR\stopbmatrix$

  $adj(A) = \startbmatrix\NC 10\NC 0\NC -15\NR\NC 5\NC -5\NC 0\NR\NC -10\NC -5\NC 10\NR\stopbmatrix$

  $\Rightarrow A^{-1} = \frac{adj(A)}{|A|} = \frac{1}{5}\startbmatrix\NC -2\NC 0\NC 3\NR\NC -1\NC 1\NC
  0\NR\NC 2\NC 1\NC -2\NR\stopbmatrix$.
\item The inverse can be calculated like previous problems and is equal to $\frac{1}{ad -
  bc}\startbmatrix\NC d\NC -b\NR\NC -c\NC a\NR\stopbmatrix$.
\item In previous problem we have formula for inverse of a $2\times2$ matrix.

  $AB = \startbmatrix\NC 14\NC 5\NR\NC 16\NC 0\NR\stopbmatrix$.

  Hence, $(AB)^{-1} = \frac{1}{-80}\startbmatrix\NC 0\NC -5\NR\NC -16\NC -14\NR\stopbmatrix$.

  $A^{-1} = \frac{1}{-4}\startbmatrix\NC 0\NC -1\NR\NC -4\NC 3\NR\stopbmatrix\;B^{-1} =
  \frac{1}{20}\startbmatrix\NC 5\NC 0\NR\NC -2\NC 4\NR\stopbmatrix$

  $A^{-1}B^{-1} = \frac{1}{-80}\startbmatrix\NC 0\NC -5\NR\NC -16\NC -14\NR\stopbmatrix$.

  Hence, $(AB)^{-1} = A^{-1}B^{-1}$.
\item Given $A = \startbmatrix\NC 1 \NC \tan x \NR\NC -\tan x \NC 1\NR\stopbmatrix$ so $A^{-1} =
  \cos^2x\startbmatrix\NC 1\NC -\tan x\NR\NC\tan x\NC 1\NR\stopbmatrix = \startbmatrix\NC \cos^2x\NC -\sin
  x\cos x\NR\NC \sin x\cos x\NC \cos^2x\NR\stopbmatrix$

  $\therefore A'A^{-1} = \startbmatrix\NC 1\NC -\tan x\NR\NC \tan x\NC 1\NR\stopbmatrix. \startbmatrix\NC
  \cos^2x\NC -\sin x\cos x\NR\NC \sin x\cos x\NC \cos^2x\NR\stopbmatrix$

  $= \startbmatrix\NC \cos 2x \NC -\sin 2x \NR\NC \sin 2x \NC \cos 2x\NR\stopbmatrix$.
\item (Hint: $(AB)^{-1} = B^{-1}A^{-1}$)

  $A^{-1} = \startbmatrix\NC 5\NC -2\NR\NC -7\NC 3\NR\stopbmatrix$ (from formula obtained in third last
  problem).

  $B^{-1} = \startbmatrix\NC -\frac{9}{2}\NC -\frac{7}{2}\NR\NC 4\NC -3\NR\stopbmatrix$

  $(AB)^{-1} = \startbmatrix\NC -47\NC \frac{39}{2}\NR\NC 41\NC -17\NR\stopbmatrix$.
\item We can write the given system of equations as $AX = B$, where $A = \startbmatrix\NC 3\NC -2\NR\NC 5\NC
  3\NR\stopbmatrix, X = \startbmatrix\NC x\NR\NC y\NR\stopbmatrix$, and $B = \startbmatrix\NC 7\NR\NC
  1\NR\stopbmatrix$.

  $|A| = 3*3 - (-2)*5 = 19\Rightarrow A^{-1} = \frac{1}{19}\startbmatrix\NC 3\NC 2\NR\NC -5\NC
  3\NR\stopbmatrix$

  $X = A^{-1}B$. We multiply the matrices and compare elements to get $x$ and $y$ as $\frac{23}{19}$ and
  $-\frac{32}{19}$ respectively.
\item Proceeding like previous problem we represent the given system of equations as $AX = B$, where $A
  = \startbmatrix\NC 2\NC -3\NC 3\NR\NC 2\NC 2\NC 3\NR\NC 3\NC -2\NC 2\NR\stopbmatrix, X = \startbmatrix\NC
  x\NR\NC y\NR\NC z\NR\stopbmatrix$, and $B = \startbmatrix\NC 1\NR\NC 2\NR\NC 3\NR\stopbmatrix$.

  $|A| = 2(2*2 - 3*-2) - (-3)(2*2 - 3*3) + 3(2*-2 - 2*3) = -25$

  Matrix of cofactors is $\startbmatrix\NC 10\NC 5\NC -10\NR\NC 0\NC -5\NC -5\NR\NC -15\NC 0\NC
  10\NR\stopbmatrix$.

  $\Rightarrow adj(A) = \startbmatrix\NC 10\NC 0\NC -15\NR\NC 5\NC -5\NC 0\NR\NC -10\NC -5\NC
  10\NR\stopbmatrix$.

  $A^{-1} = \frac{adj(A)}{|A|} = \startbmatrix\NC -\frac{2}{5}\NC 0\NC \frac{3}{5}\NR\NC -\frac{1}{5}\NC
  \frac{1}{5}\NC 0\NR\NC \frac{2}{5}\NC \frac{1}{5}\NC -\frac{2}{5}\NR\stopbmatrix$

  $X = A^{-1}B$. We multiply the matrices and compare elements to get solution as $x = \frac{7}{5}, y =
  \frac{1}{5}$ and $z = -\frac{2}{5}$.
\item We represent the gives system of equations as $AX = B$, where $A = \startbmatrix\NC 2\NC 3\NR\NC 6\NC
  9\NR\stopbmatrix, X = \startbmatrix\NC x\NR\NC y\NR\stopbmatrix$ and $B = \startbmatrix\NC 5\NC
  10\NR\stopbmatrix$

  $|A| = 0$, which means either it is inconsistent or has infinitely many solutions. However, we observe
  that $(2x + 3y = 5) (3 * 2x = 6x, 3 * 3y = 9y, 3 * 5 = 15$, and $3 * 10 = 30)$. Thus, the given system of
  equations is inconsistent
\stopitemize
