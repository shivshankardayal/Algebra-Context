% -*- mode: context; -*-
\chapter{Matrices}
\startitemize[n, 1*broad]
\item The matrix will be any one of the following type $1\times12, 12\times1, 2\times6, 6\times2, 3\times4,
  4\times3$. So the answer is $6$.
\item $a_{11} = 2.1 - 3.1 = -1, a_{12} = 2.1 - 3.2 = -4, a_{13} = 2.1 - 3.3 = -7$

  $a_{21} = 2.2 - 3.1 = 1, a_{22} = 2.2 - 3.2 = -2, a_{23} = 2.2 - 3.3 = -5$

  $\therefore A = \startbmatrix\NC-1\NC-4\NC-7\NR\NC1\NC -2\NC -5\NR\stopbmatrix$.
\item $A + B = \startbmatrix\NC a - a\NC b + b\NR\NC -b - b \NC a - a\NR\stopbmatrix = \startbmatrix\NC0\NC
  2b\NR\NC -2b \NC0\NR\stopbmatrix$.
\item $2X = (2X + Y) - Y = \startbmatrix\NC 1\NC 0\NR\NC -3 \NC 2\NR\stopbmatrix - \startbmatrix\NC 3\NC
  2\NR\NC 1\NC 4\NR\stopbmatrix = \startbmatrix\NC1 - 3\NC 0 - 2\NR\NC -3 -1\NC 2 - 4\NR\stopbmatrix
  = \startbmatrix\NC -2\NC -2\NR\NC -4\NC -2\NR\stopbmatrix$

  $\Rightarrow X = \startbmatrix\NC -1\NC -1\NR\NC -2\NC -1\NR\stopbmatrix$.
\item $x^2 - 4x = -3 \Rightarrow x = 1, 3$. $x^2 = 1 \Rightarrow x = \pm 1$. $x^2 = -x + 2 \Rightarrow x =
  -2, 1$. $x^3 = 1 \Rightarrow x = 1, \omega, \omega^2$.

  Common value of $x$ is $1$.
\item $x + 3 = 0 \Rightarrow x = -3$. $2y + x = -7 \Rightarrow 2y = -4 \Rightarrow y = -2$. $z - 1 =
  3\Rightarrow z = 4$. $4a - 6 = 2a \Rightarrow 2a = 6\Rightarrow a = 3$.
\item $4A - 3B = 4\startbmatrix\NC 1 \NC 2 \NC 3\NR\NC-1 \NC 0 \NC 2\NR\NC1 \NC -3 \NC 1\NR\stopbmatrix -
  3\startbmatrix\NC 4 \NC 5 \NC 6\NR\NC -1 \NC 0 \NC 1\NR\NC 2 \NC 1 \NC 2\NR\stopbmatrix = \startbmatrix\NC
  4\NC 8\NC 12\NR\NC -4\NC 0\NC 8\NR\NC 4\NC -12\NC 4\NR\stopbmatrix - \startbmatrix\NC
  12\NC 15\NC 18\NR\NC -3\NC 0\NC 3\NR\NC 6\NC 3\NC 6\NR\stopbmatrix$

  $=\startbmatrix\NC-8\NC -7\NC -6\NR\NC -1\NC 0\NC 5\NR\NC -2\NC -15\NC -2\NR\stopbmatrix$.
\item $A$ is a $2\times3$ matrix and $B$ is a $3\times2$ matrix. $AB$ is defined and will be a $2\times2$
  matrix.

  $AB = \startbmatrix\NC 1 \NC -2 \NC 3 \NR\NC -4 \NC 2 \NC 5\NR\stopbmatrix\startbmatrix\NC 2 \NC 3 \NR\NC
  4 \NC 5 \NR\NC 2 \NC 1\NR\stopbmatrix = \startbmatrix\NC 2 - 8 + 6\NC 3 - 10 + 3\NR\NC -8 + 8 + 10\NC
  -12 + 10 + 5\NR\stopbmatrix$

  $= \startbmatrix\NC 0\NC -4\NR\NC 10\NC 3\NR\stopbmatrix$.

  $BA$ is also defined and will be a $3\times3$ matrix.

  $BA = \startbmatrix\NC 2 \NC 3 \NR\NC 4 \NC 5 \NR\NC 2 \NC 1\NR\stopbmatrix\startbmatrix\NC 1 \NC -2 \NC 3
  \NR\NC -4 \NC 2 \NC 5\NR\stopbmatrix$

  $= \startbmatrix\NC 2 - 12\NC -4 + 6\NC 6 + 15\NR\NC 4 - 20\NC -8 + 10\NC 12 + 25\NR\NC 2 - 4\NC -4 + 2\NC
  6 + 5\NR\stopbmatrix = \startbmatrix\NC -10\NC 2\NC 21\NR\NC -16\NC 2\NC 37\NR\NC -2\NC -2\NC
  11\NR\stopbmatrix$.

  Clearly, $AB\neq BA$.
\item Because associative law holds for matrix multiplication, therefore, $A(BC) = (AB)C$.

  $BC = \startbmatrix\NC a \NC h \NC g\NR\NC h \NC b \NC f \NR\NC g \NC f \NC
  c\NR\stopbmatrix\startbmatrix\NC x \NR\NC y \NR\NC z\NR\stopbmatrix = \startbmatrix\NC ax + by + gz\NR\NC
  hx + by + fz\NC gx + fy + zc\NR\stopbmatrix$

  $ABC = [x y z]\startbmatrix\NC ax + by + gz\NR\NC hx + by + fz\NR\NC gx + fy + zc\NR\stopbmatrix$

  $= x(ax + by + gz) + y(hx + by + fz) + z(gx + fy + zc) = ax^2 + by^2 + cz^2 + 2hxy + 2gzx + 2fyz$.
\item $A' = \startbmatrix\NC 1\NC 0\NC 2\NR\NC 2\NC 5\NC 4\NR\NC 3\NC 0\NC 3\NR\stopbmatrix$

  Let $B$ be the matrix whose elements are cofactors of the corresponding elements of the matrix $A$. Then
  $B = \startbmatrix\NC 15\NC 0\NC -10\NR\NC 6\NC -3\NC 0\NR\NC -15\NC 0\NC 5\NR\stopbmatrix$

  $\therefore \adj(A) = B' = \startbmatrix\NC 15\NC 6\NC -15\NR\NC 0\NC -3\NC 0\NR\NC -10\NC 0\NC
  5\NR\stopbmatrix$.
\item Let $B$ be the matrix whose elements are cofactors of the corresponding elements of $A$. Then

  $B = \startbmatrix\NC -1\NC 8\NC 5\NR\NC 1\NC -6\NC 3\NR\NC -1\NC 2\NC -1\NR\stopbmatrix\therefore \adj(A)
  = B' = \startbmatrix\NC -1 \NC 1\NC -1\NR\NC 8\NC -6\NC 2\NR\NC -5\NC 3\NC
  -1\NR\stopbmatrix$

  $|A| = \startdeterminant\NC 0 \NC 1 \NC 2\NR\NC1 \NC 2 \NC 3\NR\NC3 \NC 1 \NC 1\NR\stopdeterminant =
  -2\therefore A^{-1} = \frac{\adj(A)}{|A|} = \startbmatrix\NC\frac{1}{2}\NC-\frac{1}{2}\NC\frac{1}{2}\NR\NC
  -4\NC 3\NC -1\NR\NC\frac{5}{2}\NC-\frac{3}{2}\NC\frac{1}{2}\NR\stopbmatrix$.
\item Let $B$ be the matrix whose elements are cofactors of the corresponding elements of $A$. Then

  $B = \startbmatrix\NC 2\NC -3\NC 5\NR\NC 3\NC 6\NC -3\NR\NC -13\NC 9\NC -1\NR\stopbmatrix\therefore
  \adj(A) = B' = \startbmatrix\NC 2\NC 3\NC -13\NR\NC -3\NC 6\NC 9\NR\NC 5\NC -3\NC -1\NR\stopbmatrix$

  $|A| = \startdeterminant\NC 1 \NC 2 \NC 5\NR\NC2 \NC 3 \NC 1\NR\NC-1 \NC 1 \NC 1\NR\stopdeterminant =
  21\therefore A^{-1} = \frac{\adj(A)}{|A|} = \frac{1}{21}\startbmatrix\NC 2\NC 3\NC -13\NR\NC -3\NC 6\NC
  9\NR\NC 5\NC -3\NC -1\NR\stopbmatrix$.

  $A^{-1}A = \frac{1}{21}\startbmatrix\NC 2 + 6 + 13\NC 4 + 9 - 13\NC 10 + 3 - 13\NR\NC -3 + 12 - 9\NC -6 +
  18 + 9\NC -15 + 6 + 9\NR\NC 5 - 6 + 1\NC 10 -9 - 1\NC 25 - 3 - 1\NR\stopbmatrix = \startbmatrix\NC 1\NC
  0\NC 0\NR\NC 0\NC 1\NC 0\NR\NC 0\NC 0\NC 1\NR\stopbmatrix = I$
\item $A^2 = A.A = \startbmatrix\NC 1 \NC 2 \NC 2\NR\NC2 \NC 1 \NC 2\NR\NC2 \NC 2 \NC
  1\NR\stopbmatrix \startbmatrix\NC 1 \NC 2 \NC 2\NR\NC2 \NC 1 \NC 2\NR\NC2 \NC 2 \NC 1\NR\stopbmatrix$

  $= \startbmatrix\NC 1 + 4 + 4\NC 2 + 2 + 4\NC 2 + 4 + 2\NR\NC 2 + 2 + 4\NC 4 + 1 + 4\NC 4 + 2 + 2\NR\NC 2
  + 4 + 2\NC 4 + 2 + 2\NC 4 + 4 + 1\NR\stopbmatrix = \startbmatrix\NC 9\NC 8\NC 8\NR\NC 8\NC 9\NC 8\NR\NC
  8\NC 8\NC 9\NR\stopbmatrix$

  $A^2 - 4A - 5I = \startbmatrix\NC 9\NC 8\NC 8\NR\NC 8\NC 9\NC 8\NR\NC 8\NC 8\NC 9\NR\stopbmatrix
  - \startbmatrix\NC 4\NC 8\NC 8\NR\NC 8\NC 4\NC 8\NR\NC 8\NC 8\NC 4\NR\stopbmatrix - \startbmatrix\NC 5\NC
  0\NC 0\NR\NC 0\NC 5\NC 0\NR\NC 0\NC 0\NC 5\NR\stopbmatrix$

  $=\startbmatrix\NC 0\NC 0\NC 0\NR\NC 0\NC 0\NC 0\NR\NC 0\NC 0\NC 0\NR\stopbmatrix = O$

  $A^2 - 4A - 5I = O\Rightarrow A^{-1}A^2 - 4A^{-1}A - 5A^{-1}I = A^{-1}O = O$

  $(A^{-1}A)A - 4(A^{-1}A) - 5A^{-1}I = O \Rightarrow IA - 4I - 5A^{-1} = O$

  $\Rightarrow 5A^{-1} = A - 4I = \startbmatrix\NC 1 \NC 2 \NC 2\NR\NC2 \NC 1 \NC 2\NR\NC2 \NC 2 \NC
  1\NR\stopbmatrix - \startbmatrix\NC 4\NC 0\NC 0\NR\NC 0\NC 4\NC 0\NR\NC 0\NC 0\NC 4\NR\stopbmatrix$

  $= \startbmatrix\NC -3\NC 2\NC 2\NR\NC 2\NC -3\NC 2\NR\NC 2\NC 2\NC -3\NR\stopbmatrix\Rightarrow A^{-1} =
  \frac{1}{5}\startbmatrix\NC -3\NC 2\NC 2\NR\NC 2\NC -3\NC 2\NR\NC 2\NC 2\NC -3\NR\stopbmatrix$.
\item Let $A = \startbmatrix\NC 5\NC 3\NC 1\NR\NC 2\NC 1\NC 3\NR\NC 1\NC 2\NC 4\NR\stopbmatrix, X
  = \startbmatrix\NC x\NR\NC y\NR\NC z\NR\stopbmatrix$ and $B = \startbmatrix\NC 16\NR\NC 19\NR\NC
  25\NR\stopbmatrix$.

  Then the matrix equation of the gives system of equation becomes $AX = B$.

  Now $A = \startdeterminant\NC 1 \NC 2 \NC 2\NR\NC2 \NC 1 \NC 2\NR\NC2 \NC 2 \NC 1\NR\stopdeterminant =
  -22\neq 0$

  Hence, $A$ is non-singular. Therefore, the given system of equations will have the unique solution given
  by $X = A^{-1}B$.

  Let $C$ be the matrix whose elements are cofactors of the corresponding elements of $A$, then

  $C = \startbmatrix\NC -2\NC -5\NC 3\NR\NC -10\NC 19\NC -7\NR\NC 8\NC -13\NC -1\NR\stopbmatrix\therefore
  \adj(A) = C' = \startbmatrix\NC -2\NC -10\NC 8\NR\NC -5\NC 19\NC -13\NR\NC 3\NC -7\NC -1\NR\stopbmatrix$

  $A^{1-} = \frac{\adj(A)}{|A|} = -\frac{1}{22}\startbmatrix\NC -2\NC -10\NC 8\NR\NC -5\NC 19\NC -13\NR\NC
  3\NC -7\NC -1\NR\stopbmatrix$

  $\Rightarrow X = A^{-1}B = \startbmatrix\NC 1\NR\NC 2\NR\NC 5\NR\stopbmatrix\therefore x = 1, y = 2, z =
  5$.
\item $AB = \startbmatrix\NC -5 \NC 1 \NC 3\NR\NC7 \NC 1 \NC -5\NR\NC1 \NC -1 \NC 1
  \NR\stopbmatrix \startbmatrix\NC 1 \NC 1 \NC 2\NR\NC 3 \NC 2 \NC 1\NR\NC 2 \NC 1 \NC 3\NR\stopbmatrix$

  $=\startbmatrix\NC -5 + 3 + 6\NC -5 + 2 + 3\NC -10 + 1 + 9\NR\NC 7 + 3 - 10\NC 7 + 2 - 5\NC 14 + 1 - 15\NR
  \NC 1 - 3 + 2\NC 1 - 2 + 1\NC 2 - 1 + 3\NR\stopbmatrix = \startbmatrix\NC 4\NC 0\NC 0\NR\NC 0\NC 4\NC
  0\NR\NC 0\NC 0\NC 4\NR\stopbmatrix = 4I_3$

  Given system of equations in matrix form is $BX = C$, where $x = \startbmatrix\NC x\NR\NC y\NR\NC
  z\NR\stopbmatrix$ and $C = \startbmatrix\NC 1\NR\NC 7\NR\NC 2\NR\stopbmatrix$

  We have $BX = C$. Multiplying both sides with $B^{-1}$, $B^{-1}BX = B^{-1}C \Rightarrow IX = X = B^{-1}C$.

  However, $AB = 4I_3 \Rightarrow \frac{A}{4}B = I_3\Rightarrow B^{-1} = \frac{A}{4}\Rightarrow X
  = \startbmatrix\NC 2\NR\NC 1\NR\NC -1\NR\stopbmatrix$

  $\therefore x = 2, y = 1, z = -1$.
\item $x + y = 3, x - y = 7 \Rightarrow x = 5, y = -2$.
\item $x - y = -1, 2x - y = 0 \Rightarrow x = 1, y = 2$. $2x + x_1 = 5 \Rightarrow x_1 = 3, 3x + y_1 = 13
  \Rightarrow y_1 = 10$

  So $P$ is $(1, 2)$ and $Q$ is $(3, 10)$. $PQ = \sqrt{(1 - 3)^2 + (2 - 10)^2} = \sqrt{68} = 2\sqrt{17}$.
\item $2X = \startbmatrix\NC 10\NC 0\NR\NC 2\NC 8\NR\stopbmatrix \Rightarrow X = \startbmatrix\NC 5\NC
  0\NR\NC 1\NC 4\NR\stopbmatrix$

  $2Y = \startbmatrix\NC 4\NC 0\NR\NC 2\NC 2\NR\stopbmatrix\Rightarrow Y = \startbmatrix\NC 2\NC 0\NR\NC
  1\NC 1\NR\stopbmatrix$.
\item $C = B - A = \startbmatrix\NC 3 \NC -1 \NC 2 \NR\NC 4 \NC 2 \NC 5 \NR\NC 2 \NC 0 \NC 3\NR\stopbmatrix -
  \startbmatrix\NC  1 \NC 2 \NC -3 \NR\NC 5 \NC 0 \NC 2 \NR\NC 1 \NC -1 \NC 1\NR\stopbmatrix
    = \startbmatrix\NC 2\NC -3\NC 5\NR\NC -1\NC 2\NC 3\NR\NC 1\NC 1\NC 2\NR\stopbmatrix$.
\item $X = 2A + 3B - C = \startbmatrix\NC 4\NC 6\NC 8\NR\NC -6\NC 0\NC 4\NR\stopbmatrix + \startbmatrix\NC 9
  \NC -12\NC -15\NR\NC 3\NC 6\NC 3\NR\stopbmatrix - \startbmatrix\NC  5 \NC -1 \NC 2 \NR\NC 7 \NC 0 \NC
  3\NR\stopbmatrix$

  $= \startbmatrix\NC 8\NC -5\NC -9\NR\NC -10\NC 6\NC 4\NR\stopbmatrix$.
\item $A - 2B + 3C = \startbmatrix\NC  1 \NC 2 \NC 3 \NR\NC -1 \NC 0 \NC 2 \NR\NC 1 \NC -3 \NC
  1\NR\stopbmatrix - \startbmatrix\NC 8 \NC 10 \NC 12 \NR\NC -2 \NC 0 \NC 2 \NR\NC 4 \NC 2 \NC
  4\NR\stopbmatrix + \startbmatrix\NC -3 \NC 6 \NC 3 \NR\NC -3 \NC 6 \NC 9 \NR\NC -3 \NC -6 \NC
  6\NR\stopbmatrix$

  $= \startbmatrix\NC -10\NC -14\NC -6\NR\NC -2\NC 6\NC 9\NR\NC -6 \NC -11\NC 3\NR\stopbmatrix.$
\item $P(x).P(y) = \startbmatrix\NC \cos x \NC \sin x \NR\NC -\sin x \NC \cos x\NR\stopbmatrix
  . \startbmatrix\NC \cos y \NC \sin y \NR\NC -\sin y \NC \cos y\NR\stopbmatrix$

  $= \startbmatrix\NC \cos x\cos y - \sin x\sin y\NC \cos x\sin y + \sin x\cos y\NR\NC -\sin x\cos y - cos
  x\sin y\NC -\sin x\sin y + \cos x\cos y\NR\stopbmatrix = \startbmatrix\NC \cos (x + y) \NC \sin (x + y)
  \NR\NC -\sin (x + y) \NC \cos (x + y)\NR\stopbmatrix$.

  Similarly, it can be proven to be equal to $P(y).P(x)$.
\item $A^2 = \startbmatrix\NC 1 \NC 0 \NC 0 \NR\NC 0 \NC 1 \NC 0 \NR\NC a \NC b \NC -1\NR\stopbmatrix
  . \startbmatrix\NC 1 \NC 0 \NC 0 \NR\NC 0 \NC 1 \NC 0 \NR\NC a \NC b \NC -1\NR\stopbmatrix$

  $=\startbmatrix\NC 1*1 + 0*0 + 0*a\NC 1*0 + 0*1 + 0*b\NC 1*0 + 0*0 + 0*-1\NR\NC 0*1 + 1*0 + 0*a\NC 0*0 +
  1*1 + 0*b\NC 0*0 + 1*0 + 0*-1\NR\NC a*1 + b*0 + -1*a\NC a*0 + b*1 + -1*b\NC a*0 + b*0 +
  -1*-1\NR\stopbmatrix$

  $=\startbmatrix\NC 1\NC 0\NC 0\NR\NC 0\NC 1\NC 0\NR\NC 0\NC 0\NC 1\NR\stopbmatrix = I_3$.
\item $A^2 = \startbmatrix\NC -1 \NC 1 \NC -1 \NR\NC 3 \NC -3 \NC 3 \NR\NC 5 \NC -5 \NC 5 \NR\stopbmatrix
  \startbmatrix\NC -1 \NC 1 \NC -1 \NR\NC 3 \NC -3 \NC 3 \NR\NC 5 \NC -5 \NC 5 \NR\stopbmatrix$

  $= \startbmatrix\NC -1*-1 + 1*3 + -1*5\NC -1*1 + 1*-3 + -1*-5\NC -1*-1 + 1*3 + -1*5\NR\NC 3*-1 + -3*3 +
    3*5\NC 3*1 + -3*-3 + 3*-5\NC 3*-1 + -3*3 + 3*5\NR\NC 5*-1 + -5*3 + 5*5\NC 5*1 + -5*-3 + 5*-5\NC 5*-1 +
    -5*3 + 5*5\NR\stopbmatrix$

  $= \startbmatrix\NC -1\NC 1\NC -1\NR\NC 3\NC -3\NC 3\NR\NC 5\NC -5\NC 5\NR\stopbmatrix = A$

  $B^2 = \startbmatrix\NC 0 \NC 4 \NC 3 \NR\NC 1 \NC -3 \NC -3 \NR\NC -1\NC 4 \NC
    4\NR\stopbmatrix \startbmatrix\NC 0 \NC 4 \NC 3 \NR\NC 1 \NC -3 \NC -3 \NR\NC -1\NC 4 \NC
    4\NR\stopbmatrix$

  $= \startbmatrix\NC 0*0 + 4*1 + 3*-1\NC 0*4 + 4*-3 + 3*4\NC 0*3 + 4*-3 + 3*4\NR\NC 1*0 + -3*1 + -3*-1\NC
    1*4 + -3*-3 + -3*4\NC 1*3 + -3*-3 + -3*4\NR\NC -1*0 + 4*1 + 4*-1\NC -1*4 + 4*-3 + 4*4\NC -1*3 + 4*-3 +
    4*4\NR\stopbmatrix$

  $= \startbmatrix\NC 1\NC 0\NC 0\NR\NC 0\NC 1\NC 0\NR\NC 0\NC 0\NC 1\NR\stopbmatrix = I$

  $\therefore A^2B^2 = A$
\item Given, $A = \startbmatrix\NC 2 \NC 3 \NC 4 \NR\NC 1\NC 2 \NC 3 \NR\NC -1 \NC 1 \NC 2\NR\stopbmatrix ,
  B = \startbmatrix\NC 1 \NC 3 \NC 0\NR\NC -1 \NC 2 \NC 1 \NR\NC 0 \NC 0 \NC 2\NR\stopbmatrix$.

  $AB = \startbmatrix\NC 2*1 + 3*-1 + 4*0\NC 2*3 + 3*2 + 4*0\NC 2*0 + 3*1 + 4*2\NR\NC 1*1 + 2*-1 + 3*0\NC
  1*3 + 2*2 + 3*0\NC 1*0 + 2*1 + 3*2\NR\NC -1*1 + 1*-1 + 2*0\NC -1*3 + 1*2 + 2*0\NC -1*0 + 1*1 +
  2*2\NR\stopbmatrix$

  $= \startbmatrix\NC -1\NC 12\NC 11\NR\NC -1\NC 7\NC 8\NR\NC -2\NC -1\NC 5\NR\stopbmatrix$

  $\BA = \startbmatrix\NC 1*2 + 3*1 + 0*-1\NC 1*3 + 3*2 + 0*1\NC 1*4 + 3*3 + 0*2\NR\NC -1*2 + 2*1 + 1*-1\NC
  -1*3 + 2*2 + 1*1\NC -1*4 + 2*3 + 1*2\NR\NC 0*2 + 0*1 + 2*-1\NC 0*3 + 0*2 + 2*1\NC 0*4 + 0*3 +
  2*2\NR\stopbmatrix$


  $= \startbmatrix\NC 5\NC 9\NC 13\NR\NC -1\NC 2\NC 4\NR\NC -2\NC 2\NC 4\NR\stopbmatrix$.


  Clearly, $AB\neq BA$.
\item Let $A = \startbmatrix\NC 0 \NC c \NC -b \NR\NC -c \NC 0 \NC a \NR\NC b \NC -a \NC 0\NR\stopbmatrix$
  and $B = \startbmatrix\NC a^2 \NC ab \NC ac \NR\NC ab \NC b^2 \NC bc \NR\NC ac \NC bc \NC
  c^2\NR\stopbmatrix$

  $AB = \startbmatrix\NC 0*a^2 + c*ab + -b*ac\NC  0*ab + c*b^2 + -b*bc\NC  0*ac + c*bc + -b*c^2\NR\NC
  -c*a^2 + 0*ab + a*ac\NC  -c*ab + 0*b^2 + a*bc\NC  -c*ac + 0*bc + a*c^2\NR\NC  b*a^2 + -a*ab + 0*ac\NC
  b*ab + -a*b^2 + 0*bc\NC  b*ac + -a*bc + 0*c^2\NR\stopbmatrix$

  $=$ a zero matrix.
\item Given $A = \startbmatrix\NC 3 \NC -5 \NR\NC -4 \NC 2\NR\stopbmatrix\therefore A^2 = \startbmatrix\NC
  3*3 + -5*-4\NC 3*-5 + -5*2\NR\NC -4*3 + 2*-4\NC -4*-5 + 2*2\NR\stopbmatrix$

  $= \startbmatrix\NC 29\NC -25\NR\NC -20\NC 24\NR\stopbmatrix$

  $\therefore A^2 - 5A - 14I = \startbmatrix\NC 0\NC 0\NR\NC 0\NC 0\NR\stopbmatrix$
\item Given $A = \startbmatrix\NC 2 \NC 3\NR\NC 1 \NC 2\NR\stopbmatrix\therefore A^2 = \startbmatrix\NC 2*2
  + 3*1\NC 2*3 + 3*2\NR\NC 1*2 + 2*1\NC 1*3 + 2*2\NR\stopbmatrix$

  $= \startbmatrix\NC 7\NC 12\NR\NC 4\NC 7\NR\stopbmatrix$

  $\therefore A^3 = \startbmatrix\NC 7*7 + 12*4\NC 7*12 + 12*7\NR\NC 4*7 + 7*4\NC 4*12 + 7*7\NR\stopbmatrix
  = \startbmatrix\NC 97\NC 168\NR\NC 56\NC 97\NR\stopbmatrix$

  Clearly, $A^3 - 4A^2 + A = \startbmatrix\NC 71\NC 123\NR\NC 41\NC 71\NR\stopbmatrix$, which can be easily
  shown to be an orthogonal matrix.
\item $A^2 = \startbmatrix\NC 0.8*0.8 + 0.6*-0.6\NC 0.8*0.6 + 0.6*0.8\NR\NC -0.6*0.8 + 0.8*-0.6\NC -0.6*0.6
  + 0.8*0.8\NR\stopbmatrix$

  Similarly we proceed for $A^3$ which turns out to be $\startbmatrix\NC -0.352\NC 0.936\NR\NC -0.936\NC
  -0.352\NR\stopbmatrix$.
\item $f(A) = A^2 - 5A + 7I$, where $A = \startbmatrix\NC 3 \NC 1 \NR\NC -1 \NC 2\NR\stopbmatrix$

  $A^2 = \startbmatrix\NC 3*3 + 1*-1\NC 3*1 + 1*2\NR\NC -1*3 + 2*-1\NC -1*1 + 2*2\NR\stopbmatrix
  = \startbmatrix\NC 8\NC 5\NR\NC -5\NC 3\NR\stopbmatrix$

  Thus, $A^2 - 5A + 7I$ is a $2\times2$ zero matrix, which is trivial to prove.
\item $AB = \startbmatrix\NC \cos\theta*\cos\phi + \sin\theta*\sin\phi\NC \cos\theta*\sin\phi +
  \sin\theta*\cos\phi\NR\NC \sin\theta*\cos\phi + \cos\theta*\sin\phi\NC \sin\theta*\sin\phi +
  \cos\theta*\cos\phi\NR\stopbmatrix$

  $= \startbmatrix\NC \cos(\theta - \phi)\NC \sin(\theta + \phi)\NR\NC \sin(\theta + \phi)\NC \cos(\theta -
  \phi)\NR\stopbmatrix$

  Similarly $BA = \startbmatrix\NC \cos(\theta - \phi)\NC \sin(\theta + \phi)\NR\NC \sin(\theta + \phi)\NC
  \cos(\theta - \phi)\NR\stopbmatrix$

  Thus, $AB = BA$.
\item $f(A) = A^2 - 5A + 6,\;A^2 = \startbmatrix\NC 2*2 + 0*2 + 1*1\NC 2*0 + 0*1 + 1*-1\NC 2*1 + 0*3 +
  1*0\NR\NC 2*2 + 1*2 + 3*1\NC 2*0 + 1*1 + 3*-1\NC 2*1 + 1*3 + 3*0\NR\NC 1*2 + -1*2 + 0*1\NC 1*0 + -1*1 +
  0*-1\NC 1*1 + -1*3 + 0*0\NR\stopbmatrix$

  $= \startbmatrix\NC 5\NC -1\NC 2\NR\NC 9\NC -2\NC 5\NR\NC 0\NC -1\NC -2\NR\stopbmatrix$

  Thus, $A^2 - 5A + 6 = \startbmatrix\NC 1\NC -1\NC -3\NR\NC -1\NC -1\NC -10\NR\NC -5\NC 4\NC
  4\NR\stopbmatrix$.
\item Given $A = \startbmatrix\NC 5 \NC 3 \NR\NC 12 \NC 7\NR\stopbmatrix\;\therefore A^2 = \startbmatrix 5*5
  + 3*12\NC 5*3 + 3*7\NR\NC 12*5 + 7*12\NC 12*3 + 7*7\NR\stopbmatrix$

  $= \startbmatrix\NC 61\NC 36\NR\NC 144\NC 85\NR\stopbmatrix$

  $\therefore A^2 - 12 A - I = 0$
\item We have $\startpmatrix\NC\startbmatrix\NC 1 \NC \omega \NC \omega^2 \NR\NC\omega \NC \omega^2 \NC 1 \NR\NC
  \omega^2 \NC 1 \NC \omega \NR\stopbmatrix + \startbmatrix\NC  \omega \NC \omega^2 \NC 1 \NR\NC \omega^2
  \NC 1 \NC \omega \NR\NC \omega \NC \omega^2 \NC 1\NR\stopbmatrix\NR\stoppmatrix \startbmatrix\NC 1 \NR\NC
  \omega \NR\NC \omega^2 \NR\stopbmatrix = \startbmatrix\NC 1 + \omega\NC \omega + \omega^2\NC \omega^2 +
  1\NR\NC \omega + \omega^2\NC \omega^2 + 1\NC 1 + \omega\NR\NC \omega^2 + \omega\NC 1 + \omega^2\NC\omega +
  1\NR\stopbmatrix\startbmatrix\NC 1\NR\NC \omega \NR\NC \omega^2 \NR\stopbmatrix$

  $= \startbmatrix1 + \omega + \omega^2 + \omega^3 + \omega^4 + \omega^2\NR\NC \omega + \omega^2 + \omega^3
  + \omega + \omega^4 + \omega^3\NR\NC \omega^2 + \omega + \omega + \omega^3 + \omega^3 +
  \omega^2\NR\stopbmatrix = 2\startbmatrix\NC 1 + \omega + \omega^2\NR\NC 1 + \omega + \omega^2\NR\NC 1 +
  \omega + \omega^2\NR\stopbmatrix = \startbmatrix\NC 0\NR\NC 0\NR\NC 0\NR\stopbmatrix$.
\item $I + A = \startbmatrix\NC 1\NC -\tan\frac{\alpha}{2}\NR\NC \tan\frac{\alpha}{2}\NC 1\NR\stopbmatrix$

  $(I - A)\startbmatrix\NC\cos\alpha\NC -\sin\alpha\NR\NC \sin\alpha\NC
  \cos\alpha\NR\stopbmatrix= \startbmatrix\NC 1\NC \tan\frac{\alpha}{2}\NR\NC -\tan\frac{\alpha}{2}\NC
  1\NR\stopbmatrix\startbmatrix\NC\cos\alpha\NC -\sin\alpha\NR\NC \sin\alpha\NC \cos\alpha\NR\stopbmatrix$

  $= \startbmatrix\NC\cos\alpha + \tan\frac{\alpha}{2}.\sin\alpha\NC \tan\frac{\alpha}{2}\cos\alpha -
  \sin\alpha\NR\NC \sin\alpha - \tan\frac{\alpha}{2}\cos\alpha\NC \tan\frac{\alpha}{2}\sin\alpha +
  \cos\alpha\NR\stopbmatrix$

  Substituting $\cos\alpha = \cos^2\frac{\alpha}{2} - \sin^2\frac{\alpha}{2}$ and $\sin\alpha =
  2\sin\frac{\alpha}{2}\cos\frac{\alpha}{2}$ we get the desired result.
\item If we multiply two matrices on left-hand side and compare the terms with right-hand side then we will
  get four equations in $x, y, z$ and $u$. We also have four unknowns, which is a solvable system of linear
  equations. The solution is left as an exercise.
\item We have $\startbmatrix\NC 1 \NC x \NC 1\NR\stopbmatrix\startbmatrix\NC  1 \NC 3 \NC 2 \NR\NC 0 \NC 5
  \NC 1 \NR\NC 0 \NC 3 \NC 2 \NR\stopbmatrix \startbmatrix\NC 1 \NR\NC 1 \NR\NC x\NR\stopbmatrix = 0$

  $\Rightarrow \startbmatrix\NC 1\NC 3 + 5x + 3\NC 2 + x + 2\NR\stopbmatrix\startbmatrix\NC 1 \NR\NC 1
  \NR\NC x\NR\stopbmatrix = 0$

  $1 + 6 + 5x + 4x + x^2 = 0 \Rightarrow x^2 + 9x + 7 = 0 \Rightarrow x = \frac{-9 \pm\sqrt{53}}{2}$.
\item Product is $\startbmatrix\NC \cos^2\theta\cos^2\phi + \sin\theta\cos\theta\cos\phi\sin\phi\NC
  \cos^2\theta\cos\phi\sin\phi + \sin\theta\cos\theta\sin^2\phi\NR\NC \cos^2\theta\sin\theta\cos\theta +
  \sin^2\theta\cos\phi\sin\phi\NC \cos\theta\sin\theta\cos\phi\sin\phi +
  \sin^2\theta\sin^2\phi\NR\stopbmatrix$

  $= \startbmatrix\NC\cos\theta\cos\phi\cos(\theta - \phi)\NC \sin\phi\sin\theta\cos(\theta - \phi)\NR\NC
  \cos\phi\sin\theta\cos(\theta - \phi)\NC \sin\theta\sin\phi\cos(\theta - \phi)\NR\stopbmatrix$

  Clearly the above matrix is a zero matrix of the difference of angles is an odd multiple of
  $\frac{\pi}{2}$.
\stopitemize
