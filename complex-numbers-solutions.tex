% -*- mode: context; -*-
\chapter{Complex Numbers}
\startitemize[n, 2*broad]
\item Let $z = 7 + 8i$, and $\sqrt{z} = \sqrt{7 + 8i} = x + iy$. Squaring $7 + 8i = (x^2 - y^2) + 2ixy$
  Comparing real and imaginary parts $x^2 - y^2 = 7, xy = 4 \Rightarrow x^2 + y^2 = \sqrt{113}$. We discaard
  $-\sqrt{113}$ as that will make $x, y$ complex.

  $\Rightarrow x = \frac{\sqrt{7 + \sqrt{113}}}{2}, y = \frac{\sqrt{\sqrt{113} - 7}}{2}$.
\item Let $\sqrt{a^2 - b^2 + 2abi} = x + iy$, then on squaring and comparison of real and imaginary parts,
  we have $x^2 - y^2 = a^2 - b^2, xy = ab\Rightarrow x^2 + y^2 = a^2 + b^2\Rightarrow x = a, y = b$.
\item $\sqrt[4]{81i^2} = \sqrt{\pm9i}$ and now we can solve it like previous problems.
\item Let $z = \frac{x^2}{y^2} + \frac{y^2}{x^2} + \frac{1}{2i}\left(\frac{x}{y} + \frac{y}{x}\right) +
  \frac{31}{16} = \left(\frac{x}{y} + \frac{y}{x}\right)^2 - 2\frac{i}{4}\left(\frac{x}{y} +
  \frac{y}{x}\right) + \frac{i^2}{4} = \left(\frac{x}{y} + \frac{y}{x} - \frac{i}{4}\right)^2$

  $\therefore$ square root $=\pm\left(\frac{x}{y} + \frac{y}{x} - \frac{i}{4}\right)$.
\item We know that $i^4 = 1$. Let $z = i^{n + 80} + i^{n + 50} = i^{n + 4.20} + i^{n + 12.4 + 2} = i^n +
  i^{n + 2} = i^n - i^n = 0$.
\item Let $z = \left(i^{17} + \frac{1}{i^{15}}\right)^3 = \left(i^{4.4 + 1} + \frac{1}{i^{4.4 - 1}}\right)^3
  = (i + i)^3 = 8i^3 = -8i$.
\item Let $z = \frac{(1 + i)^2}{2 + 3i} = \frac{2i}{2 + 3i}.\frac{2 - 3i}{2 - 3i} = \frac{-6 + 4i}{13}$.
\item Let $z = \left(\frac{1}{1 + i} + \frac{1}{1 - i}\right)\frac{7 + 8i}{7 - 8i} = \frac{2}{1 - i^2}\frac{(7 +
  8i)(7 + 8i)}{(7 - 8i)(7 + 8i)} = \frac{2}{2}\frac{-15 + 112i}{49 + 64} = \frac{-15 + 112i}{113}$.
\item Let $z = \frac{(1 + i)^{4n + 7}}{(1 - i)^{4n - 1}} = \frac{(1 + i)^{4(n + 2) - 1}}{(1 - i)^{4n - 1}} =
  \frac{1 - i}{1 + i} = \frac{(1 - i)^2}{1 - i^2} = \frac{1 - 2i + i^2}{2} = -i$.
\item Let $z = \frac{1}{1 - \cos\theta + 2i\sin\theta} = \frac{1 - \cos\theta - 2i\sin\theta}{(1 -
  \cos\theta)^2 + 4\sin^2\theta} = \frac{1 - \cos\theta - 2i\sin\theta}{1 - 2\cos\theta + 1 +
  3\sin^2\theta} = \frac{1 - \cos\theta - 2i\sin\theta}{2 - 2\cos\theta + 3\sin^2\theta}$.
\item Let $z = \frac{(\cos x + i\sin x)(\cos y + i\sin y)}{(\cot u + i)(i + \tan v)}$. Using Euler's
  formula, we have $z = \frac{e^{ix}.e^{iy}}{\frac{e^{iu}}{\sin u}.\frac{e^{iv}}{\cos v}} = \sin u\cos
  v.e^{i(x + y - u - v)} = \sin u\cos v\cos(x + y - u - v) + i\sin u\cos v\sin(x + y - u - v)$.
\item $i^5 = i^{4 + 1} = i$.
\item $i^{67} = i^{64 + 3} = i^3 = -i[\because i^2 = -1]$.
\item $i^{-59} = \frac{1}{i^{15.4 - 1}} = i$.
\item $i^{2014} = i^{4.503 + 2} = i^2 = -1$.
\item $|a| = -a \Rightarrow \sqrt{ab} = \sqrt{|a|b}i$.
\item Let $z = i^n + i^{n + 1} + i^{n + 2} + i^{n + 3} = i^n + i.i^n - i^n - i.i^n = 0$.
\item $\displaystyle\sum_{n = 1}^{13}(i^n + i^{n + 1}) = \sum_{i=1}^{13}i^n + \sum_{i=1}^{13}i^{n + 1} = (i
  + i^2 + i^3 + \cdots + i^{13}) + (i^2 + i^3 + i^4 + \cdots + i^{14}) = i - 1$.
\item $\frac{2^n}{(1 + i)^{2n}} + \frac{(1 + i)^{2n}}{2^n} = \frac{2^n}{(1 + i^2 + 2i)^n} + \frac{(1 + i^2 +
  2i)^n}{2^n} = \frac{1}{i^n} + i^n = \frac{i^n}{i^{2n}} + i^n = i^n\left(\frac{1}{(-1)^n} + 1\right) =
  i^n[(-1)^n + 1]$.
\item Let $z = i^n + \frac{1}{i^n} = \frac{i^{2n} + 1}{i^n}$. Substituting $n = 1, 2, 3, 4$, $z = 0, \pm2$
  i.e. there exists three different solutions.
\item $4x + (3x - y)i = 3 - 6i$. Comparing real and imaginary parts, $4x = 3, 3x - y = -6\Rightarrow x =
  \frac{3}{4}\Rightarrow \frac{9}{4} - y = -6 \Rightarrow y = \frac{33}{4}$.
\item $\left(\frac{1}{3} + i\frac{7}{3}\right) + \left(4 + i\frac{1}{3}\right) - \left(-\frac{4}{3} +
  i\right) = \left(\frac{1}{3} + 4 + \frac{4}{3}\right) + i\left(\frac{7}{3} + \frac{1}{3} - 1\right) =
  \frac{17}{3} + i\frac{5}{3}$.
\item $\frac{(1 + i)x - 2i}{3 + i} + \frac{(2 - 3i)y + i}{3 - i} = i \Rightarrow [(1 + i)x - 2i](3 - i) +
  [(2 - 3i)y + i](3 + i) = i(3 + i)(3 - i)\Rightarrow (4x + 9y - 3) + i(2x - 7y - 3) = 10i$. Equating real
  and imaginary parts, $4x + 9y = 3, 2x - 7y = 13 \Rightarrow x = 3, y = -1$.
\item The multiplicative inverse is $\frac{1}{z} = \frac{1}{4 - 3i} = \frac{1}{4 - 3i}.\frac{4 + 3i}{4 + 3i}
  = \frac{4 + 3i}{25}$.
\item Let $x_1 = 2, y_1 = 3, x_2 = 1$ and $y_2 = 12$. $\therefore \frac{z_!}{z_2} = \frac{[(x_1x_2 + y_1y_2)
    + i(x_2y_1 - x_1y_2)]}{x_2^2 + y_2^2} = \frac{8 - i}{5}$.
\item $z_1 = z_2 \Rightarrow 9y^2 - 4 -10xi = 8y^2 -20i$. Equating real and imaginary parts, $9y^2 - 4 =
  8y^2 \Rightarrow y = \pm2$ and $-10x = -20 \Rightarrow x = -2 \Rightarrow z = x + iy = -2\pm2i$.
\item Let $z = x + iy$ then $|x + iy + 1| = x + iy + 2(1 + i) \Rightarrow \sqrt{(x + 1)^2 + y^2} = (x + 2) +
  i(y + 2)$. Equating real and imaginary parts, $y + 2 = 0 \Rightarrow y = -2$ and $(x + 1)^2 + y^2 = (x +
  2)^2 \Rightarrow x^2 + 2x + 5 = 4 = x^2 + 4x + 4 \Rightarrow x = \frac{1}{2} \Rightarrow z = \frac{1 -
    4i}{2}$.
\item Let $z = \frac{1 + 2i}{1 - 3i} = \frac{(1 + 2i)(1 + 3i)}{1 - (3i)^2} = \frac{1 + 3i + 2i + 6i^2}{1 +
  9} = \frac{-5 + 5i}{10} = -\frac{1}{2} + \frac{1}{2}i$

  $\Rightarrow |z| = \sqrt{\left(-\frac{1}{2}\right)^2 + \frac{1}{2^2}} = \frac{1}{\sqrt{2}}$

  $\tan\theta = \frac{\frac{1}{2}}{-\frac{1}{2}} \Rightarrow \theta = \tan^{-1}-1 = \frac{3\pi}{4}$.
\item Given, $\frac{x - 3}{3 + i} + \frac{y - 3}{3 - i} = i(3 - i)(3 + i)\Rightarrow (x - 3)(3 - i) + (y -
  3)(3 + i) = 10i\Rightarrow 3x - 9 + i(3 - x) + (3y - 9) + i(y - 3) = 10i$

  Comparing real and imaginary parts, we get $3x + 3y - 18 = 0$ and $y - x = 10 \Rightarrow x = -2, y = 8$.
\item $(1 + i)^2 = 1 + 2i - i = 2i\Rightarrow (1 + i)^{50} = (2i)^{25} = 2^{25}i^{4.6 + 1} = 2^{25}i$ Thus,
  real part will be $0$.
\item Let $z = x + iy$ then $x + iy + \sqrt{x^2 + y^2} = 2 + 8i$, Comparing real and imaginary parts, we get
  $y = 8$ and $x + \sqrt{x^2 + y^2} = 2 \Rightarrow \sqrt{x^2 + y^2} = 2 - x$

  $\Rightarrow x^2 + 64 = 4 - 4x + x^2 \Rightarrow x = -15\Rightarrow z = -15 + 8i$.
\item $S = i + 2i^2 + 3i^3 +\ldots + 100i^{100}\Rightarrow iS = i^2 + 2i^3 + \ldots + 99i^{100} + 100i^{101}$

  $\Rightarrow S(1- i) = i + i^2 + \ldots + i^{100} - 100i^{101} = \frac{i(1 - i^{101})}{1 - i} - 100i^{101}$

  $S = \frac{i(1 - i^{101})}{(1- i)^2} - \frac{100i^{101}}{1 - i}$.
\item Consider $t_1 = \frac{1}{1 + i} + \frac{1}{1 - i} + \frac{1}{-1 + i} + \frac{1}{-1 - i} = \frac{1 + i
  + 1 - i}{1^2 - i^2} + \frac{-1 + i - 1 - i}{(-1)^2 - i^2} = \frac{2}{2} + \frac{-2}{2} = 0$

  $t_2 = 2\left(\frac{1}{1 + i} + \frac{1}{1 - i} + \frac{1}{-1 + i} + \frac{1}{-1 - i}\right) = 0$

  Similarly all other terms and sum will be zero.
\item Given, $z^2 - z - 5 + 5i = 0 \Rightarrow D = (-1)^2 - 4.1.(-5 + 5i) = 21 - 20i$ and we will need
  $\sqrt{D}$

  $\sqrt{D} = \sqrt{b^2 - 4ac} = \sqrt{21 - 20i} = \pm \left[\sqrt{\frac{x^2 + y^2 + x}{2}} -
  i\sqrt{\frac{x^2 + y^2 - x}{2}}\right] = \pm(5 - 2i)$

  $z = \frac{1 + 5 - 2i}{2}$ or $z = \frac{1 - 5 + 2i}{2}\Rightarrow z = 3 - i, i - 2$

  Thus, product of real parts $= -2\times 3 = -6$
\item Given, $z^3 = -\overline{z}\Rightarrow |z|^3 = |z|\Rightarrow |z|(|z| - 1)(|z| + 1) = 0 \Rightarrow
  |z| = 0, |z| = 1 [\because |z| + 1 > 0]$

  If $|z| = 0,$ then $z = 0.$ If $|z| = 1 \Rightarrow |z|^2 = 1\Rightarrow z\overline{z} = 1\Rightarrow z^3
  + \frac{1}{z} = 0 \Rightarrow z^4 + 1 = 0,$ which has four distinct roots. Thus, given equation has five
  roots.
\item Since we have to find real roots, let $z = x,$ a real value. The given equation becomes $x^3 + ix - 1=
  0\Rightarrow x^3 = 1, x = 0$ which is not possible. So there are no real solutions.
\item Let $z = x + iy,$ then $\sqrt{x^2 + y^2} > 1,$ because point $A$ is outside circle.

  $\frac{1}{z} = \frac{x - iy}{\sqrt{x^2 + y^2}}$ so $\frac{x}{\sqrt{x^2 + y^2}}, \frac{-y}{x^2 + y^2}< 1$

  This leads to the fact that point $E$ is reciprocal of point $A$.
\item $z = (3p - 7q) + i(3q + 7p),$ which is purely imaginary, $\Rightarrow 3p - 7q = 0$

  $\Rightarrow \frac{p}{q} = \frac{7}{3} \Rightarrow \frac{p}{q} + i = \frac{7}{3} + i \Rightarrow \frac{p +
  iq}{q} = \frac{7 + 3i}{3}$

  $\Rightarrow p+ iq = 7 + 3i \Rightarrow z = 21 + 9i + 49i - 21 = 58i \Rightarrow |z|^2 = 3364$.
\item Given, $\alpha = \left(\frac{a - ib}{a + ib}\right)^2 + \left(\frac{a + ib}{a - ib}\right)^2 =
  \frac{(a - ib)^4 + (a + ib)^4}{(a - ib)^2(a + (ib)^2)}$

  $= \frac{a^4 - 4a^3.ib + 6a^2i^2b^2 - 4ai^3b^3 + b^4 + a^4 + 4a^3ib + 6a^2i^2b^2 + 4ai^3b^3 + b^4}{(a^2 +
    b^2)^2} = \frac{2a^4 - 12a^2b^2 + 2b^4}{(a^2 + b^2)^2}$, which is purely real.
\item Let $z = x + iy$ then given $|z| = 1 \Rightarrow x^2 + y^2 = 1$

  Let $\beta = \frac{z - 1}{z + 1} = \frac{(x - 1) + iy}{(x + 1) + iy} = \frac{(x - 1) + iy}{(x + 1) +
  iy}.\frac{(x + 1) - iy}{(x + 1) - iy}$

  $= \frac{x^2 - 1 + y^2 + iy(x + 1 -x + 1)}{(x + 1)^2 + y^2} = \frac{2iy}{(x + 1)^2 + y^2}$ which is purely
  imaginary.
\item Let $z = x + iy \Rightarrow x^2 + (y - 3)^2 = 9 \Rightarrow x = 3\cos\theta, y = 3\sin\theta + 3$

  $z = 3[\cos\theta + i(\sin\theta + 1)] = 3\left[\sin\left(\frac{\pi}{2} - \theta\right) + i\left(1 +
  \cos\left(\frac{\pi}{2} - \theta\right)\right)\right]$

  $= 3\left[2\sin\left(\frac{\pi}{4} - \frac{\theta}{2}\right)\cos\left(\frac{\pi}{4} -
  \frac{\theta}{2}\right) + i2\cos^2\left(\frac{\pi}{4} - \frac{\theta}{2}\right)\right]$

  $= 6\cos\left(\frac{\pi}{4} - \frac{\theta}{2}\right)\left[\sin\left(\frac{\pi}{4} -
  \frac{\theta}{2}\right) + i\cos\left(\frac{\pi}{4} - \frac{\theta}{2}\right)\right]
  = 6\cos\left(\frac{\pi}{4} - \frac{\theta}{2}\right)e^{i\left(\frac{\pi}{4} + \frac{\theta}{2}\right)}$

  $\cot\left(arg(z)\right) = \cot\left(\frac{\pi}{4} + \frac{\theta}{2}\right) = \tan\left(\frac{\pi}{4} -
  \frac{\theta}{2}\right)$

  $\frac{6}{z} = \sec\left(\frac{\pi}{4} - \frac{\theta}{2}\right)e^{-i\left(\frac{\pi}{4} +
    \frac{\theta}{2}\right)} = \sec\left(\frac{\pi}{4} - \frac{\theta}{2}\right)
  \left[\sin\left(\frac{\pi}{4} - \frac{\theta}{2} - i\cos\left(\frac{\pi}{4} -
    \frac{\theta}{2}\right)\right)\right]$

  $= \tan\left(\frac{\pi}{4} - \frac{A}{2}\right) - i \Rightarrow \cot(\arg(z)) - \frac{6}{z} = i$.
\item Let $z = r(\cos\theta + i\sin\theta) = \frac{-16}{1 + \sqrt{3}} = \frac{-16}{1 + i\sqrt{3}}.\frac{1 -
  i\sqrt{3}}{1 - i\sqrt{3}} = \frac{-16(1 - i\sqrt{3})}{1 + 3}$

  $= -4 + i4\sqrt{3}$ then $r\cos\theta = 4, r\sin\theta = 4\sqrt{3} \Rightarrow r^2 = 64 \Rightarrow r = 4,
  \cos\theta = \frac{-1}{2}, \sin\theta = \frac{\sqrt{3}}{2} \Rightarrow \theta = \frac{2\pi}{3}$

  $\Rightarrow z = 8\left(\cos\frac{2\pi}{3} + i\sin\frac{2\pi}{3}\right)$.
\item Let $z = r(\cos\theta + i\sin\theta)$ then because $arg(z) + arg(w) = \pi \Rightarrow arg(w) = \pi -
  \theta$

  $\Rightarrow w = r(-\cos\theta + i\sin\theta) = -r(\cos\theta - i\sin\theta)\;\therefore\;r =
  -\overline{w}$.
\item $x - iy = \sqrt{\frac{a - ib}{c - id}} \Rightarrow x^2 - y^2 - 2ixy = \frac{a - ib}{c - id} = \frac{(a
  - ib)(c + id)}{c^2 + d^2}\Rightarrow x^2 - y^2 - 2ixy = \frac{(ac + bd) -i(bc - ad)}{c^2 + d^2}$

  Comparing real and imaginary parts, we get $x^2 - y^2 = \frac{ac + bd}{c^2 + d^2}, 2xy = \frac{bc -
    ad}{c^2 + d^2}$

  $\Rightarrow (x^2 + y^2)^2 = (x^2 - y^2)^2 + 4x^2y^2 = \frac{(ac + bd)^2 + (bc - ad)^2}{(c^2 + d^2)} = \frac{a^2c^2 +
    b^2d^2 + b^2c^2 + a^2d^2}{(c^2 + d^2)^2} = \frac{a^2 + b^2}{c^2 + d^2}$.
\item We know that for two complex numbers $z_1$ and $z_2, |z_1| + |z_2|\geq |z_1 - z_2|$

  $|z| + |z - 2| \geq |z - (z - 2)| = |2| = 2$.  Therefore, minimum value is $2$.
\item $|z_1 + z_2 + z_3| = |(z_1 - 1) + (z_2 - 2) + (z_3 - 3) + 6\leq |z_1 - 1| + |z_2 - 2| + |z_3 - 3| + 6$

  $< 1 + 2+ 3 + 6 = 12$.  Thus, maximum value of $|z_1 + z_2 + z_3|$ is $12.$
\item $|\alpha + \beta|^2 = (\alpha + \beta)(\overline{\alpha + \beta}) = (\alpha + \beta)(\overline{\alpha}
  + \overline{\beta}) = \alpha\overline{\alpha} + \alpha\overline{\beta} + \overline{\alpha}\beta +
  \beta\overline{\beta} = |\alpha|^2 + |\beta|^2 + \alpha\overline{\beta} + \overline{\alpha}\beta$

  Similarly, $|\alpha - \beta|^2 = |\alpha|^2 + |\beta|^2 - \alpha\overline{\beta} - \overline{\alpha}\beta$

  Thus, $|\alpha|^2 + |\beta|^2 = \frac{1}{2}(|\alpha + \beta|^2 + |\alpha - \beta|^2)$
\item If $|z| = 0$ then $\sqrt{x^2 + y^2} = 0 \Rightarrow x^2 + y^2 = 0$

  Above is possible if and only if $x = 0$ and $y = 0 \Rightarrow z = 0$.
\item $\frac{z_1z_2}{\overline{z_1}} = \frac{(1 - i)(2 + 7i)}{1 + i} = \frac{2 + 7 -2i + 7i}{1 + i} =
  \frac{9 + 5i}{1 + i} = \frac{9 + 5i}{1 + i}.\frac{1 - i}{1 - i} = \frac{9 + 5 + 5i -9i}{2} = 7 -
  2i\therefore Im\left(\frac{z_1z_2}{\overline{z_1}}\right) = -2$.
\item $|z + 12 - 6i| \leq |z - i| + |12 - 5i| < 1 + 13 = 14$.
\item Given, $|z + 6| = |2z + 3|,$ let $z = x + iy \Rightarrow (x + 6)^2 + y^2 = (2x + 3)^2 + 4y^2
  \Rightarrow x^2 + 12x + 36 + y^2 = 4x^2 + 12x + 9 + 4y^2$

  $\Rightarrow 3x^2 + 2y^2 = 27 \Rightarrow x^2 + y^2 = 9 \Rightarrow |z| = 3$.
\item Given $\sqrt{a - ib} = x - iy,$ squaring we get $a - ib = x^2 - y^2 - 2ixy$. Comparing real and
  imaginary parts, we get
  $a = x^2 - y^2, b = 2xy \Rightarrow a + ib = x^2 - y^2 + 2ixy = x^2 +i^2y^2 + 2ixy\Rightarrow \sqrt{a +
    ib} = x + iy$.
\item $x_1x_2x_3\ldots\infty = \left(\cos\frac{\pi}{2} + i\sin\frac{\pi}{2}\right)\left(\cos\frac{\pi}{2^2}
  + i\sin\frac{\pi}{2^2}\right) \ldots\infty= \cos\left(\frac{\pi}{2} + \frac{\pi}{2^2} + \ldots
  \infty\right) + i\sin\left(\frac{\pi}{2} + \frac{\pi}{2^2} + \ldots \infty\right)$

  $= \cos\frac{\pi}{2}.\frac{1}{1- \frac{1}{2}} + i\sin\frac{\pi}{2}.\frac{1}{1- \frac{1}{2}} = \cos\pi +
  i\sin\pi = -1$.
\item Given, $\frac{(\cos\theta + i\sin\theta)^4}{(\sin\theta + i\cos\theta)^5} = \frac{(\cos\theta +
  i\sin\theta)^4}{i^5\left(\frac{1}{i}\sin\theta + \cos\theta\right)^5}$

  $= \frac{(\cos\theta + i\sin\theta)^4}{i(\cos\theta - i\sin\theta)^5} = \frac{(\cos\theta +
  i\sin\theta)^4}{i(\cos\theta + i\sin\theta)^{-5}}= \frac{1}{i}(\cos\theta + i\sin\theta)^9 = \sin9\theta
  -i \cos9\theta$.
\item $z = \left[\cos\frac{\pi}{6} + i\sin\frac{\pi}{6}\right]^5 + \left[\cos\frac{\pi}{6} - i\sin\frac{\pi}{6}\right]^5$

  $= \cos\frac{5\pi}{6} + i\sin\frac{5\pi}{6} + \cos\frac{5\pi}{6} - i\sin\frac{5\pi}{6} =
  2\cos\frac{5\pi}{6}\;\therefore Im(z) = 0$.
\item $z = \left(\cos\frac{\pi}{3} + i\sin\frac{\pi}{3}\right)^{\frac{3}{4}} = \left(\cos\pi +
  i\sin\pi\right)^{\frac{1}{4}},$ thus general root is $\cos\frac{2n\pi + \pi}{4} + i\sin\frac{2n\pi +
    \pi}{4}$

  Thus, substituting $n = 0, 1, 2, 3$ we find four roots and the product is

  $\left(\cos\frac{\pi}{4} + i\sin\frac{\pi}{4}\right)\left(\cos\frac{3\pi}{4} +
  i\sin\frac{3\pi}{4}\right)\left(\cos\frac{5\pi}{4} + i\sin\frac{5\pi}{4}\right)\left(\cos\frac{7\pi}{4} +
  i\sin\frac{7\pi}{4}\right)$

  $= \left(\frac{1}{\sqrt{2}} + \frac{i}{\sqrt{2}}\right)\left(\frac{-1}{\sqrt{2}} +
  \frac{i}{\sqrt{2}}\right)\left(\frac{-1}{\sqrt{2}} - \frac{i}{\sqrt{2}}\right)\left(\frac{1}{\sqrt{2}} -
  \frac{i}{\sqrt{2}}\right)$

  $= \left(-\frac{1}{2} - \frac{1}{2}\right)\left(\frac{-1}{2} - \frac{1}{2}\right) = -1.-1 = 1$.
\item Let $z_1 = r_1(\cos x + i\sin x)$ and $z_2 = r_2(\cos y + i\sin y)$ Then $(r_1\cos x + r_2\cos y)^2 +
  (r_1\sin x+ r_2\sin y)^2 = r_1^2 + r_2^2 + 2r_2r_2$

  $\Rightarrow 2r_1r_2(\cos x\cos y + \sin x\sin y) = 2r_2r_2 \Rightarrow \cos(x - y) = 1 \Rightarrow x - y
  = 0 \Rightarrow \arg(z_1) - arg(z_2) = 0$.
\item Let $z = 1 - \sin\alpha + i\cos\alpha = r(\cos\theta + i\sin\theta),$ then $r = \sqrt{(1 -
  \sin\alpha)^2 + \cos^2\alpha} = \sqrt{2 - 2\sin\alpha}$

  $\tan\theta = \frac{\cos\alpha}{1 - \sin\alpha} = \frac{1 - \tan^2\frac{\alpha}{2}}{1 +
  \tan^2\frac{\alpha}{2} - 2\tan\frac{\alpha}{2}} = \frac{1 + \tan\frac{\alpha}{2}}{1 -
  \tan\frac{\alpha}{2}} = \tan\left(\frac{\pi}{4} -\frac{\alpha}{2}\right)\Rightarrow \theta = \frac{\pi}{4}
  - \frac{\alpha}{2}$.
\item Let $z = \left[\frac{1 + \sin\frac{\pi}{8} + i\cos\frac{\pi}{8}}{1 + \sin\frac{\pi}{8} -
    i\cos\frac{\pi}{8}}\right] = \left[\frac{1 + \sin\frac{\pi}{8} + i\cos\frac{\pi}{8}}{1 +
    \sin\frac{\pi}{8} - i\cos\frac{\pi}{8}}\right]. \left[\frac{1 + \sin\frac{\pi}{8} +
    i\cos\frac{\pi}{8}}{1 + \sin\frac{\pi}{8} + i\cos\frac{\pi}{8}}\right]$

  $= \frac{\left(1 + \sin\frac{\pi}{8}\right)^2 - \cos^2\frac{\pi}{8} + 2i(1 +
  \sin\frac{\pi}{8})\cos\frac{\pi}{8}}{\left(1 + \sin\frac{\pi}{8}\right)^2 + \cos^2\frac{\pi}{8}} =
  \frac{2\sin\frac{\pi}{8} + 2\sin^2\frac{\pi}{8} + 2i(1 + \sin\frac{\pi}{8})\cos\frac{\pi}{8}}{2 +
    2\sin\frac{\pi}{8}}$

  $= \sin\frac{\pi}{8} + i\cos\frac{\pi}{8} = i\left(\cos\frac{\pi}{8} - i\sin\frac{\pi}{8}\right)
  \Rightarrow z^8 = i^8(\cos\pi - i\sin\pi) = -1$.
\item $z_1z_2z_3z_4z_5 = \cos\left(\frac{2\pi}{5} + \frac{4\pi}{5} + \frac{6\pi}{5} + \frac{8\pi}{5} +
  \frac{10\pi}{5}\right) + i\sin\left(\frac{2\pi}{5} + \frac{4\pi}{5} + \frac{6\pi}{5} + \frac{8\pi}{5} +
  \frac{10\pi}{5}\right)$

  $= \cos\frac{30\pi}{5} + i\sin\frac{30\pi}{5} = \cos6\pi + i\sin6\pi = 1$.
\item $z_n = \cos\left(\frac{1}{2n + 1} - \frac{1}{2n + 3}\right).\frac{\pi}{2} + i\sin\left(\frac{1}{2n +
  1} - \frac{1}{2n + 3}\right).\frac{\pi}{2}$

  $\therefore z_1z_2z_3\ldots\infty = \cos\left(\frac{1}{3} - \frac{1}{5} + \frac{1}{5} - \frac{1}{7} +
  \frac{1}{7} - \frac{1}{9}\ldots\infty\right).\frac{\pi}{2} + i\sin\left(\frac{1}{3} - \frac{1}{5} +
  \frac{1}{5} - \frac{1}{7} + \frac{1}{7} - \frac{1}{9}\ldots\infty\right).\frac{\pi}{2}$

  $= \cos\frac{\pi}{6} + i\sin\frac{\pi}{6}$.
\item Let $z_1 = x_1 + iy_1$ and $z_2 = x_2 + iy_2\Rightarrow |az_1 - bz_2|^2 + |bz_1 + az_2|^2 = (ax_1 -
  bx_2)^2 + (ay_1 - by_2)^2 + (bx_1 + ax_2)^2 + (by_1 + ay_2)^2$

  $= a^2x_1^2 + b^2x_2^2 -2abx_1x_2 + a^2y_1^2 + b^2y_2^2 - 2aby_1y_2 + b^2x_1^2 + a^2x_2^2 + 2abx_1x_2 +
  b^2y_1^2 + a^2y_2^2 + 2aby_1y_2 = (a^2 + b^2)(x_1^2 + y_1^2 + x_2^2 + y_2^2)= (a^2 + b^2)(|z_1|^2 +
  |z_2|^2)$.
\item Let $x = y + iz,$ then given expression becomes $\frac{A^2}{y + iz - a} + \frac{B^2}{y + iz - b} +
  \ldots + \frac{H^2}{y + iz - h} = y + iz + l$

  $\frac{A^2(y - a - iz)}{(y - a)^2 + z^2} + \frac{B(y - b - iz)}{(y - b)^2 + z^2} + \ldots + \frac{H^2(y -
    iz - h)}{(y - h)^2 + z^2} = y + iz + l$.  Comparing imaginary parts, we have
  $-iz\left[\frac{A^2}{(y - a)^2 + z^2} + \frac{B^2}{(y - a)^2 + z^2} + \ldots + \frac{H^2}{(y - a)^2 +
      z^2}\right] = iz \Rightarrow iz\left[1 + \frac{A^2}{(y - a)^2 + z^2} + \frac{B^2}{(y - a)^2 + z^2} +
    \ldots + \frac{H^2}{(y - a)^2 + z^2}\right] = 0$

  Clearly the term inside brackets is non-zero. So $z = 0$.
\item Let $2^{-x} = p,$ then $|1 + 4i - p|\leq 5 \Rightarrow (1 - p)^2 + 16 \leq 25$

  $1 - p \leq \pm3 \Rightarrow p \geq 4, -2 \Rightarrow x \geq -2\;\because p \nless 0\Rightarrow p \in [-2,
  \infty]$.
\item A unimodular number has a modulus of $1. \cos\theta + i\sin\theta = \frac{c + i}{c - i} = \frac{c +
  i}{c - i}.\frac{c + i}{c - i} = \frac{c^2 - 1 + 2ic}{c^2 + 1}$

  Comparing real and imaginary parts, $\cos\theta = \frac{c^2 - 1}{c^2 + 1} \Rightarrow c =
  \pm\cot\frac{\theta}{2}$

  and $\sin\theta = \frac{2c}{c^2 + 1} \Rightarrow c = \cot\frac{\theta}{2}, \tan\frac{\theta}{2}$. So the
  common value is $c = \cot\frac{\theta}{2}$.
\item $(z^3 + 3)^2 = -16 = 16i^2 \Rightarrow z^3 = -3 \pm 4i\Rightarrow |z^3| = 5 \Rightarrow |z| =
  5^{1/3}$.
\item $z = \frac{\sin\frac{x}{2} + \cos\frac{x}{2} - i\tan x}{1 + 2i\sin\frac{x}{2}}= \frac{\sin\frac{x}{2}
  + \cos\frac{x}{2} - i\tan x}{1 + 2i\sin\frac{x}{2}}.\frac{1 - 2i\sin\frac{x}{2}}{1 - 2i\sin\frac{x}{2}}$

  Since it is real so imaginary part of this will be $0.\Rightarrow -\tan x -2\sin\frac{x}{2}\cos\frac{c}{2}
  -2\sin\frac{x}{2}\cos\frac{x}{2} = 0$

  $2\sin\frac{x}{2}\left(\sin\frac{x}{2} + \cos\frac{x}{2}\right) +
  \frac{2\sin\frac{x}{2}\cos\frac{x}{2}}{\cos x} = 0\Rightarrow \sin\frac{x}{2} = 0 \Rightarrow x = 2n\pi$
  where $n = 0,1,2,3\ldots$

  or $\left(\sin\frac{x}{2} + \cos\frac{x}{2}\right)\cos x + \cos\frac{x}{2} = 0\Rightarrow
  \tan^3\frac{x}{2} - \tan\frac{x}{2} - 2 = 0$

  If $\alpha$ is a solution of above then the set of possible values are $x = 2n\pi + 2\alpha$. Solving the
  cubic equation is left to you.
\item Let $z_1 = x_1 + iy_1$ and $z_2 = x_2 + iy_2$ then $|z_1 + z_2|^2 + |z_1 - z_2|^2 = (x_1 + x_2)^2 +
  (y_1 + y_2)^2 + (x_1 - x_2)^2 + (y_1 - y_2)^2$

  $= 2(x_1^2 + y_1^2 + x_2^2 + y_2^2) = 2(|z_1|^2 + |z_2|^2)$.
\item Given, $x^2 - x + 1 = 0\Rightarrow x = -\omega, -\omega^2$

  $\displaystyle\sum_{n = 1}^5\left(x^n + \frac{1}{x^n}\right)^2 = \sum_{n = 1}^5\left(x^{2n} +
  \frac{1}{x^{2n}} + 2\right)$

  $= \left(x^2 + \frac{1}{x^2} + 2\right) + \left(x^4 + \frac{1}{x^4} + 2\right) + \left(x^6 + \frac{1}{x^6}
  + 2\right) + \left(x^8 + \frac{1}{x^8} + 2\right) + \left(x^{10} + \frac{1}{x^{10}} + 2\right)$

  $= (x^2 + x^4 + x^6 + x^8 + x^{10}) + \left(\frac{1}{x^2} + \frac{1}{x^4} + \frac{1}{x^6} + \frac{1}{x^8}
  + \frac{1}{x^{10}}\right) + 10$

  $= (\omega^2 + \omega^4 + \omega^6 + \omega^8 + \omega^{10}) + \left(\frac{1}{\omega^2} +
  \frac{1}{\omega^4} + \frac{1}{\omega^6} + \frac{1}{\omega^8} + \frac{1}{\omega^{10}}\right) + 10$

  $= -1 - 1 + 10 = 8$.
\item $3^{49}(x + iy) = \left[i\sqrt{3}\left(\frac{1 - i\sqrt{3}}{2}\right)\right]^{100} =
  i^{100}3^{50}(-\omega)^{100} \Rightarrow 3^{49}(x + iy) = 3^{50}.\omega$

  $x + iy = -\frac{3}{2} + \frac{3\sqrt{3}}{2}i\Rightarrow x = -\frac{3}{2}, y = \frac{3\sqrt{3}}{2}$.
\item $|z_1 + z_2|^2 = x_1^2 + x_2^2 + y_1^2 + y_2^2 + 2x_1x_2 + 2y_1y_2 = |z_1|^2 + |z_2|^2 + 2(x_1x_2 +
  y_1y_2)$

  Now, $2Re(z_1\overline{z_2}) = 2Re[(x_1 + iy_1)(x_2 - iy_2)] = 2\Re[x_1x_2 + y_1y_2 -i(x_1y_2 + x_2y_1)] =
  2(x_1x_2 + y_1y_2)$

  Similalry, $2\Re(\overline{z_1}z_2) = 2(x_1x_2 + y_1y_2)$.
\item R.H.S. = $\left|\frac{1}{z_1} + \frac{1}{z_2}\right| = \left|\frac{z_2 + z_1}{z_1z_2}\right|$

  Since $|z_1| = |z_2| = 1 \therefore |z_1z_2| = 1$ and thus $|z_1 + z_2| = \left|\frac{1}{z_1} +
  \frac{1}{z_2}\right|$.
\item Let $z = x + iy,$ then $x^2 - 4x + 4 + y^2 = 4x^2 - 8x + 4 + 4y^2 \Rightarrow 3x^2 + 3y^2 = 4x$

  $\Rightarrow 3|z|^2 = 4Re(z) \Rightarrow |z|^2 = \frac{4}{3}Re(z)$.
\item Given $\sqrt[3]{a + ib} = x + iy \Rightarrow a + ib = (x + iy)^3 = x^3 - 3xy^2 + i(3x^2y - y^3)$

  Comparing real and imaginary parts, we have $a = x^3 - 3xy^2, b = 3x^2y - y^3 \Rightarrow \frac{a}{x} =
  x^2 - 3y^2, \frac{b}{y} =3x^2- y^2$

  $\therefore \frac{a}{x} + \frac{b}{y} = 4(x^2 - y^2)$.
\item $x + iy = \sqrt{\frac{a + ib}{c + id}} \Rightarrow (x + iy)^2 = \frac{a + ib}{c + id} \Rightarrow |(x
  + iy)^2| = \left|\frac{a + ib}{c + id}\right| = \frac{|a + ib|}{|c + id|} \Rightarrow (x^2 + y^2)^2 =
  \frac{a^2 + b^2}{c + d^2}$.
\item Let $z = 1 = \cos0^\circ + i\sin0^\circ = e^{i2r\pi}\;\forall i \in N \Rightarrow \sqrt[n]{z} =
  e^{\frac{i.2r\pi}{n}}$. Clearly, $|z_k| = |z_{k + 1}| = 1$.
\item $z^n  = (z + 1)^n \Rightarrow \frac{z}{z + 1} = 1^{1/n}$

  This means $\frac{z}{z + 1}$ is $n$th root of unity. $\Rightarrow \left|\frac{z}{z + 1}\right| = 1$

  $\Rightarrow |z| = |z + 1| \Rightarrow x^2 + y^2 = x^2 + 2x + 1 + y^2 \Rightarrow x = -\frac{1}{2}
  \Rightarrow Re(z) < 0$.
\item Roots of $1 + x + x^2 = 0$ are $\omega$ and $\omega^2.$ Let $f(x) = x^{3m} + x^{3n - 1} + x^{3r - 2}$

  $f(x) = x^{3m} + \frac{x^{3n}}{x} + \frac{x^{3r}}{x^2} \Rightarrow f(\omega) = 1 + \frac{1}{\omega} +
  \frac{1}{\omega^2} = \frac{1 + \omega + \omega^2}{\omega^2} = 0$

  Similarly $f(\omega^2) = 0$. Thus, we see that $f(x)$ has same roots as $1 + x + x^2= 0.$ Hence, $f(x)$
  will be divisible by $1 + x + x^2$.
\item $\sqrt{3} + i = 2\left(\frac{\sqrt{3}}{2} + i\frac{1}{2}\right) = 2\left(\cos\frac{\pi}{6} +
  i\sin\frac{\pi}{6}\right) = 2e^{i\frac{\pi}{6}}$

  Similarly, $\sqrt{3} - i = 2e^{-i\frac{\pi}{6}}$

  Since imaginary part is what prevents equality we need to get rid of it and the least value for which it
  will happen is when argument is $\pi.$ Thus, we need to raise to the power by $6$ making $n = 6$.
\item $\sqrt{3} - i = 2.\left(\cos\frac{\pi}{6} - i\sin\frac{\pi}{6}\right)$

  Thus, $(\sqrt{3} - i)^n = 2^n \Rightarrow 2^n\left(\cos\frac{n\pi}{6} - i\sin\frac{\pi}{6}\right) = 2^n$

  $\Rightarrow \cos\frac{n\pi}{6} - i\sin\frac{n\pi}{6} = 1 \Rightarrow \frac{n\pi}{6} = 2k\pi\;\forall k\in
  I \Rightarrow n = 12k$

  Thus, $n$ is a multiple of $12$.
\item Given, $z^4 + z^3 + 2z^2 + z + 1 = 0 \Rightarrow z^2(z^2 + z + 1) + z^2 + z + 1 = 0$

  $\Rightarrow (z^2 + 1)(z^2 + z + 1) = 0$. If $z^2 + 1 = 0 \Rightarrow z = i \Rightarrow |z| = 1$

  If $z^2 + z + 1 = 0 \Rightarrow z = \omega, \omega^2 \Rightarrow |z| = 1$.
\item $\because z = \sqrt[7]{-1}\Rightarrow z^7 = -1\Rightarrow z^{86} + z^{175} + z^{289} = (z^7)^{14}.z^2
  + (z^7)^{25} + (z^7)^{41}z^2 = z^2 -1 -z^2 = -1$
\item Given, $z^3 + 2z^2 + 3z + 2 = 0\Rightarrow z^3 + z^2 + 2z + z^2 + z + 2 = 0\Rightarrow (z + 1)(z^2 + z
  + 2) = 0$

  If $z + 1 = 0 \Rightarrow z = -1,$ which is real and is of no interest for us.

  If $z^2 + z + 2 = 0 \Rightarrow z = \frac{-1 + i\sqrt{7}}{2}$ which are complex roots of the given
  equation.
\item $z = \sqrt[5]{1} \Rightarrow z^5 = 1$

  $2^{|1 + z + z^2 + z^{-2} - z^{-1}|} = 2^{|1 + z + z^2 + z^3 - z^4|}[\because z^4 = 1 \Rightarrow z^{-1} =
  \frac{z^5}{z} = z^4]$

  $= 2^{|1 + z + z^2 + z^3 + z^4 - 2z^4|} = 2^{\left|\frac{1 - z^5}{1 - z} - 2z^4\right|} = 2^{|2z^4|} = 2^2
  = 4[\because |z| = 1]$.
\item Let $S = 1 + 3z + 5z^2 + \ldots + (2n - 1)z^{n - 1}$

  $\Rightarrow zS = z + 3z^2 + 5z^3 + \ldots + (2n - 3)z^{n - 1} + (2n - 1)z^n$

  $\Rightarrow (1 - z)S = 1 + 2z + 2z^2 + 2z^3 + \ldots + 2z^{n - 1} + (2n - 1)z^n$

  $\Rightarrow (1 - z)S = 1 + 2n - 1 + 2[z + z^2 + \ldots z^{n - 1}][\because z^n = 1]$

  $= 2n + 2.-1[\because 1 + z + z^2 + \ldots + z^{n - 1} = 0] \Rightarrow S = \frac{2(n - 1)}{1 - z}$.
\item Let $z = \sqrt{-1-\sqrt{-1-\sqrt{-1-\infty}}} \Rightarrow z = \sqrt{-1 - z}$

  $\Rightarrow z^2 = -1 - z \Rightarrow z^2 + z + 1 = 0 \Rightarrow z = \frac{-1 \pm i\sqrt{3}}{2}
  \Rightarrow z = \omega, \omega^2$.
\item Given, $z = e^{\frac{i2\pi}{n}},$ which is nth root of unity.

  $\therefore x^n - 1 = (x - 1)(x - z)(x - z^2 (x - z^3) ... (x - z^{n - 1})$

  Putting $x = 11, (11 - z)(11 - z^2)\ldots(11 - z^{n - 1}) = \frac{11^n - 1}{10}$.
\item Given, $\frac{3}{2 + \cos\theta + i\sin\theta} = a + ib \Rightarrow a + ib  \frac{3(2 + \cos\theta -
  i\sin\theta)}{5 + 4\cos\theta}$

  Comparing real and imaginary parts, we get $a = \frac{6 + 3\cos\theta}{5 + 4\cos\theta}, b =
  \frac{-3\sin\theta}{5 + 4\cos\theta} \Rightarrow a^2 + b^2 = \frac{36 + 36\cos\theta + 9\cos^2\theta +
    9\sin^2\theta}{(5 + 4\cos\theta)^2}$

  $= \frac{45 + 36\cos\theta}{(5 + \cos\theta)^2} = \frac{9(5 + 4\cos\theta)}{(5 + 4\cos\theta)^2} =
  \frac{9}{5 + 4\cos\theta}, 4a - 3 = \frac{24 + 12\cos\theta - 15 - 12\cos\theta}{5 +
    4\cos\theta} = \frac{9}{5 + 4\cos\theta}\Rightarrow a^2 + b^2 = 4a - 3$.
\item Let $z = x + iy, \Rightarrow |(2x - 1) + 2iy| = |(x - 2) + iy|\Rightarrow 4x^2 - 4x + 1 + 4y^2 = x^2 -
  4x + 4 + y^2 \Rightarrow 3x^2 + 3y^2 = 3\Rightarrow x^2 + y^2 = 1\Rightarrow |z| = 1$.
\item Given, $\frac{1 -ix}{1 + ix} = m + in \Rightarrow m + in = \frac{1 - ix}{1 + ix}.\frac{1 - ix}{1 - ix}$

  $m + in = \frac{1 - x^2 - 2ix}{1 + x^2}$, Comparing real and imaginary parts, $m = \frac{1 - x^2}{1 +
  x^2}, n = \frac{-2x}{1 + x^2}$

  $\Rightarrow m^2 + n^2 = \frac{(1 - x^2)^2 + 4x^2}{(1 + x^2)^2} = 1$.
\item We know that the equation of a straight line is given by $\startbmatrix\NC z \NC \overline{z} \NC
  1\NR\NC z_1 \NC \overline{z_1} \NC 1 \NR\NC z_2 \NC \overline{z_2} \NC 1\NR\stopbmatrix = 0$

  $\Rightarrow z(\overline{z_1} - \overline{z_2}) - \overline{z}(z_1 - z_2) + z_1\overline{z_2} -
  \overline{z_1}z_2 = 0$

  $\Rightarrow z(1 + i - 1 - i) - \overline{z}(1 + i -1 + i) + (1 + i)^2 - (1 - i)^2 = 0\Rightarrow z +
  \overline{z} - 2 = 0$.
\item Given, $5z_1 - 13z_2 + 8z_3 = 0 \Rightarrow z_2 = \frac{5z_1 + 8z_3}{5 + 8}$

  This means $z_1$ divides the line segment joining $z_1$ and $z_2$ in the ratio of $5:8$ which also implies
  that these three points are collinear. Thus, $\startbmatrix\NC z_1 \NC \overline{z_1} \NC 1\NR\NC z_2 \NC
  \overline{z_2} \NC 1\NR\NC z_3 \NC \overline{z_3} \NC 1\NR\stopbmatrix = 0$
\item We know that length of perpendicular from $z_1$ to $\overline{a}z + a\overline{z} + b = 0$ is given by
  $\frac{|\overline{a}z_1 + a\overline{z_1} + b|}{2|a|}.$

  Thus desired length $= \frac{|(2 - 3i)(3 + 4i) + (2 + 3i)(3 - 4i) + 9|}{2|3 - 4i|} = \frac{45}{10} =
  \frac{9}{2}$.
\item
  \startplacefigure[location={left, none}]
    \startMPcode
      draw (-1cm, 0cm) -- (1cm, 0cm);
      draw (0cm, -1cm) -- (0cm, 1cm);
      label.bot("$z_1$", (0cm, -1cm));
      label.top("$z_2$", (0cm, 1cm));
      label.urt("$\frac{z_1 + z_2}{2}$", (0cm, 0cm));
      label.lrt("$b\overline{z} + \overline{b}z = c$", (0cm, 0cm));
      pickup pencircle scaled 2pt;
      drawdot (0cm, 1cm);
      drawdot (0cm, -1cm);
    \stopMPcode
  \stopplacefigure
  Since mid-point lies on the given line, therefore $b\left(\frac{\overline{z_1} + \overline{z_2}}{2}\right) +
  \overline{b}\left(\frac{z_1 + z_2}{2}\right) = c$

  Since line segment joining $z_1$ and $z_2$ is perpedicular to the given line therefore, Slope of $z_1z_2$ + Slope of line = 0

  $\Rightarrow \frac{z_2 - z_1}{\overline{z_2} - \overline{z_1}} - \frac{b}{\overline{b}} = 0$

  Solving these two equations, we get $\overline{b}z_2 + b\overline{z_1} = c$.
\item   Let $z = 2 - i$ then after rotation new point would be $z.e^{i\pi/2} = (2 - i)\left(\cos\frac{\pi}{2} +
  i\sin\frac{\pi}{2}\right) = (2 - i)i = 1 + 2i$.
\item Coordinate of $z_0$ after moving $5$ points horizontally and $3$ points vertically away from starting pont would be
  $6 + 5i$.

  It then moves in the direction of vecor $\hat{i} + \hat{j}$ for $\sqrt{2}$ units. This vector makes angle $\pi/4$ with
  $x$-axis. So new coordinate would be $6 + \sqrt{2}\cos\pi/4 + 5 + \sqrt{2}\sin\pi/4 = 7 + 6i$.

  It then rotates by angle $\pi/2$ so new coordinate would be $(7 + 6i)e^{i\pi/2} = (7 + 6i)i = -6 + 7i$.
\item North-East direction makes angle of $\pi/4$ with $x$-axis. So coordinates of point $3$ units from origin in
  North-East direction $= 3.e^{i\pi/4} = 3\left(\cos\frac{\pi}{4} + i\sin\frac{\pi}{4}\right) = \frac{3}{\sqrt{2}} +
  i\frac{3}{\sqrt{2}}$.

  North-West direction makes angle of $3\pi/4$ with $x$-axis. A disaplacement of $4$ units in this direction will mean a shift in
  coordinates by $4.e^{i3\pi/4} = 4\left(\cos\frac{3\pi}{4} + i\sin\frac{3\pi}{4}\right) = -\frac{4}{\sqrt{2}} +
  i\sin\frac{4}{\sqrt{2}}$.

  Thus, final coordiate would be sum of the above two i.e. $-\frac{1}{\sqrt{2}} + i\frac{7}{\sqrt{2}}$.
\item Given, $\frac{z_1 - z_3}{z_2 - z_3} = \frac{1 - i\sqrt{3}}{2} = \frac{1 - i\sqrt{3}}{2}.\frac{1 +
    i\sqrt{3}}{2}$

  $= \frac{1 + 3}{2(1 + i\sqrt{3})}= \frac{2}{1 + i\sqrt{3}}$

  $\Rightarrow \frac{z_2 - z_3}{z_1 - z_3} = \frac{1 + i\sqrt{3}}{2} = \cos\frac{\pi}{3} + i\sin\frac{\pi}{3}$

  $\Rightarrow \left|\frac{z_2 - z_3}{z_1 - z_3}\right| = 1$ and $\arg\left(\frac{z_2 - z_3}{z_1 -
  z_3}\right) = \frac{\pi}{3}$

  Hence, the triangle is equilateral.
\item Since sides of an equilateral triangle make an angle of $60^\circ$ with each other, therefore
  $\frac{z_3 - z_1}{z_2 - z_1} = \cos60^\circ \pm \sin60^\circ = \frac{1 \pm i\sqrt{3}}{2}$

  $\Rightarrow 2z_3 - 2z_1 + z1 - z_2 = \pm i(z_2 - z_1)\sqrt{3}\Rightarrow (2z_3 - z_1 - z_2)^2 = 3(z_2 -
  z_1)^2\Rightarrow z_1^2 + z_2^2 + z_3^2 = z_1z_2 + z_2z_3 +z_3z_1$

  $\Rightarrow z_1z_2 + z_2z_3 + z_3z_1 - z_z^2 - z_2^2 - z_3^2 + z_1z_2 - z_1z_2 + z_2z_3 - z_2z_3 + z_1z_3
  - z_1z_3 = 0$

  $\Rightarrow (z_1 - z_2)(z_2 - z_3) + (z_2 - z_3)(z_3 - z_1) + (z_3 - z_1)(z_1 - z_2) =
  0\Rightarrow \frac{1}{z_1 - z_2} + \frac{1}{z_2 - z_3} + \frac{1}{z_3 - z_1} = 0$.
\item Since it is an equilateral triangle, therefore centroid and circumcenters would be
  identical. $\therefore z_0 = \frac{z_1+ z_2 + z_3}{3}$

  Since it is an equilateral triangle, we have just proven that $z_1^2 + z_2^2 + z_3^2 = z_1z_2 + z_2z_3
  +z_3z_1$

  From first equation, we have $\Rightarrow 9z_0^2 = z_1^2 + z_2^2 + z_3^2 + 2(z_1z_2 + z_2z_3 +z_3z_1)$

  $\Rightarrow 9z_0^2 = z_1^2 + z_2^2 + z_3^2 + 2(z_1^2 + z_2^2 + z_3^2)\Rightarrow 3z_0^2 = z_1^2 + z_2^2 +
  z_3^2$.
\item Since right angle is at $z_3,$ therefore $\frac{z_2 - z_3}{z_1 - z_3} = e^{i\pi/2} = i\Rightarrow (z_2
  - z_3)^2 = -(z_1 - z_3)^2 \Rightarrow z_2^2 + z_3^2 - 2z_2z_3 = -z_1^2 - z_3^2 + 2z_1z_3$

  $\Rightarrow z_1^2 + z_2^2 - 2z_1z_2 = -2z_3^2 + 2z_2z_3 + 2z_1z_3 - 2z_1z_2\Rightarrow (z_1 - z_2)^2 =
  2(z_1 - z_3)(z_3 - z_2)$.
\item Clearly, $|z - z_0|^2 = r^2 \Rightarrow (z - z_0)(\overline{z - z_0}) = r^2\Rightarrow (z -
  z_0)(\overline{z} - \overline{z_0}) = r^2$

  $\Rightarrow z\overline{z} - \overline{z}z_0 - z\overline{z_0} + z_0\overline{z_0} = r^2$.
\item Given, $z = 1 - t + i\sqrt{t^2 + t + 2};$ comparing real and imaginary parts, we get $x = 1 - t, y =
  \sqrt{t^2 + t + 1} \Rightarrow y^2 = t^2 + t + 2$

  $\Rightarrow y^2 = (1 - x)^2 + (1 - x) + 2 = \left(x - \frac{3}{2}\right)^2 + \frac{7}{4}$, which is
  equation of a hyperparabola.
\item Given, $\overline{z} = \overline{a} + \frac{r^2}{z - a}\Rightarrow (\overline{z} - \overline{a})(z -
  a) = r^2$, which is equation of a circle with center at $a$ and radius $r.$
\item Since $z_1$ and $z_2$ are ends of diameter $\Rightarrow |z - z_1|^2 + |z - z_2|^2 = |z_1 -
  z_2|^2\Rightarrow k = |z_1 - z_2|^2 = |2 + 3i - 4 - 3i|^2 = 4$.
\item $z = x + iy,$ then $|(x + 1) + iy| = \sqrt{2}|(x - 1) + iy|$

  Squaring both sides, we get $(x + 1)^2 + y^2 = 2[(x - 1)^2 + y^2] \Rightarrow x^2 + y^2 - 6x + 1 = 0$,
  which is equation of a circle.
\item Given, $\left|\frac{z - 1}{z - i}\right| = 1 \Rightarrow |z - 1| = |z - i|$

  Let $z = x + iy,$ then we have $|(x - 1) + iy| = |x + i(y - 1)|$

  Squaring both sides, we get $\Rightarrow (x - 1)^2 + y^2 = x^2 + (y - 1)^2 \Rightarrow 2x = 2y \Rightarrow
  x = y$, which is equation of a straight line.
\item \startplacefigure[location={left, none}]
    \startMPcode
      draw fullcircle scaled 2cm;
      draw((.707cm, .707cm) -- (.707cm, -.707cm));
      draw((-.707cm, .707cm) -- (-.707cm, -.707cm));
      draw((.707cm, .707cm) -- (-.707cm, .707cm));
      draw((-.707cm, -.707cm) -- (.707cm, -.707cm));
      label.llft("$A(z_1)$", (-.707cm, -.707cm));
      label.lrt("$B(z_2)$", (.707cm, -.707cm));
      label.urt("$C(z_3)$", (.707cm, .707cm));
      label.ulft("$D(z)$", (-.707cm, .707cm));
    \stopMPcode
  \stopplacefigure
  $\angle z_1 = \arg\left(\frac{z_1 - z_2}{z_1 - z_4}\right),$
  $\angle z_2 = \arg\left(\frac{z_3 - z_2}{z_1 - z_2}\right),$
  $\angle z_3 = \arg\left(\frac{z_3 - z_4}{z_3 - z_2}\right),$ and
  $\angle z_4 = \arg\left(\frac{z_1 - z_4}{z_3 - z_4}\right)$

  $\angle z_1 + \angle z_3 = \pi \Rightarrow \arg {\frac {z_1 - z_2} {z_1 - z_4} } + \arg \left( {\frac {z_3
      - z_4} {z_3- z_2} }\right) = \pi$

  $\Rightarrow \arg\left(\frac{(z_1 - z_2)(z_3 - z_4)}{(z_1 - z_4)(z_3 - z_2)}\right) = \pi$
  $\Rightarrow \frac{(z_1 - z_2)(z_3 - z_4)}{(z_1 - z_4)(z_3 - z_2)}$ is real number.
\item Given, $\frac{2}{z_1} = \frac{1}{z_2} + \frac{1}{z_3}\Rightarrow \frac{z_2 - z_1}{z_3 - z_1} =
  -\frac{z_2}{z_3}\Rightarrow \arg\left(\frac{z_2 - z_1}{z_3 - z_1}\right) = \pi - \arg\frac{z_3}{z_2}$

  $\Rightarrow \arg\left(\frac{z_2 - z_1}{z_3 - z_1}\right) + arg\left(\frac{z_3 - 0}{z_2 - 0}\right) = \pi$
  Thus, the given points and the origin are concyclic.
\item From the equation of circle, $r^2 = |\omega - \omega^2|^2\Rightarrow r^2 = |i\sqrt{3}|^2 = 3
  \Rightarrow r = \sqrt{3}$.
\item Let $z = x + iy\Rightarrow (x - 4)^2 + y^2 < (x - 2)^2 + y^2 \Rightarrow x^2 - 8x + 16 < x^2 - 4x +
  4\Rightarrow 4x > 12 \Rightarrow x > 3$.
\item Given, $2z_1 - 3z_2 + z_3 = 0\Rightarrow z_2 = \frac{2z_1 + z_3}{3} = \frac{2z_1 + z_3}{2 + 1}$

  Thus, $z_1$ divides the line segement $z_1z_3$ in the ratio of $2:1$ i.e. all three points are collinear.
\item Given, $|z + 1| = |z - 1| \Rightarrow (x + 1)^2 + y^2 = (x - 1)^2 + y^2 \Rightarrow x = 0$

  Also, given that $\arg\frac{z - 1}{z + 1} = \frac{\pi}{4}\Rightarrow z - 1 = (z + 1)e^{i\pi/4} \Rightarrow
  -1 + iy = (1 + iy)\left(\cos\frac{\pi}{4} + i\sin\frac{\pi}{4}\right)$

  $\Rightarrow -1 + iy = (1 + iy)\left(\frac{1}{\sqrt{2}} + i\frac{1}{\sqrt{2}}\right)\Rightarrow y =
  \sqrt{2} + 1$.
\item Given, $|z|^8 = |z - 1|^8 \Rightarrow |z| = |z - 1|, \Rightarrow x^2 + y^2 = (x - 1)^2 + y^2
  \Rightarrow x = \frac{1}{2}, y\in(\infty, \infty)$,
  which is equation of straight line parallel to $y$-axis at $x = 1/2.$
\item Given, $z\overline{z} + a\overline{z} + \overline{a}z + b = 0$
  $\Rightarrow z\overline{z} + a\overline{z} + \overline{a}z + a\overline{a} = a\overline{a} - b$

  $(z + a)(\overline{z} + \overline{a}) = |a|^2 - b$,
  which is equation of a circle if $|a|^2 - b > 0 \Rightarrow |a|^2 > b$.
\item Let $z = x + iy,$ comparing real and imaginary part gives us
  $x = \lambda + 3, y = \sqrt{3 - \lambda^2} \Rightarrow y^2 = 3 - \lambda^2$

  $\Rightarrow (x - 3)^2 + y^2 = 3$,
  which is equation of a circle with center $(3, 0)$ and radius $\sqrt{3}$.
\item Let $z = x + iy,$ then $|Re(z)| + |Im(z)| = k$ will give us four equations. $x + y = k, x - y = k, -x + y = k$ and $-x - y = k$

  These lines will intersect at $(k, 0), (0, k), (-k, 0), (0 -k)$ giving us a square as locus of $z.$
\item $z_2 = z_1^2 + i = i, z_3 = z_2^2 + i = i - 1, z_4 = z_3^2 + i = (i - 1)^2 + i = -i, z_5 = z_4^2 + i =
  i - 1, z_6 = z_5^2 + i = -i$

  Thus, we see that it is a cycle between $-i$ and $i - 1$ starting at $z_3.$
  $\Rightarrow z_{111} = z_3 = i - 1 \Rightarrow |z_{111}| = \sqrt{2}$
\item Given, $z\overline{z}^3 + z^3\overline{z} = 350 \Rightarrow z\overline{z}(\overline{z}^2 + z^2) = 350$

  Let $z = x + iy,$ then given equation becomes $2(x^2 + y^2)(x^2 - y^2) = 350 \Rightarrow (x^2 + y^2)(x^2 -
  y^2) = 175$

  Prime factors of $175$ are $5, 5, 7$ so the only solution which yields integers for $x$ and $y$ are $x^2 +
  y^2 = 25, x^2-y^2 = 7$

  $\Rightarrow x = \pm 4, y = \pm 3$ which gives a rectangle with four points and digonal with a length of
  $10$ units.
\item We know that $z_1 + z_2$ and $z_1 - z_2$ are the diagonals of a quadrilateral. Now diagonals of a
  parallelogram does not intersect at angle $\pi/2$ and diagonals of a square and rectangle are equal. Only
  rhombus satisfies the given criteria of diagonals meeting at right angle and having different lengths.
Thus, the given conditions represent a rhombus but not a square.
\item Let $\arg(z_1) = \theta, \arg(z_2) = \theta + \alpha \Rightarrow \frac{az_1}{bz_2} =
  \frac{a|z_1|e^{i\theta}}{b|z_2|e^{i(\theta + \alpha)}} = e^{-i\alpha}$

  $\Rightarrow \frac{bz_2}{az_1} = e^{i\alpha}\Rightarrow \frac{az_1}{bz_2} + \frac{bz_2}{az_1} =
  e^{i\alpha} + e^{-i\alpha} = 2\cos\alpha$

  Thus, it will lie on the line segment $[-2, 2]$ of the real axis.
\item Since $z_1, z_2, z_3$ are roots of the equation $z^3 + 3\alpha z^2 + 3\beta z + \gamma = 0\Rightarrow
  z_1 + z_2 + z_3 = -3\alpha, z_1z_2 + z_2z_3 + z_3z_1 = 3\beta, z_1z_2z_3 = \gamma$

  We know that for a triangle to be equilateral $z_1^2 + z_2^2 + z_3^2 = z_1z_2 + z_2z_3 + z_3z_1$

  $\Rightarrow (z_1 + z_2 + z_3)^2 = 3(z_1z_2 + z_2z_3 + z_3z_1)\Rightarrow 9\alpha^2 = 3.3\beta \Rightarrow
  \alpha^2 = \beta$.
\item Given, $z_1^2 + z_2^2 + 2z_1z_2\cos\theta = 0$ Dividing both sides with $z_2^2,$ we get
  $\left(\frac{z_1}{z_2}\right)^2 + 1 + 2\frac{z_1}{z_2}\cos\theta = 0$

  The above equation is a quadratic equation in $\frac{z_1}{z_2}, \therefore \frac{z_1}{z_2} =
  \frac{-2\cos\theta \pm\sqrt{4\cos^2\theta - 1}}{2}$

  $\Rightarrow \frac{z_1}{z_2} = -\cos\theta \pm i\sin\theta \Rightarrow \left|\frac{z_1}{z_2}\right| =
  1\Rightarrow |z_1| = |z_2| \Rightarrow |z_1 - 0| = |z_2 - 0|$

  Thus, $z_1, z_1$ and the origin form an isosceles triangle.
\item Since origin is circumcenter $\Rightarrow |z_1| = |z_2| = |z_3| = |z|$
  $\Rightarrow z_1\overline{z_1} = z_2\overline{z_2} = z_3\overline{z_3} = z\overline{z}$

  $\because\;AP\perp BC \therefore\;\frac{z - z_1}{\overline{z} - \overline{z_1}} + \frac{z_2 - z_3}{\overline{z_2} -
    \overline{z_3}} = 0$
  $\Rightarrow \frac{z - z_1}{\frac{z\overline{z_1}}{z} - \overline{z_1}} + \frac{z_2 - z_3}{\frac{z_3\overline{z_3}}{z} -
    \overline{z_3}} = 0$

  $\Rightarrow \frac{z(z - z_1)}{z_1\overline{z_1} - z\overline{z_1}} + \frac{z_2(z_2 - z_3)}{z_3\overline{z_3} -
    z_2\overline{z_3}} = 0$
  $\Rightarrow \frac{-z(z_1 - z)}{\overline{z_1}(z_1 - z)} - \frac{z_2(z_3 - z_2)}{\overline{z_3}(z_3 - z_2)} = 0$
  $\Rightarrow \frac{-z}{z_1} - \frac{z_2}{z_3} = 0 \Rightarrow z = -\frac{z_1z_2}{z_3}$.
\item Given $OA = OB, \Rightarrow |z_1| = |z_2| = l$ (let).
  Also given, $\arg(z_1) = \alpha + \arg(z_2) \Rightarrow z_1 = le^{i(\alpha + \arg(z_2))} = le^{i\arg(z_2)}.e^{i\alpha} =
  z_2e^{i\alpha}$

  Now, $z_1z_2 = q \Rightarrow z_2^2e^{i\alpha} = q$ and $z_1 + z_2 = -p \Rightarrow z_2(1 + e^{i\alpha}) = -p$
  $\Rightarrow 2z_2\cos\frac{\alpha}{2}.e^{i\alpha/2} = -p \Rightarrow p^2 = 4z_2^2\cos^2\frac{\alpha}{2}.e^{i\alpha}$
  $\Rightarrow p^2 = 4q\cos^2\frac{\alpha}{2}$.
\item Let $z + iy,$ then $\Re\left(\frac{z + 4}{2x - i}\right) = \Re\left(\frac{x + 4 + iy}{2x + i(2y - 1)}\right)$
  $\Rightarrow \Re\left(\frac{[(x + 4) + iy][(2x - i(2y - 1))]}{4x^2 + (2y - 1)^2}\right) = \frac{1}{2}$

  $\Rightarrow \frac{2x(x + 4) + y(2y - 1)}{4x^2 + (2y - 1)^2} = \frac{1}{2} \Rightarrow 16x + 2y - 1= 0$,
  which is equation of a straight line.
\item Since the circle is inscribed in $|z| = 2$ so center is origin. Also, since $z_1, z_2$ and $z_3$ are in
  clockwise direction $z_2 = z_1e^{-i120^\circ}, z_3 = z_2e^{-i120^\circ}$

  $\Rightarrow z_2 = (1 + \sqrt{3}i)[(\cos. -120^\circ + i.\sin -120^\circ)] = 1-\sqrt{3}i$
  $\Rightarrow z_3 = -2$.
\item Given $z_1 = \frac{a}{1 - i} \Rightarrow z_1 = \frac{a + ia}{2}, z_2 = \frac{b}{2 + i} = \frac{2b - ib}{5}$
  Also given, $z_1 - z_2 = 1 \Rightarrow 5a + i5a - 4b + i2b = 10$

  Comparing real and imaginary parts, we get $5a - 4b = 10, 5a + 2b = 0 \Rightarrow a = \frac{2}{3}, b = -\frac{5}{3}$
  Cnetroid is $\frac{z_1 + z_2 + z_3}{3} = \frac{1}{3}(1 + 7i)$.
\item From the quadratic equation we have $z_1 + z_2 = -1$ and $z_1z_2 = \frac{\lambda}{2}$.
  Since $0, z_1, z_2$ form an equilateral triangle, $\Rightarrow z_1z_2 + z_2.0 + z_1.0 = z_1^2 + z_2^2 + 0^2$

  $\Rightarrow (z_1 + z_2)^2 = 3z_1z_2 \Rightarrow (-1)^2 = 3.\frac{\lambda}{2}\Rightarrow \lambda =
  \frac{2}{3}$.
\item Let $A, B, C$ represent $a,b,c$ and $U, V, W$ represent $u, v, w.$
  $\Rightarrow AB = b - c, BC = c - b = (a - b)(1 - r), CA = a - c = r(a - b)$

  $\Rightarrow UV = v - u, VW = w - v = (u - v)(1 - r), WU = u - w = r(u - v)$
  $\Rightarrow \frac{AB}{UV} = \frac{BC}{VW} = \frac{CA}{WU}$
  Thus, the triangles are similar.
\item Let $z_1$ and $z_2$ be points on real axis which circle cuts with. Since these are on real axis and if $z$
  represents this points then $z = \overline{z}[\because z = x + i.0]$

  Substituting $z = \overline{z}$ in the equation of the circle, we get $z^2 + (\overline{\alpha} + \alpha)z + r = 0$
  Since $z_1, z_2$ are the roots $\therefore z_1 + z_2 = -\alpha, z_1z_2 = r$

  Length of intercept $=|z_1 - z_2| = \sqrt{(z_1 - z_2)^2} = \sqrt{(z_1 + z_2)^2 - 4z_1z_2} = \sqrt{(\overline{\alpha} + \alpha)^2
    - 4r}$.
\item Clearly, $a = e^{i\alpha}, b = e^{i\beta}, c= e^{i\gamma}$.
  Also given, $\frac{a}{b} + \frac{b}{c} + \frac{c}{a} = 1\Rightarrow e^{i(\alpha - \beta)} + e^{i(\beta - \gamma)} + e^{i(\gamma -
    \alpha)} = 1$.

  Comparing real parts, we get $\cos(\alpha - \beta) + \cos(\beta - \gamma) + \cos(\gamma - \alpha) = 1$.
\item Let $A(z_1), B(z_2)$ be the centers of given circles and $P$ be the center of the variable circle which
  touches given circles externally, then

  $|AP| = a + r$ and $|BP| = b + r$ where $r$ is the radius of the variable circle. Clearly,
  $|AP| - |BP| = a - b \Rightarrow ||AP| - |BP|| = |a - b| = $a constant.

  Hence, locus of $P$ is a right bisector if $a = b,$ a hyperbola if $|a - b| < |AB|$ an empty set of $|a - b|>|AB|,$ set of all
  points on line $AB$ except those which lie between $A$ and $B$ if $|a - b| = |AB|\neq 0.$
\item Let $a + ib = re^{i\theta}, r^2 = a^2 + b^2 \Rightarrow a - ib = e^{-i\theta}, \tan\theta = \frac{b}{a}$
  $\frac{a - ib}{a + ib} = e^{-2i\theta} \Rightarrow i\log\left(\frac{a - ib}{a + ib}\right) = i\log e^{-2i\theta} = 2\theta$

  $\Rightarrow \tan\left[i\log\left(\frac{a - ib}{a + ib}\right)\right] = \tan2\theta = \frac{2\tan\theta}{1 - \tan^2\theta}$
  $= \frac{2b/a}{1 - b^2/a^2} = \frac{2ab}{a^2 - b^2}$.
\item Given, $|z_1| = |z_2| = 1\Rightarrow a^2 + b^2 = c^2 + d^2 = 1$
  $\Re(z_1\overline{z_2}) = 0 \Rightarrow \Re[(a + ib)(c - id)] = 0 \Rightarrow ac + bd = 0$

  $a^2 + b^2 = c^2 + d^2 \Rightarrow (a + ic)^2 = (d - ib)^2[\because ac = =bd] \Rightarrow a + ic = d - ib
  or -d + ib$ $\Rightarrow a = d$ and $c = -b$ or $a = -d, c = b$

  $\Rightarrow a^2 + c^2 = b^2 + d^2 = 1 \Rightarrow |w_1| = |w_2| = 1$
  $\Rightarrow \Re(w_1\overline{w2}) = \Re[(a + ic)(b - id)] = ab + cd = 0$.
\item Let $z_1 = r(\cos\theta + i\sin\theta)$. Given, $\left|\frac{z_1}{z_2}\right| = 1$
  $\Rightarrow |z_1| = |z_2| = r$. Also given, $\arg(z_1z_2) = 0 \Rightarrow \arg(z_1) + \arg(z_2) = 0$

  $\Rightarrow \arg(z_2) = -\theta \Rightarrow z_2 = r[\cos(-\theta) + i\sin(-\theta)] = r[\cos\theta -
  i\sin\theta] = \overline{z_1}$
  $\Rightarrow \overline{z_2} = z_1 \Rightarrow |z_2|^2 = z_1z_2$.
\item $t_n = (n + 1)\left(n + \frac{1}{\omega}\right)\left(n + \frac{1}{\omega^2}\right)$
  $= n^3 + n^2\left(1 + \frac{1}{\omega} + \frac{1}{\omega^2}\right) + n\left(1 + \frac{1}{\omega} +
  \frac{1}{\omega^2}\right) + 1$

  $= n^3 + n^2(1 + \omega + \omega^2) + n(1 + \omega + \omega^2) + 1 = n^3 + 1$
  $\displaystyle\therefore S_n = \sum_{i = 1}^nt_i = \sum_{i = 1}(i^3 + 1) = \frac{n^2(n + 1)^2}{4} + 1$.
\item Given $|z_1 + iz_2| = |z_1 - iz_2|$
  $\Rightarrow (z_1 + iz_2)(\overline{z_1} - i\overline{z_2}) = (z_1 - iz_2)(\overline{z_1 +
  i\overline{z_2}})$

  $\Rightarrow \overline{z_1}z_2 = z_1\overline{z_2} \Rightarrow \frac{z_1}{z_2} =
  \frac{\overline{z_1}}{\overline{z_2}}$.
  Thus, $\frac{z_1}{z_2}$ is purely real.
\item $z = -2 + 2\sqrt{3}i = 4\omega$
  $\Rightarrow z^{2n} + 2^{2n}z^n + 2^{4n} = 4^{2n}[\omega^{2n} + \omega^n + 1]$

  The above expression has value of $0$ if $n$ is not a multiple of $3$ and $3.4^{2n}$ if $n$ is multiple of
  $3$.
\item $x + \frac{1}{x} = 2\cos\theta, \Rightarrow x^2 - 2\cos\theta x + 1 = 0$
  $\Rightarrow x = \frac{2\cos\theta \pm \sqrt{4\cos^2\theta - 1}}{2} = \cos\theta \pm i\sin\theta = e^{\pm
  i\theta}$

  Similarly, $y = e^{\pm i\phi}\Rightarrow \frac{x}{y} + \frac{y}{z} = 2\cos(\theta - \phi)$
  and $xy + \frac{1}{xy} = 2\cos(\theta + \phi)$.
\item Given, $|z_1| = |z_2|, \Re(z_1) > 0$ and $\Im(z_1) < 0$
  $\Re\left(\frac{z_1 + z_2}{z_1 - z_2}\right) = \frac{1}{2}\left(\frac{z_1 + z_2}{z_1 - z_2} +
  \frac{\overline{z_1} + \overline{z_2}}{\overline{z_1} - \overline{z_2}}\right)$

  $= \frac{1}{2}\left(\frac{2(|z_1|^2 - |z_2|^2)}{|z_1 - z_2|^2}\right) = 0$
  Thus, $\frac{z_1 + z_2}{z_1 - z_2}$ is purely imaginary.
\item Given, $\frac{AB}{BC} = \sqrt{2} \Rightarrow \frac{z_1 - z_2}{z_3 - z_2} = \frac{|z_1 - z_2|}{|z_3 -
  z_2|}.e^{i\pi/4}$

  $= \frac{AB}{BC}.e^{i\pi/4} = \sqrt{2}\left(\frac{1}{\sqrt{2}} + \frac{i}{\sqrt{2}}\right) = 1 + i$
  $\Rightarrow z_1 - z_2 = (1 + i)(z_3 - z_2) \Rightarrow z_2 = z_3 + i(z_1 - z_3)$.
\item Given, $z_1(z_1^2 - 3z_2^2) = 2$ and $z_2(3z_1^2 - z_2^2) = 11$
  $\Rightarrow z_1^3 - 3z_1z_2^2 + iz_2(3z_1^2 - z_2^2) = 2 + 11i \Rightarrow (z_1 + iz_2)^3 = 2 + 11i$, and

  $\Rightarrow z_1^3 - 3z_1z_2^2 - iz_2(3z_1^2 - z_2^2) = 2 - 11i \Rightarrow (z_1 - iz_2)^3 = 2 - 11i$

  Multiplying above equations, we get
  $(z_1^2 + z_2^2)^3 = 4 + 121 = 125 \Rightarrow z_1^2 + z_2^2 = 5$.
\item Given $\sqrt{1 - c^2} = nc - 1 \Rightarrow 1 - c^2 = n^2c^2 - 2nc + 1 \Rightarrow \frac{c}{2n} =
  \frac{1}{1 + n^2}$

  $\frac{c}{2n}(1 + nz)\left(1 + \frac{n}{z}\right) = \frac{1}{1 + n^2}\left[1 + n^2 + n\left(z +
    \frac{1}{z}\right)\right]$

  $= \frac{1}{1 + n^2}\left[1 + n^2 + 2\cos\theta + n\right] = 1 + \frac{2n}{1 + n^2}\cos\theta = 1 +
  c\cos\theta$.
\item If $P(z)$ is any point of the ellipse, then equation of ellipse is given by
  $|z - z_1| + |z- z_2| = \frac{|z_1 - z_2|}{e}$

  If we put $z_1$ or $z_2$ in the above equation then L.H.S. becomes $|z_1 - z_2|$.
  Thus, for any interior point of the ellipse, we have $|z - z_1| + |z - z_2| < \frac{|z_1 - z_2|}{e}$

  If $P(z)$ lies on the ellipse, we have $|z - z_1| + |z- z_2| = \frac{|z_1 - z_2|}{e}$.
  It is given that origin is an internal point, so
  $|0 - z_1| + |0 - z_2| < \frac{|z_1 - z_2|}{e}$
  $\Rightarrow e\in\left(0, \frac{|z_1 - z_2|}{|z_1| + |z_2|}\right)$.
\item Let $z = x + iy$, then we have
  $|(x - 2) + i(y - 1)| = |z|\left|\frac{1}{\sqrt{2}}\cos\theta - \frac{1}{\sqrt{2}}\sin\theta\right|$
  where, $\theta = \arg(z)$

  $\Rightarrow \sqrt{(x - 2)^2 + (y - 1)^2} = \frac{1}{\sqrt{2}}|x - y|$,
  which is equation of a parabola.
\item Since $|z - z_1| = |z - z_2|$, therefore $z$ will be one of the vertices of the isosceles triangle
  where base will be formed by $z_1$ and $z_2$.

  Also, since $\left|z - \frac{z_1 + z_2}{2}\right|\leq r$ so $z$ will lie on the circle whose center is
  $\frac{z_1 + z_2}{2}$ and radius is $r$. Thus, the distance between segment $z_1z_2$ will be $r$.
  Thus, the maximum area of the triangle will be $\frac{1}{2}|z_1 - z2|.r$.
\item Given $|z_1| = 1 \Rightarrow a_1^2 + b_1^2 = 1, |z_2| = 2 \Rightarrow a_2^2 + b_2^2 = 4$.
  Also given $\Re(z_1z_2) = 0 \Rightarrow a_1a_2 - b_1b_2 = 0 \Rightarrow a_1a_2 = b_1b_2$

  $\Rightarrow a_2^2 + b_2^2 = 4a_1^2 + 4b_1^2 \Rightarrow a_2^2 - 4a_1^2 = 4b_1^2 - b_2^2 \Rightarrow a_2^2
  - 4a_1^2 + 4ia_1a_2 = 4b_1^2 - b_2^2 + 4ib_1b_2$
  $\Rightarrow (a_2 + 2ia_1)^2 = (2b_1 + ib_2)^2 \Rightarrow a_2 = \pm 2b_1$

  $\omega_1 = a_1 + \frac{ia_2}{2} = a_1 \pm b_1 \Rightarrow |\omega_1| = \sqrt{a_1^2 + b_1^2} = 1$
  $\omega_2 = 2b_1 + ib_2 = \pm a_2 + ib_2 \Rightarrow |\omega_2| = \sqrt{a_2^2 + b_2^2} = 2$
  $\Re(\omega_1\omega_2) = 2a_1b_1 - 2a_2b_2 = 0$.
\item Given $z^2 + az + a^2 = 0 \Rightarrow z = a\omega, a\omega^2$ where $\omega$ is cube-root of unity.

  Thus, it represents a pair of straight lines and $|z| = |a|$.
  $\arg(z) = \arg(a) + \arg(\omega)$ or $\arg(a) + \arg(\omega^2) = \pm \frac{2\pi}{3}$.
\item Given $x + \frac{1}{x} = 1 \Rightarrow x^2 - x + 1 = 0 \therefore x = -\omega, -\omega^2$.  Now, for
  $x = -\omega, p = \omega^{4000} + \frac{1}{\omega^{4000}} = \omega + \frac{1}{\omega} = -1$

  Similarly, for $x = -\omega^2, p = -1\Rightarrow 2^{2^n} = 2^{4k} = 16^k =$ a number with last digit as $6
  \Rightarrow q = 6 + 1 = 7\Rightarrow p + q = -1 + 7 = 6$.
\item $A(z_1) = \frac{2i}{\sqrt{3}},B(z_2) = \frac{2}{\sqrt{3}}\left(\frac{\sqrt{3}}{2} -
  i\frac{1}{2}\right) = 1 - \frac{i}{\sqrt{3}}, C(z_3) = \frac{2}{\sqrt{3}}\left(-\frac{\sqrt{3}}{2} -
  \frac{i}{2}\right) = -1-\frac{i}{\sqrt{3}}$

  Clearly, the points lie on the circle $z=2/\sqrt{3}$ and $\triangle ABC$ is equilateral and its centroid
  coincides with circumcentre. Hence,

  $z_1 + z_2 + z_3 = 0$ and $\overline{z_1} + \overline{z_2} + \overline{z_3} = 0$.
  Clearly, radius of incircle $= \frac{1}{\sqrt{3}}$ hence any point on circle is
  $\frac{1}{\sqrt{3}}(\cos\alpha + i\sin\alpha)$. $AP^2 = |z - z_1|^2 = |z|^2 + |z_1|^2 - (z\overline{z_1} +
  \overline{z}z_1)$

  $\Rightarrow AP^2 + BP^2 + CP^2 = 3|z|^2 + |z_1|^2 + |z_2|^2 + |z_3|^2 - z(\overline{z_1} + \overline{z_2}
  + \overline{z_3}) - \overline{z}(z_1 + z_2 + z_3)$
  $= 3\times\frac{1}{3} + \frac{4}{3} + \frac{4}{3} + \frac{4}{3} - 0 - 0 = 5$.
\item Let $O$ be the center of the polygon and $z_0, z_1, \ldots, z_{n - 1}$ represent the vertices $A_1, A_2,
  \ldots, A_n$. $\therefore z_0 = 1, z_1 = \alpha, z_2 = \alpha^2, \ldots, z_{n - 1} = \alpha^{n - 1}$ where
  $\alpha = e^{i2\pi/n}$

  $|A_1A_2|^2 =|\alpha^r - 1|^2 = |1 - \alpha^r|^2 = \left|1 - \cos\frac{2r\pi}{n} +
  i\sin\frac{2r\pi}{n}\right|^2$ $= \left(1 - \cos\frac{2r\pi}{n}\right)^2 + \sin^2\frac{2r\pi}{n} = 2 -
  2\cos\frac{2r\pi}{n}$

  $\displaystyle\sum_{r=1}^n |A_1A_2|^2 = 2(n - 1) - 2\left[\cos\frac{2\pi}{n} + \cos\frac{4\pi}{3} + \ldots
    + \cos\frac{2(n - 1)\pi}{n}\right]$
  $= 2(n - 1) -2.$ real part of $(\alpha + \alpha^2 + \ldots + \alpha^{n - 1}) = 2n [\because 1 + \alpha +
    \alpha^2 + \ldots + \alpha^{n - 1} = 0]$

  $|A_1A_2||A_1A_3|\ldots |A_1A_n| = |1 - \alpha||1 - \alpha^2|\ldots|1 - \alpha^{n - 1}|$ $= |(1 -
  \alpha)(1 - \alpha^2)\ldots(1 - \alpha^{n - 1})|$

  Since $1, \alpha, \alpha^2, \ldots, \alpha^{n - 1}$ are roots of $z^n - 1 = 0$. $(z - 1)(z - \alpha)(z -
  \alpha^2)\ldots(z - \alpha^{n - 1}) = z^n - 1$
  $\Rightarrow (z - \alpha)(z - \alpha^2)\ldots(z - \alpha^{n - 1}) = \frac{z^n - 1}{z - 1} = 1 + z + z^2 +
  \ldots + z^{n - 1}$

  Putting $z = 1$, we get
  $|(1 - \alpha)(1 - \alpha^2)\ldots(1 - \alpha^{n - 1})| = n \Rightarrow \frac{a}{b} = 2$.
\item Let L.H.S. $= z_1$ and R.H.S. $= z_2$ then $\overline{z_1} = \overline{z_2}$
  $\Rightarrow z_1\overline{z_1} = z_2\overline{z_2} \Rightarrow z_1^2 = z_2^2$

  $\Rightarrow \left(1 + \frac{x^2}{a^2}\right)\left(1 + \frac{x^2}{b^2}\right)\left(1 +
  \frac{x^2}{c^2}\right)\ldots = A^2 + B^2$.
\item Given, $x + iy + \alpha\sqrt{(x - 1)^2 + y^2} + 2i = 0$. Equating real and imaginary parts, we get

  $y + 2 = 0 \Rightarrow y = -2$ and $x + \alpha\sqrt{(x - 1)^2 + y^2} = 0$.
  Substituting the value of $y$, we get
  $\alpha\sqrt{x^2 - 2x + 5} = -x \Rightarrow (\alpha^2 - 1)x^2 - 2\alpha^2x +5\alpha^2 = 0$

  Because $x$ is real, the discriminant has to be greater than zero.
  $\Rightarrow 4\alpha^4 - 20\alpha^2(\alpha^2 - 1) \geq 0$
  $\Rightarrow \alpha^2 - 5\alpha^2 + 5 \geq 0 \Rightarrow -\frac{\sqrt{5}}{2}\leq\alpha\leq
  \frac{\sqrt{5}}{2}$.
\item Let $z = x + iy \Rightarrow 2\sqrt{x^2 + y^2} - 4a(x + iy) + 1 + ia = 0$.
  Equating real and imaginary parts, we get

  $2\sqrt{x^2 + y^2} - 4ax + 1 = 0$ and $-4ay + a = 0 \Rightarrow y = \frac{1}{4}$
  $\Rightarrow 2\sqrt{x^2 + \frac{1}{16}} - 4ax + 1 = 0 \Rightarrow 4\left(x^2 + \frac{1}{16}\right) =
  16a^2x^2 - 8ax + 1$

  $x^2(4 - 16a^2) + 8ax - \frac{3}{4} = 0 \Rightarrow x = \frac{-a}{1 - 4a^2} \pm
  \frac{1}{4}\frac{\sqrt{4a^2 + 3}}{1 - 4a^2}$.
\item $(x + iy)^5 = (x^5 - 10x^3y^2 + 5xy^4) + i(5x^4y - 10x^2y^3 + y^5)$.
  Taking modulus and squaring, we get
  $(x^2 + y^2)^5 = (x^5 - 10x^3y^2 + 5xy^4) + (5x^4y - 10x^2y^3 + y^5)^2$.
\item $(x + ia)(x + ib)(x + ic) = [(x^2 - ab) + i(a + b)x](x + ic) = (x^3 - abx - acx - bcx) + i(cx^2 - abc
  + ax^2 + bx^2)$

  Taking modulus and squaring, we get $(x^2 + a^2)(x^2 + b^2)(x^2 + c^2) = [x^3 -(ab + bc + ca)x] + [(a + b
    + c)x^2 - abc]^2$.
\item Given, $(1 + x)^n = a_0 + a_1x + a_2x^2 + \ldots + a_nx^n$. Substituting $x = i$,we get

  $(1 + i)^n = a_o + ia_1 - a_2 - ia_3 + a_4 + \ldots = (a_0 - a_2 + a_4 - \ldots) + i(a_1 - a_3 + a_5 -
  \ldots)$

  Taking modulus and squaring, we get $2^n = (a_0 - a_2 + a_4 - \ldots)^2 + (a_1 - a_3 + a_5 - \ldots)^2$.
\item Let $f(z) = m(z - i) + i$ and $f(z) = n(z + i) + 1 + i$ where $m$ and $n$ are quotients upon division.
  Substituting $z = i$ in the first equation and $z = -i$ in the second we obtain $f(i) = i$ and $f(-i) = 1
  + i$.

  Let $g(z)$ be the quotient and $az + b$ be the remainder upong division of $f(z)$ by $z^2 + 1.$ Hence we
  have $f(z) = g(z)(z^2 + 1) + az + b.$ Substituting $z = i$ and $z = -i,$ we get

  $f(i) = i = ai + b$ and $f(-i) = 1 + i = -ai + b$. Adding, we get $2b = 1 + 2i \Rightarrow b = \frac{1 +
    2i}{2} \Rightarrow ai = i - \frac{1 + 2i}{2}$.
\item Let $z = r_1e^{i\theta_1}, w = r_2e^{i\theta_2}$. $\because |z|\leq 1$ and $|w|\leq 1 \Rightarrow
  r_1\leq 1$ and $r_2\leq 1$

  $|z - w|^2 = (r_1\cos\theta_1 - r_2\cos\theta_2)^2 + (r_1\sin\theta_1 - r_2\sin\theta_2)^2$
  $= r_1^2 + r_2^2 - 2r_2r_2\cos(\theta_1 - \theta_2) = (r_1 - r_2)^2 + 2r_2r_2 - 2r_2r_2\cos(\theta_1 - \theta_2)$

  $= (r_1 - r_2)^2 + 4r_1r_2\sin\left(\frac{\theta_1 - \theta_2}{2}\right)^2\leq (r_1 - r_2)^2 + (\theta_1 -
  \theta_2)^2[\because r_1, r_2\leq 1$ and $\sin\theta \leq \theta]$
  $= (|z| - |w|)^2 + [\arg(z) - \arg(w)]^2$.
\item Let $z = re^{i\theta}$, then $\frac{z}{|z|} = e^{i\theta} = \cos\theta + i\sin\theta$
  $\Rightarrow \left|\frac{z}{|z|} - 1\right| = |(\cos\theta - 1) + i\sin\theta| = \sqrt{\cos\theta^2 -
  2\cos\theta + 1 + \sin^2\theta}$

  $= \sqrt{2 - 2\cos\theta} = \sqrt{4\sin^2\frac{\theta}{2}} = 2\sin\frac{\theta}{2}\leq \theta$
  $\Rightarrow \left|\frac{z}{|z|} - 1\right| \leq |arg(z)|$.
\item Clearly, $|z - 1| = |z - |z| + |z| - 1|\leq |z - |z|| + ||z| - 1|$
  $= |z|\left|\frac{z}{|z|} - 1\right| + ||z| - 1|$

  Using the result of previous problem, we get $|z - 1| \leq ||z| - 1|+|z||argz|$.
\item Let $z = r(\cos\theta + i\sin\theta)$, then $\frac{1}{z} = \frac{1}{r}(\cos\theta - i\sin\theta)$,
  $\left|z + \frac{1}{z}\right| = \left|\left(r + \frac{1}{r}\right)\cos\theta + i\left(r -
  \frac{1}{r}\right)\sin\theta\right|$

  $\Rightarrow \left(r + \frac{1}{r}\right)^2\cos^2\theta + i\left(r - \frac{1}{r}\right)^2\sin^2\theta =
  a^2$ $\Rightarrow \left(r - \frac{1}{r}\right)^2 = a^2 - 4\cos^2\theta$

  $r$ will be greatest when $r - \frac{1}{r}$ will be greatets i.e. $\cos\theta= 0 \Rightarrow r -
  \frac{1}{r} = a$ $\Rightarrow r_{max} = \frac{a + \sqrt{a^2 + 4}}{2}$

  Similarly, for lowest value of $r, \cos\theta = 1 \Rightarrow r - \frac{1}{r} = a^2 - 4 \Rightarrow r^2 -
  (a^2 - 4)r - 1 = 0$ $r_{min} = \frac{a^2- 4 - \sqrt{a^4 - 8a^2 + 20}}{2}$.
\item We have to prove that $|z_1 + z_2|^2 < (1 + c)|z_1|^2 + \left(1 + \frac{1}{c}\right)|z_2|^2$
  $\Rightarrow (z_1 + z_2)(\overline{z_1} + \overline{z_2}) < (1 + c)|z_1|^2 + \left(1 + \frac{1}{c}\right)|z_2|^2$

  $\Rightarrow |z_1|^2 + z_1\overline{z_2} + z_2\overline{z_1} + |z_1|^2 < (1 + c)|z_1|^2 + \left(1 + \frac{1}{c}\right)|z_2|^2$
  $\Rightarrow z_1\overline{z_2} + z_2\overline{z_1} < (1 + c)|z_1|^2 + \left(1 + \frac{1}{c}\right)|z_2|^2$

  $\Rightarrow (x_1 + iy_1)(x_2 - iy_2) + (x_2 + iy_2)(x_1 - iy_1) < \frac{1}{c}[c^2(x_1^2 + y_1^2) + (x_2^2 + y_2^2)]$
  $\Rightarrow 2cx_1x_2 + 2cy_1y_2 < c^2x_1^2 + c^2y_1^2 + x_2^2 + y_2^2$

  $\Rightarrow (cx_1 - x_2)^2 + (cy_1 - y_2)^2 > 0$ which is true.
\item Given $\left|\frac{z_1 - z_2}{z_1 + z_2}\right| = 1 \Rightarrow |z_1 - z_2|^2 = |z_1 + z_2|^2$
  $\Rightarrow (z_1 - z_2)(\overline{z_1} - \overline{z_2}) = (z_1 + z_2)(\overline{z_1} + \overline{z_2})$

  $\Rightarrow 2z_1\overline{z_2} = -2z_2\overline{z_1} \Rightarrow \overline{\left(\frac{z_1}{z_2}\right)}
  = -\frac{z_1}{z_2}$ $\Rightarrow \frac{z_1}{z_2} = $ purely imaginary $\Rightarrow i\frac{z_1}{z_2} =$
  real $= x$

  Now $\frac{z_1 + z_2}{z_1 - z_2} = \frac{z_1/z_2 + 1}{z_1/z_2 - 1} = \frac{-ix + 1}{-ix - 1} = \frac{-1 +
    x^2 + 2ix}{1 + x^2}$. If $\theta$ is the angle between given lines then
  $\tan\theta = \arg\frac{z_1 + z_2}{z_1 - z_2} = \frac{2x}{x^2 - 1}$.
\item Let $z_1 = r_1(\cos\theta_1 + i\sin\theta_1), z_2 = r_2(\cos\theta_2 + i\sin\theta_2)$. Also let $a =
  r\cos\alpha, b = r\sin\alpha$.
  $|az_1 + bz_2|^2 = |rr_1(\cos\theta_1 + i\sin\theta_1)\cos\alpha + rr_2(\cos\theta_2 + i\sin\theta_2)\sin\alpha|^2$

  $= r^2(r1\cos\theta_1\cos\alpha + r_2\cos\theta_2\sin\alpha)^2 + r^2(r_1\sin\theta_1\cos\alpha + r_2\sin\theta_2\sin\alpha)^2$
  $= r^2[r_1^2\cos^2\alpha + r_2^2\sin^2\alpha + 2r_1r_2\cos\alpha\sin\alpha\cos(\theta_1 - \theta_2)]$

  $= \frac{r^2}{2}[r_1^2(1 + \cos2\alpha) + r_2^2(1 - \cos2\alpha) + 2r_1r_2\sin2\alpha\cos(\theta_1 - \theta_2)]$
  $\frac{2|az_1 + bz_2|^2}{a^2 - b^2}= r_1^2 + r_2^2 + (r_1^2 - r_2^2)\cos2\alpha + 2r_2r_2\cos(\theta_1 - \theta_2)\sin2\alpha$

  $= A + B\cos2\alpha + C\sin2\alpha$ where $A = r_1^2 + r_2^2, B = r_1^2- r_2^2, C = 2r_1r_2\cos(\theta_1 - \theta_2)$
  Clearly, $-\sqrt{B^2 + C^2}\leq B\cos2\alpha + C\sin2\alpha \leq \sqrt{B^2 + C^2}$

  $\therefore A -\sqrt{B^2 + C^2}\leq A + B\cos2\alpha + C\sin2\alpha \leq A + \sqrt{B^2 + C^2}$
  $\therefore A -\sqrt{B^2 + C^2}\leq  \frac{2|az_1 + bz_2|^2}{a^2 + b^2}\leq A + \sqrt{B^2 + C^2}$

  Now $B^2 + C^2 = r_1^4 + r_2^4 - 2r_1^2r_2^2 + 4r_1^2r_2^2\cos^2(\theta_1 - \theta_2)$.
  Again $|z_1^2 + z_2^2| = |r_1^2(\cos2\theta_1 + i\sin2\theta_1) + r_2^2(\cos2\theta_2 + i\sin2\theta_2)|$
  $= \sqrt{(r_1^2\cos2\theta_1 + r_2^2\cos2\theta_2)^2 + (r_1^2\sin2\theta_1 + r_2^2\sin2\theta_2)^2}$

  $= \sqrt{r_1^4 + r_2^4 + 2r_1^2r_2^2\cos2(\theta_1 - \theta_2)}$
  $= \sqrt{r_1^4 + r_2^4 + 2r_1^2r_2^2[2\cos^2(\theta_1 - \theta_2) - 1]} = \sqrt{B^2 + C^2}$

  $A = r_1^2 + r_2^2 = |z_1|^2 + |z_2|^2$
  Hence, $|z_1|^2 + |z_2|^2 - |z_1^2 + z_2^2| \leq 2\frac{|az_1 + bz_2|^2}{a^2 + b^2}\leq |z_1|^2 + |z_2|^2
  + |z_1^2 + z_2^2|$.
\item Given $z = \frac{b + ic}{1 + a} \therefore iz = \frac{-c + ib}{1 + a} \Rightarrow \frac{1}{iz} =
  \frac{1 + a}{-c + ib}$. Using componendo and dividendo, we get
  $\Rightarrow \frac{1 + iz}{1 - iz} = \frac{1 + a - c + ib}{1 + a + c - ib}$.
  Also, given $a^2 + b^2 + c^2 = 1 \Rightarrow a^2 + b^2 = 1 - c^2$

  $\Rightarrow (a + ib)(a - ib) = (1 + c)(1 - c)\Rightarrow \frac{a + ib}{1 - c} = \frac{1 + c}{a - ib} =
  \frac{1}{u}$(say) $\therefore \frac{1 + iz}{1 - iz} = \frac{a + ib + 1 - c}{1 + c + a - ib} = \frac{a + ib
    + u(a + ib)}{1 + c + u(1 + c)}$ $= \frac{a + ib}{1 + c}$.
\item We can write that $(x - a)(x - b)\ldots (x - k) = x^n + p_1x^{n - 1} + p_2x^{n - 2} + \ldots + p_{n -
  1}x + p_n$

  Substituting $x = i$, we get
  $(i - a)(i - b)\ldots (i - k) = i^n + p_1i^{n - 1} + p_2i^{n - 2} + \ldots + p_{n - 1}i +  p_n$.
  Dividing both sides by $i^n$, we get
  $(1 + ia)(1 + ib)\ldots(1 + ik) = 1 + \frac{p_1}{i} + \frac{p_2}{i^2} + \ldots$

  Taking modulus and squaring, we get
  $(1 + a^2)(1 + b^2)\ldots (1 + k^2) = (1 - p_2 + p_4 + \ldots)^2 + (p_1 - p_3 + \ldots)^2$.
\item $3 + 2i$ is one value of $x$ for which $f(3 + 2i) = a + ib$
  $\Rightarrow x = 3 + 2i \Rightarrow x^2 - 6x + 13 = 0$

  $f(x) = x^4 - 8x^3 + 4x^2 + 4x + 39 = (x^2 - 6x + 13)(x^2 - 2x -21) -96x + 312$
  $\Rightarrow f(3 + 2i) = -96(3 + 2i) + 312 = 24 - 192i = a + ib$
  $\Rightarrow a:b = 1:-8$.
\item Given $\frac{A}{B} + \frac{B}{A} = 1 \Rightarrow A^2 - AB + B^2 = 0$.
  $A = \frac{B \pm \sqrt{3}iB}{2} = -\omega B, -\omega^2B\Rightarrow |A| = |B|$

  $|A - B| = |-\omega B - B|$ or $|-\omega^2B - B| = |\omega^2 B|$ or $|\omega B|$
  $\Rightarrow |A - B| = |B|$.
  Thus, $|A| = |B| = |A - B|$ making the triangle equilateral.
\item Given $z^n = (z + 1)^n \Rightarrow |z|^n = |z + 1|^n$
  $\Rightarrow |z| = |z + 1|\Rightarrow x^2 = (x^2 + 2x + 1) \Rightarrow 2x + 1 = 0$,
  which is the equation of a straight line on which roots of the given equation will lie.
\item Let $z_1, z_2, z_3, z_4$ be represented by the points $A, B, C, D$ respectively.
  $\therefore AD = |z_1 - z_4|$ and $BC = |z_2 - z_3|$

  Let $a = (z_1 - z_4)(z_2 - z_3), b = (z_2 - z_4)(z_3 - z_1)$ and $c = (z_3 - z_4)(z_1 - z_2)$
  $b + c = (z_2 - z_4)(z_3 - z_1) + (z_3 - z_4)(z_1 - z_2) = -(z_1 - z_4)(z_2 - z_3) = -a$

  $|a| = |b + c| \leq |b| + |c| \Rightarrow |-(z_1 - z_4)(z_2 - z_3)| = |(z_2 - z_4)(z_3 - z_1)| + |(z_3 -
  z_4)(z_1 - z_2)|$
  $\Rightarrow AD.BC\leq BD.CA + CD.AB$.
\item Euqation of a line joining points $a$ and $ib$ is
  $\startbmatrix\NC z \NC \overline{z} \NC 1\NR\NC a \NC \overline{a} \NC 1 \NR\NC ib \NC i\overline{b} \NC
  1\NR\stopbmatrix = 0$ or $(\overline{a} +i\overline{b})z - (a - ib)\overline{z} - i(a\overline{b} +
  \overline{a}b) = 0$

  $\Rightarrow (a + ib)z - (a -ib)\overline{z} - 2abi = 0[\because a, b\in R \therefore a = \overline{a}, b
    = \overline{b}]\Rightarrow (a + ib)z - (a -ib)\overline{z} = 2abi$
  $\Rightarrow \left(\frac{1}{2a} - \frac{i}{2b}\right)z + \left(\frac{1}{2a} +
  \frac{i}{2b}\right)\overline{z} = 1$.
\item Let $z_1 = r_1e^{i\theta_1}$ and $z_2 = r_2e^{i\theta_2}$.

  Then $r_1 - r_2 = \sqrt{(r_1\cos\theta_1 - r_2\cos\theta_2)^2 + (r_1\sin\theta_1 - r_2\sin\theta_2)^2}$

  $\Rightarrow 2r_1r_2 = 2r_1r_2\cos(\theta_1 - \theta_2)\Rightarrow \cos(\theta_1- \theta_2) = \cos 2n\pi$
  $\Rightarrow \arg(z_1) - \arg(z_2) = 2n\pi$.
\item $\triangle ABC$ and $\triangle DOE$ will be similar if
  $\frac{AC}{AB} = \frac{DE}{DO}$ and $\angle BAC = \angle ODE$

  $\Rightarrow \left|\frac{z_3 - z_1}{z_2 - z_1}\right| = \left|\frac{z_5 - z_4}{0 - z_4}\right|$ and
  $\arg\left(\frac{z_3 - z_1}{z_2 - z_1}\right) = \arg\left(\frac{z_5 - z_4}{0 - z_4}\right)$

  $\Rightarrow \frac{z_3 - z_1}{z_2 - z_1} = \frac{z_5 - z_4}{0 - z_4}$.
  Solving this yields $(z_3 - z_2)z_4 = (z_1 - z_2)z_5$ and hence triangles are similar.
\item Given $OA = 1$ and $|z| = 1 = OP \Rightarrow OA = OP$. $OP_0 = |z_0|$ and $OQ = |z\overline{z_0}| =
  |z||\overline{z_0}| = |z_0|$

  $\Rightarrow OP_0 = OQ$. Also given that $\angle P_0OP = \arg\frac{z_0}{z}$.
  $\angle AOQ = \arg\left(\frac{1}{z\overline{z_0}}\right) =
  \arg\left(\frac{\overline{z}}{\overline{z_0}}\right)[\because z\overline{z} = 1]$

  $= -\arg\left(\frac{\overline{z_0}}{\overline{z}}\right)= -\arg\overline{\left(\frac{z_0}{z}\right)} =
  \arg\left(\frac{z_0}{z}\right) = \angle P_0OP$ and thus the triangles are congruent.
\item $P = \frac{az_2 + bz_1}{a + b}, Q = \frac{az_2 - bz_1}{a - b}$
  $OP^2 = \left|\frac{az_2 + bz_1}{a + b}\right|^2 = \left(\frac{az_2 + bz_1}{a +
  b}\right)\left(\frac{a\overline{z_2} + b\overline{z_1}}{a + b}\right)$

  $= \frac{1}{a^2 + b^2}[a^2|z_2|^2 + b^2|z_1|^2 + ab(z_1\overline{z_2} + \overline{z_1}z_2)]$.
  Similalry $OQ^2$ can be computed and the sum be found.
\item Let $c\neq 0$, then $c = -(a + b)$ so we can write $az_1 + bz_2 - (a + b)z_3 = 0 \Rightarrow z_3 =
  \frac{az_1 + bz_2}{a + b}$.

  Thus, we see that $z_3$ divides line segment $z_1z_2$ in the ratio of $a:b$ making all three of them
  collinear.
\item Equation of a line passing through origin is $a\overline{z} + \overline{a}z = 0$. Let us assume that
  all the points lie on the same side of the above line, so we have

  $a\overline{z_i} + \overline{a}z_i > 0$ or $< 0$ for $i = 1, 2, 3, \ldots, n$.
  Thus, $a\displaystyle\sum_{i = 1}^n\overline{z_i} + \overline{a}\sum_{i = 1}^nz_i > 0$ or $< 0$

  But it is given that $\displaystyle\sum_{i = 1}^n z_i = 0 \Rightarrow \sum_{i = 1}^n \overline{z_i} = 0$
  $\therefore\displaystyle a\sum_{i =1}^n\overline{z_i} + \overline{a}\sum_{i = 1}^nz_i = 0$,
  which is in contradiction with equation above. So all points cannot lie on the same side of line.
\item Let $OA$ and $OB$ be the unit vectors representing $z_1$ and $z_2$, then we have
  $\vec{OA} = \frac{z_1}{|z_1|}, \vec{OB} = \frac{z_2}{|z_2|}$

  Therefore equation of bisector will be $z = t\left(\frac{z_1}{|z_1|} + \frac{z_2}{|z_2|}\right) =
  \frac{6}{5}t,$ where is an arbitrary positive integer.
\item The diagram is given below:
  \startplacefigure[location={left, none}]
    \startMPcode
      pair a; pair b; pair c;
      pickup pencircle scaled 0.2pt;
      a = (0, 4cm); b = (-2cm, 0); c = (1cm, 0);
      draw a -- b -- c -- cycle;
      pair p; pair q; pair r; pair d;
      p = whatever[b, c]; a - p = whatever * (b - c) rotated 90;
      q = whatever[c, a]; b - q = whatever * (c - a) rotated 90;
      r = whatever[a, b]; c - r = whatever * (a -b) rotated 90;
      d = whatever[a, p] = whatever[b, q]; % orthocenter
      pickup pencircle scaled 2pt;
      drawdot d;
      pair l;
      l = (a -- p) intersectionpoint  (b -- c);
      pickup pencircle scaled 0.2pt;
      draw a -- l;
      pair m;
      m = (b -- q) intersectionpoint  (a -- c);
      draw b -- m;
      label.bot("$L$", l);
      label.rt("$M$", m);
      label.ulft("$H$", d);
      label.top("$A$", a);
      label.llft("$B$", b);
      label.lrt("$C$", c);
      draw unitsquare scaled 5 rotated angle (a - l) shifted l;
      draw unitsquare scaled 5 rotated angle (b - m) shifted m;
      label.bot("$a$", (b + c)/2);
      label.ulft("$c$", (b + a)/2);
      label.urt("$b$", (c + a)/2);
    \stopMPcode
  \stopplacefigure
  Let $AL$ be perpendicular on $BC$ and $H$ be orthocenter of the $\triangle ABC$.

  $\frac{BL}{LC} = \frac{c\cos B}{b\cos C} = \frac{c\sec C}{b\sec B}$, thus $L$ divides $BC$ internally in the ratio of $c\sec
  C:b\sec B$,  $L = \frac{z_3c\sec C + z_2b\sec B}{c\sec C + b\sec B}$

  $\frac{AH}{HL} = \frac{\Delta ABH}{\Delta HBL} = \frac{\frac{1}{2}AB.BH\sin\angle
    ABM}{\frac{1}{2}BL.BH.\sin\angle MBC} = \frac{c\cos A}{c\cos B\cos C}[\because \angle ABM = 90^\circ -
    A, \angle MBC = 90^\circ - C]$

  $= \frac{a\cos A}{a\cos B\cos C} = \frac{(b\cos C + c\cos B)\cos A}{a\cos B\cos C} = \frac{b\sec B + c\sec C}{a\sec A}$

  $H = \frac{z_1a\sec A + z_2b\sec B + z_3c\sec C}{a\sec A + b\sec B + c\sec C}$

  Since the above expression is similar w.r.t. $A, B$ and $C$, therefore it will also lie on the perpendiculars from $B$ and $C$ to
  opposing sides as well.
  Thus, orthocenter $H = \frac{z_1a\sec A + z_2b\sec B + z_3c\sec C}{a\sec A + b\sec B + c\sec C}$

  $H = \frac{z_1k\sin A\sec A + z_2k\sin B\sec B + z_3k\sin C\sec C}{k\sin A\sec A + k\sin B\sec B + k\sin C\sec C}$,
  $H = \frac{z_1\tan A + z_2\tan B + z_3\tan C}{\tan A + \tan B + \tan C}$.
\item The diagram is given below:
  \startplacefigure[location={left, none}]
    \startMPcode
      pair a; pair b; pair c;
      pickup pencircle scaled 0.2pt;
      a = (0, 4cm); b = (-2cm, 0); c = (1cm, 0);
      draw a -- b -- c -- cycle;
      pair p; pair q; pair r; pair d; pair m; pair n;
      p = whatever[b, c]; a - p = whatever * (b - c) rotated 90;
      q = whatever[c, a]; b - q = whatever * (c - a) rotated 90;
      r = whatever[a, b]; c - r = whatever * (a -b) rotated 90;
      d = whatever[a, p] = whatever[b, q]; % orthocenter
      n = 1/4(a + b + c + d); % remarkably...
      m = d rotatedabout(n, 180); % M is also the circumcentre
      path circumcircle;
      circumcircle = fullcircle scaled 2 abs(m - a) shifted m;
      draw circumcircle;
      pair l;
      pickup pencircle scaled 2pt;
      drawdot m;
      pickup pencircle scaled 0.2pt;
      pair ll;
      ll = (a-- (a - 100(a - m))) intersectionpoint (b -- c);
      draw a -- ll;
      draw b -- m;
      draw c -- m;
      pair t;
      t = whatever[b, c];
      t - m = whatever * (b - c) rotated 90; % perpendicular calculation
      draw m -- t;
      label.bot("$D$", ll);
      label.bot("$L$", t);
      label.rt("$O$", m + (0.2cm, 0));
      label.top("$A$", a);
      label.llft("$B$", b);
      label.lrt("$C$", c);
      draw fullcircle scaled 16 rotated angle (a - m) shifted m cutafter (m -- b) withcolor red; % angle marking
      draw fullcircle scaled 16 rotated angle (b - m) shifted m cutafter (m -- ll) withcolor .7green;
      draw fullcircle scaled 24 rotated angle (ll - m) shifted m cutafter (m -- c) withcolor .7green;
      draw fullcircle scaled 16 rotated angle (c - m) shifted m cutafter (m -- a) withcolor .7blue;
      label.lft("$2C$", m - (0.2cm, 0)) withcolor red;
      label.l("$\pi - 2C$", m - (0, 0.5cm)) withcolor .7green;
      label.lrt("$\pi - 2B$", m - (-0.5, 0.5cm)) withcolor .7green;
      label.rt("$\pi - 2B$", m + (0.4cm, 0)) withcolor .7blue;
      label.bot("$a$", (b + c)/2 + (.2cm, 0cm));
      label.ulft("$c$", (b + a)/2);
      label.urt("$b$", (c + a)/2);
    \stopMPcode
  \stopplacefigure
  Let $O$ be the circumcenter of $\triangle ABC$ where $A=z_1, B=z_2$ and $C=z_3. \frac{BD}{DC} =
  \frac{\frac{1}{2}BD.OL}{\frac{1}{2}DC.OL} = \frac{\Delta BOD}{\Delta COD}$

  $= \frac{\frac{1}{2}OB.OD.\sin(\pi - 2C)}{\frac{1}{2}OC.OD\sin(\pi - 2C)} = \frac{\sin2C}{\sin2B}$.  Thus,
  $D$ divides $BC$ internally in the ratio $\sin2C:\sin2B \Rightarrow D = \frac{z_3\sin2C +
    z_2\sin2B}{\sin2C + \sin2B}$

  The complex number dividing $AD$ internally in the ratio $\sin2B+\sin2C:\sin2A$ is $\frac{z_1\sin 2A +
    z_2\sin 2B + z_3\sin 2C}{\sin 2A + \sin 2B + \sin 2C}$

  Since the above expression is similar w.r.t. $A, B$ and $C$, therefore it will also lie on the
  perpendicular bisectors on $AC$ and $AB$ as well.

  Let $BO$ produced meet $AC$ at $E$ and $CO$ produced meet $AB$ at $F$. We can show that, the complex
  numner representing the point dividing the line segment $BE$ internally in the ratio $(\sin2C +
  \sin2A):\sin2B$ and the complex number representing the point dividing the line segment $CF$ internally in
  the ratio $(\sin2A+ \sin2B):\sin2C$ will be each $= \frac{z_1\sin 2A + z_2\sin 2B + z_3\sin 2C}{\sin 2A +
    \sin 2B + \sin 2C}$

  Thus, circumcenter is $\frac{z_1\sin 2A + z_2\sin 2B + z_3\sin 2C}{\sin 2A + \sin 2B + \sin 2C}$
\item Let $z$ be the circumcenter of the triangle represented by $A(z_1), B(z_2)$ and $C(z_3)$ respectively,
  then $|z - z_1| = |z - z_2| = |z - z_3|$ so we have $|z - z_1| = |z - z_2|$
  $\Rightarrow |z - z_1|^2 = |z - z_2|^2 \Rightarrow (z - z_1)(\overline{z} - \overline{z_1}) = (z -
  z_2)(\overline{z} - \overline{z_2})$

  $\Rightarrow z\overline{z} + z_1\overline{z_1} - \overline{z}z_1 - z\overline{z_1} = z\overline{z} +
  z_2\overline{z_1} - \overline{z}z_2 - z\overline{z_2}$
  $\Rightarrow z(\overline{z_1} - \overline{z_2})+ \overline{z}(z_1 - z_2) = z_1\overline{z_1} - z_2\overline{z_2}$

  Similarly considering $|z - z_1| = |z - z_3|$, we will have
  $\Rightarrow z(\overline{z_1} - \overline{z_3})+ \overline{z}(z_1 - z_3) = z_1\overline{z_1} - z_3\overline{z_3}\stopalign$

  We have to eliminate $\overline{z}$ from equation (1) and (2) i.e. multiplying equation (1) with $(z_1 - z_3)$ and (2) with $(z_1
  - z_2)$, we get following

  $z[\overline{z_1}(z_2 - z_3) + \overline{z_2}(z_3 - z_1) + \overline{z_3}(z_1 - z_2)] = z_1\overline{z_1}(z_2 - z_3) +
  z_2\overline{z_2}(z_3 - z_1) + z_3\overline{z_3}(z_1 - z_2)$
  $\Rightarrow z = \frac{\sum z_1\overline{z_1}(z_2 - z_3)}{\sum \overline{z_1}(z_2 - z_3)}$.
\item Let $z$ be the orthocenter of $\triangle A(z_1)B(z_2)C(z_3)$ i.e. the intersection point of
  perpendiculars on sides from opposite vertices.

  Since $AH\perp BC \therefore \arg\left(\frac{z_1 - z}{z_3 - z_2}\right) = \pm\frac{\pi}{2}$
  $\Rightarrow \frac{z_1 - z}{z_3 - z_2}$ is purely imaginary.

  $\Rightarrow \overline{\left(\frac{z_1 - z}{z_3 - z_2}\right)} = -\left(\frac{z_1 - z}{z_3 -
    z_2}\right)\Rightarrow\frac{\overline{z_1} - \overline{z}}{\overline{z_3} - \overline{z_2}} = \frac{z - z_1}{z_3 - z_2}$
  $\Rightarrow \overline{z_1} - \overline{z} = \frac{(z - z_1)(\overline{z_3} - \overline{z_2})}{z_3 - z_2}$

  Similarly for $BH\perp AC, \overline{z_2} - \overline{z} = \frac{(z - z_2)(\overline{z_1} - \overline{z_2})}{z_1 - z_3}$

  Eliminating $\overline{z}$ like last problem we arrive at the desired result.
\item We have $\angle CBA =\frac{2\pi}{3}$, therefore
  $\frac{z_3 - z_2}{z_1 - z_2} = \frac{|z_3 - z_2|}{|z_1 - z_2|}\left[\cos\frac{2\pi}{3} + i\sin\frac{2\pi}{3}\right]$
  $= -\frac{1}{2} + \frac{i\sqrt{3}}{2}[\because BC = AB]$

  $z_3 + \left(\frac{1}{2} - \frac{i\sqrt{3}}{2}\right)z_1 = \left(\frac{3}{2} - \frac{i\sqrt{3}}{2}\right)z_2$

  Solving this yields $2\sqrt{3}z_2 = (\sqrt{3} - i)z_1 + (\sqrt{3} + i)z_3$.
  Also, since diagonals bisect each other $\Rightarrow \frac{z_1 + z_3}{2} = \frac{z_2 + z_4}{2}$,
  $z_4 = z_1 + z_3 - z_2$
  Substituting the value of $z_2$, we get
  $2\sqrt{3}z_4 = (\sqrt{3} + i)z_1 + (\sqrt{3} - i)z_3$.
\item Since $\angle PQR = \angle PRQ = \frac{1}{2}(\pi - \alpha) \therefore PQ = PR$ Also, $\angle QPR = \pi
  - 2\left(\frac{\pi}{2} - \frac{\alpha}{2}\right) = \alpha$
  $\therefore \arg\frac{z_3 - z_1}{z_2 - z_1} = \alpha \Rightarrow \frac{z_3 - z_1}{z_2 - z_1} =
  \frac{PR}{RQ}(\cos\alpha + i\sin\alpha)$

  $\Rightarrow \frac{z_3 - z_1}{z_2 - z_1} -1 = (\cos\alpha - 1) + i\sin\alpha \Rightarrow \frac{z_3 -
    z_2}{z_2 - z_1} = -2\sin^2\frac{\alpha}{2} + i2\sin\frac{\alpha}{2}\cos\frac{\alpha}{2}$

  $\Rightarrow \left(\frac{z_3 - z_2}{z_2 - z_1}\right)^2 =
  -4\sin^2\frac{\alpha}{2}\left[\cos\frac{\alpha}{2} + i\sin\frac{\alpha}{2}\right]^2 =
  -4\sin^2\frac{\alpha}{2}[\cos\alpha + i\sin\alpha] = -4\sin^2\frac{\alpha}{2}.\frac{z_3 - z_1}{z_2 - z_1}$

  $\Rightarrow (z_3 - z_2)^2 = 4(z_3 - z_1)(z_1 - z_2)\sin^2\frac{\alpha}{2}$.
\item Let $C$ be the center of a regular polygon of $n$ sides. Let $A_1(z_1), A_2(z_2)$ and $A_3(z_3)$ be its three
  consecutive vertices.

  $\angle CA_2A_1 = \frac{1}{2}\left(\pi - \frac{2\pi}{n}\right) \therefore A_1A_2A_3 = \pi - \frac{2\pi}{n}$

  {\bf Case I:} When $z_1, z_2, z_3$ are in anticlockwise order. $\Rightarrow z_1 - z_2 = (z_3 - z_2)e^{i\left(\pi -
    2\pi/n\right)}[\because A_1A_2 = A_3A_2]$

  $z_1 - z_2 = (z_2 - z_3)e^{-i2\pi/n}[\because e^{i\pi} = -1] \Rightarrow z_3 = z_2 - (z_1 - z_2)e^{i2\pi/n}$

  {\bf Case II:} When $z_1, z_2, z_3$ are in clockwise order. $\Rightarrow z_3 - z_2 = (z_1 - z_2)e^{i\left(\pi -
    i2\pi/n\right)}$

  $z_3 = z_2 + (z_2 - z_1)e^{-i2\pi/n}$.
\item Let $O$ be the origin and the complex number representing $A_1$ be $z$, then $A_2, A_3, A_4$ will be
  represented by $ze^{i2\pi/n}, ze^{i4\pi/n}, ze^{i6\pi/n}$. Let $|z| = a$

  $A_1A_2 = \left|z - ze^{i2\pi/n}\right| = |z|\left|1 - \cos\frac{2\pi}{n} - i\sin\frac{2\pi}{n}\right|$
  $= a\sqrt{\left(1 - \cos\frac{2\pi}{n}\right)^2 + \sin^2\frac{2\pi}{n}} = a\sqrt{2\left(1 -
    \cos\frac{2\pi}{n}\right)} = 2a\sin\frac{\pi}{n}$

  Similarly, $A_1A_3 = 2a\sin\frac{2\pi}{n}$ and $A_1A_4 = 2a\sin\frac{3\pi}{n}$

  Given $\frac{1}{A_1A_2} = \frac{1}{A_1A_3} + \frac{1}{A_1A_4}\therefore \frac{1}{2a\sin\frac{\pi}{n}} =
  \frac{1}{2a\sin\frac{2\pi}{n}} + \frac{1}{2a\sin\frac{3\pi}{n}}$
  $\Rightarrow \sin\frac{\pi}{n}\left(\sin\frac{3\pi}{n} + \sin\frac{2\pi}{n}\right) =
  \sin\frac{2\pi}{n}\sin\frac{3\pi}{n}$

  $\Rightarrow \sin\frac{3\pi}{n} + \sin\frac{2\pi}{n} = 2\cos\frac{2\pi}{n}\sin\frac{3\pi}{n} =
  \sin\frac{4\pi}{n} + \sin\frac{2\pi}{n}$
  $\Rightarrow \sin\frac{3\pi}{n} = \sin\frac{4\pi}{n}\Rightarrow \frac{3\pi}{n} = m\pi +
  (-1)^n\frac{4\pi}{n}, m = 0,\pm1, \pm2,\ldots$

  If $m = 0\Rightarrow \frac{3\pi}{n} = \frac{4\pi}{n} \Rightarrow 3 = 4$ (not possible).
  If $m = 1\Rightarrow \frac{3\pi}{n} = \pi - \frac{4\pi}{n}\Rightarrow n = 7$.
  If $m = 2,3 \ldots, -1, -2,\ldots$ gives values of $n$ which are not possible. Thus $n = 7$.
\item Given, $|z| = 2$. Let $z_1 = -1 + 5z \Rightarrow z_1 + 1 = 5z$.

  $|z_1 + 1| = |5z| = 5|z| = 10$
  $\Rightarrow z_1$ lies on a circle with center $(-1, 0)$ having radius $10$.
\item Given, $|z - 4 + 3i|\leq 2 \Rightarrow ||z| - |4 - 3i||\leq 2\Rightarrow ||z| - 5|\leq 2 \Rightarrow
  -2 \leq |z| - 5\leq 2 \Rightarrow 3\leq |z|\leq 7$.
\item $|z - 6 - 8i|\leq |4| \Rightarrow -4 \leq ||z| - |6 + 8i|| \leq 4\Rightarrow -4 \leq |z| - 10 \leq 10
  \Rightarrow 6\leq |z|\leq 14$.
\item The diagram is given below:
  \startplacefigure[location={left,none}]
    \startMPcode
      pickup pencircle scaled 0.2pt;
      pair c;
      c = (0cm, 2.5cm);
      draw fullcircle scaled 3cm shifted c;
      drawarrow (-0.5cm, 0cm) -- (4cm, 0cm);
      drawarrow (0cm, -0.5cm) -- (0cm, 5cm);
      draw (0, 0) -- (3cm, 4cm);
      draw (0, 2.5cm) -- (1.2cm, 1.6cm);
      label.rt("$x$", (4cm, 0));
      label.top("$y$", (0, 5cm));
      label.llft("$O$", (0,0));
      label.lft("$C(0, 25)$", c);
      label.rt("$P$", (1.2cm, 1.6cm));
      draw fullcircle scaled 16 rotated angle ((1cm, 0) - (0, 0)) shifted (0, 0) cutafter ((0, 0) -- (3cm, 4cm));
      % angle marking
      label.rt("$\theta$", (0.2cm, 0.25cm));
      draw fullcircle scaled 16 rotated angle ((0, 0) - c) shifted c cutafter (c -- (1.2cm, 1.6cm));
      label.rt("$\theta$", c - (-0.1cm, 0.4cm));
    \stopMPcode
  \stopplacefigure
  Given $z - 25i \leq 15$, which represents a circle having center $(0, 25)$ and a radius $15$.
  Let $OP$ be tangent to the circle at point $P$, then $\angle XOP$ will represent least
  value of $\arg(z)$.

  Let $\angle XOP = \theta$ then $\angle OCP = \theta$. Now $OC = 25, CP = 15 \therefore OP = 20\therefore
  \tan\theta = \frac{OP}{CP} = \frac{4}{3}$. $\therefore$ Least value of $\arg(z) = \theta =
  \tan^{-1}\frac{4}{3}$
  \vskip 1.8cm
\item Given, $|z - z_1|^2 + |z - z_2|^2 = k\Rightarrow |z|^2 + |z_1|^2 - 2z\overline{z_1} + |z|^2 + |z_2|^2
  - 2z\overline{z_2} = k$

  $\Rightarrow 2|z|^2 - 2z(\overline{z_1} + \overline{z_2}) = k - (|z_1|^2 + |z_2|^2)\Rightarrow |z|^2 -
  2z\left(\frac{\overline{z_1 + z_2}}{2}\right) + \frac{1}{4}|z_1 + z_2|^2 = \frac{k}{2} + \frac{1}{4}[|z_1
    + z_2|^2 - 2|z_1|^2 -2|z_2|^2]$

  $\Rightarrow \left|z - \frac{z_1 + z_2}{2}\right|^2 = \frac{1}{2}\left[k - \frac{1}{2}|z_1 -
    z_2|^2\right]$. The above equation represents a circle with center at $\frac{z_1 + z_2}{2}$ and radius
  $\frac{1}{2}\sqrt{2k - |z_1 - z_2|^2}$ provided $k\geq \frac{|z_1 - z_2|^2}{2}$.
\item Since $|z - 1| = 1, z$ represents a circle with center $(1, 0)$ and a radius of of $1$. It is shown
  below:
  \startplacefigure[location={left,none}]
    \startMPcode
      pickup pencircle scaled 0.2pt;
      pair c; c = (1cm, 0);
      draw fullcircle scaled 2cm shifted c;
      drawarrow (-0.5cm, 0) -- (3cm, 0);
      drawarrow (0, -1.5cm) -- (0, 1.5cm);
      label.rt("$x$", (3cm, 0));
      label.top("$y$", (0, 1.5cm));
      label.llft("$O$", (0,0));
      label.bot("$C(1, 0)$", c);
    \stopMPcode
  \stopplacefigure
  Now $|z - 1| = 1$. Let $z = x + iy$ then $x^2 + y^2 = 2x$. Also,

  $\frac{z - 2}{z} = \frac{x - 2 + iy}{x + iy} = \frac{x^2 - 2x + y^2 + 2iy}{x^2 + y^2} = i\frac{y}{x}$

  \noindent{\bf Case I.} When $z$ lies in the first quadrant. This implies $\arg(z) = \theta$, where
  $\tan\theta = \frac{y}{x} \therefore i\tan[\arg(z)] = i\tan\theta = i\frac{y}{x}$.

  \noindent{\bf Case II.} When $z$ lies in the fourth quadrant. Thus, $\arg(z) = 2\pi - \theta$, where
  $\tan\theta = \frac{-y}{x}\therefore i\tan[\arg(z)] = i\tan(2\pi - \theta) = i\frac{y}{x}$.
\item Let $z = x + iy$. Now we have $\frac{z - 1}{z + 1} = \frac{(x^2 - 1) + y^2}{(x + 1)^2 + y^2} +
  i\frac{2y}{(x + 1)^2 + y^2}$

  $\therefore \arg\left(\frac{z - 1}{z + 1}\right) = \frac{\pi}{4}\Rightarrow \tan\left(\arg\left(\frac{z -
    1}{z + 1}\right)\right) = \frac{2y}{x^2 - 1 + y^2}$

  $\Rightarrow x^2 + y^2 - 1 -2 y = 0 \Rightarrow x^2 + (y - 1)^2 = 2$, which is equation of a circle having
  center at $(0, 1)$ and radius $\sqrt{2}$.
\item Let $z = x + iy$. Now, $u + iv = (z - 1)(\cos\alpha - i\sin\alpha) + \frac{1}{z - 1}(\cos\alpha +
  i\sin\alpha)= (x - 1)\cos\alpha + y\sin\alpha + i[y\cos\alpha - (x - 1)\sin\alpha] + \frac{x - 1 - iy}{(x
    - 1)^2 + y^2}(\cos\alpha + i\sin\alpha) = 0$

  Equating imaginary parts, we get
  $v = y\cos\alpha - (x - 1)\sin\alpha + \frac{(x - 1)\sin\alpha - y\cos\alpha}{(x - 1)^2 + y^2} =
  0\Rightarrow [y\cos\alpha - (x - 1)\sin\alpha][(x - 1)^2 + y^2] = 0$

  $\therefore $ Either $y\cos\alpha - (x - 1)\sin\alpha = 0 \Rightarrow y = \tan\alpha(x - 1)$, which is a
  straight line passing through $(1, 0)$ or $(x - 1)^2 + y^2 - 1 = 0$ which is a circle with center $(1, 0)$
  and unit radius.
\item Given, $1 + a_1z + a_2z^2 + \cdots + a_nz^n = 0 \Rightarrow |a_1z| + |a_2z^2| + \cdots + |a_nz^b|\geq
  1$ and

  L.H.S. $ < 2|z| + 2|z|^2 + \cdots$ to $\infty[\because |a_n| < 2]$.

  Let $|z| < 1$ then $\frac{2|z|}{1 - |z|} < 1 \Rightarrow |z| > \frac{1}{3}$

  When $|z|> 1$, clearly $|z| > \frac{1}{3}$; hence, $z$ does not lie in the interior of the circle with
  radius $\frac{1}{3}$.
\item Given, $z^n\cos\theta_0 + z^{n - 1}\cos\theta_1 + \cdots + \cos\theta_n = 2$
  $\Rightarrow 2 = |z^n\cos\theta_0 + z^{n - 1}\cos\theta_1 + \cdots + \cos\theta_n|$

  $< |z^n\cos\theta_0| + |z^{n - 1}\cos\theta_1| + \cdots + |\cos\theta_n|$
  $= |z^n||\cos\theta_0| + |z^{n - 1}||\cos\theta_1| + \cdots + |\cos\theta_n|$

  $\leq |z|^n + |z|^{n - 1}  + \cdots + 1 < 1 + |z| + |z|^2 + \cdots$ to $\infty$
  $\Rightarrow 2 = \frac{1}{1 - |z|} \Rightarrow |z| > \frac{1}{2} [$ when $|z| < 1]$

  Hence $z$ lies outside the circle $|z| = \frac{1}{2}$.
  Thus all roots of the given equation lie outside the circle $|z| = \frac{1}{2}$.
\item Recall that points $z_1, z_2, z_3$ are concyclic if $\left(\frac{z_2 - z_4}{z_1 -
  z_4}\right)\left(\frac{z_1 - z_3}{z_2 - z_3}\right)$ is real. We assume that $z_4$ is origin.

  Given, $\frac{2}{z_1} = \frac{1}{z_2} + \frac{1}{z_3} = \frac{z_2 + z_3}{z_2z_3} \therefore z_1 =
  \frac{2z_2z_3}{z_1+z_3}$.

  Putting the value of $z_1$ and $z_4$ in the concyclic condition expression we obtain

  $\left(\frac{z_2 - z_4}{z_1 - z_4}\right)\left(\frac{z_1 - z_3}{z_2 - z_3}\right) = \frac{1}{2}$.
  Thus, $z_1, z_2, z_3$ lie on a circle passing through origin.
\item The diagram given below:
  \startplacefigure[location={left,none}]
    \startMPcode
      % \draw (0, 0) circle(1);
      % \draw (.866, -.5) -- (-.866, -.5) -- (.5, .866) -- cycle;
      % \draw (.5, .866) -- (.5, -.866);
      % \filldraw (0, 0) circle(1pt);
      % \draw (0, 0) node[anchor=north] {$O$};
      % \draw (.866, -.5) node[anchor=north west] {$C(z_3)$} (-.866, -.5)
      % node[anchor=north east] {$B(z_2)$} (.5, .866) node[anchor=south]
      % {$A(z_1)$} (.5, -.866) node[anchor=north] {$P(z)$};
      pickup pencircle scaled 0.2pt;
      draw fullcircle scaled 2cm;
      draw (.866cm, -.5cm) -- (-.866cm, -.5cm) -- (.5cm, .866cm) -- cycle;
      draw (.5cm, .866cm) -- (.5cm, -.866cm);
      pickup pencircle scaled 2pt;
      drawdot (0, 0);
      label.bot("$O$", (0, 0));
      label.lrt("$C(z_3)$", (.866cm, -.5cm));
      label.llft("$B(z_2)$", (-.866cm, -.5cm));
      label.top("$A(z_1)$", (.5cm, .866cm));
      label.bot("$P(z)$", (.5cm, -.866cm));
    \stopMPcode
  \stopplacefigure
  We have $OP=OA=OB=OC \therefore |z| = |z_1| = |z_2| = |z_3| \Rightarrow |z|^2 = |z_1|^2 = |z_2|^2 =
  |z_3|^2 \Rightarrow z\overline{z} = z_1\overline{z_1} = z\overline{z_2} = z\overline{z_3}$.

  Since $AP$ is perpendicular to $BC, \therefore\arg\left(\frac{z_1 - z}{z_2 - z_3}\right) = \frac{\pi}{2}$
  or $\frac{-\pi}{2}\Rightarrow \frac{z_1 - z}{z_2 - z_3}$ is purely imaginary.

  $\Rightarrow \overline{\left(\frac{z_1 - z}{z_2 - z_3}\right)} = -\frac{z_1 - z}{z_2 - z_3}$.
  Solving the above equation gives $z = \frac{z_2z_3}{z_1}$.
\item The diagram is given below:
  \startplacefigure[location={left,none}]
    \startMPcode
      pickup pencircle scaled 0.2pt;
      draw fullcircle scaled 2cm shifted (2cm, 0);
      draw (0cm, -0.2cm) -- (3.2cm, -0.2cm);
      draw (0cm, -0.2cm) -- (3.2cm, 0.8cm);
      draw fullcircle scaled 2cm shifted (5cm, 0);
      draw (3.8cm, -.6cm) -- (6.2cm, .6cm);
      draw (3.8cm, .8cm) -- (6.2cm, -.8cm);
      label("$A$", (0.9cm, -0.4cm));
      label("$B$", (3.1cm, -0.4cm));
      label("$P$", (-0.2cm, -0.2cm));
      label("$C$", (0.9cm, 0.3cm));
      label("$D$", (2.9cm, 0.85cm));
      label("$A$", (4cm, -0.6cm));
      label("$P$", (4.95cm, -0.2cm));
      label("$B$", (5.95cm, 0.65cm));
      label("$C$", (4.15cm, 0.75cm));
      label("$D$", (5.85cm, -0.75cm));
    \stopMPcode
  \stopplacefigure
  Let $P(z)$ be the point of intersection and $A, B, C, D$ represent points $a, b, c, d$ respectively. Clearly, $P, A, B$ are
  collinear. Thus,

  $\startbmatrix\NC z \NC \overline{z} \NC 1\NR\NC a \NC \overline{a} \NC 1\NR\NC b \NC \overline{b} \NC
  1\NR\stopbmatrix = 0 \Rightarrow z(\overline{a} - \overline{b}) - \overline{z}(a - b) + (a\overline{b} -
  \overline{a}b) = 0$

  Similarly, $P, C, D$ are collinear and thus
  $\Rightarrow z(\overline{c} - \overline{d}) - \overline{z}(c - d) + (c\overline{d} - \overline{c}d) = 0$

  Eliminating $\overline{z}$ because we have to find $z$, we have
  $z(\overline{a} - \overline{b})(c - d) - z(\overline{c} - \overline{d})(a - b) = (c\overline{d} -
  \overline{c}d)(a - b) - (a\overline{b} - \overline{a}b)(c - d)$.

  $\because a, b, c, d$ lie on the circle. $|a| = |b| = |c| = |d| = r \Rightarrow a^2 = b^2 = c^2 = d^2 = r^2$
  $\Rightarrow a\overline{a} = b\overline{b} = c\overline{c} = d\overline{d} = r^2$

  $\Rightarrow \overline{a} = \frac{r^2}{a}, \overline{b} = \frac{r^2}{b}, \overline{c} = \frac{r^2}{c},
  \overline{d} = \frac{r^2}{d}$

  Putting these values in the equation we had obtained,
  $z\left(\frac{r^2}{a} - \frac{r^2}{b}\right)(c - d) - z\left(\frac{r^2}{c} - \frac{r^2}{d}\right)(a - b) = \left(\frac{cr^2}{d} -
  \frac{dr^2}{c}\right)(a - b) - \left(\frac{ar^2}{b} - \frac{br^2}{a}\right)(c - d)$

  Solving this for $z$, we arrive at desired answer.
\item Given $\startbmatrix\NC a \NC b \NC c\NR\NC b \NC c \NC a\NR\NC c \NC a \NC b\NR\stopbmatrix = 0$
    $\Rightarrow a^3 + b^3 + c^3 - 3abc = 0\Rightarrow (a + b + c)(a^2 + b^2 + c^2 - ab - bc - ca) = 0$

    $\because z_1, z_2, z_3$ are three non-zero complex numbers, hence $a^2 + b^2 + c^2 - ab - bc - ca = 0$
    $\Rightarrow (a - b)^2 + (b - c)^2 + (c - a)^2 = 0 \Rightarrow a = b = c$. This can be represented by
  following diagram:
  \startplacefigure[location={left,none}]
    \startMPcode
      % pair o = (0,0);
      % path circle = circle(o, 2);
      % draw(circle);
      % pair a = (0, 2);
      % pair b = (1.414, -1.414);
      % pair c = (-1.414, -1.414);
      % draw (a -- b -- c -- cycle);
      % draw (b -- o);
      % draw (c -- o);
      % label("$A$", a, align=N);
      % label("$B$", b, align=E);
      % label("$C$", c, align=W);
      % label("$O$", o, align=N);
      % markangle("", radius=10, c, o, b);
      pair o;
      o = (0, 0);
      draw fullcircle scaled 4cm;
      pair a; pair b; pair c;
      a = (0, 2cm); b = (1.414cm, -1.414cm);
      c = (-1.414cm, -1.414cm);
      draw a -- b -- c -- cycle;
      draw b -- o;
      draw c -- o;
      label.top("$A$", a);
      label.rt("$B$", b);
      label.lft("$C$", c);
      label.top("$O$", o);
      draw fullcircle scaled 16 rotated angle (c - o) shifted o cutafter (o -- b);
    \stopMPcode
  \stopplacefigure
  Now $OA=OB=OC$, where $O$ is the origin and $A, B$ and $C$ are the points representing $z_1, z_2$ and $z_3$ respectively.
  $\therefore O$ is the circumcenter of $\triangle ABC$.

  Now $\arg\left(\frac{z_3}{z_2}\right) = \angle BOC = 2\angle BAC = \arg\left(\frac{z_3 - z_1}{z_2 -
    z_1}\right)^2$.
  \vskip 2cm
\item The diagram is given below:
  \startplacefigure[location={left, none}]
    \startMPcode
      pair oo;
      oo = (2cm, 0cm);
      draw fullcircle scaled 4cm shifted oo;
      pair o; pair p; pair q; pair r;
      o = (0, 0); p = (4cm, 0); q = (3cm, 1.732cm); r = (1cm, 1.732cm);
      draw o -- p; draw o -- q; draw o -- r; draw p -- q; draw p -- r;
      draw unitsquare scaled 5 rotated angle (o - q) shifted q;
      draw unitsquare scaled 5 rotated angle (o - r) shifted r;
      draw fullcircle scaled 16 rotated angle (p - o) shifted o cutafter (o -- q);
      draw fullcircle scaled 16 rotated angle (q - o) shifted o cutafter (o -- r);
      label.lft("$O$", o);
      label.rt("$P(z_1)$", p);
      label.top("$Q(z_2)$", q);
      label.top("$R(z_3)$", r);
    \stopMPcode
  \stopplacefigure
  $z_2 = \frac{OQ}{OP}z_1e^{i\theta} = \cos\theta z_1e^{i\theta}$ and $z_3 = \frac{OR}{OP}z_1e^{i2\theta} =
  \cos2\theta z_1e^{i2\theta}$

  $\Rightarrow z_2^2 = \cos^2\theta z_1^2e^{i2\theta} \Rightarrow z_2^2\cos2\theta = z_1z_3\cos^2\theta$.
  \vskip 2.5cm
\item Given circles are $|z| = 1 \Rightarrow x^2 + y^2 - 1 = 0$ and $|z - 1| = 4 \Rightarrow x^2 - 2x + y^2
  - 15 = 0$.

  Let the circles cut by these two orthogonally is $x^2 + y^2 + 2gx + 2fy + c = 0$. Since first circle cuts
  this family of circles orthoginally, therefore

  $2g.0 + 2f.0 = c - 1 \Rightarrow c = 1$ and $2g(-1) + 2f.0 = c - 15 \Rightarrow g = 7$.
  Thus, required circles are $x^2 + y^2 + 14x + 2fy + 1 = 0 \Rightarrow |z + 7 + if| = \sqrt{48 + f^2}$.
\item Given, $|z + 3| = t^2 - 2t + 6$ which is equation of a circle having center $(-3, 0)$ and radius $t^2 - 2t +
  6$. Let $A = (-3, 0)$ and $r_1 = t^2 - 2t + 6$. In this case $z$ lies on the circle.

  Also, $|z - 3\sqrt{3}i| < t^2$ implies $z$ lies on the interior of the circle having center $(0,
  3\sqrt{3})$ and radius $t^2$. Let $B = (0, 3\sqrt{3})$ and $r_2 = t^2$. $AB = \sqrt{3^2 + 27} = 6$. $r_2 -
  r_1 = 2(t - 3)$

  Clearly, when the two circles are disjoint or touching each other no solution is possible. This leads to
  following cases:

  {\bf Case I:} When $t > 3$ i.e. $r2 > r_1$. In this case at least one $z$ is possible if $AB < r_1 + r_2
  \Rightarrow 6 < 2(t^2 - t + 3)\Rightarrow t < 0$ or $t > 1\Rightarrow 3 < t <\infty$

  {\bf Case II:} When $t \leq 3$ i.e. $r_1 > r_2$. In this case at least one $z$ will be possible if $|r_1 -
  r_2| \leq AB < r_1 + r_2$

  $2(3 - t)\leq 6 < 2(t^2 - t + 3)$ i.e. $t \leq 0$ and $t < 0$ or $t > 1$
  Combining all solutions we gace $1 < t < \infty$.
\item Let $z = x + iy$. $\frac{az + b}{cz + d} = \frac{ax + b + iay}{cx + d + icy} = \frac{(ax + b + iay)(cx
  + d - icy)}{(cx + d)^2 + c^2y^2}$

  $\Im\left(\frac{az + b}{cz + d}\right) = \frac{ay(cx + d) - cy(ax + b)}{(cx + d)^2 + c^2y^2} = \frac{ady -
  bcy}{(cx + d)^2 + c^2y^2}$

  $\because ad > bc$, therefore the signs of imaginary parts of $z$ and $\frac{az + b}{cz + d}$ are the same.
\item Given, $z_1 = \frac{i(z_2 + 1)}{z_2 - 1} \Rightarrow x_1 + iy_1 = \frac{-y_2 + i(x_2 + 1)}{(x_2 - 1) + iy_2}
  = \frac{[-y_2 + i(x_2 + 1)][(x_2 - 1) + iy_2]}{(x_2 - 1)^2 + y_2^2}$

  Comparing real and imaginary parts, we have

  $x_1 = \frac{-y_2(x_2 - 1) -(x_2 + 1)y_2}{(x_2 - 1)^2 + y_2^2} = \frac{-2x_2y_2}{(x_2 - 1)^2 + y_2^2}$ and $y_1 = \frac{x_2^2 - 1
    - y_2^2}{(x_2 - 1)^2 + y_2^2}$

  Substituting for $x_1$ and $y_1$ in $x_1^2 + y_1^2 - x_1$ we will arrive at the desired result.
\item $(\cos3\theta - i\sin3\theta)^6 = (e^{-i3\theta})^6 = e^{-i18\theta}$ and $(\cos2\theta +
  i\sin2\theta)^5 = (e^i2\theta)^5 = e^{i10\theta}$

  $(\sin\theta - i\cos\theta)^3 = [(-i)^3(\cos\theta + i\sin\theta)^3] = i.e^{i3\theta}$ and
  $\frac{(\cos3\theta - i\sin3\theta)^6(\sin\theta - i\cos\theta)^3}{(\cos2\theta + i\sin2\theta)^5} =
  i.e^{-i25\theta} = \sin25\theta + i\cos25\theta$.
\item Let $z = x + iy$, then we have $x^2 - y^2 + 2ixy + \sqrt{x^2 + y^2} = 0$

  Equating imaginary parts, we have $2xy = 0$ i.e. either $x = 0$ or $y = 0$.

  If $x = 0$, then $-y^2 + \sqrt{y^2} = 0 \Rightarrow y^4 + y^2 = 0 \Rightarrow y = 0, y = \pm i$.

  If $y = 0$, then $x^2 + \sqrt{x^2} = 0$ Since $x$ is real only one solution is possible i.e. $x = 0$.
  Hence, $z = 0, \pm i$.
\item Clearly $z = 0$ is one of the solutions. For other solutions divide both sides by $|z|^2$ which gives
  us $t^2 + t + 1 = 0$ where $t = \frac{z}{|z|}$.

  The equation $t^2 + t + 1 = 0$ has two roots i.e. $t = \omega, \omega^2 \Rightarrow \frac{z}{|z|} =
  \omega, \omega^2\Rightarrow z = k\omega, k\omega^2$ where $k = |z|$ is a non-negative real number.
\item Let $z = x + iy$, then $(x + iy)\sqrt{x^2 + y^2} + a(x + iy) + 1 = 0$.
  Comparing real and imaginary parts, we get

  $y\sqrt{x^2 + y^2} + ay = 0 \Rightarrow y = 0\;\because \sqrt{x^2 + y^2} + a \neq 0\;[\because a > 0]$ and
  $\therefore x\sqrt{x^2 + 0} + ax + 1 = 0 \Rightarrow x^2 + ax + 1 = 0 \Rightarrow x = \frac{-a
    \pm\sqrt{a^2 - 4}}{2}$

  Clearly, both the values of $x$ are negative, so $z$ is a negative real number.
\item Let $z = x + iy$, then $x^2 + y^2 - 2i(x + iy) + 2a(1 + i) = 0$.
  Comparing real and imaginary parts, we get

  $x^2 + y^2 + 2y + 2a = 0 \Rightarrow x^2 + (y - 1)^2 = 1 - 2a$ and $-2x + 2a = 0 \Rightarrow x = a$

  $\Rightarrow (y - 1)^2 = 1 - 2a - a^2 \Rightarrow y = 1 \pm \sqrt{1 - 2a - a^2}$.
  However $1 - 2a - a^2 > 0$. Roots of equivalent quadratic equation is $a = \frac{2 \pm \sqrt{8}}{-2}
  \Rightarrow -1\pm \sqrt{2}$ but $a > 0$ so the range for $a$ is $0 < a < \sqrt{2} - 1$.
\item i. We have $(3 + 4i)^x = 5^{\frac{x}{2}}$. Squaring both sides $(-7 + 24i)^x = 5^x \Rightarrow
  \left(\frac{-7 + 24i}{5}\right)^x = 1$ which is possible only if $x = 0$.

  ii. Given $(1 - i)^x = 2^x \Rightarrow \left(\frac{1 - i}{2}\right)^x  = 1$ which is possible only if $x =
  0$.

  iii. Given $(1 - i)^x = (1 + i)^x \Rightarrow \left(\frac{1 - i}{1 + i}\right)^x = 1 \Rightarrow (-i)^x =
  1 \Rightarrow x = 0, 4, 8, \ldots, 4n\;\forall\;4n\in I$.
\item $z^3 + 2z^2 + 2z + 1 = 0\Rightarrow (z + 1)(z^2 + z + 1) = 0 \Rightarrow z = -1, \omega, \omega^2$.

  When $z = -1, z^{1985} + z^{100} + 1 = -1 + 1 + 1 = 1 \neq 0$, when $z = \omega, \omega^{1985} +
  \omega^{100} + 1 = \omega^2 + \omega + 1 = 0$ and when $z = \omega^2, \omega^{1985*2} + \omega^{200} + 1 =
  \omega + \omega^2 + 1 = 0$. Thus common roots are $\omega, \omega^2$.
\item Adding all equations $\alpha + \beta + \gamma = 3z_1 \Rightarrow z_1 = \frac{\alpha + \beta +
  \gamma}{3}$. Similarly, multiplying second equatin with $\omega$ and third equation with $\omega^2$, and
  then adding we have $z_3 = \frac{\alpha + \beta\omega + \gamma\omega^2}{3}$. Similarly, $z_2 =
  \frac{\alpha + \beta\omega^2 + \gamma\omega}{3}$.

  $|\alpha|^2 = \alpha\overline{\alpha} = (z_1 + z_2 + z_3)(\overline{z_1} + \overline{z_2} +
  \overline{z_3}), |\beta|^2 = \beta\overline{\beta} = (z_1 + z_2\omega + z_3\omega^2)(\overline{z_1} +
  \overline{z_2}\omega^2 + \overline{z_3}\omega)$ and $\|\gamma|^2 = \gamma\overline{\gamma} = (z_1 +
  z_2\omega^2 + z_3\omega)(\overline{z_1} + \overline{z_2}\omega + \overline{z_3}\omega^2)\;[\because
    \overline{\omega} = \omega^2 \& \overline{\omega^2} = \omega]$

  $\Rightarrow |\alpha|^2 + |\beta|^2 + |\gamma|^2 = 3(|z_1|^2 + |z_2|^2 + |z_3|^2) + z_1[\overline{z_2}(1 +
    \omega + \omega^2) + \overline{z_3}(1 + \omega + \omega^2)] + z_2[\overline{z_1}(1 + \omega + \omega^2)
    + \overline{z_2}(1 + \omega + \omega^2)] + z_3[\overline{z_1}(1 + \omega + \omega^2) + \overline{z_2}(1
    + \omega + \omega^2)] = 3(|z_1|^2 + |z_2|^2 + |z_3|^2) =$ R.H.S.
\item Let $f(x) = (x + 1)^n - x^n - 1. x^3 + x^2 + x = 0 \Rightarrow x(x^2 + x + 1) = 0\Rightarrow x = 0,
  \omega, \omega^2$. So for $x^3 + x^2 + x$ to be a factor of $f(x), f(0) = 0, f(\omega) = 0, f(\omega^2) =
  0$.

  $f(0) = 1^n - 1 = 0, f(\omega) = (\omega + 1)^n - \omega^n - 1 = -\omega^{2n} - \omega^n - 1\;[\because n$
    is odd. $] = -(1 + \omega^n + \omega^{2n}) = 0$. Similarly, $f(\omega^2) = 0$. Hence proved.
\item Let $f(x, y) = (x + y)^n - x^n - y^n. xy(x + y)(x^2 + xy + y^2) = 0 \Rightarrow x = 0, y = 0, x = -y, y =
  x\omega, y = x\omega^2$. When $x = 0, f(x, y) = 0; y = 0, f(x, y) = 0; y = -x \Rightarrow f(x, y) = -x^n -(-x)^n =
  0 [\because n = 2m + 1\;\forall\;m\in\mathbb{I}], y = xw \Rightarrow f(x, y) = [x^n(1 + \omega)^n - x^n -
    x^n\omega^n] = -x^n\omega^{2n} - x^n - x^n\omega^n = 0$, and similarly when $y = x\omega^2, f(x, y) =
  0$. Hence proved.
\item R.H.S. $= \left|\frac{1}{z_1} + \frac{1}{z_2} + \cdots + \frac{1}{z_n}\right| =
  \left|\frac{\overline{z_1}}{|z_1|^2} + \frac{\overline{z_2}}{|z_2|^2} + \cdots +
  \frac{\overline{z_n}}{|z_n|^2}\right|$

  $= |\overline{z_1} + \overline{z_2} + \cdots + \overline{z_n}| = |\overline{z_1 + z_2 + \cdots + z_n}| =
  |z_1 + z_2 + \cdots + z_n| =$ R.H.S.
\item For any two complex numbers $z_1$ and $z_2$, we know that $|z_1 + z_2|^2 + |z_1 - z_2|^2 = 2|z_1|^2 +
  2|z_2|^2$. Let $z_1 = \alpha + \sqrt{\alpha^2 - \beta^2}$ and $z_2 = \alpha - \sqrt{\alpha^2 - \beta^2}$.

  Now $(|z_1| + |z_2|)^2 = |z_1|^2 + |z_2|^2 + 2|z_1||z_2| = 2|\alpha|^2 + 2|\alpha^2 - \beta^2| +
  2|\beta|^2 = |\alpha + \beta|^2 + |\alpha - \beta|^2 + 2|\alpha + \beta||\alpha -\beta|$

  $= (|\alpha + \beta| + |\alpha - \beta|)^2\Rightarrow |z_1| + |z_2| = |\alpha + \beta| + |\alpha - \beta|
  =$ R.H.S.
\item $|z_1| = |z_1| = 1 \Rightarrow a^2 + b^2 = c^2 + d^2 = 1, z_1\overline{z_2} = ac + bd + i(bc -
  ad)\;\because\;\Re(z_1\overline{z_2}) = 0 \Rightarrow ac + bd = 0 \Rightarrow \frac{a}{d} = -\frac{b}{c} =
  k$ (say). $\therefore a = kd, b = -kc$.

  $\therefore k^2d^2 + k^2c^2 = 1 \Rightarrow k^2 = 1 \Rightarrow k = \pm 1$. Now $|\omega_1| = \sqrt{a^2 +
    c^2} = \sqrt{a^2 + b^2} = 1, |\omega_2| = \sqrt{b^2 + d^2} = \sqrt{a^2 + b^2} = 1,
  \omega_1\overline{\omega_2} = (a + ic)(b - id) \therefore \Re(\omega\overline{\omega_2}) = ab + cd = 0$.
\item Given, $\left|\frac{z_1 - z_2}{1 - \overline{z_1}z_2}\right|< 1\Leftrightarrow \left|\frac{z_1 -
  z_2}{1 - \overline{z_1}z_2}\right|^2 < 1 \Leftrightarrow |z_1 - z_2|^2 < |1 - \overline{z_1}z_2|^2$

  $\Leftrightarrow (z_1 - z_2)\overline{(z_1 - z_2)} < (1 - \overline{z_1}z_2)\overline{(1 -
  \overline{z_1}z_2)} \Leftrightarrow (z_1 - z_2)(\overline{z_1} - \overline{z_2}) < (1 -
  \overline{z_1}z_2)((1 - z_1\overline{z_2}))$

  $\Leftrightarrow |z_1|^2 + |z_2|^2 > 1 + |z_1|^2|z_2|^2 \Leftrightarrow 1 - |z_1|^2 - |z_2|^2 +
  |z_1|^2|z_2|^2 > 0 \Leftrightarrow (1 - |z_1|^2)(1 - |z_2|^2) > 0 \Rightarrow (1 + |z_1|)(1 - |z_1|)(1 +
  |z_2|)(1 - |z_2|) > 0$

  $\Leftrightarrow (1 - |z_1|)(1 - |z_2|) > 0$ which is true as $|z_1| < 1$ and $|z_2| < 1$.
\item Let $z = x + iy$ then $\frac{z - z_1}{z - z_2} = \frac{(x - 10) + i(y - 6)}{(x - 4) + i(y -
  6)}$. Rationalizing $\frac{x^2 - 14x + 40 + (y - 6)^2}{(x - 4)^2 + (y - 6)^2} + \frac{i6(y - 6)}{(x - 4)^2
  + (y - 6)^2} = a + ib$ (say)

  $\because \arg(a + ib) = \frac{\pi}{4}\Rightarrow x^2 - 14x + 40 + (y - 6)^2 = 6(y - 6)\Rightarrow x^2 +
  y^2 - 14x - 18y + 112 = 0 \Rightarrow |z - 7 - 9i|^2 = 18$. Hence proved.
\item Let $z = x + iy$ then $\frac{3z - 6 - 3i}{2z - 8 - 6i} = \frac{x - 6 + i(3y - 3)}{2x - 8 + i(2y -
  6)}$. Rationalizing $\frac{6x^2 + 6y^2 - 36x - 24y + 66 + i(12x - 12y - 12)}{(2x - 8)^2 + (2y - 6)^2} = a
  + ib$ (let)

  $\because\arg(a + ib) = \frac{\pi}{4} \Rightarrow 6x^2 + 6y^2 - 36x - 24y + 66 = 12x - 12y - 12
  \Rightarrow x^2 + y^2 - 8x - 2y + 13 = 0$. Also given, $|z - 3 + i| = 3 \Rightarrow x = -2y +
  6$. Substituting this in previously obtained equation, we have

  $5y^2 - 10y + 1 = 0 \Rightarrow y = 1 \pm\frac{2}{\sqrt{5}}\Rightarrow x = 4\mp\frac{4}{\sqrt{5}}$. Hence
  we have our $z$.
\item Let $|z_1| = r_1, |w| = r_2, \arg(z_1)= \theta_1$ and $\arg(w) = \theta_2$. Then, $|z - w|^2 =
  (r_1\cos\theta_1 - r_1\sin\theta_1)^2 + (r_2\cos\theta_2 - r_2\sin\theta_2)^2 = (r_1 - r_2)^2 + 2r_1r_2 -
  2r_1r_2\cos(\theta_1 - \theta_2)$

  $= (r_1 - r_2)^2 + 4r_1r_2\sin^2\frac{\theta_1 - \theta_2}{2}\leq (r_1 - r_2)^2 +
  2.1.1.2\left(\frac{\theta_1 - \theta_2}{2}\right)^2 = (|z| - |w|)^2 + (\theta_1 - \theta_2)^2$. Hence
  proved.
\stopitemize