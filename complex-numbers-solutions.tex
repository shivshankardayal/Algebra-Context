% -*- mode: context; -*-
\chapter{Complex Numbers}
\startitemize[n, 1*broad]
\item Let $z = 7 + 8i$, and $\sqrt{z} = \sqrt{7 + 8i} = x + iy$. Squaring $7 + 8i = (x^2 - y^2) + 2ixy$
  Comparing real and imaginary parts $x^2 - y^2 = 7, xy = 4 \Rightarrow x^2 + y^2 = \sqrt{113}$. We discaard
  $-\sqrt{113}$ as that will make $x, y$ complex.

  $\Rightarrow x = \frac{\sqrt{7 + \sqrt{113}}}{2}, y = \frac{\sqrt{\sqrt{113} - 7}}{2}$.
\item Let $\sqrt{a^2 - b^2 + 2abi} = x + iy$, then on squaring and comparison of real and imaginary parts,
  we have $x^2 - y^2 = a^2 - b^2, xy = ab\Rightarrow x^2 + y^2 = a^2 + b^2\Rightarrow x = a, y = b$.
\item $\sqrt[4]{81i^2} = \sqrt{\pm9i}$ and now we can solve it like previous problems.
\item Let $z = \frac{x^2}{y^2} + \frac{y^2}{x^2} + \frac{1}{2i}\left(\frac{x}{y} + \frac{y}{x}\right) +
  \frac{31}{16} = \left(\frac{x}{y} + \frac{y}{x}\right)^2 - 2\frac{i}{4}\left(\frac{x}{y} +
  \frac{y}{x}\right) + \frac{i^2}{4} = \left(\frac{x}{y} + \frac{y}{x} - \frac{i}{4}\right)^2$

  $\therefore$ square root $=\pm\left(\frac{x}{y} + \frac{y}{x} - \frac{i}{4}\right)$.
\item We know that $i^4 = 1$. Let $z = i^{n + 80} + i^{n + 50} = i^{n + 4.20} + i^{n + 12.4 + 2} = i^n +
  i^{n + 2} = i^n - i^n = 0$.
\item Let $z = \left(i^{17} + \frac{1}{i^{15}}\right)^3 = \left(i^{4.4 + 1} + \frac{1}{i^{4.4 - 1}}\right)^3
  = (i + i)^3 = 8i^3 = -8i$.
\item Let $z = \frac{(1 + i)^2}{2 + 3i} = \frac{2i}{2 + 3i}.\frac{2 - 3i}{2 - 3i} = \frac{-6 + 4i}{13}$.
\item Let $z = \left(\frac{1}{1 + i} + \frac{1}{1 - i}\right)\frac{7 + 8i}{7 - 8i} = \frac{2}{1 - i^2}\frac{(7 +
  8i)(7 + 8i)}{(7 - 8i)(7 + 8i)} = \frac{2}{2}\frac{-15 + 112i}{49 + 64} = \frac{-15 + 112i}{113}$.
\item Let $z = \frac{(1 + i)^{4n + 7}}{(1 - i)^{4n - 1}} = \frac{(1 + i)^{4(n + 2) - 1}}{(1 - i)^{4n - 1}} =
  \frac{1 - i}{1 + i} = \frac{(1 - i)^2}{1 - i^2} = \frac{1 - 2i + i^2}{2} = -i$.
\item Let $z = \frac{1}{1 - \cos\theta + 2i\sin\theta} = \frac{1 - \cos\theta - 2i\sin\theta}{(1 -
  \cos\theta)^2 + 4\sin^2\theta} = \frac{1 - \cos\theta - 2i\sin\theta}{1 - 2\cos\theta + 1 +
  3\sin^2\theta} = \frac{1 - \cos\theta - 2i\sin\theta}{2 - 2\cos\theta + 3\sin^2\theta}$.
\item Let $z = \frac{(\cos x + i\sin x)(\cos y + i\sin y)}{(\cot u + i)(i + \tan v)}$. Using Euler's
  formula, we have $z = \frac{e^{ix}.e^{iy}}{\frac{e^{iu}}{\sin u}.\frac{e^{iv}}{\cos v}} = \sin u\cos
  v.e^{i(x + y - u - v)} = \sin u\cos v\cos(x + y - u - v) + i\sin u\cos v\sin(x + y - u - v)$.
\item $i^5 = i^{4 + 1} = i$.
\item $i^{67} = i^{64 + 3} = i^3 = -i[\because i^2 = -1]$.
\item $i^{-59} = \frac{1}{i^{15.4 - 1}} = i$.
\item $i^{2014} = i^{4.503 + 2} = i^2 = -1$.
\item $|a| = -a \Rightarrow \sqrt{ab} = \sqrt{|a|b}i$.
\item Let $z = i^n + i^{n + 1} + i^{n + 2} + i^{n + 3} = i^n + i.i^n - i^n - i.i^n = 0$.
\item $\displaystyle\sum_{n = 1}^{13}(i^n + i^{n + 1}) = \sum_{i=1}^{13}i^n + \sum_{i=1}^{13}i^{n + 1} = (i
  + i^2 + i^3 + \cdots + i^{13}) + (i^2 + i^3 + i^4 + \cdots + i^{14}) = i - 1$.
\item $\frac{2^n}{(1 + i)^{2n}} + \frac{(1 + i)^{2n}}{2^n} = \frac{2^n}{(1 + i^2 + 2i)^n} + \frac{(1 + i^2 +
  2i)^n}{2^n} = \frac{1}{i^n} + i^n = \frac{i^n}{i^{2n}} + i^n = i^n\left(\frac{1}{(-1)^n} + 1\right) =
  i^n[(-1)^n + 1]$.
\item Let $z = i^n + \frac{1}{i^n} = \frac{i^{2n} + 1}{i^n}$. Substituting $n = 1, 2, 3, 4$, $z = 0, \pm2$
  i.e. there exists three different solutions.
\item $4x + (3x - y)i = 3 - 6i$. Comparing real and imaginary parts, $4x = 3, 3x - y = -6\Rightarrow x =
  \frac{3}{4}\Rightarrow \frac{9}{4} - y = -6 \Rightarrow y = \frac{33}{4}$.
\item $\left(\frac{1}{3} + i\frac{7}{3}\right) + \left(4 + i\frac{1}{3}\right) - \left(-\frac{4}{3} +
  i\right) = \left(\frac{1}{3} + 4 + \frac{4}{3}\right) + i\left(\frac{7}{3} + \frac{1}{3} - 1\right) =
  \frac{17}{3} + i\frac{5}{3}$.
\item $\frac{(1 + i)x - 2i}{3 + i} + \frac{(2 - 3i)y + i}{3 - i} = i \Rightarrow [(1 + i)x - 2i](3 - i) +
  [(2 - 3i)y + i](3 + i) = i(3 + i)(3 - i)\Rightarrow (4x + 9y - 3) + i(2x - 7y - 3) = 10i$. Equating real
  and imaginary parts, $4x + 9y = 3, 2x - 7y = 13 \Rightarrow x = 3, y = -1$.
\item The multiplicative inverse is $\frac{1}{z} = \frac{1}{4 - 3i} = \frac{1}{4 - 3i}.\frac{4 + 3i}{4 + 3i}
  = \frac{4 + 3i}{25}$.
\item Let $x_1 = 2, y_1 = 3, x_2 = 1$ and $y_2 = 12$. $\therefore \frac{z_!}{z_2} = \frac{[(x_1x_2 + y_1y_2)
    + i(x_2y_1 - x_1y_2)]}{x_2^2 + y_2^2} = \frac{8 - i}{5}$.
\item $z_1 = z_2 \Rightarrow 9y^2 - 4 -10xi = 8y^2 -20i$. Equating real and imaginary parts, $9y^2 - 4 =
  8y^2 \Rightarrow y = \pm2$ and $-10x = -20 \Rightarrow x = -2 \Rightarrow z = x + iy = -2\pm2i$.
\item Let $z = x + iy$ then $|x + iy + 1| = x + iy + 2(1 + i) \Rightarrow \sqrt{(x + 1)^2 + y^2} = (x + 2) +
  i(y + 2)$. Equating real and imaginary parts, $y + 2 = 0 \Rightarrow y = -2$ and $(x + 1)^2 + y^2 = (x +
  2)^2 \Rightarrow x^2 + 2x + 5 = 4 = x^2 + 4x + 4 \Rightarrow x = \frac{1}{2} \Rightarrow z = \frac{1 -
    4i}{2}$.
\item Let $z = \frac{1 + 2i}{1 - 3i} = \frac{(1 + 2i)(1 + 3i)}{1 - (3i)^2} = \frac{1 + 3i + 2i + 6i^2}{1 +
  9} = \frac{-5 + 5i}{10} = -\frac{1}{2} + \frac{1}{2}i$

  $\Rightarrow |z| = \sqrt{\left(-\frac{1}{2}\right)^2 + \frac{1}{2^2}} = \frac{1}{\sqrt{2}}$

  $\tan\theta = \frac{\frac{1}{2}}{-\frac{1}{2}} \Rightarrow \theta = \tan^{-1}-1 = \frac{3\pi}{4}$.
\item Given, $\frac{x - 3}{3 + i} + \frac{y - 3}{3 - i} = i(3 - i)(3 + i)\Rightarrow (x - 3)(3 - i) + (y -
  3)(3 + i) = 10i\Rightarrow 3x - 9 + i(3 - x) + (3y - 9) + i(y - 3) = 10i$

  Comparing real and imaginary parts, we get $3x + 3y - 18 = 0$ and $y - x = 10 \Rightarrow x = -2, y = 8$.
\item $(1 + i)^2 = 1 + 2i - i = 2i\Rightarrow (1 + i)^{50} = (2i)^{25} = 2^{25}i^{4.6 + 1} = 2^{25}i$ Thus,
  real part will be $0$.
\item Let $z = x + iy$ then $x + iy + \sqrt{x^2 + y^2} = 2 + 8i$, Comparing real and imaginary parts, we get
  $y = 8$ and $x + \sqrt{x^2 + y^2} = 2 \Rightarrow \sqrt{x^2 + y^2} = 2 - x$

  $\Rightarrow x^2 + 64 = 4 - 4x + x^2 \Rightarrow x = -15\Rightarrow z = -15 + 8i$.
\item $S = i + 2i^2 + 3i^3 +\ldots + 100i^{100}\Rightarrow iS = i^2 + 2i^3 + \ldots + 99i^{100} + 100i^{101}$

  $\Rightarrow S(1- i) = i + i^2 + \ldots + i^{100} - 100i^{101} = \frac{i(1 - i^{101})}{1 - i} - 100i^{101}$

  $S = \frac{i(1 - i^{101})}{(1- i)^2} - \frac{100i^{101}}{1 - i}$.
\item Consider $t_1 = \frac{1}{1 + i} + \frac{1}{1 - i} + \frac{1}{-1 + i} + \frac{1}{-1 - i} = \frac{1 + i
  + 1 - i}{1^2 - i^2} + \frac{-1 + i - 1 - i}{(-1)^2 - i^2} = \frac{2}{2} + \frac{-2}{2} = 0$

  $t_2 = 2\left(\frac{1}{1 + i} + \frac{1}{1 - i} + \frac{1}{-1 + i} + \frac{1}{-1 - i}\right) = 0$

  Similarly all other terms and sum will be zero.
\item Given, $z^2 - z - 5 + 5i = 0 \Rightarrow D = (-1)^2 - 4.1.(-5 + 5i) = 21 - 20i$ and we will need
  $\sqrt{D}$

  $\sqrt{D} = \sqrt{b^2 - 4ac} = \sqrt{21 - 20i} = \pm \left[\sqrt{\frac{x^2 + y^2 + x}{2}} -
  i\sqrt{\frac{x^2 + y^2 - x}{2}}\right] = \pm(5 - 2i)$

  $z = \frac{1 + 5 - 2i}{2}$ or $z = \frac{1 - 5 + 2i}{2}\Rightarrow z = 3 - i, i - 2$

  Thus, product of real parts $= -2\times 3 = -6$
\item Given, $z^3 = -\overline{z}\Rightarrow |z|^3 = |z|\Rightarrow |z|(|z| - 1)(|z| + 1) = 0 \Rightarrow
  |z| = 0, |z| = 1 [\because |z| + 1 > 0]$

  If $|z| = 0,$ then $z = 0.$ If $|z| = 1 \Rightarrow |z|^2 = 1\Rightarrow z\overline{z} = 1\Rightarrow z^3
  + \frac{1}{z} = 0 \Rightarrow z^4 + 1 = 0,$ which has four distinct roots. Thus, given equation has five
  roots.
\item Since we have to find real roots, let $z = x,$ a real value. The given equation becomes $x^3 + ix - 1=
  0\Rightarrow x^3 = 1, x = 0$ which is not possible. So there are no real solutions.
\item Let $z = x + iy,$ then $\sqrt{x^2 + y^2} > 1,$ because point $A$ is outside circle.

  $\frac{1}{z} = \frac{x - iy}{\sqrt{x^2 + y^2}}$ so $\frac{x}{\sqrt{x^2 + y^2}}, \frac{-y}{x^2 + y^2}< 1$

  This leads to the fact that point $E$ is reciprocal of point $A$.
\item $z = (3p - 7q) + i(3q + 7p),$ which is purely imaginary, $\Rightarrow 3p - 7q = 0$

  $\Rightarrow \frac{p}{q} = \frac{7}{3} \Rightarrow \frac{p}{q} + i = \frac{7}{3} + i \Rightarrow \frac{p +
  iq}{q} = \frac{7 + 3i}{3}$

  $\Rightarrow p+ iq = 7 + 3i \Rightarrow z = 21 + 9i + 49i - 21 = 58i \Rightarrow |z|^2 = 3364$.
\item Given, $\alpha = \left(\frac{a - ib}{a + ib}\right)^2 + \left(\frac{a + ib}{a - ib}\right)^2 =
  \frac{(a - ib)^4 + (a + ib)^4}{(a - ib)^2(a + (ib)^2)}$

  $= \frac{a^4 - 4a^3.ib + 6a^2i^2b^2 - 4ai^3b^3 + b^4 + a^4 + 4a^3ib + 6a^2i^2b^2 + 4ai^3b^3 + b^4}{(a^2 +
    b^2)^2} = \frac{2a^4 - 12a^2b^2 + 2b^4}{(a^2 + b^2)^2}$, which is purely real.
\item Let $z = x + iy$ then given $|z| = 1 \Rightarrow x^2 + y^2 = 1$

  Let $\beta = \frac{z - 1}{z + 1} = \frac{(x - 1) + iy}{(x + 1) + iy} = \frac{(x - 1) + iy}{(x + 1) +
  iy}.\frac{(x + 1) - iy}{(x + 1) - iy}$

  $= \frac{x^2 - 1 + y^2 + iy(x + 1 -x + 1)}{(x + 1)^2 + y^2} = \frac{2iy}{(x + 1)^2 + y^2}$ which is purely
  imaginary.
\item Let $z = x + iy \Rightarrow x^2 + (y - 3)^2 = 9 \Rightarrow x = 3\cos\theta, y = 3\sin\theta + 3$

  $z = 3[\cos\theta + i(\sin\theta + 1)] = 3\left[\sin\left(\frac{\pi}{2} - \theta\right) + i\left(1 +
  \cos\left(\frac{\pi}{2} - \theta\right)\right)\right]$

  $= 3\left[2\sin\left(\frac{\pi}{4} - \frac{\theta}{2}\right)\cos\left(\frac{\pi}{4} -
  \frac{\theta}{2}\right) + i2\cos^2\left(\frac{\pi}{4} - \frac{\theta}{2}\right)\right]$

  $= 6\cos\left(\frac{\pi}{4} - \frac{\theta}{2}\right)\left[\sin\left(\frac{\pi}{4} -
  \frac{\theta}{2}\right) + i\cos\left(\frac{\pi}{4} - \frac{\theta}{2}\right)\right]
  = 6\cos\left(\frac{\pi}{4} - \frac{\theta}{2}\right)e^{i\left(\frac{\pi}{4} + \frac{\theta}{2}\right)}$

  $\cot\left(arg(z)\right) = \cot\left(\frac{\pi}{4} + \frac{\theta}{2}\right) = \tan\left(\frac{\pi}{4} -
  \frac{\theta}{2}\right)$

  $\frac{6}{z} = \sec\left(\frac{\pi}{4} - \frac{\theta}{2}\right)e^{-i\left(\frac{\pi}{4} +
    \frac{\theta}{2}\right)} = \sec\left(\frac{\pi}{4} - \frac{\theta}{2}\right)
  \left[\sin\left(\frac{\pi}{4} - \frac{\theta}{2} - i\cos\left(\frac{\pi}{4} -
    \frac{\theta}{2}\right)\right)\right]$

  $= \tan\left(\frac{\pi}{4} - \frac{A}{2}\right) - i \Rightarrow \cot(\arg(z)) - \frac{6}{z} = i$.
\item Let $z = r(\cos\theta + i\sin\theta) = \frac{-16}{1 + \sqrt{3}} = \frac{-16}{1 + i\sqrt{3}}.\frac{1 -
  i\sqrt{3}}{1 - i\sqrt{3}} = \frac{-16(1 - i\sqrt{3})}{1 + 3}$

  $= -4 + i4\sqrt{3}$ then $r\cos\theta = 4, r\sin\theta = 4\sqrt{3} \Rightarrow r^2 = 64 \Rightarrow r = 4,
  \cos\theta = \frac{-1}{2}, \sin\theta = \frac{\sqrt{3}}{2} \Rightarrow \theta = \frac{2\pi}{3}$

  $\Rightarrow z = 8\left(\cos\frac{2\pi}{3} + i\sin\frac{2\pi}{3}\right)$.
\item Let $z = r(\cos\theta + i\sin\theta)$ then because $arg(z) + arg(w) = \pi \Rightarrow arg(w) = \pi -
  \theta$

  $\Rightarrow w = r(-\cos\theta + i\sin\theta) = -r(\cos\theta - i\sin\theta)\;\therefore\;r =
  -\overline{w}$.
\item $x - iy = \sqrt{\frac{a - ib}{c - id}} \Rightarrow x^2 - y^2 - 2ixy = \frac{a - ib}{c - id} = \frac{(a
  - ib)(c + id)}{c^2 + d^2}\Rightarrow x^2 - y^2 - 2ixy = \frac{(ac + bd) -i(bc - ad)}{c^2 + d^2}$

  Comparing real and imaginary parts, we get $x^2 - y^2 = \frac{ac + bd}{c^2 + d^2}, 2xy = \frac{bc -
    ad}{c^2 + d^2}$

  $\Rightarrow (x^2 + y^2)^2 = (x^2 - y^2)^2 + 4x^2y^2 = \frac{(ac + bd)^2 + (bc - ad)^2}{(c^2 + d^2)} = \frac{a^2c^2 +
    b^2d^2 + b^2c^2 + a^2d^2}{(c^2 + d^2)^2} = \frac{a^2 + b^2}{c^2 + d^2}$.
\item We know that for two complex numbers $z_1$ and $z_2, |z_1| + |z_2|\geq |z_1 - z_2|$

  $|z| + |z - 2| \geq |z - (z - 2)| = |2| = 2$.  Therefore, minimum value is $2$.
\item $|z_1 + z_2 + z_3| = |(z_1 - 1) + (z_2 - 2) + (z_3 - 3) + 6\leq |z_1 - 1| + |z_2 - 2| + |z_3 - 3| + 6$

  $< 1 + 2+ 3 + 6 = 12$.  Thus, maximum value of $|z_1 + z_2 + z_3|$ is $12.$
\item $|\alpha + \beta|^2 = (\alpha + \beta)(\overline{\alpha + \beta}) = (\alpha + \beta)(\overline{\alpha}
  + \overline{\beta}) = \alpha\overline{\alpha} + \alpha\overline{\beta} + \overline{\alpha}\beta +
  \beta\overline{\beta} = |\alpha|^2 + |\beta|^2 + \alpha\overline{\beta} + \overline{\alpha}\beta$

  Similarly, $|\alpha - \beta|^2 = |\alpha|^2 + |\beta|^2 - \alpha\overline{\beta} - \overline{\alpha}\beta$

  Thus, $|\alpha|^2 + |\beta|^2 = \frac{1}{2}(|\alpha + \beta|^2 + |\alpha - \beta|^2)$
\item If $|z| = 0$ then $\sqrt{x^2 + y^2} = 0 \Rightarrow x^2 + y^2 = 0$

  Above is possible if and only if $x = 0$ and $y = 0 \Rightarrow z = 0$.
\item $\frac{z_1z_2}{\overline{z_1}} = \frac{(1 - i)(2 + 7i)}{1 + i} = \frac{2 + 7 -2i + 7i}{1 + i} =
  \frac{9 + 5i}{1 + i} = \frac{9 + 5i}{1 + i}.\frac{1 - i}{1 - i} = \frac{9 + 5 + 5i -9i}{2} = 7 -
  2i\therefore Im\left(\frac{z_1z_2}{\overline{z_1}}\right) = -2$.
\item $|z + 12 - 6i| \leq |z - i| + |12 - 5i| < 1 + 13 = 14$.
\item Given, $|z + 6| = |2z + 3|,$ let $z = x + iy \Rightarrow (x + 6)^2 + y^2 = (2x + 3)^2 + 4y^2
  \Rightarrow x^2 + 12x + 36 + y^2 = 4x^2 + 12x + 9 + 4y^2$

  $\Rightarrow 3x^2 + 2y^2 = 27 \Rightarrow x^2 + y^2 = 9 \Rightarrow |z| = 3$.
\item Given $\sqrt{a - ib} = x - iy,$ squaring we get $a - ib = x^2 - y^2 - 2ixy$. Comparing real and
  imaginary parts, we get
  $a = x^2 - y^2, b = 2xy \Rightarrow a + ib = x^2 - y^2 + 2ixy = x^2 +i^2y^2 + 2ixy\Rightarrow \sqrt{a +
    ib} = x + iy$.
\item $x_1x_2x_3\ldots\infty = \left(\cos\frac{\pi}{2} + i\sin\frac{\pi}{2}\right)\left(\cos\frac{\pi}{2^2}
  + i\sin\frac{\pi}{2^2}\right) \ldots\infty= \cos\left(\frac{\pi}{2} + \frac{\pi}{2^2} + \ldots
  \infty\right) + i\sin\left(\frac{\pi}{2} + \frac{\pi}{2^2} + \ldots \infty\right)$

  $= \cos\frac{\pi}{2}.\frac{1}{1- \frac{1}{2}} + i\sin\frac{\pi}{2}.\frac{1}{1- \frac{1}{2}} = \cos\pi +
  i\sin\pi = -1$.
\item Given, $\frac{(\cos\theta + i\sin\theta)^4}{(\sin\theta + i\cos\theta)^5} = \frac{(\cos\theta +
  i\sin\theta)^4}{i^5\left(\frac{1}{i}\sin\theta + \cos\theta\right)^5}$

  $= \frac{(\cos\theta + i\sin\theta)^4}{i(\cos\theta - i\sin\theta)^5} = \frac{(\cos\theta +
  i\sin\theta)^4}{i(\cos\theta + i\sin\theta)^{-5}}= \frac{1}{i}(\cos\theta + i\sin\theta)^9 = \sin9\theta
  -i \cos9\theta$.
\item $z = \left[\cos\frac{\pi}{6} + i\sin\frac{\pi}{6}\right]^5 + \left[\cos\frac{\pi}{6} - i\sin\frac{\pi}{6}\right]^5$

  $= \cos\frac{5\pi}{6} + i\sin\frac{5\pi}{6} + \cos\frac{5\pi}{6} - i\sin\frac{5\pi}{6} =
  2\cos\frac{5\pi}{6}\;\therefore Im(z) = 0$.
\item $z = \left(\cos\frac{\pi}{3} + i\sin\frac{\pi}{3}\right)^{\frac{3}{4}} = \left(\cos\pi +
  i\sin\pi\right)^{\frac{1}{4}},$ thus general root is $\cos\frac{2n\pi + \pi}{4} + i\sin\frac{2n\pi +
    \pi}{4}$

  Thus, substituting $n = 0, 1, 2, 3$ we find four roots and the product is

  $\left(\cos\frac{\pi}{4} + i\sin\frac{\pi}{4}\right)\left(\cos\frac{3\pi}{4} +
  i\sin\frac{3\pi}{4}\right)\left(\cos\frac{5\pi}{4} + i\sin\frac{5\pi}{4}\right)\left(\cos\frac{7\pi}{4} +
  i\sin\frac{7\pi}{4}\right)$

  $= \left(\frac{1}{\sqrt{2}} + \frac{i}{\sqrt{2}}\right)\left(\frac{-1}{\sqrt{2}} +
  \frac{i}{\sqrt{2}}\right)\left(\frac{-1}{\sqrt{2}} - \frac{i}{\sqrt{2}}\right)\left(\frac{1}{\sqrt{2}} -
  \frac{i}{\sqrt{2}}\right)$

  $= \left(-\frac{1}{2} - \frac{1}{2}\right)\left(\frac{-1}{2} - \frac{1}{2}\right) = -1.-1 = 1$.
\item Let $z_1 = r_1(\cos x + i\sin x)$ and $z_2 = r_2(\cos y + i\sin y)$ Then $(r_1\cos x + r_2\cos y)^2 +
  (r_1\sin x+ r_2\sin y)^2 = r_1^2 + r_2^2 + 2r_2r_2$

  $\Rightarrow 2r_1r_2(\cos x\cos y + \sin x\sin y) = 2r_2r_2 \Rightarrow \cos(x - y) = 1 \Rightarrow x - y
  = 0 \Rightarrow \arg(z_1) - arg(z_2) = 0$.
\item Let $z = 1 - \sin\alpha + i\cos\alpha = r(\cos\theta + i\sin\theta),$ then $r = \sqrt{(1 -
  \sin\alpha)^2 + \cos^2\alpha} = \sqrt{2 - 2\sin\alpha}$

  $\tan\theta = \frac{\cos\alpha}{1 - \sin\alpha} = \frac{1 - \tan^2\frac{\alpha}{2}}{1 +
  \tan^2\frac{\alpha}{2} - 2\tan\frac{\alpha}{2}} = \frac{1 + \tan\frac{\alpha}{2}}{1 -
  \tan\frac{\alpha}{2}} = \tan\left(\frac{\pi}{4} -\frac{\alpha}{2}\right)\Rightarrow \theta = \frac{\pi}{4}
  - \frac{\alpha}{2}$.
\item Let $z = \left[\frac{1 + \sin\frac{\pi}{8} + i\cos\frac{\pi}{8}}{1 + \sin\frac{\pi}{8} -
    i\cos\frac{\pi}{8}}\right] = \left[\frac{1 + \sin\frac{\pi}{8} + i\cos\frac{\pi}{8}}{1 +
    \sin\frac{\pi}{8} - i\cos\frac{\pi}{8}}\right]. \left[\frac{1 + \sin\frac{\pi}{8} +
    i\cos\frac{\pi}{8}}{1 + \sin\frac{\pi}{8} + i\cos\frac{\pi}{8}}\right]$

  $= \frac{\left(1 + \sin\frac{\pi}{8}\right)^2 - \cos^2\frac{\pi}{8} + 2i(1 +
  \sin\frac{\pi}{8})\cos\frac{\pi}{8}}{\left(1 + \sin\frac{\pi}{8}\right)^2 + \cos^2\frac{\pi}{8}} =
  \frac{2\sin\frac{\pi}{8} + 2\sin^2\frac{\pi}{8} + 2i(1 + \sin\frac{\pi}{8})\cos\frac{\pi}{8}}{2 +
    2\sin\frac{\pi}{8}}$

  $= \sin\frac{\pi}{8} + i\cos\frac{\pi}{8} = i\left(\cos\frac{\pi}{8} - i\sin\frac{\pi}{8}\right)
  \Rightarrow z^8 = i^8(\cos\pi - i\sin\pi) = -1$.
\item $z_1z_2z_3z_4z_5 = \cos\left(\frac{2\pi}{5} + \frac{4\pi}{5} + \frac{6\pi}{5} + \frac{8\pi}{5} +
  \frac{10\pi}{5}\right) + i\sin\left(\frac{2\pi}{5} + \frac{4\pi}{5} + \frac{6\pi}{5} + \frac{8\pi}{5} +
  \frac{10\pi}{5}\right)$

  $= \cos\frac{30\pi}{5} + i\sin\frac{30\pi}{5} = \cos6\pi + i\sin6\pi = 1$.
\item $z_n = \cos\left(\frac{1}{2n + 1} - \frac{1}{2n + 3}\right).\frac{\pi}{2} + i\sin\left(\frac{1}{2n +
  1} - \frac{1}{2n + 3}\right).\frac{\pi}{2}$

  $\therefore z_1z_2z_3\ldots\infty = \cos\left(\frac{1}{3} - \frac{1}{5} + \frac{1}{5} - \frac{1}{7} +
  \frac{1}{7} - \frac{1}{9}\ldots\infty\right).\frac{\pi}{2} + i\sin\left(\frac{1}{3} - \frac{1}{5} +
  \frac{1}{5} - \frac{1}{7} + \frac{1}{7} - \frac{1}{9}\ldots\infty\right).\frac{\pi}{2}$

  $= \cos\frac{\pi}{6} + i\sin\frac{\pi}{6}$.
\item Let $z_1 = x_1 + iy_1$ and $z_2 = x_2 + iy_2\Rightarrow |az_1 - bz_2|^2 + |bz_1 + az_2|^2 = (ax_1 -
  bx_2)^2 + (ay_1 - by_2)^2 + (bx_1 + ax_2)^2 + (by_1 + ay_2)^2$

  $= a^2x_1^2 + b^2x_2^2 -2abx_1x_2 + a^2y_1^2 + b^2y_2^2 - 2aby_1y_2 + b^2x_1^2 + a^2x_2^2 + 2abx_1x_2 +
  b^2y_1^2 + a^2y_2^2 + 2aby_1y_2 = (a^2 + b^2)(x_1^2 + y_1^2 + x_2^2 + y_2^2)= (a^2 + b^2)(|z_1|^2 +
  |z_2|^2)$.
\item Let $x = y + iz,$ then given expression becomes $\frac{A^2}{y + iz - a} + \frac{B^2}{y + iz - b} +
  \ldots + \frac{H^2}{y + iz - h} = y + iz + l$

  $\frac{A^2(y - a - iz)}{(y - a)^2 + z^2} + \frac{B(y - b - iz)}{(y - b)^2 + z^2} + \ldots + \frac{H^2(y -
    iz - h)}{(y - h)^2 + z^2} = y + iz + l$.  Comparing imaginary parts, we have
  $-iz\left[\frac{A^2}{(y - a)^2 + z^2} + \frac{B^2}{(y - a)^2 + z^2} + \ldots + \frac{H^2}{(y - a)^2 +
      z^2}\right] = iz \Rightarrow iz\left[1 + \frac{A^2}{(y - a)^2 + z^2} + \frac{B^2}{(y - a)^2 + z^2} +
    \ldots + \frac{H^2}{(y - a)^2 + z^2}\right] = 0$

  Clearly the term inside brackets is non-zero. So $z = 0$.
\item Let $2^{-x} = p,$ then $|1 + 4i - p|\leq 5 \Rightarrow (1 - p)^2 + 16 \leq 25$

  $1 - p \leq \pm3 \Rightarrow p \geq 4, -2 \Rightarrow x \geq -2\;\because p \nless 0\Rightarrow p \in [-2,
  \infty]$.
\item A unimodular number has a modulus of $1. \cos\theta + i\sin\theta = \frac{c + i}{c - i} = \frac{c +
  i}{c - i}.\frac{c + i}{c - i} = \frac{c^2 - 1 + 2ic}{c^2 + 1}$

  Comparing real and imaginary parts, $\cos\theta = \frac{c^2 - 1}{c^2 + 1} \Rightarrow c =
  \pm\cot\frac{\theta}{2}$

  and $\sin\theta = \frac{2c}{c^2 + 1} \Rightarrow c = \cot\frac{\theta}{2}, \tan\frac{\theta}{2}$. So the
  common value is $c = \cot\frac{\theta}{2}$.
\item $(z^3 + 3)^2 = -16 = 16i^2 \Rightarrow z^3 = -3 \pm 4i\Rightarrow |z^3| = 5 \Rightarrow |z| =
  5^{1/3}$.
\item $z = \frac{\sin\frac{x}{2} + \cos\frac{x}{2} - i\tan x}{1 + 2i\sin\frac{x}{2}}= \frac{\sin\frac{x}{2}
  + \cos\frac{x}{2} - i\tan x}{1 + 2i\sin\frac{x}{2}}.\frac{1 - 2i\sin\frac{x}{2}}{1 - 2i\sin\frac{x}{2}}$

  Since it is real so imaginary part of this will be $0.\Rightarrow -\tan x -2\sin\frac{x}{2}\cos\frac{c}{2}
  -2\sin\frac{x}{2}\cos\frac{x}{2} = 0$

  $2\sin\frac{x}{2}\left(\sin\frac{x}{2} + \cos\frac{x}{2}\right) +
  \frac{2\sin\frac{x}{2}\cos\frac{x}{2}}{\cos x} = 0\Rightarrow \sin\frac{x}{2} = 0 \Rightarrow x = 2n\pi$
  where $n = 0,1,2,3\ldots$

  or $\left(\sin\frac{x}{2} + \cos\frac{x}{2}\right)\cos x + \cos\frac{x}{2} = 0\Rightarrow
  \tan^3\frac{x}{2} - \tan\frac{x}{2} - 2 = 0$

  If $\alpha$ is a solution of above then the set of possible values are $x = 2n\pi + 2\alpha$. Solving the
  cubic equation is left to you.
\item Let $z_1 = x_1 + iy_1$ and $z_2 = x_2 + iy_2$ then $|z_1 + z_2|^2 + |z_1 - z_2|^2 = (x_1 + x_2)^2 +
  (y_1 + y_2)^2 + (x_1 - x_2)^2 + (y_1 - y_2)^2$

  $= 2(x_1^2 + y_1^2 + x_2^2 + y_2^2) = 2(|z_1|^2 + |z_2|^2)$.
\item Given, $x^2 - x + 1 = 0\Rightarrow x = -\omega, -\omega^2$

  $\displaystyle\sum_{n = 1}^5\left(x^n + \frac{1}{x^n}\right)^2 = \sum_{n = 1}^5\left(x^{2n} +
  \frac{1}{x^{2n}} + 2\right)$

  $= \left(x^2 + \frac{1}{x^2} + 2\right) + \left(x^4 + \frac{1}{x^4} + 2\right) + \left(x^6 + \frac{1}{x^6}
  + 2\right) + \left(x^8 + \frac{1}{x^8} + 2\right) + \left(x^{10} + \frac{1}{x^{10}} + 2\right)$

  $= (x^2 + x^4 + x^6 + x^8 + x^{10}) + \left(\frac{1}{x^2} + \frac{1}{x^4} + \frac{1}{x^6} + \frac{1}{x^8}
  + \frac{1}{x^{10}}\right) + 10$

  $= (\omega^2 + \omega^4 + \omega^6 + \omega^8 + \omega^{10}) + \left(\frac{1}{\omega^2} +
  \frac{1}{\omega^4} + \frac{1}{\omega^6} + \frac{1}{\omega^8} + \frac{1}{\omega^{10}}\right) + 10$

  $= -1 - 1 + 10 = 8$.
\item $3^{49}(x + iy) = \left[i\sqrt{3}\left(\frac{1 - i\sqrt{3}}{2}\right)\right]^{100} =
  i^{100}3^{50}(-\omega)^{100} \Rightarrow 3^{49}(x + iy) = 3^{50}.\omega$

  $x + iy = -\frac{3}{2} + \frac{3\sqrt{3}}{2}i\Rightarrow x = -\frac{3}{2}, y = \frac{3\sqrt{3}}{2}$.
\item $|z_1 + z_2|^2 = x_1^2 + x_2^2 + y_1^2 + y_2^2 + 2x_1x_2 + 2y_1y_2 = |z_1|^2 + |z_2|^2 + 2(x_1x_2 +
  y_1y_2)$

  Now, $2Re(z_1\overline{z_2}) = 2Re[(x_1 + iy_1)(x_2 - iy_2)] = 2\Re[x_1x_2 + y_1y_2 -i(x_1y_2 + x_2y_1)] =
  2(x_1x_2 + y_1y_2)$

  Similalry, $2\Re(\overline{z_1}z_2) = 2(x_1x_2 + y_1y_2)$.
\item R.H.S. = $\left|\frac{1}{z_1} + \frac{1}{z_2}\right| = \left|\frac{z_2 + z_1}{z_1z_2}\right|$

  Since $|z_1| = |z_2| = 1 \therefore |z_1z_2| = 1$ and thus $|z_1 + z_2| = \left|\frac{1}{z_1} +
  \frac{1}{z_2}\right|$.
\item Let $z = x + iy,$ then $x^2 - 4x + 4 + y^2 = 4x^2 - 8x + 4 + 4y^2 \Rightarrow 3x^2 + 3y^2 = 4x$

  $\Rightarrow 3|z|^2 = 4Re(z) \Rightarrow |z|^2 = \frac{4}{3}Re(z)$.
\item Given $\sqrt[3]{a + ib} = x + iy \Rightarrow a + ib = (x + iy)^3 = x^3 - 3xy^2 + i(3x^2y - y^3)$

  Comparing real and imaginary parts, we have $a = x^3 - 3xy^2, b = 3x^2y - y^3 \Rightarrow \frac{a}{x} =
  x^2 - 3y^2, \frac{b}{y} =3x^2- y^2$

  $\therefore \frac{a}{x} + \frac{b}{y} = 4(x^2 - y^2)$.
\item $x + iy = \sqrt{\frac{a + ib}{c + id}} \Rightarrow (x + iy)^2 = \frac{a + ib}{c + id} \Rightarrow |(x
  + iy)^2| = \left|\frac{a + ib}{c + id}\right| = \frac{|a + ib|}{|c + id|} \Rightarrow (x^2 + y^2)^2 =
  \frac{a^2 + b^2}{c + d^2}$.
\item Let $z = 1 = \cos0^\circ + i\sin0^\circ = e^{i2r\pi}\;\forall i \in N \Rightarrow \sqrt[n]{z} =
  e^{\frac{i.2r\pi}{n}}$. Clearly, $|z_k| = |z_{k + 1}| = 1$.
\item $z^n  = (z + 1)^n \Rightarrow \frac{z}{z + 1} = 1^{1/n}$

  This means $\frac{z}{z + 1}$ is $n$th root of unity. $\Rightarrow \left|\frac{z}{z + 1}\right| = 1$

  $\Rightarrow |z| = |z + 1| \Rightarrow x^2 + y^2 = x^2 + 2x + 1 + y^2 \Rightarrow x = -\frac{1}{2}
  \Rightarrow Re(z) < 0$.
\item Roots of $1 + x + x^2 = 0$ are $\omega$ and $\omega^2.$ Let $f(x) = x^{3m} + x^{3n - 1} + x^{3r - 2}$

  $f(x) = x^{3m} + \frac{x^{3n}}{x} + \frac{x^{3r}}{x^2} \Rightarrow f(\omega) = 1 + \frac{1}{\omega} +
  \frac{1}{\omega^2} = \frac{1 + \omega + \omega^2}{\omega^2} = 0$

  Similarly $f(\omega^2) = 0$. Thus, we see that $f(x)$ has same roots as $1 + x + x^2= 0.$ Hence, $f(x)$
  will be divisible by $1 + x + x^2$.
\item $\sqrt{3} + i = 2\left(\frac{\sqrt{3}}{2} + i\frac{1}{2}\right) = 2\left(\cos\frac{\pi}{6} +
  i\sin\frac{\pi}{6}\right) = 2e^{i\frac{\pi}{6}}$

  Similarly, $\sqrt{3} - i = 2e^{-i\frac{\pi}{6}}$

  Since imaginary part is what prevents equality we need to get rid of it and the least value for which it
  will happen is when argument is $\pi.$ Thus, we need to raise to the power by $6$ making $n = 6$.
\item $\sqrt{3} - i = 2.\left(\cos\frac{\pi}{6} - i\sin\frac{\pi}{6}\right)$

  Thus, $(\sqrt{3} - i)^n = 2^n \Rightarrow 2^n\left(\cos\frac{n\pi}{6} - i\sin\frac{\pi}{6}\right) = 2^n$

  $\Rightarrow \cos\frac{n\pi}{6} - i\sin\frac{n\pi}{6} = 1 \Rightarrow \frac{n\pi}{6} = 2k\pi\;\forall k\in
  I \Rightarrow n = 12k$

  Thus, $n$ is a multiple of $12$.
\item Given, $z^4 + z^3 + 2z^2 + z + 1 = 0 \Rightarrow z^2(z^2 + z + 1) + z^2 + z + 1 = 0$

  $\Rightarrow (z^2 + 1)(z^2 + z + 1) = 0$. If $z^2 + 1 = 0 \Rightarrow z = i \Rightarrow |z| = 1$

  If $z^2 + z + 1 = 0 \Rightarrow z = \omega, \omega^2 \Rightarrow |z| = 1$.
\item $\because z = \sqrt[7]{-1}\Rightarrow z^7 = -1\Rightarrow z^{86} + z^{175} + z^{289} = (z^7)^{14}.z^2
  + (z^7)^{25} + (z^7)^{41}z^2 = z^2 -1 -z^2 = -1$
\item Given, $z^3 + 2z^2 + 3z + 2 = 0\Rightarrow z^3 + z^2 + 2z + z^2 + z + 2 = 0\Rightarrow (z + 1)(z^2 + z
  + 2) = 0$

  If $z + 1 = 0 \Rightarrow z = -1,$ which is real and is of no interest for us.

  If $z^2 + z + 2 = 0 \Rightarrow z = \frac{-1 + i\sqrt{7}}{2}$ which are complex roots of the given
  equation.
\item $z = \sqrt[5]{1} \Rightarrow z^5 = 1$

  $2^{|1 + z + z^2 + z^{-2} - z^{-1}|} = 2^{|1 + z + z^2 + z^3 - z^4|}[\because z^4 = 1 \Rightarrow z^{-1} =
  \frac{z^5}{z} = z^4]$

  $= 2^{|1 + z + z^2 + z^3 + z^4 - 2z^4|} = 2^{\left|\frac{1 - z^5}{1 - z} - 2z^4\right|} = 2^{|2z^4|} = 2^2
  = 4[\because |z| = 1]$.
\item Let $S = 1 + 3z + 5z^2 + \ldots + (2n - 1)z^{n - 1}$

  $\Rightarrow zS = z + 3z^2 + 5z^3 + \ldots + (2n - 3)z^{n - 1} + (2n - 1)z^n$

  $\Rightarrow (1 - z)S = 1 + 2z + 2z^2 + 2z^3 + \ldots + 2z^{n - 1} + (2n - 1)z^n$

  $\Rightarrow (1 - z)S = 1 + 2n - 1 + 2[z + z^2 + \ldots z^{n - 1}][\because z^n = 1]$

  $= 2n + 2.-1[\because 1 + z + z^2 + \ldots + z^{n - 1} = 0] \Rightarrow S = \frac{2(n - 1)}{1 - z}$.
\item Let $z = \sqrt{-1-\sqrt{-1-\sqrt{-1-\infty}}} \Rightarrow z = \sqrt{-1 - z}$

  $\Rightarrow z^2 = -1 - z \Rightarrow z^2 + z + 1 = 0 \Rightarrow z = \frac{-1 \pm i\sqrt{3}}{2}
  \Rightarrow z = \omega, \omega^2$.
\item Given, $z = e^{\frac{i2\pi}{n}},$ which is nth root of unity.

  $\therefore x^n - 1 = (x - 1)(x - z)(x - z^2 (x - z^3) ... (x - z^{n - 1})$

  Putting $x = 11, (11 - z)(11 - z^2)\ldots(11 - z^{n - 1}) = \frac{11^n - 1}{10}$.
\item Given, $\frac{3}{2 + \cos\theta + i\sin\theta} = a + ib \Rightarrow a + ib  \frac{3(2 + \cos\theta -
  i\sin\theta)}{5 + 4\cos\theta}$

  Comparing real and imaginary parts, we get $a = \frac{6 + 3\cos\theta}{5 + 4\cos\theta}, b =
  \frac{-3\sin\theta}{5 + 4\cos\theta} \Rightarrow a^2 + b^2 = \frac{36 + 36\cos\theta + 9\cos^2\theta +
    9\sin^2\theta}{(5 + 4\cos\theta)^2}$

  $= \frac{45 + 36\cos\theta}{(5 + \cos\theta)^2} = \frac{9(5 + 4\cos\theta)}{(5 + 4\cos\theta)^2} =
  \frac{9}{5 + 4\cos\theta}, 4a - 3 = \frac{24 + 12\cos\theta - 15 - 12\cos\theta}{5 +
    4\cos\theta} = \frac{9}{5 + 4\cos\theta}\Rightarrow a^2 + b^2 = 4a - 3$.
\item Let $z = x + iy, \Rightarrow |(2x - 1) + 2iy| = |(x - 2) + iy|\Rightarrow 4x^2 - 4x + 1 + 4y^2 = x^2 -
  4x + 4 + y^2 \Rightarrow 3x^2 + 3y^2 = 3\Rightarrow x^2 + y^2 = 1\Rightarrow |z| = 1$.
\item Given, $\frac{1 -ix}{1 + ix} = m + in \Rightarrow m + in = \frac{1 - ix}{1 + ix}.\frac{1 - ix}{1 - ix}$

  $m + in = \frac{1 - x^2 - 2ix}{1 + x^2}$, Comparing real and imaginary parts, $m = \frac{1 - x^2}{1 +
  x^2}, n = \frac{-2x}{1 + x^2}$

  $\Rightarrow m^2 + n^2 = \frac{(1 - x^2)^2 + 4x^2}{(1 + x^2)^2} = 1$.
\item We know that the equation of a straight line is given by $\startbmatrix\NC z \NC \overline{z} \NC
  1\NR\NC z_1 \NC \overline{z_1} \NC 1 \NR\NC z_2 \NC \overline{z_2} \NC 1\NR\stopbmatrix = 0$

  $\Rightarrow z(\overline{z_1} - \overline{z_2}) - \overline{z}(z_1 - z_2) + z_1\overline{z_2} -
  \overline{z_1}z_2 = 0$

  $\Rightarrow z(1 + i - 1 - i) - \overline{z}(1 + i -1 + i) + (1 + i)^2 - (1 - i)^2 = 0\Rightarrow z +
  \overline{z} - 2 = 0$.
\item Given, $5z_1 - 13z_2 + 8z_3 = 0 \Rightarrow z_2 = \frac{5z_1 + 8z_3}{5 + 8}$

  This means $z_1$ divides the line segment joining $z_1$ and $z_2$ in the ratio of $5:8$ which also implies
  that these three points are collinear. Thus, $\startbmatrix\NC z_1 \NC \overline{z_1} \NC 1\NR\NC z_2 \NC
  \overline{z_2} \NC 1\NR\NC z_3 \NC \overline{z_3} \NC 1\NR\stopbmatrix = 0$
\item We know that length of perpendicular from $z_1$ to $\overline{a}z + a\overline{z} + b = 0$ is given by
  $\frac{|\overline{a}z_1 + a\overline{z_1} + b|}{2|a|}.$

  Thus desired length $= \frac{|(2 - 3i)(3 + 4i) + (2 + 3i)(3 - 4i) + 9|}{2|3 - 4i|} = \frac{45}{10} =
  \frac{9}{2}$.
\item
  \startplacefigure[location={left, none}]
    \startMPcode
      draw (-1cm, 0cm) -- (1cm, 0cm);
      draw (0cm, -1cm) -- (0cm, 1cm);
      label.bot("$z_1$", (0cm, -1cm));
      label.top("$z_2$", (0cm, 1cm));
      label.urt("$\frac{z_1 + z_2}{2}$", (0cm, 0cm));
      label.lrt("$b\overline{z} + \overline{b}z = c$", (0cm, 0cm));
      pickup pencircle scaled 2pt;
      drawdot (0cm, 1cm);
      drawdot (0cm, -1cm);
    \stopMPcode
  \stopplacefigure
  Since mid-point lies on the given line, therefore $b\left(\frac{\overline{z_1} + \overline{z_2}}{2}\right) +
  \overline{b}\left(\frac{z_1 + z_2}{2}\right) = c$

  Since line segment joining $z_1$ and $z_2$ is perpedicular to the given line therefore, Slope of $z_1z_2$ + Slope of line = 0

  $\Rightarrow \frac{z_2 - z_1}{\overline{z_2} - \overline{z_1}} - \frac{b}{\overline{b}} = 0$

  Solving these two equations, we get $\overline{b}z_2 + b\overline{z_1} = c$.
\item   Let $z = 2 - i$ then after rotation new point would be $z.e^{i\pi/2} = (2 - i)\left(\cos\frac{\pi}{2} +
  i\sin\frac{\pi}{2}\right) = (2 - i)i = 1 + 2i$.
\item Coordinate of $z_0$ after moving $5$ points horizontally and $3$ points vertically away from starting pont would be
  $6 + 5i$.

  It then moves in the direction of vecor $\hat{i} + \hat{j}$ for $\sqrt{2}$ units. This vector makes angle $\pi/4$ with
  $x$-axis. So new coordinate would be $6 + \sqrt{2}\cos\pi/4 + 5 + \sqrt{2}\sin\pi/4 = 7 + 6i$.

  It then rotates by angle $\pi/2$ so new coordinate would be $(7 + 6i)e^{i\pi/2} = (7 + 6i)i = -6 + 7i$.
\item North-East direction makes angle of $\pi/4$ with $x$-axis. So coordinates of point $3$ units from origin in
  North-East direction $= 3.e^{i\pi/4} = 3\left(\cos\frac{\pi}{4} + i\sin\frac{\pi}{4}\right) = \frac{3}{\sqrt{2}} +
  i\frac{3}{\sqrt{2}}$.

  North-West direction makes angle of $3\pi/4$ with $x$-axis. A disaplacement of $4$ units in this direction will mean a shift in
  coordinates by $4.e^{i3\pi/4} = 4\left(\cos\frac{3\pi}{4} + i\sin\frac{3\pi}{4}\right) = -\frac{4}{\sqrt{2}} +
  i\sin\frac{4}{\sqrt{2}}$.

  Thus, final coordiate would be sum of the above two i.e. $-\frac{1}{\sqrt{2}} + i\frac{7}{\sqrt{2}}$.
\item Given, $\frac{z_1 - z_3}{z_2 - z_3} = \frac{1 - i\sqrt{3}}{2} = \frac{1 - i\sqrt{3}}{2}.\frac{1 +
    i\sqrt{3}}{2}$

  $= \frac{1 + 3}{2(1 + i\sqrt{3})}= \frac{2}{1 + i\sqrt{3}}$

  $\Rightarrow \frac{z_2 - z_3}{z_1 - z_3} = \frac{1 + i\sqrt{3}}{2} = \cos\frac{\pi}{3} + i\sin\frac{\pi}{3}$

  $\Rightarrow \left|\frac{z_2 - z_3}{z_1 - z_3}\right| = 1$ and $\arg\left(\frac{z_2 - z_3}{z_1 -
  z_3}\right) = \frac{\pi}{3}$

  Hence, the triangle is equilateral.
\item Since sides of an equilateral triangle make an angle of $60^\circ$ with each other, therefore
  $\frac{z_3 - z_1}{z_2 - z_1} = \cos60^\circ \pm \sin60^\circ = \frac{1 \pm i\sqrt{3}}{2}$

  $\Rightarrow 2z_3 - 2z_1 + z1 - z_2 = \pm i(z_2 - z_1)\sqrt{3}\Rightarrow (2z_3 - z_1 - z_2)^2 = 3(z_2 -
  z_1)^2\Rightarrow z_1^2 + z_2^2 + z_3^2 = z_1z_2 + z_2z_3 +z_3z_1$

  $\Rightarrow z_1z_2 + z_2z_3 + z_3z_1 - z_z^2 - z_2^2 - z_3^2 + z_1z_2 - z_1z_2 + z_2z_3 - z_2z_3 + z_1z_3
  - z_1z_3 = 0$

  $\Rightarrow (z_1 - z_2)(z_2 - z_3) + (z_2 - z_3)(z_3 - z_1) + (z_3 - z_1)(z_1 - z_2) =
  0\Rightarrow \frac{1}{z_1 - z_2} + \frac{1}{z_2 - z_3} + \frac{1}{z_3 - z_1} = 0$.
\item Since it is an equilateral triangle, therefore centroid and circumcenters would be
  identical. $\therefore z_0 = \frac{z_1+ z_2 + z_3}{3}$

  Since it is an equilateral triangle, we have just proven that $z_1^2 + z_2^2 + z_3^2 = z_1z_2 + z_2z_3
  +z_3z_1$

  From first equation, we have $\Rightarrow 9z_0^2 = z_1^2 + z_2^2 + z_3^2 + 2(z_1z_2 + z_2z_3 +z_3z_1)$

  $\Rightarrow 9z_0^2 = z_1^2 + z_2^2 + z_3^2 + 2(z_1^2 + z_2^2 + z_3^2)\Rightarrow 3z_0^2 = z_1^2 + z_2^2 +
  z_3^2$.
\item Since right angle is at $z_3,$ therefore $\frac{z_2 - z_3}{z_1 - z_3} = e^{i\pi/2} = i\Rightarrow (z_2
  - z_3)^2 = -(z_1 - z_3)^2 \Rightarrow z_2^2 + z_3^2 - 2z_2z_3 = -z_1^2 - z_3^2 + 2z_1z_3$

  $\Rightarrow z_1^2 + z_2^2 - 2z_1z_2 = -2z_3^2 + 2z_2z_3 + 2z_1z_3 - 2z_1z_2\Rightarrow (z_1 - z_2)^2 =
  2(z_1 - z_3)(z_3 - z_2)$.
\item Clearly, $|z - z_0|^2 = r^2 \Rightarrow (z - z_0)(\overline{z - z_0}) = r^2\Rightarrow (z -
  z_0)(\overline{z} - \overline{z_0}) = r^2$

  $\Rightarrow z\overline{z} - \overline{z}z_0 - z\overline{z_0} + z_0\overline{z_0} = r^2$.
\item Given, $z = 1 - t + i\sqrt{t^2 + t + 2};$ comparing real and imaginary parts, we get $x = 1 - t, y =
  \sqrt{t^2 + t + 1} \Rightarrow y^2 = t^2 + t + 2$

  $\Rightarrow y^2 = (1 - x)^2 + (1 - x) + 2 = \left(x - \frac{3}{2}\right)^2 + \frac{7}{4}$, which is
  equation of a hyperparabola.
\item Given, $\overline{z} = \overline{a} + \frac{r^2}{z - a}\Rightarrow (\overline{z} - \overline{a})(z -
  a) = r^2$, which is equation of a circle with center at $a$ and radius $r.$
\item Since $z_1$ and $z_2$ are ends of diameter $\Rightarrow |z - z_1|^2 + |z - z_2|^2 = |z_1 -
  z_2|^2\Rightarrow k = |z_1 - z_2|^2 = |2 + 3i - 4 - 3i|^2 = 4$.
\item $z = x + iy,$ then $|(x + 1) + iy| = \sqrt{2}|(x - 1) + iy|$

  Squaring both sides, we get $(x + 1)^2 + y^2 = 2[(x - 1)^2 + y^2] \Rightarrow x^2 + y^2 - 6x + 1 = 0$,
  which is equation of a circle.
\item Given, $\left|\frac{z - 1}{z - i}\right| = 1 \Rightarrow |z - 1| = |z - i|$

  Let $z = x + iy,$ then we have $|(x - 1) + iy| = |x + i(y - 1)|$

  Squaring both sides, we get $\Rightarrow (x - 1)^2 + y^2 = x^2 + (y - 1)^2 \Rightarrow 2x = 2y \Rightarrow
  x = y$, which is equation of a straight line.
\item \startplacefigure[location={left, none}]
    \startMPcode
      draw fullcircle scaled 2cm;
      draw((.707cm, .707cm) -- (.707cm, -.707cm));
      draw((-.707cm, .707cm) -- (-.707cm, -.707cm));
      draw((.707cm, .707cm) -- (-.707cm, .707cm));
      draw((-.707cm, -.707cm) -- (.707cm, -.707cm));
      label.llft("$A(z_1)$", (-.707cm, -.707cm));
      label.lrt("$B(z_2)$", (.707cm, -.707cm));
      label.urt("$C(z_3)$", (.707cm, .707cm));
      label.ulft("$D(z)$", (-.707cm, .707cm));
    \stopMPcode
  \stopplacefigure
  $\angle z_1 = \arg\left(\frac{z_1 - z_2}{z_1 - z_4}\right),$
  $\angle z_2 = \arg\left(\frac{z_3 - z_2}{z_1 - z_2}\right),$
  $\angle z_3 = \arg\left(\frac{z_3 - z_4}{z_3 - z_2}\right),$ and
  $\angle z_4 = \arg\left(\frac{z_1 - z_4}{z_3 - z_4}\right)$

  $\angle z_1 + \angle z_3 = \pi \Rightarrow \arg {\frac {z_1 - z_2} {z_1 - z_4} } + \arg \left( {\frac {z_3
      - z_4} {z_3- z_2} }\right) = \pi$

  $\Rightarrow \arg\left(\frac{(z_1 - z_2)(z_3 - z_4)}{(z_1 - z_4)(z_3 - z_2)}\right) = \pi$
  $\Rightarrow \frac{(z_1 - z_2)(z_3 - z_4)}{(z_1 - z_4)(z_3 - z_2)}$ is real number.
\item Given, $\frac{2}{z_1} = \frac{1}{z_2} + \frac{1}{z_3}\Rightarrow \frac{z_2 - z_1}{z_3 - z_1} =
  -\frac{z_2}{z_3}\Rightarrow \arg\left(\frac{z_2 - z_1}{z_3 - z_1}\right) = \pi - \arg\frac{z_3}{z_2}$

  $\Rightarrow \arg\left(\frac{z_2 - z_1}{z_3 - z_1}\right) + arg\left(\frac{z_3 - 0}{z_2 - 0}\right) = \pi$
  Thus, the given points and the origin are concyclic.
\item From the equation of circle, $r^2 = |\omega - \omega^2|^2\Rightarrow r^2 = |i\sqrt{3}|^2 = 3
  \Rightarrow r = \sqrt{3}$.
\item Let $z = x + iy\Rightarrow (x - 4)^2 + y^2 < (x - 2)^2 + y^2 \Rightarrow x^2 - 8x + 16 < x^2 - 4x +
  4\Rightarrow 4x > 12 \Rightarrow x > 3$.
\item Given, $2z_1 - 3z_2 + z_3 = 0\Rightarrow z_2 = \frac{2z_1 + z_3}{3} = \frac{2z_1 + z_3}{2 + 1}$

  Thus, $z_1$ divides the line segement $z_1z_3$ in the ratio of $2:1$ i.e. all three points are collinear.
\item Given, $|z + 1| = |z - 1| \Rightarrow (x + 1)^2 + y^2 = (x - 1)^2 + y^2 \Rightarrow x = 0$

  Also, given that $\arg\frac{z - 1}{z + 1} = \frac{\pi}{4}\Rightarrow z - 1 = (z + 1)e^{i\pi/4} \Rightarrow
  -1 + iy = (1 + iy)\left(\cos\frac{\pi}{4} + i\sin\frac{\pi}{4}\right)$

  $\Rightarrow -1 + iy = (1 + iy)\left(\frac{1}{\sqrt{2}} + i\frac{1}{\sqrt{2}}\right)\Rightarrow y =
  \sqrt{2} + 1$.
\item Given, $|z|^8 = |z - 1|^8 \Rightarrow |z| = |z - 1|, \Rightarrow x^2 + y^2 = (x - 1)^2 + y^2
  \Rightarrow x = \frac{1}{2}, y\in(\infty, \infty)$,
  which is equation of straight line parallel to $y$-axis at $x = 1/2.$
\item Given, $z\overline{z} + a\overline{z} + \overline{a}z + b = 0$
  $\Rightarrow z\overline{z} + a\overline{z} + \overline{a}z + a\overline{a} = a\overline{a} - b$

  $(z + a)(\overline{z} + \overline{a}) = |a|^2 - b$,
  which is equation of a circle if $|a|^2 - b > 0 \Rightarrow |a|^2 > b$.
\item Let $z = x + iy,$ comparing real and imaginary part gives us
  $x = \lambda + 3, y = \sqrt{3 - \lambda^2} \Rightarrow y^2 = 3 - \lambda^2$

  $\Rightarrow (x - 3)^2 + y^2 = 3$,
  which is equation of a circle with center $(3, 0)$ and radius $\sqrt{3}$.
\item Let $z = x + iy,$ then $|Re(z)| + |Im(z)| = k$ will give us four equations. $x + y = k, x - y = k, -x + y = k$ and $-x - y = k$

  These lines will intersect at $(k, 0), (0, k), (-k, 0), (0 -k)$ giving us a square as locus of $z.$
\item $z_2 = z_1^2 + i = i, z_3 = z_2^2 + i = i - 1, z_4 = z_3^2 + i = (i - 1)^2 + i = -i, z_5 = z_4^2 + i =
  i - 1, z_6 = z_5^2 + i = -i$

  Thus, we see that it is a cycle between $-i$ and $i - 1$ starting at $z_3.$
  $\Rightarrow z_{111} = z_3 = i - 1 \Rightarrow |z_{111}| = \sqrt{2}$
\item Given, $z\overline{z}^3 + z^3\overline{z} = 350 \Rightarrow z\overline{z}(\overline{z}^2 + z^2) = 350$

  Let $z = x + iy,$ then given equation becomes $2(x^2 + y^2)(x^2 - y^2) = 350 \Rightarrow (x^2 + y^2)(x^2 -
  y^2) = 175$

  Prime factors of $175$ are $5, 5, 7$ so the only solution which yields integers for $x$ and $y$ are $x^2 +
  y^2 = 25, x^2-y^2 = 7$

  $\Rightarrow x = \pm 4, y = \pm 3$ which gives a rectangle with four points and digonal with a length of
  $10$ units.
\item We know that $z_1 + z_2$ and $z_1 - z_2$ are the diagonals of a quadrilateral. Now diagonals of a
  parallelogram does not intersect at angle $\pi/2$ and diagonals of a square and rectangle are equal. Only
  rhombus satisfies the given criteria of diagonals meeting at right angle and having different lengths.
Thus, the given conditions represent a rhombus but not a square.
\item Let $\arg(z_1) = \theta, \arg(z_2) = \theta + \alpha \Rightarrow \frac{az_1}{bz_2} =
  \frac{a|z_1|e^{i\theta}}{b|z_2|e^{i(\theta + \alpha)}} = e^{-i\alpha}$

  $\Rightarrow \frac{bz_2}{az_1} = e^{i\alpha}\Rightarrow \frac{az_1}{bz_2} + \frac{bz_2}{az_1} =
  e^{i\alpha} + e^{-i\alpha} = 2\cos\alpha$

  Thus, it will lie on the line segment $[-2, 2]$ of the real axis.
\item Since $z_1, z_2, z_3$ are roots of the equation $z^3 + 3\alpha z^2 + 3\beta z + \gamma = 0\Rightarrow
  z_1 + z_2 + z_3 = -3\alpha, z_1z_2 + z_2z_3 + z_3z_1 = 3\beta, z_1z_2z_3 = \gamma$

  We know that for a triangle to be equilateral $z_1^2 + z_2^2 + z_3^2 = z_1z_2 + z_2z_3 + z_3z_1$

  $\Rightarrow (z_1 + z_2 + z_3)^2 = 3(z_1z_2 + z_2z_3 + z_3z_1)\Rightarrow 9\alpha^2 = 3.3\beta \Rightarrow
  \alpha^2 = \beta$.
\item Given, $z_1^2 + z_2^2 + 2z_1z_2\cos\theta = 0$ Dividing both sides with $z_2^2,$ we get
  $\left(\frac{z_1}{z_2}\right)^2 + 1 + 2\frac{z_1}{z_2}\cos\theta = 0$

  The above equation is a quadratic equation in $\frac{z_1}{z_2}, \therefore \frac{z_1}{z_2} =
  \frac{-2\cos\theta \pm\sqrt{4\cos^2\theta - 1}}{2}$

  $\Rightarrow \frac{z_1}{z_2} = -\cos\theta \pm i\sin\theta \Rightarrow \left|\frac{z_1}{z_2}\right| =
  1\Rightarrow |z_1| = |z_2| \Rightarrow |z_1 - 0| = |z_2 - 0|$

  Thus, $z_1, z_1$ and the origin form an isosceles triangle.
\item Since origin is circumcenter $\Rightarrow |z_1| = |z_2| = |z_3| = |z|$
  $\Rightarrow z_1\overline{z_1} = z_2\overline{z_2} = z_3\overline{z_3} = z\overline{z}$

  $\because\;AP\perp BC \therefore\;\frac{z - z_1}{\overline{z} - \overline{z_1}} + \frac{z_2 - z_3}{\overline{z_2} -
    \overline{z_3}} = 0$
  $\Rightarrow \frac{z - z_1}{\frac{z\overline{z_1}}{z} - \overline{z_1}} + \frac{z_2 - z_3}{\frac{z_3\overline{z_3}}{z} -
    \overline{z_3}} = 0$

  $\Rightarrow \frac{z(z - z_1)}{z_1\overline{z_1} - z\overline{z_1}} + \frac{z_2(z_2 - z_3)}{z_3\overline{z_3} -
    z_2\overline{z_3}} = 0$
  $\Rightarrow \frac{-z(z_1 - z)}{\overline{z_1}(z_1 - z)} - \frac{z_2(z_3 - z_2)}{\overline{z_3}(z_3 - z_2)} = 0$
  $\Rightarrow \frac{-z}{z_1} - \frac{z_2}{z_3} = 0 \Rightarrow z = -\frac{z_1z_2}{z_3}$.
\item Given $OA = OB, \Rightarrow |z_1| = |z_2| = l$ (let).
  Also given, $\arg(z_1) = \alpha + \arg(z_2) \Rightarrow z_1 = le^{i(\alpha + \arg(z_2))} = le^{i\arg(z_2)}.e^{i\alpha} =
  z_2e^{i\alpha}$

  Now, $z_1z_2 = q \Rightarrow z_2^2e^{i\alpha} = q$ and $z_1 + z_2 = -p \Rightarrow z_2(1 + e^{i\alpha}) = -p$
  $\Rightarrow 2z_2\cos\frac{\alpha}{2}.e^{i\alpha/2} = -p \Rightarrow p^2 = 4z_2^2\cos^2\frac{\alpha}{2}.e^{i\alpha}$
  $\Rightarrow p^2 = 4q\cos^2\frac{\alpha}{2}$.
\item Let $z + iy,$ then $\Re\left(\frac{z + 4}{2x - i}\right) = \Re\left(\frac{x + 4 + iy}{2x + i(2y - 1)}\right)$
  $\Rightarrow \Re\left(\frac{[(x + 4) + iy][(2x - i(2y - 1))]}{4x^2 + (2y - 1)^2}\right) = \frac{1}{2}$

  $\Rightarrow \frac{2x(x + 4) + y(2y - 1)}{4x^2 + (2y - 1)^2} = \frac{1}{2} \Rightarrow 16x + 2y - 1= 0$,
  which is equation of a straight line.
\item Since the circle is inscribed in $|z| = 2$ so center is origin. Also, since $z_1, z_2$ and $z_3$ are in
  clockwise direction $z_2 = z_1e^{-i120^\circ}, z_3 = z_2e^{-i120^\circ}$

  $\Rightarrow z_2 = (1 + \sqrt{3}i)[(\cos. -120^\circ + i.\sin -120^\circ)] = 1-\sqrt{3}i$
  $\Rightarrow z_3 = -2$.
\item Given $z_1 = \frac{a}{1 - i} \Rightarrow z_1 = \frac{a + ia}{2}, z_2 = \frac{b}{2 + i} = \frac{2b - ib}{5}$
  Also given, $z_1 - z_2 = 1 \Rightarrow 5a + i5a - 4b + i2b = 10$

  Comparing real and imaginary parts, we get $5a - 4b = 10, 5a + 2b = 0 \Rightarrow a = \frac{2}{3}, b = -\frac{5}{3}$
  Cnetroid is $\frac{z_1 + z_2 + z_3}{3} = \frac{1}{3}(1 + 7i)$.
\item From the quadratic equation we have $z_1 + z_2 = -1$ and $z_1z_2 = \frac{\lambda}{2}$.
  Since $0, z_1, z_2$ form an equilateral triangle, $\Rightarrow z_1z_2 + z_2.0 + z_1.0 = z_1^2 + z_2^2 + 0^2$

  $\Rightarrow (z_1 + z_2)^2 = 3z_1z_2 \Rightarrow (-1)^2 = 3.\frac{\lambda}{2}\Rightarrow \lambda =
  \frac{2}{3}$.
\item Let $A, B, C$ represent $a,b,c$ and $U, V, W$ represent $u, v, w.$
  $\Rightarrow AB = b - c, BC = c - b = (a - b)(1 - r), CA = a - c = r(a - b)$

  $\Rightarrow UV = v - u, VW = w - v = (u - v)(1 - r), WU = u - w = r(u - v)$
  $\Rightarrow \frac{AB}{UV} = \frac{BC}{VW} = \frac{CA}{WU}$
  Thus, the triangles are similar.
\item Let $z_1$ and $z_2$ be points on real axis which circle cuts with. Since these are on real axis and if $z$
  represents this points then $z = \overline{z}[\because z = x + i.0]$

  Substituting $z = \overline{z}$ in the equation of the circle, we get $z^2 + (\overline{\alpha} + \alpha)z + r = 0$
  Since $z_1, z_2$ are the roots $\therefore z_1 + z_2 = -\alpha, z_1z_2 = r$

  Length of intercept $=|z_1 - z_2| = \sqrt{(z_1 - z_2)^2} = \sqrt{(z_1 + z_2)^2 - 4z_1z_2} = \sqrt{(\overline{\alpha} + \alpha)^2
    - 4r}$.
\item Clearly, $a = e^{i\alpha}, b = e^{i\beta}, c= e^{i\gamma}$.
  Also given, $\frac{a}{b} + \frac{b}{c} + \frac{c}{a} = 1\Rightarrow e^{i(\alpha - \beta)} + e^{i(\beta - \gamma)} + e^{i(\gamma -
    \alpha)} = 1$.

  Comparing real parts, we get $\cos(\alpha - \beta) + \cos(\beta - \gamma) + \cos(\gamma - \alpha) = 1$.
\item Let $A(z_1), B(z_2)$ be the centers of given circles and $P$ be the center of the variable circle which
  touches given circles externally, then

  $|AP| = a + r$ and $|BP| = b + r$ where $r$ is the radius of the variable circle. Clearly,
  $|AP| - |BP| = a - b \Rightarrow ||AP| - |BP|| = |a - b| = $a constant.

  Hence, locus of $P$ is a right bisector if $a = b,$ a hyperbola if $|a - b| < |AB|$ an empty set of $|a - b|>|AB|,$ set of all
  points on line $AB$ except those which lie between $A$ and $B$ if $|a - b| = |AB|\neq 0.$
\item Let $a + ib = re^{i\theta}, r^2 = a^2 + b^2 \Rightarrow a - ib = e^{-i\theta}, \tan\theta = \frac{b}{a}$
  $\frac{a - ib}{a + ib} = e^{-2i\theta} \Rightarrow i\log\left(\frac{a - ib}{a + ib}\right) = i\log e^{-2i\theta} = 2\theta$

  $\Rightarrow \tan\left[i\log\left(\frac{a - ib}{a + ib}\right)\right] = \tan2\theta = \frac{2\tan\theta}{1 - \tan^2\theta}$
  $= \frac{2b/a}{1 - b^2/a^2} = \frac{2ab}{a^2 - b^2}$.
\item Given, $|z_1| = |z_2| = 1\Rightarrow a^2 + b^2 = c^2 + d^2 = 1$
  $\Re(z_1\overline{z_2}) = 0 \Rightarrow \Re[(a + ib)(c - id)] = 0 \Rightarrow ac + bd = 0$

  $a^2 + b^2 = c^2 + d^2 \Rightarrow (a + ic)^2 = (d - ib)^2[\because ac = =bd] \Rightarrow a + ic = d - ib
  or -d + ib$ $\Rightarrow a = d$ and $c = -b$ or $a = -d, c = b$

  $\Rightarrow a^2 + c^2 = b^2 + d^2 = 1 \Rightarrow |w_1| = |w_2| = 1$
  $\Rightarrow \Re(w_1\overline{w2}) = \Re[(a + ic)(b - id)] = ab + cd = 0$.
\item Let $z_1 = r(\cos\theta + i\sin\theta)$. Given, $\left|\frac{z_1}{z_2}\right| = 1$
  $\Rightarrow |z_1| = |z_2| = r$. Also given, $\arg(z_1z_2) = 0 \Rightarrow \arg(z_1) + \arg(z_2) = 0$

  $\Rightarrow \arg(z_2) = -\theta \Rightarrow z_2 = r[\cos(-\theta) + i\sin(-\theta)] = r[\cos\theta -
  i\sin\theta] = \overline{z_1}$
  $\Rightarrow \overline{z_2} = z_1 \Rightarrow |z_2|^2 = z_1z_2$.
\item $t_n = (n + 1)\left(n + \frac{1}{\omega}\right)\left(n + \frac{1}{\omega^2}\right)$
  $= n^3 + n^2\left(1 + \frac{1}{\omega} + \frac{1}{\omega^2}\right) + n\left(1 + \frac{1}{\omega} +
  \frac{1}{\omega^2}\right) + 1$

  $= n^3 + n^2(1 + \omega + \omega^2) + n(1 + \omega + \omega^2) + 1 = n^3 + 1$
  $\displaystyle\therefore S_n = \sum_{i = 1}^nt_i = \sum_{i = 1}(i^3 + 1) = \frac{n^2(n + 1)^2}{4} + 1$.
\item Given $|z_1 + iz_2| = |z_1 - iz_2|$
  $\Rightarrow (z_1 + iz_2)(\overline{z_1} - i\overline{z_2}) = (z_1 - iz_2)(\overline{z_1 +
  i\overline{z_2}})$

  $\Rightarrow \overline{z_1}z_2 = z_1\overline{z_2} \Rightarrow \frac{z_1}{z_2} =
  \frac{\overline{z_1}}{\overline{z_2}}$.
  Thus, $\frac{z_1}{z_2}$ is purely real.
\item $z = -2 + 2\sqrt{3}i = 4\omega$
  $\Rightarrow z^{2n} + 2^{2n}z^n + 2^{4n} = 4^{2n}[\omega^{2n} + \omega^n + 1]$

  The above expression has value of $0$ if $n$ is not a multiple of $3$ and $3.4^{2n}$ if $n$ is multiple of
  $3$.
\item $x + \frac{1}{x} = 2\cos\theta, \Rightarrow x^2 - 2\cos\theta x + 1 = 0$
  $\Rightarrow x = \frac{2\cos\theta \pm \sqrt{4\cos^2\theta - 1}}{2} = \cos\theta \pm i\sin\theta = e^{\pm
  i\theta}$

  Similarly, $y = e^{\pm i\phi}\Rightarrow \frac{x}{y} + \frac{y}{z} = 2\cos(\theta - \phi)$
  and $xy + \frac{1}{xy} = 2\cos(\theta + \phi)$.
\item Given, $|z_1| = |z_2|, \Re(z_1) > 0$ and $\Im(z_1) < 0$
  $\Re\left(\frac{z_1 + z_2}{z_1 - z_2}\right) = \frac{1}{2}\left(\frac{z_1 + z_2}{z_1 - z_2} +
  \frac{\overline{z_1} + \overline{z_2}}{\overline{z_1} - \overline{z_2}}\right)$

  $= \frac{1}{2}\left(\frac{2(|z_1|^2 - |z_2|^2)}{|z_1 - z_2|^2}\right) = 0$
  Thus, $\frac{z_1 + z_2}{z_1 - z_2}$ is purely imaginary.
\item Given, $\frac{AB}{BC} = \sqrt{2} \Rightarrow \frac{z_1 - z_2}{z_3 - z_2} = \frac{|z_1 - z_2|}{|z_3 -
  z_2|}.e^{i\pi/4}$

  $= \frac{AB}{BC}.e^{i\pi/4} = \sqrt{2}\left(\frac{1}{\sqrt{2}} + \frac{i}{\sqrt{2}}\right) = 1 + i$
  $\Rightarrow z_1 - z_2 = (1 + i)(z_3 - z_2) \Rightarrow z_2 = z_3 + i(z_1 - z_3)$.
\item Given, $z_1(z_1^2 - 3z_2^2) = 2$ and $z_2(3z_1^2 - z_2^2) = 11$
  $\Rightarrow z_1^3 - 3z_1z_2^2 + iz_2(3z_1^2 - z_2^2) = 2 + 11i \Rightarrow (z_1 + iz_2)^3 = 2 + 11i$, and

  $\Rightarrow z_1^3 - 3z_1z_2^2 - iz_2(3z_1^2 - z_2^2) = 2 - 11i \Rightarrow (z_1 - iz_2)^3 = 2 - 11i$

  Multiplying above equations, we get
  $(z_1^2 + z_2^2)^3 = 4 + 121 = 125 \Rightarrow z_1^2 + z_2^2 = 5$.
\item Given $\sqrt{1 - c^2} = nc - 1 \Rightarrow 1 - c^2 = n^2c^2 - 2nc + 1 \Rightarrow \frac{c}{2n} =
  \frac{1}{1 + n^2}$

  $\frac{c}{2n}(1 + nz)\left(1 + \frac{n}{z}\right) = \frac{1}{1 + n^2}\left[1 + n^2 + n\left(z +
    \frac{1}{z}\right)\right]$

  $= \frac{1}{1 + n^2}\left[1 + n^2 + 2\cos\theta + n\right] = 1 + \frac{2n}{1 + n^2}\cos\theta = 1 +
  c\cos\theta$.
\item If $P(z)$ is any point of the ellipse, then equation of ellipse is given by
  $|z - z_1| + |z- z_2| = \frac{|z_1 - z_2|}{e}$

  If we put $z_1$ or $z_2$ in the above equation then L.H.S. becomes $|z_1 - z_2|$.
  Thus, for any interior point of the ellipse, we have $|z - z_1| + |z - z_2| < \frac{|z_1 - z_2|}{e}$

  If $P(z)$ lies on the ellipse, we have $|z - z_1| + |z- z_2| = \frac{|z_1 - z_2|}{e}$.
  It is given that origin is an internal point, so
  $|0 - z_1| + |0 - z_2| < \frac{|z_1 - z_2|}{e}$
  $\Rightarrow e\in\left(0, \frac{|z_1 - z_2|}{|z_1| + |z_2|}\right)$.
\item Let $z = x + iy$, then we have
  $|(x - 2) + i(y - 1)| = |z|\left|\frac{1}{\sqrt{2}}\cos\theta - \frac{1}{\sqrt{2}}\sin\theta\right|$
  where, $\theta = \arg(z)$

  $\Rightarrow \sqrt{(x - 2)^2 + (y - 1)^2} = \frac{1}{\sqrt{2}}|x - y|$,
  which is equation of a parabola.
\item Since $|z - z_1| = |z - z_2|$, therefore $z$ will be one of the vertices of the isosceles triangle
  where base will be formed by $z_1$ and $z_2$.

  Also, since $\left|z - \frac{z_1 + z_2}{2}\right|\leq r$ so $z$ will lie on the circle whose center is
  $\frac{z_1 + z_2}{2}$ and radius is $r$. Thus, the distance between segment $z_1z_2$ will be $r$.
  Thus, the maximum area of the triangle will be $\frac{1}{2}|z_1 - z2|.r$.
\item Given $|z_1| = 1 \Rightarrow a_1^2 + b_1^2 = 1, |z_2| = 2 \Rightarrow a_2^2 + b_2^2 = 4$.
  Also given $\Re(z_1z_2) = 0 \Rightarrow a_1a_2 - b_1b_2 = 0 \Rightarrow a_1a_2 = b_1b_2$

  $\Rightarrow a_2^2 + b_2^2 = 4a_1^2 + 4b_1^2 \Rightarrow a_2^2 - 4a_1^2 = 4b_1^2 - b_2^2 \Rightarrow a_2^2
  - 4a_1^2 + 4ia_1a_2 = 4b_1^2 - b_2^2 + 4ib_1b_2$
  $\Rightarrow (a_2 + 2ia_1)^2 = (2b_1 + ib_2)^2 \Rightarrow a_2 = \pm 2b_1$

  $\omega_1 = a_1 + \frac{ia_2}{2} = a_1 \pm b_1 \Rightarrow |\omega_1| = \sqrt{a_1^2 + b_1^2} = 1$
  $\omega_2 = 2b_1 + ib_2 = \pm a_2 + ib_2 \Rightarrow |\omega_2| = \sqrt{a_2^2 + b_2^2} = 2$
  $\Re(\omega_1\omega_2) = 2a_1b_1 - 2a_2b_2 = 0$.
\item Given $z^2 + az + a^2 = 0 \Rightarrow z = a\omega, a\omega^2$ where $\omega$ is cube-root of unity.

  Thus, it represents a pair of straight lines and $|z| = |a|$.
  $\arg(z) = \arg(a) + \arg(\omega)$ or $\arg(a) + \arg(\omega^2) = \pm \frac{2\pi}{3}$.
\item Given $x + \frac{1}{x} = 1 \Rightarrow x^2 - x + 1 = 0 \therefore x = -\omega, -\omega^2$.  Now, for
  $x = -\omega, p = \omega^{4000} + \frac{1}{\omega^{4000}} = \omega + \frac{1}{\omega} = -1$

  Similarly, for $x = -\omega^2, p = -1\Rightarrow 2^{2^n} = 2^{4k} = 16^k =$ a number with last digit as $6
  \Rightarrow q = 6 + 1 = 7\Rightarrow p + q = -1 + 7 = 6$.
\item $A(z_1) = \frac{2i}{\sqrt{3}},B(z_2) = \frac{2}{\sqrt{3}}\left(\frac{\sqrt{3}}{2} -
  i\frac{1}{2}\right) = 1 - \frac{i}{\sqrt{3}}, C(z_3) = \frac{2}{\sqrt{3}}\left(-\frac{\sqrt{3}}{2} -
  \frac{i}{2}\right) = -1-\frac{i}{\sqrt{3}}$

  Clearly, the points lie on the circle $z=2/\sqrt{3}$ and $\triangle ABC$ is equilateral and its centroid
  coincides with circumcentre. Hence,

  $z_1 + z_2 + z_3 = 0$ and $\overline{z_1} + \overline{z_2} + \overline{z_3} = 0$.
  Clearly, radius of incircle $= \frac{1}{\sqrt{3}}$ hence any point on circle is
  $\frac{1}{\sqrt{3}}(\cos\alpha + i\sin\alpha)$. $AP^2 = |z - z_1|^2 = |z|^2 + |z_1|^2 - (z\overline{z_1} +
  \overline{z}z_1)$

  $\Rightarrow AP^2 + BP^2 + CP^2 = 3|z|^2 + |z_1|^2 + |z_2|^2 + |z_3|^2 - z(\overline{z_1} + \overline{z_2}
  + \overline{z_3}) - \overline{z}(z_1 + z_2 + z_3)$
  $= 3\times\frac{1}{3} + \frac{4}{3} + \frac{4}{3} + \frac{4}{3} - 0 - 0 = 5$.
\item Let $O$ be the center of the polygon and $z_0, z_1, \ldots, z_{n - 1}$ represent the vertices $A_1, A_2,
  \ldots, A_n$. $\therefore z_0 = 1, z_1 = \alpha, z_2 = \alpha^2, \ldots, z_{n - 1} = \alpha^{n - 1}$ where
  $\alpha = e^{i2\pi/n}$

  $|A_1A_2|^2 =|\alpha^r - 1|^2 = |1 - \alpha^r|^2 = \left|1 - \cos\frac{2r\pi}{n} +
  i\sin\frac{2r\pi}{n}\right|^2$ $= \left(1 - \cos\frac{2r\pi}{n}\right)^2 + \sin^2\frac{2r\pi}{n} = 2 -
  2\cos\frac{2r\pi}{n}$

  $\displaystyle\sum_{r=1}^n |A_1A_2|^2 = 2(n - 1) - 2\left[\cos\frac{2\pi}{n} + \cos\frac{4\pi}{3} + \ldots
    + \cos\frac{2(n - 1)\pi}{n}\right]$
  $= 2(n - 1) -2.$ real part of $(\alpha + \alpha^2 + \ldots + \alpha^{n - 1}) = 2n [\because 1 + \alpha +
    \alpha^2 + \ldots + \alpha^{n - 1} = 0]$

  $|A_1A_2||A_1A_3|\ldots |A_1A_n| = |1 - \alpha||1 - \alpha^2|\ldots|1 - \alpha^{n - 1}|$ $= |(1 -
  \alpha)(1 - \alpha^2)\ldots(1 - \alpha^{n - 1})|$

  Since $1, \alpha, \alpha^2, \ldots, \alpha^{n - 1}$ are roots of $z^n - 1 = 0$. $(z - 1)(z - \alpha)(z -
  \alpha^2)\ldots(z - \alpha^{n - 1}) = z^n - 1$
  $\Rightarrow (z - \alpha)(z - \alpha^2)\ldots(z - \alpha^{n - 1}) = \frac{z^n - 1}{z - 1} = 1 + z + z^2 +
  \ldots + z^{n - 1}$

  Putting $z = 1$, we get
  $|(1 - \alpha)(1 - \alpha^2)\ldots(1 - \alpha^{n - 1})| = n \Rightarrow \frac{a}{b} = 2$.
\item Let L.H.S. $= z_1$ and R.H.S. $= z_2$ then $\overline{z_1} = \overline{z_2}$
  $\Rightarrow z_1\overline{z_1} = z_2\overline{z_2} \Rightarrow z_1^2 = z_2^2$

  $\Rightarrow \left(1 + \frac{x^2}{a^2}\right)\left(1 + \frac{x^2}{b^2}\right)\left(1 +
  \frac{x^2}{c^2}\right)\ldots = A^2 + B^2$.
\item Given, $x + iy + \alpha\sqrt{(x - 1)^2 + y^2} + 2i = 0$. Equating real and imaginary parts, we get

  $y + 2 = 0 \Rightarrow y = -2$ and $x + \alpha\sqrt{(x - 1)^2 + y^2} = 0$.
  Substituting the value of $y$, we get
  $\alpha\sqrt{x^2 - 2x + 5} = -x \Rightarrow (\alpha^2 - 1)x^2 - 2\alpha^2x +5\alpha^2 = 0$

  Because $x$ is real, the discriminant has to be greater than zero.
  $\Rightarrow 4\alpha^4 - 20\alpha^2(\alpha^2 - 1) \geq 0$
  $\Rightarrow \alpha^2 - 5\alpha^2 + 5 \geq 0 \Rightarrow -\frac{\sqrt{5}}{2}\leq\alpha\leq
  \frac{\sqrt{5}}{2}$.
\item Let $z = x + iy \Rightarrow 2\sqrt{x^2 + y^2} - 4a(x + iy) + 1 + ia = 0$.
  Equating real and imaginary parts, we get

  $2\sqrt{x^2 + y^2} - 4ax + 1 = 0$ and $-4ay + a = 0 \Rightarrow y = \frac{1}{4}$
  $\Rightarrow 2\sqrt{x^2 + \frac{1}{16}} - 4ax + 1 = 0 \Rightarrow 4\left(x^2 + \frac{1}{16}\right) =
  16a^2x^2 - 8ax + 1$

  $x^2(4 - 16a^2) + 8ax - \frac{3}{4} = 0 \Rightarrow x = \frac{-a}{1 - 4a^2} \pm
  \frac{1}{4}\frac{\sqrt{4a^2 + 3}}{1 - 4a^2}$.
\item $(x + iy)^5 = (x^5 - 10x^3y^2 + 5xy^4) + i(5x^4y - 10x^2y^3 + y^5)$.
  Taking modulus and squaring, we get
  $(x^2 + y^2)^5 = (x^5 - 10x^3y^2 + 5xy^4) + (5x^4y - 10x^2y^3 + y^5)^2$.
\item $(x + ia)(x + ib)(x + ic) = [(x^2 - ab) + i(a + b)x](x + ic) = (x^3 - abx - acx - bcx) + i(cx^2 - abc
  + ax^2 + bx^2)$

  Taking modulus and squaring, we get $(x^2 + a^2)(x^2 + b^2)(x^2 + c^2) = [x^3 -(ab + bc + ca)x] + [(a + b
    + c)x^2 - abc]^2$.
\item Given, $(1 + x)^n = a_0 + a_1x + a_2x^2 + \ldots + a_nx^n$. Substituting $x = i$,we get

  $(1 + i)^n = a_o + ia_1 - a_2 - ia_3 + a_4 + \ldots = (a_0 - a_2 + a_4 - \ldots) + i(a_1 - a_3 + a_5 -
  \ldots)$

  Taking modulus and squaring, we get $2^n = (a_0 - a_2 + a_4 - \ldots)^2 + (a_1 - a_3 + a_5 - \ldots)^2$.
\item Let $f(z) = m(z - i) + i$ and $f(z) = n(z + i) + 1 + i$ where $m$ and $n$ are quotients upon division.
  Substituting $z = i$ in the first equation and $z = -i$ in the second we obtain $f(i) = i$ and $f(-i) = 1
  + i$.

  Let $g(z)$ be the quotient and $az + b$ be the remainder upong division of $f(z)$ by $z^2 + 1.$ Hence we
  have $f(z) = g(z)(z^2 + 1) + az + b.$ Substituting $z = i$ and $z = -i,$ we get

  $f(i) = i = ai + b$ and $f(-i) = 1 + i = -ai + b$. Adding, we get $2b = 1 + 2i \Rightarrow b = \frac{1 +
    2i}{2} \Rightarrow ai = i - \frac{1 + 2i}{2}$.
\item Let $z = r_1e^{i\theta_1}, w = r_2e^{i\theta_2}$. $\because |z|\leq 1$ and $|w|\leq 1 \Rightarrow
  r_1\leq 1$ and $r_2\leq 1$

  $|z - w|^2 = (r_1\cos\theta_1 - r_2\cos\theta_2)^2 + (r_1\sin\theta_1 - r_2\sin\theta_2)^2$
  $= r_1^2 + r_2^2 - 2r_2r_2\cos(\theta_1 - \theta_2) = (r_1 - r_2)^2 + 2r_2r_2 - 2r_2r_2\cos(\theta_1 - \theta_2)$

  $= (r_1 - r_2)^2 + 4r_1r_2\sin\left(\frac{\theta_1 - \theta_2}{2}\right)^2\leq (r_1 - r_2)^2 + (\theta_1 -
  \theta_2)^2[\because r_1, r_2\leq 1$ and $\sin\theta \leq \theta]$
  $= (|z| - |w|)^2 + [\arg(z) - \arg(w)]^2$.
\item Let $z = re^{i\theta}$, then $\frac{z}{|z|} = e^{i\theta} = \cos\theta + i\sin\theta$
  $\Rightarrow \left|\frac{z}{|z|} - 1\right| = |(\cos\theta - 1) + i\sin\theta| = \sqrt{\cos\theta^2 -
  2\cos\theta + 1 + \sin^2\theta}$

  $= \sqrt{2 - 2\cos\theta} = \sqrt{4\sin^2\frac{\theta}{2}} = 2\sin\frac{\theta}{2}\leq \theta$
  $\Rightarrow \left|\frac{z}{|z|} - 1\right| \leq |arg(z)|$.
\item Clearly, $|z - 1| = |z - |z| + |z| - 1|\leq |z - |z|| + ||z| - 1|$
  $= |z|\left|\frac{z}{|z|} - 1\right| + ||z| - 1|$

  Using the result of previous problem, we get $|z - 1| \leq ||z| - 1|+|z||argz|$.
\item Let $z = r(\cos\theta + i\sin\theta)$, then $\frac{1}{z} = \frac{1}{r}(\cos\theta - i\sin\theta)$,
  $\left|z + \frac{1}{z}\right| = \left|\left(r + \frac{1}{r}\right)\cos\theta + i\left(r -
  \frac{1}{r}\right)\sin\theta\right|$

  $\Rightarrow \left(r + \frac{1}{r}\right)^2\cos^2\theta + i\left(r - \frac{1}{r}\right)^2\sin^2\theta =
  a^2$ $\Rightarrow \left(r - \frac{1}{r}\right)^2 = a^2 - 4\cos^2\theta$

  $r$ will be greatest when $r - \frac{1}{r}$ will be greatets i.e. $\cos\theta= 0 \Rightarrow r -
  \frac{1}{r} = a$ $\Rightarrow r_{max} = \frac{a + \sqrt{a^2 + 4}}{2}$

  Similarly, for lowest value of $r, \cos\theta = 1 \Rightarrow r - \frac{1}{r} = a^2 - 4 \Rightarrow r^2 -
  (a^2 - 4)r - 1 = 0$ $r_{min} = \frac{a^2- 4 - \sqrt{a^4 - 8a^2 + 20}}{2}$.
\item We have to prove that $|z_1 + z_2|^2 < (1 + c)|z_1|^2 + \left(1 + \frac{1}{c}\right)|z_2|^2$
  $\Rightarrow (z_1 + z_2)(\overline{z_1} + \overline{z_2}) < (1 + c)|z_1|^2 + \left(1 + \frac{1}{c}\right)|z_2|^2$

  $\Rightarrow |z_1|^2 + z_1\overline{z_2} + z_2\overline{z_1} + |z_1|^2 < (1 + c)|z_1|^2 + \left(1 + \frac{1}{c}\right)|z_2|^2$
  $\Rightarrow z_1\overline{z_2} + z_2\overline{z_1} < (1 + c)|z_1|^2 + \left(1 + \frac{1}{c}\right)|z_2|^2$

  $\Rightarrow (x_1 + iy_1)(x_2 - iy_2) + (x_2 + iy_2)(x_1 - iy_1) < \frac{1}{c}[c^2(x_1^2 + y_1^2) + (x_2^2 + y_2^2)]$
  $\Rightarrow 2cx_1x_2 + 2cy_1y_2 < c^2x_1^2 + c^2y_1^2 + x_2^2 + y_2^2$

  $\Rightarrow (cx_1 - x_2)^2 + (cy_1 - y_2)^2 > 0$ which is true.
\item Given $\left|\frac{z_1 - z_2}{z_1 + z_2}\right| = 1 \Rightarrow |z_1 - z_2|^2 = |z_1 + z_2|^2$
  $\Rightarrow (z_1 - z_2)(\overline{z_1} - \overline{z_2}) = (z_1 + z_2)(\overline{z_1} + \overline{z_2})$

  $\Rightarrow 2z_1\overline{z_2} = -2z_2\overline{z_1} \Rightarrow \overline{\left(\frac{z_1}{z_2}\right)}
  = -\frac{z_1}{z_2}$ $\Rightarrow \frac{z_1}{z_2} = $ purely imaginary $\Rightarrow i\frac{z_1}{z_2} =$
  real $= x$

  Now $\frac{z_1 + z_2}{z_1 - z_2} = \frac{z_1/z_2 + 1}{z_1/z_2 - 1} = \frac{-ix + 1}{-ix - 1} = \frac{-1 +
    x^2 + 2ix}{1 + x^2}$. If $\theta$ is the angle between given lines then
  $\tan\theta = \arg\frac{z_1 + z_2}{z_1 - z_2} = \frac{2x}{x^2 - 1}$.
\item Let $z_1 = r_1(\cos\theta_1 + i\sin\theta_1), z_2 = r_2(\cos\theta_2 + i\sin\theta_2)$. Also let $a =
  r\cos\alpha, b = r\sin\alpha$.
  $|az_1 + bz_2|^2 = |rr_1(\cos\theta_1 + i\sin\theta_1)\cos\alpha + rr_2(\cos\theta_2 + i\sin\theta_2)\sin\alpha|^2$

  $= r^2(r1\cos\theta_1\cos\alpha + r_2\cos\theta_2\sin\alpha)^2 + r^2(r_1\sin\theta_1\cos\alpha + r_2\sin\theta_2\sin\alpha)^2$
  $= r^2[r_1^2\cos^2\alpha + r_2^2\sin^2\alpha + 2r_1r_2\cos\alpha\sin\alpha\cos(\theta_1 - \theta_2)]$

  $= \frac{r^2}{2}[r_1^2(1 + \cos2\alpha) + r_2^2(1 - \cos2\alpha) + 2r_1r_2\sin2\alpha\cos(\theta_1 - \theta_2)]$
  $\frac{2|az_1 + bz_2|^2}{a^2 - b^2}= r_1^2 + r_2^2 + (r_1^2 - r_2^2)\cos2\alpha + 2r_2r_2\cos(\theta_1 - \theta_2)\sin2\alpha$

  $= A + B\cos2\alpha + C\sin2\alpha$ where $A = r_1^2 + r_2^2, B = r_1^2- r_2^2, C = 2r_1r_2\cos(\theta_1 - \theta_2)$
  Clearly, $-\sqrt{B^2 + C^2}\leq B\cos2\alpha + C\sin2\alpha \leq \sqrt{B^2 + C^2}$

  $\therefore A -\sqrt{B^2 + C^2}\leq A + B\cos2\alpha + C\sin2\alpha \leq A + \sqrt{B^2 + C^2}$
  $\therefore A -\sqrt{B^2 + C^2}\leq  \frac{2|az_1 + bz_2|^2}{a^2 + b^2}\leq A + \sqrt{B^2 + C^2}$

  Now $B^2 + C^2 = r_1^4 + r_2^4 - 2r_1^2r_2^2 + 4r_1^2r_2^2\cos^2(\theta_1 - \theta_2)$.
  Again $|z_1^2 + z_2^2| = |r_1^2(\cos2\theta_1 + i\sin2\theta_1) + r_2^2(\cos2\theta_2 + i\sin2\theta_2)|$
  $= \sqrt{(r_1^2\cos2\theta_1 + r_2^2\cos2\theta_2)^2 + (r_1^2\sin2\theta_1 + r_2^2\sin2\theta_2)^2}$

  $= \sqrt{r_1^4 + r_2^4 + 2r_1^2r_2^2\cos2(\theta_1 - \theta_2)}$
  $= \sqrt{r_1^4 + r_2^4 + 2r_1^2r_2^2[2\cos^2(\theta_1 - \theta_2) - 1]} = \sqrt{B^2 + C^2}$

  $A = r_1^2 + r_2^2 = |z_1|^2 + |z_2|^2$
  Hence, $|z_1|^2 + |z_2|^2 - |z_1^2 + z_2^2| \leq 2\frac{|az_1 + bz_2|^2}{a^2 + b^2}\leq |z_1|^2 + |z_2|^2
  + |z_1^2 + z_2^2|$.
\item Given $z = \frac{b + ic}{1 + a} \therefore iz = \frac{-c + ib}{1 + a} \Rightarrow \frac{1}{iz} =
  \frac{1 + a}{-c + ib}$. Using componendo and dividendo, we get
  $\Rightarrow \frac{1 + iz}{1 - iz} = \frac{1 + a - c + ib}{1 + a + c - ib}$.
  Also, given $a^2 + b^2 + c^2 = 1 \Rightarrow a^2 + b^2 = 1 - c^2$

  $\Rightarrow (a + ib)(a - ib) = (1 + c)(1 - c)\Rightarrow \frac{a + ib}{1 - c} = \frac{1 + c}{a - ib} =
  \frac{1}{u}$(say) $\therefore \frac{1 + iz}{1 - iz} = \frac{a + ib + 1 - c}{1 + c + a - ib} = \frac{a + ib
    + u(a + ib)}{1 + c + u(1 + c)}$ $= \frac{a + ib}{1 + c}$.
\item We can write that $(x - a)(x - b)\ldots (x - k) = x^n + p_1x^{n - 1} + p_2x^{n - 2} + \ldots + p_{n -
  1}x + p_n$

  Substituting $x = i$, we get
  $(i - a)(i - b)\ldots (i - k) = i^n + p_1i^{n - 1} + p_2i^{n - 2} + \ldots + p_{n - 1}i +  p_n$.
  Dividing both sides by $i^n$, we get
  $(1 + ia)(1 + ib)\ldots(1 + ik) = 1 + \frac{p_1}{i} + \frac{p_2}{i^2} + \ldots$

  Taking modulus and squaring, we get
  $(1 + a^2)(1 + b^2)\ldots (1 + k^2) = (1 - p_2 + p_4 + \ldots)^2 + (p_1 - p_3 + \ldots)^2$.
\item $3 + 2i$ is one value of $x$ for which $f(3 + 2i) = a + ib$
  $\Rightarrow x = 3 + 2i \Rightarrow x^2 - 6x + 13 = 0$

  $f(x) = x^4 - 8x^3 + 4x^2 + 4x + 39 = (x^2 - 6x + 13)(x^2 - 2x -21) -96x + 312$
  $\Rightarrow f(3 + 2i) = -96(3 + 2i) + 312 = 24 - 192i = a + ib$
  $\Rightarrow a:b = 1:-8$.
\item Given $\frac{A}{B} + \frac{B}{A} = 1 \Rightarrow A^2 - AB + B^2 = 0$.
  $A = \frac{B \pm \sqrt{3}iB}{2} = -\omega B, -\omega^2B\Rightarrow |A| = |B|$

  $|A - B| = |-\omega B - B|$ or $|-\omega^2B - B| = |\omega^2 B|$ or $|\omega B|$
  $\Rightarrow |A - B| = |B|$.
  Thus, $|A| = |B| = |A - B|$ making the triangle equilateral.
\item Given $z^n = (z + 1)^n \Rightarrow |z|^n = |z + 1|^n$
  $\Rightarrow |z| = |z + 1|\Rightarrow x^2 = (x^2 + 2x + 1) \Rightarrow 2x + 1 = 0$,
  which is the equation of a straight line on which roots of the given equation will lie.
\item Let $z_1, z_2, z_3, z_4$ be represented by the points $A, B, C, D$ respectively.
  $\therefore AD = |z_1 - z_4|$ and $BC = |z_2 - z_3|$

  Let $a = (z_1 - z_4)(z_2 - z_3), b = (z_2 - z_4)(z_3 - z_1)$ and $c = (z_3 - z_4)(z_1 - z_2)$
  $b + c = (z_2 - z_4)(z_3 - z_1) + (z_3 - z_4)(z_1 - z_2) = -(z_1 - z_4)(z_2 - z_3) = -a$

  $|a| = |b + c| \leq |b| + |c| \Rightarrow |-(z_1 - z_4)(z_2 - z_3)| = |(z_2 - z_4)(z_3 - z_1)| + |(z_3 -
  z_4)(z_1 - z_2)|$
  $\Rightarrow AD.BC\leq BD.CA + CD.AB$.
\item Euqation of a line joining points $a$ and $ib$ is
  $\startbmatrix\NC z \NC \overline{z} \NC 1\NR\NC a \NC \overline{a} \NC 1 \NR\NC ib \NC i\overline{b} \NC
  1\NR\stopbmatrix = 0$ or $(\overline{a} +i\overline{b})z - (a - ib)\overline{z} - i(a\overline{b} +
  \overline{a}b) = 0$

  $\Rightarrow (a + ib)z - (a -ib)\overline{z} - 2abi = 0[\because a, b\in R \therefore a = \overline{a}, b
    = \overline{b}]\Rightarrow (a + ib)z - (a -ib)\overline{z} = 2abi$
  $\Rightarrow \left(\frac{1}{2a} - \frac{i}{2b}\right)z + \left(\frac{1}{2a} +
  \frac{i}{2b}\right)\overline{z} = 1$.
\item Let $z_1 = r_1e^{i\theta_1}$ and $z_2 = r_2e^{i\theta_2}$.

  Then $r_1 - r_2 = \sqrt{(r_1\cos\theta_1 - r_2\cos\theta_2)^2 + (r_1\sin\theta_1 - r_2\sin\theta_2)^2}$

  $\Rightarrow 2r_1r_2 = 2r_1r_2\cos(\theta_1 - \theta_2)\Rightarrow \cos(\theta_1- \theta_2) = \cos 2n\pi$
  $\Rightarrow \arg(z_1) - \arg(z_2) = 2n\pi$.
\item $\triangle ABC$ and $\triangle DOE$ will be similar if
  $\frac{AC}{AB} = \frac{DE}{DO}$ and $\angle BAC = \angle ODE$

  $\Rightarrow \left|\frac{z_3 - z_1}{z_2 - z_1}\right| = \left|\frac{z_5 - z_4}{0 - z_4}\right|$ and
  $\arg\left(\frac{z_3 - z_1}{z_2 - z_1}\right) = \arg\left(\frac{z_5 - z_4}{0 - z_4}\right)$

  $\Rightarrow \frac{z_3 - z_1}{z_2 - z_1} = \frac{z_5 - z_4}{0 - z_4}$.
  Solving this yields $(z_3 - z_2)z_4 = (z_1 - z_2)z_5$ and hence triangles are similar.
\item Given $OA = 1$ and $|z| = 1 = OP \Rightarrow OA = OP$. $OP_0 = |z_0|$ and $OQ = |z\overline{z_0}| =
  |z||\overline{z_0}| = |z_0|$

  $\Rightarrow OP_0 = OQ$. Also given that $\angle P_0OP = \arg\frac{z_0}{z}$.
  $\angle AOQ = \arg\left(\frac{1}{z\overline{z_0}}\right) =
  \arg\left(\frac{\overline{z}}{\overline{z_0}}\right)[\because z\overline{z} = 1]$

  $= -\arg\left(\frac{\overline{z_0}}{\overline{z}}\right)= -\arg\overline{\left(\frac{z_0}{z}\right)} =
  \arg\left(\frac{z_0}{z}\right) = \angle P_0OP$ and thus the triangles are congruent.
\item $P = \frac{az_2 + bz_1}{a + b}, Q = \frac{az_2 - bz_1}{a - b}$
  $OP^2 = \left|\frac{az_2 + bz_1}{a + b}\right|^2 = \left(\frac{az_2 + bz_1}{a +
  b}\right)\left(\frac{a\overline{z_2} + b\overline{z_1}}{a + b}\right)$

  $= \frac{1}{a^2 + b^2}[a^2|z_2|^2 + b^2|z_1|^2 + ab(z_1\overline{z_2} + \overline{z_1}z_2)]$.
  Similalry $OQ^2$ can be computed and the sum be found.
\item Let $c\neq 0$, then $c = -(a + b)$ so we can write $az_1 + bz_2 - (a + b)z_3 = 0 \Rightarrow z_3 =
  \frac{az_1 + bz_2}{a + b}$.

  Thus, we see that $z_3$ divides line segment $z_1z_2$ in the ratio of $a:b$ making all three of them
  collinear.
\item Equation of a line passing through origin is $a\overline{z} + \overline{a}z = 0$. Let us assume that
  all the points lie on the same side of the above line, so we have

  $a\overline{z_i} + \overline{a}z_i > 0$ or $< 0$ for $i = 1, 2, 3, \ldots, n$.
  Thus, $a\displaystyle\sum_{i = 1}^n\overline{z_i} + \overline{a}\sum_{i = 1}^nz_i > 0$ or $< 0$

  But it is given that $\displaystyle\sum_{i = 1}^n z_i = 0 \Rightarrow \sum_{i = 1}^n \overline{z_i} = 0$
  $\therefore\displaystyle a\sum_{i =1}^n\overline{z_i} + \overline{a}\sum_{i = 1}^nz_i = 0$,
  which is in contradiction with equation above. So all points cannot lie on the same side of line.
\item Let $OA$ and $OB$ be the unit vectors representing $z_1$ and $z_2$, then we have
  $\vec{OA} = \frac{z_1}{|z_1|}, \vec{OB} = \frac{z_2}{|z_2|}$

  Therefore equation of bisector will be $z = t\left(\frac{z_1}{|z_1|} + \frac{z_2}{|z_2|}\right) =
  \frac{6}{5}t,$ where is an arbitrary positive integer.
\item The diagram is given below:
  \startplacefigure[location={left, none}]
    \startMPcode
      pair a; pair b; pair c;
      pickup pencircle scaled 0.2pt;
      a = (0, 4cm); b = (-2cm, 0); c = (1cm, 0);
      draw a -- b -- c -- cycle;
      pair p; pair q; pair r; pair d;
      p = whatever[b, c]; a - p = whatever * (b - c) rotated 90;
      q = whatever[c, a]; b - q = whatever * (c - a) rotated 90;
      r = whatever[a, b]; c - r = whatever * (a -b) rotated 90;
      d = whatever[a, p] = whatever[b, q]; % orthocenter
      pickup pencircle scaled 2pt;
      drawdot d;
      pair l;
      l = (a -- p) intersectionpoint  (b -- c);
      pickup pencircle scaled 0.2pt;
      draw a -- l;
      pair m;
      m = (b -- q) intersectionpoint  (a -- c);
      draw b -- m;
      label.bot("$L$", l);
      label.rt("$M$", m);
      label.ulft("$H$", d);
      label.top("$A$", a);
      label.llft("$B$", b);
      label.lrt("$C$", c);
      draw unitsquare scaled 5 rotated angle (a - l) shifted l;
      draw unitsquare scaled 5 rotated angle (b - m) shifted m;
      label.bot("$a$", (b + c)/2);
      label.ulft("$c$", (b + a)/2);
      label.urt("$b$", (c + a)/2);
    \stopMPcode
  \stopplacefigure
  Let $AL$ be perpendicular on $BC$ and $H$ be orthocenter of the $\triangle ABC$.

  $\frac{BL}{LC} = \frac{c\cos B}{b\cos C} = \frac{c\sec C}{b\sec B}$, thus $L$ divides $BC$ internally in the ratio of $c\sec
  C:b\sec B$,  $L = \frac{z_3c\sec C + z_2b\sec B}{c\sec C + b\sec B}$

  $\frac{AH}{HL} = \frac{\Delta ABH}{\Delta HBL} = \frac{\frac{1}{2}AB.BH\sin\angle
    ABM}{\frac{1}{2}BL.BH.\sin\angle MBC} = \frac{c\cos A}{c\cos B\cos C}[\because \angle ABM = 90^\circ -
    A, \angle MBC = 90^\circ - C]$

  $= \frac{a\cos A}{a\cos B\cos C} = \frac{(b\cos C + c\cos B)\cos A}{a\cos B\cos C} = \frac{b\sec B + c\sec C}{a\sec A}$

  $H = \frac{z_1a\sec A + z_2b\sec B + z_3c\sec C}{a\sec A + b\sec B + c\sec C}$

  Since the above expression is similar w.r.t. $A, B$ and $C$, therefore it will also lie on the perpendiculars from $B$ and $C$ to
  opposing sides as well.
  Thus, orthocenter $H = \frac{z_1a\sec A + z_2b\sec B + z_3c\sec C}{a\sec A + b\sec B + c\sec C}$

  $H = \frac{z_1k\sin A\sec A + z_2k\sin B\sec B + z_3k\sin C\sec C}{k\sin A\sec A + k\sin B\sec B + k\sin C\sec C}$,
  $H = \frac{z_1\tan A + z_2\tan B + z_3\tan C}{\tan A + \tan B + \tan C}$.
\item The diagram is given below:
  \startplacefigure[location={left, none}]
    \startMPcode
      pair a; pair b; pair c;
      pickup pencircle scaled 0.2pt;
      a = (0, 4cm); b = (-2cm, 0); c = (1cm, 0);
      draw a -- b -- c -- cycle;
      pair p; pair q; pair r; pair d; pair m; pair n;
      p = whatever[b, c]; a - p = whatever * (b - c) rotated 90;
      q = whatever[c, a]; b - q = whatever * (c - a) rotated 90;
      r = whatever[a, b]; c - r = whatever * (a -b) rotated 90;
      d = whatever[a, p] = whatever[b, q]; % orthocenter
      n = 1/4(a + b + c + d); % remarkably...
      m = d rotatedabout(n, 180); % M is also the circumcentre
      path circumcircle;
      circumcircle = fullcircle scaled 2 abs(m - a) shifted m;
      draw circumcircle;
      pair l;
      pickup pencircle scaled 2pt;
      drawdot m;
      pickup pencircle scaled 0.2pt;
      pair ll;
      ll = (a-- (a - 100(a - m))) intersectionpoint (b -- c);
      draw a -- ll;
      draw b -- m;
      draw c -- m;
      pair t;
      t = whatever[b, c];
      t - m = whatever * (b - c) rotated 90; % perpendicular calculation
      draw m -- t;
      label.bot("$D$", ll);
      label.bot("$L$", t);
      label.rt("$O$", m + (0.2cm, 0));
      label.top("$A$", a);
      label.llft("$B$", b);
      label.lrt("$C$", c);
      draw fullcircle scaled 16 rotated angle (a - m) shifted m cutafter (m -- b) withcolor red; % angle marking
      draw fullcircle scaled 16 rotated angle (b - m) shifted m cutafter (m -- ll) withcolor .7green;
      draw fullcircle scaled 24 rotated angle (ll - m) shifted m cutafter (m -- c) withcolor .7green;
      draw fullcircle scaled 16 rotated angle (c - m) shifted m cutafter (m -- a) withcolor .7blue;
      label.lft("$2C$", m - (0.2cm, 0)) withcolor red;
      label.l("$\pi - 2C$", m - (0, 0.5cm)) withcolor .7green;
      label.lrt("$\pi - 2B$", m - (-0.5, 0.5cm)) withcolor .7green;
      label.rt("$\pi - 2B$", m + (0.4cm, 0)) withcolor .7blue;
      label.bot("$a$", (b + c)/2 + (.2cm, 0cm));
      label.ulft("$c$", (b + a)/2);
      label.urt("$b$", (c + a)/2);
    \stopMPcode
  \stopplacefigure
  Let $O$ be the circumcenter of $\triangle ABC$ where $A=z_1, B=z_2$ and $C=z_3. \frac{BD}{DC} =
  \frac{\frac{1}{2}BD.OL}{\frac{1}{2}DC.OL} = \frac{\Delta BOD}{\Delta COD}$

  $= \frac{\frac{1}{2}OB.OD.\sin(\pi - 2C)}{\frac{1}{2}OC.OD\sin(\pi - 2C)} = \frac{\sin2C}{\sin2B}$.  Thus,
  $D$ divides $BC$ internally in the ratio $\sin2C:\sin2B \Rightarrow D = \frac{z_3\sin2C +
    z_2\sin2B}{\sin2C + \sin2B}$

  The complex number dividing $AD$ internally in the ratio $\sin2B+\sin2C:\sin2A$ is $\frac{z_1\sin 2A +
    z_2\sin 2B + z_3\sin 2C}{\sin 2A + \sin 2B + \sin 2C}$

  Since the above expression is similar w.r.t. $A, B$ and $C$, therefore it will also lie on the
  perpendicular bisectors on $AC$ and $AB$ as well.

  Let $BO$ produced meet $AC$ at $E$ and $CO$ produced meet $AB$ at $F$. We can show that, the complex
  numner representing the point dividing the line segment $BE$ internally in the ratio $(\sin2C +
  \sin2A):\sin2B$ and the complex number representing the point dividing the line segment $CF$ internally in
  the ratio $(\sin2A+ \sin2B):\sin2C$ will be each $= \frac{z_1\sin 2A + z_2\sin 2B + z_3\sin 2C}{\sin 2A +
    \sin 2B + \sin 2C}$

  Thus, circumcenter is $\frac{z_1\sin 2A + z_2\sin 2B + z_3\sin 2C}{\sin 2A + \sin 2B + \sin 2C}$
\item Let $z$ be the circumcenter of the triangle represented by $A(z_1), B(z_2)$ and $C(z_3)$ respectively,
  then $|z - z_1| = |z - z_2| = |z - z_3|$ so we have $|z - z_1| = |z - z_2|$
  $\Rightarrow |z - z_1|^2 = |z - z_2|^2 \Rightarrow (z - z_1)(\overline{z} - \overline{z_1}) = (z -
  z_2)(\overline{z} - \overline{z_2})$

  $\Rightarrow z\overline{z} + z_1\overline{z_1} - \overline{z}z_1 - z\overline{z_1} = z\overline{z} +
  z_2\overline{z_1} - \overline{z}z_2 - z\overline{z_2}$
  $\Rightarrow z(\overline{z_1} - \overline{z_2})+ \overline{z}(z_1 - z_2) = z_1\overline{z_1} - z_2\overline{z_2}$

  Similarly considering $|z - z_1| = |z - z_3|$, we will have
  $\Rightarrow z(\overline{z_1} - \overline{z_3})+ \overline{z}(z_1 - z_3) = z_1\overline{z_1} - z_3\overline{z_3}\stopalign$

  We have to eliminate $\overline{z}$ from equation (1) and (2) i.e. multiplying equation (1) with $(z_1 - z_3)$ and (2) with $(z_1
  - z_2)$, we get following

  $z[\overline{z_1}(z_2 - z_3) + \overline{z_2}(z_3 - z_1) + \overline{z_3}(z_1 - z_2)] = z_1\overline{z_1}(z_2 - z_3) +
  z_2\overline{z_2}(z_3 - z_1) + z_3\overline{z_3}(z_1 - z_2)$
  $\Rightarrow z = \frac{\sum z_1\overline{z_1}(z_2 - z_3)}{\sum \overline{z_1}(z_2 - z_3)}$.
\item Let $z$ be the orthocenter of $\triangle A(z_1)B(z_2)C(z_3)$ i.e. the intersection point of
  perpendiculars on sides from opposite vertices.

  Since $AH\perp BC \therefore \arg\left(\frac{z_1 - z}{z_3 - z_2}\right) = \pm\frac{\pi}{2}$
  $\Rightarrow \frac{z_1 - z}{z_3 - z_2}$ is purely imaginary.

  $\Rightarrow \overline{\left(\frac{z_1 - z}{z_3 - z_2}\right)} = -\left(\frac{z_1 - z}{z_3 -
    z_2}\right)\Rightarrow\frac{\overline{z_1} - \overline{z}}{\overline{z_3} - \overline{z_2}} = \frac{z - z_1}{z_3 - z_2}$
  $\Rightarrow \overline{z_1} - \overline{z} = \frac{(z - z_1)(\overline{z_3} - \overline{z_2})}{z_3 - z_2}$

  Similarly for $BH\perp AC, \overline{z_2} - \overline{z} = \frac{(z - z_2)(\overline{z_1} - \overline{z_2})}{z_1 - z_3}$

  Eliminating $\overline{z}$ like last problem we arrive at the desired result.
\item We have $\angle CBA =\frac{2\pi}{3}$, therefore
  $\frac{z_3 - z_2}{z_1 - z_2} = \frac{|z_3 - z_2|}{|z_1 - z_2|}\left[\cos\frac{2\pi}{3} + i\sin\frac{2\pi}{3}\right]$
  $= -\frac{1}{2} + \frac{i\sqrt{3}}{2}[\because BC = AB]$

  $z_3 + \left(\frac{1}{2} - \frac{i\sqrt{3}}{2}\right)z_1 = \left(\frac{3}{2} - \frac{i\sqrt{3}}{2}\right)z_2$

  Solving this yields $2\sqrt{3}z_2 = (\sqrt{3} - i)z_1 + (\sqrt{3} + i)z_3$.
  Also, since diagonals bisect each other $\Rightarrow \frac{z_1 + z_3}{2} = \frac{z_2 + z_4}{2}$,
  $z_4 = z_1 + z_3 - z_2$
  Substituting the value of $z_2$, we get
  $2\sqrt{3}z_4 = (\sqrt{3} + i)z_1 + (\sqrt{3} - i)z_3$.
\item Since $\angle PQR = \angle PRQ = \frac{1}{2}(\pi - \alpha) \therefore PQ = PR$ Also, $\angle QPR = \pi
  - 2\left(\frac{\pi}{2} - \frac{\alpha}{2}\right) = \alpha$
  $\therefore \arg\frac{z_3 - z_1}{z_2 - z_1} = \alpha \Rightarrow \frac{z_3 - z_1}{z_2 - z_1} =
  \frac{PR}{RQ}(\cos\alpha + i\sin\alpha)$

  $\Rightarrow \frac{z_3 - z_1}{z_2 - z_1} -1 = (\cos\alpha - 1) + i\sin\alpha \Rightarrow \frac{z_3 -
    z_2}{z_2 - z_1} = -2\sin^2\frac{\alpha}{2} + i2\sin\frac{\alpha}{2}\cos\frac{\alpha}{2}$

  $\Rightarrow \left(\frac{z_3 - z_2}{z_2 - z_1}\right)^2 =
  -4\sin^2\frac{\alpha}{2}\left[\cos\frac{\alpha}{2} + i\sin\frac{\alpha}{2}\right]^2 =
  -4\sin^2\frac{\alpha}{2}[\cos\alpha + i\sin\alpha] = -4\sin^2\frac{\alpha}{2}.\frac{z_3 - z_1}{z_2 - z_1}$

  $\Rightarrow (z_3 - z_2)^2 = 4(z_3 - z_1)(z_1 - z_2)\sin^2\frac{\alpha}{2}$.
\item Let $C$ be the center of a regular polygon of $n$ sides. Let $A_1(z_1), A_2(z_2)$ and $A_3(z_3)$ be its three
  consecutive vertices.

  $\angle CA_2A_1 = \frac{1}{2}\left(\pi - \frac{2\pi}{n}\right) \therefore A_1A_2A_3 = \pi - \frac{2\pi}{n}$

  {\bf Case I:} When $z_1, z_2, z_3$ are in anticlockwise order. $\Rightarrow z_1 - z_2 = (z_3 - z_2)e^{i\left(\pi -
    2\pi/n\right)}[\because A_1A_2 = A_3A_2]$

  $z_1 - z_2 = (z_2 - z_3)e^{-i2\pi/n}[\because e^{i\pi} = -1] \Rightarrow z_3 = z_2 - (z_1 - z_2)e^{i2\pi/n}$

  {\bf Case II:} When $z_1, z_2, z_3$ are in clockwise order. $\Rightarrow z_3 - z_2 = (z_1 - z_2)e^{i\left(\pi -
    i2\pi/n\right)}$

  $z_3 = z_2 + (z_2 - z_1)e^{-i2\pi/n}$.
\item Let $O$ be the origin and the complex number representing $A_1$ be $z$, then $A_2, A_3, A_4$ will be
  represented by $ze^{i2\pi/n}, ze^{i4\pi/n}, ze^{i6\pi/n}$. Let $|z| = a$

  $A_1A_2 = \left|z - ze^{i2\pi/n}\right| = |z|\left|1 - \cos\frac{2\pi}{n} - i\sin\frac{2\pi}{n}\right|$
  $= a\sqrt{\left(1 - \cos\frac{2\pi}{n}\right)^2 + \sin^2\frac{2\pi}{n}} = a\sqrt{2\left(1 -
    \cos\frac{2\pi}{n}\right)} = 2a\sin\frac{\pi}{n}$

  Similarly, $A_1A_3 = 2a\sin\frac{2\pi}{n}$ and $A_1A_4 = 2a\sin\frac{3\pi}{n}$

  Given $\frac{1}{A_1A_2} = \frac{1}{A_1A_3} + \frac{1}{A_1A_4}\therefore \frac{1}{2a\sin\frac{\pi}{n}} =
  \frac{1}{2a\sin\frac{2\pi}{n}} + \frac{1}{2a\sin\frac{3\pi}{n}}$
  $\Rightarrow \sin\frac{\pi}{n}\left(\sin\frac{3\pi}{n} + \sin\frac{2\pi}{n}\right) =
  \sin\frac{2\pi}{n}\sin\frac{3\pi}{n}$

  $\Rightarrow \sin\frac{3\pi}{n} + \sin\frac{2\pi}{n} = 2\cos\frac{2\pi}{n}\sin\frac{3\pi}{n} =
  \sin\frac{4\pi}{n} + \sin\frac{2\pi}{n}$
  $\Rightarrow \sin\frac{3\pi}{n} = \sin\frac{4\pi}{n}\Rightarrow \frac{3\pi}{n} = m\pi +
  (-1)^n\frac{4\pi}{n}, m = 0,\pm1, \pm2,\ldots$

  If $m = 0\Rightarrow \frac{3\pi}{n} = \frac{4\pi}{n} \Rightarrow 3 = 4$ (not possible).
  If $m = 1\Rightarrow \frac{3\pi}{n} = \pi - \frac{4\pi}{n}\Rightarrow n = 7$.
  If $m = 2,3 \ldots, -1, -2,\ldots$ gives values of $n$ which are not possible. Thus $n = 7$.
\item Given, $|z| = 2$. Let $z_1 = -1 + 5z \Rightarrow z_1 + 1 = 5z$.

  $|z_1 + 1| = |5z| = 5|z| = 10$
  $\Rightarrow z_1$ lies on a circle with center $(-1, 0)$ having radius $10$.
\item Given, $|z - 4 + 3i|\leq 2 \Rightarrow ||z| - |4 - 3i||\leq 2\Rightarrow ||z| - 5|\leq 2 \Rightarrow
  -2 \leq |z| - 5\leq 2 \Rightarrow 3\leq |z|\leq 7$.
\item $|z - 6 - 8i|\leq |4| \Rightarrow -4 \leq ||z| - |6 + 8i|| \leq 4\Rightarrow -4 \leq |z| - 10 \leq 10
  \Rightarrow 6\leq |z|\leq 14$.
\item The diagram is given below:
  \startplacefigure[location={left,none}]
    \startMPcode
      pickup pencircle scaled 0.2pt;
      pair c;
      c = (0cm, 2.5cm);
      draw fullcircle scaled 3cm shifted c;
      drawarrow (-0.5cm, 0cm) -- (4cm, 0cm);
      drawarrow (0cm, -0.5cm) -- (0cm, 5cm);
      draw (0, 0) -- (3cm, 4cm);
      draw (0, 2.5cm) -- (1.2cm, 1.6cm);
      label.rt("$x$", (4cm, 0));
      label.top("$y$", (0, 5cm));
      label.llft("$O$", (0,0));
      label.lft("$C(0, 25)$", c);
      label.rt("$P$", (1.2cm, 1.6cm));
      draw fullcircle scaled 16 rotated angle ((1cm, 0) - (0, 0)) shifted (0, 0) cutafter ((0, 0) -- (3cm, 4cm));
      % angle marking
      label.rt("$\theta$", (0.2cm, 0.25cm));
      draw fullcircle scaled 16 rotated angle ((0, 0) - c) shifted c cutafter (c -- (1.2cm, 1.6cm));
      label.rt("$\theta$", c - (-0.1cm, 0.4cm));
    \stopMPcode
  \stopplacefigure
  Given $z - 25i \leq 15$, which represents a circle having center $(0, 25)$ and a radius $15$.
  Let $OP$ be tangent to the circle at point $P$, then $\angle XOP$ will represent least
  value of $\arg(z)$.

  Let $\angle XOP = \theta$ then $\angle OCP = \theta$. Now $OC = 25, CP = 15 \therefore OP = 20\therefore
  \tan\theta = \frac{OP}{CP} = \frac{4}{3}$. $\therefore$ Least value of $\arg(z) = \theta =
  \tan^{-1}\frac{4}{3}$
  \vskip 2.4cm
\item Given, $|z - z_1|^2 + |z - z_2|^2 = k\Rightarrow |z|^2 + |z_1|^2 - 2z\overline{z_1} + |z|^2 + |z_2|^2
  - 2z\overline{z_2} = k$

  $\Rightarrow 2|z|^2 - 2z(\overline{z_1} + \overline{z_2}) = k - (|z_1|^2 + |z_2|^2)\Rightarrow |z|^2 -
  2z\left(\frac{\overline{z_1 + z_2}}{2}\right) + \frac{1}{4}|z_1 + z_2|^2 = \frac{k}{2} + \frac{1}{4}[|z_1
    + z_2|^2 - 2|z_1|^2 -2|z_2|^2]$

  $\Rightarrow \left|z - \frac{z_1 + z_2}{2}\right|^2 = \frac{1}{2}\left[k - \frac{1}{2}|z_1 -
    z_2|^2\right]$. The above equation represents a circle with center at $\frac{z_1 + z_2}{2}$ and radius
  $\frac{1}{2}\sqrt{2k - |z_1 - z_2|^2}$ provided $k\geq \frac{|z_1 - z_2|^2}{2}$.
\item Since $|z - 1| = 1, z$ represents a circle with center $(1, 0)$ and a radius of of $1$. It is shown
  below:
  \startplacefigure[location={left,none}]
    \startMPcode
      pickup pencircle scaled 0.2pt;
      pair c; c = (1cm, 0);
      draw fullcircle scaled 2cm shifted c;
      drawarrow (-0.5cm, 0) -- (3cm, 0);
      drawarrow (0, -1.5cm) -- (0, 1.5cm);
      label.rt("$x$", (3cm, 0));
      label.top("$y$", (0, 1.5cm));
      label.llft("$O$", (0,0));
      label.bot("$C(1, 0)$", c);
    \stopMPcode
  \stopplacefigure
  Now $|z - 1| = 1$. Let $z = x + iy$ then $x^2 + y^2 = 2x$. Also,

  $\frac{z - 2}{z} = \frac{x - 2 + iy}{x + iy} = \frac{x^2 - 2x + y^2 + 2iy}{x^2 + y^2} = i\frac{y}{x}$

  \noindent{\bf Case I.} When $z$ lies in the first quadrant. This implies $\arg(z) = \theta$, where
  $\tan\theta = \frac{y}{x} \therefore i\tan[\arg(z)] = i\tan\theta = i\frac{y}{x}$.

  \noindent{\bf Case II.} When $z$ lies in the fourth quadrant. Thus, $\arg(z) = 2\pi - \theta$, where
  $\tan\theta = \frac{-y}{x}\therefore i\tan[\arg(z)] = i\tan(2\pi - \theta) = i\frac{y}{x}$.
\item Let $z = x + iy$. Now we have $\frac{z - 1}{z + 1} = \frac{(x^2 - 1) + y^2}{(x + 1)^2 + y^2} +
  i\frac{2y}{(x + 1)^2 + y^2}$

  $\therefore \arg\left(\frac{z - 1}{z + 1}\right) = \frac{\pi}{4}\Rightarrow \tan\left(\arg\left(\frac{z -
    1}{z + 1}\right)\right) = \frac{2y}{x^2 - 1 + y^2}$

  $\Rightarrow x^2 + y^2 - 1 -2 y = 0 \Rightarrow x^2 + (y - 1)^2 = 2$, which is equation of a circle having
  center at $(0, 1)$ and radius $\sqrt{2}$.
\item Let $z = x + iy$. Now, $u + iv = (z - 1)(\cos\alpha - i\sin\alpha) + \frac{1}{z - 1}(\cos\alpha +
  i\sin\alpha)= (x - 1)\cos\alpha + y\sin\alpha + i[y\cos\alpha - (x - 1)\sin\alpha] + \frac{x - 1 - iy}{(x
    - 1)^2 + y^2}(\cos\alpha + i\sin\alpha) = 0$

  Equating imaginary parts, we get
  $v = y\cos\alpha - (x - 1)\sin\alpha + \frac{(x - 1)\sin\alpha - y\cos\alpha}{(x - 1)^2 + y^2} =
  0\Rightarrow [y\cos\alpha - (x - 1)\sin\alpha][(x - 1)^2 + y^2] = 0$

  $\therefore $ Either $y\cos\alpha - (x - 1)\sin\alpha = 0 \Rightarrow y = \tan\alpha(x - 1)$, which is a
  straight line passing through $(1, 0)$ or $(x - 1)^2 + y^2 - 1 = 0$ which is a circle with center $(1, 0)$
  and unit radius.
\item Given, $1 + a_1z + a_2z^2 + \cdots + a_nz^n = 0 \Rightarrow |a_1z| + |a_2z^2| + \cdots + |a_nz^b|\geq
  1$ and

  L.H.S. $ < 2|z| + 2|z|^2 + \cdots$ to $\infty[\because |a_n| < 2]$.

  Let $|z| < 1$ then $\frac{2|z|}{1 - |z|} < 1 \Rightarrow |z| > \frac{1}{3}$

  When $|z|> 1$, clearly $|z| > \frac{1}{3}$; hence, $z$ does not lie in the interior of the circle with
  radius $\frac{1}{3}$.
\item Given, $z^n\cos\theta_0 + z^{n - 1}\cos\theta_1 + \cdots + \cos\theta_n = 2$
  $\Rightarrow 2 = |z^n\cos\theta_0 + z^{n - 1}\cos\theta_1 + \cdots + \cos\theta_n|$

  $< |z^n\cos\theta_0| + |z^{n - 1}\cos\theta_1| + \cdots + |\cos\theta_n|$
  $= |z^n||\cos\theta_0| + |z^{n - 1}||\cos\theta_1| + \cdots + |\cos\theta_n|$

  $\leq |z|^n + |z|^{n - 1}  + \cdots + 1 < 1 + |z| + |z|^2 + \cdots$ to $\infty$
  $\Rightarrow 2 = \frac{1}{1 - |z|} \Rightarrow |z| > \frac{1}{2} [$ when $|z| < 1]$

  Hence $z$ lies outside the circle $|z| = \frac{1}{2}$.
  Thus all roots of the given equation lie outside the circle $|z| = \frac{1}{2}$.
\item Recall that points $z_1, z_2, z_3$ are concyclic if $\left(\frac{z_2 - z_4}{z_1 -
  z_4}\right)\left(\frac{z_1 - z_3}{z_2 - z_3}\right)$ is real. We assume that $z_4$ is origin.

  Given, $\frac{2}{z_1} = \frac{1}{z_2} + \frac{1}{z_3} = \frac{z_2 + z_3}{z_2z_3} \therefore z_1 =
  \frac{2z_2z_3}{z_1+z_3}$.

  Putting the value of $z_1$ and $z_4$ in the concyclic condition expression we obtain

  $\left(\frac{z_2 - z_4}{z_1 - z_4}\right)\left(\frac{z_1 - z_3}{z_2 - z_3}\right) = \frac{1}{2}$.
  Thus, $z_1, z_2, z_3$ lie on a circle passing through origin.
\item The diagram given below:
  \startplacefigure[location={left,none}]
    \startMPcode
      % \draw (0, 0) circle(1);
      % \draw (.866, -.5) -- (-.866, -.5) -- (.5, .866) -- cycle;
      % \draw (.5, .866) -- (.5, -.866);
      % \filldraw (0, 0) circle(1pt);
      % \draw (0, 0) node[anchor=north] {$O$};
      % \draw (.866, -.5) node[anchor=north west] {$C(z_3)$} (-.866, -.5)
      % node[anchor=north east] {$B(z_2)$} (.5, .866) node[anchor=south]
      % {$A(z_1)$} (.5, -.866) node[anchor=north] {$P(z)$};
      pickup pencircle scaled 0.2pt;
      draw fullcircle scaled 2cm;
      draw (.866cm, -.5cm) -- (-.866cm, -.5cm) -- (.5cm, .866cm) -- cycle;
      draw (.5cm, .866cm) -- (.5cm, -.866cm);
      pickup pencircle scaled 2pt;
      drawdot (0, 0);
      label.bot("$O$", (0, 0));
      label.lrt("$C(z_3)$", (.866cm, -.5cm));
      label.llft("$B(z_2)$", (-.866cm, -.5cm));
      label.top("$A(z_1)$", (.5cm, .866cm));
      label.bot("$P(z)$", (.5cm, -.866cm));
    \stopMPcode
  \stopplacefigure
  We have $OP=OA=OB=OC \therefore |z| = |z_1| = |z_2| = |z_3| \Rightarrow |z|^2 = |z_1|^2 = |z_2|^2 =
  |z_3|^2 \Rightarrow z\overline{z} = z_1\overline{z_1} = z\overline{z_2} = z\overline{z_3}$.

  Since $AP$ is perpendicular to $BC, \therefore\arg\left(\frac{z_1 - z}{z_2 - z_3}\right) = \frac{\pi}{2}$
  or $\frac{-\pi}{2}\Rightarrow \frac{z_1 - z}{z_2 - z_3}$ is purely imaginary.

  $\Rightarrow \overline{\left(\frac{z_1 - z}{z_2 - z_3}\right)} = -\frac{z_1 - z}{z_2 - z_3}$.
  Solving the above equation gives $z = \frac{z_2z_3}{z_1}$.
\item The diagram is given below:
  \startplacefigure[location={left,none}]
    \startMPcode
      pickup pencircle scaled 0.2pt;
      draw fullcircle scaled 2cm shifted (2cm, 0);
      draw (0cm, -0.2cm) -- (3.2cm, -0.2cm);
      draw (0cm, -0.2cm) -- (3.2cm, 0.8cm);
      draw fullcircle scaled 2cm shifted (5cm, 0);
      draw (3.8cm, -.6cm) -- (6.2cm, .6cm);
      draw (3.8cm, .8cm) -- (6.2cm, -.8cm);
      label("$A$", (0.9cm, -0.4cm));
      label("$B$", (3.1cm, -0.4cm));
      label("$P$", (-0.2cm, -0.2cm));
      label("$C$", (0.9cm, 0.3cm));
      label("$D$", (2.9cm, 0.85cm));
      label("$A$", (4cm, -0.6cm));
      label("$P$", (4.95cm, -0.2cm));
      label("$B$", (5.95cm, 0.65cm));
      label("$C$", (4.15cm, 0.75cm));
      label("$D$", (5.85cm, -0.75cm));
    \stopMPcode
  \stopplacefigure
  Let $P(z)$ be the point of intersection and $A, B, C, D$ represent points $a, b, c, d$ respectively. Clearly, $P, A, B$ are
  collinear. Thus,

  $\startbmatrix\NC z \NC \overline{z} \NC 1\NR\NC a \NC \overline{a} \NC 1\NR\NC b \NC \overline{b} \NC
  1\NR\stopbmatrix = 0 \Rightarrow z(\overline{a} - \overline{b}) - \overline{z}(a - b) + (a\overline{b} -
  \overline{a}b) = 0$

  Similarly, $P, C, D$ are collinear and thus
  $\Rightarrow z(\overline{c} - \overline{d}) - \overline{z}(c - d) + (c\overline{d} - \overline{c}d) = 0$

  Eliminating $\overline{z}$ because we have to find $z$, we have
  $z(\overline{a} - \overline{b})(c - d) - z(\overline{c} - \overline{d})(a - b) = (c\overline{d} -
  \overline{c}d)(a - b) - (a\overline{b} - \overline{a}b)(c - d)$.

  $\because a, b, c, d$ lie on the circle. $|a| = |b| = |c| = |d| = r \Rightarrow a^2 = b^2 = c^2 = d^2 = r^2$
  $\Rightarrow a\overline{a} = b\overline{b} = c\overline{c} = d\overline{d} = r^2$

  $\Rightarrow \overline{a} = \frac{r^2}{a}, \overline{b} = \frac{r^2}{b}, \overline{c} = \frac{r^2}{c},
  \overline{d} = \frac{r^2}{d}$

  Putting these values in the equation we had obtained,
  $z\left(\frac{r^2}{a} - \frac{r^2}{b}\right)(c - d) - z\left(\frac{r^2}{c} - \frac{r^2}{d}\right)(a - b) = \left(\frac{cr^2}{d} -
  \frac{dr^2}{c}\right)(a - b) - \left(\frac{ar^2}{b} - \frac{br^2}{a}\right)(c - d)$

  Solving this for $z$, we arrive at desired answer.
\item Given $\startbmatrix\NC a \NC b \NC c\NR\NC b \NC c \NC a\NR\NC c \NC a \NC b\NR\stopbmatrix = 0$
    $\Rightarrow a^3 + b^3 + c^3 - 3abc = 0\Rightarrow (a + b + c)(a^2 + b^2 + c^2 - ab - bc - ca) = 0$

    $\because z_1, z_2, z_3$ are three non-zero complex numbers, hence $a^2 + b^2 + c^2 - ab - bc - ca = 0$
    $\Rightarrow (a - b)^2 + (b - c)^2 + (c - a)^2 = 0 \Rightarrow a = b = c$. This can be represented by
  following diagram:
  \startplacefigure[location={left,none}]
    \startMPcode
      % pair o = (0,0);
      % path circle = circle(o, 2);
      % draw(circle);
      % pair a = (0, 2);
      % pair b = (1.414, -1.414);
      % pair c = (-1.414, -1.414);
      % draw (a -- b -- c -- cycle);
      % draw (b -- o);
      % draw (c -- o);
      % label("$A$", a, align=N);
      % label("$B$", b, align=E);
      % label("$C$", c, align=W);
      % label("$O$", o, align=N);
      % markangle("", radius=10, c, o, b);
      pair o;
      o = (0, 0);
      draw fullcircle scaled 4cm;
      pair a; pair b; pair c;
      a = (0, 2cm); b = (1.414cm, -1.414cm);
      c = (-1.414cm, -1.414cm);
      draw a -- b -- c -- cycle;
      draw b -- o;
      draw c -- o;
      label.top("$A$", a);
      label.rt("$B$", b);
      label.lft("$C$", c);
      label.top("$O$", o);
      draw fullcircle scaled 16 rotated angle (c - o) shifted o cutafter (o -- b);
    \stopMPcode
  \stopplacefigure
  Now $OA=OB=OC$, where $O$ is the origin and $A, B$ and $C$ are the points representing $z_1, z_2$ and $z_3$ respectively.
  $\therefore O$ is the circumcenter of $\triangle ABC$.

  Now $\arg\left(\frac{z_3}{z_2}\right) = \angle BOC = 2\angle BAC = \arg\left(\frac{z_3 - z_1}{z_2 -
    z_1}\right)^2$.
  \vskip 2cm
\item The diagram is given below:
  \startplacefigure[location={left, none}]
    \startMPcode
      pair oo;
      oo = (2cm, 0cm);
      draw fullcircle scaled 4cm shifted oo;
      pair o; pair p; pair q; pair r;
      o = (0, 0); p = (4cm, 0); q = (3cm, 1.732cm); r = (1cm, 1.732cm);
      draw o -- p; draw o -- q; draw o -- r; draw p -- q; draw p -- r;
      draw unitsquare scaled 5 rotated angle (o - q) shifted q;
      draw unitsquare scaled 5 rotated angle (o - r) shifted r;
      draw fullcircle scaled 16 rotated angle (p - o) shifted o cutafter (o -- q);
      draw fullcircle scaled 16 rotated angle (q - o) shifted o cutafter (o -- r);
      label.lft("$O$", o);
      label.rt("$P(z_1)$", p);
      label.top("$Q(z_2)$", q);
      label.top("$R(z_3)$", r);
    \stopMPcode
  \stopplacefigure
  $z_2 = \frac{OQ}{OP}z_1e^{i\theta} = \cos\theta z_1e^{i\theta}$ and $z_3 = \frac{OR}{OP}z_1e^{i2\theta} =
  \cos2\theta z_1e^{i2\theta}$

  $\Rightarrow z_2^2 = \cos^2\theta z_1^2e^{i2\theta} \Rightarrow z_2^2\cos2\theta = z_1z_3\cos^2\theta$.
  \vskip 2.5cm
\item Given circles are $|z| = 1 \Rightarrow x^2 + y^2 - 1 = 0$ and $|z - 1| = 4 \Rightarrow x^2 - 2x + y^2
  - 15 = 0$.

  Let the circles cut by these two orthogonally is $x^2 + y^2 + 2gx + 2fy + c = 0$. Since first circle cuts
  this family of circles orthoginally, therefore

  $2g.0 + 2f.0 = c - 1 \Rightarrow c = 1$ and $2g(-1) + 2f.0 = c - 15 \Rightarrow g = 7$.
  Thus, required circles are $x^2 + y^2 + 14x + 2fy + 1 = 0 \Rightarrow |z + 7 + if| = \sqrt{48 + f^2}$.
\item Given, $|z + 3| = t^2 - 2t + 6$ which is equation of a circle having center $(-3, 0)$ and radius $t^2 - 2t +
  6$. Let $A = (-3, 0)$ and $r_1 = t^2 - 2t + 6$. In this case $z$ lies on the circle.

  Also, $|z - 3\sqrt{3}i| < t^2$ implies $z$ lies on the interior of the circle having center $(0,
  3\sqrt{3})$ and radius $t^2$. Let $B = (0, 3\sqrt{3})$ and $r_2 = t^2$. $AB = \sqrt{3^2 + 27} = 6$. $r_2 -
  r_1 = 2(t - 3)$

  Clearly, when the two circles are disjoint or touching each other no solution is possible. This leads to
  following cases:

  {\bf Case I:} When $t > 3$ i.e. $r2 > r_1$. In this case at least one $z$ is possible if $AB < r_1 + r_2
  \Rightarrow 6 < 2(t^2 - t + 3)\Rightarrow t < 0$ or $t > 1\Rightarrow 3 < t <\infty$

  {\bf Case II:} When $t \leq 3$ i.e. $r_1 > r_2$. In this case at least one $z$ will be possible if $|r_1 -
  r_2| \leq AB < r_1 + r_2$

  $2(3 - t)\leq 6 < 2(t^2 - t + 3)$ i.e. $t \leq 0$ and $t < 0$ or $t > 1$
  Combining all solutions we gace $1 < t < \infty$.
\item Let $z = x + iy$. $\frac{az + b}{cz + d} = \frac{ax + b + iay}{cx + d + icy} = \frac{(ax + b + iay)(cx
  + d - icy)}{(cx + d)^2 + c^2y^2}$

  $\Im\left(\frac{az + b}{cz + d}\right) = \frac{ay(cx + d) - cy(ax + b)}{(cx + d)^2 + c^2y^2} = \frac{ady -
  bcy}{(cx + d)^2 + c^2y^2}$

  $\because ad > bc$, therefore the signs of imaginary parts of $z$ and $\frac{az + b}{cz + d}$ are the same.
\item Given, $z_1 = \frac{i(z_2 + 1)}{z_2 - 1} \Rightarrow x_1 + iy_1 = \frac{-y_2 + i(x_2 + 1)}{(x_2 - 1) + iy_2}
  = \frac{[-y_2 + i(x_2 + 1)][(x_2 - 1) + iy_2]}{(x_2 - 1)^2 + y_2^2}$

  Comparing real and imaginary parts, we have

  $x_1 = \frac{-y_2(x_2 - 1) -(x_2 + 1)y_2}{(x_2 - 1)^2 + y_2^2} = \frac{-2x_2y_2}{(x_2 - 1)^2 + y_2^2}$ and $y_1 = \frac{x_2^2 - 1
    - y_2^2}{(x_2 - 1)^2 + y_2^2}$

  Substituting for $x_1$ and $y_1$ in $x_1^2 + y_1^2 - x_1$ we will arrive at the desired result.
\item $(\cos3\theta - i\sin3\theta)^6 = (e^{-i3\theta})^6 = e^{-i18\theta}$ and $(\cos2\theta +
  i\sin2\theta)^5 = (e^i2\theta)^5 = e^{i10\theta}$

  $(\sin\theta - i\cos\theta)^3 = [(-i)^3(\cos\theta + i\sin\theta)^3] = i.e^{i3\theta}$ and
  $\frac{(\cos3\theta - i\sin3\theta)^6(\sin\theta - i\cos\theta)^3}{(\cos2\theta + i\sin2\theta)^5} =
  i.e^{-i25\theta} = \sin25\theta + i\cos25\theta$.
\item Let $z = x + iy$, then we have $x^2 - y^2 + 2ixy + \sqrt{x^2 + y^2} = 0$

  Equating imaginary parts, we have $2xy = 0$ i.e. either $x = 0$ or $y = 0$.

  If $x = 0$, then $-y^2 + \sqrt{y^2} = 0 \Rightarrow y^4 - y^2 = 0 \Rightarrow y = 0, y = \pm 1$.

  If $y = 0$, then $x^2 + \sqrt{x^2} = 0$ Since $x$ is real only one solution is possible i.e. $x = 0$.
  Hence, $z = 0, \pm i$.
\item Clearly $z = 0$ is one of the solutions. For other solutions divide both sides by $|z|^2$ which gives
  us $t^2 + t + 1 = 0$ where $t = \frac{z}{|z|}$.

  The equation $t^2 + t + 1 = 0$ has two roots i.e. $t = \omega, \omega^2 \Rightarrow \frac{z}{|z|} =
  \omega, \omega^2\Rightarrow z = k\omega, k\omega^2$ where $k = |z|$ is a non-negative real number.
\item Let $z = x + iy$, then $(x + iy)\sqrt{x^2 + y^2} + a(x + iy) + 1 = 0$.
  Comparing real and imaginary parts, we get

  $y\sqrt{x^2 + y^2} + ay = 0 \Rightarrow y = 0\;\because \sqrt{x^2 + y^2} + a \neq 0\;[\because a > 0]$ and
  $\therefore x\sqrt{x^2 + 0} + ax + 1 = 0 \Rightarrow x^2 + ax + 1 = 0 \Rightarrow x = \frac{-a
    \pm\sqrt{a^2 - 4}}{2}$

  Clearly, both the values of $x$ are negative, so $z$ is a negative real number.
\item Let $z = x + iy$, then $x^2 + y^2 - 2i(x + iy) + 2a(1 + i) = 0$.
  Comparing real and imaginary parts, we get

  $x^2 + y^2 + 2y + 2a = 0 \Rightarrow x^2 + (y - 1)^2 = 1 - 2a$ and $-2x + 2a = 0 \Rightarrow x = a$

  $\Rightarrow (y - 1)^2 = 1 - 2a - a^2 \Rightarrow y = 1 \pm \sqrt{1 - 2a - a^2}$.
  However $1 - 2a - a^2 > 0$. Roots of equivalent quadratic equation is $a = \frac{2 \pm \sqrt{8}}{-2}
  \Rightarrow -1\pm \sqrt{2}$ but $a > 0$ so the range for $a$ is $0 < a < \sqrt{2} - 1$.
\item i. We have $(3 + 4i)^x = 5^{\frac{x}{2}}$. Squaring both sides $(-7 + 24i)^x = 5^x \Rightarrow
  \left(\frac{-7 + 24i}{5}\right)^x = 1$ which is possible only if $x = 0$.

  ii. Given $(1 - i)^x = 2^x \Rightarrow \left(\frac{1 - i}{2}\right)^x  = 1$ which is possible only if $x =
  0$.

  iii. Given $(1 - i)^x = (1 + i)^x \Rightarrow \left(\frac{1 - i}{1 + i}\right)^x = 1 \Rightarrow (-i)^x =
  1 \Rightarrow x = 0, 4, 8, \ldots, 4n\;\forall\;4n\in I$.
\item $z^3 + 2z^2 + 2z + 1 = 0\Rightarrow (z + 1)(z^2 + z + 1) = 0 \Rightarrow z = -1, \omega, \omega^2$.

  When $z = -1, z^{1985} + z^{100} + 1 = -1 + 1 + 1 = 1 \neq 0$, when $z = \omega, \omega^{1985} +
  \omega^{100} + 1 = \omega^2 + \omega + 1 = 0$ and when $z = \omega^2, \omega^{1985*2} + \omega^{200} + 1 =
  \omega + \omega^2 + 1 = 0$. Thus common roots are $\omega, \omega^2$.
\item Adding all equations $\alpha + \beta + \gamma = 3z_1 \Rightarrow z_1 = \frac{\alpha + \beta +
  \gamma}{3}$. Similarly, multiplying second equatin with $\omega$ and third equation with $\omega^2$, and
  then adding we have $z_3 = \frac{\alpha + \beta\omega + \gamma\omega^2}{3}$. Similarly, $z_2 =
  \frac{\alpha + \beta\omega^2 + \gamma\omega}{3}$.

  $|\alpha|^2 = \alpha\overline{\alpha} = (z_1 + z_2 + z_3)(\overline{z_1} + \overline{z_2} +
  \overline{z_3}), |\beta|^2 = \beta\overline{\beta} = (z_1 + z_2\omega + z_3\omega^2)(\overline{z_1} +
  \overline{z_2}\omega^2 + \overline{z_3}\omega)$ and $\|\gamma|^2 = \gamma\overline{\gamma} = (z_1 +
  z_2\omega^2 + z_3\omega)(\overline{z_1} + \overline{z_2}\omega + \overline{z_3}\omega^2)\;[\because
    \overline{\omega} = \omega^2 \& \overline{\omega^2} = \omega]$

  $\Rightarrow |\alpha|^2 + |\beta|^2 + |\gamma|^2 = 3(|z_1|^2 + |z_2|^2 + |z_3|^2) + z_1[\overline{z_2}(1 +
    \omega + \omega^2) + \overline{z_3}(1 + \omega + \omega^2)] + z_2[\overline{z_1}(1 + \omega + \omega^2)
    + \overline{z_2}(1 + \omega + \omega^2)] + z_3[\overline{z_1}(1 + \omega + \omega^2) + \overline{z_2}(1
    + \omega + \omega^2)] = 3(|z_1|^2 + |z_2|^2 + |z_3|^2) =$ R.H.S.
\item Let $f(x) = (x + 1)^n - x^n - 1. x^3 + x^2 + x = 0 \Rightarrow x(x^2 + x + 1) = 0\Rightarrow x = 0,
  \omega, \omega^2$. So for $x^3 + x^2 + x$ to be a factor of $f(x), f(0) = 0, f(\omega) = 0, f(\omega^2) =
  0$.

  $f(0) = 1^n - 1 = 0, f(\omega) = (\omega + 1)^n - \omega^n - 1 = -\omega^{2n} - \omega^n - 1\;[\because n$
    is odd. $] = -(1 + \omega^n + \omega^{2n}) = 0$. Similarly, $f(\omega^2) = 0$. Hence proved.
\item Let $f(x, y) = (x + y)^n - x^n - y^n. xy(x + y)(x^2 + xy + y^2) = 0 \Rightarrow x = 0, y = 0, x = -y, y =
  x\omega, y = x\omega^2$. When $x = 0, f(x, y) = 0; y = 0, f(x, y) = 0; y = -x \Rightarrow f(x, y) = -x^n -(-x)^n =
  0 [\because n = 2m + 1\;\forall\;m\in\mathbb{I}], y = xw \Rightarrow f(x, y) = [x^n(1 + \omega)^n - x^n -
    x^n\omega^n] = -x^n\omega^{2n} - x^n - x^n\omega^n = 0$, and similarly when $y = x\omega^2, f(x, y) =
  0$. Hence proved.
\item R.H.S. $= \left|\frac{1}{z_1} + \frac{1}{z_2} + \cdots + \frac{1}{z_n}\right| =
  \left|\frac{\overline{z_1}}{|z_1|^2} + \frac{\overline{z_2}}{|z_2|^2} + \cdots +
  \frac{\overline{z_n}}{|z_n|^2}\right|$

  $= |\overline{z_1} + \overline{z_2} + \cdots + \overline{z_n}| = |\overline{z_1 + z_2 + \cdots + z_n}| =
  |z_1 + z_2 + \cdots + z_n| =$ L.H.S.
\item For any two complex numbers $z_1$ and $z_2$, we know that $|z_1 + z_2|^2 + |z_1 - z_2|^2 = 2|z_1|^2 +
  2|z_2|^2$. Let $z_1 = \alpha + \sqrt{\alpha^2 - \beta^2}$ and $z_2 = \alpha - \sqrt{\alpha^2 - \beta^2}$.

  Now $(|z_1| + |z_2|)^2 = |z_1|^2 + |z_2|^2 + 2|z_1||z_2| = 2|\alpha|^2 + 2|\alpha^2 - \beta^2| +
  2|\beta|^2 = |\alpha + \beta|^2 + |\alpha - \beta|^2 + 2|\alpha + \beta||\alpha -\beta|$

  $= (|\alpha + \beta| + |\alpha - \beta|)^2\Rightarrow |z_1| + |z_2| = |\alpha + \beta| + |\alpha - \beta|
  =$ R.H.S.
\item $|z_1| = |z_1| = 1 \Rightarrow a^2 + b^2 = c^2 + d^2 = 1, z_1\overline{z_2} = ac + bd + i(bc -
  ad)\;\because\;\Re(z_1\overline{z_2}) = 0 \Rightarrow ac + bd = 0 \Rightarrow \frac{a}{d} = -\frac{b}{c} =
  k$ (say). $\therefore a = kd, b = -kc$.

  $\therefore k^2d^2 + k^2c^2 = 1 \Rightarrow k^2 = 1 \Rightarrow k = \pm 1$. Now $|\omega_1| = \sqrt{a^2 +
    c^2} = \sqrt{a^2 + b^2} = 1, |\omega_2| = \sqrt{b^2 + d^2} = \sqrt{a^2 + b^2} = 1,
  \omega_1\overline{\omega_2} = (a + ic)(b - id) \therefore \Re(\omega\overline{\omega_2}) = ab + cd = 0$.
\item Given, $\left|\frac{z_1 - z_2}{1 - \overline{z_1}z_2}\right|< 1\Leftrightarrow \left|\frac{z_1 -
  z_2}{1 - \overline{z_1}z_2}\right|^2 < 1 \Leftrightarrow |z_1 - z_2|^2 < |1 - \overline{z_1}z_2|^2$

  $\Leftrightarrow (z_1 - z_2)\overline{(z_1 - z_2)} < (1 - \overline{z_1}z_2)\overline{(1 -
  \overline{z_1}z_2)} \Leftrightarrow (z_1 - z_2)(\overline{z_1} - \overline{z_2}) < (1 -
  \overline{z_1}z_2)((1 - z_1\overline{z_2}))$

  $\Leftrightarrow |z_1|^2 + |z_2|^2 > 1 + |z_1|^2|z_2|^2 \Leftrightarrow 1 - |z_1|^2 - |z_2|^2 +
  |z_1|^2|z_2|^2 > 0 \Leftrightarrow (1 - |z_1|^2)(1 - |z_2|^2) > 0 \Rightarrow (1 + |z_1|)(1 - |z_1|)(1 +
  |z_2|)(1 - |z_2|) > 0$

  $\Leftrightarrow (1 - |z_1|)(1 - |z_2|) > 0$ which is true as $|z_1| < 1$ and $|z_2| < 1$.
\item Let $z = x + iy$ then $\frac{z - z_1}{z - z_2} = \frac{(x - 10) + i(y - 6)}{(x - 4) + i(y -
  6)}$. Rationalizing $\frac{x^2 - 14x + 40 + (y - 6)^2}{(x - 4)^2 + (y - 6)^2} + \frac{i6(y - 6)}{(x - 4)^2
  + (y - 6)^2} = a + ib$ (say)

  $\because \arg(a + ib) = \frac{\pi}{4}\Rightarrow x^2 - 14x + 40 + (y - 6)^2 = 6(y - 6)\Rightarrow x^2 +
  y^2 - 14x - 18y + 112 = 0 \Rightarrow |z - 7 - 9i|^2 = 18$. Hence proved.
\item Let $z = x + iy$ then $\frac{3z - 6 - 3i}{2z - 8 - 6i} = \frac{x - 6 + i(3y - 3)}{2x - 8 + i(2y -
  6)}$. Rationalizing $\frac{6x^2 + 6y^2 - 36x - 24y + 66 + i(12x - 12y - 12)}{(2x - 8)^2 + (2y - 6)^2} = a
  + ib$ (let)

  $\because\arg(a + ib) = \frac{\pi}{4} \Rightarrow 6x^2 + 6y^2 - 36x - 24y + 66 = 12x - 12y - 12
  \Rightarrow x^2 + y^2 - 8x - 2y + 13 = 0$. Also given, $|z - 3 + i| = 3 \Rightarrow x = -2y +
  6$. Substituting this in previously obtained equation, we have

  $5y^2 - 10y + 1 = 0 \Rightarrow y = 1 \pm\frac{2}{\sqrt{5}}\Rightarrow x = 4\mp\frac{4}{\sqrt{5}}$. Hence
  we have our $z$.
\item Let $|z| = r_1, |w| = r_2, \arg(z)= \theta_1$ and $\arg(w) = \theta_2$. Then, $|z - w|^2 =
  (r_1\cos\theta_1 - r_1\sin\theta_1)^2 + (r_2\cos\theta_2 - r_2\sin\theta_2)^2 = (r_1 - r_2)^2 + 2r_1r_2 -
  2r_1r_2\cos(\theta_1 - \theta_2)$

  $= (r_1 - r_2)^2 + 4r_1r_2\sin^2\frac{\theta_1 - \theta_2}{2}\leq (r_1 - r_2)^2 +
  2.1.1.2\left(\frac{\theta_1 - \theta_2}{2}\right)^2 = (|z| - |w|)^2 + (\theta_1 - \theta_2)^2$. Hence
  proved.
\item Let $z = r(\cos\theta + i\sin\theta) \Rightarrow \frac{z}{|z|}= \cos\theta + i\sin\theta\;\therefore
  \left|\frac{z}{|z|} - 1\right| = |(\cos\theta - 1) + i\sin\theta| = \sqrt{(\cos\theta - 1)^2 +
    \sin^2\theta} = \sqrt{4\sin^2\frac{\theta}{2}} = 2|\sin\frac{\theta}{2}|\leq |\theta|$.

  Now, $|z - |z|| = |z - 1 - (|z| - 1)|\geq |z - 1| - ||z| - 1|\;\therefore |z - 1| - ||z| - 1|\leq |z -
  |z||$

  $\Rightarrow |z - |z|| = |r(\cos\theta + i\sin\theta) - r| = \sqrt{4r^2\sin^2\frac{\theta}{2}}\leq
  2r\left|\frac{\theta}{2}\right| = r|\theta| = |z||\arg(z)|$

  $\Rightarrow |z - 1| - ||z| - 1|\leq |z||\arg(z)|\Rightarrow |z - 1|\leq ||z| - 1| + |z||\arg(z)|$.
\item Let $z = r(\cos\theta + i\sin\theta)$ then $\frac{1}{z} = \frac{1}{r}(\cos\theta -
  i\sin\theta)$. Given $\left|z + \frac{1}{z}\right| = a \Rightarrow \left|\left(r +
  \frac{1}{r}\right)\cos\theta + i\left(r - \frac{1}{r}\right)\sin\theta\right| = a$

  $\Rightarrow \left(r + \frac{1}{r}\right)^2\cos^2\theta + \left(r - \frac{1}{r}\right)^2\sin^2\theta =
  a^2 \Rightarrow \left(r - \frac{1}{r}\right) = a^2 - 4\cos^2\theta$. Clearly, $r$ will be greatest if
  $\cos\theta = 0 \Rightarrow r^2 - ar - 1 = 0 \Rightarrow r = \frac{a \pm\sqrt{a^2 + 4}}{2}$. This also
  implies that $z$ is a purely imaginary number.
\item $|z_1 + z_2|^2 < |z_1|^2 + c|z_1|^2 + |z_2|^2 + \frac{1}{c}|z_2|^2 \Rightarrow (z_1 +
  z_2)(\overline{z_1} + \overline{z_2}) < |z_1|^2 + |z_2|^2 + \frac{c^2|z_1|^2 + |z_2|^2}{c} \Rightarrow
  z_2\overline{z_1} + z_1\overline{z_2} < \frac{1}{c}(c^2|z_1|^2 + |z_2|^2)$

  $\Rightarrow (x_2 + iy_2)(x_1 - iy_1) + (x_1 + iy_1)(x_2 - iy_2) < \frac{1}{c}[c^2(x_1^2 + y_1^2) + x_2^2
    + y_2^2] \Rightarrow (cx_1 - x_2)^2 + (cy_1 - y_2)^2 > 0$ which is true.
\item Given, $\left|\frac{z_1 - z_2}{z_1 + z_2}\right| = 1\Rightarrow |z_1 - z_2|^2 = |z_1 + z_2|^2
  \Rightarrow (z_1 - z_2)(\overline{z_1} - \overline{z_2}) = (z_1 + z_2)(\overline{z_1} + \overline{z_2})$

  $\Rightarrow -z_2\overline{z_1} - z_1\overline{z_2} = z_2\overline{z_1} + z_1\overline{z_2} \Rightarrow
  z_1\overline{z_2} = -2z_2\overline{z_1} \Rightarrow \overline{\left(\frac{z_1}{z_2}\right)} =
  -\frac{z_1}{z_2}$

  $\Rightarrow \frac{z_1}{z_2}$ is purely imaginary $\Rightarrow \frac{iz_1}{z_2}$ is real, which we take as
  $x$.

  $\frac{z_1 + z_2}{z_1 - z_2} = \frac{z_1/z_2 + 1}{z_1/z_2 - 1} = \frac{-ix + 1}{-ix - 1} = \frac{-1 + x^2
    + 2ix}{1 + x^2}$

  If $\theta$ is the angle between the lines joining the origin to the points $z_1 + z_2$ and $z_1 - z_2$,
  then $\tan\theta = \left|\arg\left(\frac{z_1 + z_2}{z_1 - z_2}\right)\right| = \left|\frac{2x}{x^2 -
    1}\right|$.
\item Let $z_1 = r_1(\cos\theta_1 + i\sin\theta_1), z_2= r_2(\cos\theta_2 + i\sin\theta_2)$. Let $\sqrt{a^2
  + b^2} = r$. Let $a = r\cos\alpha, b = r\cos\alpha$. Now $|az_1 + bz_2|^2 = |rr_1(\cos\theta_1 +
  i\sin\theta_1)\cos\alpha + rr_2(\cos\theta_2 + i\sin\theta_2)\sin\alpha|^2$

  $= r^2[r_1^2\cos^2\alpha + r_2^2\sin^2\alpha + 2r_1r_2\cos\alpha\sin\alpha\cos(\theta_1 - \theta_2)] =
  \frac{r^2}{2}[r_1^2 + r_2^2 + (r_1^2 - r_2^2)\cos2\alpha + 2r_2r_2\cos(\theta_1 - \theta_2)\sin2\alpha]$

  Thus, $|az_1 + bz_2|^2 = \frac{r^2}{2}[A + B\cos2\alpha + C\sin2\alpha] \Rightarrow  \frac{2|az_1 +
    bz_2|^2}{r^2}[A + B\cos2\alpha + C\sin2\alpha]$, where $A = r_1^2 + r_2^2, B = r_1^2 - r_2^2$ and $C =
  2r_1r_2\cos(\theta_1 - \theta_2)$.

  Since $A - \sqrt{B^2 + C^2} \leq A + B\cos2\alpha +
  C\sin2\alpha \leq A + \sqrt{B^2 + C^2}$

  $B^2 + C^2 = r_1^4 + r_2^4 - 2r_1^2r_2^2 + 4r_1^2r_2^2\cos^2(\theta_1 - \theta_2)$.

  $|z_1^2 + z_2^2| = |r_1^2(\cos2\theta_1 + i\sin2\theta_1) + r_2^2(\cos2\theta_2 + i\sin2\theta_2)| =
  \sqrt{B^2 + C^2}$. Hence proved.
\item Given $z = \frac{b + ic}{1 + a} \Rightarrow iz = \frac{-c + ib}{1 + a} \Rightarrow \frac{1 + iz}{1 -
  iz} = \frac{1 + a - c + ib}{1 + a + c - ib}$

  Given, $a^2 + b^2 + c^2 = 1 \Rightarrow (a + ib)(a - ib) = (1 + c)(1 - c) \Rightarrow \frac{1 + iz}{1 -
    iz} = \frac{a + ib}{1 + c}$.
\item Let $z_1 = x_1 + iy_1$ and $z_2 = x_2 + iy_2$. L.H.S. $= |az_1 - bz_2|^2 + |bz_1 - az_2|^2 = (ax_1 -
  bx_2)^2 + (ay_1 - by_2)^2 + (bx_1 - ax_2)^2 + (by_1 - by_2)^2$

  $= (a^2 + b^2)(x_1^2 + y_1^2) + (a^2 + b^2)(x_2^2 + y_2^2) = (a^2 + b^2)(|z_1|^2 + |z_2|^2) =$ R.H.S.
\item Let $\alpha = x_1 + iy_1$ and $\beta = x_2 + iy_2$. Then $|\alpha + \beta|^2 = (x_1 + x_2)^2 + (y_1 +
  y_2)^2 = x_1^2 + x_2^2 + y_1^2 + y_2^2 + 2x_1x_2 + 2y_1y_2$.

  $|\alpha|^2 = x_1^2 + y_1^2, |\beta|^2 = x_2^2 + y_2^2, \Re(\alpha\overline{\beta}) = x_1x_2 + y_1y_2$ and
  $\Re(\overline{\alpha}\beta) = x_1x_2 + y_1y_2$. Now it is trivial to prove the equality.
\item $|1 - \overline{z_1}z_2|^2 - |z_1 - z_2|^2 = (1 - \overline{z_1}z_2)(1 - z_1\overline{z_2}) - (z_1 -
  z_2)(\overline{z_1} - \overline{z_2}) = (1 - \overline{z_1}z_2 - z_1\overline{z_2} + |z_1|^2|z_2|^2) -
  (|z_1|^2 - \overline{z_1}z_2 - z_1\overline{z_2} + |z_2|^2) = 1 - |z_1|^2 - |z_2|^2 + |z_1|^2|z_2|^2 = (1
  - |z_1|^2)(1 - |z_2|^2) =$ R.H.S.
\item Consider two complex numbers $z_1 = a_1 + ib_1$ and $z_2 = a_2 + ib_2.$ Now we have to prove $|z_1 +
  z_2| \le |z_1| + |z_2|$ which can be further extended to prove the result.

  $\Rightarrow \sqrt{(a_1 + a_2)^2 + (b_2 + b_2)^2} \le \sqrt{a_1^2 + b_1^2} + \sqrt{a_2^2 +
    b_2^2}$.

  Squaring both sides and simplifying

  $\Rightarrow a_1a_2 + b_1b_2 \le \sqrt{(a_1^2 + b_1^2)(a_2^2 + b_2^2)} \Rightarrow (a_1a_2 + b_1b_2)^2 -
  (a_1^2 + b_1^2)(a_2^2 + b_2^2) \le 0\Rightarrow -(a_1b_2 - a_2b_1)^2 \le 0$.
\item Given, $\left|\frac{\overline{z_1} - 2\overline{z_2}}{2 - z_1\overline{z_2}}\right| = 1\Rightarrow
  |\overline{z_1} - 2\overline{z_2}|^2 = |2 - z_1\overline{z_2}|^2$

  $\Rightarrow (\overline{z_1} - 2\overline{z_2})(z_1 - 2z_2) = (2 - z_1\overline{z_2})(2 -
  \overline{z_1}z_2) \Rightarrow |z_1|^2 - 2z_1\overline{z_2} - 2\overline{z_1}z_2 + 4|z_2|^2 = 4 -
  2z_1\overline{z_2} - 2\overline{z_1}z_2+ |z_1|^2|z_2|^2$

  $\Rightarrow |z_1|^2|z_2|^2 - 4|z_2|^2 - |z_1|^2 - 4 = 0 \Rightarrow |z_2| = 2\;\because |z_1|\neq 1$.
\item $\left|\frac{z_1 + z_2}{2} + \sqrt{z_1z_2}\right| + \left|\frac{z_1 + z_2}{2} - \sqrt{z_1z_2}\right|$

  $= \frac{1}{2}\left|(\sqrt{z_1} + \sqrt{z_2})^2\right| + \frac{1}{2}\left|(\sqrt{z_1} -
  \sqrt{z_2})^2\right| = |z_1| + |z_2|$
\item We have proven that $|a + \sqrt{a^2 - b^2}| + |a - \sqrt{a^2 - b^2}| = |a + b| + |a -
  b|$. Substituting $a = \beta$ and $b = \sqrt{\alpha\gamma}$ we have

  $|\beta + \sqrt{\alpha\gamma}| + |\beta - \sqrt{\alpha\gamma}| = |\alpha|\left(|\frac{\beta}{\alpha} +
  \sqrt{\frac{\gamma}{\alpha}}| +|\frac{\beta}{\alpha} - \sqrt{\frac{\gamma}{\alpha}}|\right)$

  $= |\alpha|\left(|-z_1 - z_2 + \sqrt{z_1z_2}| + |-z_1 - z_2 - \sqrt{z_1z_2}|\right) = |\alpha|(|z_1| +
  |z_2|)$.
\item We have $|a| = 1 \Rightarrow |a|^2 = 1 \Rightarrow a\overline{a} = 1 \Rightarrow \overline{a} =
  \frac{1}{a}$. Thus, $\frac{1}{a} + \frac{1}{b} + \frac{1}{c} = \overline{a} + \overline{b} +\overline{c} =
  0 [\because a + b + c = 0]$
\item $|z + 4|\leq 3 \Rightarrow -3 \leq z + 4\leq 4 \Rightarrow 0\leq z + 1\leq 6$.
\item We have to prove that $(|z_1| + |z_2|)\left|\frac{z_1}{|z_1|} + \frac{z_2}{|z_2|}\right| \leq 2|z_1 +
  z_2|$. Let $z_1 = r_1(\cos\theta_1 + i\sin\theta_1)$ and $z_2 = r_2(\cos\theta_2 + i\sin\theta_2)$. Then

  $(|z_1| + |z_2|)\left|\frac{z_1}{|z_1|} + \frac{z_2}{|z_2|}\right| = (r_1 + r_2)\left|(\cos\theta_1 +
  \cos\theta_2) + i(\sin\theta_1 + \sin\theta_2)\right| = (r_1 + r_2)\sqrt{2 + 2\cos(\theta_1 - \theta_2)}$

  Also, $4|z_1 + z_2|^2 = 4[(r_1\cos\theta_1 + r_2\cos\theta_2)^2 + (r_1\sin\theta_1 + r_2\sin\theta_2)^2] =
  4[r_1^2 + r_2^2 + r_1r_2\cos(\theta_1 - \theta_2)]$ and squaring L.H.S. we have $2(r_1 + r_2)^2[1 +
    \cos(\theta_1 - \theta_2)]^2$. Clearly, L.H.S. $\leq$ R.H.S.
\item Given equation is $z^2 + az + b = 0$. Let $p, q$ are two of its roots. Then we have $p + q = -a$ and
  $pq = b$. Taking modulus of both we have $|p + q| = |a|$ and $|pq| = b$. Now it is required that $|p| =
  |q| = 1.$ Therefore we have $|p + q| \le |p| + |q| = 2 \therefore |a| \le 2.$ Similarly, $|b| = |pq| =
  |p||q| = 1$. Since $p, q$ have unit modulii, we can have them as $p = cos\theta_1 + isin\theta_1$ and $q =
  cos\theta_2 + isin\theta_2$.

  $\arg(b) = \arg(pq) = \arg(\cos(\theta_1 + \theta_2) + i\sin(\theta_1 + \theta_2)) = \theta_1 + \theta_2$

  $\arg(a) = \arg(p + q) = \arg[(\cos\theta_1 + \cos\theta_2) + i(\sin\theta_1 + \sin\theta_2)] =
  \arg\left[\left(\cos^2\frac{\theta_1}{2} + i^2\sin\frac{\theta_1}{2} +
    2i\sin\frac{\theta_1}{2}\cos\frac{\theta_1}{2}\right) + \left(\cos^2\frac{\theta_2}{2} +
    i^2\sin\frac{\theta_2}{2} +2i\sin\frac{\theta_2}{2}\cos\frac{\theta_2}{2}\right)\right]$

  $=\arg\left[\cos\frac{\theta_1 + \theta_2}{2} + i\sin\frac{\theta_1 + \theta_2}{2}\right] = \frac{\theta_1
    + \theta_2}{2}$ and hence $\arg(b) = 2\arg(a)$.
\item Let $z = x + iy$. First we consider first two inequalities $|z| \le |\Re(z)| + |\Im(z)| \Rightarrow
  \sqrt{x^2 + y^2} \le x + y$. Squaring, we have $x^2 + y^2 \le x^2 + y^2 + 2xy \Rightarrow 2xy \ge 0$,

  which is true. Now we consider last two inequalities, $|\Re(z)| + |\Im(z)| \le \sqrt{2}|z|\Rightarrow x + y
  \le \sqrt{2(x^2 + y^2)}$. Squaring, we have $x^2 + y^2 + 2xy \le 2(x^2 + y^2) \Rightarrow (x - y)^2 \ge 0$,
  which is also true.
\item $\left|z - \frac{4}{z}\right| = 2 \Rightarrow |z| - \frac{4}{|z|} \geq 2 \Rightarrow |z|^2 - 2|z| -
  4\geq 0$. The greatest root of this equation is $\sqrt{5} + 1$. Hence proven.
\item Since $\alpha, \beta, \gamma, \delta$ are roots of the equation. $\therefore a(x - \alpha)(x - \beta)(x
  - \gamma)(x - \delta) = ax^4 + bx^3 + cx^2 + dx + e$. Substituting $x = i$, we get following

  $a(i - \alpha)(i - \beta)(i - \gamma)(i - \delta) = ai^4 + bi^3 + ci^2 + di + e \Rightarrow a(1 + i\alpha)(1
  + i\beta)(1 + i\gamma)(1 + i\delta) = a - ib - c + id + e$.

  Taking modulus and squaring we get our desired result.
\item $\because \alpha_1, \alpha_2, \ldots, \alpha_n$ are the roots of the given equation. $\therefore (x -
  \alpha_1)(x - \alpha_2)\cdots(x - \alpha_n) = x^n + a_1x^{n - 1} + a_2x^{n - 2} + \ldots + a_{n- 1}x +
  a_n= 0$.

  Substituting $x = i$, we get following $(i - \alpha_1)(i - \alpha_2)\cdots(i - \alpha_n) = i^n + a_1i^{n -
    1} + a_2i^{n - 2} + \ldots + a_{n- 1}i +a_n$.

  Taking modulus and squaring we get our desired result.
\item Let $|z_1| = |z_2| = |z_3| = R. \therefore$ Origin is the circumcenter of triangle. Since triangle is
  also equilateral circumcenter and centroid coincide. Therefore, origin is also centroid. Thus,

  $\frac{z_1 + z_2 + z_3}{3} = 0 \Rightarrow z_1 + z_2 + z_3 = 0$.
\item $z_1 + z_2 + z_3 = 0$ implies centroid of the triangle is the origin. Circumcenter is also origin as
  $Z_i$ lies on the circle $|z| = 1$. Hence, circumcenter is same as centroid making the triangle an
  equilateral triangle having circumcircle with unit radius.
\item Since the triangle is equilateral therefore the circumcenter and centroid will be same i.e. $z_0 =
  \frac{z_1 + z_2 + z_3}{3}$. Also for equilateral triangle, $z_1^2 + z_2^2 + z_3^2 = z_1z_2 + z_2z_3 +
  z_3z_1$.

  Squaring the first equation $9z_0^2 = z_1^2 + z_2^2 + z_3^2 + 2(z_1z_2 + z_2z_3 + z_3z_2) = z_1^2 + z_2^2
  + z_3^2 + 2(z_1^2 + z_2^2 + z_3^2) \Rightarrow z_1^2 + z_2^2 + z_3^2 = 3z_0^2$.
\item Since $z_1, z_2$ and origin form an equilateral triangle we have
  $z_1^2 + z_2^2 + 0^2 - z_1z_2 - z_2.0 - z_1.0 = 0$. Hence, proven.
\item From previous probelm $z_1, z_2$ and origin will form a triangle if $z_1^2 + z_2^2 - z_1z_2 = 0.$
  Therefore,  $(z_1 + z_2)^2 = 3z_1z_2 \Rightarrow a^2 = 3b$.
\item Since $z_1, z_2, z_3$ are roots of the equation $z^3 + 3\alpha z^2 + 3\beta z + \gamma = 0\Rightarrow
  z_1 + z_2 + z_3 = -3\alpha, z_1z_2 + z_2z_3 + z_3z_1 = 3\beta$ and $z_1z_2z_3 = -\gamma$.

  Centroid is given by $\frac{z_1 + z_2 + z_3}{3} = -\alpha$. Triangle will be equilateral if $z_1^2 + z_2^2
  + z_3^3 = z_1z_2 + z_2z_3 + z_3z_1 \Rightarrow (z_1 + z_2 + z_3)^2 = 3(z_1z_2 + z_2z_3 + z_3z_1)
  \Rightarrow \alpha^2 = \beta$.
\item Given $2z_2 = z_1 + z_3$. Clearly, from section formula we can deduce that $z_2$ divides line segment
  joining $z_1$ and $z_3$ in two equal segments hence the complex numbers are collinear.
\item If $z_1, z_2, z_3$ are collinear then either $z_2$ divides $z_1z_3$ internally/externally or $z_3$
  divides $z_1z_2$ internally/externally. Now we can apply the condition for collinearity
  i.e. $\startdeterminant\NC z_1\NC z_2 \NC z_3\NR\NC \overline{z_1} \NC \overline{z_2} \NC
  \overline{z_3}\NR\NC 1\NC 1\NC 1\NR\stopdeterminant = 0$ and hence we can show desired conditions.
\item $z$ represents the ring between the concentric circles whose center is at $(3, 4i)$ having radii $1$
  and $2$.
\item Let $z = x + iy \Rightarrow |z|^2 = x^2 + y^2, |z - 1|^2 = (x - 1)^2 + y^2, |z + 1|^2 = (x + 1)^2 +
  y^2$. From given inequailty $|z + 1|^2 = 16 + |z - 1|^2 - 8|z - 1|\Rightarrow 4x = 16 - 8|z -
  1|\Rightarrow 4|z - 1|^2 = (4 - x)^2 \Rightarrow 3x^2 + 4y^2 = 12$, which is an equation of an ellipse.
\item Let $z = x + iy$, then $x = t + 5 \Rightarrow x - 5 = t$ and $y = \sqrt{4 - t^2}\Rightarrow y^2 = 4 -
  t^2 \Rightarrow (x - 5)^2 + y^2 = 4$, which is a circle with center $(5, 0)$ and radius $2$.
\item Let $z = x + iy$, then $\frac{z^2}{z - 1} = \frac{(x^2 - y^2 + 2ixy)[(x - 1) - iy]}{(x - 1)^2 +
  y^2}$. Since it is real, we can equate the imaginary part to zero.

  $\Rightarrow y(y^2 - x^2) + 2x^2y - 2xy = 0 \Rightarrow y = 0$ or $x^2 + y^2 - 2x = 0 \Rightarrow (x -
  1)^2 + y^2 = 1$. However, $y\neq 0$ else $z$ won't remain a complex number. $\Rightarrow x^ + y^2 - 2x =
  (x - 1)^2 + y^2 = 1$, which represents a circle with center at $(1, 0)$ and unit radius.
\item Let $z = x + iy$, then $|z^2 - 1| = |z|^2 + 1 \Rightarrow (x^2 - y^2 - 1)^ + 4x^2y^2 = (x^2 + y^2 +
  1)^2 \Rightarrow x = 0$. Hence, locus of $z$ is a straight line specifically imginary axis.
\item Let $z = x + iy$ then $\frac{y}{x} \geq \tan\frac{\pi}{3} \Rightarrow y\geq \sqrt{3}x$. Similarly,
  $\frac{y}{x} \leq \tan\frac{3\pi}{2}=-\infty$.

  This represents the set of straight lines whose slope is greater than $\sqrt{3}$ and less than or equal
  to $-\infty$.
\item Let $z = x + iy$, then $\arg\left(\frac{z - 2}{z + 2}\right) = \frac{\pi}{3} \Rightarrow
  \arg\left(\frac{x - 2 + iy}{x + 2 + iy}\right) = \frac{\pi}{3}$

  $\Rightarrow \arg\left(\frac{x^2 + y^2 - 4 + 4iy}{(x + 2)^2 + y^2}\right) = \frac{\pi}{3}\Rightarrow
  \frac{4y}{x^2 + y^2 - 4} = \sqrt{3}$, which is equation of a circle.
\item Let $z = x + iy$. Given, $\arg\left(\frac{z - 1}{z + 1}\right) = \frac{\pi}{2} \Rightarrow
  \arg\left(\frac{(x - 1) + iy}{(x + 1) + iy}\right) = \frac{\pi}{2} \Rightarrow \frac{2y}{x^2 + y^2 - 1} =
  \infty$.

  The above equation implies $x^2 + y^2 - 1 = 0$ and $y > 0$ which is circle at $(0, 0)$ with unit circle
  above $x$-axis. The points $(-1, 0)$ and $(1, 0)$ are excluded because that will make the above equation
  indeterminate.
\item $\log_{\sqrt{3}}\frac{|z|^2 - |z| + 1}{2 + |z|} < 2\Rightarrow \frac{|z|^2 - |z| + 1}{2 + |z|} <
  (\sqrt{3})^2\Rightarrow |z|^2 - 4|z| - 5 < 0\Rightarrow |z| < 5$.
\item Clearly $A$ is $(1, 0)$ or $(-1, 0)$. Let A is (1, 0). Then $z = \cos0^\circ + i\sin0^\circ.$ Clearly,
  $B$ and $C$ would be $\cos120^\circ + i\sin120^\circ$ and $\cos240^\circ + i\sin240^\circ$. Similarly, $B$
  and $C$ can be found if $A$ is $(-1, 0)$.
\item Let $z$ represent $A$, then $\frac{z - (2 - i)}{1 + i - (2 - i)} = \frac{AM}{MD}e^{\frac{2\pi i}{2}}
  \Rightarrow z = (2 - i) + \frac{i}{2}(-1 + 2i) \Rightarrow z = 1 - \frac{3}{2}i$ or $3 -\frac{i}{2}$.
\item $\frac{z_1 - z_2}{z_3 - z_2} = r^{\frac{i\pi}{2}} = i\Rightarrow z_3 = -iz_1 + z_2(1 + i)$. Similarly,
  $z_4$ can be found.
\item $z_1 =2\left(\frac{1}{2} + \frac{\sqrt{3}}{2}i\right) = 2(\cos60^\circ + i\sin60^\circ)$. Therefore,
  $z_2 = 2(\cos180^\circ + i\sin180^\circ) = -2$ and $z_3 = 2(\cos300^\circ + i\sin300^\circ)$.
\item We know that three vertices represent an equilateral triangle if $z_1^2 + z_2^2 + z_3^2 - z_1z_2 -
  z_2z_3 -z_1z_3 = 0.$ Substituting the respective values, we get

  $a^2 - 1 + 2ai + 1 - b^2 + 2bi - a + b - abi - i = 0 \Rightarrow a^2 - b^2 - a + b = 0 \Rightarrow (a -
  b)(a + b + 1) = 0$. So either $a = b$ or $a + b = -1$ but if we choose $a + b = -1$ then the other part
  leads us to $ab = 3$ which is not possible.

  Choosing $a = b$, the imaginary part becomes $2a + 2b - ab - 1 = 0 \Rightarrow a = 2 \pm \sqrt{3}$. But $a
  = 2 + \sqrt{3}$ does not make triangle equilateral. So $a = b = 2 - \sqrt{3}$.
\item Let $O = z$ represent center of the sqsuare then $z = \frac{A + C}{2} \Rightarrow C = 4 + 0i = 4$.
  $AC = AB.\sqrt{2}.e^{\pi/4} \Rightarrow B = 1 + 2i$ and $AD = AB.e^{\pi/2} = 6 + 3i$.
\item Let $O$ be the origin and $A_1$ the vertex $z_1$. Let the vertex adjacent to $A_1$ be $A_2$. Then $z_2
  = z_1e^{2\pi i/n}\;\because \angle A_1OA_2 = \frac{2\pi}{n}$. Similarly, $z_3, z_4, \ldots, z_n$ are other
  vertices in order, then $z_3 = e^{4\pi i/n}, z_4 = e^{6\pi i/n}, \ldots$. Thus, all vertices are given by
  $z_{r + 1} = z_1e^{2\pi ri/n} = z_1(\cos2r\pi/n + i\sin2r\pi/n), \ldots$, where $r = 1, 2, \ldots, n - 1$.
\item $z_1, z_2, z_3$ are collinear if $\startdeterminant\NC z_1 \NC \overline{z_1}\NC1\NR\NC z_2 \NC
  \overline{z_2}\NC1\NR\NC z_3 \NC \overline{z_3}\NC1 \NR\stopdeterminant=0$. Substituting $a, b, c$ in this
  and expnading the determinant it is trivial to obtain the given condition.
\item $PA^2 = 4PB^2 \Rightarrow |z - 6i|^2 = 4|z - 3|^2 \Rightarrow x^2 + (y - 6)^2 = 4[(x - 3)^2 + y^2]
  \Rightarrow x^2 + y^2 - 8x + 4y = 0$, which represents a circle with center at $(4 -2)$ and radius
  $\sqrt{20}$.

  $x^2 + y^2 - 8x + 4y = 0 \Rightarrow x^2 + y^2 = 4(2x) + 2i(2iy) \Rightarrow |z|^2 = 4(z +
  \overline{z}) + 2i(z - \overline{z}) = (4 + 2i)z + (4 - 2i)\overline{z}$.
\item The diagram is given below:
  \startplacefigure[location={left, none}]
    \externalfigure[concyclic_complex.pdf][height=4cm]
  \stopplacefigure
  Let three non-collinear points be $A(z_1), B(z_2)$ and $C(z_3)$. Let $P(x)$ be any point on the circle.

  Then either $\angle ACB = \angle APB$ (when they are in the same segment) or $\angle ACB + \angle APB =
  \pi$ (when they are in the opposite segment).

  $\arg\left(\frac{z_3 - z_2}{z_3 - z_1}\right) - \arg\left(\frac{z - z_2}{z - z_1}\right) = 0$ or
  $\arg\left(\frac{z_3 - z_2}{z_3 - z_1}\right) + \arg\left(\frac{z - z_1}{z - z_2}\right) = \pi$

  $\arg\left[\left(\frac{z_3 - z_2}{z_3 - z_1}\right)\left(\frac{z - z_1}{z - z_2}\right)\right] = 0$ or
  $\arg\left[\left(\frac{z_3 - z_2}{z_3 - z_1}\right)\left(\frac{z - z_1}{z - z_2}\right)\right] = \pi$

  In any case, we get $\frac{(z_3 - z_2)}{(z_3 - z_1)}\frac{(z - z_1)}{(z - z_2)}$ is purely real. Hence,
  proved.
\item Following from previous problem we have one equation for the condition for the four vertices to be
  cyclic. Also, sum of all four angles of the quadrilateral is equal to be $2\pi$. From these two equations,
  the results can be deduced.
\item Consider the following diagram:
  \startplacefigure[location={left,none}]
    \startMPcode
      % draw((-2, 0) -- (2, 0)^^(-2, 0)--(0, 2)^^(0, 2) -- (2,0));
      % label("$A(z_1)$", (-2, 0), S);
      % label("$B(z_2)$", (2, 0), S);
      % label("$C(z_3)$", (0, 2), N);
      % draw((-2, -3) -- (2, -3)^^(-2, -3)--(0, -1)^^(0, -1) -- (2,-3));
      % label("$P({z^\prime}_1)$", (-2, -3), S);
      % label("$Q({z^\prime}_2)$", (2, -3), S);
      % label("$R({z^\prime}_3)$", (0, -1), N);
      draw (-1cm, 0) -- (1cm, 0) -- (0, 1cm) -- cycle;
      label.bot("$A(z_1)$", (-1cm, 0));
      label.bot("$B(z_2)$", (1cm, 0));
      label.top("$C(z_3)$", (0, 1cm));
      draw (-1cm, -1.5cm) -- (1cm, -1.5cm) -- (0, -.5cm) -- cycle;
      label.bot("$P(z_1')$", (-1cm, -1.5cm));
      label.bot("$Q(z_2')$", (1cm, -1.5cm));
      label.top("$R(z_3')$", (0, -.5cm));
    \stopMPcode
  \stopplacefigure
  $\triangle ABC$ and $\triangle PQR$ will be similar if all their angles are equal and ratios of sides as
  well.

  $arg\left(\frac{z_3 - z_1}{z_2 - z_1}\right) = arg\left(\frac{{z^\prime}_3 - {z^\prime}_1}{{z^\prime}_2 -
    {z^\prime}_1}\right)$

  $\frac{AB}{PQ} = \frac{AC}{PR}$ or $\frac{AC}{AB} = \frac{PR}{PQ}$ or $\frac{z_3 - z_1}{z_2 - z_1} =
  \frac{{z^\prime}_3 - {z^\prime}_1}{{z^\prime}_2 - {z^\prime}_1}$

  Simplifying these two equations gives us our determinant.
  \vskip .8cm
\item From these two equations we have $r = \frac{c - a}{b - a}$ and $r = \frac{\omega - u}{v -u}$. Equating
  these two equations and taking modulus and argument, it follows from the previous problem that the two
  triangles are similar.
\item We know that points on a perpendicular bisector is equidistant from the two points of the line to
  which it is perperndicular bisector.

  $\Rightarrow |z - z_1| = |z - z_2| \Rightarrow |z - z_1|^2 = |z - z_2|^2 \Rightarrow (z -
  z_1)(\overline{z} - \overline{z_1}) = (z - z_2)(\overline{z} - \overline{z_2})$, which can be written in
  the form of $\overline{a}z + a\overline{z} + b = 0$, which is equation of a straight line.
\item Mid-point of such a diameter is $\frac{z_1 + z_2}{2}$. Let $P$ be a point lying on this circle, which,
  is represented by complex number $z$. Thus, the equation of circle is $\left|z - \frac{z_1 +
    z_2}{2}\right| = \left|z_1 - \frac{z_1 + z_2}{2}\right|$ or $\left|z - \frac{z_1 + z_2}{2}\right| =
  \left|z_2 - \frac{z_1 + z_2}{2}\right|$. Square and simplify to arrive at the equation.
\item The equation can be written as $\left|z - z_1\right| = c\left|z - z_2\right|$, which, when substituted
  with $z_1 = x_1 + iy_1$ and $z_2 = x_2 + iy_2$ gives following

  $\left|(x - x_1) + i(y - y_1)\right| = c\left|(x - x_2) + i(y - y_2)\right| \Rightarrow (x - x_1)^2 + (y -
  y_1)^2 = c^2\{(x - x_2)^2 + (y - y_2)^2\}$, which is equation of a circle.
\item Given, $|z| = 1 \Rightarrow 2z\overline{z} = 2 \Rightarrow \frac{2}{z} = 2\overline{z}$ which gives us
  a circle.
\item  Let $z_1 = r_1(\cos\theta_1 + i\sin\theta_1)$ and $z_2 = r_2(\cos\theta_2 + i\sin\theta_2)$. Then
  L.H.S. $= \left|z_1 + z_2\right|$ $\Rightarrow \left|z_1 + z_2\right|^2 = {r^2}_1 + {r^2}_2 +
  2r_1r_2\cos(\theta_1 - \theta_2)$.

  Similarly, $\left(\left|z_1\right| + \left|z_1\right|\right)^2 = \left({r^2}_1 + {r^2}_2 + 2r_1r_2\right)$.

  Thus, $\cos(\theta_1 - \theta_2) = 0$ $\Rightarrow \arg(z_1) - \arg(z_2) = 2n\pi$.
\item The diagram is given below:
  \startplacefigure[location={left, none}]
    \startMPcode

      pair c; c = (2cm, -2cm);
      draw fullcircle scaled 2cm shifted c;
      label.llft("$C$", c);
      drawarrow (-0.1cm, 0) -- (3cm, 0);
      drawarrow (0, 0.1cm) -- (0, -3cm);
      draw (-.1cm, .1cm) -- (3.2cm, -3.2cm);
      label.top("$P$", (1.29cm, -1.29cm));
      label.urt("$O$", (0, 0));
      label.rt("$x$", (3cm, 0));
      label.bot("$y$", (0, -3cm));
    \stopMPcode
  \stopplacefigure
  The equation $\left|z - 2 + 2i\right| = 1$ represents a circle with center at $(2, -2i)$ with unity
  radius. Since, the line between $(2, -2i)$ and origin will make an angle of $45^\circ$. Therefore, $P$ is
  $2 - \frac{1}{\sqrt{2}} + i(\frac{1}{\sqrt{2}} -2)$.
  \vskip 2.2cm
\item The diagram is given below:
  \startplacefigure[location={left,none}]
    \startMPcode
      pickup pencircle scaled 0.2pt;
      pair c;
      c = (0cm, 1.25cm);
      draw fullcircle scaled 1.5cm shifted c;
      drawarrow (-0.5cm, 0cm) -- (2cm, 0cm);
      drawarrow (0cm, -0.5cm) -- (0cm, 2.5cm);
      draw (0, 0) -- (1.5cm, 2cm);
      draw (0, 1.25cm) -- (.6cm, .8cm);
      label.rt("$x$", (2cm, 0));
      label.top("$y$", (0, 2.5cm));
      label.llft("$O$", (0,0));
      label.top("$C(0, 5)$", c);
      label.rt("$P$", (.6cm, .8cm));
      draw fullcircle scaled 16 rotated angle ((.5cm, 0) - (0, 0)) shifted (0, 0) cutafter ((0, 0) -- (1.5cm, 2cm));
      % angle marking
      label.rt("$\theta$", (0.2cm, 0.2cm));
      draw fullcircle scaled 16 rotated angle ((0, 0) - c) shifted c cutafter (c -- (.6cm, .8cm));
      label.rt("$\theta$", c - (0cm, 0.4cm));
    \stopMPcode
  \stopplacefigure
  Given equation is a circle with center $(0, 5)$ and radius $3 \therefore OC = 5, CP = 3$.

  The point having
  least argument will have a tangent from origin which makes $\triangle OCP$ right angle
  triangle.

  $\Rightarrow CP = 4\Rightarrow \tan\theta = \frac{4}{3}$. Therefore, the point would be
  $4(\cos\theta + i\sin\theta) = \frac{12}{5} + \frac{16i}{5}$.
  \vskip .5cm
\item From given equation, $\left(\frac{\left|z-1\right| + 4}{3\left|z - 1\right| - 2}\right) < \frac{1}{2}$

  $\Rightarrow |z - 1| > 10$. This represents area which lies outside a circle with center at $(1, 0)$ and
  radius $10$.
\item Let $z = x + iy$ then the equation becomes $x^2 - y^2 + x + 1 + iy(1 + 2x) = 0$. Clearly, imaginary
  part has to be zero i.e. either $y = 0$ or $x = -\frac{1}{2}$. So, it is real and positive for all points
  on the x-axis. When, $x = -\frac{1}{2}$ the real part becomes $y^2 = \frac{3}{4}$. Thus, for points $x =
  -\frac{1}{2}$ and $-\frac{\sqrt{3}}{2}<y<\frac{\sqrt{3}}{2}$ the required condition is satisfied.
\item First equation represents a circle whose center is at $(0, ia)$ and radius equal to $\sqrt{a +
  4}$. The second equation represents interior of a circle with center at $(2, 0)$ and radius unity. Now,
  for the possibility of existence of $z$ the two circles must intersect each other.

  $\Rightarrow \sqrt{a^2 + 4} \leq a + 4 + 1 \Rightarrow a \geq -\frac{21}{10}$ and $a + 4 - 1\leq \sqrt{a^2
    + 4} \Rightarrow a\leq -\frac{5}{6}$. Combining these two gives us the range for values of $a$.
\item Let $z = x + iy$ then $|z + \sqrt{2}| = \sqrt{x^2 + 2\sqrt{2}x + 2 + y^2} = t^2 - 3t + 2$ and $|z +
  i\sqrt{2}| = \sqrt{x^2 + y^2 + 2\sqrt{2}y + 2} < t^2$.

  Because $|z + \sqrt{2}| > 0 \Rightarrow t^2 - 3t + 2 > 0 \Rightarrow t < 1, t > 2$ and $t > 0$. Both the
  equations are circles so they must intersect for $t$ to exist. The distance between centers
  i.e. $(-\sqrt{2}, 0)$ and $(0, -i\sqrt{2})$ is $2$.

  $\Rightarrow r_1 + r_2 > 2 \Rightarrow 2t^2 - 3t + 2 > 2 \Rightarrow t(2t - 3) > 0 \Rightarrow t < 0, t >
  \frac{3}{2}$ and $r_1 < r_2 + 2 \Rightarrow t^2 - 2t + 2 < t^2 + 2 \Rightarrow t  > 0$. Combining all the
  inequalities, $t > 2$.
\item Let $z = x + iy$ then $\sqrt{x^2 + 8x + 16 + y^2} = \sqrt{a^2 - 12a + 28}$ and $\sqrt{x^2 - 8\sqrt{3}x
  + 48 + y^2} < 1$.

  Becaise $|z + 4| > 0 \Rightarrow a^2 - 12a + 28 > 0 \Rightarrow a > 6 + 2\sqrt{2}, a < 6 - 2\sqrt{2}$ and
  $a > 0$. Both the equations are  circles so they must intersect for $a$ to exist. The distance between
  centers i.e. $(0, -4i)$ and $(4\sqrt{3}, 0)$ is $8$.

  $\Rightarrow r_1 + r_2 > 8 \Rightarrow \sqrt{a^2 - 12a + 28} + a > 8 \Rightarrow a > 9$ and $r_1 < r_2 + 8
  \Rightarrow a < -\frac{9}{7}$. Combining all these inequalities we have $a > 9$.
\item Let $z = x + iy \Rightarrow (1 + i)z^2 = (1 + i)(x^2 - y^2 + 2ixy)\Rightarrow \Re[(1 + i)z^2] = x^2 -
  y^2 - 2xy > 0 \Rightarrow x$ has two limits $y(1 \pm\sqrt{2})$.
\item Let $z = x + iy$ then $2z = |z| + 2i \Rightarrow 2(x + iy) = \sqrt{x^2 + y^2} + 2iy$. Equating real
  and imaginary parts, $y = 1, 2x = \sqrt{x^2 + 1}$. Squaring $4x^2 = x^2 + 1 \Rightarrow x =
  \pm\frac{1}{\sqrt{3}}$.
\item We have earlier proven that if there are two non-parallel lines cutting a circle at $a, b$ and $c, d$
  then their point of intersection is given by $\frac{a^{-1} + b^{-1} - c^{-1} - d^{-1}}{a^{-1}b^{-1}-
    c^{-1}d^{-1}}$. Now if $c$ and $d$ coincide then that line will become a tangent. So putting $d = c$ we
  have

  $z = \frac{a^{-1} + b^{-1} - 2c^{-1}}{a^{-1}b^{-1} - c^{-2}}$.
\item Given $a_1z^3 + a_2z^2 + a_3z + a_4 = 3 \Rightarrow |a_1z^3 + a_2z^2 + a_3z + a_4| = 3 \Rightarrow
  |a_1z^3| + |a_2z^2| + |a_3z| + |a_4| \geq 3$

  $\Rightarrow |a_1||z^3| + |a_2||z^2| + |a_3||z| + |a_4|\geq 3 \Rightarrow |z|^3 + |z|^2 + |z| + 1\geq
  3\;[\because |a_i|\leq 1]$

  $\Rightarrow 1 + |z| + |z|^2 + |z|^3 + \cdots $ to $\infty > 3 \Rightarrow \frac{1}{1 - |z|} > 3
  \Rightarrow |z| > \frac{2}{3}$, which shows that roots lie outside the circle with center origin and
  radius $\frac{2}{3}$.
\item The diagram is given below:
  \startplacefigure[location={left, none}]
    \startMPcode
      draw fullcircle scaled 2cm;
      pair a; pair b; pair c; pair d;
      a = (.707cm, .707cm); b = (-.707cm, .707cm); c = (-.707cm, -.707cm); d = (.707cm, -.707cm);
      draw a -- b -- c -- d -- cycle;
      draw a -- c; draw b -- d;
      label.urt("$A$", a);
      label.ulft("$B$", b);
      label.llft("$C$", c);
      label.lrt("$D$", d);
      label.rt("$O$", (a + c)/2);
    \stopMPcode
  \stopplacefigure
  Given, $b_1z_1 + b_3z_3 = -(b_2z_2 + b_4z_4)$ and $b_1 + b_3 = -(b_2 + b_4)\;\therefore \frac{b_1z_1 +
    b_3z_3}{b_1 + b_3} = \frac{b_2z_2 + b_4z_4}{b_2 + b_4}$.

  This means that the point dividing $AC$ in the ratio $b_3:b_1$ also divides $BC$ in the ratio
  $b_4:b_2$. Let this point be $O$. Let $b_1b_2|z_1 - z_2|^2 = b_3b_4|z_3 - z_4|^2$

  $\Rightarrow b_1b_2(b_3^2 + b_4^2 - 2b_3b_4\cos\alpha) = b_3b_4(b_2^2 + b_1^2 - 2b_1b_2\cos\alpha)$

  $\Rightarrow \frac{b_3}{b_4} + \frac{b_4}{b_3} = \frac{b_1}{b_2} + \frac{b_2}{b_1}\Rightarrow
  \frac{b_3}{b_4} = \frac{b_1}{b_2}$ or $\frac{b_2}{b_1}$

  If $\frac{b_3}{b_4} = \frac{b_1}{b_2}$, then $\frac{b_3}{b_1} = \frac{b_4}{b_1} \Rightarrow \frac{AO}{CO}
  = \frac{BO}{DO}$

  $\Rightarrow \triangle AOB\sim\triangle BCO \Rightarrow \angle BAO = \angle CDO \Rightarrow AB\parallel
  CD$ which is not possible.

  If $\frac{b_3}{b_4} = \frac{b_2}{b_1}$ then $\frac{AO}{BO} = \frac{DO}{CO}\Rightarrow \triangle
  ADO\sim\triangle BCO\Rightarrow \angle DAO = \angle OBC\Rightarrow A, B, C, D$ are concyclic.
\item The diagram is given below:
  \startplacefigure[location={left, none}]
    \startMPcode
      draw (1cm, -1.5cm) -- (1cm, 1.5cm);
      draw (-1cm, 0) -- (2cm, 0);
      draw (1cm, -1.5cm) -- (-1cm, 0) -- (1cm, 1.5cm);
      draw (1cm, -1.5cm) -- (3cm, 0) -- (1cm, 1.5cm);
      draw (-1cm, 0) -- (3cm, 0);
      draw (1cm, -1.5cm) -- (0, 0) -- (1cm, 1.5cm);
      label.top("$B(\beta, \gamma)$", (1cm, 1.5cm));
      label.bot("$C(\beta, -\gamma)$", (1cm, -1.5cm));
      label.lft("$P$", (-1cm, 0));
      label.rt("$Q$", (3cm, 0));
      label.lrt("$L(\beta, 0)$", (1cm, 0));
      label.llft("$A(a, 0)$", (.5cm, 0));
    \stopMPcode
  \stopplacefigure
  Let $f(x) = k(x - a)(x - \beta - i\gamma)(x - \beta + i\gamma) = k(x - a)[(x - \beta)^2 + \gamma^2]$

  $\Rightarrow f'(x) = k[3x^2 - 2(a + 2\beta)x + \beta^2 + \gamma^2 + 2a\beta]$. Discriminant of $f'(x)$ is
  given by $D = 4[(a + 2\beta)^2 - 3(\beta^2 + \gamma^2 + 2a\beta)] = 4(a^2 + \beta^2 - 3\gamma^2 -
  2a\beta)$

  $BC = 2|\gamma| \Rightarrow PL = \sqrt{3}|\gamma|$. If $A$ lies inside the equilateral triangle having
  $BC$ as base, then $|\beta - a|< \sqrt{3}\gamma \Rightarrow (\beta - a)^2 < 3\gamma^2 \Rightarrow a^2 +
  \beta^2 - 3\gamma^2 - 2a\beta < 0 \Rightarrow D < 0$. Thus roots will be complex numbers.
\item Let $a = \alpha + i\beta$ and $z = x + iy$, then $\overline{a}z + a\overline{z} = 0$ becomes as
  $\alpha x + \beta y = 0$ or $y = \left(\frac{-\alpha}{\beta}\right)x$.

  Its reflection in the x-axis is $y = \frac{\alpha}{\beta}x$ or $\alpha x - \beta y = 0 \Rightarrow
  \left(\frac{a + \overline{a}}{2}\right)\left(\frac{z + \overline{z}}{2}\right) - \left(\frac{a -
    \overline{a}}{2}\right)\left(\frac{z -\overline{z}}{2}\right) = 0$

  $\Rightarrow az + \overline{a}\overline{z} = 0$
\item We have $z = \frac{\alpha + \beta t}{\gamma + \delta t} \Rightarrow t = \frac{\alpha - \gamma
  z}{\delta z - \beta}$. As $t$ is real, $\frac{\alpha - \gamma z}{\delta z - \beta} =
  \frac{\overline{\alpha} - \overline{\gamma z}}{\overline{\delta z} - \overline{\beta}}$

  $\Rightarrow \Rightarrow (\alpha - \gamma z)(\overline{\delta z} - \overline{\beta}) = (\overline{\alpha}
  - \overline{\gamma z})(\delta z - \beta)$

  $\Rightarrow (\overline{\gamma}\delta - \gamma\overline{\delta})z\overline{z} +
  (\gamma\overline{\beta}-\overline{\alpha}\delta)z +(\alpha\overline{\delta} -
  \beta\overline{\gamma})\overline{z} = (\alpha\overline{\beta} - \overline{\alpha}\beta)$

  Since $\frac{\gamma}{\delta}$ is real, $\frac{\gamma}{\delta} =
  \frac{\overline{\gamma}}{\overline{\delta}}$ or $\gamma\overline{\delta} - \delta\overline{\gamma} = 0$.

  Thus, $\overline{a}z + a\overline{z} = c$, where $a = i(\alpha\overline{\delta}) - \beta\overline{\gamma}$
  and $c = i(\overline{\alpha}\beta - \alpha\overline{\beta})$.

  Note that $a \ne 0$ for if $a = 0$ then $\alpha\overline{\delta} - \beta\overline{\gamma} = 0\Rightarrow
  \frac{\alpha}{\beta} = \frac{\overline{\gamma}}{\overline{\delta}} = \frac{\gamma}{\delta}\Rightarrow
  \alpha\delta - \beta\gamma = 0$, which is against the hypothesis.

  Also, note that $c = i(\overline{\alpha}\beta - \alpha\overline{\beta})$ is a purely real number. Thus, $z
  = \frac{\alpha + \beta t}{\gamma + \delta t}$ represents a straight line.
\item The solutions are given below:
  \startitemize[i]
  \item L.H.S. $= (3 + 3\omega + 5\omega^2)^6 - (2 + 6\omega + 2\omega^2)^3 = [(3 + 3\omega + 3\omega^2 +
    2\omega^2)^6 - (2 + 2\omega + 2\omega^2 + 4\omega)^3] = [\{3(1 + \omega + \omega^2) + 2\omega^2\}^6] -
    [\{2(1 + \omega + \omega^2) + 4\omega\}^3]$

    $= 64\omega^{12} - 64\omega^3 = 0 =$ R.H.S. $[\because 1 + \omega + \omega^2 = 0]$.
  \item L.H.S. $= (2 - \omega)(2 - \omega^2)(2 - \omega^{10})(2 - \omega^{11}) = (2 - \omega)(2 -
    \omega^2)(2 - \omega)(2 - \omega^2) = [(2 - \omega)(2 - \omega^2)]^2$

    $= (4 - 2\omega - 2\omega^2 + \omega^3)^2 = [5 - 2(\omega + \omega^2)]^2 = (5 + 2)^2 = 49 =$ R.H.S.
  \item L.H.S. $= (1 - \omega)(1 - \omega^2)(1 - \omega^4)(1 - \omega^5) = (1 - \omega)^2(1 - \omega^2)^2 =
    (1 - \omega - \omega^2 + \omega^3)^2$

    $= [2 - (-1)]^2 = 9 =$ R.H.S.
  \item L.H.S. $= (1 - \omega + \omega^2)^5 + (1 + \omega - \omega^2)^5 = (-2\omega)^5 + (-2\omega^2)^5 =
    -32(\omega + \omega^2) = 32 =$ R.H.S.
  \item L.H.S. $= 1 + \omega^n + \omega^{2n}$, where $n = 3m\;\forall\;m\in \mathbb{I}$ L.H.S. $= 1 +
    \omega^{3m} + \omega^{6m} = 1 + (\omega^3)^m + (\omega^3)^{2m} = 1 + 1 + 1 = 3 =$ R.H.S.
  \item We have to prove that $1 + \omega^n + \omega^{2n} = 0$. If $n = 3m + 1\;\forall\;m\in\mathbb{I}$
    then L.H.S. $= 1 + \omega^{3m + 1} + \omega^{6m + 2} = 1 + \omega + \omega^2 = 0 =$ R.H.S.

    If $n = 3m + 2,\;\forall\;m\in\mathbb{I}$ then L.H.S. $= 1 + \omega^{3m + 2} + \omega^{6m + 4} = 1 +
    \omega^2 + \omega = 0 =$ R.H.S.
  \stopitemize
\item We have $a^2 + b^2 + c^2 - ab - bc - ca = a^2 + \omega^3b^2 + \omega^3c^2 + (\omega + \omega^2)ab +
  (\omega + \omega^2)bc + (\omega + \omega^2)ca$

  $= (a^2 + ab\omega + ca\omega^2) + (ab\omega^2 + b^2\omega^3 + bc\omega) + (ca\omega + bc\omega^2 +
  c^2\omega^3)$

  $= a(a + b\omega + c\omega^2) + b\omega^2(a + b\omega + c\omega^2) + c\omega(a + b\omega + c\omega^2)$

  $= (a + b\omega + c\omega^2)(a + b\omega^2 + c\omega)$.
\item $x^3 + y^3 + z^3 = (a + b)^3 + (a\omega + b\omega^2)^3 + (a\omega^2 + b\omega)^3 = a^3 + b^3 + 3a^2b +
  3ab^2 + a^3\omega^3 + b^3\omega^6 + 3a^2b\omega^4 + 3ab^2\omega^5 + a^3\omega^6 + b^3\omega^3 +
  3a^2b\omega^5 + 3ab^2\omega^4 = 3[a^3 + b^3 + 3a^2b(1 + \omega + \omega^2) + 3ab^2(1 + \omega + \omega^2)]
  = 3(a^3 + b^3) = $R.H.S.

  $xyz = (a + b)(a\omega + b\omega^2)(a\omega^2 + b\omega) = (a + b)(a^2 + ab\omega + ab\omega^2 + b^2) = (a
  + b)(a^2 + b^2 - ab) = a^3 + b^3 =$ R.H.S.
\item Given below are the factorization of the expressions:
  \startitemize[i]
  \item $a^2 - ab + b^2 = a^2 + (\omega + \omega^2)ab + b^2\omega^3 = (a + b\omega)(a + b\omega^2)$.
  \item $a^2 + ab + b^2 = a^2 - (\omega + \omega^2)ab + b^2\omega^3 = (a - b\omega)(a - b\omega^2)$.
  \item $a^3 + b^3 = (a + b)(a^2 - ab + b^2) = (a + b)(a + b\omega)(a + b\omega^2)$.
  \item $a^3 - b^3 = (a - b)(a^2 + ab + b^2) = (a + b)(a - b\omega)(a - b\omega^2)$.
  \item $a^3 + b^3 + c^3 - 3abc = (a + b + c)(a^2 + b^2 + c^2 - ab - bc - ca) = (a + b + c)(a + b\omega +
    c\omega^2)(a + b\omega^2 + c\omega)$.
  \stopitemize
\item $x^{3p} + x^{3q + 1} + x^{3r + 2}$ will be divisible by $x^2 + x + 1$ only if all the factors of $x^2
  + x + 1$ satisfy $x^{3p} + x^{3q + 1} + x^{3r + 2}$.

  $x^2 + x + 1 = 0\Rightarrow x = \omega, \omega^2$. If $x = \omega$ then $x^{3p} + x^{3q + 1} + x^{3r + 2}
  = (\omega^3)^p + (\omega^3)^q.\omega + (\omega^3)^r.\omega^2 = 1 + \omega + \omega^2 = 0$.

  If $x = \omega^2$ then $x^{3p} + x^{3q + 1} + x^{3r + 2} = (\omega^6)p + (\omega^6)^q.\omega^2 +
  (\omega^6)^r.\omega^4 = 1 + \omega^2 + \omega = 0$. Hence proved.
\item Following like previous problem $x^3 + x^2 + x + 1 = (x + 1)(x^2 + 1) = 0 \Rightarrow x = -1, \pm i$.

  If $x = -1$ then $x^{4p} + x^{4q + 1} + x^{4r + 2} + x^{4s + 3} = (-1)^{4p} + (-1)^{4q + 1} + (-1)^{4r +
    2} + (-1)^{4s + 3} = 1 - 1 + 1 - 1 = 0$.

  If $x = i$, then $x^{4p} + x^{4q + 1} + x^{4r + 2} + x^{4s + 3} = i^{4p} + i^{4q + 1} + i^{4r + 2} + i^{4s
    + 3} = 1 + i - 1 - i = 0$.

  If $x = -i$, then $x^{4p} + x^{4q + 1} + x^{4r + 2} + x^{4s + 3} = (-i)^{4p} + (-i)^{4q + 1} + (-i)^{4r +
    2} + (-i)^{4s + 3} = 1 - i - 1 + i = 0$. Hence proved.
\item $p^3 + q^3 + r^3 - 3pqr = (p + q + r)(p^2 + q^2 + r^2 - pq - qr - rp) = (p + q + r)(p + q\omega +
  r\omega^2)(p + q\omega^2 + r\omega)$

  $p + q + r = 3a + b(1 + \omega + \omega^2) + c(1 + \omega^2 + \omega) = 3a$. Similarly, $p + q\omega +
  r\omega^2 = 3c$ and $p + q\omega^2 + r\omega = 3b$. Hence, $p^3 + q^3 + r^3 - 3pqr = 27abc$, proved.
\item Let $p = (a + b\omega + c\omega^2)$ and $q = (a + b\omega^2 + c\omega)$ then we know that $p^3 + q^3 =
  (p + q)(p + q\omega)(p + \omega^2)$.

  $p + q = 2a - b - c, p + q\omega = 2b - c - a, p + q\omega^2 = 2c - a - b$, and hence

  $(a + b\omega + c\omega^2)^3 + (a + b\omega^2 + c\omega)^3 = (2a - b - c)(2b - a - c)(2c - a - b)$.
\item The solutions are given below:
  \startitemize[i]
  \item $(a^2 + b^2 + c^2 - ab - bc - ca)(x^2 + y^2 + z^2 - xy - yz - zx) = (a + b\omega + c\omega^2)(a +
    b\omega^2 + c\omega)(x + y\omega + z\omega^2)(x + y\omega^2 + z\omega)$

    $= (a + b\omega + c\omega^2)(x + y\omega + z\omega^2)[(a + b\omega^2 + c\omega)(x + y\omega^2 +
      z\omega)]$

    $= (ax + cy\omega^3 + bz\omega^3 + cx\omega^2 + by\omega^2 + za\omega^2 + bx\omega + ay\omega +
    cz\omega^4)(ax + cy\omega^3 + bz\omega^3 + cx\omega + by\omega^4 + az\omega + bz\omega^2 + ay\omega^2
    + cz\omega^2)$

    $= [(ax+cy+bz)(cx+by+az)\omega^2 + (bx+ay+cz)\omega][(ax+cy+bz)(cx+by+az)\omega + (bx+ay+cz)\omega^2]$

    $= (X+Y\omega^2+Z\omega)(X+Y\omega+Z\omega^2) = (X^2+Y^2+Z^2
    -YZ-ZX-XY)$.
  \item We just introduce two new factors to previous problem $a + b + c$ and $x + y + z$ and then it is
    only a matter of simplification to obtain the result.
  \stopitemize
\item L.H.S. $= \left(\frac{\cos\theta + i\sin\theta}{\sin\theta + i\cos\theta}\right)^4 =
  \left(\frac{\cos\theta + i\sin\theta}{i(\cos\theta - i\sin\theta)}\right)^4 =
  \frac{e^{i4\theta}}{e^{-i4\theta}} = e^{i8\theta} = \cos8\theta + i\sin8\theta =$ R.H.S.
\item Roots of the quadratic equation $z^2 - 2z\cos\theta + 1 = 0$ are given by $z = \cos\theta\pm
  i\sin\theta$.

  $\Rightarrow z^2 + z^{-2} = \cos2\theta \pm i\sin2\theta + \cos2\theta \mp i\sin2\theta = 2\cos2\theta =$
  R.H.S.
\item $1 + i = \sqrt{2}\left(\cos\frac{\pi}{4} + i\sin\frac{\pi}{4}\right)$ and $(1 - i) =
  \sqrt{2}\left(\cos\frac{\pi}{4} - i\sin\frac{\pi}{4}\right)$

  L.H.S. $= (1 + i)^n + (1 - i)^n = (\sqrt{2})^n.2\cos\frac{n\pi}{4} = 2^{\frac{n}{2} + 1}.cos\frac{n\pi}{4}
  =$ R.H.S.
\item $\displaystyle\sum_{k = 1}^6\left(sin\frac{2\pi k}{7} -icos\frac{2\pi k}{7}\right) = -i \sum_{k =
  1}^6\left(cos\frac{2\pi k}{7} + isin\frac{2\pi k}{7}\right)$

  $= -i \sum_{k = 1}^6 e^{\frac{i2\pi k}{7}} = -i \left[e^{\frac{i2\pi}{7}} + e^{\frac{i4\pi}{7}} + .. +
  e^{\frac{i12\pi}{7}}\right] = -i \left[\left(\frac{1 - e^{2\pi}}{1 - e^{\frac{i2\pi}{7}}}\right) -
  1\right] = -i [0 - 1] = i$.
\item Let $cot^{-1}p = \theta$, then $cot\theta = p$. Now, L. H. S. is

  $e^{2mi\theta}\left(\frac{icot\theta + 1}{icot\theta - 1}\right)^m  = e^{2mi\theta}\left[\frac{i(cot\theta
    - i)}{i(cot\theta + i)}\right]^m$

  $= e^{2mi\theta}\left(\frac{cos\theta - isin\theta}{cos\theta + isin\theta}\right)^m$

  $= e^{2mi\theta}\left(\frac{e^{-i\theta}}{e^i\theta}\right)^m = e^{2mi\theta} . e^{-2mi\theta} = e^0 = 1
  =$ R.H.S.
\item Let $1 + \sin\phi + i \cos\phi = r(\cos\theta + i \sin\theta) \therefore 1 + \sin\phi = r\cos\theta$
  and $\cos\phi = r\sin\theta$

  Now $(1 + \sin\phi + i \cos\phi)^n = r^n(\cos n\theta + i\sin n\theta)$. Taking conjugates, we get $(1
  + \sin\phi - i \cos\phi)^n = r^n(\cos n\theta - i\sin n\theta)$

  From these two, we get $\left(\frac{1 + \sin\phi + i \cos\phi}{1 + \sin\phi - i \cos\phi}\right)^n =
  \frac{\cos n\theta + i\sin n\theta}{\cos n\theta - i\sin n\theta} = \frac{e^{in\theta}}{e^{-in\theta}}$

  $= e^{2in\theta} = \cos 2n\theta + \sin 2n\theta$

  $\tan \theta = \frac{\cos \phi}{1 + \sin \phi} = \frac{\cos^2\frac{\phi}{2} -
    \sin^2\frac{\phi}{2}}{\left(\cos\frac{\phi}{2} + \sin\frac{\phi}{2}\right)^2} =
  \frac{\cos\frac{\phi}{2} - \sin\frac{\phi}{2}}{\cos\frac{\phi}{2} + \sin\frac{\phi}{2}} = \frac{1 -
    \tan\frac{\phi}{2}}{1 + \tan\frac{\phi}{2}} = \tan\left(\frac{\pi}{4} - \frac{\phi}{2}\right)$

  $\therefore \theta = \frac{\pi}{4} - \frac{\phi}{2}\;\therefore 2n\theta = \left(\frac{n\pi}{2} -
  n\phi\right)$. Hence, proved.
\item Let $a = \cos\alpha + i \sin\alpha, b = \cos\beta + i \sin\beta, c = \cos\gamma + i \sin\gamma$

  Now, $a + b + c = (\cos\alpha + \cos\beta + \cos\gamma) + i(\sin\alpha + \sin\beta + \sin\gamma) = 0 +
  i.0 = 0$

  Now, $a^3 + b^3 + c^3 - 3abc = (a + b + c)(a^2 + b^2 + c^2 - ab - bc - ca) = 0$ $[\because a + b + c = 0]$

  $\therefore a^3 + b^3 + c^3 = 3abc\therefore \cos3\alpha + \cos3\beta + \cos3\gamma = 3\cos(\alpha +
  \beta + \gamma)$ and $\sin3\alpha + \sin3\beta + \sin3\gamma = 3\sin(\alpha + \beta + \gamma)$.
\item Proceeding similarly as last problem and with an extra calculation we have

  $\frac{1}{a} + \frac{1}{b} + \frac{1}{c} = (\cos\alpha + \cos\beta + \cos\gamma) - i(\sin\alpha +
  \sin\beta + \sin\gamma) = 0$

  $\therefore a^2 + b^2 + c^2 = (a + b + c)^2 - 2(ab + bc + ca) = (a + b + c)^2 - 2 abc\left(\frac{1}{a} +
  \frac{1}{b} + \frac{1}{c}\right)$

  $\Rightarrow 0^2 - 2abc.0 = 0\therefore L.H.S. = (\cos2\alpha + \cos2\beta + \cos2\gamma) + i(\sin2\alpha
  + \sin2\beta + \sin2\gamma) = 0$

  Equating real and imaginary parts we have our desired result.
\item $t^2 -2t + 2 = 0 \Leftrightarrow t = \frac{2 \pm \sqrt{4 - 8}}{2} = 1 \pm i$

  Let $\alpha = 1+ i$ and $\beta = 1 - i\;\therefore x + \alpha = (x + 1) + i, x + \beta = (x + 1) - i$
  and $\alpha - \beta = 2i$

  Let $x + 1 = r\cos\phi$ and $1 = r\sin\phi$. We have, $\frac{(x + \alpha)^n - (x + \beta)^n}{(\alpha -
    \beta)} = \frac{\sin\theta}{\sin^n\theta}$

 $ \Leftrightarrow \frac{r^n(\cos n\phi + i \sin n\phi) - r^n(\cos n\phi - i\sin n\phi)}{2i} =
  \frac{\sin\theta}{\sin^n\theta} \Leftrightarrow r^n \sin n\phi = \frac{\sin\theta}{\sin^n\theta}$

  $\Leftrightarrow \frac{\sin n\phi}{\sin^n\phi} = \frac{\sin\theta}{\sin^n\theta} \Leftrightarrow$ one of
  the values of $\phi$ is $\theta$. $\left[\because r\sin\phi = 1\Rightarrow r^n =
    \frac{1}{\sin^n\phi}\right]$

  $\therefore x + 1 = r \cos\theta$ and $1 = r \sin\theta$. Dividing and evaluating we get $x = \cot\theta -
  1$.
\item Given, $(1 + x)^n = p_0 + p_1x + p_2x^2 + \cdots + p_nx^n$. Putting $x = i$, we get $(1 + i)^n = p_0 + p_1i + p_2i^2 + \cdots + p_ni^n$

  $= (p_0 - p_2 + p_4 - \cdots) + i(p_1 - p_3 + p_5 - \cdots)\Rightarrow \left[\sqrt{2}\left(\cos\frac{\pi}{4} +
  i\sin\frac{\pi}{4}\right)\right]^n = (p_0 - p_2 + p_4 - \cdots) + i(p_1 - p_3 + p_5 - \cdots)$

  Equating real and imaginary parts, we have $p_0 - p_2 + p_4 \cdots = 2^{\frac{n}{2}}\cos\frac{n\pi}{4}$ and
  $p_1 - p_3 + p_5 - \cdots = 2^{\frac{n}{2}}\sin\frac{n\pi}{4}$.
\item Given, $(1 - x + x^2)^n = a_0 + a_1 + a_2x^2 + \cdots a_{2n}x^{2n}$. Putting $x = 1, \omega$ and $\omega^2$, we get

  $1 = a_0 + a_1 + a_2 + \cdots + a_{2n}, (-2\omega)^n = a_0 + a_1\omega + a_2\omega^2 + \cdots +
  a_{2n}\omega^{2n}, (-2\omega^2)^n = a_0 + a_1\omega^2 + a_2\omega^4 + \cdots +
  a_{2n}\omega^{4n}$

  Adding these we get, $3(a_0 + a_3 + a_6 + \cdots) = 1 + (-2)^n(\omega^n + \omega^{2n})$. Now $\omega =
  \frac{-1 + \sqrt{3}i}{2} = \left(\cos\frac{2\pi}{3} + i\sin \frac{2\pi}{3}\right)$

  $\omega^n = \cos\frac{2n\pi}{3} + i\sin\frac{2n\pi}{3}$. Now $\omega^2 = \frac{-1 - \sqrt{3}i}{2} =
  \left(\cos\frac{2\pi}{3} - i\sin \frac{2\pi}{3}\right)\therefore \omega^n + \omega^{2n} =
  2\cos\frac{2n\pi}{3} = 2\cos\left(n\pi - \frac{n\pi}{3}\right)$

  $= 2(-1)^n\cos\frac{n\pi}{3}$. Thus, $3(a_0 + a_3 + a_6 + \cdots) = 1 + (-2)^n2(-1)^n\cos\frac{n\pi}{3} = 1 +2^{n + 1}\cos\frac{n\pi}{3}$.

  $a_0 + a_3 + a_6 + \cdots = \frac{1}{3}\left(1 + 2^{n+1}\cos\frac{n\pi}{3}\right)$.
\item Given, $(1 + x)^n = c_0 + c_1x + c_2x^2 + \cdots + c_nx^n$. Putting $x = 1$ and $x = -1$, we get $2^n = c_0 + c_1 + c_2 + \cdots + c_n$

  and $0 = c_0 - c_1 + c_2 - \cdots + (-1)^nc_n$. Adding these two, we get $2^n = 2(c_0 + c_2 + c_4 +
  \cdots)$ or $c_0 + c_2 + c_4 + \cdots = 2^{n - 1}$

  Putting $x = i$, we get $(1 + i)^n = c_0 + c_1i + c_2i^2 + c_3i^3 + \cdots + c_ni^n\Rightarrow
  \left[\sqrt{2}\left(\cos\frac{\pi}{4}+i\sin\frac{\pi}{4}\right)\right]^n = (c_0 -c_2 + c_4 - \cdots) +
  i(c_1 - c_3 + \cdots)$

  $\Rightarrow 2^{\frac{n}{2}}\left(\cos\frac{n\pi}{4}+i\sin\frac{i\pi}{4}\right) = (c_0 -c_2 + c_4 - \cdots) + i(c_1 - c_3 + \cdots)$

  Equating real parts, we get $c_0 - c_2 + c_4 - \cdots = 2^{\frac{n}{2}}\cos\frac{n\pi}{4}$. Adding this
  result with the one obtained previously, we have $2[c_0 + c_4 + c_8 + \cdots] = 2^{n - 1} + 2^{\frac{n}{2}}\cos\frac{n\pi}{4}$.
\item $z^8 + 1 = 0 \Rightarrow z^8 = -1 = \cos\pi + i \sin\pi\therefore z = (\cos\pi + i \sin\pi)^{\frac{1}{8}} =
  \cos\frac{2r\pi + \pi}{8} + i \sin\frac{2r\pi + \pi}{8}, r = 0, 1, 2, \ldots, 7$

  $\therefore z = \cos\frac{\pi}{8} \pm \sin\frac{\pi}{8}, \cos\frac{3\pi}{8} \pm \sin\frac{3\pi}{8},
  \cos\frac{5\pi}{8} \pm \sin\frac{5\pi}{8}, \cos\frac{7\pi}{8} \pm \sin\frac{7\pi}{8}$

  Now, quadratic equation whose roots are $\cos\frac{\pi}{8} \pm \sin\frac{\pi}{8},$ is $z^2 -
  2\cos\frac{\pi}{8}z + 1 = 0$

  Similarly, we can find the quadratic equations for remaining three pairs of roots. Thus,

  $z^8 + 1 = \left(z^2 - 2\cos\frac{\pi}{8}z + 1\right)\left(z^2 - 2\cos\frac{3\pi}{8}z + 1\right)\left(z^2
  - 2\cos\frac{5\pi}{8}z + 1\right)\left(z^2 - 2\cos\frac{7\pi}{8}z + 1\right)$

  Dividing both sides by $z^4$, we get

  $z^4 + \frac{1}{z^4} = \left(z + \frac{1}{z} - 2\cos\frac{\pi}{8}\right)\left(z + \frac{1}{z} -
  2\cos\frac{3\pi}{8}\right)\left(z + \frac{1}{z} - 2\cos\frac{5\pi}{8}\right)\left(z + \frac{1}{z} -
  2\cos\frac{7\pi}{8}\right)$

  Putting $z = \cos\theta + i\sin\theta$, so that $z^n + \frac{1}{z^n} = 2n\cos n\theta$, we get

  $2\cos 4\theta = 2\left(\cos \theta - \cos\frac{\pi}{8}\right)2\left(\cos \theta -
  \cos\frac{3\pi}{8}\right)2\left(\cos \theta - \cos\frac{5\pi}{8}\right)2\left(\cos \theta -
  \cos\frac{5\pi}{8}\right)$

  $\therefore \cos 4\theta = 8\left(\cos \theta - \cos\frac{\pi}{8}\right)\left(\cos \theta -
  \cos\frac{3\pi}{8}\right)\left(\cos \theta - \cos\frac{5\pi}{8}\right)\left(\cos \theta -
  \cos\frac{7\pi}{8}\right)$
\item Let $z = \cos\theta + i \sin\theta$, then $z^7 = \cos 7\theta + i \sin 7\theta$. If

  $\theta = \frac{\pi}{7}, \frac{3\pi}{7}, \frac{5\pi}{7}, \frac{7\pi}{7}, \frac{9\pi}{7}, \frac{11\pi}{7},
  \frac{13\pi}{7}$ then $z^7 = \cos 7\theta + i \sin 7\theta = 1$ or $z^7 + 1 =0$

  Thus, $z = \cos\theta + i \sin\theta$, where $\theta = \frac{\pi}{7}, \frac{3\pi}{7}, \frac{5\pi}{7},
  \frac{7\pi}{7}, \frac{9\pi}{7}, \frac{11\pi}{7}, \frac{13\pi}{7}$ are the roots of the equation.

  Also, when $\theta = \pi, z = -1$. Now, $z^7 + 1 = 0 \Rightarrow (z + 1)(z^6 - z^5 + z^4 - z^3 + z^2 - z +
  1) = 0$

  Root of equation $z + 1 = 0$ is $\cos \theta + i \sin \theta$, where $\theta = \pi$

  Roots of equation $z^6 - z^5 + z^4 - z^3 + z^2 - z + 1 = 0\;\;\;\;(1)$

  are $\cos \theta + i \sin \theta,$ where $\theta = \frac{\pi}{7}, \frac{3\pi}{7}, \frac{5\pi}{7},
  \frac{7\pi}{7}, \frac{9\pi}{7}, \frac{11\pi}{7}, \frac{13\pi}{7}$

  Let $x = \cos \theta$, then $z + \frac{1}{z} = \cos \theta + i \sin \theta + \frac{1}{\cos \theta +
    i \sin \theta} = 2\cos\theta = 2x$

  But $\cos\left(\frac{13\pi}{7}\right) = cos\left(2\pi - \frac{\pi}{7}\right) = \cos\frac{\pi}{7},
  \cos\frac{11\pi}{7} = \cos\frac{3\pi}{7}, \cos\frac{9\pi}{7} = \cos\frac{5\pi}{7}$

  Dividing (1) by $z^3$, we get $z^3 - z^2 + z - 1 + \frac{1}{z} - \frac{1}{z^2} + \frac{1}{z^3} = 0$

  $\left(z^3 + \frac{1}{z^3}\right) - \left(z^2 + \frac{1}{z^2}\right) + \left(z + \frac{1}{z}\right) - 1 =
  0$

  $\left(z + \frac{1}{z}\right)^3 - 3z.\frac{1}{z}\left(z + \frac{1}{z}\right) - \left[\left(z +
    \frac{1}{z}\right)^2 - 2z.\frac{1}{z}\right] + z + \frac{1}{z} - 1 = 0$

  $\Rightarrow 8x^3 - 4x^2 -4x + 1 = 0$. Roots of this equation are $\cos \frac{\pi}{7}, \cos
  \frac{3\pi}{7}$ and $\cos \frac{5\pi}{7}$.
\item Given, $z^{10} - 1 = 0 \Rightarrow z^{10} = 1 = \cos 0 + i \sin 0\therefore z = (\cos 0 + i \sin
  0)^{\frac{1}{10}} = \cos\frac{2r\pi}{10} + i \sin\frac{2r\pi}{10}$

  $= \pm 1, \cos\frac{\pi}{5} \pm i\sin\frac{\pi}{5}, \cos\frac{2\pi}{5} \pm i\sin\frac{2\pi}{5},
  \cos\frac{3\pi}{5} \pm i\sin\frac{3\pi}{5}, \cos\frac{4\pi}{5} \pm i\sin\frac{4\pi}{5}$

  Quadratic equation whose roots are $\pm 1$ is $z^2 - 1 = 0$. And quadratic equation whose roots are
  $\cos\frac{\pi}{5} \pm \sin\frac{\pi}{5}$ is $z^2 - 2\cos\frac{\pi}{5}z + 1 = 0$. Thus,

  $z^{10} - 1 = (z^2 - 1)\left(z^2 - 2\cos\frac{\pi}{5}z + 1\right)\left(z^2 - 2\cos\frac{2\pi}{5}z +
  1\right)\left(z^2 - 2\cos\frac{3\pi}{5}z + 1\right)\left(z^2 - 2\cos\frac{4\pi}{5}z + 1\right)$

  Dividing both sides by $z^5$, we get

  $z^5 - \frac{1}{z^5} = \left(z - \frac{1}{z}\right)\left(z + \frac{1}{z}
  - 2\cos\frac{\pi}{5}\right)\left(z + \frac{1}{z}
  - 2\cos\frac{2\pi}{5}\right)\left(z + \frac{1}{z}
  - 2\cos\frac{3\pi}{5}\right)\left(z + \frac{1}{z}
  - 2\cos\frac{4\pi}{5}\right)$

  Putting $z = \cos \theta + i \sin \theta$ in the above equation, so
  that $z^5 - \frac{1}{z^5} = 2i\sin 5\theta$, we get

  $2i\sin 5\theta = 2i\sin\theta.2\left(\cos \theta -
  \cos\frac{\pi}{5}\right)2\left(\cos \theta -
  \cos\frac{2\pi}{5}\right)2\left(\cos \theta -
  \cos\frac{3\pi}{5}\right)2\left(\cos \theta - \cos\frac{4\pi}{5}\right)$

  $\therefore \sin 5\theta = 16 \sin \theta\left(\cos \theta -
  \cos\frac{\pi}{5}\right)\left(\cos \theta -
  \cos\frac{2\pi}{5}\right)\left(\cos \theta -
  \cos\frac{3\pi}{5}\right)\left(\cos \theta - \cos\frac{4\pi}{5}\right)$

  $= 16\sin \theta\left(\cos\theta
  -\cos\frac{\pi}{5}\right)\left(\cos\theta
  +\cos\frac{\pi}{5}\right)\left(\cos\theta
  -\cos\frac{2\pi}{5}\right)\left(\cos\theta +\cos\frac{2\pi}{5}\right)$

  $= 16\sin \theta\left(\cos^2\theta -
  \cos^2\frac{\pi}{5}\right)\left(\cos^2\theta -
  \cos^2\frac{2\pi}{5}\right)$

  $= 16\sin \theta\left(\sin^2\frac{\pi}{5} -
  \sin^2\theta\right)\left(\sin^2\frac{2\pi}{5} - \sin^2\theta\right)$

  $= 16\sin\theta\sin^2\frac{\pi}{5}\sin^2\frac{2\pi}{5}\left(1 -
  \frac{\sin^2\theta}{\sin^2\frac{\pi}{5}}\right)\left(1 -
  \frac{\sin^2\theta}{\sin^2\frac{2\pi}{5}}\right)$

  $= 16\sin\theta\sin^2{36^\circ}\sin^2{72^\circ}\left(1 -
  \frac{\sin^2\theta}{\sin^2\frac{\pi}{5}}\right)\left(1 -
  \frac{\sin^2\theta}{\sin^2\frac{2\pi}{5}}\right)$

  $= 16\sin\theta\left(\frac{\sqrt{10 - 2\sqrt{5}}}{4}\right)^2\left(\frac{\sqrt{10 +
      2\sqrt{5}}}{4}\right)^2\left(1 - \frac{\sin^2\theta}{\sin^2\frac{\pi}{5}}\right)\left(1 -
  \frac{\sin^2\theta}{\sin^2\frac{2\pi}{5}}\right)$

  Thus, $\sin5\theta = 5\sin\theta\left(1 - \frac{\sin^2\theta}{\sin^2\frac{\pi}{5}}\right)\left(1 -
  \frac{\sin^\theta}{\sin^2\frac{2\pi}{5}}\right)$.
\item Given, $x^7 + 1 = 0$ or $x^7 = -1 = \cos\pi + i \sin\pi$

  $\therefore x = \left(\cos\pi + i \sin\pi\right)^{\frac{1}{7}} =
  \cos\frac{2r\pi + \pi}{7} + i \sin \frac{2r\pi + \pi}{7}, r = 0, 1, 2,
  \ldots, 6$

  $= \cos \frac{\pi}{7} \pm i \sin\frac{\pi}{7}, \cos \frac{2\pi}{7} \pm i
  \sin\frac{2\pi}{7}, \cos \frac{3\pi}{7} \pm i \sin\frac{3\pi}{7}, \cos
  \pi + i \sin \pi(= -1)$

  $x^7 + 1 = (x + 1)\left(x^2 - 2\cos\frac{\pi}{7}x + 1\right)\left(x^2 -
  2\cos\frac{2\pi}{7}x + 1\right)\left(x^2 - 2\cos\frac{3\pi}{7}x +
  1\right)$. Putting $x = i$, we get

  $i^7 + 1 = (1 +
  i)\left(-2i\cos\frac{\pi}{7}\right)\left(-2i\cos\frac{2\pi}{7}\right)\left(-2i\cos\frac{3\pi}{7}\right)$

  $1 - i = 8(1 + i)\cos\frac{\pi}{7}\cos\frac{2\pi}{7}\cos\frac{3\pi}{7} = -8(1 -
  i)\cos\frac{\pi}{7}\cos\frac{2\pi}{7}\cos\frac{3\pi}{7}$

  $\therefore \cos\frac{\pi}{7}\cos\frac{2\pi}{7}\cos\frac{3\pi}{7} = -\frac{1}{8}$.
\item $(\cos\alpha + i \sin\alpha)^n = \cos^n\alpha + i.{{}^nC_1}\cos^{n-1}\alpha
  \sin\alpha + i^2.{^nC_2}\cos^{n-2}\alpha \sin^2\alpha + \cdots +
  i^n.{^nCn}\sin^n\alpha$

  $\Rightarrow \cos n\alpha + i \sin n\alpha = (\cos^n\alpha -
  {^nC_2}\cos^{n-2}\alpha \sin^2\alpha) + i({^nC_1}\cos^{n-1}\alpha
  \sin\alpha)$. Equating imaginary parts, we get

  $\therefore \sin n\alpha = {^nC_1}\cos^{n-1}\alpha \sin\alpha - {}^nC_3\cos^{n-3}\alpha \sin^3\alpha +
  \cdots$

  $\therefore \sin (2n+1)\alpha = {}^{2n+1}C_1\cos^{2n}\alpha \sin\alpha - {}^{2n+1}C_3\cos^{2n-2}\alpha
  \sin^3\alpha + \cdots$

  $\Rightarrow \sin (2n+1)\alpha = \sin^{2n+1}\alpha[{^{2n+1}C_1\cot^{2n}}\alpha -
    {}^{2n+1}C_3\cot^{2n-2}\alpha + \cdots]$

  $\text{when} \alpha = \frac{\pi}{2n+1}, \frac{2\pi}{2n+1}, \cdots, \frac{n\pi}{2n+1}, \sin(2n+1)\alpha =
  0$

  $\therefore \cot^2\frac{\pi}{2n+1}, \cot^2\frac{2\pi}{2n+1}, \cdots, \cot^2\frac{n\pi}{2n+1}$ are the
  roots of the equation. From the second term coefficient we get sum of roots in a polynomial.

  $\therefore \cot^2\frac{\pi}{2n+1}+ \cot^2\frac{2\pi}{2n+1}+ \cdots+ \cot^2\frac{n\pi}{2n+1} =
  \frac{{}^{2n+1}C_3}{{}^{2n+1}C_1}$.
\item Let $C = \cos\theta \cos\theta + \cos^2\theta \cos 2\theta + \cdots + cos^n\theta \cos n\theta$
  and

  $S = \cos\theta \sin\theta + \cos^2\theta \sin 2\theta + \cdots + cos^n\theta \sin n\theta$

  Now, $C + iS = \cos\theta(cos\theta + i \sin\theta) + \cos^2\theta(\cos2\theta + i \sin 2\theta) + \cdots
  + \cos^n\theta(\cos n\theta + i \sin n\theta)$

  $= \cos\theta.e^{i\theta} + \cos^2\theta.e^{2i\theta} + \cdots + \cos^n\theta.e^{ni\theta}$
  $= x + x^2 + \cdots + x^n$, where $x = \cos\theta e^{i\theta}= \frac{x(x^n - 1)}{x - 1} = \frac{\cos\theta
    e^{i\theta}(\cos^n\theta e^{in\theta} - 1)}{\cos\theta e^{i\theta} - 1}$

  $= \frac{\cos\theta[\cos^n\theta(\cos n\theta + i \sin n\theta) - 1]}{\cos\theta - e^{-i\theta}} =
  \frac{\cos\theta[(\cos^n\theta\cos n\theta -1) + i cos^n\theta\sin n\theta]}{\cos\theta - (\cos\theta
    - i\sin\theta)}$

  $= -i \cot\theta(\cos^n\theta \cos n\theta - 1) + i \cos^n\theta \sin n\theta$

  Equating imaginary parts, we get

  $S = -\cot\theta(\cos^n\theta \cos n\theta - 1) = \cot\theta(1 - \cos^n\theta\cos n\theta)$.
\item L.H.S. $= -3 -4i = 5\left(-\frac{3}{5} - i\frac{4}{5}\right) = 5\left(\cos\left(\pi +
  \tan^{-1}\frac{4}{5}\right) +i \sin\left(\pi + \tan^{-1}\frac{4}{5}\right)\right)$

  $= 5e^{i\left(\pi + \tan^{-1}\frac{4}{5}\right)} =$ R.H.S.
\item Putting $x^4 = \frac{\sqrt{3} - 1}{2\sqrt{2}} + i\frac{\sqrt{3} + 1}{2\sqrt{2}}$ in polar form we get

  $x^4 = \cos\frac{5\pi}{12} + i \sin\frac{5\pi}{12} \therefore x = \cos\frac{(24r + 5)\pi}{48} + i
  \sin\frac{(24r + 5)\pi}{48}, r = 0, 1, 2, 4$.
\item L.H.S. $= z_1z_2z_3\ldots = \left(\cos\frac{\pi}{3} +
  i\sin\frac{\pi}{3}\right)\left(\cos\frac{\pi}{3^2} + i\sin\frac{\pi}{3^2}\right)\left(\cos\frac{\pi}{3^3}
  + i\sin\frac{\pi}{3^3}\right)\cdots$

  $= \cos\left(\frac{\frac{\pi}{3}}{1 - \frac{1}{3}}\right) + i\sin\left(\frac{\frac{\pi}{3}}{1 -
    \frac{1}{3}}\right) = \cos\frac{\pi}{2} + i\sin\frac{\pi}{2} = i =$ R.H.S.
\item Given $p_0x^n + p_1x^{n - 1} + p_2x^{n - 2} + \cdots + p_n = 0$, prove that $p_1\sin\theta +
  p_2\sin2\theta + \cdots + p_n = 0 \Rightarrow p_0(\cos n\theta + i\sin n\theta) + p_1[\cos(n - 1)\theta +
    \sin(n - 1)\theta] + p_2[\cos(n - 2)\theta + i\sin(n - 2)\theta] + \cdots + p_n = 0\;[\because
    \cos\theta + i\sin\theta]$ is a solution.

  Dividing both sides by $\cos n\theta + i\sin n\theta$, we have

  $p_0 + p_1(\cos\theta - i\sin\theta) + p_2(\cos2\theta - i\sin2\theta) + \cdots + p_n(\cos n\theta - i\sin
  n\theta) = 0$. Equating real and imaghinary parts we have required equations.
\item L.H.S. $= \left(\frac{1 + \cos \phi + i \sin\phi}{1 + \cos \phi - i\sin\phi}\right)^n = \left(\frac{(1
  + \cos\phi + i \sin\phi)(1 + \cos\phi + i\sin\phi)}{(1 + \cos\phi)^2 + \sin^2\phi}\right)^n$

  $= \left(\frac{1+ 2\cos\phi + \cos^2\phi - \sin^2\phi + 2i \sin\phi(1 + \cos\phi)}{1 + 2\cos\phi +
  \cos^2\phi + \sin^2\phi}\right)^n= \left(\frac{2(1 + \cos\phi) + 2i \sin\phi(1 +
  \cos\phi)}{2\cos\phi(1 + \cos\phi)}\right)^n$

  $= (\cos\phi + i \sin\phi^n) = \cos n\phi + i \sin n\phi =$ R.H.S.
\item Given $2\cos\theta = x + \frac{1}{x} \Rightarrow x^2 - 2\cos\theta x + 1 = 0 \Rightarrow x =
  \cos\theta\pm i\sin\theta$. Similarly, $y = \cos\phi\pm i\sin\phi$.
  \startitemize[i]
  \item $\frac{x}{y} = \cos(\theta - \phi) \pm i\sin(\theta - \phi)$ and $\frac{y}{x} = \cos(\phi -
    \theta)\pm i\sin(\phi - \theta)$

    $\therefore$ L.H.S. $= 2\cos(\theta - \phi) =$ R.H.S. $[\because \cos(-\theta) = \cos\theta,
      \sin(-\theta) = -\sin\theta]$
  \item $xy = \cos(\theta + \phi) \pm i\sin(\theta + \phi), \frac{1}{xy} = \cos(\theta + \phi)\mp
    i\sin(\theta + \phi)$

    $\therefore$ L.H.S. $= 2\cos(\theta + \phi) =$ R.H.S.
  \item $x^my^n = (\cos m\theta \pm i\sin m\theta)(\cos n\phi \pm i\sin n\phi) = \cos(m\theta + n\phi)\pm
    i\sin(m\theta + n\phi)$ and $\frac{1}{x^my^n} = \cos(m\theta + n\phi)\mp i\sin(m\theta + n\phi)$
    $\therefore$ L.H.S. $= 2\cos(m\theta + n\phi) =$ R.H.S.
  \item $\frac{x^m}{y^n} = \cos(m\theta - n\phi)\pm i\sin(m\theta - n\phi)$ and $\frac{y^n}{x^m} =
    \cos(n\phi - m\theta)\pm i\sin(n\phi - m\theta)$
    $\therefore$ L.H.S. $= 2\cos(m\theta - n\phi) =$ R.H.S.
  \stopitemize
\item Given equation is $x^2 - 2x + 4 = 0$ whose roots are $\alpha, \beta = 1\pm i\sqrt{3} =
  2\left(\cos\frac{\pi}{3}\pm i\sin\frac{\pi}{3}\right)\Rightarrow \alpha^n, \beta^n =
  2\left(\cos\frac{n\pi}{3}\pm i\sin\frac{n\pi}{3}\right)$

  $\therefore \alpha^n + \beta^n = 2^{n + 1}\cos\frac{n\pi}{3} =$ R.H.S.
\item Given equation is $x^2 - 2x\cos\theta + 1 = 0$, whose roots are $\cos\theta \pm i\sin\theta$, $n$th
  power of which are $\cos n\theta \pm i\sin n\theta$. Therefore, the equation having these roots are $x^2 -
  2\cos n\theta + 1 = 0$.
\item L.H.S. $= A(\cos2\theta + i\sin2\theta) + B(\cos2\theta - i\sin2\theta) = 5\cos2\theta +
  7i^2\sin2\theta$.

  $\Rightarrow A + B = 5, A - B = 7i \Rightarrow A = \frac{5 + 7i}{2}, B = \frac{5 - 7i}{2}$.
\item Given $x = \cos\theta + i\sin\theta$ and $\sqrt{1 - c^2} = nc - 1$. Squaring the second equaiton
  $n^2c^2 + c^2 - 2nc  = 0 \Rightarrow c = \frac{2n}{n^2 + 1}$. We have to prove that $1 + \cos\theta =
  \frac{c}{2n}(1 + nx)\left(1 + \frac{n}{x}\right)$.

  R.H.S. $= \frac{1}{n^2 + 1}\left(1 + n^2 + 2n\cos\theta\right) = 1 + \frac{2n}{n^2 + 1}\cos\theta = 1 +
  c\cos\theta =$ L.H.S.
\item From the given equality, we have $\left(\frac{1+z}{1-z}\right)^n = 1\Rightarrow 1 + z = (1 -
  z)(\cos\frac{2r\pi}{n} +i \sin\frac{2r\pi}{n})$

  Let $\frac{2r\pi}{n} = \theta$ then $1 + z = (1 - z)(\cos\theta + i \sin\theta) \Rightarrow z((1 +
  \cos\theta) + i \sin\theta) = (\cos\theta - 1) + i \sin\theta
  \Rightarrow z = \frac{(\cos\theta - 1) + i \sin\theta}{(1 + \cos\theta) + i \sin\theta}$

  $z = i \tan\frac{\theta}{2} = i \tan\frac{2\pi}{n}, \; r = 0, 1, 2, ..., (n - 1)$

  Clearly, the above equation is invalid if $n$ is even and $r =
  \frac{n}{2}$ as it will cause the value of $\tan$ function to reach infinity.
\item L.H.S. $= \frac{xy(x + y) - (x + y)}{xy(x - y)+(x - y)}$. Dividing numerator and denominator by $xy$

  $= \frac{x + y - \frac{1}{x} - \frac{1}{y}}{x - y + \frac{1}{y} - \frac{1}{x}}$
  $= \frac{\cos\alpha + i \sin\alpha + \cos\beta + i \sin\beta -
  \cos\alpha + i\sin\alpha - \cos\beta + i \sin\beta}{\cos\alpha + i
  \sin\alpha - \cos\beta - i \sin\beta - \cos\alpha + i\sin\alpha +
  \cos\beta - i \sin\beta}$
  $= \frac{\sin\alpha + \sin\beta}{sin\alpha - \sin\beta} =$ R.H.S.
\item $(1 + x)^n = {^nC_0} + {^nC_1}x + {^nC_3}x^2 + {^nC_3}x^3 + \cdots$

  We know that $\omega, \omega^2 = \frac{-1\pm\sqrt{3}i}{2} = -cos\frac{\pi}{3} \pm \sin\frac{\pi}{3}$.

  Putting $x = 1, \omega, \omega^2$ and adding we get

  $2^n + 2\cos\frac{n\pi}{3} = 3[{}^nC_0 + {}^nC_3 + {}^nC_6 + \cdots] \Rightarrow {}^nC_0 + {}^nC_3 +
  {}^nC_6 + \cdots = \frac{1}{3}\left(2^n + 2\cos\frac{n\pi}{3}\right)$.
\item Proceeding like previous question,

  $2^n = {}^nC_0 + {}^nC_1 + {}^nC_2 + {}^nC_3 + {}^nC_4 + {}^nC_5 + \cdots$

  $(-\omega^2)^n = {}^nC_0 + {}^nC_1\omega + {}^nC_2\omega^2 + {}^nC_3\omega^3 + {}^nC_4\omega^4 +
  {}^nC_5\omega^5 + \cdots$

  $\Rightarrow (-\omega^2)^{n}\omega^2 = {}^nC_0\omega^2 + {}^nC_1\omega^3 + {}^nC_2\omega^4 + {}^nC_3\omega^5 + {}^nC_4\omega^6 +
  {}^nC_5\omega^7 + \cdots$

  and $(-\omega)^n = {}^nC_0 + {}^nC_1\omega^2 + {}^nC_2\omega^4 + {}^nC_3\omega^6 + {}^nC_4\omega^8 +
  {}^nC_5\omega^{10} + \cdots$

  $\Rightarrow (-\omega)^n\omega = {}^nC_0\omega + {}^nC_1\omega^3 + {}^nC_2\omega^5 + {}^nC_3\omega^7 + {}^nC_4\omega^9 +
  {}^nC_5\omega^{11} + \cdots$

  Adding $2^{n - 2} + 2\cos\frac{(n- 2)\pi}{3} = 3[{}^nC_1 + {}^nC_4 + {}^nC_7 + \cdots] \Rightarrow {}^nC_1
  + {}^nC_4 + {}^nC_7 + \cdots = \frac{1}{3}\left[2^{n - 2} + 2\cos\frac{(n - 2)\pi}{3}\right]$
\item This problem can be solved like previous problem. Put $x = 1, \omega, \omega^2$ and multiply with $1,
  \omega, \omega^2$ and then add to obtain the result.
\item $C_0 + C_1x + C_2x^2 + C_3x^3 + C_4x^4 + \cdots = (1 + x)^{4n}$. Putting $x = 1, -1, i, -i$ and adding

  $4[C_0 + C_4 + C_8 + \cdots] = 2^{4n} + (1 + i)^{4n} + (1 - i)^{4n}$

  $\Rightarrow C_0 + C_4 + C_8 + \cdots = 2^{4n - 2} + (-1)^n2^{2n - 1}$.
\item Given $(1 - x + x^2)^{6n} = a_0 + a_1x + a_2x^2 + \cdots$. Putting $x = 1, \omega, \omega^2$

  $1^{6n} = a_0 + a_1 + a_2 + a_3 + \cdots$

  $(-2\omega)^{6n}= 2^{6n} = a_0 + a_1\omega + a_2\omega^2 + a_3\omega^3 + \cdots$

  $(-2\omega^2)^{6n}= 2^{6n} = a_0 + a_1\omega^2 + _2\omega^4 + a_3\omega^6 + \cdots$

  Adding $2^{6n + 1} + 1 = 3[a_0 + a_3 + a_6 + \cdots]$
  $\Rightarrow a_0 + a_3 + a_6 + \cdots = \frac{1}{3}[2^{6n + 1} + 1]$.
\item Proceeding like previous problem we obtain $3[a_0 + a_3 + a_6 + \cdots]$.

  R.H.S. becomes $1^n + (-2\omega)^n + (-2\omega^2)^n$ but $-\omega = \cos\frac{\pi}{3} +
  i\sin\frac{\pi}{3}$ and $-\omega^2 = \cos\frac{\pi}{3} - i\sin\frac{\pi}{3}$ and hence we have R.H.S.
\item Clearly, $x'' = \frac{AA' + BB' + CC'}{3}, y'' = \frac{AA' + BB'\omega^2 + CC'\omega}{3}$ and $z'' =
  \frac{AA' + BB'\omega + CC'\omega^2}{3}$,

  and $AA' + BB' + CC' = (x + y + z)(x' + y' + z') + (x + y\omega + z\omega^2)((x' + y'\omega + z'\omega^2)
  + (x + y\omega^2 + z\omega)((x' + y'\omega^2 + z'\omega) = 3(xx' + zy' + yz')$. Analogously $y'' = yy' +
  zx' + xz', z'' = zz' + xy' + yz'$.
\item We have the identity $(\alpha\delta - \beta\gamma)(\alpha'\delta' - \beta'\gamma') = (\alpha\alpha' +
  \beta\gamma')(\gamma\beta' + \delta\delta') - (\alpha\beta' + \beta\delta')(\gamma\alpha' + \gamma\alpha'
  + \delta\gamma')$

  Putting $\alpha = x + yi, \beta = z + ti, \gamma = -(z - ti), \delta = x - yi, \alpha' = a + bi, \beta' =
  c + di, \gamma' = -(c - di)$ and $\delta' = a - bi$ then

  $\alpha\delta - \beta\gamma = x^2 + y^2 + z^2 + t^2$ and $\alpha'\delta' - \beta'\gamma' = a^2 + b^2 + c^2
  + d^2$

  $\Rightarrow \alpha\alpha' + \beta\gamma' = (ax - by - ca - dt) + i(bx + ay + dz - ct), \gamma\beta' +
  \delta\delta' = \overline{\beta\gamma'} + \overline{\alpha\alpha'} = \overline{\beta\gamma' +
    \alpha\alpha'}$

  $\therefore \alpha\beta' + \beta\delta' = (cx - dy + az + bt) + i(dx + cy - bz + at), \gamma\alpha' +
  \delta\gamma' = -(cx - dy + az + bt) + i(dx + cy - bz + at)$

  Thus, $-(\alpha\beta' + \beta\delta')(\gamma\alpha' + \delta\gamma') = (cx - dy + az + bt)^2 + (dx + cy -
  bz + at)^2$

  Substituting obtained expression in the original idendity we have the required result.
\item $(\cos\theta + i\sin\theta)^n = \cos^n\theta + iC_1^^n\cos^{n - 1}\theta\sin\theta +
  i^2C_2^^n\cos^{(n - 2)}\theta\sin^2\theta + \cdots + i^rC_r^^n\cos{(n - r + 1)}\theta\sin^{r - 1}\theta +
  \cdots$

  Separating real part, $\cos n\theta = \cos^n\theta - C_2^^n\cos^{n - 2}\theta\sin^2\theta +
  \cdots$

  Taking into account the parity of $n$ and dividing both members of these equalities by $\cos^n\theta$, we
  get the required formulas.
\item Replacing real part with imaginary part in previous problem we arrive at required formula.
\item $\cos\theta = \frac{(\cos\theta + i\sin\theta) + (\cos\theta - i\sin\theta)}{2}$. Let $\cos\theta
  + i\sin\theta = z$ then $\cos\theta - i\sin\theta = z^{-1} $.

  $\therefore \cos^{2m}\theta = \left(\frac{z + z^{-1}}{2}\right)^{2m} =
  \frac{1}{2^{2m}}\displaystyle\sum_{k=0}^{2m}C_k^^{2m}z^{2m - k}.z^{-k}$

  Moreover $2^{2m}\cos^{2m}\theta = \displaystyle\sum_{k = 0}^{m - 1}C_k^^{2m}z^{2(m - k)} + C_m^^{2m} +
  \sum_{k = m + 1}^{2m}C_{k}^^{2m}z^{2(m - k)}$

  Putting $m - k = -(m - k')$, we rewrite the sum $\displaystyle\sum_{k' = m - 1}^0C_{2m - k'}^^{2m}z^{-2(m
    - k')} = \sum_{k = 0}^{m - 1}C_{k}^^{2m}z^{-2(m - k)}$

  And so $2^{2m}\cos^{2m}\theta = \sum_{k = 0}^{m - 1}C_k^^{2m}(z^{2(m - k)} + z^{-2(m - k)}) + C_m^^{2m}$.

  However, $z^{2(m - k)} + z^{-2(m - k)} = 2\cos2(m - k)$.

  $\therefore\displaystyle 2^{2m}\cos^{2m}\theta = \sum_{k = 0}^{m - 1}2\binom{2m}{k}\cos2(m - k)x +
  \binom{2m}{m}$.
\item Putting $\theta = \frac{\pi}{2} - \theta$ in the previous problem, we get the required formula.
\item This is deduced like previous problem.
\item This is deduced like previous problem.
\item We have the expression $u_n + iv_n = (\cos\alpha + i\sin\alpha) + r[\cos(\alpha + \theta) +
  i\sin(\alpha + \theta)] + \cdots + r^n[\cos(\alpha + n\theta) + i\sin(\alpha + n\theta)]$

  $= (\cos\alpha + i\sin\alpha)[1 + (\cos\theta + i\sin\theta) + \cdots + r^n(\cos n\theta + i\sin
  n\theta)]$. Putting $z = \cos\theta + i\sin\theta$, then

  $u_n + iv_n = (\cos\alpha + i\sin\alpha)[1 + rz + \cdots + r^nz^n] = e^{i\alpha}\frac{(rz)^{n + 1} - 1}{rz
  - 1}$

  Transforming $\frac{(rz)^{n + 1} - 1}{rz - 1}$, separating real part from the imaginary one.

  $\frac{(rz)^{n + 1} - 1}{rz - 1} = \frac{[(rz)^{n + 1} - 1][\overline{rz} - 1]}{(rz - 1)(\overline{rz} -
    1)}$

  $= \frac{r^{n + 2}[\cos n\theta + i\sin n\theta] - r[\cos\theta - i\sin\theta]}{1 - 2r\cos\theta + r^2} +
  \frac{-r^{n + 1}[\cos(n + 1)\theta + i\sin(n + 1)\theta] + 1}{1 - 2r\cos\theta + r^2}$

  Multiplying above with $(\cos\alpha + i\sin\alpha)$ and separating real and imaginary parts we have

  $u_n + iv_n = \frac{\cos\alpha - r\cos(\alpha - \theta) - r^{n + 1}\cos[\alpha + (n + 1)\theta] + r^{n + 2}\cos(\alpha + n\theta)}{1 -
    2r\cos\theta + r^2} + $

  $i\frac{\sin\alpha - r\sin(\alpha - \theta) - r^{n + 1}\sin[\alpha + (n + 1)\theta] + r^{n +
      2}\sin(\alpha + n\theta)}{1 - 2r\cos\theta + r^2}$.

  {\bf Note:} Putting $\alpha = 0, r = 1$, we obtain $1 + \cos\theta + \cos2\theta + \cdots + \cos n\theta =
  \frac{\sin\frac{n + 1}{2}\theta\cos\frac{n\theta}{2}}{\sin\frac{\theta}{2}}$

  and $\sin\theta + \sin2\theta + \cdots + \sin n\theta = \frac{\sin\frac{n +
      1}{2}\theta\sin\frac{n\theta}{2}}{\sin\frac{\theta}{2}}$.
\item $S + iS' = \displaystyle\sum_{k=0}^nC_k^^n(\cos k\theta + i\sin k\theta) = \sum_{k = 0}^n(\cos\theta
  + i\sin\theta)^k$

  $= (1 + \cos\theta + i\sin\theta)^n= \left[2\cos^2\frac{\theta}{2} + 2i\sin\frac{\theta}{2}\cos\frac{\theta}{2}\right]^n =
  2^n\cos^n\frac{\theta}{2}\left(\cos\frac{\theta}{2} + i\sin\frac{\theta}{2}\right)^n$

  $= 2^n\cos^n\frac{\theta}{2}\left(\cos\frac{n\theta}{2} + i\sin\frac{n\theta}{2}\right)$.

  Equating real and imaginary parts we have $S$ and $S'$.
\item Put $S = \sin^{2p}\alpha + \sin^{2p}2\alpha + \cdots + \sin^{2p}2\alpha = \displaystyle\sum_{l =
  1}^n\sin^{2p}l\alpha$

  But we have proved earlier $\sin^{2p}l\alpha = \frac{1}{2^{2p - 1}})(-1)^p\displaystyle\sum_{k = 0}^{p -
    1}C_k^^{2p}\cos2(p - k)l\alpha + \frac{1}{2^{2p}}C_p^^{2p}$, therefore

  $S = \frac{(-1)^p}{2^{2p - 1}}\displaystyle\sum_{k = 0}^{p - 1}(-1)^kC_k^^{2p}\sum_{l = 1}^n\cos2(p - k)l\alpha +
  \frac{1}{2^{2p}}C_p^^{2p}$

  Put $2(p - k)\alpha = \theta, \sum_{l = 1}^n\cos2(p - k)\alpha = \cos\theta + \cdots + \cos n\theta =
  \frac{\sin\frac{n\theta}{2}\cos\frac{n + 1}{2}\theta}{\sin\frac{\theta}{2}}$

  Denoting $\frac{\sin\frac{n\theta}{2}\cos\frac{n + 1}{2}\theta}{\sin\frac{\theta}{2}} = \sigma_k$, we can
  prove that $\sigma_k = 0$ if $k$ is of the same parity as $p\{k\equiv p(\mod\;2)\}$ and $\sigma_k
  = -1$ if $k$ and $p$ are of different parity $\{k\equiv p + 1(\mod)\;2\}$, and we get

  $S = \frac{(-1)^{p + 1}}{2^{2p - 1}}\displaystyle\sum_{\mstack{k=0, k\equiv p + 1 (\mod 2)}}^{p -
    1}(-1)^kC_k^^{2p} + \frac{n}{2^{2p}}C_p^^{2p}$.

  Hence, $S = \frac{1}{2^{2p - 1}}\displaystyle\sum_{\mstack{k=0, k\equiv p + 1 (\mod 2)}}^{p - 1}C_k^^{2p} +
  \frac{n}{2^{2p}}C_p^^{2p}$.

  But we can prove that $\displaystyle\sum_{\mstack{k=0, k\equiv p + 1 (\mod 2)}}^{p - 1}C_k^^{2p} = 2^{2p -
    2}$ (check binomial theorem chapter) and hence our formula is deduced.
\item Considering the given expression as a polynomial in $y$ we see that at $y = 0$ the polynomial
  vanishes. Therefore, our polynomial is divisible by $y$. Since it is symmetrical both w.r.t. to $x$ and
  $y$ this must also be true for $x$ i.e. the polynomial being divisible by $x$. Hence, the polynomial is
  divisible by $xy$. Putting $y = -x$(we do this for checking divisibility by $x + y$), we have $(x - x)^n -
  x^n - (-x)^n = 0$. Consequently, the polynomial is divisible by $x + y$.

  Now it remains to prove that the polynomial is divisible by $x^2 + xy + y^2$. Expansind this into linear
  factors we have $x^2 + xy + y^2 = (y - x\omega)(y - x\omega^2)$ where $\omega$ is cube root of unity,
  which leads to $1 + \omega + \omega^2 = 0$.

  Since $n = 3m + 1, 3m + 2\;\forall\;m\in \mathbb{I}$, we substitute $y = x\omega$ and $y = x\omega^2$ and
  find that it vanishes for both. Consequently, we have proven the divisibility condition.
\item Let the quantities $-x, -y$ and $x + y$ be the roots of the cubic equation $x^3 - rx^2 - px - q =
  0$. Then $r = -x - y + x + y = 0, -p = xy - x(x + y) - y(x + y), q = xy(x + y)$ reducing our equation to
  $x^3 - px + q = 0$.

  Putting $(x + y)^n - x^n - y^n = S_n$ we find that between successive values of $S_n$ their exists
  relationship $S_{n + 3} = pS_{n + 1} + qS_n$. We will use mathematical induction to prove that $S_n$ is
  divisible by $p^2$ with the knowledge that $S_1 = 0$.

  Let $S_n$ be divisble by $p^2$ then let $S_{n + 6}$ be also divisible by $p^2$. We have, $S_{n + 6} = pS_{n +
    4} + qS_{n + 3}, S_{n + 4} = pS_{n + 2} + qS_{n + 1}$. Therefore,

  $S_{n + 6} = p(pS_{n + 2} + qS_{n + 1}) + q(pS_{n + 1} + qS_n) = p^2S_{n + 2} + 2pqS_{n + 1} + q^2S_n$.

  Since by supposition, $S_n$ is divisible by $p^2$, it suffices to prove that $S_{n + 1}$ is divisible by
  $p$. Thus, we only have to prove that given expression is divisible by $x^2 + xy + y^2$ if $n\equiv
  2(\mod 6)$, which can be proved by proceeding like previous problem.
\item Let $f(x) = (\cos\theta + x\sin\theta)^n - \cos n\theta - x\sin n\theta$. But $x^2 + 1 = (x + i)(x -
  i)$ and $f(i) = \cos n\theta + i\sin n\theta - \cos n\theta - i\sin n\theta = 0$. Similarly, $f(-i) =
  0$. And hence, required condition is proved.
\item Roots of the equation $x^2 - 2px\cos\theta + p^2 = 0$ are $p(\cos\theta\pm\sin\theta)$. Let $f(x) =
  x^n\sin\theta - p^{n - 1}x\sin n\theta + p^n\sin(n - 1)\theta$, then

  $f[p(\cos\theta + i\sin\theta)] = p^n(\cos n\theta + i\sin n\theta)\sin\theta - p^n(\cos\theta + i\sin
  \theta)\sin n\theta + p^n\sin(n - 1)\theta$. Separating real and imaginary parts

  $\cos n\theta\sin\theta - \cos\theta\sin n\theta + \sin(n - 1)\theta = -\sin(n - 1)\theta + \sin(n -
  1)\theta = 0$

  and $\sin\theta\sin n\theta - \sin\theta\sin n\theta = 0$. Hence, $f(x)$ is divisible by $p(\cos\theta +
  i\sin\theta)$ and similarly we can prove it for the other root.
\item Let $x^4 + 1 = (x^2 + px + q)(x^2 + p'x + q') = x^4 + (p + p')x^3 + (pp' + q + q')x^2 + (pq' + p'q)x +
  qq'$ which gives us four equations $p + p' = 0, pp' + q + q' = 0, pq' + p'q = 0$ and $qq' = 1$.

  Assuming $p = 0, p' = 0, q + q' = 0, qq' = 1, q^2 = -1, q = \pm i, q' = \mp i$.

  Consequently, corresponding factorization has form $x^4 + 1 = (x^2 + i)(x^2 - i)$.

  Let $q = q', q^2 = 1, q = \pm 1$. First let $q = q' = 1$. Then $pp' = -2, p + p' = 0, p^2 = 2, p =
  \pm\sqrt{2}, p' = \mp\sqrt{2}$. The corresponding factorization is $x^4 + 1 = (x^2 - \sqrt{2}x + 1)(x^2 +
  \sqrt{2}x + 1)$.

  Then we assume $q = q' = -1, p + p' = 0, pp' = 2, p = \pm \sqrt{2}i, p' = \mp\sqrt{2}i$.

  The factorization will be $(x^2 + \sqrt{2}ix - 1)(x^2 - \sqrt{2}ix - 1)$.
\item Let $S = \displaystyle\sum_{k = 1}^{n - 1}x_k^p = \sum_{k = 1}^{n - 1}z^{kp}$ where $z =
  \cos\frac{2\pi}{n} + i\sin\frac{2\pi}{n}$.

  Thus, $\displaystyle\sum_{k = 1}^{n - 1}x_k^p = 1 + z^p + z^{2p} + \cdots + z^{(n - 1)p}$ but $z^p =
  \cos\frac{2p\pi}{n} + i\sin\frac{2p\pi}{n}$. Obviously $z^p = 1$ if and only if $p$ is divisible by $n$,
  in which case $S = n$. If $z^p\neq 1$, then $S = \frac{z^{np - 1}}{z^p - 1} = 0\because z^{np} = 1$.
\item We have $\displaystyle\sum_{k=0}^{n - 1}|A_k|^2 = \sum_{k=0}^{n - 1}A_k\overline{A_k}$.

  But $A_k\overline{A_k} = (x + y\epsilon^k + z\epsilon^{2k} + \ldots + w\epsilon^{(n - 1)k})(\overline{x} +
  \overline{y}\epsilon^{-k} + \overline{z}\epsilon^{-2k} + \cdots + \overline{w}\epsilon^{-(n - 1)k})$

  $= (x\overline{x} + y\overline{y} + \cdots + w\overline{w})+ x(\overline{y}\epsilon^{-k} +
  \overline{z}\epsilon^{-2k} + \cdots + \overline{w}\epsilon^{-(n - 1)k}) + y\epsilon^k(\overline{x} +
  \overline{x}\epsilon^{-2k} + \cdots + \overline{w}\epsilon^{-(n - 1)k}) + \cdots + w\epsilon^{(n -
    1)k}(\overline{x} + \overline{y}\epsilon^{-k} + \cdots + \overline{u}\epsilon^{-(n - 2)k})$

  Therefore, $\displaystyle\sum_{k=0}^{n - 1}|A_k|^2 = n(|x|^2 + |y|^2 + \cdots + |w|^2) + x\sum_{k=1}^{n -
    1}(\overline{y}\epsilon^{-k} + \overline{z}\epsilon^{-2k} + \cdots + \overline{w}\epsilon^{-(n - 1)k}) +
  y\sum_{k=0}^{n - 1}(\overline{x}\epsilon^k + \overline{z}\epsilon^{-k} + \cdots +
  \overline{w}\epsilon^{-(n - 2)k}) + \cdots + w\sum_{k = 0}^{n - 1}(\overline{x}\epsilon^{(n - 1)k}+
  \overline{y}\epsilon^{(n - 2)k} + \cdots + \overline{u}\epsilon^k)$

  But $\displaystyle\sum_{k=0}^{n - 1}\epsilon^{lk} = 0$ if $l$ is not divisible by $n$ from previous
  problem. Therefore all the sums in the right vanish and we get

  $\displaystyle\sum_{k = 0}^{n - 1}|A_k|^2 = n(|x|^2 + |y|^2 + \ldots + |w|^2)$.
\item Considering $2n$th root of unity $x_s = \cos\frac{2s\pi}{n} + i\sin\frac{2s\pi}{n}\;\;(s = 1, 2, 3,
  \ldots, n)$.

  Therefore, $x^{2n} - 1 = \displaystyle\prod_{s=1}^{2n}(x - x_s) = \prod_{s=1}^{n - 1}(x - x_s)
  \prod_{s=n + 1}^{2n - 1}(x - x_s)(x^2 - 1)\;\because x_n = -1, x_{2n} = 1$. But $x_{2n - s} =
  \overline{x_s}$, consequently,

  $x^{2n} - 1 = (x^2 - 1)\displaystyle\prod_{s = 1}^{n - 1}(x - x_s)(x - \overline{x_s}) = (x^2 - 1)\prod_{s
    = 1}^{n - 1}(x^2 - 2x\cos\frac{s\pi}{n} + 1)$.
\item Considering $2n + 1$th root of unity $x_s = \cos\frac{2(2s + 1)\pi}{2n + 1} + i\sin\frac{(2s +
  1)\pi}{2n + 1}\;\;(s = 1, 2, 3, \ldots, n)$.

  Therefore $x^{2n + 1} - 1 = \displaystyle\prod_{s = 1}^{2n + 1}(x - x_s)$. However, $x_{2n + 1} = 1$,
  therefore

  $x^{2n + 1} - 1 = \displaystyle(x - 1)\prod_{s=1}^{2n}(x - x_s)$, but $x_{2n - s} = \displaystyle\overline{x_s}
  \Rightarrow x^{2n + 1} - 1 = (x - 1)\prod_{x=1}^{n}(x - x_s)(x - \overline{x_s}) = (x +
  1)\displaystyle\prod_{k = 1}^n\left(x^2 - 2x\cos\frac{2k\pi}{2n + 1} + 1\right)$.
\item Considering $2n + 1$th root of $-1, x_s = -\cos\frac{2(2s + 1)\pi}{2n + 1} + i\sin\frac{(2s +
  1)\pi}{2n + 1}\;\;(s = 1, 2, 3, \ldots, n)$.
  Therefore $x^{2n + 1} + 1 = \displaystyle\prod_{s = 1}^{2n + 1}(x - x_s)$. However, $x_{2n + 1} = -1$,
  therefore

  $x^{2n + 1} + 1 = \displaystyle(x + 1)\prod_{s=1}^{2n}(x - x_s)$, but $x_{2n - s} = \displaystyle\overline{x_s}
  \Rightarrow x^{2n + 1} + 1 = (x + 1)\prod_{x=1}^{n}(x - x_s)(x - \overline{x_s}) = (x +
  1)\displaystyle\prod_{k = 1}^n\left(x^2 + 2x\cos\frac{2k\pi}{2n + 1} + 1\right)$.
\item This problem can be solved like previous problem.
\item We have proven that $x^{2n} - 1 = (x^2 - 1)\displaystyle\prod_{k = 1}^{n - 1}\left(x^2 -
  2x\cos\frac{k\pi}{n} + 1\right)$

  $\Rightarrow x^{2n - 2} + x^{2n - 4} + \cdots + x^2 + 1 = \displaystyle\prod_{k = 1}^{n - 1}\left(x^2 -
  2x\cos\frac{k\pi}{n} + 1\right)$

  Putting $x = 1$, we have $n = \displaystyle\prod_{k = 1}^{n - 1}\left(2 - 2\cos\frac{k\pi}{n}\right) = \prod_{k=1}^{n
  - 1}4\sin^2\frac{k\pi}{2n} = 2^{2(n - 1)}\sin^2\frac{\pi}{2n}\sin^2\frac{2\pi}{2n}\cdots\sin^2\frac{(n -
    1)\pi}{2n}$

  $\Rightarrow \sin\frac{\pi}{2n}\sin\frac{2\pi}{2n}\ldots \sin\frac{(n - 1)\pi}{2n} = \frac{\sqrt{n}}{2^{n
      - 1}}$.
\item This problem can be solved like previous problem.
\item Since $\cos\alpha + i\sin\alpha$ is the root of the given equation, we have $\displaystyle\sum_{k =
  0}^np_k(\cos\alpha + i\sin\alpha)^{n - k} = 0\;\;(p_0 = 1)$

  $\Rightarrow \displaystyle(\cos\alpha + i\sin\alpha)^n\sum_{k=0}^np_k(\cos\alpha + i\sin\alpha)^{-k} = 0 \Rightarrow
  \sum_{k= 0}^np_k(\cos\alpha k - i\sin\alpha k) = 0$.

  Hence, $\displaystyle\sum_{k= 0}^np_k\sin\alpha k = p_1\sin\alpha + p_2\sin2\alpha + \cdots + p_n\sin
  n\alpha = 0$.
\item The roots of the equation $x^7 = 1$ are $\cos\frac{2k\pi}{7} + i\sin\frac{2k\pi}{7}\;\;(k = 0, 1, 2,
  \ldots, 6)$.

  Therefore, the roots of the equation $x^6 + x^5 + x^4 + x^3 + x^2 + x + 1 = 0$ will be $x_k =
  \cos\frac{2k\pi}{7} + i\sin\frac{2k\pi}{7}\;\;(k = 1, 2, 3, \ldots, 6)$.

  Putting $x + \frac{1}{x} = y$, then $x^2 + \frac{1}{x^2} = y^2 - 2$ and $x^3 + \frac{1}{x^3} = y^3 -
  3y$. Rewriting the above equation $\left(x^3 + \frac{1}{x^3}\right) + \left(x^2 + \frac{1}{x^2}\right) +
  \left(x + \frac{1}{x}\right) + 1 = 0$.

  Clearly, $x_1 = \overline{x_6}, x_2 = \overline{x_5}, x_3 = \overline{x_4}, x_k + \frac{1}{x_k} = x_k +
  \overline{x_k} = 2\cos\frac{2k\pi}{7}$.

  Hence we can say that quantities $2\cos\frac{2\pi}{7},
  2\cos\frac{4\pi}{7}, 2\cos\frac{6\pi}{7}$ are the rootss of the equation $y^3 + y^2 - 2y - 1 = 0$.

  Let the roots of the cubic equation $x^3 - ax^2 + bx - c = 0$ be $\alpha, \beta, \gamma$. Then $\alpha +
  \beta + \gamma = a, \alpha\beta + \beta\gamma + \gamma\alpha = b, \alpha\beta\gamma = c$.

  Let the equation, whose roots are $\sqrt[3]{\alpha}, \sqrt[3]{\beta}, \sqrt[3]{\gamma}$, be $x^3 - Ax^2 +
  Bx - C = 0$. Then,

  $\sqrt[3]{\alpha} + \sqrt[3]{\beta} + \sqrt[3]{\gamma} = A, \sqrt[3]{\alpha\beta} + \sqrt[3]{\beta\gamma}
  + \sqrt[3]{\gamma\alpha} = B, \sqrt[3]{\alpha\beta\gamma} = C$.

  We know that $(m + p + q)^3 = m^3 + p^3 + q^3 + 3(m + p + q)(mp + mq + pq) - 3mpq$. Substituting
  $\sqrt[3]{\alpha}, \sqrt[3]{\beta}, \sqrt[3]{\gamma}$ and $\sqrt[3]{\alpha\beta}, \sqrt[3]{\beta\gamma},
  \sqrt[3]{\gamma\alpha}$ for $m, p, q$ we obtain

  $A^3 = a + 3AB - 3C, B^3 = b + 3BCA - 3C^2$. In our case, $a = -1, b = -2, c = 1, C = 1$. Hence, $A^3 = 3AB
  - 4, B^3 = 3AB - 5$.

  Multiplying these equations and putting $AB = z$, we find

  $z^3 - 9z^2 + 27z - 20 = 0 \Rightarrow (z - 3)^3 + 7 = 0 \Rightarrow z = 3 - \sqrt[3]{7}$

  But $A^3 = 3z - 4 \Rightarrow A = \sqrt[3]{5 - 3\sqrt[3]{7}}$ and hence

  $\sqrt[3]{\cos\frac{2\pi}{7}} + \sqrt[3]{\cos\frac{4\pi}{7}} + \sqrt[3]{\cos\frac{8\pi}{7}} =
  \sqrt[3]{\frac{1}{2}(5 - 3\sqrt[3]{7})}$.
\item This problem can be solved like previous problem.
\item Squaring the first trimonial, $A^2 = (x_1^2 + 2x_2x_3) + (x_3^2 + 2x_1x_2)\omega + (x_2^2 +
  2x_1x_3)\omega^2$.

  Then $A^3 = (x_1^2 + x_2^2 + x_3^2 + 6x_1x_2x_3) + (3x_1^2x_2 + 3x_2^2x_1 + 3x_2^2x_3)\omega + (3x_1^2x_3
  + 3x_2^2x_1 + 3x_3^2x_2)\omega^2$

  Putting $3\alpha = 3x_1^2x_2 + 3x_2^2x_1 + 3x_2^2x_3$ and $3\beta = 3x_1^2x_3 + 3x_2^2x_1 + 3x_3^2x_2$.

  Now $x_1^3 + x_2^3 + x_3^3 = -(px_1 + q) - (px_2 + q) - (px_3 + q) = -3q$ since $x_1 + x_2 + x_3 =
  0$. Moreover, $x_1x_2x_3 = -q$, therefore

  $A^3 = -9q + 3\alpha\omega + 3\beta\omega^2$, we also find $B^3 = -9q + 3\alpha\omega^2 + 3\beta\omega$.

  Hence, $A^3 + B^3 = -18q - 3\alpha - 3\beta = -27q$, and similarly, $A^3B^3 = -27p^3$.
\item Let $f(x) = \frac{5x^4 + 10x^2 + 1}{x^4 + 10x^2 + 5}$ then the equation takes the form $f(x).f(a) =
  ax$.

  $x - f(x) = \frac{(x - 1)^5}{x^4 + 10x^2 + 5}$ and $x + f(x) = \frac{(x + 1)^5}{x^4 + 10x^2 +
    5}$. Dividing,

  $\frac{x - f(x)}{x + f(x)} = \left(\frac{x - 1}{x + 1}\right)^5$. Let $\frac{x - 1}{x + 1} = y$ and
  $\frac{a - 1}{a + 1} = b$.

  $\Rightarrow x - f(x) = y^5x + y^5f(x), x(1 - y^5) = f(x)(1 + y^5)\Rightarrow \frac{f(x)}{x} = \frac{1 -
    y^5}{1 + y^5}$.

  Similarly, $\frac{f(a)}{a} = \frac{1 - b^5}{1 + b^5}$. So we can write the equation as $\frac{1 - y^5}{1 +
    y^5} = \frac{1 + b^5}{1 - b^5}\Rightarrow y^5 = -b^5$.

  The last equation has five roots. $y_k = -b\epsilon^k$, where $\epsilon = \cos\frac{2\pi}{5} +
  i\sin\frac{2\pi}{5}$.

  But $x = \frac{1 + y}{1 - y}\Rightarrow x_k = \frac{(a + 1) - (a - 1)\epsilon^k}{(a + 1) + (a -
    1)\epsilon^k} = \frac{\cos\frac{k\pi}{5} - ia\sin\frac{k\pi}{5}}{a\cos\frac{k\pi}{5} -
    i\sin\frac{k\pi}{5}}$.
\item $\left(-\frac{1}{2} + i\frac{\sqrt{3}}{2}\right)^n = \cos\frac{2n\pi}{3} + i\sin\frac{2n\pi}{3}$

  Further $\left(-\frac{1}{2} + i\frac{\sqrt{3}}{2}\right)^n = \frac{(-1)^n}{2^n}(1 - i\sqrt{3})^n =
  \frac{(-1)^n}{2^n}[1 + C_1^^n(-i\sqrt{3}) + C_2^^n(-i\sqrt{3})^2 + C_3^^n(-i\sqrt{3})^3 + \cdots]$

  $= \frac{(-1)^n}{2^n}[1 - 3C_2^^n + \cdots] -i\sqrt{3}[C_1^^n -3C_3^^n + 3^2C_5^^n - 3^3C_7^^n + \cdots]$

  Equating coefficient of $i$ in both the equations, $S = (-1)^{n +
    1}\frac{2^n}{\sqrt{3}}\sin\frac{2n\pi}{3}$.
\item We have $(1 + i)^n = 1 + C_1^^n i + C_2^^n i^2 + C_3^^n i^3 + \cdots = 1 + C_1^^n i - C_2^^n - C_3^^n
  i + \cdots$

  But $1 + i = \sqrt{2}\left(\cos\frac{\pi}{4} + i\sin\frac{\pi}{4}\right)$

  Therefore, $\sigma = 1 - C_2^^n + C_4^^n -C_6^^n + \cdots = 2^{\frac{n}{2}}\cos\frac{n\pi}{4}$,

  $\sigma' = C_1^^n - C_3^^n + C_5^^n - C_7^^n + \cdots = 2^{\frac{n}{2}}\sin\frac{n\pi}{4}$.

  Hence, if $n = 0(\mod 4)$ i.e. $n = 4m\;\forall\;m\in\mathbb{I}$, then $\sigma = (-1)^m2^{2m}, \sigma' =
  0$. If $n = 4m + 1$, then $\sigma = \sigma' = (-1)^m2^{2m}$, for $n = 4m + 2, \sigma = 0, \sigma' =
  (-1)2^{2m + 1}$ and for $n = 4m + 3, \sigma = (-1)^{m + 1}2^{2m + 1}, \sigma' = (-1)^m2^{2m + 1}$.

\stopitemize
