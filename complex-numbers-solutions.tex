% -*- mode: context; -*-
\chapter{Complex Numbers}
\startitemize[n, 1*broad]
\item Let $z = 7 + 8i$, and $\sqrt{z} = \sqrt{7 + 8i} = x + iy$. Squaring $7 + 8i = (x^2 - y^2) + 2ixy$
  Comparing real and imaginary parts $x^2 - y^2 = 7, xy = 4 \Rightarrow x^2 + y^2 = \sqrt{113}$. We discaard
  $-\sqrt{113}$ as that will make $x, y$ complex.

  $\Rightarrow x = \frac{\sqrt{7 + \sqrt{113}}}{2}, y = \frac{\sqrt{\sqrt{113} - 7}}{2}$.
\item Let $\sqrt{a^2 - b^2 + 2abi} = x + iy$, then on squaring and comparison of real and imaginary parts,
  we have $x^2 - y^2 = a^2 - b^2, xy = ab\Rightarrow x^2 + y^2 = a^2 + b^2\Rightarrow x = a, y = b$.
\item $\sqrt[4]{81i^2} = \sqrt{\pm9i}$ and now we can solve it like previous problems.
\item Let $z = \frac{x^2}{y^2} + \frac{y^2}{x^2} + \frac{1}{2i}\left(\frac{x}{y} + \frac{y}{x}\right) +
  \frac{31}{16} = \left(\frac{x}{y} + \frac{y}{x}\right)^2 - 2\frac{i}{4}\left(\frac{x}{y} +
  \frac{y}{x}\right) + \frac{i^2}{4} = \left(\frac{x}{y} + \frac{y}{x} - \frac{i}{4}\right)^2$

  $\therefore$ square root $=\pm\left(\frac{x}{y} + \frac{y}{x} - \frac{i}{4}\right)$.
\item We know that $i^4 = 1$. Let $z = i^{n + 80} + i^{n + 50} = i^{n + 4.20} + i^{n + 12.4 + 2} = i^n +
  i^{n + 2} = i^n - i^n = 0$.
\item Let $z = \left(i^{17} + \frac{1}{i^{15}}\right)^3 = \left(i^{4.4 + 1} + \frac{1}{i^{4.4 - 1}}\right)^3
  = (i + i)^3 = 8i^3 = -8i$.
\item Let $z = \frac{(1 + i)^2}{2 + 3i} = \frac{2i}{2 + 3i}.\frac{2 - 3i}{2 - 3i} = \frac{-6 + 4i}{13}$.
\item Let $z = \left(\frac{1}{1 + i} + \frac{1}{1 - i}\right)\frac{7 + 8i}{7 - 8i} = \frac{2}{1 - i^2}\frac{(7 +
  8i)(7 + 8i)}{(7 - 8i)(7 + 8i)} = \frac{2}{2}\frac{-15 + 112i}{49 + 64} = \frac{-15 + 112i}{113}$.
\item Let $z = \frac{(1 + i)^{4n + 7}}{(1 - i)^{4n - 1}} = \frac{(1 + i)^{4(n + 2) - 1}}{(1 - i)^{4n - 1}} =
  \frac{1 - i}{1 + i} = \frac{(1 - i)^2}{1 - i^2} = \frac{1 - 2i + i^2}{2} = -i$.
\item Let $z = \frac{1}{1 - \cos\theta + 2i\sin\theta} = \frac{1 - \cos\theta - 2i\sin\theta}{(1 -
  \cos\theta)^2 + 4\sin^2\theta} = \frac{1 - \cos\theta - 2i\sin\theta}{1 - 2\cos\theta + 1 +
  3\sin^2\theta} = \frac{1 - \cos\theta - 2i\sin\theta}{2 - 2\cos\theta + 3\sin^2\theta}$.
\item Let $z = \frac{(\cos x + i\sin x)(\cos y + i\sin y)}{(\cot u + i)(i + \tan v)}$. Using Euler's
  formula, we have $z = \frac{e^{ix}.e^{iy}}{\frac{e^{iu}}{\sin u}.\frac{e^{iv}}{\cos v}} = \sin u\cos
  v.e^{i(x + y - u - v)} = \sin u\cos v\cos(x + y - u - v) + i\sin u\cos v\sin(x + y - u - v)$.
\item \startitemize[i]\item $i^5 = i^{4 + 1} = i$.
\item $i^{67} = i^{64 + 3} = i^3 = -i[\because i^2 = -1]$.
\item $i^{-59} = \frac{1}{i^{15.4 - 1}} = i$.
\item $i^{2014} = i^{4.503 + 2} = i^2 = -1$.
\stopitemize
\item $|a| = -a \Rightarrow \sqrt{ab} = \sqrt{|a|b}i$.
\item Let $z = i^n + i^{n + 1} + i^{n + 2} + i^{n + 3} = i^n + i.i^n - i^n - i.i^n = 0$.
\item $\displaystyle\sum_{n = 1}^{13}(i^n + i^{n + 1}) = \sum_{i=1}^{13}i^n + \sum_{i=1}^{13}i^{n + 1} = (i
  + i^2 + i^3 + \cdots + i^{13}) + (i^2 + i^3 + i^4 + \cdots + i^{14}) = i - 1$.
\item $\frac{2^n}{(1 + i)^{2n}} + \frac{(1 + i)^{2n}}{2^n} = \frac{2^n}{(1 + i^2 + 2i)^n} + \frac{(1 + i^2 +
  2i)^n}{2^n} = \frac{1}{i^n} + i^n = \frac{i^n}{i^{2n}} + i^n = i^n\left(\frac{1}{(-1)^n} + 1\right) =
  i^n[(-1)^n + 1]$.
\item Let $z = i^n + \frac{1}{i^n} = \frac{i^{2n} + 1}{i^n}$. Substituting $n = 1, 2, 3, 4$, $z = 0, \pm2$
  i.e. there exists three different solutions.
\item $4x + (3x - y)i = 3 - 6i$. Comparing real and imaginary parts, $4x = 3, 3x - y = -6\Rightarrow x =
  \frac{3}{4}\Rightarrow \frac{9}{4} - y = -6 \Rightarrow y = \frac{33}{4}$.
\item $\left(\frac{1}{3} + i\frac{7}{3}\right) + \left(4 + i\frac{1}{3}\right) - \left(-\frac{4}{3} +
  i\right) = \left(\frac{1}{3} + 4 + \frac{4}{3}\right) + i\left(\frac{7}{3} + \frac{1}{3} - 1\right) =
  \frac{17}{3} + i\frac{5}{3}$.
\item $\frac{(1 + i)x - 2i}{3 + i} + \frac{(2 - 3i)y + i}{3 - i} = i \Rightarrow [(1 + i)x - 2i](3 - i) +
  [(2 - 3i)y + i](3 + i) = i(3 + i)(3 - i)\Rightarrow (4x + 9y - 3) + i(2x - 7y - 3) = 10i$. Equating real
  and imaginary parts, $4x + 9y = 3, 2x - 7y = 13 \Rightarrow x = 3, y = -1$.
\item The multiplicative inverse is $\frac{1}{z} = \frac{1}{4 - 3i} = \frac{1}{4 - 3i}.\frac{4 + 3i}{4 + 3i}
  = \frac{4 + 3i}{25}$.
\item Let $x_1 = 2, y_1 = 3, x_2 = 1$ and $y_2 = 12$. $\therefore \frac{z_!}{z_2} = \frac{[(x_1x_2 + y_1y_2)
    + i(x_2y_1 - x_1y_2)]}{x_2^2 + y_2^2} = \frac{8 - i}{5}$.
\item $z_1 = z_2 \Rightarrow 9y^2 - 4 -10xi = 8y^2 -20i$. Equating real and imaginary parts, $9y^2 - 4 =
  8y^2 \Rightarrow y = \pm2$ and $-10x = -20 \Rightarrow x = -2 \Rightarrow z = x + iy = -2\pm2i$.
\item Let $z = x + iy$ then $|x + iy + 1| = x + iy + 2(1 + i) \Rightarrow \sqrt{(x + 1)^2 + y^2} = (x + 2) +
  i(y + 2)$. Equating real and imaginary parts, $y + 2 = 0 \Rightarrow y = -2$ and $(x + 1)^2 + y^2 = (x +
  2)^2 \Rightarrow x^2 + 2x + 5 = 4 = x^2 + 4x + 4 \Rightarrow x = \frac{1}{2} \Rightarrow z = \frac{1 -
    4i}{2}$.
\item Let $z = \frac{1 + 2i}{1 - 3i} = \frac{(1 + 2i)(1 + 3i)}{1 - (3i)^2} = \frac{1 + 3i + 2i + 6i^2}{1 +
  9} = \frac{-5 + 5i}{10} = -\frac{1}{2} + \frac{1}{2}i$

  $\Rightarrow |z| = \sqrt{\left(-\frac{1}{2}\right)^2 + \frac{1}{2^2}} = \frac{1}{\sqrt{2}}$

  $\tan\theta = \frac{\frac{1}{2}}{-\frac{1}{2}} \Rightarrow \theta = \tan^{-1}-1 = \frac{3\pi}{4}$.
\item Given, $\frac{x - 3}{3 + i} + \frac{y - 3}{3 - i} = i(3 - i)(3 + i)\Rightarrow (x - 3)(3 - i) + (y -
  3)(3 + i) = 10i\Rightarrow 3x - 9 + i(3 - x) + (3y - 9) + i(y - 3) = 10i$

  Comparing real and imaginary parts, we get $3x + 3y - 18 = 0$ and $y - x = 10 \Rightarrow x = -2, y = 8$.
\item $(1 + i)^2 = 1 + 2i - i = 2i\Rightarrow (1 + i)^{50} = (2i)^{25} = 2^{25}i^{4.6 + 1} = 2^{25}i$ Thus,
  real part will be $0$.
\item Let $z = x + iy$ then $x + iy + \sqrt{x^2 + y^2} = 2 + 8i$, Comparing real and imaginary parts, we get
  $y = 8$ and $x + \sqrt{x^2 + y^2} = 2 \Rightarrow \sqrt{x^2 + y^2} = 2 - x$

  $\Rightarrow x^2 + 64 = 4 - 4x + x^2 \Rightarrow x = -15\Rightarrow z = -15 + 8i$.
\item $S = i + 2i^2 + 3i^3 +\ldots + 100i^{100}\Rightarrow iS = i^2 + 2i^3 + \ldots + 99i^{100} + 100i^{101}$

  $\Rightarrow S(1- i) = i + i^2 + \ldots + i^{100} - 100i^{101} = \frac{i(1 - i^{101})}{1 - i} - 100i^{101}$

  $S = \frac{i(1 - i^{101})}{(1- i)^2} - \frac{100i^{101}}{1 - i}$.
\item Consider $t_1 = \frac{1}{1 + i} + \frac{1}{1 - i} + \frac{1}{-1 + i} + \frac{1}{-1 - i} = \frac{1 + i
  + 1 - i}{1^2 - i^2} + \frac{-1 + i - 1 - i}{(-1)^2 - i^2} = \frac{2}{2} + \frac{-2}{2} = 0$

  $t_2 = 2\left(\frac{1}{1 + i} + \frac{1}{1 - i} + \frac{1}{-1 + i} + \frac{1}{-1 - i}\right) = 0$

  Similarly all other terms and sum will be zero.
\item Given, $z^2 - z - 5 + 5i = 0 \Rightarrow D = (-1)^2 - 4.1.(-5 + 5i) = 21 - 20i$ and we will need
  $\sqrt{D}$

  $\sqrt{D} = \sqrt{b^2 - 4ac} = \sqrt{21 - 20i} = \pm \left[\sqrt{\frac{x^2 + y^2 + x}{2}} -
  i\sqrt{\frac{x^2 + y^2 - x}{2}}\right] = \pm(5 - 2i)$

  $z = \frac{1 + 5 - 2i}{2}$ or $z = \frac{1 - 5 + 2i}{2}\Rightarrow z = 3 - i, i - 2$

  Thus, product of real parts $= -2\times 3 = -6$
\item Given, $z^3 = -\overline{z}\Rightarrow |z|^3 = |z|\Rightarrow |z|(|z| - 1)(|z| + 1) = 0 \Rightarrow
  |z| = 0, |z| = 1 [\because |z| + 1 > 0]$

  If $|z| = 0,$ then $z = 0.$ If $|z| = 1 \Rightarrow |z|^2 = 1\Rightarrow z\overline{z} = 1\Rightarrow z^3
  + \frac{1}{z} = 0 \Rightarrow z^4 + 1 = 0,$ which has four distinct roots. Thus, given equation has five
  roots.
\item Since we have to find real roots, let $z = x,$ a real value. The given equation becomes $x^3 + ix - 1=
  0\Rightarrow x^3 = 1, x = 0$ which is not possible. So there are no real solutions.
\item Let $z = x + iy,$ then $\sqrt{x^2 + y^2} > 1,$ because point $A$ is outside circle.

  $\frac{1}{z} = \frac{x - iy}{\sqrt{x^2 + y^2}}$ so $\frac{x}{\sqrt{x^2 + y^2}}, \frac{-y}{x^2 + y^2}< 1$

  This leads to the fact that point $E$ is reciprocal of point $A$.
\item $z = (3p - 7q) + i(3q + 7p),$ which is purely imaginary, $\Rightarrow 3p - 7q = 0$

  $\Rightarrow \frac{p}{q} = \frac{7}{3} \Rightarrow \frac{p}{q} + i = \frac{7}{3} + i \Rightarrow \frac{p +
  iq}{q} = \frac{7 + 3i}{3}$

  $\Rightarrow p+ iq = 7 + 3i \Rightarrow z = 21 + 9i + 49i - 21 = 58i \Rightarrow |z|^2 = 3364$.
\item Given, $\alpha = \left(\frac{a - ib}{a + ib}\right)^2 + \left(\frac{a + ib}{a - ib}\right)^2 =
  \frac{(a - ib)^4 + (a + ib)^4}{(a - ib)^2(a + (ib)^2)}$

  $= \frac{a^4 - 4a^3.ib + 6a^2i^2b^2 - 4ai^3b^3 + b^4 + a^4 + 4a^3ib + 6a^2i^2b^2 + 4ai^3b^3 + b^4}{(a^2 +
    b^2)^2} = \frac{2a^4 - 12a^2b^2 + 2b^4}{(a^2 + b^2)^2}$, which is purely real.
\item Let $z = x + iy$ then given $|z| = 1 \Rightarrow x^2 + y^2 = 1$

  Let $\beta = \frac{z - 1}{z + 1} = \frac{(x - 1) + iy}{(x + 1) + iy} = \frac{(x - 1) + iy}{(x + 1) +
  iy}.\frac{(x + 1) - iy}{(x + 1) - iy}$

  $= \frac{x^2 - 1 + y^2 + iy(x + 1 -x + 1)}{(x + 1)^2 + y^2} = \frac{2iy}{(x + 1)^2 + y^2}$ which is purely
  imaginary.
\item Let $z = x + iy \Rightarrow x^2 + (y - 3)^2 = 9 \Rightarrow x = 3\cos\theta, y = 3\sin\theta + 3$

  $z = 3[\cos\theta + i(\sin\theta + 1)] = 3\left[\sin\left(\frac{\pi}{2} - \theta\right) + i\left(1 +
  \cos\left(\frac{\pi}{2} - \theta\right)\right)\right]$

  $= 3\left[2\sin\left(\frac{\pi}{4} - \frac{\theta}{2}\right)\cos\left(\frac{\pi}{4} -
  \frac{\theta}{2}\right) + i2\cos^2\left(\frac{\pi}{4} - \frac{\theta}{2}\right)\right]$

  $= 6\cos\left(\frac{\pi}{4} - \frac{\theta}{2}\right)\left[\sin\left(\frac{\pi}{4} -
  \frac{\theta}{2}\right) + i\cos\left(\frac{\pi}{4} - \frac{\theta}{2}\right)\right]
  = 6\cos\left(\frac{\pi}{4} - \frac{\theta}{2}\right)e^{i\left(\frac{\pi}{4} + \frac{\theta}{2}\right)}$

  $\cot\left(arg(z)\right) = \cot\left(\frac{\pi}{4} + \frac{\theta}{2}\right) = \tan\left(\frac{\pi}{4} -
  \frac{\theta}{2}\right)$

  $\frac{6}{z} = \sec\left(\frac{\pi}{4} - \frac{\theta}{2}\right)e^{-i\left(\frac{\pi}{4} +
    \frac{\theta}{2}\right)} = \sec\left(\frac{\pi}{4} - \frac{\theta}{2}\right)
  \left[\sin\left(\frac{\pi}{4} - \frac{\theta}{2} - i\cos\left(\frac{\pi}{4} -
    \frac{\theta}{2}\right)\right)\right]$

  $= \tan\left(\frac{\pi}{4} - \frac{A}{2}\right) - i \Rightarrow \cot(\arg(z)) - \frac{6}{z} = i$.
\item Let $z = r(\cos\theta + i\sin\theta) = \frac{-16}{1 + \sqrt{3}} = \frac{-16}{1 + i\sqrt{3}}.\frac{1 -
  i\sqrt{3}}{1 - i\sqrt{3}} = \frac{-16(1 - i\sqrt{3})}{1 + 3}$

  $= -4 + i4\sqrt{3}$ then $r\cos\theta = 4, r\sin\theta = 4\sqrt{3} \Rightarrow r^2 = 64 \Rightarrow r = 4,
  \cos\theta = \frac{-1}{2}, \sin\theta = \frac{\sqrt{3}}{2} \Rightarrow \theta = \frac{2\pi}{3}$

  $\Rightarrow z = 8\left(\cos\frac{2\pi}{3} + i\sin\frac{2\pi}{3}\right)$.
\item Let $z = r(\cos\theta + i\sin\theta)$ then because $arg(z) + arg(w) = \pi \Rightarrow arg(w) = \pi -
  \theta$

  $\Rightarrow w = r(-\cos\theta + i\sin\theta) = -r(\cos\theta - i\sin\theta)\;\therefore\;r =
  -\overline{w}$.
\item $x - iy = \sqrt{\frac{a - ib}{c - id}} \Rightarrow x^2 - y^2 - 2ixy = \frac{a - ib}{c - id} = \frac{(a
  - ib)(c + id)}{c^2 + d^2}\Rightarrow x^2 - y^2 - 2ixy = \frac{(ac + bd) -i(bc - ad)}{c^2 + d^2}$

  Comparing real and imaginary parts, we get $x^2 - y^2 = \frac{ac + bd}{c^2 + d^2}, 2xy = \frac{bc -
    ad}{c^2 + d^2}$

  $\Rightarrow (x^2 + y^2)^2 = (x^2 - y^2)^2 + 4x^2y^2 = \frac{(ac + bd)^2 + (bc - ad)^2}{(c^2 + d^2)} = \frac{a^2c^2 +
    b^2d^2 + b^2c^2 + a^2d^2}{(c^2 + d^2)^2} = \frac{a^2 + b^2}{c^2 + d^2}$.
\item We know that for two complex numbers $z_1$ and $z_2, |z_1| + |z_2|\geq |z_1 - z_2|$

  $|z| + |z - 2| \geq |z - (z - 2)| = |2| = 2$.  Therefore, minimum value is $2$.
\item $|z_1 + z_2 + z_3| = |(z_1 - 1) + (z_2 - 2) + (z_3 - 3) + 6\leq |z_1 - 1| + |z_2 - 2| + |z_3 - 3| + 6$

  $< 1 + 2+ 3 + 6 = 12$.  Thus, maximum value of $|z_1 + z_2 + z_3|$ is $12.$
\item $|\alpha + \beta|^2 = (\alpha + \beta)(\overline{\alpha + \beta}) = (\alpha + \beta)(\overline{\alpha}
  + \overline{\beta}) = \alpha\overline{\alpha} + \alpha\overline{\beta} + \overline{\alpha}\beta +
  \beta\overline{\beta} = |\alpha|^2 + |\beta|^2 + \alpha\overline{\beta} + \overline{\alpha}\beta$

  Similarly, $|\alpha - \beta|^2 = |\alpha|^2 + |\beta|^2 - \alpha\overline{\beta} - \overline{\alpha}\beta$

  Thus, $|\alpha|^2 + |\beta|^2 = \frac{1}{2}(|\alpha + \beta|^2 + |\alpha - \beta|^2)$
\item If $|z| = 0$ then $\sqrt{x^2 + y^2} = 0 \Rightarrow x^2 + y^2 = 0$

  Above is possible if and only if $x = 0$ and $y = 0 \Rightarrow z = 0$.
\item $\frac{z_1z_2}{\overline{z_1}} = \frac{(1 - i)(2 + 7i)}{1 + i} = \frac{2 + 7 -2i + 7i}{1 + i} =
  \frac{9 + 5i}{1 + i} = \frac{9 + 5i}{1 + i}.\frac{1 - i}{1 - i} = \frac{9 + 5 + 5i -9i}{2} = 7 -
  2i\therefore Im\left(\frac{z_1z_2}{\overline{z_1}}\right) = -2$.
\item $|z + 12 - 6i| \leq |z - i| + |12 - 5i| < 1 + 13 = 14$.
\item Given, $|z + 6| = |2z + 3|,$ let $z = x + iy \Rightarrow (x + 6)^2 + y^2 = (2x + 3)^2 + 4y^2
  \Rightarrow x^2 + 12x + 36 + y^2 = 4x^2 + 12x + 9 + 4y^2$

  $\Rightarrow 3x^2 + 2y^2 = 27 \Rightarrow x^2 + y^2 = 9 \Rightarrow |z| = 3$.
\item Given $\sqrt{a - ib} = x - iy,$ squaring we get $a - ib = x^2 - y^2 - 2ixy$. Comparing real and
  imaginary parts, we get
  $a = x^2 - y^2, b = 2xy \Rightarrow a + ib = x^2 - y^2 + 2ixy = x^2 +i^2y^2 + 2ixy\Rightarrow \sqrt{a +
    ib} = x + iy$.
\item $x_1x_2x_3\ldots\infty = \left(\cos\frac{\pi}{2} + i\sin\frac{\pi}{2}\right)\left(\cos\frac{\pi}{2^2}
  + i\sin\frac{\pi}{2^2}\right) \ldots\infty= \cos\left(\frac{\pi}{2} + \frac{\pi}{2^2} + \ldots
  \infty\right) + i\sin\left(\frac{\pi}{2} + \frac{\pi}{2^2} + \ldots \infty\right)$

  $= \cos\frac{\pi}{2}.\frac{1}{1- \frac{1}{2}} + i\sin\frac{\pi}{2}.\frac{1}{1- \frac{1}{2}} = \cos\pi +
  i\sin\pi = -1$.
\item Given, $\frac{(\cos\theta + i\sin\theta)^4}{(\sin\theta + i\cos\theta)^5} = \frac{(\cos\theta +
  i\sin\theta)^4}{i^5\left(\frac{1}{i}\sin\theta + \cos\theta\right)^5}$

  $= \frac{(\cos\theta + i\sin\theta)^4}{i(\cos\theta - i\sin\theta)^5} = \frac{(\cos\theta +
  i\sin\theta)^4}{i(\cos\theta + i\sin\theta)^{-5}}= \frac{1}{i}(\cos\theta + i\sin\theta)^9 = \sin9\theta
  -i \cos9\theta$.
\item $z = \left[\cos\frac{\pi}{6} + i\sin\frac{\pi}{6}\right]^5 + \left[\cos\frac{\pi}{6} - i\sin\frac{\pi}{6}\right]^5$

  $= \cos\frac{5\pi}{6} + i\sin\frac{5\pi}{6} + \cos\frac{5\pi}{6} - i\sin\frac{5\pi}{6} =
  2\cos\frac{5\pi}{6}\;\therefore Im(z) = 0$.
\item $z = \left(\cos\frac{\pi}{3} + i\sin\frac{\pi}{3}\right)^{\frac{3}{4}} = \left(\cos\pi +
  i\sin\pi\right)^{\frac{1}{4}},$ thus general root is $\cos\frac{2n\pi + \pi}{4} + i\sin\frac{2n\pi +
    \pi}{4}$

  Thus, substituting $n = 0, 1, 2, 3$ we find four roots and the product is

  $\left(\cos\frac{\pi}{4} + i\sin\frac{\pi}{4}\right)\left(\cos\frac{3\pi}{4} +
  i\sin\frac{3\pi}{4}\right)\left(\cos\frac{5\pi}{4} + i\sin\frac{5\pi}{4}\right)\left(\cos\frac{7\pi}{4} +
  i\sin\frac{7\pi}{4}\right)$

  $= \left(\frac{1}{\sqrt{2}} + \frac{i}{\sqrt{2}}\right)\left(\frac{-1}{\sqrt{2}} +
  \frac{i}{\sqrt{2}}\right)\left(\frac{-1}{\sqrt{2}} - \frac{i}{\sqrt{2}}\right)\left(\frac{1}{\sqrt{2}} -
  \frac{i}{\sqrt{2}}\right)$

  $= \left(-\frac{1}{2} - \frac{1}{2}\right)\left(\frac{-1}{2} - \frac{1}{2}\right) = -1.-1 = 1$.
\item Let $z_1 = r_1(\cos x + i\sin x)$ and $z_2 = r_2(\cos y + i\sin y)$ Then $(r_1\cos x + r_2\cos y)^2 +
  (r_1\sin x+ r_2\sin y)^2 = r_1^2 + r_2^2 + 2r_2r_2$

  $\Rightarrow 2r_1r_2(\cos x\cos y + \sin x\sin y) = 2r_2r_2 \Rightarrow \cos(x - y) = 1 \Rightarrow x - y
  = 0 \Rightarrow \arg(z_1) - arg(z_2) = 0$.
\item Let $z = 1 - \sin\alpha + i\cos\alpha = r(\cos\theta + i\sin\theta),$ then $r = \sqrt{(1 -
  \sin\alpha)^2 + \cos^2\alpha} = \sqrt{2 - 2\sin\alpha}$

  $\tan\theta = \frac{\cos\alpha}{1 - \sin\alpha} = \frac{1 - \tan^2\frac{\alpha}{2}}{1 +
  \tan^2\frac{\alpha}{2} - 2\tan\frac{\alpha}{2}} = \frac{1 + \tan\frac{\alpha}{2}}{1 -
  \tan\frac{\alpha}{2}} = \tan\left(\frac{\pi}{4} -\frac{\alpha}{2}\right)\Rightarrow \theta = \frac{\pi}{4}
  - \frac{\alpha}{2}$.
\item Let $z = \left[\frac{1 + \sin\frac{\pi}{8} + i\cos\frac{\pi}{8}}{1 + \sin\frac{\pi}{8} -
    i\cos\frac{\pi}{8}}\right] = \left[\frac{1 + \sin\frac{\pi}{8} + i\cos\frac{\pi}{8}}{1 +
    \sin\frac{\pi}{8} - i\cos\frac{\pi}{8}}\right]. \left[\frac{1 + \sin\frac{\pi}{8} +
    i\cos\frac{\pi}{8}}{1 + \sin\frac{\pi}{8} + i\cos\frac{\pi}{8}}\right]$

  $= \frac{\left(1 + \sin\frac{\pi}{8}\right)^2 - \cos^2\frac{\pi}{8} + 2i(1 +
  \sin\frac{\pi}{8})\cos\frac{\pi}{8}}{\left(1 + \sin\frac{\pi}{8}\right)^2 + \cos^2\frac{\pi}{8}} =
  \frac{2\sin\frac{\pi}{8} + 2\sin^2\frac{\pi}{8} + 2i(1 + \sin\frac{\pi}{8})\cos\frac{\pi}{8}}{2 +
    2\sin\frac{\pi}{8}}$

  $= \sin\frac{\pi}{8} + i\cos\frac{\pi}{8} = i\left(\cos\frac{\pi}{8} - i\sin\frac{\pi}{8}\right)
  \Rightarrow z^8 = i^8(\cos\pi - i\sin\pi) = -1$.
\item $z_1z_2z_3z_4z_5 = \cos\left(\frac{2\pi}{5} + \frac{4\pi}{5} + \frac{6\pi}{5} + \frac{8\pi}{5} +
  \frac{10\pi}{5}\right) + i\sin\left(\frac{2\pi}{5} + \frac{4\pi}{5} + \frac{6\pi}{5} + \frac{8\pi}{5} +
  \frac{10\pi}{5}\right)$

  $= \cos\frac{30\pi}{5} + i\sin\frac{30\pi}{5} = \cos6\pi + i\sin6\pi = 1$.
\item $z_n = \cos\left(\frac{1}{2n + 1} - \frac{1}{2n + 3}\right).\frac{\pi}{2} + i\sin\left(\frac{1}{2n +
  1} - \frac{1}{2n + 3}\right).\frac{\pi}{2}$

  $\therefore z_1z_2z_3\ldots\infty = \cos\left(\frac{1}{3} - \frac{1}{5} + \frac{1}{5} - \frac{1}{7} +
  \frac{1}{7} - \frac{1}{9}\ldots\infty\right).\frac{\pi}{2} + i\sin\left(\frac{1}{3} - \frac{1}{5} +
  \frac{1}{5} - \frac{1}{7} + \frac{1}{7} - \frac{1}{9}\ldots\infty\right).\frac{\pi}{2}$

  $= \cos\frac{\pi}{6} + i\sin\frac{\pi}{6}$.
\item Let $z_1 = x_1 + iy_1$ and $z_2 = x_2 + iy_2\Rightarrow |az_1 - bz_2|^2 + |bz_1 + az_2|^2 = (ax_1 -
  bx_2)^2 + (ay_1 - by_2)^2 + (bx_1 + ax_2)^2 + (by_1 + ay_2)^2$

  $= a^2x_1^2 + b^2x_2^2 -2abx_1x_2 + a^2y_1^2 + b^2y_2^2 - 2aby_1y_2 + b^2x_1^2 + a^2x_2^2 + 2abx_1x_2 +
  b^2y_1^2 + a^2y_2^2 + 2aby_1y_2 = (a^2 + b^2)(x_1^2 + y_1^2 + x_2^2 + y_2^2)= (a^2 + b^2)(|z_1|^2 +
  |z_2|^2)$.
\item Let $x = y + iz,$ then given expression becomes $\frac{A^2}{y + iz - a} + \frac{B^2}{y + iz - b} +
  \ldots + \frac{H^2}{y + iz - h} = y + iz + l$

  $\frac{A^2(y - a - iz)}{(y - a)^2 + z^2} + \frac{B(y - b - iz)}{(y - b)^2 + z^2} + \ldots + \frac{H^2(y -
    iz - h)}{(y - h)^2 + z^2} = y + iz + l$.  Comparing imaginary parts, we have
  $-iz\left[\frac{A^2}{(y - a)^2 + z^2} + \frac{B^2}{(y - a)^2 + z^2} + \ldots + \frac{H^2}{(y - a)^2 +
      z^2}\right] = iz \Rightarrow iz\left[1 + \frac{A^2}{(y - a)^2 + z^2} + \frac{B^2}{(y - a)^2 + z^2} +
    \ldots + \frac{H^2}{(y - a)^2 + z^2}\right] = 0$

  Clearly the term inside brackets is non-zero. So $z = 0$.
\item Let $2^{-x} = p,$ then $|1 + 4i - p|\leq 5 \Rightarrow (1 - p)^2 + 16 \leq 25$

  $1 - p \leq \pm3 \Rightarrow p \geq 4, -2 \Rightarrow x \geq -2\;\because p \nless 0\Rightarrow p \in [-2,
  \infty]$.
\item A unimodular number has a modulus of $1. \cos\theta + i\sin\theta = \frac{c + i}{c - i} = \frac{c +
  i}{c - i}.\frac{c + i}{c - i} = \frac{c^2 - 1 + 2ic}{c^2 + 1}$

  Comparing real and imaginary parts, $\cos\theta = \frac{c^2 - 1}{c^2 + 1} \Rightarrow c =
  \pm\cot\frac{\theta}{2}$

  and $\sin\theta = \frac{2c}{c^2 + 1} \Rightarrow c = \cot\frac{\theta}{2}, \tan\frac{\theta}{2}$. So the
  common value is $c = \cot\frac{\theta}{2}$.
\item $(z^3 + 3)^2 = -16 = 16i^2 \Rightarrow z^3 = -3 \pm 4i\Rightarrow |z^3| = 5 \Rightarrow |z| =
  5^{1/3}$.
\item $z = \frac{\sin\frac{x}{2} + \cos\frac{x}{2} - i\tan x}{1 + 2i\sin\frac{x}{2}}= \frac{\sin\frac{x}{2}
  + \cos\frac{x}{2} - i\tan x}{1 + 2i\sin\frac{x}{2}}.\frac{1 - 2i\sin\frac{x}{2}}{1 - 2i\sin\frac{x}{2}}$

  Since it is real so imaginary part of this will be $0.\Rightarrow -\tan x -2\sin\frac{x}{2}\cos\frac{c}{2}
  -2\sin\frac{x}{2}\cos\frac{x}{2} = 0$

  $2\sin\frac{x}{2}\left(\sin\frac{x}{2} + \cos\frac{x}{2}\right) +
  \frac{2\sin\frac{x}{2}\cos\frac{x}{2}}{\cos x} = 0\Rightarrow \sin\frac{x}{2} = 0 \Rightarrow x = 2n\pi$
  where $n = 0,1,2,3\ldots$

  or $\left(\sin\frac{x}{2} + \cos\frac{x}{2}\right)\cos x + \cos\frac{x}{2} = 0\Rightarrow
  \tan^3\frac{x}{2} - \tan\frac{x}{2} - 2 = 0$

  If $\alpha$ is a solution of above then the set of possible values are $x = 2n\pi + 2\alpha$. Solving the
  cubic equation is left to you.
\item Let $z_1 = x_1 + iy_1$ and $z_2 = x_2 + iy_2$ then $|z_1 + z_2|^2 + |z_1 - z_2|^2 = (x_1 + x_2)^2 +
  (y_1 + y_2)^2 + (x_1 - x_2)^2 + (y_1 - y_2)^2$

  $= 2(x_1^2 + y_1^2 + x_2^2 + y_2^2) = 2(|z_1|^2 + |z_2|^2)$.
\item Given, $x^2 - x + 1 = 0\Rightarrow x = -\omega, -\omega^2$

  $\displaystyle\sum_{n = 1}^5\left(x^n + \frac{1}{x^n}\right)^2 = \sum_{n = 1}^5\left(x^{2n} +
  \frac{1}{x^{2n}} + 2\right)$

  $= \left(x^2 + \frac{1}{x^2} + 2\right) + \left(x^4 + \frac{1}{x^4} + 2\right) + \left(x^6 + \frac{1}{x^6}
  + 2\right) + \left(x^8 + \frac{1}{x^8} + 2\right) + \left(x^{10} + \frac{1}{x^{10}} + 2\right)$

  $= (x^2 + x^4 + x^6 + x^8 + x^{10}) + \left(\frac{1}{x^2} + \frac{1}{x^4} + \frac{1}{x^6} + \frac{1}{x^8}
  + \frac{1}{x^{10}}\right) + 10$

  $= (\omega^2 + \omega^4 + \omega^6 + \omega^8 + \omega^{10}) + \left(\frac{1}{\omega^2} +
  \frac{1}{\omega^4} + \frac{1}{\omega^6} + \frac{1}{\omega^8} + \frac{1}{\omega^{10}}\right) + 10$

  $= -1 - 1 + 10 = 8$.
\item $3^{49}(x + iy) = \left[i\sqrt{3}\left(\frac{1 - i\sqrt{3}}{2}\right)\right]^{100} =
  i^{100}3^{50}(-\omega)^{100} \Rightarrow 3^{49}(x + iy) = 3^{50}.\omega$

  $x + iy = -\frac{3}{2} + \frac{3\sqrt{3}}{2}i\Rightarrow x = -\frac{3}{2}, y = \frac{3\sqrt{3}}{2}$.
\item $|z_1 + z_2|^2 = x_1^2 + x_2^2 + y_1^2 + y_2^2 + 2x_1x_2 + 2y_1y_2 = |z_1|^2 + |z_2|^2 + 2(x_1x_2 +
  y_1y_2)$

  Now, $2Re(z_1\overline{z_2}) = 2Re[(x_1 + iy_1)(x_2 - iy_2)] = 2\Re[x_1x_2 + y_1y_2 -i(x_1y_2 + x_2y_1)] =
  2(x_1x_2 + y_1y_2)$

  Similalry, $2\Re(\overline{z_1}z_2) = 2(x_1x_2 + y_1y_2)$.
\item R.H.S. = $\left|\frac{1}{z_1} + \frac{1}{z_2}\right| = \left|\frac{z_2 + z_1}{z_1z_2}\right|$

  Since $|z_1| = |z_2| = 1 \therefore |z_1z_2| = 1$ and thus $|z_1 + z_2| = \left|\frac{1}{z_1} +
  \frac{1}{z_2}\right|$.
\item Let $z = x + iy,$ then $x^2 - 4x + 4 + y^2 = 4x^2 - 8x + 4 + 4y^2 \Rightarrow 3x^2 + 3y^2 = 4x$

  $\Rightarrow 3|z|^2 = 4Re(z) \Rightarrow |z|^2 = \frac{4}{3}Re(z)$.
\item Given $\sqrt[3]{a + ib} = x + iy \Rightarrow a + ib = (x + iy)^3 = x^3 - 3xy^2 + i(3x^2y - y^3)$

  Comparing real and imaginary parts, we have $a = x^3 - 3xy^2, b = 3x^2y - y^3 \Rightarrow \frac{a}{x} =
  x^2 - 3y^2, \frac{b}{y} =3x^2- y^2$

  $\therefore \frac{a}{x} + \frac{b}{y} = 4(x^2 - y^2)$.
\item $x + iy = \sqrt{\frac{a + ib}{c + id}} \Rightarrow (x + iy)^2 = \frac{a + ib}{c + id} \Rightarrow |(x
  + iy)^2| = \left|\frac{a + ib}{c + id}\right| = \frac{|a + ib|}{|c + id|} \Rightarrow (x^2 + y^2)^2 =
  \frac{a^2 + b^2}{c + d^2}$.
\item Let $z = 1 = \cos0^\circ + i\sin0^\circ = e^{i2r\pi}\;\forall i \in N \Rightarrow \sqrt[n]{z} =
  e^{\frac{i.2r\pi}{n}}$. Clearly, $|z_k| = |z_{k + 1}| = 1$.
\item $z^n  = (z + 1)^n \Rightarrow \frac{z}{z + 1} = 1^{1/n}$

  This means $\frac{z}{z + 1}$ is $n$th root of unity. $\Rightarrow \left|\frac{z}{z + 1}\right| = 1$

  $\Rightarrow |z| = |z + 1| \Rightarrow x^2 + y^2 = x^2 + 2x + 1 + y^2 \Rightarrow x = -\frac{1}{2}
  \Rightarrow Re(z) < 0$.
\item Roots of $1 + x + x^2 = 0$ are $\omega$ and $\omega^2.$ Let $f(x) = x^{3m} + x^{3n - 1} + x^{3r - 2}$

  $f(x) = x^{3m} + \frac{x^{3n}}{x} + \frac{x^{3r}}{x^2} \Rightarrow f(\omega) = 1 + \frac{1}{\omega} +
  \frac{1}{\omega^2} = \frac{1 + \omega + \omega^2}{\omega^2} = 0$

  Similarly $f(\omega^2) = 0$. Thus, we see that $f(x)$ has same roots as $1 + x + x^2= 0.$ Hence, $f(x)$
  will be divisible by $1 + x + x^2$.
\item $\sqrt{3} + i = 2\left(\frac{\sqrt{3}}{2} + i\frac{1}{2}\right) = 2\left(\cos\frac{\pi}{6} +
  i\sin\frac{\pi}{6}\right) = 2e^{i\frac{\pi}{6}}$

  Similarly, $\sqrt{3} - i = 2e^{-i\frac{\pi}{6}}$

  Since imaginary part is what prevents equality we need to get rid of it and the least value for which it
  will happen is when argument is $\pi.$ Thus, we need to raise to the power by $6$ making $n = 6$.
\item $\sqrt{3} - i = 2.\left(\cos\frac{\pi}{6} - i\sin\frac{\pi}{6}\right)$

  Thus, $(\sqrt{3} - i)^n = 2^n \Rightarrow 2^n\left(\cos\frac{n\pi}{6} - i\sin\frac{\pi}{6}\right) = 2^n$

  $\Rightarrow \cos\frac{n\pi}{6} - i\sin\frac{n\pi}{6} = 1 \Rightarrow \frac{n\pi}{6} = 2k\pi\;\forall k\in
  I \Rightarrow n = 12k$

  Thus, $n$ is a multiple of $12$.
\item Given, $z^4 + z^3 + 2z^2 + z + 1 = 0 \Rightarrow z^2(z^2 + z + 1) + z^2 + z + 1 = 0$

  $\Rightarrow (z^2 + 1)(z^2 + z + 1) = 0$. If $z^2 + 1 = 0 \Rightarrow z = i \Rightarrow |z| = 1$

  If $z^2 + z + 1 = 0 \Rightarrow z = \omega, \omega^2 \Rightarrow |z| = 1$.
\item $\because z = \sqrt[7]{-1}\Rightarrow z^7 = -1\Rightarrow z^{86} + z^{175} + z^{289} = (z^7)^{14}.z^2
  + (z^7)^{25} + (z^7)^{41}z^2 = z^2 -1 -z^2 = -1$
\item Given, $z^3 + 2z^2 + 3z + 2 = 0\Rightarrow z^3 + z^2 + 2z + z^2 + z + 2 = 0\Rightarrow (z + 1)(z^2 + z
  + 2) = 0$

  If $z + 1 = 0 \Rightarrow z = -1,$ which is real and is of no interest for us.

  If $z^2 + z + 2 = 0 \Rightarrow z = \frac{-1 + i\sqrt{7}}{2}$ which are complex roots of the given
  equation.
\item $z = \sqrt[5]{1} \Rightarrow z^5 = 1$

  $2^{|1 + z + z^2 + z^{-2} - z^{-1}|} = 2^{|1 + z + z^2 + z^3 - z^4|}[\because z^4 = 1 \Rightarrow z^{-1} =
  \frac{z^5}{z} = z^4]$

  $= 2^{|1 + z + z^2 + z^3 + z^4 - 2z^4|} = 2^{\left|\frac{1 - z^5}{1 - z} - 2z^4\right|} = 2^{|2z^4|} = 2^2
  = 4[\because |z| = 1]$.
\item Let $S = 1 + 3z + 5z^2 + \ldots + (2n - 1)z^{n - 1}$

  $\Rightarrow zS = z + 3z^2 + 5z^3 + \ldots + (2n - 3)z^{n - 1} + (2n - 1)z^n$

  $\Rightarrow (1 - z)S = 1 + 2z + 2z^2 + 2z^3 + \ldots + 2z^{n - 1} + (2n - 1)z^n$

  $\Rightarrow (1 - z)S = 1 + 2n - 1 + 2[z + z^2 + \ldots z^{n - 1}][\because z^n = 1]$

  $= 2n + 2.-1[\because 1 + z + z^2 + \ldots + z^{n - 1} = 0] \Rightarrow S = \frac{2(n - 1)}{1 - z}$.
\item Let $z = \sqrt{-1-\sqrt{-1-\sqrt{-1-\infty}}} \Rightarrow z = \sqrt{-1 - z}$

  $\Rightarrow z^2 = -1 - z \Rightarrow z^2 + z + 1 = 0 \Rightarrow z = \frac{-1 \pm i\sqrt{3}}{2}
  \Rightarrow z = \omega, \omega^2$.
\item Given, $z = e^{\frac{i2\pi}{n}},$ which is nth root of unity.

  $\therefore x^n - 1 = (x - 1)(x - z)(x - z^2 (x - z^3) ... (x - z^{n - 1})$

  Putting $x = 11, (11 - z)(11 - z^2)\ldots(11 - z^{n - 1}) = \frac{11^n - 1}{10}$.
\item Given, $\frac{3}{2 + \cos\theta + i\sin\theta} = a + ib \Rightarrow a + ib  \frac{3(2 + \cos\theta -
  i\sin\theta)}{5 + 4\cos\theta}$

  Comparing real and imaginary parts, we get $a = \frac{6 + 3\cos\theta}{5 + 4\cos\theta}, b =
  \frac{-3\sin\theta}{5 + 4\cos\theta} \Rightarrow a^2 + b^2 = \frac{36 + 36\cos\theta + 9\cos^2\theta +
    9\sin^2\theta}{(5 + 4\cos\theta)^2}$

  $= \frac{45 + 36\cos\theta}{(5 + \cos\theta)^2} = \frac{9(5 + 4\cos\theta)}{(5 + 4\cos\theta)^2} =
  \frac{9}{5 + 4\cos\theta}, 4a - 3 = \frac{24 + 12\cos\theta - 15 - 12\cos\theta}{5 +
    4\cos\theta} = \frac{9}{5 + 4\cos\theta}\Rightarrow a^2 + b^2 = 4a - 3$.
\item Let $z = x + iy, \Rightarrow |(2x - 1) + 2iy| = |(x - 2) + iy|\Rightarrow 4x^2 - 4x + 1 + 4y^2 = x^2 -
  4x + 4 + y^2 \Rightarrow 3x^2 + 3y^2 = 3\Rightarrow x^2 + y^2 = 1\Rightarrow |z| = 1$.
\item Given, $\frac{1 -ix}{1 + ix} = m + in \Rightarrow m + in = \frac{1 - ix}{1 + ix}.\frac{1 - ix}{1 - ix}$

  $m + in = \frac{1 - x^2 - 2ix}{1 + x^2}$, Comparing real and imaginary parts, $m = \frac{1 - x^2}{1 +
  x^2}, n = \frac{-2x}{1 + x^2}$

  $\Rightarrow m^2 + n^2 = \frac{(1 - x^2)^2 + 4x^2}{(1 + x^2)^2} = 1$.
\item We know that the equation of a straight line is given by $\startbmatrix\NC z \NC \overline{z} \NC
  1\NR\NC z_1 \NC \overline{z_1} \NC 1 \NR\NC z_2 \NC \overline{z_2} \NC 1\NR\stopbmatrix = 0$

  $\Rightarrow z(\overline{z_1} - \overline{z_2}) - \overline{z}(z_1 - z_2) + z_1\overline{z_2} -
  \overline{z_1}z_2 = 0$

  $\Rightarrow z(1 + i - 1 - i) - \overline{z}(1 + i -1 + i) + (1 + i)^2 - (1 - i)^2 = 0\Rightarrow z +
  \overline{z} - 2 = 0$.
\item Given, $5z_1 - 13z_2 + 8z_3 = 0 \Rightarrow z_2 = \frac{5z_1 + 8z_3}{5 + 8}$

  This means $z_1$ divides the line segment joining $z_1$ and $z_2$ in the ratio of $5:8$ which also implies
  that these three points are collinear. Thus, $\startbmatrix\NC z_1 \NC \overline{z_1} \NC 1\NR\NC z_2 \NC
  \overline{z_2} \NC 1\NR\NC z_3 \NC \overline{z_3} \NC 1\NR\stopbmatrix = 0$
\item We know that length of perpendicular from $z_1$ to $\overline{a}z + a\overline{z} + b = 0$ is given by
  $\frac{|\overline{a}z_1 + a\overline{z_1} + b|}{2|a|}.$

  Thus desired length $= \frac{|(2 - 3i)(3 + 4i) + (2 + 3i)(3 - 4i) + 9|}{2|3 - 4i|} = \frac{45}{10} =
  \frac{9}{2}$.
\item
  \startplacefigure[location={left, none}]
    \startMPcode
      draw (-1cm, 0cm) -- (1cm, 0cm);
      draw (0cm, -1cm) -- (0cm, 1cm);
      label.bot("$z_1$", (0cm, -1cm));
      label.top("$z_2$", (0cm, 1cm));
      label.urt("$\frac{z_1 + z_2}{2}$", (0cm, 0cm));
      label.lrt("$b\overline{z} + \overline{b}z = c$", (0cm, 0cm));
      pickup pencircle scaled 2pt;
      drawdot (0cm, 1cm);
      drawdot (0cm, -1cm);
    \stopMPcode
  \stopplacefigure
  Since mid-point lies on the given line, therefore $b\left(\frac{\overline{z_1} + \overline{z_2}}{2}\right) +
  \overline{b}\left(\frac{z_1 + z_2}{2}\right) = c$

  Since line segment joining $z_1$ and $z_2$ is perpedicular to the given line therefore, Slope of $z_1z_2$ + Slope of line = 0

  $\Rightarrow \frac{z_2 - z_1}{\overline{z_2} - \overline{z_1}} - \frac{b}{\overline{b}} = 0$

  Solving these two equations, we get $\overline{b}z_2 + b\overline{z_1} = c$.
\item   Let $z = 2 - i$ then after rotation new point would be $z.e^{i\pi/2} = (2 - i)\left(\cos\frac{\pi}{2} +
  i\sin\frac{\pi}{2}\right) = (2 - i)i = 1 + 2i$.
\item Coordinate of $z_0$ after moving $5$ points horizontally and $3$ points vertically away from starting pont would be
  $6 + 5i$.

  It then moves in the direction of vecor $\hat{i} + \hat{j}$ for $\sqrt{2}$ units. This vector makes angle $\pi/4$ with
  $x$-axis. So new coordinate would be $6 + \sqrt{2}\cos\pi/4 + 5 + \sqrt{2}\sin\pi/4 = 7 + 6i$.

  It then rotates by angle $\pi/2$ so new coordinate would be $(7 + 6i)e^{i\pi/2} = (7 + 6i)i = -6 + 7i$.
\item North-East direction makes angle of $\pi/4$ with $x$-axis. So coordinates of point $3$ units from origin in
  North-East direction $= 3.e^{i\pi/4} = 3\left(\cos\frac{\pi}{4} + i\sin\frac{\pi}{4}\right) = \frac{3}{\sqrt{2}} +
  i\frac{3}{\sqrt{2}}$.

  North-West direction makes angle of $3\pi/4$ with $x$-axis. A disaplacement of $4$ units in this direction will mean a shift in
  coordinates by $4.e^{i3\pi/4} = 4\left(\cos\frac{3\pi}{4} + i\sin\frac{3\pi}{4}\right) = -\frac{4}{\sqrt{2}} +
  i\sin\frac{4}{\sqrt{2}}$.

  Thus, final coordiate would be sum of the above two i.e. $-\frac{1}{\sqrt{2}} + i\frac{7}{\sqrt{2}}$.
\item Given, $\frac{z_1 - z_3}{z_2 - z_3} = \frac{1 - i\sqrt{3}}{2} = \frac{1 - i\sqrt{3}}{2}.\frac{1 +
    i\sqrt{3}}{2}$

  $= \frac{1 + 3}{2(1 + i\sqrt{3})}= \frac{2}{1 + i\sqrt{3}}$

  $\Rightarrow \frac{z_2 - z_3}{z_1 - z_3} = \frac{1 + i\sqrt{3}}{2} = \cos\frac{\pi}{3} + i\sin\frac{\pi}{3}$

  $\Rightarrow \left|\frac{z_2 - z_3}{z_1 - z_3}\right| = 1$ and $\arg\left(\frac{z_2 - z_3}{z_1 -
  z_3}\right) = \frac{\pi}{3}$

  Hence, the triangle is equilateral.
\item Since sides of an equilateral triangle make an angle of $60^\circ$ with each other, therefore
  $\frac{z_3 - z_1}{z_2 - z_1} = \cos60^\circ \pm \sin60^\circ = \frac{1 \pm i\sqrt{3}}{2}$

  $\Rightarrow 2z_3 - 2z_1 + z1 - z_2 = \pm i(z_2 - z_1)\sqrt{3}\Rightarrow (2z_3 - z_1 - z_2)^2 = 3(z_2 -
  z_1)^2\Rightarrow z_1^2 + z_2^2 + z_3^2 = z_1z_2 + z_2z_3 +z_3z_1$

  $\Rightarrow z_1z_2 + z_2z_3 + z_3z_1 - z_z^2 - z_2^2 - z_3^2 + z_1z_2 - z_1z_2 + z_2z_3 - z_2z_3 + z_1z_3
  - z_1z_3 = 0$

  $\Rightarrow (z_1 - z_2)(z_2 - z_3) + (z_2 - z_3)(z_3 - z_1) + (z_3 - z_1)(z_1 - z_2) =
  0\Rightarrow \frac{1}{z_1 - z_2} + \frac{1}{z_2 - z_3} + \frac{1}{z_3 - z_1} = 0$.
\item Since it is an equilateral triangle, therefore centroid and circumcenters would be
  identical. $\therefore z_0 = \frac{z_1+ z_2 + z_3}{3}$

  Since it is an equilateral triangle, we have just proven that $z_1^2 + z_2^2 + z_3^2 = z_1z_2 + z_2z_3
  +z_3z_1$

  From first equation, we have $\Rightarrow 9z_0^2 = z_1^2 + z_2^2 + z_3^2 + 2(z_1z_2 + z_2z_3 +z_3z_1)$

  $\Rightarrow 9z_0^2 = z_1^2 + z_2^2 + z_3^2 + 2(z_1^2 + z_2^2 + z_3^2)\Rightarrow 3z_0^2 = z_1^2 + z_2^2 +
  z_3^2$.
\item Since right angle is at $z_3,$ therefore $\frac{z_2 - z_3}{z_1 - z_3} = e^{i\pi/2} = i\Rightarrow (z_2
  - z_3)^2 = -(z_1 - z_3)^2 \Rightarrow z_2^2 + z_3^2 - 2z_2z_3 = -z_1^2 - z_3^2 + 2z_1z_3$

  $\Rightarrow z_1^2 + z_2^2 - 2z_1z_2 = -2z_3^2 + 2z_2z_3 + 2z_1z_3 - 2z_1z_2\Rightarrow (z_1 - z_2)^2 =
  2(z_1 - z_3)(z_3 - z_2)$.
\stopitemize