% -*- mode: context; -*-
\chapter{Determinants}
\startitemize[n, 1*broad]
\item Let $\Delta = \startdeterminant\NC  4 \NC 9 \NC 7\NR\NC 3 \NC 5 \NC 7\NR\NC 5 \NC 4 \NC
  5\NR\stopdeterminant$

  $\Delta = \startdeterminant\NC  1 \NC 4 \NC 0\NR\NC 3 \NC 5 \NC 7\NR\NC 5 \NC 4 \NC
  5\NR\stopdeterminant[R_1\rightarrow R_1 - R_2] = \startdeterminant\NC  1 \NC 4 \NC 0\NR\NC 0 \NC -7 \NC
  7\NR\NC 0 \NC -16 \NC 5\NR\stopdeterminant[R_2\rightarrow R_2 - 3R_1$ and $R_3\rightarrow R_3 - 5R_1]$

  $= 1(-35 + 112) = 77$.
\item Let $\Delta = \startdeterminant\NC  1 \NC a\NC a^2\NR\NC 1 \NC b \NC b^2\NR\NC 1 \NC c \NC
  c^2\NR\stopdeterminant$

  $= \startdeterminant\NC  0 \NC a - b\NC a^2 - b^2\NR\NC 0 \NC b - c \NC b^2 - c^2\NR\NC 1 \NC c \NC
  c^2\NR\stopdeterminant[R_1\rightarrow R_1 - R_2; R_2\rightarrow R_2 - R_3]$

  $= (a - b)(b - c)\startdeterminant\NC  0 \NC 1\NC a + b\NR\NC 0 \NC 1 \NC b + c\NR\NC 1 \NC c \NC
  c^2\NR\stopdeterminant = (a - b)(b - c)(b + c - a - b) = (a - b)(b - c)(c - a)$.
\item Let $a = 2, b = 3, c = 4$ then $\Delta = \startdeterminant\NC  1 \NC 2\NC 4\NR\NC 1 \NC 3 \NC 9\NR\NC
  1 \NC 4 \NC 16\NR\stopdeterminant = \startdeterminant\NC  1 \NC a\NC a^2\NR\NC 1 \NC b \NC b^2\NR\NC 1 \NC
  c \NC c^2\NR\stopdeterminant$

  We can solve this like previous problem which gives us $\Delta = (2 - 3)(3 - 4)(4 - 2) = 2$.
\item Let $\Delta = \startdeterminant\NC  4 \NC 9 \NC 2\NR\NC 3 \NC 5 \NC 7\NR\NC 8 \NC 1 \NC
  6\NR\stopdeterminant =\startdeterminant\NC  1 \NC 4 \NC -5\NR\NC 3 \NC 5 \NC 7\NR\NC 8 \NC 1 \NC
  6\NR\stopdeterminant[R_1\rightarrow R_1 - R_2]$

  $= \startdeterminant\NC  1 \NC 4 \NC -5\NR\NC 0 \NC -7 \NC 22\NR\NC 0 \NC -31 \NC
  46\NR\stopdeterminant[R_2\rightarrow R_2 - 3R_1; R_3\rightarrow R_3 - 5R_1]$

  $= (-322 + 682) = 360$.
\item Let $\Delta = \startdeterminant\NC  18 \NC 1 \NC 17\NR\NC 22 \NC 3 \NC 19\NR\NC 26 \NC 5 \NC
  21\NR\stopdeterminant = \startdeterminant\NC  18 \NC 1 \NC 17\NR\NC 4 \NC 2 \NC 2\NR\NC 4 \NC 2 \NC
  2\NR\stopdeterminant[R_3\rightarrow R_3 - R_2; R_2\rightarrow R_2 - R_1]$

  $= 0$(bceuase two rows are identical).
\item Let $\Delta = \startdeterminant\NC  4 \NC 9 \NC 7\NR\NC 3 \NC 5 \NC 7\NR\NC 5 \NC 4 \NC
  5\NR\stopdeterminant = \startdeterminant\NC  1 \NC 4 \NC 0\NR\NC 3 \NC 5 \NC 7\NR\NC 5 \NC 4 \NC
  5\NR\stopdeterminant[R_1\rightarrow R_2 - R_1]$

  $= \startdeterminant\NC 1 \NC 4 \NC 0\NR\NC 0 \NC -7 \NC 7\NR\NC 0 \NC -16 \NC
  5\NR\stopdeterminant[R_2\rightarrow R_2 - 3R_1;R_3\rightarrow R_3 - R_1]$

  $= (-35 + 112) = 77$.
\item $\Delta = \startdeterminant\NC 1 \NC 4\NC 9\NR\NC 3\NC 5\NC 7\NR\NC 5 \NC 7\NC
  9\NR\stopdeterminant[R_3\rightarrow R_3 - R_2; R_2\rightarrow R_2 - R_1]= \startdeterminant\NC 1\NC
  4\NC 9\NR\NC 3\NC 5\NC 7\NR\NC 2\NC 2\NC
  2\NR\stopdeterminant[R_3\rightarrow R_3 - R_2]$

  $=\startdeterminant\NC 1\NC 3\NC 5\NR\NC 3\NC 2\NC 2\NR\NC 2\NC 0\NC
  0\NR\stopdeterminant[C_3\rightarrow C_3 - C_2; C_2\rightarrow C_2 - C_1] = 2(6 - 10) = -8$.
\item Let $\Delta = \startdeterminant\NC  a \NC b \NC c\NR\NC b \NC c \NC a\NR\NC c \NC a \NC
  b\NR\stopdeterminant = \startdeterminant\NC  a + b + c \NC b \NC c\NR\NC b + c + a \NC c \NC a\NR\NC c + a
  + b\NC a \NC b\NR\stopdeterminant[C_1\rightarrow C_1 + C_2 + C_3]$

  $= (a + b + c)\startdeterminant\NC 1 \NC b \NC c\NR\NC 1 \NC c \NC a\NR\NC 1 \NC a \NC
  b\NR\stopdeterminant = (a + b + c)\startdeterminant\NC  1 \NC b \NC c\NR\NC 0 \NC c - b \NC a - c\NR\NC 0
  \NC a - b \NC b - c\NR\stopdeterminant[R_2\rightarrow R_2 - R_1; R_3\rightarrow R_3 - R_1]$

  $= (a + b + c)[(c - b)(b - c) - (a - b)(a - c)] = (a + b + c)(ab + bc + ca - a^2 - b^2 - c^2) =
  -\frac{1}{2}(a + b + c)[(a - b)^2 + (b - c)^2 + (c - a)^2]$

  Given that $a, b, c$ are positive so $a + b + c > 0$ and since $a, b, c$ are unequal so $(a - b)^2 + (b -
  c)^2 + (c - a)^2 > 0$. Thus, $\Delta < 0$.
\item Let $\Delta = \startdeterminant\NC  b + c\NC a + b \NC a\NR\NC c + a \NC b + c \NC b\NR\NC a + b \NC c
  + a \NC c\NR\stopdeterminant = \startdeterminant\NC  a + b + c\NC a + b \NC a\NR\NC b + c + a \NC b + c
  \NC b\NR\NC a + b + c \NC c + a \NC c\NR\stopdeterminant[C_1\rightarrow C_1 + C_3]$

  $= (a + b + c)\startdeterminant\NC 1\NC a + b \NC a\NR\NC 1 \NC b + c \NC b\NR\NC 1 \NC c + a \NC
  c\NR\stopdeterminant = (a + b + c)\startdeterminant\NC 1\NC a + b \NC a\NR\NC 0 \NC c - a \NC b - a\NR\NC
  0 \NC c - b \NC c - a\NR\stopdeterminant[R_2\rightarrow R_2 - R_1; R_3\rightarrow R_3 - R_1]$

  $= (a + b + c)[(c - a)^2 - (c - b)(b - a)] = (a + b + c)(a^2 + b^2 + c^2 - ab - bc - ca) = a^3 + b^3 + c^3
  - 3abc$.
\item $\Delta = \startdeterminant\NC  1 + a_1 + a_2 + a_3\NC a_2 \NC a_3\NR\NC 1 + a_1 + a_2 + a_3 \NC 1+
  a_2 \NC a_3\NR\NC 1 + a_1 + a_2 + a_3 \NC a_2 \NC 1 + a_3\NR\stopdeterminant[C_1\rightarrow C_1 + C_2 +
  C_3]$

  $= (1 + a_1 + a_2 + a_3)\startdeterminant\NC  1\NC a_2 \NC a_3\NR\NC 1\NC 1 + a_2 \NC a_3\NR\NC 1\NC a_2
  \NC 1 + a_3\NR\stopdeterminant = (1 + a_1 + a_2 + a_3)\startdeterminant\NC  0\NC -1 \NC 0\NR\NC 0\NC 1 \NC
  -1\NR\NC 1\NC a_2 \NC 1 + a_3\NR\stopdeterminant[R_1\rightarrow R_1 - R_2; R_2\rightarrow R_2 - R_3]$

  $= (1 + a_1 + a_2 + a_3)\startdeterminant\NC -1 \NC 0\NR\NC 1\NC -1\NR\stopdeterminant = 1 + a_1 + a_2
  + a_3$.
\item $\Delta = \startdeterminant\NC  2a + 2b + 2c \NC a \NC b\NR\NC 2a + 2b + 2c \NC b + c + 2a \NC b\NR\NC
  2a + 2b + 2c \NC a \NC c + a + 2b\NR\stopdeterminant[C_1\rightarrow C_1 + C_2 + C_3]$

  $= 2(a + b + c)\startdeterminant\NC  1 \NC a \NC b\NR\NC 1 \NC b + c + 2a \NC b\NR\NC 1 \NC a \NC c + a +
  2b\NR\stopdeterminant$

  $= 2(a + b + c)\startdeterminant\NC  0 \NC -(a + b + c) \NC 0\NR\NC 0 \NC b + c + a
  \NC -(a + b + c)\NR\NC 1 \NC a \NC c + a + 2b\NR\stopdeterminant [R_1\rightarrow R_1 - R_2; R_2\rightarrow
  R_2 - R_3]$

  $= 2(a + b + c)^3\startdeterminant\NC 0\NC -1\NC 0\NR\NC 0\NC 1\NC -1\NR\NC 1\NC a\NC c + a +
  2b\NR\stopdeterminant$(taking $a + b + c$ common from first and second row)

  $= 2(a + b + c)^3[-1.-1 - 1.0] = 2(a + b + c)^3$.
\item $\Delta = \startdeterminant\NC 2a\NC 2b \NC a - b - c\NR\NC 2b\NC 2c\NC b - c - a\NR\NC 2c\NC 2a\NC
  c - a - b\NR\stopdeterminant[C_1\rightarrow C_1 + C_2; C_2\rightarrow C_2 - C_3]$

  $= 4\startdeterminant\NC a\NC b \NC a - b - c\NR\NC b\NC c\NC b - c - a\NR\NC c\NC a\NC
  c - a - b\NR\stopdeterminant$

  $= 4\startdeterminant\NC a \NC b\NC c\NR\NC b\NC c\NC a\NR\NC c\NC a\NC b\NR\stopdeterminant = 4(a + b
  + c)(a^2 + b^2 + c^2 - ab - bc - ca) = 4(a^3 + b^3 + c^3 - 3abc)$.
\item $\Delta = \startdeterminant\NC  a + b + c \NC a + b + c \NC a + b + c\NR\NC 2b \NC b - c - a\NR\NC 2c
  \NC 2c \NC c - a - b\NR\stopdeterminant[R_1\rightarrow R_1 + R_2 + R_3]$

  $= (a + b + c)\startdeterminant\NC 1 \NC 1 \NC 1\NR\NC 2b \NC b - c - a\NC 2b\NR\NC 2c \NC 2c\NC c - a - b
  \NR\stopdeterminant$

  $= (a + b + c)\startdeterminant\NC 1 \NC 0 \NC 0\NR\NC 2b \NC -b - c - a\NC 0\NR\NC 2c \NC 0\NC -c - a - b
  \NR\stopdeterminant[C_1\rightarrow C_2 - C_1; C_3\rightarrow C_3 - C_1]$

  $= (a + b + c)\startdeterminant\NC -b - c - a\NC 0\NR\NC 0 \NC -c - a - b\NR\stopdeterminant = (a + b +
  c)^3$.
\item $\Delta = \frac{1}{x.y.z}\startdeterminant\NC x^2\NC y^2\NC z^2\NR\NC x^3\NC y^3\NC z^3\NR\NC xyz\NC
  xyz\NC xyz\NR\stopdeterminant[C_1\rightarrow xC_1; C_2\rightarrow yC_2; C_3\rightarrow zC_3]$

  $= \frac{xyz}{xyz}\startdeterminant\NC x^2\NC y^2\NC z^2\NR\NC x^3\NC y^3\NC z^3\NR\NC 1\NC
  1\NC 1\NR\stopdeterminant = \startdeterminant\NC1 \NC 1\NC 1\NR\NC x^2 \NC y^2\NC z^2\NR\NC x^3 \NC y^3\NC
  z^3\NR\stopdeterminant$(exchanging rows twice)

  $= \startdeterminant\NC 1\NC 0\NC\NR\NC x^2 \NC y^2-x^2 \NC z^2 - x^2\NR\NC x^3\NC y^3 - x^3\NC z^3 -
  x^3\NR\stopdeterminant[C_2\rightarrow C_2 - C_1; C_3\rightarrow C_3 - C_1]$

  $= \startdeterminant\NC(y - x)(y + x)\NC (z - x)(z + x)\NR\NC (y - x)(y^2 + xy + x^2)\NC (z - x)(z^2 + zx
  + x^2)\NR\stopdeterminant$

  $= (y - x)(z - x)\startdeterminant\NC y + x\NC z + x\NR\NC y^2 + xy + y^2\NC z^2 + zx +
  x^2\NR\stopdeterminant$

  $= (y - x)(z - x)\startdeterminant\NC y + x\NC z - y\NR\NC y^2 + yx + x^2 \NC (z^2 - y^2) + zx -
  zy\NR\stopdeterminant[C_2\rightarrow C_2 - C_1]$

  $= (y - x)(z - x)(z - y)\startdeterminant\NC y + z\NC 1\NR\NC y^2 + xy + x^2 \NC x + y +
  z\NR\stopdeterminant$

  $= (y - x)(z - x)(z - y)[(y + x)(x + y + z) - y^2 - xy - x^2] = (x - y)(y - z)(z - x)(xy + yz + zx)$.
\item $\Delta = \frac{1}{abc}\startdeterminant\NC  a(a^2 + 1) \NC ab^2 \NC ac^2\NR\NC a^2b \NC b(b^2 + 1)
  \NC bc^2\NR\NC a^2c \NC b^2c \NC c(c^2 + 1)\NR\stopdeterminant[C_1\rightarrow aC_1; C_2\rightarrow bC_2;
  C_3\rightarrow cC_3]$

  $= \frac{abc}{abc}\startdeterminant\NC a^2 + 1\NC b^2\NC c^2\NR\NC a^2 \NC b^2 + 1\NC c^2\NR\NC a^2 \NC
  b^2\NC c^2 + 1\NR\stopdeterminant$

  $= \startdeterminant\NC 1 + a^2 + b^2 + c^2\NC b^2\NC c^2\NR\NC 1 + a^2 + b^2 + c^2 \NC b^2 + 1\NC
  c^2\NR\NC 1 + a^2 + b^2 + c^2 \NC b^2\NC c^2 + 1\NR\stopdeterminant[C_1\rightarrow C_1 + C_2 + C_3]$

  $= (1 + a^2 + b^2 + c^2)\startdeterminant\NC 1\NC b^2\NC c^2\NR\NC 1\NC b^2 + 1\NC c^2\NR\NC 1\NC b^2\NC
  c^2 + 1\NR\stopdeterminant$

  $= (1 + a^2 + b^2 + c^2)\startdeterminant\NC 1 \NC b^2\NC c^2\NR\NC 0 \NC 1\NC 0\NR\NC 0 \NC 0\NC
  1\NR\stopdeterminant[R_2\rightarrow R_2 - R_1; R_3\rightarrow R_3 - R_1]$

  $= (1 + a^2 + b^2 + c^2)\startdeterminant\NC1 \NC 0\NR\NC 0\NC 1\NR\stopdeterminant = 1 + a^2 + b^2 +
  c^2$.
\item $\Delta = a_1a_2a_3\startdeterminant\NC \frac{1}{a_1} + 1\NC \frac{1}{a_2}\NC \frac{1}{a_3}\NR\NC
  \frac{1}{a_1}\NC \frac{1}{a_2} + 1\NC \frac{1}{a_3}\NR\NC \frac{1}{a_1}\NC \frac{1}{a_2} \NC \frac{1}{a_3}
  + 1\NR\stopdeterminant[C_1\rightarrow \frac{C_1}{a_1}; C_2\rightarrow \frac{C_2}{a_2}; C_3\rightarrow
  \frac{C_3}{a_3}]$

  $= a_1a_2a_2\startdeterminant\NC 1 + \frac{1}{a_1} + \frac{1}{a_2} + \frac{1}{a_3} \NC \frac{1}{a_2}\NC
  \frac{1}{a_3}\NR\NC 1 + \frac{1}{a_1} + \frac{1}{a_2} + \frac{1}{a_3}\NC \frac{1}{a_2} + 1\NC
  \frac{1}{a_3}\NR\NC 1 + \frac{1}{a_1} + \frac{1}{a_2} + \frac{1}{a_3}\NC \frac{1}{a_2}\NC \frac{1}{a_3} +
  1\NR\stopdeterminant[C_1\rightarrow C_1 + C_2 + C_3]$

  $= a_1a_2a_3\left(1 + \frac{1}{a_1} + \frac{1}{a_2} + \frac{1}{a_3}\right)\startdeterminant\NC 1 \NC
  \frac{1}{a_2}\NC \frac{1}{a_3}\NR\NC 1\NC \frac{1}{a_2} + 1\NC \frac{1}{a_3}\NR\NC 1\NC \frac{1}{a_2}\NC
  \frac{1}{a_3} + 1\NR\stopdeterminant$

  $= a_1a_2a_3\left(1 + \frac{1}{a_1} + \frac{1}{a_2} + \frac{1}{a_3}\right)\startdeterminant\NC1
  \NC\frac{1}{a_2}\NC \frac{1}{a_3}\NR\NC 0\NC 1\NC 0\NR\NC 0\NC 0\NC 1\NR\stopdeterminant[R_2\rightarrow
  R_2 - R_1; R_3\rightarrow R_3 - R_1]$

  $= a_1a_2a_3\left(1 + \frac{1}{a_1} + \frac{1}{a_2} + \frac{1}{a_3}\right)$.
\item $\Delta = \startdeterminant\NC  x \NC x^2 \NC 1 + x^3\NR\NC y \NC y^2 \NC 1 + y^3\NR\NC z \NC z^2 \NC
  1 + z^3\NR\stopdeterminant = \startdeterminant\NC  x \NC x^2 \NC x^3\NR\NC y \NC y^2 \NC y^3\NR\NC z \NC
  z^2 \NC z^3\NR\stopdeterminant + \startdeterminant\NC  x \NC x^2 \NC 1\NR\NC y \NC y^2 \NC 1\NR\NC z \NC
  z^2 \NC 1\NR\stopdeterminant$

  $= xyz\startdeterminant\NC 1\NC x\NC x^2\NR\NC 1\NC y\NC y^2\NR\NC 1\NC z\NC z^2\NR\stopdeterminant +
  \startdeterminant\NC 1\NC x\NC x^2\NR\NC 1\NC y\NC y^2\NR\NC 1\NC z\NC z^2\NR\stopdeterminant$(by
  exchanging two columns)

  $= (1 + xyz)\startdeterminant\NC 1\NC x\NC x^2\NR\NC 1\NC y\NC y^2\NR\NC 1\NC z\NC z^2\NR\stopdeterminant
  = (1 + xyz)(x - y)(y - z)(z - x)$(from the value of a circular determinant)

  $\Delta = 0 \Rightarrow 1 + xyz = 0[\because x\neq y;y\neq z;z\neq x]\Rightarrow xyz = -1$.
\item $\Delta = \startdeterminant\NC 0 \NC -2c \NC -2b\NR\NC b \NC c + a\NC b\NR\NC a\NC c\NC a +
  b\NR\stopdeterminant[R_1\rightarrow R_1 - R_2 - R_3]$

  $= \frac{1}{c}\startdeterminant\NC 0 \NC -2c\NC -2b\NR\NC 0\NC c(c + a - b)\NC b(c - a - b)\NR\NC c\NC
  c\NC a + b\NR\stopdeterminant[R_2\rightarrow cR_2 - bR_3]$

  $= \frac{1}{c}c(-2bc)[c - a - b - (c + a - b)] = -2bc.-2a = 4abc$.
\item $\Delta = \startdeterminant\NC(b + c)^2 - a^2\NC 0\NC a^2\NR\NC b^2 - (c + a)^2\NC (c + a)^2 - b^2\NC
  b^2\NR\NC 0\NC c^2 - (a + b)^2 \NC (a + b)^2\NR\stopdeterminant[C_1\rightarrow C_1 - C_2; C_2\rightarrow
  C_2 - C_3]$

  $= (a + b + c)^2\startdeterminant\NC b + c - a\NC 0\NC a^2\NR\NC b+ c - a\NC c + a - b\NC b^2\NR\NC 0 \NC
  c - a - b\NC (a + b)^2\NR\stopdeterminant$

  $= (a + b + c)^2\startdeterminant\NC b + c - a\NC 0\NC a^2\NR\NC b+ c - a\NC c + a - b\NC b^2\NR\NC 2a -
  2b\NC -2a\NC 2ab\NR\stopdeterminant[R_3\rightarrow R_3 - R_1 - R_2]$

  $= (a + b + c)^2\startdeterminant\NC b + c - a\NC 0\NC a^2\NR\NC 0\NC c + a - b\NC b^2\NR\NC -2b\NC -2a\NC
  2ab\NR\stopdeterminant[C_1\rightarrow C_1 + C_2]$

  $= \frac{(a + b + c)^2}{ab}\startdeterminant\NC a(b + c)\NC a^2\NC a^2\NR\NC b^2\NC b(c + a) \NC b^2\NR\NC
  0\NC 0\NC 2ab\NR\stopdeterminant[C_1\rightarrow aC_1 + C_3; C_2\rightarrow bC_2 + C_3]$

  $= \frac{(a + b + c)^2}{ab}.ab.2ab\startdeterminant\NC b + c\NC a\NC a\NR\NC b\NC c + a\NC b\NR\NC 0\NC
  0\NC 1\NR\stopdeterminant$

  $= 2ab(a + b + c)^2[(b + c)(c + a) - ab] = 2abc(a + b + c)^3$.
\item $\Delta = \startdeterminant\NC 15 - x\NC 1\NC 10\NR\NC -4 -2x\NC 0\NC 6\NR\NC -8\NC 0\NC
  3\NR\stopdeterminant = 0[R_2\rightarrow R_2 - R_1; R_3\rightarrow R_3 - R_1]$

  $\Rightarrow (12 + 6x - 48) = 0 \Rightarrow x = 6$.
\item $\Delta = \startdeterminant\NC a + b + c - x\NC c\NC b\NR\NC a + b + c - x\NC b - x\NC a\NR\NC a + b +
  c - x\NC a\NC c - x\NR\stopdeterminant = 0[C_1\rightarrow C_1 + C_2 + C_3]$

  $\Rightarrow \startdeterminant\NC -x\NC c\NC b\NR\NC -x\NC b - x\NC a\NR\NC -x\NC a\NC c -
  x\NR\stopdeterminant = 0[\because a + b + c = 0]$

  $\Rightarrow (-x)\startdeterminant\NC 1\NC c\NC b\NR\NC 1\NC b - x\NC a\NR\NC 1\NC a\NC c -
  x\NR\stopdeterminant = 0$

  $\Rightarrow \startdeterminant\NC1 \NC c\NC b\NR\NC 0\NC b - c - x\NC a - b\NR\NC 0\NC a- c\NC c - b -
  x\NR\stopdeterminant = 0[R_2\rightarrow R_2 - R_1; R_3\rightarrow R_3 - R_1]$

  $\Rightarrow x[(b - c - x)(c - b - x) - (a - c)(a - b)] = 0 \Rightarrow x(x^2 - a^2 - b^2 - c^2 + ab + bc
  + ca) = 0$

  $\Rightarrow x = 0; x^2 = (a^2 + b^2 + c^2) - \frac{1}{2}[(a + b + c)^2 - (a^2 + b^2 + c^2)] =
  \frac{3}{2}(a^2 + b^2 + c^2)[\because a + b + c = 0]\Rightarrow x = \pm\sqrt{\frac{3}{2}(a^2 + b^2 +
    c^2)}$.
\item $D_1 = \startdeterminant\NC  a \NC b \NC c\NR\NC d \NC e \NC f\NR\NC g \NC h \NC k\NR\stopdeterminant
  = \startdeterminant\NC  a \NC b \NC c\NR\NC tx \NC ty \NC tc\NR\NC g \NC h \NC k\NR\stopdeterminant$

  $= t\startdeterminant\NC  a \NC b \NC c\NR\NC x \NC y \NC z\NR\NC g \NC h \NC k\NR\stopdeterminant =
  t\startdeterminant\NC  a \NC x \NC g\NR\NC b \NC y \NC h\NR\NC c \NC z \NC k\NR\stopdeterminant$(changing
  rows into corresponding columns)

  $= -t\startdeterminant\NC a\NC g \NC x\NR\NC b \NC h \NC y\NR\NC c \NC k \NC z\NR\stopdeterminant =
  -tD_2$.
\item $\startdeterminant\NC  a \NC bc \NC abc \NR\NC b \NC ca \NC abc\NR\NC c \NC ab \NC
  abc\NR\stopdeterminant = \frac{1}{a^2b^2c^2}\startdeterminant\NC a^3\NC a^2bc\NC a^3bc\NR\NC b^3\NC
  ab^2c\NC ab^3c\NR\NC c^3\NC abc^2\NC abc^3\NR\stopdeterminant[R_1\rightarrow a^2R_1; R_2\rightarrow
  b^2R_2; R_3\rightarrow c^2R_3]$

  $= \frac{abc.abc}{a^2b^2c^2}\startdeterminant\NC a^3\NC a\NC a^2\NR\NC b^3\NC b\NC b^2\NR\NC c^3\NC c\NC
  c^2\NR\stopdeterminant$

  $= \startdeterminant\NC  a \NC a^2 \NC a^3\NR\NC b \NC b^2 \NC b^3\NR\NC c \NC c^2 \NC
  c^3\NR\stopdeterminant$(exchanging columns twice).
\item Let $x$ be the first term and $y$ the common ratio of the G.P. then

  $a = xy^{p - 1}\Rightarrow \log a = \log x + (p - 1)\log y; b = xy^{q - 1}\Rightarrow \log b = \log x + (q
  - 1)\log y; c = xy^{r - 1}\Rightarrow \log c = \log x + (r - 1)\log(r - 1)$

  $\Rightarrow \startdeterminant\NC \log a \NC p \NC 1\NR\NC \log b \NC q \NC 1\NR\NC \log c \NC r \NC
  1\NR\stopdeterminant = \startdeterminant\NC \log x + (p - 1)\log y \NC p \NC 1\NR\NC \log x + (q -
  1)\log y \NC q \NC 1\NR\NC \log x + (r - 1)\log y \NC r \NC 1\NR\stopdeterminant$

  $= \startdeterminant\NC (p - 1)\log y \NC p \NC 1\NR\NC (q - 1)\log y \NC q \NC 1\NR\NC (r - 1)\log y \NC
  r \NC 1\NR\stopdeterminant[C_1\rightarrow C_1 - \log x.C_3]$

  $= \log y\startdeterminant\NC p - 1\NC p\NC 1\NR\NC q - 1\NC q\NC 1\NR\NC r - 1\NC r\NC
  1\NR\stopdeterminant = \log y\startdeterminant\NC p\NC p\NC 1\NR\NC q\NC q\NC 1\NR\NC r\NC r\NC
  1\NR\stopdeterminant = 0[C_1\rightarrow C_1 + C_3]$.
\item $\Delta = \startdeterminant\NC 1\NC 0\NC 0\NR\NC 1\NC x\NC 0\NR\NC 1\NC 0\NC
  y\NR\stopdeterminant[C_2\rightarrow C_2 - C_1; C_3\rightarrow C_3 - C_1]$

  $= xy$.
\item $\Delta = \startdeterminant\NC 0 \NC 0 \NC 1\NR\NC a - b\NC b - c\NC c\NR\NC a^3 - b^3\NC b^3 - c^3\NC
  c^3\NR\stopdeterminant[C_1\rightarrow C_1 - C_2; C_2\rightarrow C_2 - C_3]$

  $= (a - b)(b - c)\startdeterminant\NC 0 \NC 0 \NC 1\NR\NC 1\NC 1\NC c\NR\NC a^2 + ab + b^2\NC b^2 + bc +
  c^2\NC c^3\NR\stopdeterminant$

  $= (a - b)(b - c)[b^2 + bc + c^2 - a^2 - ab - b^2] = (a - b)(b - c)[(c - a)(a + b + c)]$.
\item $\Delta = \startdeterminant\NC 0\NC b - a\NC b^2 - a^2\NR\NC 0\NC c - b\NC b^2 - c^2\NR\NC 1\NC a +
  b\NC a^2 + b^2\NR\stopdeterminant[R_1\rightarrow R_1 - R_2; R_2\rightarrow R_2 - R_3]$

  $= (b - a)(c - b)\startdeterminant\NC 0 \NC 1\NC b + a\NR\NC 0 \NC 1\NC c + b\NR\NC 1\NC a + b\NC a^2 +
  b^2\NR\stopdeterminant$

  $= (b - a)(c - b)(c + b - b - a) = (a - b)(b - c)(c - a)$.
\item $\Delta = \startdeterminant\NC 0 \NC a - b\NC a^2 - b^2 + ca - bc\NR\NC 0\NC b - c\NC b^2 - c^2 + ab -
  ca\NR\NC 1\NC c\NC c^2 - ab\NR\stopdeterminant[R_1\rightarrow R_1 - R_2; R_2\rightarrow R_2 - R_3]$

  $= (a - b)(b - c)\startdeterminant\NC 0\NC 1\NC a + b + c\NR\NC 0 \NC 1\NC a + b + c\NR\NC 1\NC c\NC c^2 -
  ab\NR\stopdeterminant$

  $= 0$(two rows are identical).
\item $\Delta = \frac{1}{abc}\startdeterminant\NC a\NC abc \NC abc(b + c)\NR\NC b \NC abc\NC abc(c + a)\NR\NC c \NC
  abc\NC abd(a + b)\NR\stopdeterminant[R_1\rightarrow aR_1; R_2\rightarrow bR_2; R_3\rightarrow cR_3]$

  $= \frac{a^2b^2c^2}{abc}\startdeterminant\NC a\NC 1\NC b + c\NR\NC b\NC 1\NC c + a\NR\NC c\NC 1 \NC a +
  b\NR\stopdeterminant= abc\startdeterminant\NC a - b\NC 0 \NC b - a\NR\NC b - c\NC 0\NC c - b\NR\NC c\NC
  1\NC a + b\NR\stopdeterminant[R_1\rightarrow R_1 - R_2; R_2\rightarrow R_2 - R_3] = 0$.
\item Check previous problem.
\item Let $x$ be the first term and $d$ be the common difference of the corresponding A.P. Then4

  $\frac{1}{a} = x + (p - 1)d; \frac{1}{b} = x + (q - 1)d; \frac{1}{c} = x + (r - 1)d$

  $\Rightarrow \frac{1}{a} - \frac{1}{b} = \frac{b - a}{ab} = (p - q)d; \frac{c - b}{bc} = (q - r)d$.

  $\Delta = \startdeterminant\NC c(b - a) \NC p - q \NC 0\NR\NC a(c - b)\NC q - r\NC 0\NR\NC ab\NC r\NC
  1\NR\stopdeterminant[R_1\rightarrow R_1 - R_2; R_2\rightarrow R_2 - R_3]$

  $= \startdeterminant\NC c(b - a) \NC \frac{b - a}{abd} \NC 0\NR\NC a(c - b)\NC \frac{c - b}{bcd}\NC 0\NR\NC
  ab\NC r\NC 1\NR\stopdeterminant = (b - a)(c - b)\startdeterminant\NC c \NC \frac{1}{abd} \NC 0\NR\NC a\NC
  \frac{1}{bcd}\NC 0\NR\NC ab\NC r\NC 1\NR\stopdeterminant = 0$
\item $\Delta = \startdeterminant\NC 0\NC -1\NC 3\NR\NC 1\NC 1\NC -4\NR\NC -2 \NC 4\NC
  0\NR\stopdeterminant$(putting $x = 0$) $= \startdeterminant\NC 0\NC -1\NC 0\NR\NC 1\NC 1\NC -1\NR\NC -2\NC
  4\NC 12\NR\stopdeterminant[C_3\rightarrow C_3 + 3C_2]$

  $=10 = t$
\item $\Delta = abc\startdeterminant\NC 1 \NC 1 \NC 1\NR\NC a \NC b \NC c\NR\NC a^2 \NC b^2 \NC
  c^2\NR\stopdeterminant$

  We have evaluated this new determinant equal to $(a - b)(b - c)(c - a)$ earlier and thus our required
  result is proven.
\item We can write $b = a + d$ and $c = a + 2d$, where $d$ is the common difference of the A.P. Then,

  $\Delta = \startdeterminant\NC  x + 1 \NC x + 2 \NC x + a\NR\NC  1 \NC 1 \NC d \NR\NC  1 \NC 1 \NC
  d\NR\stopdeterminant[R_3\rightarrow R_3 - R_2; R_2\rightarrow R_2 - R_1]$

  $= 0$(because two rows are equal).
\item $\Delta = \startdeterminant\NC  1 + \omega + \omega^2 \NC 1 + \omega + \omega^2 \NC 1 + \omega +
  \omega^2\NR\NC \omega \NC \omega^2 \NC 1\NR\NC \omega \NC 1 \NC
  \omega^2\NR\stopdeterminant[R_1\rightarrow R_1 + R_2 + R_3]$

  $= 0$(because $1 + \omega + \omega^2 = 0$).
\item $\Delta = k\startdeterminant\NC  1 \NC 1 \NC 1\NR\NC 1 \NC 2 \NC 3\NR\NC  1\NC 3 \NC
  6\NR\stopdeterminant = \startdeterminant\NC  1 \NC 1 \NC 1\NR\NC 0 \NC 1 \NC 2\NR\NC  0\NC 1 \NC
  3\NR\stopdeterminant[R_3\rightarrow R_3 - R_2; R_2\rightarrow R_2 - R_1]$

  $= k$.
\item $\Delta = \startdeterminant\NC  a^2 + b^2 + c^2 + x \NC b^2 \NC c^2\NR\NC a^2 + b^2 + c^2 + x \NC b^2
  + x \NC c^2\NR\NC a^2 + b^2 + c^2 + x \NC b^2 \NC c^2 + x\NR\stopdeterminant[C_1\rightarrow C_1 + C_2 +
  C_3]$

  $= (a^2 + b^2 + c^2 + x)\startdeterminant\NC  1\NC b^2 \NC c^2\NR\NC 1\NC b^2
  + x \NC c^2\NR\NC 1\NC b^2 \NC c^2 + x\NR\stopdeterminant = \startdeterminant\NC  1\NC b^2 \NC c^2\NR\NC 0
  \NC x\NC 0\NR\NC 0\NC 0\NC x\NR\stopdeterminant[R_3\rightarrow R_3 - R_1; R_2\rightarrow R_2 - R_1]$

  $= (a^2 + b^2 + c^2 + x)x^2$.
\item $\Delta = \startdeterminant\NC a - b \NC b - c\NC a^2 - b^2\NR\NC b  - c\NC c - b\NC b^2 - c^2\NR\NC
  c\NC a + b\NC c^2\NR\stopdeterminant[R_1\rightarrow R_1 - R_2; R_2\rightarrow R_2 - R_3]$

  $= (a - b)(b - c)\startdeterminant\NC 0 \NC 0 \NC a - c\NR\NC 1\NC -1\NC b + c\NR\NC c\NC a + b\NC
  c^2\NR\stopdeterminant[R_1\rightarrow R_1 - R_2] = -(a - b)(b - c)(c - a)(a + b + c)$.
\item $\Delta = \startdeterminant\NC  a + b + c \NC a - b \NC a\NR\NC b + c + a \NC b - c \NC b\NR\NC c +a +
  b \NC c - a \NC c\NR\stopdeterminant[C_1\rightarrow C_1 + C_3]$

  $= (a + b + c)\startdeterminant\NC 0\NC a + c - 2b\NC a - b\NR\NC 0\NC b + a - 2c\NC b - c\NR\NC 1\NC c -
  a\NC c\NR\stopdeterminant[R_1\rightarrow R_1 - R_2; R_2\rightarrow R_2 - R_3]$

  $= (a + b + c)(ab + bc + ca - a^2 - b^2 - c^2) = 3abc - a^3 - b^3 - c^3$.
\item $\Delta = 2(a + b + c)\startdeterminant\NC 1 \NC b + c \NC c + a\NR\NC 1 \NC c + a \NC a + b\NR\NC 1
  \NC a + b \NC b + c\NR\stopdeterminant[C_1\rightarrow C_1 + C_2 + C_3]$

  $= 2(a + b + c)\startdeterminant\NC 0\NC b - a\NC c - b\NR\NC 0\NC c - b\NC a - c\NR\NC 1\NC a + b\NC a +
  c\NR\stopdeterminant[R_1\rightarrow R_1 - R_2; R_2\rightarrow R_2 - R_3]$

  $= 2(a + b + c)(ab + bc + ca - a^2 - b^2 - c^2) = -2(a^3 + b^3 + c^3 - 3abc)$.
\item $\Delta = \startdeterminant\NC a - b \NC b - c\NC x + c\NR\NC a - b\NC b - c\NC x + b\NR\NC a - b\NC b
  - c\NC x + c\NR\stopdeterminant[C_1\rightarrow C_1 - C_2; C_2\rightarrow C_2 - C_3]$

  $= 0$(because two columns are equal).
\item Multiplying each row with $-1$ gives us

  $\Delta = -\startdeterminant\NC  0 \NC q - p \NC r - p\NR\NC  p - q \NC 0 \NC r - q\NR\NC  p - r \NC q - r
  \NC 0\NR\stopdeterminant$

  Changing rows into corresponding columns

  $\Delta = -\startdeterminant\NC  0 \NC p - q \NC p - r\NR\NC  q - p \NC 0 \NC q - r\NR\NC  r - p \NC r - q
  \NC 0\NR\stopdeterminant = -\Delta \Rightarrow \Delta = 0$.
\item Let L.H.S.\ $= \Delta = (3a + 3b)\startdeterminant\NC 1\NC a + b\NC a + 2b\NR\NC 1\NC a\NC a + b\NC\NR 1\NC a +
  2b\NC a\NR\stopdeterminant[C_1\rightarrow C_1 + C_2 + C_3]$

  $= (3a + 3b)\startdeterminant\NC 0\NC b\NC b\NR\NC 0\NC -2b\NC b\NC\NR 1\NC a + 2b\NC
  a\NR\stopdeterminant[R_1\rightarrow R_1 - R_2; R_2\rightarrow R_2 - R_3]$

  $= 9b^2(a + b) =$ R.H.S.
\item $\Delta = \frac{1}{a}\startdeterminant\NC a^2 + b^2 + c^2\NC b - c\NC c + b\NR\NC a^2 + b^2 + c^2\NC
  b\NC c - a\NR\NC a^2 + b^2 + c^2\NC b + c\NC c\NR\stopdeterminant[C_1\rightarrow aC_1 + bC_2 + cC_3]$

  $= \frac{a^2 + b^2 + c^2}{a}\startdeterminant\NC 0\NC -c\NC a + b\NR\NC 0\NC -a\NC -a\NR\NC 1\NC b + a\NC
  \NR\stopdeterminant[R_1\rightarrow R_1 - R_2; R_2\rightarrow R_2 - R_3]$

  $= \frac{a^2 + b^2 + c^2}{a}(ac + a^2 + ab) = (a^2 + b^2 + c^2)(a + b + c)$.

  Thus, given determinant has the same sign as $a + b + c$ because $a^2 + b^2 + c^2$ is always positive for
  real values of $a, b, c$.
\item $\Delta = \frac{1}{ab}\startdeterminant\NC 0\NC a(bc' + b'c) - b(ca' + c'a) \NC ab'c' - bc'a'\NR\NC
  0\NC b(ca' + c'a) - c(ab' - ab)\NC bc'a' - ca'b'\NR\NC ab \NC ab' + a'b\NC
  a'b'\NR\stopdeterminant[R_1\rightarrow aR_1 - bR_2; R_2\rightarrow bR_2 - cR_3]$

  $= \frac{ab}{ab}(ab' - a'b)(b'c - bc')(a'c - c'a) = (ab' - a'b)(b'c - bc')(a'c - c'a)$.
\item $\Delta = \frac{1}{abc}\startdeterminant\NC a(b^2 + c^2)\NC a^2b\NC a^2c\NR\NC ab^2\NC b(c^2 + a^2)\NC b^2c\NR\NC
  ac^2\NC c^2b\NC c(a^2 + b^2)\NR\stopdeterminant[R_1\rightarrow aR_1; R_2\rightarrow bR_2; R_3\rightarrow
  cR_3]$

  $= \frac{abc}{abc}\startdeterminant\NC b^2 + c^2\NC a^2\NC a^2\NR\NC b^2\NC c^2 + a^2\NC b^2\NR\NC c^2\NC
  c^2\NC a^2 + b^2\NR\stopdeterminant$

  $= \startdeterminant\NC 0 \NC -2c^2 \NC -2b^2\NR\NC 0 \NC c^4 + c^2a^2 - b^2c^2 \NC b^2c^2 - a^2b^2 -
  b^4\NR\NC c^2\NC c^2\NC a^2 + b^2\NR\stopdeterminant[R_1\rightarrow R_1 - R_2 - R_3; R_2\rightarrow c^2R_2
  - b^2R_3]$

  $= 4a^2b^2c^2$.
\item $\Delta = \frac{1}{a^2b^2c^2}\startdeterminant\NC(\alpha + \gamma)^2 \NC\beta^2\NC\beta^2 \NR\NC
  \gamma^2 \NC(\alpha + \beta)^2 \NC \gamma^2\NR\NC \alpha^2\NC\alpha^2\NC (\beta +
  \gamma)^2\NR\stopdeterminant[C_1\rightarrow a^2C_1; C_2\rightarrow b^2C_2; C_3\rightarrow c^2C_3$ and then
  applying $ab = \alpha, bc = \beta, ca = \gamma]$

  We have evaluated this determinant earlier to be equal to $2\alpha\beta\gamma(\alpha + \beta + \gamma)^3$
  and $\alpha\beta\gamma = a^2b^2c^2$.

  Thus, $\Delta = 2(ab + bc + ca)^3$.
\item $\Delta = \frac{1}{abc}\startdeterminant\NC c(a + b)^2\NC c^2a \NC bc^2\NR\NC ca^2\NC a(b +
  c)^2\NC a^2b\NR\NC b^2c\NC ab^2 \NC b(c + a)^2\NR\stopdeterminant[R_1\rightarrow cR_1; R_2\rightarrow
  aR_2; R_3\rightarrow bR_3]$

  $=\startdeterminant\NC (a + b)^2\NC c^2\NC c^2\NR\NC a^2\NC (b + c)^2\NC a^2\NR\NC b^2\NC\b^2\NC (c +
  a)^2\NR\stopdeterminant$

  We have evaluated this determinant to be equal to $2abc(a + b + c)^3$ earlier.

  $\therefore \Delta = 2abc(a + b + c)^3$.
\item $\Delta = \frac{1}{abc}\startdeterminant\NC a^2 + b^2\NC c^2\NC c^2\NR\NC a^2\NC b^2 + c^2 \NC
  a^2\NR\NC b^2\NC b^2\NC c^2 + a^2\NR\stopdeterminant[R_1\rightarrow cR_1; R_2\rightarrow
  aR_2; R_3\rightarrow bR_3]$

  We have evaluated this determinant to be equal to $4a^2b^2c^2$.

  $\therefore \Delta = 4abc$.
\item $\Delta = \startdeterminant\NC 0\NC 0\NC x - a\NR\NC a\NC a\NC a\NR\NC b\NC x\NC
  b\NR\stopdeterminant[R_1\rightarrow R_1 - R_2]$

  $= (x - a)(ax - ab) = 0 \Rightarrow x = a, b$.
\item $\Delta = \startdeterminant\NC x \NC 2\NC 1\NR\NC 6 \NC x + 4\NC -x\NR\NC 7 \NC 8\NC
  x\NR\stopdeterminant[C_3\rightarrow C_3 - C_2] = \startdeterminant\NC x \NC 2\NC 1\NR\NC 13\NC x + 12\NC
  0\NR\NC 7 \NC 8\NC x\NR\stopdeterminant[R_2\rightarrow R_2 + R_3]$

  $=\startdeterminant\NC 0\NC 0\NC 1\NR\NC 13\NC x + 12\NC 0\NR\NC 7 - x^2 \NC 8 - 2x\NC
  x\NR\stopdeterminant[C_1\rightarrow C_1 - xC_3; C_2\rightarrow C_2 - 2C_3]$

  $= x^3 + 12x^2 - 33x + 20 = 0$

  We observe that sum of coefficients is zero so $1$ would be a factor. The other factor is $x^2 + 13x - 20
  = 0$; whose roots are $\frac{-13\pm\sqrt{249}}{2}$.
\item $\Delta = \startdeterminant\NC x - 12\NC 2 - 3x\NC 0\NR\NC 4 \NC x\NC 1\NR\NC x - 20\NC 2- 5x\NC
  0\NR\stopdeterminant[R_1\rightarrow R_1 - 3R_2; R_3\rightarrow R_3 - 5R_2]$

  $= (x - 20)(3x - 2) - (x - 12)(5x -2)\Rightarrow x = \pm2\sqrt{2}$.
\item $\Delta = \startdeterminant\NC  x + a + b + c\NC b \NC c\NR\NC x + a + b + c \NC x + b \NC c\NR\NC x +
  a + b + c\NC b \NC x + c\NR\stopdeterminant[C_1 \rightarrow C_1 + C_2 + C_3]$

  $= (x + a + b + c)\startdeterminant\NC 1\NC b\NC c\NR\NC 0\NC x\NC 0\NR\NC 0\NC 0\NC x\NR\stopdeterminant
  [R_2\rightarrow R_2- R_1; R_3\rightarrow R_3 - R_1] = (x + a + b + c)x^2$

  $\Rightarrow x = -(a + b + c), 0$.
\item $\Delta = \startdeterminant\NC x + 10\NC 5\NC 2\NR\NC x + 10\NC 7 + x \NC 6\NR\NC x + 10\NC 5\NC 3 +
  x\NR\stopdeterminant[C_1\rightarrow C_1 + C_2 + C_3]$

  $= \startdeterminant\NC 0\NC 0\NC -x -1\NR\NC x + 14\NC x + 7\NC 6\NR\NC x + 10\NC 5\NC 3 +
  x\NR\stopdeterminant[R_1\rightarrow R_1 - R_3] = (x + 1)(x^2 + 12x)$

  $\Rightarrow x = 0, -1, -12$.
\item $\Delta = \startdeterminant\NC a \NC b + c\NC c + a\NR\NC b\NC c + a\NC a + b\NR\NC c\NC a + b\NC b +
  c\NR\stopdeterminant + \startdeterminant\NC b \NC b + c\NC c + a\NR\NC c\NC c + a\NC a + b\NR\NC a\NC a +
  b\NC b + c\NR\stopdeterminant$

  $=\startdeterminant\NC a \NC b\NC c + a\NR\NC b\NC c\NC a + b\NR\NC c\NC a\NC b + c\NR\stopdeterminant +
  \startdeterminant\NC a \NC c\NC c + a\NR\NC a\NC c\NC a + b\NR\NC b\NC a\NC b + c\NR\stopdeterminant +
  \startdeterminant\NC b \NC b\NC c + a\NR\NC c\NC c\NC a + b\NR\NC a\NC a \NC b + c\NR\stopdeterminant[=0]
  + \startdeterminant\NC b \NC c\NC c + a\NR\NC a\NC c\NC a + b\NR\NC a\NC b \NC b + c\NR\stopdeterminant$

  $= \startdeterminant\NC a \NC b\NC c\NR\NC b \NC c\NC a\NR\NC c\NC a \NC b\NR\stopdeterminant
  + \startdeterminant\NC a\NC b\NC a\NR\NC b\NC c\NC b\NR\NC c\NC a\NC c\NR\stopdeterminant[=0]
  + \startdeterminant\NC a\NC c\NC c\NR\NC b\NC a\NC a\NR\NC c\NC b\NC b\NR\stopdeterminant[=0]
  + \startdeterminant\NC a\NC c\NC a\NR\NC b\NC a\NC b\NR\NC c\NC b\NC c\NR\stopdeterminant[=0]
  + \startdeterminant\NC b\NC c\NC c\NR\NC c\NC a\NC a\NR\NC a\NC b\NC b\NR\stopdeterminant[=0]
  + \startdeterminant\NC b\NC c\NC a\NR\NC c\NC a\NC b\NR\NC a\NC b\NC c\NR\stopdeterminant$

  $= 2\startdeterminant\NC  a \NC b \NC c\NR\NC b \NC c \NC a\NR\NC  c \NC a \NC
  b\NR\stopdeterminant$(exchanging columns of second determinant twice).
\item This problem is similar to previous problem and can be solved similarly.
\item This problem is similar to previous problem and can be solved similarly.
\item $\Delta = \startdeterminant\NC  -2a \NC a^2 + 1 \NC a\NR\NC -2b \NC b^2 + 1 \NC b\NR\NC -2c \NC c^2 +
  1 \NC c\NR\stopdeterminant[C_1\rightarrow C_1 - C_2]$

  Taking out $-2$ from $C_1$ makes $C_1$ and $C_3$ equal, and thus, $\Delta = 0$.
\item Multiplying all rows by $-1$ and changing rows intocorresponding columns we observe that $\Delta =
  -\Delta \Rightarrow \Delta = 0$.
\item $\Delta = \frac{1}{abc}\startdeterminant\NC a\NC a^2\NC abc\NR\NC b\NC b^2\NC abc\NR\NC c\NC c^2\NC
  abc\NR\stopdeterminant[R_1\rightarrow aR_1; R_2\rightarrow bR_2; R_3\rightarrow cR_3]$

  $= \frac{abc}{abc}\startdeterminant\NC  1 \NC a \NC a^2\NR\NC 1 \NC b \NC b^2\NR\NC  1 \NC c \NC
  c^2\NR\stopdeterminant$(exchanging columns twice).
\item $\Delta = \frac{1}{xyz}\startdeterminant\NC  ax \NC by \NC cz\NR\NC x^2 \NC y^2 \NC z^2\NR\NC xyz \NC
  yzx \NC zxy\NR\stopdeterminant[C_1\rightarrow xC_1; C_2\rightarrow yC_2; C_3\rightarrow zC_3]$

  $= \frac{xyz}{xyz}\startdeterminant\NC  ax \NC by \NC cz\NR\NC  x^2 \NC y^2 \NC z^2\NR\NC  1 \NC 1 \NC
  1\NR\stopdeterminant$.
\item Changing rows twice gives us third determinant from first determinant.

  $\Delta = -\startdeterminant\NC x\NC y\NC z\NR\NC a\NC b\NC c\NR\NC p\NC q\NC
  r\NR\stopdeterminant[R_1\leftrightarrow R_2] = \startdeterminant\NC y\NC x\NC z\NR\NC b\NC a\NC c\NR\NC
  q\NC p\NC r\NR\stopdeterminant[C_1\leftrightarrow C_2]$

  $= \startdeterminant\NC  x \NC y \NC z\NR\NC  p \NC q \NC r\NR\NC  a \NC b \NC
  c\NR\stopdeterminant$(changing rows into corresponding columns).
\item $\Delta = m!(m + 1)!(m + 2)!\startdeterminant\NC 1\NC m + 1\NC(m + 1)(m + 2)\NR\NC 1\NC m + 2\NC(m +
  2)(m + 3)\NR\NC 1\NC m + 3\NC(m + 3)(m + 4)\NR\stopdeterminant$

  $= m!(m + 1)!(m + 2)!\startdeterminant\NC 1\NC m + 1\NC(m + 1)^2\NR\NC 1\NC m + 2\NC(m + 2)^2\NR\NC 1\NC m
  + 3\NC(m + 3)^2\NR\stopdeterminant[C_3\rightarrow C_3 - C_2]$

  Using the result obtained earlier

  $= m!(m + 1)!(m + 2)!(-1).(-1).2$.

  $\frac{\Delta}{(m!)^3} = 2m^3 + 8m^2 + 10m + 4$, and thus, divisibility condition is fulfilled.
\item $\Delta = \startdeterminant\NC 1\NC 1\NR\NC 2\NC -3\NR\stopdeterminant = -3 -2 = -5\neq 0$.

  $\Delta_1 = \startdeterminant\NC 4\NC 1\NR\NC 9 \NC -3\NR\stopdeterminant = -21, \Delta_2
  = \startdeterminant\NC 1\NC 4\NR\NC 2\NC 9\NR\stopdeterminant = 1$

  By Cramer's rule, $x = \frac{\Delta_1}{\Delta} = \frac{21}{5}, y = \frac{\Delta_2}{\Delta} =
  -\frac{1}{5}$.
\item $\Delta = \startdeterminant\NC 2\NC -1\NC 3\NR\NC 1 \NC 1\NC 1\NR\NC 1\NC -1\NC 1\NR\stopdeterminant =
  2(1 + 1) + 1(1 - 1) + 3(-1 -1) = -2$.

  $\Delta_1 = \startdeterminant\NC 9\NC -1\NC 3\NR\NC 6 \NC 1\NC 1\NR\NC 2\NC -1\NC 1\NR\stopdeterminant =
  9(1 + 1) + 1(6 - 2) + 3(-6 - 2) = -2$

  $\Delta_2 = \startdeterminant\NC 2\NC 9\NC 3\NR\NC 1 \NC 6\NC 1\NR\NC 2\NC -1\NC 1\NR\stopdeterminant =
  2(6 - 2) - 9(1 - 1) + 3(2 - 6) = -4$.

  $\Delta_3 = \startdeterminant\NC 2\NC -1\NC 9\NR\NC 1 \NC 1\NC 6\NR\NC 1\NC -1\NC 2\NR\stopdeterminant =
  2(2 + 6) + 1(2 - 6) + 9(-1 - 1) = -6$

  By Cramer's rule, $x = \frac{\Delta_1}{\Delta} = 1, y = \frac{\Delta_2}{\Delta} = 2, z =
  \frac{\Delta_3}{\Delta} = 3$.
\item $\Delta = \startdeterminant\NC 2\NC 3\NR\NC 4\NC 6\NR\stopdeterminant = 0, \Delta_1
  = \startdeterminant\NC 6\NC 3\NR\NC 10\NC 6\NR\stopdeterminant = 6 \neq 0$.

  Hence, given system of equations is inconsistent and has no solution.
\item $\Delta = \startdeterminant\NC 1 \NC 1\NC -1\NR\NC 2\NC 3\NC 1\NR\NC 4\NC 3\NC 1\NR\stopdeterminant
  = \startdeterminant\NC 0\NC 0\NC -1\NR\NC 3\NC 4\NC 1\NR\NC 5\NC 4\NC 1\NR\stopdeterminant[C_1\rightarrow
    C_1 + C_3; C_2\rightarrow C_2 + C_3]$

  $= -1(12 - 20) = 8\neq 0$.

  Hence, given system of equations is consistent and has unique solution.
\item $\Delta = \startdeterminant\NC 1 \NC 1\NR\NC 2\NC 2\NR\stopdeterminant = 0$, $\Delta_1
  = \startdeterminant\NC 2\NC 1\NR\NC 4\NC 2\NR\stopdeterminant = 0$, $\Delta_2 = \startdeterminant\NC 1\NC
  2\NR\NC 2\NC 4\NR\stopdeterminant = 0$.

  Hence, given system of equations is consistent and has infinite number of solutions.
\item $\Delta = \startdeterminant\NC 2\NC 1\NC 13\NR\NC 6 \NC 3\NC 18\NR\NC 1\NC -1\NC -3\NR\stopdeterminant
  = 2(-9+ 18) -1(-18 - 18) + 13(-6 - 3) = -63\neq 0$.

  Thus, we have case of inconsistent solutions.
\item $\Delta = \startdeterminant\NC 1\NC 1\NC -6\NR\NC 3\NC -1\NC -2\NR\NC 1\NC -1\NC 2\NR\stopdeterminant
  = 1(-2 - 2) - 1(6 + 2) -6(-3 + 1) = 0$

  Hence, given system of equations has non-trivial solution.
\item $\Delta = \startdeterminant\NC 1\NC 1\NC -k\NR\NC 3\NC -1\NC -2\NR\NC 1\NC -1\NC 2\NR\stopdeterminant
  = 0$(for non-trivial solution)$\Rightarrow k = 6$.

  Solving the system of equation gives us $z = \frac{x}{2}$ and $y = 2x$. Thus, solution is given by $x = t,
  y = 2t, z = \frac{t}{2}$, where $t$ is an arbitrary number.
\item $\Delta = \startdeterminant\NC 1\NC -2\NR\NC 7\NC 6\NR\stopdeterminant = 6 + 14 = 20, \Delta_1
  = \startdeterminant\NC 0\NC -2\NR\NC 40\NC 6\NR\stopdeterminant = 80$

  $\Delta_2 = \startdeterminant\NC 1\NC 0\NR\NC 7\NC 40\NR\stopdeterminant = 40\Rightarrow x =
  \frac{\Delta_1}{\Delta} = 4, y = \frac{\Delta_2}{\Delta} = 2$.
\item $\Delta = \startdeterminant\NC 1\NC 1\NC 1\NR\NC 3\NC 2\NC -3\NR\NC -1\NC 0\NC 1\NR\stopdeterminant =
  1(2) -1(3 - 3) + 1(2) = 4$

  $\Delta_1 = \startdeterminant\NC 9\NC 1\NC 1\NR\NC 0\NC 2\NC -3\NR\NC 2\NC 0\NC 1\NR\stopdeterminant =
  9(2) -1(6) + 1(-4) = 8$

  $\Delta_2 = \startdeterminant\NC 1\NC 9\NC 1\NR\NC 3\NC 0\NC -3\NR\NC -1\NC 2\NC 1\NR\stopdeterminant =
  1(6) -9(3 - 3) + 1(6) = 12$

  $\Delta_3 = \startdeterminant\NC 1\NC 1\NC 9\NR\NC 3\NC 2\NC 0\NR\NC -1\NC 0\NC 2\NR\stopdeterminant =
  1(4) -1(6) + 9(2) = 16$

  $\Rightarrow x = \frac{\Delta_1}{\Delta} = 2, y = \frac{\Delta_2}{\Delta} = 3, z = \frac{\Delta_3}{\Delta}
  = 4$.
\item $\Delta = \startdeterminant\NC 1\NC -1\NC 1\NR\NC 2\NC 3\NC -5\NR\NC 3\NC -4\NC 2\NR\stopdeterminant =
  1(6 - 20) +1(4 + 15) + 1(-17) = 12$

  $\Delta_1 = \startdeterminant\NC 0\NC -1\NC 1\NR\NC 7\NC 3\NC -5\NR\NC -1\NC -4\NC 2\NR\stopdeterminant =
  -1(9) + 1(25) = 16$

  $\Delta_2 = \startdeterminant\NC 1\NC 0\NC 1\NR\NC 2\NC 7\NC -5\NR\NC 3\NC -1\NC 2\NR\stopdeterminant =
  1(9) + 1(-23) = -14$

  $\Delta_3 = \startdeterminant\NC 1\NC -1\NC 0\NR\NC 2\NC 3\NC 7\NR\NC 3\NC -4\NC -1\NR\stopdeterminant =
  1(25) + 1(-23) = 2$

  $\Rightarrow x = \frac{4}{3}, y = -\frac{7}{6}, z = \frac{1}{6}$.
\item $\Delta = \startdeterminant\NC 2\NC 3\NC -3\NR\NC 5\NC -2\NC 2\NR\NC 1\NC 7\NC -5\NR\stopdeterminant =
  2(-4) -3(-25 - 2) -3(37) = -38$

  $\Delta_1 = \startdeterminant\NC 0\NC 3\NC -3\NR\NC 19\NC -2\NC 2\NR\NC 5\NC 7\NC -5\NR\stopdeterminant =
  -3(-105) -3(143) = -114$

  $\Delta_2 = \startdeterminant\NC 2\NC 0\NC -3\NR\NC 5\NC 19\NC 2\NR\NC 1\NC 5\NC -5\NR\stopdeterminant =
  2(-105) - 3(6) = -228$

  $\Delta_3 = \startdeterminant\NC 2\NC 3\NC 0\NR\NC 5\NC -2\NC 19\NR\NC 1\NC 7\NC 5\NR\stopdeterminant =
  2(-143) - 3(6) = -304$

  $\Rightarrow x = 3, y = 6, z = 8$.
\item $\Delta = \startdeterminant\NC 1\NC 1\NC 1\NR\NC a\NC b\NC c\NR\NC a^2\NC b^2\NC
  c^2\NR\stopdeterminant = (a - b)(b - c)(c - a)$

  $\Delta_1 = \startdeterminant\NC 1\NC 1\NC 1\NR\NC k\NC b\NC c\NR\NC k^2\NC b^2\NC
  c^2\NR\stopdeterminant = (k - b)(b - c)(c - k)$

  $\Delta_2 = \startdeterminant\NC 1\NC 1\NC 1\NR\NC a\NC k\NC c\NR\NC a^2\NC k^2\NC
  c^2\NR\stopdeterminant = -(a - k)(k - c)(c - a)$

  $\Delta_3 = \startdeterminant\NC 1\NC 1\NC 1\NR\NC a\NC b\NC k\NR\NC a^2\NC b^2\NC
  k^2\NR\stopdeterminant = (a - b)(b - k)(k - a)$

  $\Rightarrow x = \frac{(k - b)(k - c)}{(a - b)(a - c)},  y = \frac{(k - a)(k - c)}{(b - a)(b - c)}, z =
  \frac{(k - a)(k - b)}{(c - a)(c - b)}$.
\item $\Delta = \startdeterminant\NC 3\NC 9\NR\NC 9\NC 27\NR\stopdeterminant = 0, \Delta_1
  = \startdeterminant\NC 5\NC 9\NR\NC 10\NC 27\NR\stopdeterminant \neq 0$.

  Hence, the given system of equations is inconsistent and has no solution.
\item $\Delta = \startdeterminant\NC 5\NC -3\NR\NC 1\NC 1\NR\stopdeterminant \neq 0, \Delta_1
  = \startdeterminant\NC 3\NC -3\NR\NC 7\NC 1\NR\stopdeterminant \neq 0, \Delta_2 = \startdeterminant\NC
  5\NC 3\NR\NC 1\NC 7\NR\stopdeterminant \neq 0$.

  Hence, the given system of equations has unique solution.
\item $\Delta = \startdeterminant\NC 1\NC 2\NR\NC 3\NC 6\NR\stopdeterminant = 0, \Delta_1
  = \startdeterminant\NC 5\NC 2\NR\NC 15\NC 2\NR\stopdeterminant = 0, \Delta_2 = \startdeterminant\NC 1\NC
  5\NR\NC 3\NC 15\NR\stopdeterminant = 0$.

  Hence, the given system of equations has infinite solutions.
\item $\Delta = \startdeterminant\NC 2\NC 3\NC 1\NR\NC 3\NC 1\NC 5\NR\NC 1\NC 4\NC -2\NR\stopdeterminant =
  2(-22) -3(-11) + 1(11) = 0$.

  $\Delta_1 = \startdeterminant\NC 5\NC 3\NC 1\NR\NC 7\NC 1\NC 5\NR\NC 3\NC 4\NC -2\NR\stopdeterminant =
  5(-22) -3(-29) + 1(25) = 2$.

  Hence, the given system of equations is inconsistent and has no solution.
\item $\Delta = \startdeterminant\NC 1\NC 1\NC -1\NR\NC 6\NC 4\NC 6\NR\NC 2\NC 7\NC 4\NR\stopdeterminant =
  1(-26) - 1(12) -1(34) = -72$

  $\Delta_1 = \startdeterminant\NC -2\NC 1\NC -1\NR\NC 26\NC 4\NC 6\NR\NC 31\NC 7\NC 4\NR\stopdeterminant =
  -2(-26) -1(-82) -1(38) \neq 0$

  $\Delta_2 = \startdeterminant\NC 1\NC -2\NC -1\NR\NC 6\NC 26\NC 6\NR\NC 2\NC 31\NC 4\NR\stopdeterminant =
  1(-82) + 2(12) -1(134)\neq 0$

  $\Delta_3 = \startdeterminant\NC 1\NC 1\NC -2\NR\NC 6\NC 4\NC 26\NR\NC 2\NC 7\NC 31\NR\stopdeterminant =
  1(-38) - 1(134) -2(34) \neq 0$

  Hence, the given system of equations has unique solution.
\item $\Delta = \startdeterminant\NC 1\NC k\NC 3\NR\NC 3\NC k\NC -2\NR\NC 2\NC 3\NC -4\NR\stopdeterminant =
  1(-4k + 6) - k(-8) + 3(9 - 2k) = 0 \Rightarrow k = \frac{33}{2}$.

  Solving the system of equations for this value of $k$ gives us $2x + 15y = 0$ and $x + 5z = 0$.

  Therefore, $x = t, y = \frac{-2t}{15}, z = \frac{-t}{5}$, where $t$ is an arbitrary number.
\item $\Delta = \startdeterminant\NC  a \NC b \NC c\NR\NC b \NC c \NC a\NR\NC c \NC a \NC
  b\NR\stopdeterminant = \startdeterminant\NC  a + b + c \NC b \NC c\NR\NC b + c + a \NC c \NC a\NR\NC c + a
  + b\NC a \NC b\NR\stopdeterminant[C_1\rightarrow C_1 + C_2 + C_3]$

  $= (a + b + c)\startdeterminant\NC 1 \NC b \NC c\NR\NC 1 \NC c \NC a\NR\NC 1 \NC a \NC
  b\NR\stopdeterminant = (a + b + c)\startdeterminant\NC  1 \NC b \NC c\NR\NC 0 \NC c - b \NC a - c\NR\NC 0
  \NC a - b \NC b - c\NR\stopdeterminant[R_2\rightarrow R_2 - R_1; R_3\rightarrow R_3 - R_1]$

  $= (a + b + c)[(c - b)(b - c) - (a - b)(a - c)] = (a + b + c)(ab + bc + ca - a^2 - b^2 - c^2) =
  -\frac{1}{2}(a + b + c)[(a - b)^2 + (b - c)^2 + (c - a)^2]$

  Since $a, b, c$ are different $\Delta$ will acquire value zero only if $a + b + c = 0$ for non-trivial
  solution.
\item $\Delta = \startdeterminant\NC 3\NC -1\NC 4\NR\NC 1\NC 2\NC -3\NR\NC 6\NC 5\NC
  \lambda\NR\stopdeterminant = 3(2\lambda + 15) + 1(\lambda + 18) + 4(-7) = 0$

  $\Rightarrow \lambda = -5$.
\item Let $A28 = A\times 100 + 2\times 10 + 8 = pk, 3B9 = 3\times 100 + B\times 10 + 9 = qk 62C = 6\times
  100 + 2\times 10 + C = rk$ where $p , q, r$ are integers.

  $\Delta = \startdeterminant\NC A \NC 3 \NC 6 \NR\NC pk \NC qk \NC rk \NR\NC 2 \NC B \NC
  2\NR\stopdeterminant [R_2\rightarrow R_2 + 10R_3 + 100R_1]$

  $= k \startdeterminant\NC A \NC 3 \NC 6 \NR\NC p \NC q \NC r \NR\NC 2 \NC B \NC 2\NR\stopdeterminant$, which
  is divisible by $k$.
\item $\Delta = \startdeterminant\NC x \NC \frac{x(x - 1)}{2} \NC \frac{x(x - 1)(x -
  2)}{6} \NR\NC y \NC \frac{y(y - 1)}{2} \NC \frac{y(y - 1)(y - 2)}{6} \NR\NC z \NC
  \frac{z(z - 1)}{2} \NC \frac{z(z - 1)(z - 2)}{6}\NR\stopdeterminant= \frac{xyz}{2.6}\startdeterminant\NC
  1 \NC x - 1 \NC (x - 1)(x - 2) \NR\NC 1 \NC
  y - 1 \NC (y - 1)(y - 2) \NR\NC 1 \NC z - 1 \NC (z - 1)(z - 2)\NR\stopdeterminant$

  $= \frac{xyz}{12}\startdeterminant\NC 1 \NC x - 1 \NC (x - 1)^2 \NR\NC 1 \NC y - 1 \NC
  (y - 1)^2 \NR\NC 1 \NC z - 1 \NC (z - 1)^2\NR\stopdeterminant[C_3\rightarrow C_3 + C_2]=
  \frac{xyz}{12}(x - y)(y - z)(z - x)[\because \startdeterminant\NC 1 \NC a
    \NC a^2 \NR\NC 1 \NC b \NC b^2 \NR\NC 1 \NC c \NC c^2\NR\stopdeterminant = (a - b)(b - c)(c - a)]$.
\item $\Delta = \startdeterminant\NC p - a \NC b - q \NC 0 \NR\NC 0 \NC q - b \NC c - r \NR\NC a
  \NC b \NC r\NR\stopdeterminant[R_1\rightarrow R_1 - R-2; R_2 \rightarrow R_2 - R_3] =
  0$

  $\Rightarrow (p -a)[r(q - b) - b(c - r)] - (b - q)[0 - a(c - r)] = r(p - a)(q - b) + b(p -
  a)(r - c) + a(q - b)(r - c) = 0$

  $\Rightarrow\frac{r}{r - c} + \frac{b}{q - b} + \frac{a}{p - a} = 0 \Rightarrow  \frac{r}{r - c} +
  \left(\frac{b}{q - b} + 1\right) + \left(\frac{a}{p - a} + 1\right) = 0 + 1 + 1$

  $\Rightarrow \frac{p}{p - a} + \frac{q}{q - b} + \frac{r}{r - c} = 2$
\item $\Delta = \startdeterminant\NC x(x - 2a) \NC x(2b - x) \NC 0 \NR\NC 0 \NC -(x - 2b) \NC
  x(ac - x) \NR\NC a^2 \NC b^2 \NC (x - c)^2\NR\stopdeterminant[R_1\rightarrow R_1
    -R_2;R_2\rightarrow R_2 - R_3]$

  $= x^2\startdeterminant\NC x - 2a \NC -(x - 2b) \NC 0 \NR\NC 0 \NC x - 2b \NC -(x - 2c) \NR\NC a^2
  \NC b^2 \NC (x - c)^2\NR\stopdeterminant$

  $= x^2(x - 2a)(x - 2b)(x - 2c)\startdeterminant\NC 1 \NC -1 \NC 0 \NR\NC 0 \NC 1 \NC -1
  \NR\NC \frac{a^2}{x - 2a} \NC \frac{b^2}{x - 2b} \NC x + \frac{c^2}{x -
    2c}\NR\stopdeterminant$

  $= x^2(x - 2a)(x - 2b)(x - 2c)\left(x + \frac{a^2}{x - 2a} +
  \frac{b^2}{x - 2b} + \frac{c^2}{x - 2c}\right)\NR\NC\startdeterminant\NC 1 \NC -1 \NC 0 \NR\NC
  0 \NC 1 \NC 0 \NR\NC \frac{a^2}{x - 2a} \NC \frac{b^2}{x - ac} \NC
  1\NR\stopdeterminant[C_3\rightarrow C_1 + C_2 + C_3]$

  $= x^2(x - 2a)(x - 2b)(x - 2c)\left(x + \frac{a^2}{x - 2a} +
  \frac{b^2}{x - 2b} + \frac{c^2}{x - 2c}\right)$.
\item $\Delta = \frac{1}{a(a + d)^2(a + 2d)^3(a + 3d)^2(a +
  4d)}\startdeterminant\NC (a + d)(a + 2d)  \NC a + 2d \NC a \NR\NC (a + 2d)(a + 3d) \NC a +
  3d \NC a + d \NR\NC (a + 3d)(a + 4d) \NC a + 3d \NC a + 2d\NR\stopdeterminant$

  $= \frac{1}{a(a + d)^2(a + 2d)^3(a + 3d)^2(a + 4d)} \startdeterminant\NC (a
  + d)(a + 2d) \NC 2d \NC a \NR\NC (a + 2d)(a + 3d) \NC 2d \NC a + d \NR\NC (a + 3d)(a + 4d)
  \NC 2d \NC a + 2d\NR\stopdeterminant[C_2\rightarrow C_2 - C_3]$

  $= \frac{1}{a(a + d)^2(a + 2d)^3(a + 3d)^2(a + 4d)} \startdeterminant\NC(a
  + d)(a + 2d) \NC 2d \NC a \NR\NC (a + 2d)2d \NC 0 \NC d \NR\NC (a + 3d)2d \NC 0 \NC
  d\NR\stopdeterminant[R_2\rightarrow R_2 - R_1; R_3\rightarrow R_3 - R_2]$

  $= \frac{1}{a(a + d)^2(a + 2d)^3(a + 3d)^2(a + 4d)}.-2d[2d^2(a + 2d -
    a - 3d)]$

  $= \frac{4d^4}{a(a + d)^2(a + 2d)^3(a + 3d)^2(a + 4d)}$
\item $\Delta = \frac{1}{(a + x)(b + c)(c + x)(a + y)(b + y)(c + y)(a +
  z)(b + z)(c + z)}\Delta_1$,

  where $\Delta_1 = \startdeterminant\NC(b + x)(c + x) \NC (b + y)(c + y) \NC (b
  + z)(c + z) \NR\NC (c + x)(a + x) \NC (c + y)(a + y) \NC (c + z)(a + z) \NR\NC (a +
  x)(b + x) \NC (a + y)(b + y) \NC (a + z)(b + z)\NR\stopdeterminant$

  $\Delta_1 = \startdeterminant\NC(b + x)(c + x) \NC (b + y)(c + y) \NC (b
  + z)(c + z) \NR\NC (c + x)(a - b) \NC (c + y)(a - b) \NC (c - z)(a - b) \NR\NC (b +
  x)(a - c) \NC (b + y)(a - c) \NC (b + z)(a - c)\NR\stopdeterminant[R_2\rightarrow R_2
    - R-1; R_3\rightarrow R_3 - R_1]$

  $= (a - b)(a - c)\startdeterminant\NC(b + x)(c + x) \NC (b + y)(c + y) \NC (b
  + z)(c + z) \NR\NC c + x \NC c + y \NC c + z \NR\NC b + x \NC b + y \NC b + z\NR\stopdeterminant$

  $= (a - b)(a - c)\startdeterminant\NC x(c + x) \NC y(c + y) \NC z(c + z) \NR\NC c +
  x \NC c + y \NC c + z \NR\NC b - c \NC b - c \NC b - c\NR\stopdeterminant[R_1\rightarrow
    R_1 - bR_2; R_3\rightarrow R_3 - R_2]$

  $= (a - b)(b - c)(a - c)\startdeterminant\NC(x - z)(c + x + z) \NC (y - z)(c
  + y + z) \NC z(c + z) \NR\NC x - z \NC y - z \NC c + z \NR\NC 0 \NC 0 \NC
  1\NR\stopdeterminant[C_1\rightarrow C_1 - C_2; C_2\rightarrow C_2 - C_3]$

  $= (a - b)(b - c)(a - c)(x - z)(y - z)\startdeterminant\NC c + x + z \NC c +
  y + z \NC z(c + z) \NR\NC 1 \NC 1 \NC c + z \NR\NC 0 \NC 0 \NC 1\NR\stopdeterminant$

  $= (a - b)(b - c)(a - c)(x - z)(y - z)[c + x + z - c - y - z] = (a - b)(b - c)(a - c)(x - z)(y - z)(x -
  y)$

  $\Rightarrow  \Delta = \frac{(a - b)(b - c)(c - a)(x - y)(y -
  z)(z - x)}{(a + x)(b + x)(c + x)(b + x)(b + y)(b + z)(c + x)(c + y)(c + z)}$.
\item Let $\alpha = s - a, \beta = s - b, \gamma = s - c,$ then

  $\beta + \gamma = 2s - (b + c) = a, \gamma + \alpha = b, \alpha +
  \beta = c, \alpha + \beta + \gamma = 3s - (a + b + c) = 3s - 2s = s$

  $\Delta = \startdeterminant\NC(\beta + \gamma)^2 \NC \alpha^2 \NC \alpha^2 \NR\NC
  \beta^2 \NC (\gamma + \alpha)^2 \NC \beta^2 \NR\NC \gamma^2 \NC \gamma ^2 \NC (\alpha +
  \beta)^2\NR\stopdeterminant$

  Follwing like problem solved earlier

  $= 2\alpha\beta\gamma(\alpha + \beta + \gamma)^3 = 2(s - a)(s - b)(s - c)s^3$.
\item $\Delta = \frac{1}{a}(a^2 + b^2 + c^2)\startdeterminant\NC x \NC ay + bx \NC cx
  + az \NR\NC y \NC by - cz -ax \NC bz + cy \NR\NC z \NC bz + cy \NC cz - ax -
  by\NR\stopdeterminant[C_1\rightarrow aC_1 + bC_2 + cC_3]$

  $= \frac{1}{ax}(a^2 + b^2 + c^2)\startdeterminant\NC x^2 + y^2 + z^2 \NC
  b(x^2 + y^2 + z^2) \NC c(x^2 + y^2 + z^2) \NR\NC y \NC by - cz -ax \NC bz + cy \NR\NC z \NC
  bz + cy \NC cz - ax - by\NR\stopdeterminant[R_1\rightarrow xR_1 + yR_2 + zR_3]$

  $= \frac{(a^2 + b^2 + c^2)(x^2 + y^2 + z^2)}{ax}\startdeterminant\NC 1 \NC b
  \NC c \NR\NC y \NC by - cz -ax \NC bz + cy \NR\NC z \NC bz + cy \NC cz - ax -
  by\NR\stopdeterminant$

  $= \frac{(a^2 + b^2 + c^2)(x^2 + y^2 + z^2)}{ax}\startdeterminant\NC 1 \NC b
  \NC c \NR\NC 0 \NC -cz - ax \NC bz \NR\NC 0 \NC cy \NC -ax - by\NR\stopdeterminant[R_2\rightarrow
    R_2 - yR_1; R_3\rightarrow R_3 - zR_1]$

  $= \frac{(a^2 + b^2 + c^2)(x^2 + y^2 + z^2)}{ax}[(cz + ax)(ax + by) - bcyz]$

  $= (a^2 + b^2 + c^2)(x^2 + y^2 + z^2)(ax + by + cz)$
\item $\Delta = \startdeterminant\NC 2 + 4\sin 4\theta \NC \sin^2\theta \NC 4\sin\theta \NR\NC2 +
  4\sin 4\theta \NC 1 + \sin^2\theta \NC 4\sin\theta \NR\NC 2 + 4\sin 4\theta \NC
  \sin^2 \theta \NC 1 + 4\sin 4\theta\NR\stopdeterminant = 0[C_1\rightarrow C_1 + C_2
    + C_3]$

  $= (2 + 4\sin 4\theta)\startdeterminant\NC 1 \NC \sin^2\theta \NC
  4\sin 4\theta \NR\NC 0 \NC 1 \NC 0 \NR\NC 0 \NC 0 \NC 1\NR\stopdeterminant = 0[R_2\rightarrow R_2
    - R_1; R_3\rightarrow R_3 - R_1]$

  $\Rightarrow 2(2 + 4\sin 4\theta) = 0$

  $\sin 4\theta = -\frac{1}{2} \Rightarrow 4\theta = \frac{7\pi}{6},
  \frac{11\pi}{6}\Rightarrow \theta = \frac{7\pi}{24}, \frac{11\pi}{24}$
\item $\Delta = \frac{1}{abc}\startdeterminant\NC a[a^2 + (b^2 + c^2)\cos\phi] \NC
  ba^2[1 - \cos\phi] \NC ca^2(1 - cos\phi) \NR\NC ab^2(1 - \cos\phi) \NC b[b^2 + (c^2
    + a^2)\cos\phi] \NC cb^2(1 - \cos\phi) \NR\NC ac^2(1 - \cos\phi) \NC bc^2(1 -
  \cos\phi) \NC c[c^2 + (a^2 + b^2)\cos\phi]\NR\stopdeterminant[R_1\rightarrow aR_1
    + bR_2 + cR_3]$

  $= \startdeterminant\NC a^2 + (b^2 + c^2)\cos\phi \NC a^2(1 - \cos\phi) \NC
  a^2(1 - \cos\phi) \NR\NC b^2(1 - \cos\phi) \NC b^2 + (c^2 + a^2)\cos\phi \NC b^2(1 -
  \cos\phi) \NR\NC c^2(1 - \cos\phi) \NC c^2(1 - \cos\phi) \NC c^2 + (a^2 +
  b^2)\cos\phi\NR\stopdeterminant$

  $= (a^2 + b^2 + c^2)\startdeterminant\NC 1 \NC 1 \NC 1 \NR\NC \NR\NC b^(1 - \cos\phi) \NC
  b^2 + (c^2 + a^2)\cos\phi \NC b^2(1 - \cos\phi) \NR\NC c^2(1 - \cos\phi) \NC
  c^2(1 - \cos\phi) \NC c^2 + (a^2 + b^2)\cos\phi\NR\stopdeterminant[R_1\rightarrow
    R_1 + R_2 + R_3]$

  Performing $C_1\rightarrow C_1 - C_2; C_2\rightarrow C_2 - C_3$, we
  get

  $= (a^2 + b^2 + c^2)\startdeterminant\NC 0 \NC 0 \NC 1 \NR\NC -(a^2 + b^2 +
  c^2)\cos\phi \NC (a^2 + b^2 + c^2)\cos\phi \NC b^2(1 - \cos\phi) \NR\NC 0 \NC -(a^2 +
  b^2 + c^2)\cos\phi \NC c^2 + (a^2 + b^2)\cos\phi\NR\stopdeterminant$

  $= (a^2 + b^2 + c^2)(a^2 + b^2 + c^2)^2\cos^2\phi = \cos^2\phi$
\item $\Delta = \frac{1}{abc}\startdeterminant\NC -abc \NC ab^2 + abc \NC ac^2 + abc
  \NR\NC a^2b \NC -abc \NC bc^2 + abc \NR\NC a^c + abc \NC b^2c + abc \NC
  -abc\NR\stopdeterminant[R_1\rightarrow aR_1; R-2 \rightarrow bR_2; R_3\rightarrow
    cR_3]$

  $= \startdeterminant\NC -bc \NC ab + ac \NC ac + ab \NR\NC ab + bc \NC -ac \NC bc + ab
  \NR\NC ac + bc \NC bc + ac \NC ab\NR\stopdeterminant$

  $= (ab + bc + ca)\startdeterminant\NC 1 \NC 1 \NC 1 \NR\NC ab + bc \NC -ac \NC bc + ab
  \NR\NC ac + bc \NC bc + ac \NC ab\NR\stopdeterminant[R_1\rightarrow R_1 + R_2 + R_3]$

  $= (ab + bc + ca)\startdeterminant\NC 1 \NC 0 \NC 0 \NR\NC ab + bc \NC -(ab + bc +
  ca) \NC 0 \NR\NC ac + bc \NC 0 \NC -(ab + bc + ca)\NR\stopdeterminant[C_2\rightarrow C_2 -
    C_1; C_3\rightarrow C_3 - C_1]$

  $= (ab + bc + ca)^3$
\item Given $y = \frac{u}{v},\; \frac{dy}{dx} = \frac{vu' - uv'}{v^2}
  \Rightarrow v^2\frac{dy}{dx} = vu' - uv'$

  $= v^3\frac{dy}{dx} = v^2u' - uvv'$

  Again differentiating w.r.t.\ $x$, we get

  $v^3\frac{d^2y}{dx^2} + 3v^2v'\frac{dy}{dx} = 2vv'u' + v^2u'' - uvv'' -
  (uv' + u'v)v'$

  $v^3\frac{dy^2}{dx^2} = -2u'vv' + 2uv'^2 + v^2u'' - uvv'' = \Delta$
\item $\Delta = \startdeterminant\NC x  \NC x \NC x \NR\NC x \NC x + a \NC x \NR\NC x \NC x \NC x +
  a^2\NR\stopdeterminant + \startdeterminant\NC 1 \NC x \NC x \NR\NC 0 \NC x + a \NC x \NR\NC 0 \NC x \NC
  x + a^2\NR\stopdeterminant$

  $= \startdeterminant\NC x \NC x \NC x \NR\NC 0 \NC a \NC 0 \NR\NC 0 \NC 0 \NC
  a^2\NR\stopdeterminant[R_2\rightarrow R_2 - R_1; R_3\rightarrow R_3 - R_1] + (x +
  a)(x + a^2) - x^2$

  $= xa^3 + x(a + a^2) + a^3 = a^3\left[1 + x\left(1 + \frac{1}{a} + \frac{1}{a^2}\right)\right]$

  $= a^3\left[1 + \frac{x(a^3 - 1)}{a^2(a - 1)}\right]$
\item L.H.S.\ $= pa(qra^2 - p^2bc) - qb(q^2ca - prb^2) + rc(pqc^2 - r^2ab) = pqra^3 - abcp^3 - abcq^3 +
  pqrb^3 + pqrc^3 - abcr^3$

  $= pqr(a^3 + b^3 + c^3) - abc(p^3 + q^3 + r^3) = pqr(a^3 + b^3 + c^3 - 3abc) - abc(p^3 + q^3 + r^3 -
  3pqr)$

  $= pqr(a^3 + b^3 + c^3 - 3abc) - 0[\because p + q + r = 0]$

  R.H.S.\ $= pqr \startdeterminant\NC a \NC b \NC c \NR\NC c \NC a \NC b \NR\NC b \NC c \NC
  a\NR\stopdeterminant$

  $= pqr(a + b + c)\startdeterminant\NC 1 \NC b \NC c \NR\NC 1 \NC a \NC b \NR\NC 1 \NC c \NC
  a\NR\stopdeterminant[C_1\rightarrow C_1 + C_2 + C_3]$

  $= pqr(a + b + c)\startdeterminant\NC o \NC b- a \NC c - b \NR\NC 0 \NC a - c \NC b -
  a \NR\NC 1 \NC c \NC a\NR\stopdeterminant[R_1\rightarrow R_1-R_2;R_2\rightarrow R_2 -
    R_3]$

  $= pqr(a^3 + b^3 + c^3 - 3abc) =$ L.H.S
\item R.H.S.\ $= \frac{1}{abc}\startdeterminant\NC a \NC abc \NC a(b + c) \NR\NC b \NC abc
  \NC b(c + a) \NR\NC c \NC abc \NC c(a + b) \NR\stopdeterminant[R_1\rightarrow aR_1;
    R_2\rightarrow bR_2; R_3\rightarrow cR_3]$

  $= -\frac{abc}{abc}\startdeterminant\NC 1 \NC a \NC ab + ac \NR\NC 1 \NC b \NC bc + ba
  \NR\NC 1 \NC c \NC ca + cb\NR\stopdeterminant$ Taking $abc$ out and then applying
  $C_1\leftrightarrow C_2$

  $= -\startdeterminant\NC 1 \NC a \NC -bc \NR\NC 1 \NC b \NC - ca \NR\NC 1 \NC c \NC -ab
  \NR\stopdeterminant[C_3\rightarrow C_3 - (ab + bc + ca)C_1]$

  $= \startdeterminant\NC 1 \NC a \NC bc \NR\NC 1 \NC b \NC ca \NR\NC 1 \NC c \NC
  ab\NR\stopdeterminant = \frac{1}{abc}\startdeterminant\NC a \NC a^2 \NC abc \NR\NC b \NC b^2 \NC
  abc \NR\NC c \NC c^2 \NC abc\NR\stopdeterminant[R_1\rightarrow aR_1; R_2\rightarrow
    bR_2; R_3\rightarrow cR_3]$

  $= \frac{abc}{abc}\startdeterminant\NC a \NC a^2 \NC 1 \NR\NC b \NC b^2 \NC 1 \NR\NC c \NC
  c^2 \NC 1\NR\stopdeterminant= \startdeterminant\NC 1 \NC a \NC a^2 \NR\NC 1 \NC b \NC b^2 \NR\NC 1 \NC c
  \NC c^2\NR\stopdeterminant[C_2\leftrightarrow C_3; C_1\leftrightarrow C_2]$
\item $\Delta = \startdeterminant\NC x^2 \NC x + 1 \NC x - 2 \NR\NC 2x^2 \NC 3x \NC 3x - 3
  \NR\NC x^2 \NC 2x - 1 \NC 2x - 1\NR\stopdeterminant + \startdeterminant\NC x \NC x + 1 \NC x - 2
  \NR\NC 3x - 1 \NC 3x \NC 3x - 3 \NR\NC 2x + 3 \NC 2x - 1 \NC 2x - 1\NR\stopdeterminant$

  $= \startdeterminant\NC 2x^2 \NC 3x \NC 3x - 3 \NR\NC 2x^2 \NC 3x \NC 3x - 3 \NR\NC x^2 \NC
  2x - 1 \NC 2x - 1\NR\stopdeterminant[R_1\rightarrow R_1 + R_3] + \\\startdeterminant\NC 2
  \NC 3 \NC x - 2 \NR\NC 2 \NC 3 \NC 3x - 3 \NR\NC 4 \NC 0 \NC 2x -1\NR\stopdeterminant[C_1\rightarrow
    C_1-C_3;C_2\rightarrow C_2 - C_3]$

  $= 0 + \startdeterminant\NC 2 \NC 3 \NC x \NR\NC 2 \NC 3 \NC 3x \NR\NC 4 \NC 0 \NC
  2x\NR\stopdeterminant + \startdeterminant\NC 2 \NC 3 \NC -2 \NR\NC 2 \NC 3 \NC -3 \NR\NC 4 \NC 0 \NC
  -1\NR\stopdeterminant$

  $= xA + B$, where $A = \startdeterminant\NC 2 \NC 3 \NC 1 \NR\NC 2 \NC 3 \NC 3 \NR\NC
  4 \NC 0 \NC 2\NR\stopdeterminant$ and $B = \startdeterminant\NC 2 \NC 3 \NC -2 \NR\NC 2 \NC 3
  \NC -3 \NR\NC 4 \NC 0 \NC -1\NR\stopdeterminant$ which are determinants of 3rd order
  independent of $x$.
\item $\displaystyle\sum_{r = 1}^n D_r = D_1 + D_2 + \cdots + D_n$

  $= \startdeterminant\NC\displaystyle\sum_{r=1}^nr \NC x \NC \frac{n(n + 1)}{2} \NR\NC
  \displaystyle\sum_{r=1}^n(2r - 1) \NC y \NC n^2 \NR\NC\displaystyle \sum_{r=1}^n(3r - 2) \NC z \NC
  \frac{n(3n - 1)}{2}\NR\stopdeterminant = \startdeterminant\NC\frac{n(n + 1)}{2} \NC x \NC \frac{n(n +
    1)}{2} \NR\NC n^2 \NC y \NC n^2 \NR\NC \frac{n(3n - 1)}{2} \NC z \NC \frac{n(3n -
    1)}{2}\NR\stopdeterminant$

  $= 0$ because first and third columns are identical.
\item $\Delta = \startdeterminant\NC -5 \NC 3 + 5i \NC \frac{3}{2} - 4i \NR\NC 3 - 5i \NC
  8 \NC 4 + 5i \NR\NC \frac{3}{2} + 4i \NC 4 - 5i \NC 9\NR\stopdeterminant$

  $\overline{\Delta} = \startdeterminant\NC -5 \NC 3 - 5i \NC \frac{3}{2} + 4i \NR\NC 3 + 5i \NC 8 \NC 4 -
  5i \NR\NC \frac{3}{2} - 4i \NC 4 + 5i \NC 9\NR\stopdeterminant$

  Exchanging rows and columns

  $\overline{\Delta} = \Delta. \therefore \Delta$ is purely real.
\item Putting $b = -c$, we have

  $\Delta = \startdeterminant\NC -2a \NC a - c \NC a + c \NR\NC -c + a \NC 2c \NC 0 \NR\NC c
  + a \NC 0 \NC -2c\NR\stopdeterminant$

  $= \startdeterminant\NC c - a \NC a - c \NC a - c \NR\NC a - c \NC 2c \NC 0 \NR\NC c + a \NC
  0 \NC -2c\NR\stopdeterminant[R_1 \rightarrow R_1 + R_3]$

  $= \startdeterminant\NC c - a \NC 0 \NC 0 \NR\NC a - c \NC a + c \NC a - c \NR\NC c + a \NC
  a + c \NC a - c\NR\stopdeterminant[C_2\rightarrow C_2 + C_1; C_3\rightarrow C_3 +
    C_1]$

  $= (c - a)[(a ^2 - c^2) - (a^2 - c^2)] = 0$

  Hence, $b + c$ is a factor of $\Delta$. Similarly it can be
  proven that $a + b$ and $c + a$ are factors of $\Delta$.


  We see that, upon expansion of determinant, each term of the L.H.S.\ and
  R.H.S.\ is a homogeneous expression in $a,b,c$ of 3rd degree.

  Let $\startdeterminant\NC -2a \NC a + b \NC b + c \NR\NC b + a \NC -2b \NC b +
  c \NR\NC c + a \NC c + b \NC -2c\NR\stopdeterminant = k(b + c)(c + a)(a + b)$, where
  $k$ is independent of $a,b,c$

  Putting $a = 0, b = 1, c = 1$ we get

  $\startdeterminant\NC 0 \NC 1 \NC 1  \NR\NC 1 \NC -2 \NC 2 \NR\NC 1 \NC 2 \NC -2 \NR\stopdeterminant
  = 2k$

  $k = 4$.

  Thus, we have proven the required condition.
\item $F'(a) = \startdeterminant\NC f_1'(a) \NC f_2'(a) \NC f_3'(x) \NR\NC g_1(a) \NC
  g_2(x) \NC g_3(a) \NR\NC h_1(a) \NC h_2(a) \NC h_3(a)\NR\stopdeterminant +
  \startdeterminant\NC f_1(a) \NC f_2(a) \NC f_3(x) \NR\NC g_1'(a) \NC
  g_2'(x) \NC g_3'(a) \NR\NC h_1(a) \NC h_2(a) \NC h_3(a)\NR\stopdeterminant +
  \startdeterminant\NC f_1(a) \NC f_2(a) \NC f_3(x) \NR\NC g_1(a) \NC g_2(x) \NC g_3(a) \NR\NC
  h_1'(a) \NC h_2'(a) \NC h_3'(a)\NR\stopdeterminant$

  $= 0 + 0 + 0$

  $\because f_r(a)=g_r(a)=h_r(a), r=1,2,3$ in the first determinant
  last two, in the second determinant first and third, in the third
  determinant first two, rows are identical. Therefore, all determinants are
  zero.
\item Since $f(x) = 0$ is a quadratic equation with repeated root
  $\alpha, \therefore f(x) = a_r(x - \alpha)^2$, where $a_r$ is a
  constant.

  Clearly $\Delta(x)$ is a polynomial of degree having a maximum value
  of $5$.

  $\Delta(\alpha) = \startdeterminant\NC A(\alpha) \NC B(\alpha) \NC C(\alpha)
  \NR\NC A(\alpha) \NC B(\alpha) \NC C(\alpha) \NR\NC A'(\alpha) \NC B'(\alpha) \NC
  C'(\alpha)\NR\stopdeterminant$

  $\Delta(\alpha) = 0 [\because R_1$ and $R_2$ are identical$]$.

  $\Delta'(\alpha) = \startdeterminant\NC A'(\alpha) \NC B'(\alpha) \NC
  C'(\alpha) \NR\NC A(\alpha) \NC B(\alpha) \NC C(\alpha) \NR\NC A'(\alpha) \NC B'(\alpha)
  \NC C'(\alpha)\NR\stopdeterminant = 0 [\because R_1$ and $R_3$ are
    identical$]$.

  Thus, we can say that $\Delta(x) = 0$ has two roots equal to $\alpha$.

  $\Rightarrow \Delta(x) = (x - \alpha)^2g(x)$, where $g(x)$ is a polynomial of degree $3$ at most.

  $\Delta(x) = a(x - \alpha)^2\frac{g(x)}{a} = a(x - \alpha)^2.h(x)$, where $h(x) = \frac{g(x)}{a_1}$.

  Thus, $\Delta(x) = f(x).h(x)$, where $h(x)$ is a polynomial in $x$. Hence, $\Delta(x)$ is divisible by
  $f(x)$.
\item Let $\Delta$ be the determinant. Then,

  $\frac{d\Delta}{d\theta} = \startdeterminant\NC-\sin(\theta + \alpha) \NC
  -\sin(\theta + \beta) \NC -\sin(\theta + \gamma) \NR\NC \sin(\theta + \alpha) \NC
  \sin(\theta + \beta) \NC \sin(\theta + \gamma) \NR\NC \sin(\beta + \gamma) \NC
  \sin(\gamma - \alpha) \NC \sin(\alpha - \beta)\NR\stopdeterminant +
  \startdeterminant\NC\cos(\theta + \alpha) \NC cos(\theta + \beta) \NC \cos(\theta +
  \gamma) \NR\NC \cos(\theta + \alpha) \NC cos(\theta + \beta) \NC \cos(\theta +
  \gamma) \NR\NC \sin(\beta + \gamma) \NC \sin(\gamma - \alpha) \NC \sin(\alpha -
  \beta)\NR\stopdeterminant + \\ \startdeterminant\NC\cos(\theta + \alpha) \NC
  cos(\theta + \beta) \NC \cos(\theta + \gamma) \NR\NC \cos(\theta + \alpha) \NC
  cos(\theta + \beta) \NC \cos(\theta + \gamma) \NR\NC 0 \NC 0 \NC 0\NR\stopdeterminant$

  $= 0 + 0 + 0$

  Thus, $\Delta$ is a constant, which will be independent of $\theta$.
\item $\Delta = \startdeterminant\NC f \NC g \NC h \NR\NC xf' + f \NC xg' + g \NC xh' + h \NR\NC
  x^2f'' + 4xf' + 2f \NC x^2g'' + 4xg' + 2g \NC x^2h'' + 4xh' + 2h\NR\stopdeterminant$

  $= \startdeterminant\NC f \NC g \NC h \NR\NC xf' \NC xg' \NC xh' \NR\NC x^2f'' + 4xf' \NC
  x^2g'' + 2xg' \NC x^2h'' + 2xh'\NR\stopdeterminant[R_2\rightarrow R_2 - R_1;
    R_3\rightarrow R_3 - 2R_1]$

  $= \startdeterminant\NC f \NC g \NC h \NR\NC xf' \NC xg' \NC xh' \NR\NC x^2f'' \NC x^2g'' \NC
  x^2h''\NR\stopdeterminant[R_3 \rightarrow R_3 - 4R_2]$

  $= x^3\startdeterminant\NC f \NC g \NC h\NR\NC f' \NC g' \NC h' \NR\NC f'' \NC g'' \NC
  h''\NR\stopdeterminant$

  $\Delta' = \startdeterminant\NC f' \NC g ' \NC h' \NR\NC f' \NC g ' \NC h' \NR\NC x^3f'' \NC
  x^3g'' \NC x^3h''\NR\stopdeterminant + \startdeterminant\NC f \NC g \NC h \NR\NC f'' \NC g'' \NC h''
  \NR\NC x^3f'' \NC x^3g'' \NC x^3h''\NR\stopdeterminant + \startdeterminant\NC f \NC g \NC h \NR\NC f'
  \NC g' \NC h' \NR\NC (x^2f'')' \NC (x^2g'')' \NC (x^2h'')'\NR\stopdeterminant$

  $= 0 + 0 + \startdeterminant\NC f \NC g \NC h \NR\NC f' \NC g' \NC h' \NR\NC
  (x^2f'')' \NC (x^2g'')' \NC (x^2h'')'\NR\stopdeterminant$ because two rows of first
  two determinants are equal.
\item $\frac{d^n\{f(x)\}}{dx^n} =
  \startdeterminant\NC \frac{d^nx^n}{dx^n} \NC \frac{d^n\sin x}{dx^n} \NC
  \frac{d^n\cos x}{dx^n} \NR\NC n! \NC \sin \frac{n\pi}{2} \NC \cos \frac{n\pi}{2}
  \NR\NC a \NC a ^2 \NC a^2\NR\stopdeterminant$

  $y = x^n, y_1 = \frac{dy}{dx} = nx^{n - 1}, y_2 =
  \frac{d^2y}{dy^2} = n(n - 1)x^{n - 1}, \ldots y_n = n(n - 1)\ldots 3.2.1 =
  n!$

  $y = \sin x, y_1 = \cos x = \sin\left(\frac{\pi}{2} + x\right), y_2
  = \cos\left(\frac{\pi}{2} + x\right) = \sin\left(\frac{\pi}{2} +
  \frac{\pi}{2} + x\right)$

  Proceeding in the same way $y_n = \sin\left(\frac{n\pi}{2} + x\right)$

  Now $y = \cos x, y_1 = -\sin x = \cos \left(\frac{\pi}{2} + x\right),
  y_2 = -sin\left(\frac{\pi}{2} + x\right) = \cos\left(2\frac{\pi}{2} +
  x\right)$

  Proceeding in the same way $y_n = \cos\left(n\frac{\pi}{2} + x\right)$

  $\frac{d^n\{f(x)\}}{dx^n} = \startdeterminant\NC n!
  \NC\sin\left(\frac{n\pi}{2} + x\right) \NC \cos\left(\frac{n\pi}{2} + x\right)
  \NR\NC n! \NC \sin\frac{n\pi}{2} \NC \cos\frac{n\pi}{2} \NR\NC a \NC a^2 \NC a^3\NR\stopdeterminant$

  $f^n(0) = \startdeterminant\NC n! \NC\sin\frac{n\pi}{2} \NC
  \cos\frac{n\pi}{2} \NR\NC n! \NC \sin\frac{n\pi}{2} \NC \cos\frac{n\pi}{2} \NR\NC a \NC
  a^2 \NC a^3\NR\stopdeterminant = 0$ because first two rows are identical.
\item $\Delta = \startdeterminant\NC\cos A\cos P + \sin A\sin P \NC \cos A\cos
  Q + \sin A\sin Q \NC \cos A\cos R + \sin A\sin R \NR\NC \cos B\cos P + \sin
  B\sin P \NC \cos B\cos Q + \sin B\sin Q \NC \cos B\cos R + \sin B\sin R \NR\NC
  \cos C\cos P + \sin C\sin P \NC \cos C\cos Q + \sin C\sin Q \NC \cos C\cos R +
  \sin C\sin R \NR\stopdeterminant$

  $= \startdeterminant\NC\cos A \NC \sin A \NC 0 \NR\NC \cos B \NC \sin B \NC 0 \NR\NC \cos
  C \NC \sin C \NC 0\NR\stopdeterminant + \startdeterminant\NC\cos P \NC \sin P \NC 0 \NR\NC \cos Q
  \NC \sin Q \NC 0 \NR\NC \cos R \NC \sin R \NC 0\NR\stopdeterminant$

  $= 0 + 0 = 0$.

  As an alternative we can expand the determinant along first column and then split the addition taking
  $\cos P$ and $\sin P$ common and repeat it for others to get the desired result.
\item We know that $\startdeterminant\NC a \NC b \NC c \NR\NC b \NC c \NC a \NR\NC c \NC a \NC
  b\NR\stopdeterminant = 3abc - a^3 - b^3 - c^3$

  $\Rightarrow (a^3 + b^3 + c^3 - 3abc)^2 = \startdeterminant\NC a \NC b \NC c \NR\NC b \NC c \NC a
  \NR\NC c \NC a \NC b\NR\stopdeterminant^2$

  $= \startdeterminant\NC a \NC b \NC c \NR\NC b \NC c \NC a \NR\NC c \NC a \NC
  b\NR\stopdeterminant\startdeterminant\NC -a \NC c \NC b \NR\NC -b \NC a \NC c \NR\NC -c \NC b \NC
  a\NR\stopdeterminant[C_1\rightarrow -C_1; C_2\leftrightarrow C_3]= \startdeterminant\NC 2bc - a^2 \NC c^2
  \NC b^2 \NR\NC c^2 \NC 2bc - b^2 \NC
  a^2 \NR\NC b^2 \NC a^2 \NC 2bc - c^2\NR\stopdeterminant$
\item L.H.S.\ $= \startdeterminant\NC\sin\alpha \NC \cos\alpha \NC 0 \NR\NC \sin\beta \NC
  \cos\beta \NC 0 \NR\NC \sin\gamma \NC \cos\gamma \NC 0 \NR\stopdeterminant \startdeterminant\NC
  \sin\alpha \NC \cos\alpha \NC 0 \NR\NC \sin\beta \NC \cos\beta \NC 0 \NR\NC \sin\gamma \NC
  \cos\gamma \NC 0 \NR\stopdeterminant$

  $=0.0 = 0$.
\item $\Delta = \startdeterminant\NC3 \NC m \NR\NC 2 \NC -5\NR\stopdeterminant = -(15 + 2m)$

  {\bf Case I:} When $\Delta = 0, m = \frac{-15}{2}$

  $\Delta_1 = \startdeterminant\NC m \NC m \NR\NC 20 \NC -5\NR\stopdeterminant = -25m\neq 0$

  Hence, given system of equation has no solution when $m = \frac{-15}{2}$

  {\bf Case II:} When $m \neq \frac{-15}{2}$

  $\Delta_2 = \startdeterminant\NC2 \NC m \NR\NC 2 \NC 20\NR\stopdeterminant = 2(30 - m)$

  $x = \frac{\Delta_1}{\Delta} = \frac{25m}{15 + 2m} > 0[\because x >0]$

  $\Rightarrow -\infty < m < \frac{-15}{2}$ or $0 < m < \infty$

  $y = \frac{\Delta_2}{\Delta} = \frac{2(m - 30)}{15 + 2m} > 0[\because y > 0]$

  $\Rightarrow -\infty < m < \frac{-15}{2}$ or $30 < m < \infty$

  Combining both we get, $-\infty < m < \frac{-15}{2}$ or $30 < m < \infty$.
\item $\Delta = \startdeterminant\NC3 \NC -1 \NC 4 \NR\NC 1 \NC 2 \NC -3 \NR\NC 6 \NC 5 \NC
  \lambda\NR\stopdeterminant = 7(\lambda + 5)$

  {\bf Case I:} When $\lambda \neq 5 \Rightarrow \Delta \neq 0$ which
  means the system of equations has unique solution.

  {\bf Case II:} When $\lambda = -5 \Rightarrow \Delta = 0$

  Also, $\Delta_1 = \startdeterminant\NC3 \NC -1 \NC 4 \NR\NC -2 \NC 2 \NC -3 \NR\NC 3 \NC 5
  \NC -5\NR\stopdeterminant = 0, \Delta_2 = \startdeterminant\NC3 \NC 3 \NC 4 \NR\NC 1 \NC -2 \NC -3
  \NR\NC 6 \NC -3 \NC -5\NR\stopdeterminant = 0$

  $\Delta_3 = \startdeterminant\NC3 \NC -1 \NC 3 \NR\NC 1 \NC 2 \NC -2 \NR\NC 6 \NC 5 \NC
  -3\NR\stopdeterminant = 0$

  Since all the determinants are zero, in this case we have infinite solutions for given system of equations.

  Putting the value of $\lambda$ the set of equation becomes

  $3x - y + 4z = 3; x + 2y - 3z = -2; 6x + 5y - 5z = -3$

  From first two equations we get, $z = \frac{4 - 7x}{5}$

  Substituting this in first we get $y = \frac{1 - 13x}{5}$

  Thus the set of solutions is $x = t, y = \frac{1 - 13t}{5}, z = \frac{4 - 7t}{5}$.
\item $\Delta = \startdeterminant\NC2 \NC p \NC 6 \NR\NC 1 \NC 2 \NC q \NR\NC 1 \NC 1 \NC
  3\NR\stopdeterminant = (p - 2)(q - 3)$

  $\Delta_1 = \startdeterminant\NC8 \NC p \NC 8 \NR\NC 5 \NC 2 \NC q \NR\NC 4 \NC 1 \NC
  3\NR\stopdeterminant = (p - 2)(4q - 15)$

  $\Delta_2 = \startdeterminant\NC 2 \NC 8 \NC 6 \NR\NC 1 \NC 5 \NC 1 \NR\NC 1 \NC 4 \NC
  3\NR\stopdeterminant = 0$

  $\Delta_3 = \startdeterminant\NC2 \NC p \NC 8 \NR\NC 1 \NC 2 \NC 5 \NR\NC 1 \NC 1 \NC
  4\NR\stopdeterminant = p -2$

  {\bf Case I:} When $\Delta \neq 0$ i.e. $p \neq 2, q\neq 3$, given system of equations has unique
  solution.

  {\bf Case II:} When $\Delta = 0, p = 2,$ or $q = 3$

  When $p = 2 \Rightarrow \Delta_1 = 0, \Delta_2 = 0, \Delta_3 = 0$

  Thus, given system of equations has inifinite solutions.

  When $q = 3\Rightarrow \Delta_1\neq 0$

  Thus, given system of equations has no solutions.
\item For non-trivial solution

  $\Delta = 0$ or $\startdeterminant\NC\lambda \NC \sin\alpha \NC
  \cos\alpha \NR\NC 1 \NC \cos\alpha \NC \sin\alpha \NR\NC -1 \NC \sin\alpha \NC \cos\alpha
  \NR\stopdeterminant = 0$

  $\Rightarrow \lambda = \sin 2\alpha + \cos 2\alpha$

  If $\lambda = 1, \sin 2\alpha + \cos 2\alpha = 1$

  $\Rightarrow \sin 2\alpha = 1 - \cos 2\alpha = 2\sin^2\alpha$

  $\Rightarrow 2\sin\alpha(\cos\alpha - \sin\alpha) = 0$

  $\therefore \sin\alpha = 0$ or $\tan\alpha = 1$

  $\therefore \alpha = n\pi$ or $\alpha = n\pi + \frac{\pi}{4}, n\in I$.
\item $\Delta = (a + b + c)\startdeterminant\NC 1 \NC b + c \NC a^2 \NR\NC 1 \NC c + a \NC b^2 \NR\NC 1
  \NC a + b \NC c^2\NR\stopdeterminant[C_1\rightarrow C_1 + C_2]$

  $=(a + b + c)\startdeterminant\NC 1 \NC b + a \NC a^2 \NR\NC 0 \NC a - b \NC b^2 - a^2
  \NR\NC 0 \NC a - c \NC c^2 - a^2\NR\stopdeterminant[R_3\rightarrow R_3 - R_1; R_2\rightarrow R_2 - R_1]$

  $= (a + b + c)[(a - b)(c^2 - a^2) - (a - c)(b^2 - a^2)] = (a + b + c)(a - b)(c - a)(c + a - b - a)$

  $= -(a + b + c)(a - b)(b - c)(c - a)$.
\item $\Delta = \startdeterminant\NC\sqrt{13} + \sqrt{3} \NC 2\sqrt{5} \NC \sqrt{5}
  \NR\NC \sqrt{15} - \sqrt{6} \NC 5 - 2\sqrt{10}  \NC 0 \NR\NC 3 - \sqrt{15} \NC \sqrt{15}
  - 10 \NC 0\NR\stopdeterminant[R_2\rightarrow R_2 - \sqrt{2}R_1;R_3\rightarrow
    R_3 - \sqrt{5}R_1]$

  $= 15\sqrt{2} - 25\sqrt{3}$.
\item $\Delta = \startdeterminant\NC x \NC x(x^2 + 1) \NC x \NR\NC y \NC y(y^2 + 1) \NC
  y \NR\NC z \NC z(z^2 + 1) \NC z\NR\stopdeterminant + \startdeterminant\NC x \NC x(x^2 + 1) \NC 1
  \NR\NC y \NC y(y^2 + 1) \NC 1 \NR\NC z \NC z(z^2 + 1) \NC 1\NR\stopdeterminant$

  Observe that first and third columns of first determinant are identical.

  $\Rightarrow \Delta = \startdeterminant\NC x \NC x(x^2 + 1) \NC 1
  \NR\NC y \NC y(y^2 + 1) \NC 1 \NR\NC z \NC z(z^2 + 1) \NC 1\NR\stopdeterminant$

  $= \startdeterminant\NC x \NC x^3 \NC 1 \NR\NC y \NC y^3 \NC 1 \NR\NC z \NC z^3 \NC
  1\NR\stopdeterminant + \startdeterminant\NC x \NC x^3 \NC x \NR\NC y \NC y^3 \NC y \NR\NC z \NC z^3 \NC
  z\NR\stopdeterminant$

  Again second and third columns are identical in second determinant.

  $\Delta = \startdeterminant\NC x \NC x^3 \NC 1 \NR\NC y \NC y^3 \NC 1 \NR\NC z \NC z^3 \NC
  1\NR\stopdeterminant$

  $= (x - y)(y - z)(z - x)(x + y + z)$.
\item Let $a$ and $d$ be the first term and common difference of
  corresponding A.P.

  $\frac{1}{x} = a + (l - 1)d, \frac{1}{y} = a + (2m - 1)d,
  \frac{1}{z} = a + (3n - 1)d$

  $\Delta = \frac{1}{xyz}\startdeterminant\NC \frac{1}{x} \NC \frac{1}{y} \NC \frac{1}{z} \NR\NC l \NC 2m
  \NC 3n \NR\NC 1 \NC 1 \NC 1\NR\stopdeterminant$

  $= \frac{1}{xyz}\startdeterminant\NC a + (l - 1)d \NC a + (2m - 1)d \NC a +
  (3n - 1)d\NR\NC l \NC 2m \NC 3n \NR\NC 1 \NC 1 \NC 1\NR\stopdeterminant$

  $= \frac{1}{xyz}\startdeterminant\NC ld - d \NC 2md - 2d \NC 3nd - 3d \NR\NC l \NC
  2m \NC 3n \NR\NC 1 \NC 1 \NC 1\NR\stopdeterminant[R_1\rightarrow R_1 - aR_3]$

  $= \frac{1}{xyz}\startdeterminant\NC 0 \NC 0 \NC 0 \NR\NC l\NC 2m \NC 3n \NR\NC 1 \NC 1 \NC
  1\NR\stopdeterminant[R_1\rightarrow R_1 - (R_2 - 1)d]$

  $=0$.
\item $\Delta = \startdeterminant\NC 1 \NC a^2 \NC a^3 \NR\NC 0 \NC b^2 - a^2 \NC b^3 - a^3
  \NR\NC 0 \NC c^2 -a^2\NC c^3 - a^3\NR\stopdeterminant[R_2 \rightarrow R_2 - R_1; R_3
    \rightarrow R_3 - R_1]$

  $= (b^2 - a^2)(c^3 - a^3) - (c^2 - a^2)(b^3 - a^3)$

  $= (b - a)(c - a)[(b + a)(c^2 + ac + a^2) - (c + a)(b^2 + ab + a^2)]$

  $= (b - a)(c - a)(bc^2 + abc + a^2b + ac^2 + a^2c + a^3 - b^2c - abc - a^2c - ab^2 - a^2b - a^3)$

  $= (b - a)(c - a)(bc^2 + ac^2 - b^2c - ab^2) = (b - a)(c - a)[bc(c - b) + a(c^2 - b^2)]$

  $= (b - a)(c - a)(c - b)(ab + bc + ca)$

  We know that $\startdeterminant\NC 1 \NC a \NC a^2 \NR\NC 1 \NC b \NC b^2
  \NR\NC 1 \NC c \NC c^2\NR\stopdeterminant = (b - a)(c - a)(c - b)$

  Hence, L.H.S.\ $=$ R.H.S.
\item $\Delta = \startdeterminant\NC b^2 + c^2 \NC a^2 \NC bc \NR\NC c^2 + a^2 \NC b^2 \NC
  ca \NR\NC a^2 + b^2 \NC c^2 \NC ab\NR\stopdeterminant[C_1\rightarrow -2C_3]$

  $= (a^2 + b^2 + c^2)\startdeterminant\NC 1 \NC a^2 \NC bc \NR\NC 1 \NC b^2 \NC ca \NR\NC 1
  \NC c^2 \NC ab\NR\stopdeterminant[C_1\rightarrow C_1 + C_2 + C_3]$

  We know that $\startdeterminant\NC1 \NC a^2 \NC bc \NR\NC 1 \NC b^2 \NC ca \NR\NC 1
  \NC c^2 \NC ab\NR\stopdeterminant = (a - b)(b - c)(c - a)(a + b + c)$

  $\Delta = (a^2 + b^2 + c^2)(a + b + c)(a - b)(b - c)(c - a)$.
\item Perform $C_2\rightarrow C_2 - C_1$, take out $-1$ common then $C_1\rightarrow C_1 + C_2 + 2C_3$, take
  out $x^2 + y^2 + z^2$ commong then $R_3\rightarrow R_3 - R_2$ and $R_2\rightarrow R_2 - R_1$ will give two
  zeroes in first column which upon expansion gives the result.
\item Let $a_1b_1c_1 = 100\times a_1 + 10\times b_1 + c_1 = pk$, where
  $p\in I$

  $a_2b_2c_2 = 100\times a_2 + 10\times b_2 + c_2 = qk$, where
  $q\in I$

  $a_3b_3c_3 = 100\times a_3 + 10\times b_3 + c_3 = rk$, where
  $r\in I$

  $\Delta = \startdeterminant\NC a_1 \NC b_1 \NC pk \NR\NC a_2 \NC b_2 \NC qk \NR\NC a_3 \NC
  b_3 \NC rk\NR\stopdeterminant[C_3\rightarrow 100C_1 + 10C_2 + C_3]$

  Thus, given determinant is divisible by $k$.
\item $\Delta = \startdeterminant\NC a_1 \NC a_1x + b_1 \NC c_1 \NR\NC a_2 \NC a_2x + b_2
  \NC c_2 \NR\NC a_3 \NC a_3x + b_3 \NC c_3\NR\stopdeterminant + \startdeterminant\NC b_1x \NC
  a_1x + b_1 \NC c_1 \NR\NC b_2x \NC a_2x + b_2 \NC c_2 \NR\NC b_3x \NC a_3x + b_3 \NC
  c_3\NR\stopdeterminant$

  $= \startdeterminant\NC a_1 \NC a_1x + b_1 \NC c_1 \NR\NC a_2 \NC a_2x + b_2
  \NC c_2 \NR\NC a_3 \NC a_3x + b_3 \NC c_3\NR\stopdeterminant + x\startdeterminant\NC b_1 \NC
  a_1x + b_1 \NC c_1 \NR\NC b_2 \NC a_2x + b_2 \NC c_2 \NR\NC b_3 \NC a_3x + b_3 \NC
  c_3\NR\stopdeterminant$

  $= x\startdeterminant\NC a_1 \NC a_1 \NC c_1 \NR\NC a_2 \NC a_2 \NC c_2 \NR\NC a_3 \NC a_3
  \NC c_3\NR\stopdeterminant + \startdeterminant\NC a_1 \NC b_1 \NC c_1 \NR\NC a_2 \NC b_2 \NC c_3 \NR\NC
  a_3 \NC b_3 \NC c_3\NR\stopdeterminant + \\ x\startdeterminant\NC b_1 \NC a_1x \NC c_1 \NR\NC b_2
  \NC a_2x \NC c_2 \NR\NC b_3 \NC a_3x \NC c_3 \NR\stopdeterminant + \startdeterminant\NC b_1 \NC b_1 \NC
  c_1 \NR\NC b_2 \NC b_2 \NC c_2 \NR\NC b_3 \NC b_3 \NC c_3\NR\stopdeterminant$

  Clearly, first and last determinants are zero as they have identical columns.

  $= \startdeterminant\NC a_1 \NC b_1 \NC c_1 \NR\NC a_2 \NC b_2 \NC c_3 \NR\NC a_3 \NC b_3 \NC
  c_3\NR\stopdeterminant + x^2\startdeterminant\NC b_1 \NC a_1 \NC c_1 \NR\NC b_2 \NC a_2 \NC c_2 \NR\NC
  b_3 \NC a_3 \NC c_3\NR\stopdeterminant$

  Exchanging first two columns of second determinant

  $= (1 - x^2)\startdeterminant\NC a_1 \NC b_1 \NC c_1 \NR\NC a_2 \NC b_2 \NC c_2 \NR\NC a_3
  \NC b_3 \NC c_3\NR\stopdeterminant$.
\item $\Delta = abc\startdeterminant\NC\frac{1}{a} + 1 \NC \frac{1}{a}
  \NC \frac{1}{a} \NR\NC \frac{1}{b} \NC \frac{1}{b} + 1 \NC \frac{1}{b} \NR\NC
  \frac{1}{c} \NC \frac{1}{c} \NC \frac{1}{c} + 1\NR\stopdeterminant$

  $= abc\left(\frac{1}{a} + \frac{1}{b} + \frac{1}{c} + 1\right)
  \startdeterminant\NC 1 \NC 1 \NC 1 \NR\NC \frac{1}{b} \NC \frac{1}{b} + 1 \NC \frac{1}{b}
  \NR\NC \frac{1}{c} \NC \frac{1}{c} \NC \frac{1}{c} + 1\NR\stopdeterminant$

  $= abc\left(\frac{1}{a} + \frac{1}{b} + \frac{1}{c} + 1\right)
  \startdeterminant\NC1 \NC 0 \NC 0 \NR\NC \frac{1}{b} \NC 1 \NC 0 \NR\NC \frac{1}{c} \NC 0 \NC
  1\NR\stopdeterminant[C_2 \rightarrow C_2 - C_1; C_3\rightarrow C_3 - C_1]$

  $= abc\left(\frac{1}{a} + \frac{1}{b} + \frac{1}{c} + 1\right)$

  Now since $a, b, c$ are roots of $px^3 + qx^2 + rx + s = 0$

  $\therefore px^3 + qx^2 + rx + s = (x - a)(x - b)(x - c)$

  Comparing coefficients, $a + b + c = \frac{-q}{p}$

  $ab + bc + ca = \frac{r}{p}; abc = \frac{-s}{p}$

  Thus, $abc\left(\frac{1}{a} + \frac{1}{b} + \frac{1}{c} + 1\right) =
  \frac{r - s}{p}$.
\item $\Delta = \startdeterminant\NC 1 \NC a \NC a^4 \NR\NC 0 \NC b - a \NC b^4 - a^4 \NR\NC 0 \NC
  c - a \NC c^4 - a^4\NR\stopdeterminant[R_2\rightarrow R_2 - R_1; R_3 \rightarrow
    R_3 - R-1]$

  $= (b - a)(c^4 - a^4) - (c - a)(b^4 - a^4)$

  $= (b - a)(c - a)[(c + a)(c^2 + a^2) - (b + a)(b^2 + a^2)] >
  0\;\forall\;a<b<c$
\item $\Delta = \startdeterminant\NC a \NC a^3 \NC a^4 \NR\NC b \NC b^3 \NC b^4 \NR\NC c \NC c^3 \NC
  c^4\NR\stopdeterminant - \startdeterminant\NC a \NC a^3 \NC 1 \NR\NC b \NC b^3 \NC 1 \NR\NC c \NC c^3 \NC
  1\NR\stopdeterminant$

  $= abc \startdeterminant\NC 1 \NC a^2 \NC a^3 \NR\NC 1 \NC b^2 \NC b^3 \NR\NC 1 \NC c^2 \NC
  c^3\NR\stopdeterminant - \startdeterminant\NC a \NC a^3 \NC 1 \NR\NC b \NC b^3 \NC 1 \NR\NC c \NC c^3 \NC
  1\NR\stopdeterminant$

  $= abc \startdeterminant\NC 0 \NC a^2 - b^2 \NC a^3 - b^3\NR\NC 0 \NC b^2 - c^2 \NC
  b^3 - c^3 \NR\NC 1 \NC c^2 \NC c^3\NR\stopdeterminant - \startdeterminant\NC0 \NC a - b \NC a^3 -
  b^3 \NR\NC 0 \NC b - c \NC b^3 - c^3 \NR\NC 1 \NC c \NC c^3\NR\stopdeterminant[R_1 \rightarrow
    R_1 - R_2; R_2 \rightarrow R_2 -R_3]$

  $\Rightarrow abc[(a^2 - b^2)(b^3 - c^3) - (b^2 - c^2)(a^3 - b^3)] - [(a -
    b)(b^3 - c^3) - (b - c)(a^3 - b^3)] = 0$

  $abc(a - b)(b - c)[(a + b)(b^2 + bc + c^2) - (b + c)(a^2 + ab + b^2)] =\NR\NC
  (a - b)(b - c)(b^2 + bc + c^2 - a^2 - ab - b^2)$

  Becasue $a, b, c$ are distinct we $a - b \neq 0; b - c \neq 0;
  c - a\neq 0$

  $abc(a + b)(b^2 + bc + c^2) - (b + c)(a^2 + ab + b^2) = (b^2 + bc +
  c^2 - a^2 - ab - b^2)$

  $abc(ab^2 + abc + ac^2 + b^3 + b^2c + bc^2 - a^2b - ab^2 - b^3 -
  a^2c - abc - b^2c) = bc + c^2 - a^2 -ab$

  $abc(ac^2 + bc^2 - a^2b - a^2c) = b(c - a) + (c^2 - a^2)$

  $abc[ac(c - a) + b(c^2 - a^2)] = (c - a)(a + b + c)$

  $abc(ab + bc + ca) = a + b + c$
\item Taking $b_1, b_2, b_3$ common from columns and multiplying rows with
  them, we get

  $\Delta = \startdeterminant\NC x_1 + a_1b_1 \NC a_1b_1 \NC a_1b_1 \NR\NC a_2b_2 \NC
  x_2 + a_2b_2 \NC a_2b_2 \NR\NC a_3b_3 \NC a_3b_3 \NC x + a_3b_3\NR\stopdeterminant$

  Taking $x_1,x_2,x_3$ common from rows

  $= x_1x_2x_3\startdeterminant\NC1 + \frac{a_1b_1}{x_1} \NC \frac{a_1b_1}{x_1} \NC
  \frac{a_1b_1}{x_1} \NR\NC \frac{a_2b_2}{x_2} \NC 1 + \frac{a_2b_2}{x_2} \NC
  \frac{a_2b_2}{x_2} \NR\NC \frac{a_3b_3}{x_3} \NC \frac{a_3b_3}{x_3} \NC 1 +
  \frac{a_3b_3}{x_3}\NR\stopdeterminant$

  $= x_1x_2x_3\left(1 + \frac{a_1b_1}{x_1} + \frac{a_2b_2}{x_2} +
  \frac{a_3b_3}{c_3}\right) \startdeterminant\NC1 \NC 1 \NC 1 \NR\NC \frac{a_2b_2}{x_2} \NC
  1 + \frac{a_2b_2}{x_2} \NC \frac{a_2b_2}{x_2} \NR\NC \frac{a_3b_3}{x_3} \NC
  \frac{a_3b_3}{x_3} \NC 1 + \frac{a_3b_3}{x_3}\NR\stopdeterminant[R_1 \rightarrow
    R_1 + R_2 + R_3]$

  $= x_1x_2x_3\left(1 + \frac{a_1b_1}{x_1} + \frac{a_2b_2}{x_2} +
  \frac{a_3b_3}{c_3}\right)\startdeterminant\NC1 \NC 0 \NC 0 \NR\NC \frac{a_2b_2}{x_2} \NC
  1 \NC 0 \NR\NC \frac{a_3b_3}{x_3} \NC 0 \NC 1\NR\stopdeterminant[C_2\rightarrow C_2 - C_1;
    C_3 \rightarrow C_3 - C_1]$

  $= x_1x_2x_3\left(1 + \frac{a_1b_1}{x_1} + \frac{a_2b_2}{x_2} +
  \frac{a_3b_3}{c_3}\right)$.
\item This problem is similar to problem $90$ and can be solved similarly.
\item $\Delta = abc\startdeterminant\NC a \NC c \NC a + c \NR\NC a + b \NC b \NC a \NR\NC b \NC
  b + c \NC c\NR\stopdeterminant$

  $= abc \startdeterminant\NC-2a \NC -2b \NC 0 \NR\NC a + b \NC b \NC a \NR\NC b \NC
  b + c \NC c\NR\stopdeterminant[R_1 \rightarrow R_1 - R_2 - R_3]$

  $= 4a^2b^2c^2$.
\item $\Delta = \startdeterminant\NC 1 + a^2 + b^2 \NC 0 \NC -2b \NR\NC 0 \NC 1 + a^2 +
  b^2 \NC 2a \NR\NC b(1 + a^2 + b^2) \NC -a(1 + a^2 + b^2) \NC 1 - a^2 -
  b^2\NR\stopdeterminant[C_1\rightarrow C_1 - bC_3; C_2\rightarrow C_2 + aC_3]$

  $= (1 + a^2 + b^2)^2(1 -a^2 - b^2 + 2a^2) + 2b^2(1 + a^2 + b^2)^2$

  $= (1 + a^2 + b^2)^3$.
\item We know that $P = \frac{a + b + c}{a}; A = \sqrt{s(s - a)(s - b)(s -
  c)}$

  After that this problem is same as $91$, and we just need to
  substitute for the values of $A$ and $P$.
\item Taking $a, b, c$ common from rows and multiplying with columns gives is
  the same determinant as in problem $88$ and can be solved in same fashion.
\item $\Delta = \startdeterminant\NC x^3 \NC 6x^2a + 2a^3 \NC (x - a)^3 \NR\NC y^3 \NC
  6y^2a + 2a^3 \NC (y - a)^3 \NR\NC z^3 \NC 6z^2a + 2a^3 \NC (z - a)^3\NR\stopdeterminant
  [C_2 \rightarrow C_2 - C_3]$

  $= 2\startdeterminant\NC x^3 \NC 3x^2a + a^3 \NC (x - a)^3 \NR\NC y^3 \NC 3y^2a + a^3
  \NC (y - a)^3 \NR\NC z^3 \NC 3z^2a + a^3 \NC (z - a)^3\NR\stopdeterminant$

  $= 2 \startdeterminant\NC x^3 \NC 3x^2a + a^3 \NC 3xa^2 \NR\NC y^3 \NC 3y^2a + a^3 \NC
  3ya^2 \NR\NC z^3 \NC 3z^2a + a^3 \NC 3za^2\NR\stopdeterminant [C_3 \rightarrow C_3 - C_1
    - C_2]$

  $= 2a^3 \startdeterminant\NC x^3 \NC 3x^2 + a^2 \NC 3x \NR\NC y^3 - x^3 \NC 3(y^2 -
  x^2) \NC 3(y - x) \NR\NC z^3 - x^3 \NC 3(z^2 - x^2) \NC 3(z - x)\NR\stopdeterminant
  [R_2\rightarrow R_2 - R_1; R_3 \rightarrow R_2 - R_1]$

  Now we can take $y - x$ and $z - x$ common followed by expasion so that desired condition can be proven
  easily.
\item $\Delta = \startdeterminant\NC 1- x \NC a \NC 0 \NR\NC a \NC a^2 - x \NC x \NR\NC a^2 \NC
  a^3 \NC -x\NR\stopdeterminant[C_3 \rightarrow C_3 -aC_2]$

  $= x\startdeterminant\NC 1 - x \NC a \NC 0 \NR\NC a + a^2 \NC a^2 - x + a^3 \NC 0 \NR\NC
  a^2 \NC a^3 \NC -1\NR\stopdeterminant[R_2 \rightarrow R_2 + R_3]$

  $= x[a(a + a^2) - (1 - x)(a^2 + a^3 - x)]$

  $= x(a^2 + a^3 - a^2 - a^3 + x + xa^2 + xa^3 - x^2)$

  $= x^2(1 + a ^2 + a^3)- x^3$.
\item $y_1 = p\cos px, y_2 = -p^2\sin px, y_3 = -p^3\cos px, y_4 = p^4
  \sin px$

  $y_5 = p^5 \cos px, y_6 = -p^6\sin px, y_7 = -p^7\cos px, y_8 =
  p^8\sin px$

  $\Delta = -p^6 \startdeterminant\NC\sin px \NC p\cos px \NC -p^2 \sin px \NR\NC
  -p^3\cos px \NC p^4 \sin px \NC p^5 \cos px \NR\NC \sin px \NC p\cos px \NC -p^2 \sin
  px\NR\stopdeterminant$

  Clearly first and last rows are identical.

  $\Delta = 0$.
\item $\Delta = \startdeterminant\NC 1 \NC 0 \NC -\sin\theta \NR\NC 0 \NC 1 \NC \cos\theta
  \NR\NC \sin\theta \NC -\cos\theta \NC 0\NR\stopdeterminant[C_1 \rightarrow C_1
    - \sin\theta C_3; C_2 \rightarrow C_2 + \cos\theta C_3]$

  $= \cos^2\theta + \sin^2\theta = 1$.
\item $\Delta = \startdeterminant\NC\cos\alpha \NC \sin\alpha\cos\beta \NC 0 \NR\NC
  -\sin\alpha \NC \cos\alpha\cos\beta \NC 0 \NR\NC 0 \NC -\sin\beta \NC \frac{1}{\cos\beta}
  \NR\stopdeterminant[C_3 \rightarrow C_3 - \tan\beta C_2]$

  $= \frac{1}{\cos\beta}[\cos\beta(\cos^2\alpha + \sin^2\alpha)] = 1$.
\item Multiplying columns with $a, b, c,$ we get

  $\Delta = \frac{1}{abc}\startdeterminant\NC a(a^2 + x) \NC ab^2 \NC ac^2 \NR\NC
  a^2b \NC b(b^2 + x) \NC bc^2 \NR\NC a^2c \NC b^2c \NC c(c^2 + x)\NR\stopdeterminant$

  Now taking out $a, b, c$ from rows, we have

  $= \startdeterminant\NC a^2 + x \NC b^2 \NC c^2 \NR\NC a^2 \NC b^2 + x \NC c^2 \NR\NC a^2
  \NC b^2 \NC c^2 + x\NR\stopdeterminant$

  $= \startdeterminant\NC x \NC 0 \NC -x \NR\NC 0 \NC x \NC -x \NR\NC a^2 \NC b^2 \NC c^2 +
  x\NR\stopdeterminant[R_1\rightarrow R_1 - R_3; R_2\rightarrow R_2 - R_3]$

  $= \startdeterminant\NC x \NC 0 \NC 0 \NR\NC 0 \NC x \NC -x \NR\NC a^2 \NC b^2 \NC a^2 + c^2 +
  x\NR\stopdeterminant[C_3 \rightarrow C_3 + C_1]$

  $\Rightarrow x^2(a^2 + b^2 + c^2 + x) = 0$

  $\Rightarrow x=0, -(a^2 + b^2 + c^2)$.
\item By observation if $x = n - 1$ then first two columns are same. Similarly, if $x = n$ then first column
  is equal to sum of two other columns. Thus, $x = n, n - 1$ are two possible solutions.

  If we take $\frac{x!}{r!(x - r)!}$ common from first column and similarly for second and third, then
  we get a quadratic equation which will have two roots and we have found both of them.
\item $\Delta = \frac{1}{a^2}\startdeterminant\NC u + a^2x \NC aw' - bu \NC av' - cu
  \NR\NC w' + abx \NC av - bw' \NC au' - cw' \NR\NC v' + acx \NC au' - bv' \NC aw -
  cv'\NR\stopdeterminant[C_2\rightarrow aC_2 - bC_1; C_3\rightarrow aC_3 - cC_1]$

  $\Rightarrow x = - \startdeterminant\NC u \NC  aw' - bu \NC av' - cu
  \NR\NC w' \NC av - bw' \NC au' - cw' \NR\NC v' \NC au' - bv' \NC aw -
  cv' \NR\stopdeterminant \div \startdeterminant\NC a^2 \NC  aw' - bu \NC av' - cu
  \NR\NC ab \NC av - bw' \NC au' - cw' \NR\NC ac \NC au' - bv' \NC aw -
  cv'\NR\stopdeterminant$.
\item We know that value of the determinant in denominator is $(a - b)(b -
  c)(c - a) = k$ (say)

  $f(a, b, c) = \startdeterminant\NC f(a) - f(b)\NC f(b) - f(c) \NC f(c)
  \NR\NC 0 \NC 0 \NC 1 \NR\NC a - b \NC b - c \NC c\NR\stopdeterminant[C_1\rightarrow C_1 - C_2;
    C_2\rightarrow C_2 - C_3] \div k$

  $= -(b - c)(f(a) - f(b)) - (a - b)(f(b) - f(c))\div k$

  $= -(a - b)(b - c)\left[\frac{f(a) - f(b)}{a - b} - \frac{f(b) -
      f(c)}{(b - c)}\right]\div k$

  $= -(a - b)(b - c)(f(a, b) - f(b, c))\div k$

  $= (a - b)(b - c)(c - a)\frac{f(b, c) - f(a, b)}{c - a} = (a -
  b)(b - c)(c - a)f(a, b, c)\div (a - b)(b - c)(c - a)$

  $= f(a, b, c)$.
\item Becasue $A, B, C$ are angles of a triangle. $A + B + C = \pi$

  Also, $e^{i\pi} = \cos\pi + i\sin\pi = -1$

  Taking $e^{iA}, e^{iB}, e^{iC}$ common from $R_1, R_2, R_3,$
  we get

  $\Delta = e^{i(A + B + C)}\startdeterminant\NC e^{iA} \NC e^{-i(A +C)} \NC
  e^{-i(A + B)} \NR\NC e^{-i(B + C)} \NC e^{iB} \NC e^{-i(A + B)} \NR\NC e^{-i(B + C)} \NC
  e^{-i(A + C)} \NC e^{iC}\NR\stopdeterminant$

  $= -\startdeterminant\NC e^{iA} \NC -e^{iB} \NC -e^{iC} \NR\NC -e^{iA} \NC e^{iB} \NC
  -e^{iC} \NR\NC -e^{iA} \NC -e^{iB} \NC e^{iC}\NR\stopdeterminant$

  Taking $e^{iA}, e^{iB}, e^{iC}$ common from $C_1, C_2, C_3,$
  we get

  $= \startdeterminant\NC 1 \NC -1 \NC -1 \NR\NC -1 \NC 1 \NC -1 \NR\NC -1 \NC -1 \NC
  1\NR\stopdeterminant = -4$, which is purely real.
\item $\Delta = \startdeterminant\NC1 \NC \sin A\cos A \NC \cos^2 A \NR\NC 1 \NC \sin
  B\cos B \NC \cos^2 B \NR\NC 1 \NC \sin C\cos C \NC \cos^2 C \NR\stopdeterminant[C_1
    \rightarrow C_1 + C_3]$

  Performing $R_3\rightarrow R_3 - R_1; R_2\rightarrow R_2 - R_1,$ we get

  $= \sin(A - B)\sin(B - C)\sin(C - A)\geq 0$.

  Now it is trivial to prove the second part.
\item Performing $C_1 \rightarrow C_1 - aC_2; C_2\rightarrow C_2 -aC_3]$

  $\Delta = \startdeterminant\NC 0 \NC 0 \NC 1 \NR\NC \cos nx - a\cos(n + 1)x \NC
  \cos(n + 1)x - a\cos(n + 2)x \NC \cos(n + 2)x \NR\NC \sin nx - a\sin(n + 1)x \NC
  \sin(n + 1)x - a\sin(n + 2)x \NC \sin(n + 2)x\NR\stopdeterminant$

  $= \sin(n+1)x\cos nx - a\sin(n+1)x\cos(n+1)x - a\sin(n+2)x\cos nx \NR\NC
  + a^2\sin(n+2)x\cos(n+1)x -\sin nx\cos(n+1)x +a\sin nx\cos(n + 2)x \NR\NC
  + a\sin(n + 1)x\cos(n + 1)x - a^2\sin(n + 1)x\cos(n + 2)x$

  $= \sin(n + 1 - n)x - a\sin(n + 2 - n)x + a^2\sin(n + 2 - n - 1)x$

  $= \sin x - a\sin 2x + a^2\sin x$

  $= (a^2 -2a\cos x + 1)\sin x$.
\item $\Delta = \startdeterminant\NC2 \NC \cos^2x \NC 4\sin 2x \NR\NC 2 \NC 1 + \cos^2x \NC
  4\sin 2x \NR\NC 1 \NC \cos^2x \NC 1 + 4\sin 2x\NR\stopdeterminant[C_1 \rightarrow C_1 +
    C_2]$

  $= \startdeterminant\NC0 \NC -1 \NC 0 \NR\NC 2 \NC 1 + \cos^2x \NC 4\sin 2x \NR\NC 1 \NC
  \cos^2x \NC 1 + 4\sin 2x\NR\stopdeterminant[R_1 \rightarrow R_1 - R_2]$

  $= 2 - 4\sin 2x$

  The above expression has maximum value for $0 < x < \frac{\pi}{2}$ when $x = \frac{\pi}{4}$.
\item Expanding the determinant we get $\Delta = -1 + 2\cos A\cos B \cos C
  + \cos^2 A + \cos^2 B + \cos^2 C$

  Consider the expression $2(\cos^2 A + \cos^2 B + \cos^2 C)$

  $= 1 + \cos 2A + 1 + \cos 2B + 1 + \cos 2C = 2 + 2\cos(A + B)\cos(A - B) + 2\cos^2 C$

  $= 2 + 2\cos(\pi - C)\cos(A - B) + 2\cos^2C = 2 - 2\cos C[\cos(A - B) - \cos C]$

  $= 2 - 2\cos C[\cos(A - B) 0 \cos (\pi - (A + B))] = 2 - 4\cos A\cos B\cos C$

  Thus, $\Delta = 0$.
\item Since $A, B, C$ are angles of an isosceles triangle, let $A = B$

  Thus, first two columns become equal leading determinant to be zero.
\item $\Delta = \startdeterminant\NC1 \NC \frac{\log y}{\log x} \NC \frac{\log
  z}{\log x} \NR\NC \frac{\log x}{\log y} \NC 1 \NC \frac{\log z}{\log y} \NR\NC
  \frac{\log x}{\log z} \NC \frac{\log y}{\log z} \NC 1\NR\stopdeterminant$

  $= \frac{1}{\log x\log y\log z}\startdeterminant\NC\log x \NC \log y \NC \log
  z \NR\NC \log x \NC \log y \NC \log z \NR\NC \log x \NC \log y \NC \log x\NR\stopdeterminant$

  $= 0$ because all three rows are identical.
\item $\Delta = \startdeterminant\NC a^{2x} + a^{-2x} + 2 \NC a^{2x} + a^{-2x} - 2
  \NC 1 \NR\NC b^{2x} + b^{-2x} + 2 \NC b^{2x} + b^{-2x} - 2 \NC 1 \NR\NC c^{2x} +
  c^{-2x} + 2 \NC c^{2x} + c^{-2x} - 2 \NC 1\NR\stopdeterminant$

  $= \startdeterminant\NC a^{2x} + a^{-2x} \NC a^{2x} + a^{-2x} \NC 1 \NR\NC
  b^{2x} + b^{-2x} \NC b^{2x} + b^{-2x} \NC 1 \NR\NC c^{2x} + c^{-2x} \NC c^{2x} +
  c^{-2x} \NC 1\NR\stopdeterminant[C_1\rightarrow C_1 - 2C_3; C_2\rightarrow C_2 +
    2C_3]$

  $= 0$ because first two columns are identical.
\item Considering first determinant only:

  $\Delta = \startdeterminant\NC115 \NC 114 \NC 103 \NR\NC 108 \NC 106 \NC 111 \NR\NC 113 \NC
  116 \NC 104\NR\stopdeterminant[C_1\leftrightarrow C_2; C_2\leftrightarrow C_3]$

  Performing $R_1\leftrightarrow R_3$

  $\Delta = -\startdeterminant\NC113 \NC 116 \NC 104 \NR\NC 108 \NC 106 \NC 111 \NR\NC
  115 \NC 114 \NC 103\NR\stopdeterminant$

  Thus, given condition is satisfied.
\item $\displaystyle\sum_{n = 1}^N U_n = \startdeterminant\NC\displaystyle\sum_{n = 1}^N n \NC 1 \NC 5
  \NR\NC \displaystyle\sum_{n = 1}^N n^2 \NC 2N + 1 \NC 2N + 1 \NR\NC \displaystyle\sum_{n = 1}^N n^3 \NC
  3N^2 \NC 3N\NR\stopdeterminant$

  $= \startdeterminant\NC\frac{N(N + 1)}{2} \NC 1 \NC 5 \NR\NC \frac{N(N + 1)(2N +
    1)}{6} \NC 2N + 1 \NC 2N + 1 \NR\NC \left\{\frac{N(N + 1)}{2}\right\}^2 \NC 3N^2 \NC
  3N\NR\stopdeterminant$

  Taking $\frac{N(N + 1)}{2}$ common from first column and then
  performing $C_1\rightarrow C_1 - \frac{1}{6}(C_2 + C_3)$

  $= \frac{N(N + 1)}{2}\startdeterminant\NC0 \NC 1 \NC 5 \NR\NC 0 \NC 2N + 1 \NC 2N + 1
  \NR\NC 0 \NC 3N^2 \NC 3N\NR\stopdeterminant$

  Since first column has only $0$ as element, therefore, the sum of determinants
  is zero.
\item $\because A, B, C$ are angles of a triangle, therefore $A +
  B + C = \pi; \sin(A + B + C) = 0; \cos(A + B) = -\cos C$

  $\therefore \Delta = \startdeterminant\NC0 \NC \sin B \NC \cos C \NR\NC -\sin B
  \NC 0 \NC \tan A \NR\NC -\cos C \NC -\tan A \NC 0 \NR\stopdeterminant$

  Changing rows into corresponding columns

  $= \startdeterminant\NC0 \NC -\sin B \NC -\cos C \NR\NC \sin B \NC 0 \NC -\tan A \NR\NC
  \cos C \NC \tan A \NC 0\NR\stopdeterminant$

  Taking $-1$ common from second and third columns, we have

  $= \startdeterminant\NC0 \NC \sin B \NC \cos C \NR\NC \sin B \NC 0 \NC \tan A \NR\NC
  \cos C \NC -\tan A \NC 0\NR\stopdeterminant = -\Delta$

  $\Rightarrow 2\Delta = 0 \Rightarrow \Delta = 0$.
\item Taking $b - a$ common from first and third columns

  $\Delta = (b - a)^2 \startdeterminant\NC b \NC b - c \NC c \NR\NC a \NC a - b \NC b \NR\NC
  c \NC c - a \NC a\NR\stopdeterminant$

  $= (b - a)^2 \startdeterminant\NC b - c \NC b - c \NC c \NR\NC a - b \NC a - b \NC b
  \NR\NC c - a \NC c - a \NC a\NR\stopdeterminant[C_1 \rightarrow C_1 - C_3]$

  Since the first two columns are same; the determinant is zero.
\item We can rewrite it as $\displaystyle\sum_{j = 0}^{n - 1}\Delta_j =
  \startdeterminant\NC\displaystyle\sum_{j = 0}^{n - 1} j \NC n \NC 6 \NR\NC \displaystyle\sum_{j = 0}^{n - 1} j^2
  \NC 2n^2 \NC 4n - 2 \NR\NC \displaystyle\sum_{j = 0}^{n - 1}j^3 \NC 3n^3 \NC 3n^2 - 3n\NR\stopdeterminant$

  $= \startdeterminant\NC\frac{n(n - 1)}{2} \NC n \NC 6 \NR\NC \frac{n(n - 1)(2n -
    1)}{6} \NC 2n^2 \NC 4n - 2 \NR\NC \left\{\frac{n(n - 1)}{2}\right\}^2 \NC 3n^3 \NC
  3n^2 - 3n\NR\stopdeterminant$

  $= \frac{n(n - 1)}{2}\startdeterminant\NC1 \NC n \NC 6 \NR\NC \frac{2n - 1}{3} \NC
  2n^2 \NC 4n - 2 \NR\NC \frac{n(n - 1)}{2} \NC 3n^3 \NC 3n^2 - 3n\NR\stopdeterminant$

  $= \frac{n(n - 1)}{2}\startdeterminant\NC0 \NC n \NC 6 \NR\NC 0 \NC 2n^2 \NC 4n - 2
  \NR\NC 0 \NC 3n^3 \NC 3n^2 -3n\NR\stopdeterminant[C_1\rightarrow C_1 - \frac{C_3}{6}]$

  Since first column is entirely made up of zeros the value of determinant
  is zero, which is a constant as desired.
\item $\displaystyle\sum_{r = 0}^m(2r - 1) = \frac{1}{2}(m + 1)(2m - 1 - 1) = m^2 - 1$

  $\displaystyle\sum_{r = 0}^m({}^nC_r) = 2^m$

  $\displaystyle\sum_{r = 0}^m1 = m + 1$

  Thus, first two rows of determinant become zero leading the desired sum to
  be $0$.
\item $\Delta = \startdeterminant\NC C_r^^x \NC C_{r + 1}^^{x + 1} \NC C_{r +
  2}^^{x + 1} \NR\NC C_r^^y \NC C_{r + 1}^^{y + 1} \NC C_{r + 2}^^{y + 1} \NR\NC C_r^^z \NC
  C_{r + 1}^^{z + 1} \NC C_{r + 2}^^{z + 1}\NR\stopdeterminant[C_3 \rightarrow
    C_3 + C_2; C_2\rightarrow C_2 + C_1]$

  Performing $C_3\rightarrow C_3 + C_2$ we get the determinant on R.H.S.
\item $\displaystyle\sum_{r = 1}^n \Delta_r = \startdeterminant\NC\displaystyle\sum_{r = 1}^nr \NC n + 1 \NC
  1 \NR\NC \displaystyle\sum_{r = 1}^nr^2 \NC 2n - 1 \NC \frac{2n + 1}{3} \NR\NC \displaystyle\sum_{r =
    1}^nr^3 \NC 3n + 2 \NR\NC \frac{n(n + 1)}{2}\NR\stopdeterminant$

  $= \startdeterminant\NC\frac{n(n + 1)}{2} \NC n + 1 \NC 1 \NR\NC \frac{n(n +
    1)(2n + 1)}{6} \NC 2n - 1 \NC \frac{2n + 1}{3} \NR\NC \left\{\frac{n(n +
    1)}{2}\right\}^2 \NC 3n + 2 \NC \frac{n(n + 1)}{2}\NR\stopdeterminant$

  If we take $\frac{n(n + 1)}{2}$ common from first column then first
  and third column become same. Thus, $\sum_{r = 1}^n \Delta_r = 0$.
\item $\displaystyle\sum_{r = 1}^n 2^{r - 1} = 1 + 2 + \ldots + 2^{n - 1} = \frac{2^n -
  1}{2 - 1} = 2^n - 1$

  $\displaystyle\sum_{r = 1}^n 2.3^{r - 1} = 2.\frac{3^n - 1}{3 - 1} = 3^n - 1$

  $\displaystyle\sum_{r = 1}^n 4.5^{r - 1} = 4.\frac{35^n - 1}{5 - 1} = 5^n - 1$

  Thus, we see that first row and third rows are equal leading the sum of
  the determinants to zero.
\item $\Delta = \startdeterminant\NC 2x - 1 \NC 2x - 3 \NC x^2 - 4x + 4 \NR\NC 2x - 3 \NC
  2x - 5 \NC x^2 - 6x + 9 \NR\NC 2x -5 \NC 2x -7 \NC x^2 - 8x + 16\NR\stopdeterminant[C_1
    \rightarrow C_1 - C_1; C_2\rightarrow C_2 - C_3]$

  $= \startdeterminant\NC 2x - 1 \NC 2x - 3 \NC x^2 \NR\NC 2x - 3 \NC 2x - 5 \NC x^2 \NR\NC
  2x - 5 \NC 2x - 7 \NC x^2\NR\stopdeterminant + \startdeterminant\NC2x - 1 \NC 2x - 3
  \NC -4x \NR\NC 2x - 3 \NC 2x - 5 \NC -6x \NR\NC 2x - 5 \NC 2x - 7 \NC -8x\NR\stopdeterminant +
  \startdeterminant\NC2x - 1 \NC 2x - 3 \NC 4 \NR\NC 2x - 3 \NC 2x - 5 \NC 9 \NR\NC 2x - 5 \NC 2x -
    7 \NC 16\NR\stopdeterminant$

  Clearly, if we perform $R_1\rightarrow R_1- R_2; R_2\rightarrow
  R_2 - R_3$ will make $R_1$ and $R_3$ same in the first
  determinant.

  This is also true for second determinant.

  $= \startdeterminant\NC2 \NC 2 \NC -5 \NR\NC 2 \NC 2 \NC -7 \NR\NC 2x - 5 \NC 2x -
  7 \NC 16\NR\stopdeterminant$

  Clearly, the determinant is independent of $x$.
\item $\Delta = \startdeterminant\NC2 \NC 1 + i \NC 3 \NR\NC 1 - i \NC
  0 \NC 2 + i \NR\NC 3 \NC 2 - i \NC 1\NR\stopdeterminant$

  Taking complex conjugate and exchanging rows into corresponding columns

  $\overline{\Delta} = \startdeterminant\NC2 \NC 1 + i \NC 3 \NR\NC 1 - i \NC
  0 \NC 2 + i \NR\NC 3 \NC 2 - i \NC 1\NR\stopdeterminant = \Delta$

  Since $\overline{\Delta} = \Delta,$ the determinant is purely real.
\item $\Delta = \startdeterminant\NC x - 3 \NC 2x \NC 2 \NR\NC 3x + 2 \NC x \NC 1 \NR\NC 5x + 1
  \NC 5x \NC 5\NR\stopdeterminant + \startdeterminant\NC x - 3 \NC 1 \NC 2 \NR\NC 3x + 2 \NC 2 \NC 1 \NR\NC
  5x + 1 \NC 4 \NC 5\NR\stopdeterminant$

  If we take out $x$ common from second column of first determinant
  then second and third columns are same, making it zero. Now expandng
  second determinant

  $= \startdeterminant\NC x \NC 1 \NC 2 \NR\NC 3x \NC 2 \NC 1 \NR\NC 5x \NC 4 \NC 5
  \NR\stopdeterminant +$ a determinant of constants(say $k$)

  $= x \startdeterminant\NC0 \NC 1 \NC 2 \NR\NC 1 \NC 2 \NC 1 \NR\NC 1 \NC 4 \NC 5\NR\stopdeterminant
  [C_1\rightarrow C_1 - C_2] + k$

  $= x \startdeterminant\NC0 \NC 1 \NC 2 \NR\NC 1 \NC 2 \NC 1 \NR\NC 0 \NC 2 \NC 4\NR\stopdeterminant
  [R_3\rightarrow R_3 - R_2] + k$

  $= x\startdeterminant\NC0 \NC 1 \NC 2 \NR\NC 1 \NC 2 \NC 1 \NR\NC 0 \NC 0 \NC 0\NR\stopdeterminant
  [C_3\rightarrow C_3 - 2C_1] + k$

  $= k$.
\item $\Delta = \startdeterminant\NC a^n - x \NC a^{n+ 1} - x \NC a^{n + 2} - x \NR\NC
  a^{n + 3} - a^n \NC a^{n + 4} - a^{n + 1} \NC a^{n + 5} - a^{n + 2} \NR\NC a^{n + 6} - a^{n + 3} \NC
  a^{n + 7} - a^{n + 4} \NC a^{n + 8} - a^{n + 5}\NR\stopdeterminant[R_2\rightarrow R_2 - R_1;
    R_3\rightarrow R_3 - R_2]$

  $=a^{n(n + 3)}\startdeterminant\NC a^n - x \NC a^{n + 1} - x \NC a^{n + 2} - x \NR\NC a^{3} -
  1 \NC a^4 - a \NC a^5 - a^2 \NR\NC a^{3} - 1 \NC a^4 - a \NC a^5 - a^2\NR\stopdeterminant = 0$

  Since second and third rows are same, the edterminant is zero.
\item $\Delta = \displaystyle\sum_{r = 2}^n (-2)^r \startdeterminant\NC C_r^^{n- 2} \NC C_{r - 1}^^{n - 2}
  \NC C_r^^{n - 2} \NR\NC 0 \NC 1 \NC 1 \NR\NC 0 \NC -1 \NC 9\NR\stopdeterminant[C_1\rightarrow C_1 + 2C_2 +
    C_3]$

  $= \sum_{r = 0}^n (-2)^rC_r^^n - (C_0^^n -2C_1^^n)$

  $= 2n - 1 + (-1)^n$.
\item Performing $R_1\rightarrow aR_1, R_2\rightarrow bR_2, R_3\rightarrow
  cR_3$ and then taking out $abc$ out from first two columns,

  $\Delta = abc\startdeterminant\NC bc \NC 1 \NC a(b + c) \NR\NC ca \NC 1 \NC b(c + a)
  \NR\NC ab \NC 1 \NC c(a + b)\NR\stopdeterminant$

  Performing $C_3\rightarrow C_3 + C_1$ and then taking $ab + bc + ca$ out

  $= abc(ab + bc + ca)\startdeterminant\NC bc \NC 1 \NC 1 \NR\NC ca \NC 1 \NC 1 \NR\NC ab \NC
  1 \NC 1\NR\stopdeterminant$

  Since last two columns are same, the determinant is zero.
\item Putting $b = c$, we see that the determinant reduces to $0$. Similarly, $c = a$ or $a = b$ or $a = d$
  or $b = d$ or $c = d$ also reduces the determinant to zero.

  We also see that the degree of polynomial of the determinant is six, and thus,

  $\startdeterminant\NC  b + c - a - d \NC bc - ad \NC bc(a + d) - ad(b + d)\NR\NC c + a - b - d \NC ca - bd
  \NC ca(b + d) - bd(c + a)\NR\NC  a + b - c - d \NC ab - cd \NC ab(c + d) - cd(a + b)\NR\stopdeterminant =
  k(b - c)(c - a)(a - b)(a - d)(b - d)(c - d)$

  Putting $a = 0, b = 1, c = 2, d = 3$ we evaluate $k = -2$, and thus, we have proven the desired equality.
\item Putting $b = c$, we see that the determinant reduces to zero. Similarly, $c = a$ or $a = b$ also
  reduced the determinant to zero. Also, putting $a = -b - c$ or $ab = -bc - ca$ makes the determinant zero.

  We also see that the degree of polynomial of the determinant is six, and thus,

  $\startdeterminant\NC  bc - a^2 \NC ca - b^2 \NC ab - c^2\NR\NC ca + ab - bc \NC bc + ab - ca \NC bc + ca
  - ab\NR\NC (a + b)(a + c) \NC (b + c)(b + a) \NC (c + a)(c + b)\NR\stopdeterminant= k(b - c)(c - a)(a -
  b)(a + b + c)(ab + bc + ca)$

  Putting $a = 0, b = 1, c = 2$ we evaluate $k = 3$.
\item Putting $l = m$ we see that the determinant reduces to zero. Similarly $l = n, n = p, m = n, m = p, n
  = p$ also reduce the determinant to zero.

  We also see that the degree of polynomial of the determinant is six, and thus,

  $\startdeterminant\NC 1 \NC (m + n - l - p)^2 \NC (m + n - l - p)^4\NR\NC 1 \NC (n + l - m - p)^2 \NC (n +
  l - m - p)^4\NR\NC 1 \NC (l + m - n - p)^2 \NC (l + m - n - p)^4\NR\stopdeterminant = k(l - m)(l - n)(l -
  p)(m - n)(m -p)(n - p)$

  Putting $l = 0, m = 1, n = 2, p = 3$ we find that $k = 64$, and hence we prove the required equality.
\item $\frac{d}{dt}\startdeterminant\NC  u_1 \NC v_1 \NC w_1\NR\NC u_2 \NC v_2 \NC w_2\NR\NC u_3 \NC v_3 \NC
  w_3\NR\stopdeterminant = \startdeterminant\NC u_2\NC v_2\NC w_2\NR\NC u_2\NC v_2\NC w_2\NR\NC u_3\NC
  v_3\NC w_3\NR\stopdeterminant \\ + \startdeterminant\NC u_1\NC v_1\NC w_1\NR\NC u_3\NC v_3\NC w_3\NR\NC
  u_3\NC v_3\NC w_3\NR\stopdeterminant + \startdeterminant\NC u_1\NC v_1\NC w_1\NR\NC u_2\NC v_2\NC
  w_2\NR\NC u_4\NC v_4\NC w_4\NR\stopdeterminant$

  First two determinants are zero because two rows are identical. Hence,

  $\frac{d}{dt}\startdeterminant\NC  u_1 \NC v_1 \NC w_1\NR\NC u_2 \NC v_2 \NC w_2\NR\NC u_3 \NC v_3 \NC
  w_3\NR\stopdeterminant = \startdeterminant\NC  u_1 \NC v_1 \NC w_1\NR\NC u_2\NC v_2 \NC w_2\NR\NC u_4 \NC
  v_4 \NC w_4\NR\stopdeterminant$.
\item Given $Y = sX \Rightarrow Y_1 = s_1X + sX_1 \Rightarrow Y_2 = s_2X + s_1X_1 + s_1X_1 + sX_2 = sX_2 +
  2s_1X_1 + s_2X$, and similarly $Z = tX \Rightarrow Z_1 = t_1X + tX_1 \Rightarrow Z_2 = tX_2 + 2t_1X_1 +
  t_2X$.

  L.H.S.\ $= \startdeterminant\NC  X \NC Y \NC Z\NR\NC X_1\NC Y_1 \NC Z_1\NR\NC X_2 \NC Y_2 \NC
  Z_2\NR\stopdeterminant = \startdeterminant\NC X\NC sX\NC tX\NR\NC X_1\NC s_1X + sX_1\NC t_1X + tX_1\NR\NC
  X_2\NC s_2X + 2s_1X_1 + sX_2\NC t_2X + 2t_1X_1 + tX_2\NR\stopdeterminant$

  $=X\startdeterminant\NC 1\NC s\NC t\NR\NC X_1\NC s_1X + sX_1\NC t_1X + tX_1\NR\NC
  X_2\NC s_2X + 2s_1X_1 + sX_2\NC t_2X + 2t_1X_1 + tX_2\NR\stopdeterminant$

  $= X\startdeterminant\NC 1\NC s\NC t\NR\NC 0\NC s_1X\NC t_1X\NR\NC 0\NC s_2X + 2s_1X_1\NC t_2X +
  2t_1x_1\NR\stopdeterminant[R_2\rightarrow R_2 - R_1X_1; R_3\rightarrow R_3 - R_1X_2]$

  $= X^3\startdeterminant\NC  s_1 \NC t_1\NR\NC s_2 \NC t_2\NR\stopdeterminant$.
\item Let $F(x) = \startdeterminant\NC  f(x) \NC g(x) \NC h(x)\NR\NC f(\alpha) \NC
  g(\alpha) \NC h(\alpha)\NR\NC  f(\beta) \NC g(\beta) \NC h(\beta)\NR\stopdeterminant$

  Clearly, $F(\alpha) = 0$ because first two rows become equal. $F'(x) = \startdeterminant\NC  f'(x) \NC g'(x)
  \NC h'(x)\NR\NC f(\alpha) \NC
  g(\alpha) \NC h(\alpha)\NR\NC  f(\beta) \NC g(\beta) \NC h(\beta)\NR\stopdeterminant$

  If $F'(x) = 0$ then $F'(\alpha) = 0$ making $\alpha$ a repeated root.

  $F(\beta) = 0$ because first and last rows are identical. Thus, $(x - \alpha)^2(x - \beta)$ is a factor of
  $F(x)$ so the required condition is $\startdeterminant\NC  f'(x) \NC g'(x)
  \NC h'(x)\NR\NC f(\alpha) \NC
  g(\alpha) \NC h(\alpha)\NR\NC  f(\beta) \NC g(\beta) \NC h(\beta)\NR\stopdeterminant = 0$
\item We see that $\frac{d\Delta}{dx} = 0$ and hence it is a constant, independent of $x$.
\item Applying $C_1\rightarrow C_1 - 2\sin xC_3; C_2\rightarrow C_2 + 2\cos xC_3$

  $\Delta = \startdeterminant\NC 2\NC 0\NC -\sin x\NR\NC 0\NC 2\NC \cos x\NR\NC \sin x\NC -\cos x\NC
  0\NR\stopdeterminant = 2\sin^2x + 2\cos^2x = 2$.

  $\Rightarrow f'(x) = 0$.

  $\therefore \displaystyle\int_0^{\frac{\pi}{2}}[f(x) + f'(x)]dx = \int_0^{\frac{\pi}{2}}2dx = \pi$.
\item L.H.S.\ $=\startdeterminant\NC a_1\NC b_1\NC 0\NR\NC a_2\NC b_2\NC 0\NR\NC a_3\NC b_3\NC
  0\NR\stopdeterminant\startdeterminant\NC \alpha_1\NC \beta_1\NC 0\NR\NC \alpha_2\NC \beta_2\NC 0\NR\NC
  \alpha_3\NC \beta_3\NC 0\NR\stopdeterminant = 0$.
\item Let $\vec{V_r} = l_r\vec{i} + m_r\vec{j} + n_r\vec{k}$. Then for $r = 1$ let $\vec{V_1} = \pm\vec{i}$,
  for $r = 2, \vec{V_2} = \pm\vec{j},$ and $r = 3, \vec{V_3}= \pm\vec{k}$.

  Thus, we get determinant as $\startdeterminant\NC1\NC 0\NC 0\NR\NC 0 \NC 1\NC 0\NR\NC 0\NC 0\NC
  1\NR\stopdeterminant$ or $\startdeterminant\NC-1\NC 0\NC 0\NR\NC 0 \NC -1\NC 0\NR\NC 0\NC 0\NC
  -1\NR\stopdeterminant$

  Hence, $\Delta = \pm 1$.
\item Let $A = \startbmatrix\NC a_1\NC b_1\NC c_1\NR\NC a_2\NC b_2\NC c_2\NR\NC a_3\NC b_3\NC
  c_3\NR\stopbmatrix$

  Cofactor matrix $C = \startbmatrix\NC A_1\NC B_1\NC C_1\NR\NC A_2\NC B_2\NC C_2\NR\NC A_3\NC B_3\NC
  C_3\NR\stopbmatrix$

  $\adj(A) = C^{T}$ but $A(\adj(A)) = |A|I \Rightarrow |A||\adj(A)| = |A|^3 \Rightarrow |\adj(A)| = |A|^2$

  $\therefore |C| = |A|^2 \Rightarrow \startbmatrix\NC A_1\NC B_1\NC C_1\NR\NC A_2\NC B_2\NC C_2\NR\NC
  A_3\NC B_3\NC C_3\NR\stopbmatrix$

  We can proceed similarly for second level of cofactors whose determinant will be $\Delta^4$.

  Thus, $\startdeterminant\NC  A_1 \NC B_1 \NC C_1\NR\NC A_2 \NC B_2 \NC C_2\NR\NC A_3 \NC B_3 \NC
  C_3\NR\stopdeterminant\startdeterminant\NC \alpha_1 \NC \beta_1 \NC \gamma_1\NR\NC \alpha_2 \NC \beta_2
  \NC \gamma_2\NR\NC \alpha_3 \NC \beta_3 \NC \gamma_3\NR\stopdeterminant = \Delta^6$.
\item Using Cramer's rule, $\Delta = \startdeterminant\NC 1\NC 2\NC 3\NR\NC 2\NC 4\NC 1\NR\NC 3\NC 2\NC
  9\NR\stopdeterminant = 34 - 20 - 24 = -20$

  $\Delta_x = \startdeterminant\NC 17\NC 4\NC 1\NR\NC 2\NC 2\NC 9\NR\NC 1\NC 6\NC 3\NR\stopdeterminant = 204
  - 302 + 78 = -20$

  $\Delta_y = \startdeterminant\NC 1\NC 6\NC 3\NR\NC 2\NC 17\NC 1\NR\NC 3\NC 2\NC 9\NR\stopdeterminant = 151
  - 90 - 141 = -80$

  $\Delta_z = \startdeterminant\NC1\NC2\NC6\NR\NC 2\NC 4\NC 17\NR\NC 3\NC 2\NC 2\NR\stopdeterminant = -26 +
  94 - 48 = 20$

  $\Rightarrow x = \frac{\Delta_x}{\Delta} = 1, y = \frac{\Delta_y}{\Delta} = 4, z = \frac{\Delta_z}{\Delta}
  = -1$.
\item $\Delta = \startdeterminant\NC a\NC b\NC c\NR\NC a^2\NC b^2\NC c^2\NR\NC a^3\NC b^3\NC
  c^3\NR\stopdeterminant = abc(a - b)(b - c)(c - a)$.

  Similarly, $\Delta_x = dbc(d - b)(b - c)(c - d)$, $\Delta_y = acd(a - d)(d - c)(c - a)$, and $\Delta_z =
  abd(a - b)(b - d)(d - a)$

  $\Rightarrow x = \frac{d(d - b)(c - d)}{a(a - b)(c - a)}, y = \frac{d(a - d)(d - c)}{b(a - b)(b - c)}, z =
  \frac{d(b - d)(d - a)}{c(b - c)(c - a)}$.

  If only two of $a, b, c$ are zero the given system of equations has no solution. If $a = b = c$ and any of
  $a, b, c$ is zero then the system of equations has infinite number of solutions.
\item Given $f(1) = 0 \Rightarrow a + b + c = 0; f(2) = -2 \Rightarrow 4a + 2b + c = -2; f(3) = -6
  \Rightarrow 9a + 3b + c = -6$

  $\Delta = \startdeterminant\NC 1\NC 1\NC 1\NR\NC 4\NC 2\NC 1\NR\NC 9\NC 3\NC 1\NR\stopdeterminant = -1 + 5
  - 6 = -2$

  $\Delta_a = \startdeterminant\NC 0\NC 1\NC 1\NR\NC -2\NC 2\NC 1\NR\NC -6\NC 3\NC 1\NR\stopdeterminant = -4
  + 6 = 2$

  $\Delta_b = \startdeterminant\NC 1\NC 0\NC 1\NR\NC 4\NC -2\NC 1\NR\NC 9\NC -6\NC 1\NR\stopdeterminant = 4
  - 6 = -2$

  $\Delta_c = \startdeterminant\NC 1\NC 1\NC 0\NR\NC 4\NC 2\NC -2\NR\NC 9\NC 3\NC -6\NR\stopdeterminant = 0$

  $\Rightarrow a = -1, b = 1, c = 0$.
\item $f(0) = 6 \Rightarrow c = 6, f(2) = 11\Rightarrow 4a + 2b + c = 11, f(-3) = 6 \Rightarrow 9a - 3b + c
  = 6$.

  $\Delta = \startdeterminant\NC 0\NC 0\NC 1\NR 4\NC 2\NC 1\NR\NC 9\NC -3\NC 1\NR\stopdeterminant = -30$

  $\Delta_a = \startdeterminant\NC 6\NC 0\NC 1\NR\NC 11\NC 2\NC 1\NR\NC 0\NC 6\NC 1\NR\stopdeterminant =
  -15$

  $\Delta_b =\startdeterminant\NC 0\NC 6\NC 1\NR\NC 4\NC 11\NC 1\NR\NC 9\NC 6\NC 1\NR\stopdeterminant = -45$

  $\Rightarrow a = \frac{1}{2}, b = \frac{3}{2}, c = 6\Rightarrow f(1) = a + b + c = 8$.
\item $\Delta = \startdeterminant\NC -a\NC b + c\NC b + c\NR\NC c + a\NC -b\NC c + a\NR\NC a + b\NC a + b\NC
  -c\NR\stopdeterminant = \startdeterminant\NC a + b + c\NC a + b + c\NC a + b + c\NR\NC c + a\NC -b\NC c +
  a\NR\NC a + b\NC a + b\NC -c\NR\stopdeterminant[R_1\rightarrow R_1 + R_2 + R_3]$

  $= (a + b + c)\startdeterminant\NC 0\NC 1\NC 1\NR\NC 0\NC -b\NC c + a\NR\NC a + b + c\NC a + b\NC
  -c\NR\stopdeterminant[C_1\rightarrow C_1 - C_3]$

  $= (a + b + c)^3$

  $\Delta_x = \startdeterminant\NC b - c\NC b + c\NC b + c\NR\NC c - a\NC -b\NC c + a\NR\NC a - b\NC a +
  b\NC -a\NR\stopdeterminant = \startdeterminant\NC b - c\NC 0\NC b + c\NR\NC c - a\NC -(a + b + c)\NC c +
  a\NR\NC a - b\NC a + b + c\NC -c\NR\stopdeterminant[C_2\rightarrow C_2 - C_3]$

  $=(a + b + c)\startdeterminant\NC b - c\NC 0\NC b + c\NR\NC c - b\NC 0\NC a\NR\NC a - b\NC 1\NC -
  c\NR\stopdeterminant[R_2\rightarrow R_2 + R_3]$

  $= (c - b)(a + b + c)^2$

  $\Rightarrow x = \frac{c - b}{a + b + c}$. Since the given system of equations is cyclic, therefore, $y =
  \frac{a - c}{a + b + c}, z = \frac{b - a}{a + b + c}$.
\item $\Delta = \startdeterminant\NC 7\NC -7\NC 5\NR\NC 3\NC 1\NC 5\NR\NC 2\NC 3\NC 5\NR\stopdeterminant =
  -70 + 35 + 35 = 0$

  $\Delta_x = \startdeterminant\NC 3\NC -7\NC 5\NR\NC 7\NC 1\NC 5\NR\NC 5\NC 3\NC 5\NR\stopdeterminant =
  -30 -70 + 80 = -20$

  Therefore, the system of equations is inconsistent and has no solution.
\item For system of equations to be consistent $\Delta = \startdeterminant\NC 1\NC 1\NC 3\NR\NC 1 + k\NC 2 +
  k\NC 8\NR\NC 1\NC -(1 + k)\NC -(2 + k)\NR\stopdeterminant = 0$

  $\Rightarrow 3k^2 + 2k - 5 = 0 \Rightarrow k = 1, -\frac{5}{3}$.
\item $\Delta = \startdeterminant\NC(k + 1)^3\NC(k + 2)^2\NC(k + 3)^3\NR\NC k + 1\NC k + 2\NC k + 3\NR\NC 1\NC
  1\NC 1\NR\stopdeterminant$

  $\Rightarrow k = -2$.
\stopitemize
