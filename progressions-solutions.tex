% -*- mode: context; -*-
\chapter{Progressions}
\startitemize[n, 1*broad]
\item Given $t_n = 2n^2 + 1 \Rightarrow t_{n - 1} = 2(n - 1)^2 + 1$

  $\therefore d = t_n - t_{n - 1} = 4n - 2$, which is not constant. Hence, the sequence is not in A.P.
\item Given, $t_1 = 1, t_2 = 2$ and $t_{n+2} = t_n + t_{n + 1}$

  $\therefore t_3 = t_1 + t_2 = 3, t_4 = t_2 + t_3 = 5, t_5 = t_3 + t_4 = 8$.
\item Given $t_n = 3n + 5 \Rightarrow t_1 = 3\times1 + 5 = 8, t_2 = 3\times2 + 5 = 11, t_3 = 3\times3 + 5 = 14$. So the seuquence
  is $8, 11, 14, \ldots, 3n + 5$.
\item Given $t_n = 2n^2 + 3 \Rightarrow t_1 = 2\times1^2 + 3 = 5, t_2 = 2\times2^2 + 3 = 11, t_3 = 2\times3^2 + 5 = 23$. So the
  sequence is $5, 11, 23, \ldots, 2n^2 + 3$.
\item Given, $t_n = \frac{3n}{2n + 4}\Rightarrow t_1 = \frac{3\times1}{2\times1 + 4} = \frac{3}{6} = \frac{1}{2}, t_2 =
  \frac{3\times2}{2\times2 + 4} = \frac{6}{8} = \frac{3}{4}, t_3 = \frac{3\times3}{2\times3 + 4} = \frac{9}{10}$. So the
  sequence is $\frac{1}{2}, \frac{3}{4}, \frac{9}{10}, \cdots, \frac{3n}{2n + 4}$.
\item Given, $t_1 = 2, t_{n + 1} = \frac{2t_n + 1}{t_n + 3} \Rightarrow t_2 = \frac{2t_1 + 1}{t_1 + 3} = \frac{2\times1 + 1}{1 + 3}
  = \frac{3}{4}, t_3 = \frac{2t_2 + 1}{t_2 + 3} = \frac{2\times\tfrac{3}{4} + 1}{\tfrac{3}{4} + 3} = \frac{10}{15} = \frac{2}{3}$.
  So the sequence is $2, \frac{3}{4}, \frac{2}{3}, \cdots$.
\item Given, $t_n = 4n^2 + 1 \Rightarrow t_{n - 1} 4(n - 1)^2 + 1$

  $\therefore d = t_n - t_{n - 1} = 8n - 4$, which is not constant. Hence the sequence is not in A.P.
\item Given $t_n = 2an + b \Rightarrow t_{n - 1} = 2a(n - 1) + b$

  $\therefore d = t_n - t_{n - 1} = 2a$. which is a constant. Hence the sequence will be an A.P.
\item Given, $t_1 = 3, t_2 = 3, t_3 = 6$ and $t_{n + 2} = t_n + t_{n + 1}$

  $\therefore t_4 = t_2 + t_3 = 3 + 6 = 9$ and $t_5 = t_3 + t_4 = 6 + 9 = 15$.
\item $t_1 = 1 = a + b + c, t_2 = 5 = 4a + 2b + c$ and $t_3 = 11 = 9a + 3b + c$

  $\therefore t_2 - t_1 = 4 = 3a + b$ and $t_3 - t_2 = 6 = 5a + b$

  $\Rightarrow 2a = 2 \Rightarrow a = 1 \Rightarrow b = 1 \Rightarrow c = -1$

  $\Rightarrow t_{10} = 1\times10^2 + 1\times10 - 1 = 109$.
\item Difference between successive terms i.e. commond difference, $d = 12 - 9 = 15 - 12 = 18 - 15 = 3$ which is a constant, hence,
  the given sequence is an A.P.

  Here first term $t_1 = 9$ and $d = 3 \therefore t_{16} = 9 + (16 - 1)3 = 54$ and $t_n = 9 + (n - 1)3 = 3(n + 2)$.
\item $t_1 = \log a, t_2 = \log(ab) = \log a + \log b, t_3 = \log(ab^2) = \log a + 2\log b$

  $t_2 - t_1 = t_3 - t_2 = \log b$. Clearly, $t_1 = \log a, d = \log b$ which is constant so the sequence is an A.P.

  $\therefore t_n = \log a + (n - 1)\log b = \log(ab^{n - 1})$.
\item Given, $t_n = 5 - 6n \Rightarrow t_1 = 5 - 6 = -1$

  $S_n = \frac{n}{2}[t_1 + t_n] = n(2 - 3n)$.
\item $d = 7 - 3 = 11 -7 = 4, t_n = 407 = 3 + (n - 1)d \Rightarrow n = \frac{404}{4} + 1 = 102$.
\item Since $a, b, c, d, e$ are in A.P. $\therefore a + e = b + d = 2c = k$(say)

  $\therefore a - 4b + 6c - 4d + e = (a + e) - 4(b + d) + 3.2c = k - 4k + 3k = 0$.
\item Let $a$ be the first term and $d$ be the common difference of the given A.P.

  Given, $5t_5 = 8t_8 \Rightarrow 5a + 20d = 8a + 56d \Rightarrow 3a = -36d \Rightarrow a = -12d$

  $\Rightarrow t_{13} = a + 12d = 0$.
\item Let $n$th term be the smallest positive number. From the sequence we obtain that $t_1 = 25$ and $d = -2\frac{1}{4} = -\frac{9}{4}$.

  Then $t_n > 0 \Rightarrow 25 - (n - 1)\frac{9}{4} > 0\Rightarrow n < \frac{25\times4}{9} + 1 \Rightarrow n = 12$.
\item The given pay scale represents an A.P. with $t_1 = 700, d = 40$ and $t_n = 1500$.

  $\therefore t_n = t_1 + (n - 1)d \Rightarrow n = \frac{t_n - t_1}{d} + 1 = \frac{1500 - 700}{40} + 1 = 21$.

  Thus, the person will reach maximum payment in $21$ years.
\item Let $a$ be the first term and $d$ be the common difference of the A.P. According to the question,

  $t_7 = a + 6d = 34$ and $t_{13} = a + 12d = 64$

  Subtracting $6d = 30 \Rightarrow d = 5 \Rightarrow a = 4$. So the A.P. is $4, 9, 14, \ldots$.
\item If $55$ is the $n$th term then $n$ will have to be an integer. From the given sequence $a = 1, d = 3 - 1 = 5 - 3 = 2$.

  $55 = 1 + (n - 1)2 \Rightarrow n = 28$, which is an integer and hence, $55$ will be $28$th term of the A.P.
\item From the given sequence $a = 2000, d = 1995 - 2000 = 1990 - 1995 = -5$.

  Let $n$th term be first negative term, then, $a + (n - 1)d < 0 \Rightarrow 2000 -(n - 1)5 < 0$

  \Rightarrow $n > 401 \Rightarrow n = 402 \Rightarrow t_{402} = 2000 - (402 - 1)5 = -5$.
\item Common different of the sequence $2, 4, 6, 8, \ldots$ is $2$ and common difference of the seqquence $3, 6, 9, \ldots$ is $3$.

  Thus, common terms will have a common different which is L.C.M. of these two commond differences i.e. $6$.

  Last term of first sequence is $200$ and last term of second sequence is $240$. Clearly, last identical(common) number will be
  less than $200$. We also observe that $6$ is the first identical term. Let there be $n$ such terms. Then

  $6 + (n - 1)6 \leq 200 \Rightarrow n\leq \frac{194}{6} + 1 \Rightarrow n = 33$. Thus there will be $33$ identical terms in the two
  given A.P.
\item Clearly the first number of three digits divisible by $5$ is $100$; while the last such number is $995$. Since these numbers
  are all divisible by $5$ they will form an A.P. with common difference $5$.

  Clearly, $t_1 = 100, t_n = 995, d = 5$ and we have to find $n$.

  $t_n = 995 = 100 + (n - 1)5\Rightarrow n = 180$.
\item Given sequence is $4, 9, 14, \ldots$. So $a = 4, d = 9 - 4 = 14 - 9 = 5$. Let $105$ be $n$th term of this A.P. then $n$ has
  to be an integer for this assumption to be true.

  $105 = 4 + (n - 1)5 \Rightarrow n = \frac{106}{5}$ which is not an integer and therefore $105$ is not a term in the given A.P.
\item This problem is same as problem $21$ and has been left as an exercise.
\item This problem is same as problem $22$ and has been left as an exercise.
\item Let $a$ be the first term and $d$ be the common difference of the A.P. Given,

  $mt_m = nt_n \Rightarrow ma + (m - 1)md = na + (n - 1)nd \Rightarrow (m - n)a = (n^2 - n - m^2 + m)d$

  $\Rightarrow a = -(m + n - 1)d\,\therefore t_{m + n} = a + (m + n - 1)d = 0$.
\item Let $x$ be the first term and $y$ be the common difference of the A.P. Then,

  $a = x + (p - 1)y, b = x + (q - 1)y, c = x + (r - 1)y$

  We have to prove that $a(q - r) + b(r - p) + c(p - q) = 0$.

  Substituting the values of $a, b$ and $c$ in the above equation

  L.H.S. $= [x + (p - 1)y](q - r) + [x + (q - 1)y](r - p) + [x + (r - 1)y](p - q)$

  $= x(q - r + r - p + p - q) + y[(p - 1)(q - r) + (q - 1)(r - p) + (r - 1)(p - q)]$

  $= 0 =$ R.H.S.
\item First number after $100$ which is divisible by $7$ is $105$. The last number divisible by $7$ before $1000$ is $994$.

  Let $n$ be the numbers divisible by $7$ between $100$ and $1000$. Then $994 = 105 + (n - 1)7$

  $\Rightarrow n = 128$. Then no. of numbers not divisible by $7$ is $1000 - 100 - 128 = 772$.
\item Let $x$ be the first term and $y$ be the common difference of the A.P. Then,

  $a = x + (p - 1)y, b = x + (q - 1)y, c = x + (r - 1)y$

  We have to prove that $(a - b)r + (b - c)p + (c - a)q = 0$

  Substituting the values of $a, b$ and $c$ in the above equation

  L.H.S. $= (p - q)yr + (q - r)yp + (r - p)yq = 0 =$ R.H.S.
\item Let the numbers in A.P. be $a - d, a$ and $a + d$. Given their sum is $27$ and sum of squares is $293$.

  $\therefore a - d + a + a + d = 27 \Rightarrow a = 9$

  $\therefore (a - d)^2 + a^2 + (a + d)^2 = 293 \Rightarrow 3a^2 + 2d^2 = 293 \Rightarrow 3\times81 + 2d^2 = 293$

  $\Rightarrow 2d^2 = 50 \Rightarrow d = \pm5$

  So the numbers are $4, 9, 14$ or $14, 9, 4$.
\item Let the numbers in A.P. be $a - 3d, a - d, a + d, a + 3d$. Given their sum is $24$ and product is $945$.

  $\therefore a - 3d + a - d + a + d + a + 3d = 24 \Rightarrow 4a = 24 \Rightarrow a = 6$

  Also, $(a - 3d)(a - d)(a + d)(a + 3d) = 945 \Rightarrow (a^2 - 9d^2)(a^2 - d^2) = 945$

  $\Rightarrow a^4 - 10a^2d^2 + 9d^4 = 945 \Rightarrow 9d^4 - 360d^2 + 1296 - 945 = 0$

  $\Rightarrow 9d^4 - 360d^2 + 351 = 0 \Rightarrow d^4 - 40d^2 + 39 = 0$

  $\Rightarrow (d^2 - 1)(d^2 - 39) = 0$. Since the numbers are integers $\Rightarrow d^2 \neq 39$.

  $\Rightarrow d = \pm 1$. So the numbers are $3, 5, 7, 9$ or $9, 7, 5, 3$.
\item Let $a$ be the first term and $d$ be the common ratio of the A.P. Given,

  $t_p = a + (p - 1)d = q$ and $t_q = a + (q - 1)d = p$

  $\Rightarrow (p - q)d = q - p \Rightarrow d = -1 \Rightarrow a = p + q - 1$

  $\Rightarrow t_{p + q} = a + (p + q - 1)d = p + q - 1 - (p + q - 1) = 0$.
\item Let $a$ be the first term and $d$ be the common ratio of the A.P.

  $\Rightarrow t_m = a + (m - 1)d, t_{2n + m} = a + (2n + m - 1)d$

  $\Rightarrow t_m + t_{2n + m} = 2a + (2m + 2n - 2)d = 2[a + (m + n - 1)d] = 2t_{m + n}$
\item Let the three numbers be $a - d, a, a + d$. Given that their sum is $15$ and sum of their square is $83$.

  $\Rightarrow a - d + a + a + d = 15 \Rightarrow 3a = 15 \Rightarrow a = 5$

  $\Rightarrow (a - d)^2 + a^2 + (a + d)^2 = 83 \Rightarrow 3a^2 + 2d^2 = 83 \Rightarrow 3\times5^2 + 2d^2 = 83^2$

  $\Rightarrow d = \pm2$. So the numbers are $3, 5, 7$ or $7, 5, 3$.
\item This problem is similar to previous problem and has been left as an exercise.
\item Let the three numbers be $a - d, a, a + d$. Given their sum as $12$ and sum of cubes as $408$.

  $\therefore a - d + a + a + d = 12 \Rightarrow 3a = 12 \Rightarrow a = 4$

  $\therefore (a - d)^3 + a^3 + (a + d)^3 = 3a^3 + 6ad^2 = 408 \Rightarrow 24d^2 = 216 \Rightarrow d = \pm 3$

  Hence, the numbers are $1, 4, 7$ or $7, 4, 1$.
\item Let the numbers in A.P. be $a - 3d, a - d, a + d, a + 3d$. Given their sum is $24$ and product of first and
  fourth to product of second and third is $2:3$.

  $\therefore a - 3d + a - d + a + d + a + 3d = 20 \Rightarrow 4a = 20 \Rightarrow a = 5$

  $\therefore \frac{(a - 3d)(a + 3d)}{(a - d)(a + d)} = \frac{2}{3}$

  $\Rightarrow 3a^2 - 27d^2 = 2a^2 - 2d^2 \Rightarrow a^2 = 25d^2 \Rightarrow d = \pm1$.

  Therefore numbers are $2, 4, 6, 8$ or $8, 6, 4, 2$.
\item Let the three numbers be $a - d, a, a + d$. Given their sum is $-3$ and product is $8$.

  $\therefore a - d + a + a + d = -3 \Rightarrow 3a = -3 \Rightarrow a = -1$

  $\therefore (a - d).a.(a + d) = 8 \Rightarrow a^2 - d^2 = -8 \Rightarrow d = \pm3$

  Hence the numbers are $-4, -1, 2$ or $2, -1, -4$.
\item This problem is similar to problem 38 and has been left as an exercise.
\item Given $\frac{b + c - a}{a}, \frac{c + a - b}{b}, \frac{a + b - c}{c}$ are in A.P.

  Adding $2$ to each term will give us another A.P. [refer properties of A.P.]

  $\therefore \frac{a + b + c}{a}, \frac{a + b + c}{b}, \frac{a + b + c}{c}$ will be in A.P.

  Dividing each term with $a + b + c$ will yield another A.P.

  $\therefore \frac{1}{a}, \frac{1}{b}, \frac{1}{c}$ will be in A.P.
\item Given $a, b, c$ are in A.P.

  Dividing each term by $abc$ will yield another A.P.

  $\therefore \frac{1}{bc}, \frac{1}{ca}, \frac{1}{ab}$ will be in A.P.

  Multiplying each term with $abc + 1$ will yield another A.P.

  $\therefore a + \frac{1}{bc}, b + \frac{1}{ca}, c + \frac{1}{ab}$ will be in A.P.
\item Given $a, b, c$ are in A.P. $\therefore b - a = c - b$

  $\Rightarrow \frac{1}{b - a} = \frac{1}{c - b} \Rightarrow \frac{ab + bc + ca}{b - a} = \frac{ab + bc + ca}{c - b}$

  $\Rightarrow ab(b - a) + c(b^2 - a^2) = bc(c - a) + a(c^2 - b^2)$

  $\Rightarrow b^2a + b^2c - a^2b - a^2c = c^2a + c^2b - b^2c - b^2a \Rightarrow b^2(a + c) - a^2(b + c) = c^2(a + b) - b^2(c + a)$

  $\therefore a^2(b + c), b^2(c + a), c^2(a + b)$ are in A.P.
\item We will prove this in reverse. We assume that $\frac{1}{\sqrt{b} + \sqrt{c}}, \frac{1}{\sqrt{c} + \sqrt{a}},
  \frac{1}{\sqrt{a} + \sqrt{b}}$ are in A.P.

  $\Rightarrow \frac{1}{\sqrt{c} + \sqrt{a}} - \frac{1}{\sqrt{b} + \sqrt{c}} = \frac{1}{\sqrt{a} + \sqrt{b}} +
  \frac{1}{\sqrt{c} + \sqrt{a}}$

  $\Rightarrow \frac{2}{\sqrt{c} + \sqrt{a}} = \frac{1}{\sqrt{b} + \sqrt{c}} + \frac{1}{\sqrt{a} + \sqrt{b}}$

  $\Rightarrow \frac{2}{\sqrt{c} + \sqrt{a}} = \frac{\sqrt{a} + \sqrt{b} + \sqrt{b} + \sqrt{c}}{(\sqrt{b} + \sqrt{c})(\sqrt{a}
    + \sqrt{b})}$

  $\Rightarrow 2(\sqrt{b} + \sqrt{c})(\sqrt{a} + \sqrt{b}) = (\sqrt{c} + \sqrt{a})(\sqrt{a} + 2\sqrt{b} + \sqrt{c})$

  $\Rightarrow 2(\sqrt{ab} + b + \sqrt{ac} + \sqrt{bc}) = \sqrt{ac} + 2\sqrt{bc} + c + a + 2\sqrt{ab} + \sqrt{ac}$

  $\Rightarrow 2b = a + c$, which implies that $a, b, c$ are in A.P. So the reverse is also true.
\item Given $a, b, c$ are in A.P.

  Dividing each term by $abc$ will yield another A.P.

  $\Rightarrow \frac{1}{bc}, \frac{1}{ca}, \frac{1}{ab}$ will be in A.P.

  Multiplying each term with $ab + bc + ca$ will yield another A.P.

  $\Rightarrow \frac{ab + ca}{bc} + 1, \frac{ab + bc}{ca} + 1, \frac{bc + ca}{ab} + 1$ will be in A.P.

  Subtracting $1$ from each term yields desired terms in A.P.
\item We have to prove that $\frac{1}{b - c}, \frac{1}{c - a}, \frac{1}{a - b}$ are in A.P.

  i.e. $\frac{1}{c - a} - \frac{1}{b - c} = \frac{1}{a - b} - \frac{1}{c - a}$

  $\Rightarrow \frac{b -2c + a}{(c - a)(b - c)} = \frac{c - 2a + b}{(a - b)(c - a)}$

  $\Rightarrow (a + b - 2c)(a - b) = (b + c - 2a)(b - c)$

  Now, given that $(b - c)^2, (c - a)^2, (a - b)^2$ are in A.P.

  $\Rightarrow (c - a)^2 - (b - c)^2 = (a - b)^2 - (c - a)^2$

  $\Rightarrow (b - a)(2c - a - b) = (c - b)(2a - b - c)$

  Thus, we have proven the desierd result.
\item Given $a, b, c$ are in A.P.

  Subtracting $a, b, c$ from each term will yield another A.P.

  $\Rightarrow -(b + c), -(c + a), -(a + b)$ will be in A.P.

  Multiplying each term with $-1$ will yield the desired A.P.
\item We have to prove that $\frac{1}{b + c}, \frac{1}{c + a}, \frac{1}{a + b}$ are in A.P.

  i.e. $\frac{1}{c + a} - \frac{1}{b + c} = \frac{1}{a + b} - \frac{1}{c + a}$

  $\Rightarrow \frac{b - a}{(b + c)} = \frac{c - b}{(a + b)}$

  $\Rightarrow b^2 - a^2 = c^2 - b^2 \Rightarrow a^2, b^2, c^2$ are in A.P.

  Thus, we have proven the desired result in reverse.
\item Given that $a, b, c$ are in A.P. $\Rightarrow b - a = c - b = k$ (say)

  $\Rightarrow c - a = 2k \Rightarrow 2(a - b) = a - c = 2(b - c) = -2k$.
\item Given that $a, b, c$ are in A.P. Let $b = a + d \Rightarrow c = a + 2d$

  Now, $(a - c)^2 = 4d^2, 4(b^2 - ac) = 4[(a + d)^2 - a(a + 2d)] = 4d^2$

  $\Rightarrow (a - c)^2 = 4(b^2 - ac)$
\item Let $n = 2m + 1$ where $m\in N. \Rightarrow S_1 = \frac{n}{2}[t_1 + t_n]$ where $d$ is the commond difference.

  For $S_2$ the no. of terms will be $m. \Rightarrow S_2 = \frac{m}{2}[t_2 + t_{n - 1}]$

  We know that $t_1 + t_n = t_2 + t_{n - 1}$

  $\therefore \frac{S_1}{S_2} = \frac{n}{m} = \frac{n}{\tfrac{n - 1}{2}} = \frac{2n}{n - 1}$.
\item The degree is the highest power of $x$ which will be $1 + 6 + 11 + \cdots + 101$.

  Clearly, the above sequence is an A.P. having first term $1$, common difference $5$ and last term as $101$.

  $n = \frac{t_n - t_1}{d} + 1 = \frac{101 - 1}{5} + 1 = 21$.

  $\Rightarrow S = \frac{21}{2}[t_1 + t_n] = \frac{21}{2}[1 + 101] = 21\times 51 = 1071$

  Therefore, the degree of the polynomial will be $1071$.
\item Consider an A.P. with first term as $a$, commond difference as $d$ and no. of terms as $n$. Then sum is given by

  $S = \frac{n}{2}[2a + (n - 1)]d = \frac{n^2d^2}{2} + \frac{(2a - d)n}{2}$

  which is of the form $An^2 + Bn$ where $A = \frac{d^2}{2}$ and $B = \frac{2a - d}{2}$.
\item Let the common difference of the A.P. be $d$.

  L.H.S. $= a_1^2 - a_2^2 + a_3^2 - a)4^2 + \cdots + a_{2n - 1}^2 -a_{2n}^2$

  $= (a_1 - a_2)(a_1 + a_2) + (a_3 - a_4)(a_3 + a_4) + \cdots + (a_{2n - 1} - a_{2n})(a_{2n - 1} + a_{2n})$

  $= -d(a_1 + a_2 + a_3 + a_4 + \cdots + a_{2n - 1} + a_{2n})$

  $= -\frac{2nd}{2}[a_1 + a_{2n}]$

  $= \frac{n}{2n - 1}(a_1^2 - a_{2n}^2)\left[\because d = \frac{a_{2n} - a_1}{2n - 1}\right]$
\item We know that sum of equidistant terms from start and end of an A.P. is equal.

  $\therefore a_1 + a_{24} = a_5 + a_{20} = a_{10} + a_{15} = k$ (say)

  $\therefore a_1 + a_5 + a_{10} + a_{15} + a_{24} = 3k = 225 \Rightarrow k = 75$

  Sum of first $24$ terms $S = a_1 + a_2 + \cdots + a_{24} = \frac{24}{2}[a_1 + a_{24}] = 12\times75 = 600$.
\item Let $a$ be the first term and $d$ be the common difference. Also let $S_1$ denote the sum of first $3n$ terms and $S_2$
  denote the sum of next $n$ terms.

  $S_1 = \frac{3n}{2}[2a + (3n - 1)d], S_2 = \frac{n}{2}[2a + 6nd + (n - 1)d][\because t_{3n+1} = a + 3nd]$

  Given, $S_1 = S_2 \Rightarrow \frac{3n}{2}[2a + (3n - 1)d] = \frac{n}{2}[2a + 6nd + (n - 1)d]$

  $\Rightarrow 6a + (9n - 3)d = 2a + (7n - 1)d \Rightarrow 2a + (n - 2)d = 0$

  Let $S_3$ be sum of first $2n$ terms and $S_4$ be sum of next $2n$ terms, then

  $\frac{S_3}{S_4} = \frac{\tfrac{2n}{2}[2a + (2n - 1)d]}{\tfrac{2n}{2}[2a + 4nd + (2n - 1)]d}$

  $\Rightarrow = \frac{nd}{5nd} = \frac{1}{5}[\because 2a + (n - 1)d = 0xs]$
\item Given $S_n = 5n^2 + 3n \Rightarrow t_n = S_n - S_{n - 1} = 5n^2 + 3n - 5(n - 1)^2 - 3(n - 1)$

  $= 10n - 5 + 3 = 10n - 2 \Rightarrow d = t_n - t_{n - 1} = 10n - 2 - 10(n - 1) + 2 = 10$,

  Since common difference is a constant the series is in A.P.
\item Common difference of the series $d = (a^2 + b^2) - (a + b)^2 = (a - b)^2 - (a^2 + b^2) = -2ab$

  $S = \frac{n}{2}[2(a + b)^2 - (n - 1)2ab] = \frac{n}{2}[2a^2 + 2b^2 - 2(n + 1)ab]$

  $= n[a^2 + b62 - (n + 1)ab]$.
\item There will be two cases. First $n$ being odd and second $n$ being even.

  {\bf Case I:} When $n$ is odd i.e. $n = 2m + 1$, where $m = 0, 1, 2, \ldots$

  $S = 1 + 5 + 9 + \cdots$ up to $m + 1$ terms $- 3 - 7 - 11$ up to $m$ terms

  $= \frac{m + 1}{2}[2 + 4m] - \frac{m}{2}[6 + 4m - 4] = (m + 1)(1 + 2m) - m(2m + 1)$

  $= 2m^2 + 3m + 1 - 2m^2 - m = 2m + 1 = n$.

  {\bf Case II:} When $n$ is even i.e. $n = 2m$, where $m =1, 2, 3, \ldots$

  $S = 1 + 5 + 9 + \cdots$ up to $m$ terms $- 3 - 7 - 11$ up to $m$ terms

  $= \frac{m}{2}[2 + 4m - 4] - \frac{m}{2}[6 + 4m - 4] = -2m = -n$.
\item Let there be $n$ sides of the polygon. From geometry, we know that sum of angles of the polygon $= (n - 2)180^\circ$

  From the formula for sum of an A.P. $S = \frac{n}{2}[2\times120^\circ + (n - 1)5^\circ] = (n - 2)180^\circ$

  $\frac{n}[240^\circ + (n - 1)5^\circ] = (n - 2)360^\circ\Rightarrow n[48^\circ + (n - 1)] = (n - 2)72^circ$

  $\Rightarrow n^2 - 25n + 144 = 0 \Rightarrow n = 9, 16$
\item To water first tree the gardener will have to travel $10$ m. To water second tree he will
  have tp travel back $10$ m to well and then $15$ m to the tree i.e. $25$ m. Similarly, for third tree he will
  have to travel $15$ m to well and $20$ m i.e a total of $35$ m.

  Thus, total distance travelled will be $10 + 25 + 35 + \cdots$

  Clearly, $25$ will be the first term of the A.P. and there will be $24$ such terms because distance travelled
  for first tree is noty part of the A.P. Note that common difference would be $10$.

  Total distance travelled $= 10 + \frac{24}{2}[2\times25 + (24 - 1)10] = 10 + 3360 = 3370$ m.
\item Let $d$ be the common difference. Given $S_p = 0 \Rightarrow \frac{p}{2}[2a + (p - 1)d] = 0$

  $\Rightarrow 2a + (p - 1)d = 0 \Rightarrow d = \frac{2a}{1 - p}$

  $p + 1$th term $t_{p + 1} = a + pd$, so the sum of next $q$ terms $S = \frac{q}{2}[2a + 2pd + (q - 1)d]$

  $= \frac{q}{2}[2a + (2p + q - 1)d] = \frac{q}{2}\left[2a + (2p + q - 1).\frac{2a}{1 - p}\right]$

  $= \frac{q}{2}\left[\frac{2a.(p + q)}{1 - p}\right] = -\frac{a(p + q)}{p - 1}q$.
\item Sum of first $p$ terms, $S_p = \frac{p}{2}[2a + (p - 1)d]$; sum of first $q$ terms $S_q = \frac{q}{2}[2a + (q - 1)d]$

  $2ap + (p^2 - p)d = 2aq + (q^2 - q)d \Rightarrow 2a(p - q) = (q^2 - p^2 + p - q)d$

  $2a = (1 - p - q)d$

  Sum of $(p + q)$ terms, $S_{p + q} = \frac{p + q}{2}[2a + (p + q - 1)d] = \frac{p + q}{2}[(1 - p - q)d + (p + q - 1)d] = 0$.
\item Sum of latter half of $2n$ terms means $n + 1$th term to $2n$th term. $t_{n + 1} = a + nd$ and $t_{2n} = a + (2n - 1)d$
  where $a$ and $d$ are the first term and common difference respectively.

  Sum of latter half of terms, $S = \frac{n}{2}[t_{n + 1} + t_{2n}] = \frac{n}{2}[2a + (3n - 1)d]$

  Sum of first $3n$ terms, $S_{3n} = \frac{3n}{2}[2a + (3n - 1)d]$

  Clearly, $S/S_{3n} = 1:3$.
\item Let $S_r$ be the $r$th A.P. whose first term is $r$ and common difference is also $r$.

  $S_r = \frac{n}{2}[2r + (n - 1)r] = \frac{n}{2}[(n + 1)r] = \frac{n(n + 1)r}{2}$

  $S_1 + S_2 + S_3 + \cdots + S_p = \displaystyle\sum_{r=1}^pS_r$

  $= \frac{n(n + 1)}{2}\displaystyle\sum_{r = 1}^pr = \frac{np}{4}(n + 1)(p + 1)\left[\because\displaystyle\sum_{i=1}^ni =
  \frac{n(n + 1)}{2}\right]$.
\item Let $x$ be the first term and $y$ be the common difference of the A.P.

  Then, according to the question $a = \frac{p}{2}[2x + (p - 1)y], b = \frac{q}{2}[2x + (q - 1)y], c = \frac{r}{2}[2x + (r - 1)y]$

  We have to prove that $\frac{a}{p}(q - r) + \frac{b}{q}(r - p) + \frac{c}{r}(p - q) = 0$

  L.H.S. $= x(q - r + r - p + p - q) + \frac{y}{2}[(p - 1)(q - r) + (q - 1)(r - p) + (r - 1)(p - q)]$

  $= 0$.
\item Let $a$ be the first term and $d$ be the common difference of the A.P.

  Given, $S_m = \frac{1}{2}S_{m + n}\Rightarrow \frac{m}{2}[2a + (m - 1)d] = \frac{1}{2}.\frac{m + n}{2}[2a + (m + n - 1)d]$

  Let $2a + (m - 1)d = x$, then the above equation can be written as

  $mx = \frac{m + n}{2}[x + nd] \Rightarrow 2mx = (m + n)[x + nd]\Rightarrow mx = n(x + nd) + mnd$

  $\Rightarrow (m - n)x = (m + n)nd$

  Similarly, $(m - p)x = (m + p)pd$

  Dividing, we get

  $(m - n)(m + p)p = (m + n)(m - p)n$

  Dividing both sides with $mnp$ we arrive at the desired result.
\item Let $a$ be the first term and $d$ be the common difference of the A.P. For odd terms, the no. of terms will be $n + 1$, first
  term will be $a$ and common difference will be $2d$.

  $\therefore S_{odd} = \frac{n + 1}{2}[2a + 2nd]$

  For even terms, the no. of terms will be $n$, first term will be $a + d$ and common difference will be $2d$.

  $\therefore S_{even} = \frac{n}{2}[2a + 2d + 2(n - 1)d] = \frac{n}{2}[2a + 2nd]$

  $\therefore \frac{S_{odd}}{S_{even}} = \frac{n + 1}{n}$.
\item Let $a_1$ and $a_2$ be the first terms and $d_1$ and $d_2$ be the common differences of the two series in A.P.

  Given, $\frac{\tfrac{n}{2}[2a_1 + (n - 1)d_1]}{\tfrac{n}{2}[2a_2 + (n - 1)d_2]} = \frac{3n - 12}{5n + 21}$

  $\Rightarrow \frac{2a_1 + (n - 1)d_1}{2a_2 + (n - 1)d_2} = \frac{3n - 13}{5n + 21}$

  We need to find ratio of the $24$th terms i.e. $\frac{a_1 + 23d_1}{a_2 + 23d_2} = \frac{2a_1 + 46d_1}{2a_2 + 46d_2}$

  Putting $n = 47$ in the ratio of sums, we have

  $\frac{2a_1 + 46d_1}{2a_2 + 46d_2} = \frac{3\times 47 - 13}{5\times47 + 21} = \frac{1}{2}$
\item Let $a$ be the first term and $d$ be the common difference of the A.P.

  Given, $t_m = a + (m - 1)d = \frac{1}{n},\,t_n = a + (n - 1)d = \frac{1}{m}$

  Subtracting, we get $(m - n)d = \frac{m - n}{mn} \Rightarrow d = \frac{1}{mn} \Rightarrow a = \frac{1}{mn}$

  $\therefore S_{mn} = \frac{mn}{2}\left[\frac{2}{mn} + \frac{mn - 1}{mn}\right] = \frac{mn + 1}{2}$.
\item Let $a$ be the first term and $d$ be the common difference of the A.P.

  Given, $S_m = n = \frac{m}{2}[2a + (m - 1)d] \Rightarrow 2a + (m - 1)d = \frac{2n}{m}$

  and $S_n = m = \frac{n}{2}[2a + (n - 1)d] \Rightarrow 2a + (n - 1)d = \frac{2m}{n}$

  $\Rightarrow d = -\frac{2(m + n)}{mn}\Rightarrow a = \frac{m^2 + n^2 + mn - m - n}{mn}$

  $\Rightarrow S_{m + n} = \frac{m + n}{2}[2a + (m + n - 1)d] = -(m + n)$.
\item Let $a$ be the first term and $d$ be the common difference of the A.P.

  $\therefore S = \frac{2n + 1}{2}[2a + 2nd]$

  For $S_1$ first term would be $a$, common difference would be $2d$ and no. of terms would be $n + 1$.

  $\therefore S_1 = \frac{n + 1}{2}[2a + 2nd]$

  $\therefore \frac{S}{S_1} = \frac{2n + 1}{n +1}$.
\item Let $d$ be the common difference, then $b = a + 2d \Rightarrow d = \frac{b - a}{2}$

  $c = a + (n - 1)d \Rightarrow n - 1 = \frac{c - a}{d} = \frac{2(c - a)}{b - a}$

  $\Rightarrow n = \frac{2(c - a)}{b - a} + 1$

  $\therefore S = \frac{n}{2}[2a + (n - 1)d] = \frac{1}{2}\left[\frac{2(c - a)}{b - a} + 1\right]\left[2a + \frac{2(c - a)}{b -
    a}.\frac{b - a}{2}\right]$

  $= \frac{c + a}{2} + \frac{c^2 - a^2}{b - a}$.
\item Let $a_1, a_2$ be the first terms and $d_1, d_2$ be the common differences of the two series in A.P.

  According to the question $\frac{2a_1 + (n - 1)a_1}{2a_2 + (n - 1)d_2} = \frac{3n + 8}{7n + 15}$.

  We have to find ratio of $12$th terms i.e. $\frac{a_1 + 11d_1}{a_2 + 11d_2} = \frac{2a_1 + 22d_1}{2a_2 + 22d_2}$

  Putting $n = 23$ in previous equation, we get

  $\frac{2a_1 + 22d_1}{2a_2 + 22d_2} = \frac{77}{176} = \frac{7}{16}$.
\item Let $a$ be the first term and $d$ be the common difference of the A.P.

  Given, $\frac{S_m}{S_n} = \frac{\tfrac{m}{2}[2a + (m - 1)d]}{\tfrac{n}{2}[2a + (n - 1)d]} = \frac{m^2}{n^2}$

  $\Rightarrow \frac{2a + (m - 1)d}{2a + (n - 1)d} = \frac{m}{n}$

  $\Rightarrow 2a(n - m) + [(m - 1)n - (n - 1)m]d = 0 \Rightarrow a = \frac{d}{2}$

  We have to find $\frac{t_m}{t_n} = \frac{a + (m - 1)d}{a + (n - 1)d} = \frac{2m - 1}{2n - 1}$
\item Let $n$ be the no. of terms. Clearly, common ratio $r = \frac{20}{5} = \frac{80}{20} = 4$

  Then $t_n = 5120 = 5.r^{n - 1} \Rightarrow 4^{n - 1} = 1024 = 4^5 \Rightarrow n = 6$.
\item Let $n$ be the no. of terms. Clearly, common ratio $r = \frac{0.06}{0.03} = \frac{0.12}{0.06} = 2$

  Then $t_n = 3.84 = 0.03r^{n - 1} \Rightarrow 2^{n - 1} = 128 \Rightarrow n = 8$.
\item From the question we deduce that it is a G.P. with $a = 1, r = 2, n = 20$. We have to find $t_{20}$.

  $t_{20} = 1.2^{20 - 1} = 524288$.
\item This is a G.P. with $a = 20000, r = 1.02, n = 11$. We have to find $t_{11}$.

  $t_{11} = 20000\times(1.02)^{11 - 1} = 24380$.
\item Given, $S_n = 2^n - 1 \Rightarrow t_n = S_n - S_{n - 1} = 2^n - 1 - (2^{n - 1} - 1) = 2^{n - 1}$

  $r = \frac{t_n}{t_{n - 1}} = \frac{2^{n - 1}}{2^{n - 2}} = 2$, which is a constant and hence the sequence is in G.P.
\item Let the first term of the G.P. be $a$ and common ratio is $r$.

  Then $t_2 = ar = 24$ and $t_5 = ar^4 = 81$, Dividing, we have $r^3 = \frac{81}{24} = \frac{27}{8}$

  $\Rightarrow r = \frac{3}{2} \Rightarrow a = 16$.

  Hence the G.P. is $16, 24, 36, 54, 81, \ldots$.
\item Let the first term of the G.P. be $a$ and common ratio is $r$.

  Given $t_7 = 8t_4 \Rightarrow ar^6 = 8ar^3 \Rightarrow r = 2$. Also given, $t_5 = 48 \Rightarrow ar^4 = 48$

  $\Rightarrow a = 3$. Hence, the G.P. is $3, 6, 12, 24, \ldots$.
\item Let the first term of the G.P. be $a$ and common ratio is $r$.

  Given, $t_5 = ar^4 = 48$ and $t_8 = ar^7 = 384 \Rightarrow r^3 = 8 \Rightarrow r = 2$

  $\Rightarrow a = 3$. Hence, the G.P. is $3, 6, 12, 24, \ldots$.
\item Let the first term of the G.P. be $a$ and common ratio is $r$.

  Given $t_6 = ar^5 = \frac{1}{16}$ and $t_{10} = ar^9 = \frac{1}{256} \Rightarrow r = \pm\frac{1}{2}$

  $\Rightarrow a = \pm 2$. Hence the G.P. is $2, 1, \frac{1}{2}, \ldots$ or $-2, 1, -\frac{1}{2}, \ldots$.
\item Let the first term of the G.P. be $x$ and common ratio is $y$. Then

  $a = xy^{p - 1}, b = xy^{q - 1},  c = xy^{r - 1}$

  Taking $\log$ of both sides for these three terms

  $\log a = \log x + (p - 1)\log y, \log b = \log x + (q - 1)\log y, \log c = \log x + (r - 1)\log y$

  Clearly, $(q - r)\log a + (r - p)\log b + (p - q)\log r = 0$.
\item Let the first term of the G.P. be $x$ and common ratio is $r$.

  Given, $t_{p + q} = a = xr^{p + q - 1}$ and $t_{p - q} = b = xr^{p - q - 1}$

  Multiplying the two terms, we have

  $x^2r^{2p - 2} = (xr^{p - 1})^2 = t_p^2 = ab \Rightarrow t_p = \sqrt{ab}$.
\item Let $a$ be the first term and $b$ be the common ratio. Then,

  $x = ab^{p - 1}, y = ab^{q - 1}, z = ab^{r -1}$

  We have to prove that $x^{q - r}.y^{r - p}.z^{p - q} = 1$

  L.H.S. $= (ab^{p - 1})^{q - r}.(ab^{q - 1})^{r - p}.(ab^{r - 1})^{p - q}$

  $= a^{(q - r + r - p + p - q)}b^{[(p - 1)(q - r) + (q - 1)(r - p) + (r - 1)(p - q)]}$

  $= a^0b^0 = 1 =$ R.H.S.
\item Let $r$ be the common ratio and first term is given as $1$.

  $t_3 + t_5 = 90 \Rightarrow r^4 + r^2 = 90 \Rightarrow r^2 = 9 \Rightarrow r = pm 3$.

  $r^2$ cannot be $-10$ as that would mean that it is an imaginary number.
\item Let $a$ be the first term and $r$ be the common ratio of the G.P.

  Gibem $t_5 = ar^4 = 2$ and we have to find the product of the first nine terms. Let the required product be $S$.

  $S = a.ar.ar^2.\ldots.ar^8 = a^9r^{1 + 2 + \cdots + 8} = a^9r^{\tfrac{8.9}{2}} = a^9r^{36} = (ar^4)^9 = 2^9 = 512$.
\item Let $a$ be the first term, $r$ be the common ratio and $n$ be the number of terms.

  Given, $t_4 = ar^3 = 10, t_7 = ar^6 = 80, t_n = ar^{n - 1} = 2560$

  $\therefore \frac{t_7}{t_4} = r^3 = 8 \Rightarrow r = 2 \Rightarrow a = \frac{10}{8}$

  $\Rightarrow \frac{10}{8}2^{n- 1} = 2560 \Rightarrow 2^{n - 1} = 2048 \Rightarrow n = 12$.
\item Let the three numbers in G.P. be $a, ar, ar^2$. According to question, on doubling $ar$ the numbers form an A.P.

  $\Rightarrow 2ar - a = ar^2 - 2ar \Rightarrow r^2 - 4r + 1 = 0 \Rightarrow r = \frac{4\pm\sqrt{12}}{2} = 2\pm\sqrt{3}$.
\item Given, $p, q, r$ are in A.P. i.e. $q - p = r - q$.

  Let $x$ be the first term and $y$ be the common ratio of the G.P. We have to prove that $t_p, t_q, t_r$ are in G.P.

  $\Rightarrow \frac{t_q}{t_p} = \frac{t_r}{t_q} \Rightarrow \frac{xy^{q - 1}}{xy ^{p - 1}} = \frac{xy^{r - 1}}{xy^{q - 1}}$

  $\Rightarrow y^{q - p} = y^{r - q}$ which is true from the condition for A.P.
\item Let $r$ be the common ratio of the G.P. Then, $b = ar, c = ar^2, d = ar^3$

  L.H.S. $= (a.ar + ar.ar^2 + ar^2.ar^3)^2 = a^4r^2(1 + r^2 + r^4)^2$

  R.H.S. $= (a^2 + a^2r^2 + a^2r^4)(a^2r^2 + a^2r^4 + a^2r^6) = a^2(1 + r^2 + r^4).a^2r^2(1 + r^2 + r^4)$

  $= a^2r^4(1 + r^2 + r^4)^2 =$ L.H.S.
\item Given $a, b, c$ are in A.P. $\Rightarrow 2b = a + c$

  If we increase $a$ by $1$ then they are in G.P. $\Rightarrow b^2 = (a + 1)c \Rightarrow b^2 = (a + 1)(2b - a)$

  $\Rightarrow b^2 = 2ab - a^2 + 2b - a \Rightarrow (a - b)^2 = 2b - a$

  If we increase $c$ by $2$ then again they are in G. P $\Rightarrow b^2 = a(c + 2) = a(2b - a + 2)$

  $\Rightarrow b^2 = 2ab - a^2 + 2a \Rightarrow (a - b)^2 = 2a \Rightarrow 2b - a = 2a \Rightarrow 2b = 3a$

  $\Rightarrow \left(a - \frac{3a}{2}\right)^2 = 2a \Rightarrow a = 8 \Rightarrow b = 12 \Rightarrow c = 16$.
\item Let the three numbers in G.P. be $\frac{a}{r}, a, ar$. Then,

  $\frac{a}{r} + a + ar = 70$ and $10a = \frac{4a}{r} + 4ar \Rightarrow \frac{10a}{4} = \frac{a}{r} + ar$

  $\Rightarrow \frac{10a}{4} + a = 70 \Rightarrow a = 20$

  $\Rightarrow \frac{20}{r} + 20r = 50 \Rightarrow r = 2, \frac{1}{2}$

  So the numbers are $10, 20, 40$ or $40, 20, 10$.
\item Let the three numbers in G.P. be $\frac{a}{r}, a, ar$. Given that product of these numbers is $216$.

  $\Rightarrow \frac{a}{r}.a.ar = 216 \Rightarrow a^3 = 216 \Rightarrow a = 6$

  Also, given that their sum is $19 \Rightarrow \frac{6}{r} + 6 + 6r = 19$

  $\Rightarrow 6r^2 - 13r + 6 = 0 \Rightarrow r = \frac{2}{3}, \frac{3}{2}$.

  So the numbers are $9, 6, 4$ or $4, 6, 9$.
\item Let the number be $100a + 10ar + ar^2$.

  According to question $a + ar^2 = 2ar + 1$ and $a + ar = \frac{2}{3}(ar + ar^2)$

  $\Rightarrow a(r - 1)^2 = 1$ and $3 + 3r = 2r + 2r^2 \Rightarrow r = -1, \frac{3}{2}$

  If $r = -1, a = \frac{1}{4}$, but $a$ cannot be a fraction.

  If $r = \frac{3}{2} \Rightarrow a = 4$ and the number is $469$.
\item Given that three of four numbers are in A.P. and so we choose them as $a - d, a, a + d$. Also, since first number is same as
  first so the numbers are $a + d, a - d, a, a + d$. The first three are in G.P. Given $d = 6$

  $\therefore (a - d)^2 = a(a + d)\Rightarrow (a - 6)^2 = a(a + 6) \Rightarrow 18a = 36 \Rightarrow a = 2$.

  So the numbers are $8, -4, 2, 8$.
\item Let the three numbers are $a, ar, ar^2$. The sum is given as $21 \Rightarrow a + ar + ar^2 = 21$.

  Also, sum of squares is given as $189 \Rightarrow a^2 + a^2r^2 + a^2r^4 = 189$

  $\Rightarrow \frac{441(1 + r^2 + r^4)}{(1 + r + r^2)^2} = 189$

  $\Rightarrow 7(1 + 2r^2 + r^4 - r^2) = 3(r + r + r^2)^2 \Rightarrow 7(1 - r + r^2) - 3(1 + r + r^2)$

  $\Rightarrow 2r^2 - 5r + 2 = 0 \Rightarrow r = 2, \frac{1}{2}$

  When $r = 2, a = 3$ and so the numbers are $3, 6, 12$.

  When $r = \frac{1}{2}, a = 12$ and so the numbers are $12, 6, 3$.
\item Let the terms in G.P. be $\frac{a}{r}, a, ar$. Given that the product of these is $-64$.

  $\therefore \frac{a}{r}.a.ar = -64 \Rightarrow a^3 = -64 \Rightarrow a = -4$.

  Also given that the first term is four times the third. $\Rightarrow \frac{a}{r} = 4.ar \Rightarrow r^2 = \frac{1}{4} \Rightarrow r = \pm\frac{1}{2}$.

  If $r = \frac{1}{2},$ the terms will be $-8, -4, -2$. If $r = -\frac{1}{2}$, the terms will be $8, -4, 2$.
\item Let the numbers be $a - d, a, a + d$. Given that sum is $15. \Rightarrow a - d + a + a + d = 15 \Rightarrow a = 5$.

  Also given that if $1, 4, 19$ are added to them then they are in G.P.

  $\Rightarrow (5 + 4)^2 = (5 - d + 1)(5 + d + 19) \Rightarrow 81 = (6 - d)(24 + d)$

  $\Rightarrow d^2 + 18d - 63 = 0 \Rightarrow d = -21, 3$.

  If $d = -15$, the numbers will be $26, 5, -16$ and if $d = 3$ the numbers will be $2, 5, 8$.
\item Let the two sets of three numbers in G.P. are $a_1, a_1r_1, a_1r_1^2$ and $a_2, a_2r_2, a_2r_2^2$.

  Given that the difference is also in G.P.

  $\Rightarrow (a_1r_1 - a_2r_2)^2 = (a_1r_1^2 - a_2r_2^2)(a_1 - a_2)$

  $\Rightarrow a_1^2r_1^2 + a_2^2r_2^2 - 2a_1a_2r_1r_2 = a_1^2r_1^2 - a_1a_2r_2^2 - a_1a_2r_1^2 + a_2^2r_2^2$

  $\Rightarrow 2a_1a_2r_1r_2 = a_1a_2r_2^2 + a_1a_2r_1^2 \Rightarrow 2r_1r_2 = r_1^2 + r_2^2 \Rightarrow (r_1 - r_2)^2 = 0$

  $\Rightarrow r _1 = r_2$ which implies that they have same common ratio.
\item Let $r$ be the common ratio. Then $b = ar, c = ar^2, d = ar^3$

  L.H.S. $= (b - c)^2 + (c - a)^2 + (d - b)^2 = (ar - ar^2)^2 + (ar^2 - a)^2 + (ar^3 - ar)^2$

  $= a^2(r - r^2)^2 + a^2(r^2 - 1)^2 + a^2(r^3 - r)^2 = a^2(r^2 + r^4 - 2r^3 + r^4 + 1 - 2r^2 + r^6 + r^2 - 2r^4)$

  $= a^2(r^6 - 2r^3 + 1) = (ar^3 - a)^2 = (d - a)^2 = $ R.H.S.
\item This problem can be solved like previous problem.
\item Given that $x, y, z$ are in G.P. Let $p$ be the first term and $r$ be the common ratio of this G.P.

  Also given, $a^x = b^y = c^z \Rightarrow x\log a = y\log b = z\log c$

  $\Rightarrow \frac{\log a}{\log b} = \frac{y}{x}$ and $\frac{\log b}{\log c} = \frac{z}{y}$. Clearly, $\frac{y}{x} = \frac{z}{y} =
  r \Rightarrow \log_ba = \log_cb$.
\item Let $\frac{a}{r}, a, ar$ be the terms in G.P. Given that continued product is $216$ i.e.

  $\frac{a}{r}.a.ar = 216 \Rightarrow a^3 = 216 \Rightarrow a = 6$

  Sum of products when taken in pair is given as $156$.

  $\Rightarrow \frac{a}{r}.a + a.ar + \frac{a}{r}.ar = 156 \Rightarrow \frac{1}{r} + r + 1 = \frac{26}{6}$

  $\Rightarrow 6r^2 -20r + 6 = 0 \Rightarrow r = \frac{1}{3}, 3$

  So the numbers are $18, 6, 2$ or $2, 6, 18$.
\item Let $r$ be the common ratio. Then, $\frac{(b + c)^2}{(a + b)^2} = \frac{(ar + ar^2)^2}{(a + ar)^2} = r^2$.

  Similarly, $\frac{(c + d)^2}{(b + c)^2} = r^2 = \frac{(b + c)^2}{(a + b)^2}$.

  Thus, $(a + b)^2, (b + c)^2, (c + d)^2$ are also in G.P.
\item This problem can be solved like previous problem.
\item This problem can be solved like previous problem.
\item This problem can be solved like previous problem.
\item Let $r$ be the common ratio. Then, $a(b - c)^3 = a(ar - ar^2)^3 = a^4r^3(1 - r)^3$ and $d(a - b)^3 =
  ar^3(a - ar)^3 = a^4r^3(1 - r)^3$.

  Thus, $a(b - c)^3 = d(a - b)^3$.
\item We have to prove that $(a + b + c + d)^2 = (a + b)^2 + (c + d)^2 + 2(b + c)^2$ where $a, b, c, d$ are
  in G.P.

  Now, $(a + b + c + d)^2 = (a + b)^2 + (c + d)^2 + 2(a + b)(c + d)$ so it is enough to prove that $(a +
  b)(c + d) = (b + c)^2$.

  $(a + b)(c + d) = (a + ar)(ar^2 + ar^3) = a^2r^2(1 + r)^2$ and $(b + c)^2 = (ar + ar^2)^2 = a^2r^2(1 +
  r)^2$ which proves the required equality.
\item Let $r$ be the common ratio. L.H.S. $= a^2b^2c^2\left(\frac{1}{a^3} + \frac{1}{b^3} +
  \frac{1}{c^3}\right) = \frac{b^2c^2}{a} + \frac{a^2c^2}{b} + \frac{a^2b^2}{c}$

  $= a^3r^6 + a^3r^3 + a^3 = a^3 + b^3 + c^3 =$ R.H.S.
\item Let $r$ be the common ratio. L.H.S. $=(a^2 - b^2)(b^2 + c^2) = (a^2 - ar^2)(a^2r^2 + a^2r^4) = r^2(a^2
  - a^2r^2)(a^2 + a^2r^2) = (a^2r^2 - a^2r^4)(a^2 + a^2r^2) = (b^2 - c^2)(a^2 + b^2) =$ R.H.S.
\item Let $r$ be the common ratio. Given $a, b, c$ are in G.P. i.e. $a, ar, ar^2$ are in G.P.

  Taking $\log$ of $a, b, c$, we have

  $\log a, \log a + \log r, \log a + 2\log r$ are in A.P. with $\log a$ being the first term and $\log r$ be
  the common difference.
\item Given series is $1 + \frac{1}{2} + \frac{1}{4} + \frac{1}{8} + \cdots$ to $n$ terms. Let $S$ be the
  sum, $a = 1, r = \frac{1}{2}$, then

  $S = \frac{a(1 - r^n)}{1 - r} = 2\left(\frac{2^n - 1}{2^n}\right)$
\item Given series is $1 + 2 + 4 + 8 + \cdots$ to $12$ terms. First term $a = 1$, common ratio $r = 2$
  and no. of terms $n = 12$. Let $S$ be the sum of the series. Then,

  $S = \frac{a(r^n - 1)}{r - 1} = \frac{1(2^{12} - 1)}{2 - 1} = 4095$.
\item Given series is $1 - 3 + 9 - 27 + \cdots$ to $9$ terms. First terms $a = 1$, common ratio $r = -3$ and
  no. of terms $n = 9$. Let $S$ be the sum of the series. Then,

  $S = \frac{a(1 - r^n)}{1 - r} = \frac{1 - (-3)^9}{1 - (-3)} = 4921$
\item This problem is similar to $115$, and has been left as an exercise.
\item Given series is $(a + b) + (a^2 + 2b) + (a^3 + 3b) + \cdots $ to $n$ terms. We can rewrite the series
  as $a + a^2 + a^3 + \cdots$ to $n$ terms + $b + 2b + 3b + \cdots$ to $n$ terms.

  We know that $a + a^2 + a^3 + \cdots$ to $n$ terms $= \frac{a(a^n - 1)}{a - 1}$ and for the second series
  applying the A.P. formula, $b + 2b + 3b + \cdots$ to $n$ terms $= \frac{n}{2}[2b + (n - 1)b] =
  \frac{n}{2}[(n + 1)b] = \frac{n(n + 1)b}{2}$.
\item Clearly the given situation forms a G.P. with $a = 1$, common ratio $r = 2$ and $n = 120$. Let $S$ be
  the sum which he gets at the end of $120$ days. Then,

  $S = \frac{a(r^n - 1)}{r - 1} = 2^{120} - 1 = 1329227995784915872903807060280344575$.
\item Given series is $S = 8 + 88 + 888 + \cdots = \frac{8}{9}[9 + 99 + 999 + \cdots]$

  $= \frac{8}{9}[(10 - 1) + (100 - 1) + (1000 - 1) + \cdots]$

  $= \frac{8}{9}\left[\frac{10(10^n - 1)}{10 - 1} - n\right] = \frac{8}{81}[10^{n + 1} - 10 - 9n]$.
\item This problem can be solved like previous problem.
\item This problem can be solved like previous problem.
\item This problem can be solved like previous problem.
\item Let $S = 1 - \frac{1}{2} + \frac{1}{4} - \frac{1}{8} + \cdots$ to $n$ terms. Clearly, $a = 1$ and $r =
  -\frac{1}{2}$.

  $\Rightarrow S = \frac{a(1 - r^n)}{1 - r} = \frac{1 - (-1)^n\frac{1}{2^n}}{1 -(-)\frac{1}{2}} =
  \frac{2}{3}.\frac{2^n - (-1)^n}{2^n}$.
\item When we make $1000$ per day for $31$ days total amount received will be $31,000$.

  When we receive $1$ for the first day and doubling every day then that would be a G.P. with $a = 1, r =
  2, n = 31 \Rightarrow S = \frac{a(r^n - 1)}{r - 1} = 2^{31} - 1 = 2,147,483,647$ which is clearly way more
  than we make in the first case so we will happily take the second option.
\item We assume that $n$ terms of the series $1 + 3 + 3^2 + \cdots$ make for $3280$. Then

  $S = \frac{1(3^n - 1)}{3 - 1} \Rightarrow 3^n = 6561 \Rightarrow n = 8$.
\item Let $S = 1 + 3 + 3^2 + \cdots + 3^{n - 1}\Rightarrow S = \frac{3^n - 1}{3 - 1} > 1000 \Rightarrow 3^n
  > 2001 \Rightarrow n = 7$.
\item Let the sum be $S$. Clearly it is a G.P. with $a = 1, r = \frac{1}{2}$. We know that when $|r| < 1$
  the sum of an infinite G.P. is given by $S = \frac{a}{1 - r}$. Thus, $S = \frac{1}{1- \frac{1}{2}} = 2$.
\item Clearly, it is a G.P. with $a = 1, r = 3$ and $n = 20$. Thus sum is given by $S = \frac{3^{20} - 1}{3
  - 1} = 1,743,392,200$.
\item We can represent the given series as three series like $(x^2 + x^4 + x^6 + \cdots)$ to $n$ terms $+
  \left(\frac{1}{x^2} + \frac{1}{x^4} + \frac{1}{x^6} + \cdots\right)$ to $n$ terms $+ 2 + 5 + 8 + \cdots$
  to $n$ terms. Let the sum be $S$.

  $S = x^2\frac{(x^2)^n - 1}{x^2 - 1} + \frac{1}{x^2}.\frac{\frac{1}{(x^2)^n - 1}}{\frac{1}{x^2} - 1}
  + \frac{n}{2}[3n + 1]$.
\item Let $n$ be the no. of terms required to make the sum of given G.P. with $a = 1, r = 2$ equal to $511$.

  $511 = \frac{2^n - 1}{2 - 1} \Rightarrow 2^n = 512 \Rightarrow n = 9$.
\item Let the sum be $S. S = 1 + 2 + 2^2 + \cdots + 2^{n - 1} = \frac{2^n - 1}{2 - 1}\geq 300 \Rightarrow
  2^n\geq 301 \Rightarrow n = 9$.
\item Let $r$ be the common ratio. $a_n = ar^{n - 1} = 96. S = \frac{a_1(r^n - 1)}{r - 1} = \frac{a_nr -
  a_1}{r - 1} = \frac{96r - 3}{r - 1} = 189 \Rightarrow 32r - 1 = 63r - 63 \Rightarrow r = 2 \Rightarrow n =
  6$.
\item $0.4\dot{2}\dot{3} = 0.4232323\ldots$ to $\infty = \frac{4}{10} + \frac{23}{1000} + \frac{23}{100000}
  + \cdots$ to $\infty$

  $= \frac{4}{10} + \frac{23}{100}\left[1 + \frac{1}{100} + \frac{1}{10000} +
    \cdots\mathrm{\;to\;}\infty\right]$
  $= \frac{4}{10} + \frac{23}{100}\frac{1}{1 - \frac{1}{100}} = \frac{419}{990}$.
\item Given series can be written as $S = \frac{1}{5} + \frac{1}{5^2} + \cdots$ to $\infty + \frac{1}{7} +
  \frac{1}{7^2} + \cdots $ to $\infty$

  $= \frac{1}{5}.\frac{1}{1 - \frac{1}{5}} + \frac{1}{7}.\frac{1}{1 - \frac{1}{7}} = \frac{1}{4} +
  \frac{1}{6} = \frac{5}{12}$.
\item Let the sum be $S$, then $S = (10 + 1) + (100 + 3) + (1000 + 5) + \cdots$ to $n$ terms

  $= \frac{10(10^n - 1)}{10 - 1} + \frac{n}{2}[2 + (n - 1)2] = \frac{10}{9}(10^n - 1) + n^2$.
\item The general term of the series is $t_n = \left(x^n + \frac{1}{x^n}\right)^2 = x^{2n} +
  \frac{1}{x^{2n}} + 2$ so we can write it as three series and solve like problem $132$.
\item Let $a$ be the first term and $r$ be the common ratio of the G.P. Then,

  $S = \frac{a(r^n - 1)}{r - 1}, P = a.ar.ar^2\ldots ar^{n - 1} = a^nr^{\tfrac{n(n - 1)}{2}}, R =
  \frac{1}{a}\frac{1 - \frac{1}{r^n}}{1 - \frac{1}{r}} = \frac{1}{a}\frac{r^n - 1}{r - 1}.\frac{1}{r^{n -
  1}}$

  $P^2 = a^{2n}r^{n(n - 1)}, \frac{S}{R} = a^2.r^{n - 1}\therefore  \left(\frac{S}{R}\right)^n = P^2$.
\item Clearly, the given series is a G.P. with $a = 1, r = \frac{x}{1 + x} \Rightarrow S = \frac{1}{1 -
  \frac{x}{1 + x}} = 1 + x$.
\item We consider the $n$-th term. $t_n = ar^{n - 1}$, where $a$ is the first term. Sum of all succeeding
  terms $S = \frac{ar^n}{1 - r}\;\;\therefore \frac{t_n}{S} = \frac{1 - r}{r}$. Hence proven.
\item $S_1 = \frac{1}{1 - \frac{1}{2}} = 2, S_2 = \frac{2}{1 - \frac{1}{3}} = 3, S_3 = \frac{3}{1 -
  \frac{1}{4}}, \ldots, S_p = \frac{p}{1 - \frac{1}{p + 1}} = p + 1$.

  Clearly, $S_1, S_2, \ldots, S_p$ forms an A.P. with $2$ as first term and $1$ as c.d.

  $S_1 + S_2 + \cdots + S_p = \frac{p}{2}[2.2 + (p - 1)] = \frac{p(p + 3)}{2}$.
\item $x = \frac{1}{1 - a} \Rightarrow a = 1 - \frac{1}{x} = \frac{x - 1}{x}$ and similarly $b = \frac{y -
  1}{y}$.

  $1 + ab + a^2b^2 + \cdots$ to $\infty = \frac{1}{1 - ab} = \frac{1}{1 - \frac{x - 1}{x}.\frac{y - 1}{y}} =
  \frac{xy}{x + y - 1}$.
\item Let $S$ be the sum, then $S = \frac{1}{1 - r} + \frac{a}{1 - r} + \frac{a^2}{1 - r} + \cdots$ to
  $\infty$

  $\Rightarrow S = \frac{a}{1 - r}.\frac{1}{1 - a} = \frac{a}{(1 - r)(1 - a)}$.
\item When the ball is dropped it will first travel $120$ mts. Then it will bounce back $120.\frac{4}{5} =
  96$ m and fall $96$ m as well. It will then bounce back $96.\frac{4}{5}$ m and fall the same distance as
  well.

  Thus, total distance travelled $120 + 120\times2\times\frac{4}{5} + 120\times2\times\frac{4^2}{5^2} +
  \cdots$ to $\infty$

  $= 120 + 192 \left[1 + \frac{4}{5} + \frac{4^2}{5^2} + \cdots\right]$ to $\infty = 120 + 192.\frac{1}{1 -
    \frac{4}{5}} = 120 + 960 = 1080$ meters.
\item Let $r$ be the common ratio. Then $b = ar^{n - 1} \Rightarrow (ab)^n = a^{2n}r^{n(n - 1)}$

  $p = a.ar.ar^2.ar^3.\ldots ar^{n - 1} = a^nr^{1 + 2 + 3 + \cdots + (n - 1)} = a^nr^{\tfrac{n(n - 1)}{2}}$

  $\Rightarrow p^2 = (ab)^n$.
\item Let the first terms are $a$ and $b$; and the common ratio is $r$. Ratio of sums would be $a:b$ which
  is equal to $ar^{n -1}:br^{n -1}$ i.e. ratio of $n$th terms.
\item Let $a$ be the first term. Then, $S_1 = \frac{a(r^n - 1)}{r - 1}, S_2 = \frac{a(r^{2n} - 1)}{r - 1}$
  and $S_3 = \frac{a(r^{3n} - 1)}{r - 1}$.

  $S_2 - S_1 = \frac{a(r^{2n} - r^n)}{r - 1} = \frac{ar^n(r^n - 1)}{r - 1}$

  $S_1(S_3 - S_2) = \frac{a(r^n - 1)}{r - 1}\left(\frac{ar^{2n}(r^n - 1)}{r - 1}\right) =
  \frac{a^2r^{2n}(r^n - 1)^2}{(r - 1)^2}$

  $\Rightarrow S_1(S_3 - S_2) = (S_2 - S_1)^2$.
\item $S_1 = a, S_2 = \frac{a(r^2 - 1)}{r - 1}, S_2 = \frac{a(r^3 - 1)}{r - 1}, \cdots, S_{2n - 1} =
  \frac{a(r^{2n - 1} - 1)}{r - 1}$

  $S_1 + S_2 + S_3 + \cdots + S_{2n - 1} = \frac{a}{r - 1}\left[r + r^2 + r^3 + \cdots + r^{2n - 1} -(1 + 1
    + \cdots + \mathrm{to}\;2n - 1 \mathrm{terms})\right]$

  $= \frac{a}{r - 1}\left[\frac{r(r^{2n - 1} - 1)}{r - 1} - (2n - 1)\right]$.
\item Given, $S_n = a.2^n - b; t_n = S_n - S_{n- 1} = a.2^n - b - a.2^{n - 1} + b = a.2^{n - 1}; r =
  \frac{t_n}{t_{n - 1}} = 2$ which is a constant independent of $n$ hence the given series is in G.P.
\item Given $x\geq 0 \therefore \frac{2x}{1 + x^2} < 1$ therefore we can apply the sum formula of a G.P. for
  infinite terms.

  Let $S$ be the required sum, then $S = \frac{1}{1 + x^2}.\frac{1}{1 - \frac{2x}{1 + x^2}} = \frac{1}{(1 -
    x)^2}$.
\item Let $a$ be the first term and $r$ be the common ratio. Then given, $a + ar = 24$ and $S_\infty =
  \frac{a}{1 - r} = 32$

  $a = \frac{24}{1 + r}$ and $a = 32(1 - r) \Rightarrow 1 - r^2 = \frac{24}{32} = \frac{3}{4} \Rightarrow r
  = \pm \frac{1}{2}$

  If $r = \frac{1}{2}$ then series is $16, 8, 4, \ldots$. If $r = -\frac{1}{2}$ then series is $48, -24, 12,
  -6, \ldots$.
\item Let $a$ be the first term and $r$ be the common ratio. Sum of this G.P. $\frac{a}{1 - r} = 4$ and sum
  of squares of terms $\frac{a^2}{1 - r^2} = \frac{16}{3}$.

  $\Rightarrow \frac{16(1 - r)^2}{1 - r^2} = \frac{16}{3}\Rightarrow \frac{1 - r}{1 + r} = \frac{1}{3}
  \Rightarrow r = \frac{1}{2} \Rightarrow a = 2$. So the G.P. is $2, 1, \frac{1}{2}, \frac{1}{4}, \ldots$.
\item $p(x) = \tfrac{\tfrac{x^{2n} - 1}{x^2 - 1}}{\tfrac{x^n - 1}{x - 1}} = \frac{x^n + 1}{x + 1}$ so clearly
  $n$ is an odd number for $p(x)$ to be a polynomial in $x$.
\item $x = a + \frac{a}{r} + \frac{a}{r^2} + \cdots$ to $\infty = \frac{a}{1 - \frac{1}{r}} = \frac{ar}{r -
  1}$. Similarly $y = \frac{br}{r + 1}$ and $z = \frac{cr^2}{r^2 - 1},\;\therefore \frac{xy}{z} =
  \frac{ab}{c}$.
\item Let $a$ be the first term, $r$ be the common ratio and $2n$ be the no. of terms. Then sum of all terms
  $S = \frac{a(r^{2n} - 1)}{r - 1}$ and sum of odd terms $S_{\mathrm{odd}} = \frac{a(r^{2n} - 1)}{r^2 - 1}$.

  Given, $S = 5S_{\mathrm{odd}}\Rightarrow r = 4$.
\item $S_n = 3 - \frac{3^{n + 1}}{4^{2n}} \Rightarrow t_n = S_n - S_{n - 1} = \frac{3^n}{4^{2(n - 1)}} -
  \frac{3^{n + 1}}{4^{2n}} = \frac{16.3^n - 3^{n + 1}}{4^{2n}} = \frac{3^n.13}{4^{2n}}$.

  $\Rightarrow r = \frac{t_n}{t_{n - 1}} = \frac{3}{16}$.
\item Let $a$ be the first term and $r$ be the common ratio; then $t_n = ar^{n -1}$. Let the sum of all
  terms succeeding $t_n$ be $S$. Then $S = \frac{ar^n}{1 - r}$.

  $\frac{t_r}{S} = \frac{1 - r}{r}$. If $\frac{1 - r}{r} > 1$ then $r < \frac{1}{2}$, if $\frac{1 - r}{r} =
  1$ then $r = \frac{1}{2}$ and $\frac{1 - r}{r} < 1$ then $r > \frac{1}{2}$.
\item $666\ldots n\;\mathrm{digits} = \frac{6}{9}(10^n - 1) = \frac{2}{3}(10^n - 1)$.

  $888\ldots n\;\mathrm{digits} = \frac{8}{9}(10^n - 1). \Rightarrow$ L.H.S. $= \frac{4}{9}(10^{2n} - 2.10^n + 1 -
  2.10^n -2) = \frac{4}{9}(10^{2n} - 1)$

  R.H.S. $= 444\ldots 2n\;\mathrm{digits} = \frac{4}{9}(10^{2n} - 1) =$ L.H.S.
\item Let $S = (x + y) + (x^2 + xy + y^2) + (x^3 + x^2y + xy^2 + y^3) + \cdots$ to $n$ terms

  $S = \frac{1}{x - y}[(x^2 - y^2) + (x^3 - y^3) + (x^4 + y^4) + \cdots]$ to $n$ terms

  $= \frac{1}{x - y}\left[\frac{x^2(x^n - 1)}{x - 1} - \frac{y^2(y^n - 1)}{y - 1}\right]$.
\item $S = \frac{1}{1 - r}\Rightarrow r = \frac{S - 1}{S}$. Let $S' =
  \displaystyle\sum_{n=0}^{\infty}r^{2n}$ then
  $S' = \frac{1}{1 - r^2} = \frac{S^2}{2S - 1}$.
\item Let $a$ be the first term and $r$ be the common ratio. Then $t_m = ar^{m - 1} = \frac{1}{n^2}$ and
  $t_n = ar^{n - 1} = \frac{1}{m^2} \Rightarrow \frac{t_m}{t_n} = r^{m - n} = \frac{m^2}{n^2} \Rightarrow r
  = \sqrt[m - n]{\frac{m^2}{n^2}}$.

  $ar^{m - 1} = \frac{1}{n^2} \Rightarrow a = \frac{1}{n^2}\left(\frac{n^2}{m^2}\right)^{\tfrac{m - 1}{m -
      n}}$

  $\Rightarrow t_{\tfrac{m + n}{2}} = ar^{\tfrac{m + n - 2}{2}} =
  \frac{1}{n^2}\left(\frac{n^2}{m^2}\right)^{\tfrac{m - 1}{m - n}}.\left(\frac{m^2}{n^2}\right)^{\tfrac{m +
      n -2}{2(m - n)}} = \frac{1}{mn}$.

  This can be alternatively computed with G.M. formula i.e. $t_{\tfrac{m + n}{2}} = \sqrt{t_mt_n} =
  \frac{1}{mn}$.
\item Given condition is $c > 4b - 3a \Rightarrow c - 4b + 3a > 0 \Rightarrow r^2 - 4r + 3 < 0\;[\because a
  < 0] \Rightarrow r> 3$ or $r < 1$.
\item Given, $(1 - k)(1 + 2x + 4x^2 + 8x^3 + 16x^4 + 32x^5) = 1 - k^6 \Rightarrow (1 - k)\frac{64x^6 - 1}{x
  - 1} = 1 - k^6 \Rightarrow k = 2x \Rightarrow \frac{k}{x} = 2$.
\item Given, $(a^2 + b^2 + c^2)(b^2 + c^2 + d^2)\leq(ab + bc + cd)^2 \Rightarrow (b^2 - ac)^2 + (c^2 - ad)^2
  + (ad - bc)^2 \leq 0$

  Since $a, b, c, d$ are non-zero real numbers therefore the above conditiion leads to equality if and only
  if $b^2 = ac, c^2 = ad, ad = bc$ i.e. $a, b, c, d$ are in G.P.
\item This problem is generalization of previous problem and can be solved similarly.
\item Let $r$ be the common ratio, then $\beta = \alpha r, \gamma = \alpha r^2, \delta = \alpha r^3$.

  From roots of quadratic equaiton $\alpha + \beta = 3, \alpha\beta = a, \gamma + \delta = 12, \gamma\delta
  = b$

  $\frac{\gamma + \delta}{\alpha + \beta} = r^2 = 4 \Rightarrow r = 2$ because G.P. is increasing so we
  discard the negative root.

  $\Rightarrow \alpha = 1 \Rightarrow a = 2, \Rightarrow b = 32$.
\item Let $a$ be the first term of the A.P. Then $t_{2n + 1} = a + 4n$. So the first term of the G.P. is $a
  + 4n$.

  Middle term of A.P. $t_{n + 1} = a + 2n$ and middle term of G.P. $= \frac{a + 4n}{2^n}$

  Given, $a + 2n = \frac{a + 4n}{2^n}$ thus, $a$ can found and hence $a + 4n$ which is the mid term can be
  deduced.
\item $f(x) = 2x + 1, f(2x) = 4x + 1, f(4x) = 8x + 1$. Given that $f(x), f(2x), f(4x)$ are in G.P.

  $\Rightarrow \frac{f(2x)}{f(x)} = \frac{f(4x)}{f(2x)} \Rightarrow (4x + 1)^2 = (2x + 1)(8x + 1)
  \Rightarrow 8x + 1 = 10x + 1 \Rightarrow x = 0$.
\item Let $r$ be the common ratio then $a + b + c = xb \Rightarrow 1 + r + r^2 = xr \Rightarrow x = \frac{1
  + r + r^2}{r} = \frac{1}{r} + 1 + r$. We know that if $r > 0, r + \frac{1}{r} > 2 \Rightarrow x > 3$ and
  if $r < 0, r + \frac{1}{r} < -2 \Rightarrow x < -1$.
\item $x = \frac{1}{1 - a}, y = \frac{1}{1 - b}, z = \frac{1}{1 - c} \Rightarrow \frac{1}{x} = 1 - a,
  \frac{1}{y} = 1 - b, \frac{1}{z} = 1 - c$

  Thus, $\because a, b, c$ are in A.P. where $|a|, |b|, |c| < 1\;\therefore x, y, z$ are also in A.P.
\item $p = \frac{1}{1 + \tan^2x} = \cos^2x; q = \frac{1}{1 + \cot^2y} = \sin^2y$

  $\displaystyle\sum_{k = 0}^{\infty}\tan^{2k}x\cot^{2k}y = \frac{1}{1 - \tan^2x\cot^2y}$

  $\frac{1}{\frac{1}{p} + \frac{1}{q} - \frac{1}{pq}} = \frac{\cos^2x\sin^2y}{\cos^2x + \sin^2y - 1}$

  Dividing numerator and denominator with $\cos^2x\sin^2y$, we get

  $= \frac{1}{\csc^2y + \sec^2x - \csc^2y\sec^2x} = \frac{1}{\tan^2x + \cot^2y + 2 - 1 -\tan^2x - cot^2y
    - \tan^2xcot^2y} = \displaystyle\sum_{k = 0}^{\infty}\tan^{2k}x\cot^{2k}y$.
\item We know that area of an equilateral triangle is $\frac{\sqrt{3}}{4}a^2$, where $a$ is one of the
  sides. In this case $\Delta = \frac{3}{4}$.

  Now the area of sides joining mid-point will have side $\frac{a}{2}$ and terefore area will be
  $\frac{1}{4}$th of the original triangle. This ratio of $\frac{1}{4}$ will continue and areas of all
  triangles will form a G.P. with common ratio of $\frac{1}{4}$. Thus sum of areas of all these triangles $=
  \frac{\tfrac{3}{4}}{1 - \tfrac{1}{4}} = 1$.
\item $1 + |\cos x| + |\cos^2x| + |\cos^3x| + \cdots$ to $\infty = \frac{1}{1 - |\cos x|} = p$(let).

  $\Rightarrow e^{p.\log_e4} = 4^p$. Now given equation is $t^2 - 20t + 64 = 0 \Rightarrow t = 4, 16
  \Rightarrow p = 1, 2 \Rightarrow |\cos x| = 0, 1/2 \Rightarrow x = \pi/2, \pi/3, 2\pi/3$.
\item $1 + |\cos x| + |\cos^2x| + |\cos^3x| + \cdots$ to $\infty = \frac{1}{1 - |\cos x|} \Rightarrow
  \frac{1}{1 - |\cos x|} = 2 \Rightarrow |\cos x| = \frac{1}{2}\Rightarrow \cos x = \pm\frac{1}{2}
  \Rightarrow S = \left\{\frac{\pi}{3}, \frac{2\pi}{3}\right\}$.
\item $\sin^2x + \sin^4x + \cdots$ to $\infty = \frac{\sin^2x}{1 - \sin^2x} = \tan^2x$

  Roots of $x^2 - 9x + 8 = 0$ are $1, 8$ i.e. $2^0, 2^3 \Rightarrow \tan x = 0, \sqrt{3}$ (rejecting
  $-\sqrt{3}$ as for $0< x< \frac{\pi}{2}, \tan x$  cannot be negative.)

  $\frac{\cos x}{\cos x + \sin x} = \frac{1}{1 + \tan x} = 1, \frac{1}{1 + \sqrt{3}}$.
\item $S_\lambda = \frac{\lambda}{\lambda - 1}$ [Hint: It is a G.P.]
  $\displaystyle\sum_{\lambda=1}^n(\lambda - 1)S_\lambda = \sum_{\lambda=1}^n\lambda = \frac{n(n + 1)}{2}$.
\item Let $2^{ax + 1}, 2^{bx + 1}, 2^{cx + 1}$ are in G.P. $\Rightarrow \frac{2^{bx + 1}}{2^{ax + 1}} =
  \frac{2^{cx + 1}}{2^{bx + 1}} \Rightarrow (b - a)x = (c - b)x \Rightarrow b - a = c - b$

  which implies that $a, b, c$ are in A.P. which is a given and hence we have proven required condition in
  reverse.
\item Given $\frac{a + be^x}{a - be^x} = \frac{b + ce^x}{b - ce^x} \Rightarrow ab - ace^x + b^2e^x -
  bce^{2x} = ab + ace^x - b^2e^x - bce^{2x} \Rightarrow 2ace^x = b^2e^x \Rightarrow 2ac = b^2$, which
  implies $a, b, c$ are in G.P. Similarly it can be proven that $b, c, d$ are in G.P. making $a, b, c, d$
  are in G.P.
\item Given, $2\tan^{-1}y = \tan^{-1}x + \tan^{-1}z \Rightarrow \frac{2y}{1 - y^2} = \frac{x + z}{1 - zx}$

  But we are also given that $y^2 = zx \Rightarrow 2y = x + z \Rightarrow x, y, z$ are in A.P. Now $4y^2 =
  (x + z)^2 = 2(x + z) \Rightarrow x = z = y$ but the common values are not necessarily $0$.
\item Given, $b - c = a - b\;[\because a, b, c$ are in A.P.$]$. From second condition $(c - b)^2 = (b - a)a
  \Rightarrow (a - b)^2 = (a - b)a \Rightarrow 2a = b \Rightarrow 3a = c \Rightarrow a:b:c = 1:2:3$.
\item Since $a, b, c$ are in G.P. $\Rightarrow b^2 = ac$. From second condition, $2(\log 2b - \log 3c) =
  \log 3c - \log 2b \Rightarrow 3\log 2b = 3\log 3c \Rightarrow 2b = 3c \Rightarrow b = \frac{2a}{3}, c =
  \frac{4a}{9}$. Clearly, $a$ is the greatest side. Using $\cos$ rule,

  $\cos A = \frac{b^2 + c^2 - a^2}{2bc} = -\frac{1}{2}$ and thus $A > 90^\circ$ making the triangle
  obtuse-angled triangle.
\item Let $\alpha, \beta, \gamma$ are the roots. Then $\alpha + \beta + \gamma = -\frac{b}{c}, \alpha\beta +
  \beta\gamma + \gamma\alpha = \frac{c}{a}, \alpha\beta\gamma = -\frac{d}{a}$. Let $r$ be the common ratio
  of the G.P. then $\beta = \alpha r, \gamma = \alpha r^2$. Also let $\alpha = x$.

  $\frac{c^3}{b^3} = \frac{c^3}{a^3}.\frac{a^3}{b^3} = -\frac{(\alpha\beta + \beta\gamma +
    \gamma\alpha)^3}{(\alpha + \beta + \gamma)^3} = -\left(\frac{x^2r + x^2r^3 + x^2r^2}{x + xr +
    xr^2}\right)^3 = -x^3r^3 = -\alpha\beta\gamma = \frac{d}{a} \Rightarrow c^3a = b^3d$.
\item Clearly $t_n = \frac{1}{2n - 1} \Rightarrow t_{100} = \frac{1}{199}$.
\item The corresponding $p$th and $q$th term in the A.P.would be $\frac{1}{qr}$ and $\frac{1}{rp}$. Let $a$
  be the first term and $d$ be the commond difference of this A.P. Then, $a + (p - 1)d = \frac{1}{qr}$ and
  $a + (q - 1)d = \frac{1}{rp}$. Subtracting $(p - q)d = \frac{p - q}{pqr} \Rightarrow d = \frac{1}{pqr}$.

  $\Rightarrow a = \frac{1}{qr} - \frac{p - 1}{pqr} = \frac{1}{pqr}. \Rightarrow t_r = \frac{1}{pqr} +
  \frac{r - 1}{pqr} = \frac{1}{pq}$. Therefore $r$th term in H.P. would be $pq$.
\item Corrsponding $p$th, $q$th and $r$th term of the A.P. would be $\frac{1}{a}, \frac{1}{b}$ and
  $\frac{1}{c}$. Let $x$ be the first term and $y$ be the c.d. of this A.P. Then,

  $x + (p - 1)y = \frac{1}{a}, x + (q - 1)y = \frac{1}{b}, x + (r - 1)y = \frac{1}{c}$

  $(p - q)y = \frac{b - a}{ab}\Rightarrow (p - q)ab = \frac{b - a}{y}$. Similarly, $(q - r)bc = \frac{c -
  b}{y}$ and $(r - p)ca = \frac{c - a}{y}$. Clearly, $(q - r)bc + (r - p)ca + (p - q)ab = 0$.
\item We have to prove that $\frac{a - b}{b - c} = \frac{a}{c}\Rightarrow ac - bc = ab - ac \Rightarrow 2ac
  = ab + bc$ which prove that $a, b, c$ are in H.P. Thus required equality is proven in reverse.
\item Given $\frac{1}{a}, \frac{1}{b}, \frac{1}{c}, \frac{1}{d}$ are in A.P. Let $p$ be the c.d. of this
  A.P. $\Rightarrow \frac{1}{b} - \frac{1}{a} = p \Rightarrow ab = \frac{a - b}{p}$. Similarly, $bc =
  \frac{b - c}{p}, cd = \frac{c - d}{p}$. Adding these we have $ab + bc + cd = \frac{a - d}{p}$. Now
  $\frac{1}{d} - \frac{1}{a} = 3p \Rightarrow 3ad = \frac{a - d}{p}$. Thus, $ab + bc + cd = 3ad$.
\item Let $d$ be the common difference of the corresponding $A.P.$ Then, $\frac{1}{x_n} - \frac{1}{x_1} = (n
  - 1)d \Rightarrow \frac{x_1 - x_n}{d} = (n - 1)x_1x_n =$ R.H.S.

  Now, $\frac{1}{x_1} - \frac{1}{x_2} = d \Rightarrow \frac{x_1 - x_2}{d} = x_1x_2$. Similarly, $\frac{x_2
    - x_3}{d} = x_2x_3$ and so on till $\frac{x_{n - 1} - x_n}{d} = x_{n - 1}x_n$. Adding these and
  comparing with R.H.S. we get the required equality.
\item $\frac{1}{a}, \frac{1}{b}, \frac{1}{c}$ are in A.P.

  $\Rightarrow \frac{a + b + c}{a}, \frac{a + b + c}{b}, \frac{a + b + c}{c}$ are in A.P.

  $\Rightarrow \frac{a + b + c}{a} -1, \frac{a + b + c}{b} - 1, \frac{a + b + c}{c} -1$ are in A.P.

  $\Rightarrow \frac{a}{b + c}, \frac{b}{c + a}, \frac{c}{a + b}$ are in H.P.
\item $a^2, b^2, c^2$ are in A.P. $\Rightarrow a^2 + ab + bc + ca, b^2 + ab + bc + ca, c^2 + ab + bc + ca$
  are in A.P.

  $\Rightarrow (a + b)(c + a), (b + c)(a + b), (c + a)(b + c)$ are in A.P.

  Dividing each term by $(a + b)(b + c)(c + a)$, we have

  $\frac{1}{b + c}, \frac{1}{c + a}, \frac{1}{a + b}$ are in A.P.

  $\Rightarrow b + c, c + a, a + b$ are in H.P.
\item If $t_n = \frac{1}{3n - 2}$ then the sequence is $1, \frac{1}{4}, \frac{1}{7}, \frac{1}{10}, \cdots$

  Let us assume that it is in H.P. then corresponding $n$th term in A.P. is $3n - 2$. Thus, c.d. $= 3n - 2 -
  (3n - 1) - 2 = 3$ which is a constant so the sequence is in A.P. Thus our assumption is correct and given
  sequence is in H.P.
\item Let $a$ be the first term and $d$ be the c.d. of the corresponding A.P. Then,

  $a + (m - 1)d = \frac{1}{n}$ and $a + (n - 1)d = \frac{1}{m}$. Subtracting, $(m - n)d = \frac{m -
  n}{mn}\Rightarrow d = \frac{1}{mn} \Rightarrow a = \frac{1}{n} - \frac{m - 1}{mn} = \frac{1}{mn}$.

  Then $t_{m + n} = \frac{1}{mn} + (m + n - 1)\frac{1}{mn} = \frac{m + n}{mn}$ thus corrsponding term in
  H.P. would be $\frac{mn}{m + n}$. Also, $t_{mn} = \frac{1}{mn} + \frac{mn - 1}{mn} = 1$ and hence
  corresponding term in H.P. is $1$.
\item Let the three numbers in H.P. are $a, b, c$ then $\frac{1}{a}, \frac{1}{b}, \frac{1}{c}$ will be in
  A.P. Given, $a + b + c = 37, \frac{1}{a} + \frac{1}{b} + \frac{1}{c} = \frac{1}{4}$. Let $d$ be the
  c.d. of the A.P. then $\frac{3}{b} = \frac{1}{4} \Rightarrow b = 12$

  $\Rightarrow \frac{12}{1 - 12d} + 12 + \frac{12}{1 + 12d} = 37 \Rightarrow d = \frac{1}{60}$. So the
  numbers are $15, 12, 10$.
\item $\because a, b, c$ are in H.P. $\therefore b = \frac{2ac}{a + c}$.

  L.H.S. $= \frac{1}{b - a} + \frac{1}{b - c} = \frac{a + c}{ac - a^2} + \frac{a + c}{ac - c^2} = \frac{a + c}{ac} =
  \frac{1}{a} + \frac{1}{c} =$ R.H.S.
\item $\because a, b, c$ are in H.P. $\therefore b = \frac{2ac}{a + c}$.

  L.H.S. $= \frac{b + a}{b - a} + \frac{b + c}{b - c} = \frac{a^2 + 3ac}{ac - a^2} + \frac{c^2 + 3ac}{ac -
    c^2} = \frac{3ac^2 + a^2c - 3a^2c - ac^2}{ac(c - a)} = \frac{2ac^2 - 2a^2c}{ac(c - a)} = 2 =$ R.H.S.
\item Let $d$ be the c.d. of corresponding A.P., then $\frac{1}{x_2} - \frac{1}{x_1} = d \Rightarrow x_1x_2
  = \frac{x_1 - x_2}{d}$ and similarly, $x_2x_3 = \frac{x_2 - x_3}{d}, x_3x_4 = \frac{x_3 - x_4}{d}, x_4x_5
  = \frac{x_4 - x_5}{d}$.

  Adding toegther, $\frac{x_1 - x_5}{d} = x_1x_2 + x_2x_3 + x_3x_4 + x_4x_5 =
  \frac{x_1x_5}{d}\left[\frac{1}{x_1} - \frac{1}{x_5}\right] = 4x_1x_5$. Hence proved.
\item Like previous problem $x_1 - x_3 = 2x_1x_3d$ and $x_2 - x_4 = 2x_2x_4d$ so L.H.S. $= 4x_1x_2x_3x_4d^2$

  And $x_1 - x_2 = x_1x_2d$ and $x_3 - x_4 = x_3x_4d$ so R.H.S. $= 4x_1x_2x_3x_4d^2$ and thus L.H.S. =
  R.H.S.
\item Given $\frac{1}{b + c}, \frac{1}{c + a}, \frac{1}{a + b}$ are in A.P.

  Multiplying with $a + b + c$ and then subtracting $1$ from each term we get required condition.
\item Given $\frac{1}{b + c}, \frac{1}{c + a}, \frac{1}{a + b}$ are in A.P.

  Multiplying each term with $a + b+ c$ and then subtracting $ab + bc + ca$ from each term we get the
  required condition.
\item Given that $a, b, c$ are in A.P. Dividing each term by $abc$, we get that $\frac{1}{bc}, \frac{1}{ca},
  \frac{1}{ab}$ are in A.P. Multiplying each term with $ab + bc + ca$ and then subtracting $1$ from each
  term we get the desired condition.
\item Given that $\frac{1}{a}, \frac{1}{b}, \frac{1}{c}$ are in A.P. Multiplying each term with $a + b + c$
  and then subtracting $2$ from each term we get the desired condition.
\item Given that $\frac{1}{a}, \frac{1}{b}, \frac{1}{c}$ are in A.P. Multiplying each term with $a + b + c$
  and then subtracting $1$ from each term we get the desired condition.
\item Let $d$ be the c.d. of the A.P. and $r$ be the common ratio of the G.P.

  $\Rightarrow b - c = -d, c - a = 2d, a - b = -d$ and $y = xr, z = xr^2$.

  L.H.S. $= x^{b - c}y^{c - a}z^{a - b} = x^{-d}(xr)^{2d}(xr^2)^{-d} = x^0y^0 = 1$.
\item Let $a$ be the first term and $d$ be the c.d. of the A.P. Then,

  $\frac{a + (q - 1)d}{a + (p - 1)d} = \frac{a + (r - 1)d}{a + (q - 1)d} = \frac{a + (s - 1)d}{a + (r -
  1)d}$

  $\Rightarrow \frac{[a + (q - 1)d] - [a + (r - 1)d]}{[a + (p - 1)d] - [a + (q - 1)d]} = \frac{[a + (r -
    1)d] - [a + (s - 1)d]}{[a + (q - 1)d] - [a + (r - 1)d]}$

  $\Rightarrow \frac{q - r}{p - q} = \frac{r - s}{q - r}$ which proves the required condition.
\item Let $x$ be the first term and $d$ be the c.d. of the A.P. Then $a = x + (p - 1)d, b = x + (q - 1)d, c
  = x + (r - 1)d$

  $\Rightarrow b - c = (q - r)d, c - a = (r - p)d$ and $a - b = (p - q)d$

  Also let $m$ be the first term and $n$ be the common ratio of the G.P. Then $a = mn^{p - 1}, b = mn^{q - 1}, c = mn^{r
    -1}$

  L.H.S. $= a^{b - c}b^{c - a}c^{a - b} = (mn^{p - 1})^{(q - r)d}(mn^{q - 1})^{(r - p)d}(mn^{r - 1})^{(p -
    q)d} = m^0n^0 = 1 =$ R.H.S.
\item Given, $a, b, c$ are in A.P. $\Rightarrow 2b = a + c$ and $b, c, d$ are in H.P. $\Rightarrow c =
  \frac{2bd}{b + d}$

  $\Rightarrow bc = \frac{a + c}{2}\frac{2bd}{b + d} = \frac{(a + c)bd}{b + d} \Rightarrow b^2c + bcd = abbd
  + bcd \Rightarrow bc = ad$.
\item Given $a^x = b^y = c^z = p$(let) $\Rightarrow a = p^{\tfrac{1}{x}}, b = p^{\tfrac{1}{y}}, c =
  p^{\frac{1}{z}}$.

  Also given, $a, b, c$ are in G.P. $\Rightarrow \frac{b}{a} = \frac{c}{b} \Rightarrow p^{\tfrac{1}{y} -
    \tfrac{1}{x}} = p^{\tfrac{1}{z} - \tfrac{1}{y}} \Rightarrow \frac{1}{y} - \frac{1}{x} = \frac{1}{z} -
  \frac{1}{y}$

  $\therefore \frac{1}{x}, \frac{1}{y}, \frac{1}{z}$ are in H.P.
\item $\because \frac{x + y}{2}, y, \frac{y + z}{2}$ are in H.P. $\therefore y = \frac{2\left(\tfrac{x +
    y}{2}\right)\left(\tfrac{y + z}{2}\right)}{\tfrac{x + y}{2} + \tfrac{y + z}{2}}$

  $\Rightarrow xy + 2y^2 + yz = xy + y^2 + zx + yz \Rightarrow y^2 = zx \Rightarrow a, b, c$ are in G.P.
\item $\because x, y, z$ are in G.P. $\therefore y^2 = zx$. Also, $x + a, y + a, z + a$ are in
  H.P. $\Rightarrow y + a = \frac{2(x + a)(z + a)}{x + a + z + a} \Rightarrow xy + yz + 2ay + ax + az + 2a^2
  = 2(zx + az + ax + a^2)\Rightarrow (y - a)(x + z - 2y)$

  But $x + z - 2y \neq 0$ else $x + z = 2y$ i.e. $x, y, z$ are in A.P. $\Rightarrow x = y = z \therefore y =
  a$.
\item $\because a, b, c$ are in A.P., G.P. and H.P. $\therefore 2b = a + c, b^2 = ac, b = \frac{2ac}{a + c}
  \Rightarrow \left(\frac{a + c}{2}\right)^2 = ac \Rightarrow (a + c)^2 = 4ac \Rightarrow a = c = b$.
\item $\because a, b, c$ are in A.P. $\Rightarrow 2b = a + c$. $\because b, c, d$ are in G.P. $\therefore
  c^2 = bd$. $\because c, d, e$ are in H.P. $\therefore d = \frac{2ce}{c + e}$.

  $c^2 = bd = \frac{a + c}{2}.\frac{2ce}{c + e} \Rightarrow c(c + e) = (a + c)e \Rightarrow c^2 = ae
  \Rightarrow a, c, e$ are in G.P.
\item $\because a, b, c$ are in A.P. $\therefore 2b = a + c$. $\because a^2, b^2, c^2$ are in
  H.P. $\therefore b^2 = \frac{2a^2c^2}{a^2 + c^2}$

  $\Rightarrow \left(\frac{a + c}{2}\right)^2 = \frac{2a^2c^2}{a^2 + c^2} \Rightarrow (a^2 + c^2)(a + c)^2 =
  8a^2c^2 \Rightarrow (a - c)^2[(a + c)^2 + 2ac] = 0$

  If $(a - c)^2 = 0 \Rightarrow a = c \Rightarrow a = b = c$ else $(a + c)^2 + 2ac = 0 \Rightarrow ac =
  -2b^2 \Rightarrow b^2 = -\frac{a}{2}.c \Rightarrow -\frac{a}{2}, b, c$ are in G.P.
\item  $a^bb^cc^a = a^cb^ac^b  \Rightarrow a^{b - c}b^{c - a}c^{a - b} = 1$ which has been proved
  previously.
\item Let $a$ be the first terms of both the A.P. and G.P. $d$ be c.d. of the A.P. and $r$ be the common
  ratio of the G.P. Given,

  $a + a = 1 \Rightarrow a = \frac{1}{2}, a + d + ar = \frac{1}{2} \Rightarrow d = -ar \Rightarrow 2d = -r$
  and $a + 2d + ar^2 = 2 \Rightarrow -r + \frac{r^2}{2} = \frac{3}{2} \Rightarrow r^2 -2r + 3 = 0$. Now $r$
  and sum of fourth term can be easily found.
\item $\because p, q, r$ are in A.P. $\therefore 2q = p + r$. Also, let $\frac{a - x}{px} = \frac{a - y}{qy}
  = \frac{a - z}{rz} = f$

  $\therefore p = \frac{a}{fx} - \frac{1}{f}, q = \frac{a}{fy} - \frac{1}{f}, r = \frac{a}{fz}-
  \frac{1}{f}$. Substituting these in $2q = p + r$

  $\frac{2a}{fy} - \frac{2}{f} = \frac{a}{fx} - \frac{1}{f} + \frac{a}{fy} - \frac{1}{f} \Rightarrow
  \frac{2}{y} = \frac{1}{x} + \frac{1}{z} \Rightarrow x, y, z$ are in H.P.
\item Let $d$ be c.d. of the A.P. and $d'$ be the c.d. of the A.P. corrsponding to H.P. then,

  $b = a + (n - 1)d$ and $\frac{1}{b} = \frac{1}{a} + (n - 1)d'$ $\Rightarrow d = \frac{b - a}{n - 1}, d' =
  \frac{a - b}{ab(n - 1)}$

  Product of the $r$th term of the A.P. and $(n - r + 1)$ th term of the H.P. $= \left[a + (r -
    1)\frac{b - a}{n - 1}\right].\frac{1}{\frac{1}{a} + (n - r).\frac{a - b}{ab(n - 1)}} = ab$.
\item Let $a, b, c$ be three consecutive terms of an H.P. then $b = \frac{2ac}{a + c}$.

  Terms after subtraction will be $a - \frac{b}{2}, \frac{b}{2}, c - \frac{b}{2}$. The condition for these
  to be in G.P. is $b^2 = (2a - b)(2c - b) = 4ac - 2b(a + c) + b^2 \Rightarrow b = \frac{2ac}{a + c}$ which
  is a given.
\item $\because y - x, 2(y - a), y - z$ are in H.P. $\therefore \frac{1}{2(y - z)} - \frac{1}{y - x} =
  \frac{1}{y - z} - \frac{1}{2(y - a)} = \frac{2a - y - z}{(y - x)} = \frac{y + z - 2a}{y - z}$

  $= \frac{(x - a) + (y - a)}{(x - a) - (y - a)} = \frac{(y - a) + (z - a)}{(y - a) - (z - a)} = \frac{x -
    a}{y - a} = \frac{y - a}{z - a}$
  Hence, $x - a, y - a, z - a$ are in G.P.
\item From given conditions we have $2b = a + c, q = \frac{2pr}{p + r}$ and $b^2q^2 = acpr$. Substituting
  the values of $b$ and $q$ in third equations, we arrive at

  $\left[\left(\frac{a + c}{2}\right)^2\left(\frac{2pr}{p + r}\right)^2\right] = acpr = \frac{(a + c)^2}{(r
    + p)^2}.p^2r^2 \Rightarrow \frac{pr}{(r + p)^2} = \frac{ac}{(a + c)^2}$

  $\Rightarrow \frac{(r + p)^2}{pr} = \frac{(a + c)^2}{ac} \Rightarrow \frac{p}{r} + \frac{r}{p} =
  \frac{a}{c} + \frac{c}{a}$.
\item From given conditions we have, $2b = a + x, b^2 = ay$ and $\frac{2}{b} = \frac{1}{a} + \frac{1}{x}
  \Rightarrow x = 2b - a, y = \frac{b^2}{a}$ and $z = \frac{ab}{2a - b}$

  Now we can substitute in the required result and prove the equality.
\item From given equations $2 = x + z$ and $4 = zx$, we have to prove that $4 = \frac{2zx}{x +
  z}$. Substituting the values from given conditions to required equality we find that equality holds.
\item Given that $t_n = 12n^2 - 6n + 5$ then $\displaystyle S_n = 12\sum_{i = 1}^ni^2 - 6\sum_{i=1}^ni + 5\sum_{i=1}^n1$

  $= 12.\frac{n(n + 1)(2n + 1)}{6} - 6\frac{n(n + 1)}{2} + 5n = n\left[4n^2 + 6n + 2 - 3n - 3 + 5\right]
  = n(4n^2 + 3n  + 4)$.
\item Clearly $t_n = (2n - 1)^2 = 4n^2 - 4n + 1 \Rightarrow \displaystyle S_n = 4\sum_{i =1}^ni^2 -
  4\sum_{i=1}^ni + \sum_{i=1}^n1$

  $= 4\frac{n(n + 1)(2n + 1)}{6}- 4\frac{n(n + 1)}{2} + n = n\left[\frac{4n^2 + 6n + 2 - 6n - 6 +
      3}{3}\right] = \frac{n(4n^2 -1)}{3}$.
\item Clearly, $t_n = n(n + 1)(n + 2) = n^3 + 3n^2 + 2n \Rightarrow \displaystyle S_n = \sum_{i= 1}^ni^3 +
  3\sum_{i=1}^n i^2 + \sum_{i=1}^ni$

  $= \left[\frac{n(n + 1)}{2}\right]^2 + 3.\frac{n(n + 1)(2n + 1)}{6} + \frac{n(n + 1)}{2} = \frac{n(n +
    1)}{2}\left[\frac{n(n + 1)}{2} + 2n + 1 + 1\right] = \frac{n(n + 1)}{2}.\frac{n^2 + 5n + 4}{2} =
  \frac{n(n + 1)^2(n + 4)}{4}$.
\item $r$th term of the series, $t_r = r(n - r + 1)\Rightarrow\displaystyle S_n = n\sum_{r=1}^nr -
  \sum_{r=1}^nr^2 + \sum_{r=1}^nr$

  $= \frac{n.n(n + 1)}{2} - \frac{n(n + 1)(2n + 1)}{6} + \frac{n(n + 1)}{2} = \frac{n(n + 1)}{2}\left[n -
    \frac{2n + 1}{3} + 1\right] = \frac{n(n + 1)}{2}\left[\frac{3n - 2n - 1 + 3}{3}\right] = \frac{n(n +
    1)(n + 2)}{6}$.
\item   If you see carefully this series is same as previous problem hence sum will be same.
  $t_n = 1 + 2 + 3 + \cdots + n = \frac{n^2 + n)}{2}\Rightarrow \displaystyle t_n =
  \frac{1}{2}\left[\sum_{i=1}^ni^2 + \sum_{i=1}^ni\right]$

  $=\frac{1}{2}\left[\frac{n(n + 1)(2n + 1)}{6} + \frac{n(n + 1)}{2}\right] = \frac{n(n +
    1)}{4}\left[\frac{2n + 1}{3} + 1\right] = \frac{n(n + 1)(n + 2)}{6}$.
\item First term contains $1$ integer, second term contains $2$ and so on. So before $t_n$ we will have $1 +
  2 + \cdots + (n - 1)$ integers i.e. $\frac{n(n - 1)}{2}$ integers. So $t_n$ will start with $\frac{n(n -
    1) + 2}{2}$ and will have $n$ integers. So $t_n = \frac{n^2 - n + 2}{2} $ and now it is trivial to find
  the sum, which will be $\displaystyle S_n = \frac{1}{2}\sum_{i=1}^ni^2 -
  \frac{1}{2}\sum_{i=1}^ni + \sum_{i=1}^n1 = \frac{n(n + 1)(2n + 1)}{12} - \frac{n(n + 1)}{2} + n$
  simplification is left to you.
\item Let $nt_n$ represent numerator and $dt_n$ be the denominator of the $n$th term $t_n$. Then $nt_n =
  \left[\frac{n(n + 1)}{2}\right]^3$ and $dt_n = \frac{n}{2}[2 + (n - 1)2] = n^2$

  $\displaystyle\Rightarrow t_n = \left(\frac{n + 1}{2}\right)^2 = \frac{n^2 + 2n + 1}{2}\Rightarrow S_n =
  \frac{1}{2}\sum_{i=1}^ni^2 + \sum_{i=1}^ni + \frac{1}{2}\sum_{i=1}^n1 = \frac{n(n + 1)(2n + 1)}{12} +
  \frac{n(n + 1)}{2} + \frac{n}{2}$. Simplify and put $n=16$ to arrive at the answer.
\item $t_n = [(2n + 1)^3 - (2n)^3] = 12n^2 + 6n + 1\Rightarrow\displaystyle S_n = 12\sum_{i=1}^ni^3 +
  6\sum_{i=1}^n i + \sum_{i=1}^n1 = 12\frac{n(n + 1)(2n + 1)}{6} + 6\frac{n(n + 1)}{2} + n = 2n(n + 1)(2n +
  1) + 3n(n + 1) + n$. Simplify and put $n=10$ to get the answer.
\item $t_1 = \frac{1}{1} - \frac{1}{2}, t_2 = \frac{1}{2} - \frac{1}{3} \cdots t_n = \frac{1}{n} -
  \frac{1}{n + 1}$. Adding $S_n = \frac{1}{1} - \frac{1}{n + 1} = \frac{n}{n + 1}$.
\item $t_n = \frac{1}{n(n + 1)(n + 2)} = \frac{1}{2}\left[\frac{1}{n(n + 1)} - \frac{1}{(n + 1)(n +
    2)}\right] = \frac{1}{2}\left[\frac{1}{n} - \frac{2}{n + 1} + \frac{1}{n + 2}\right]$

  Then, $t_1 = \frac{1}{2.1} - \frac{1}{2} + \frac{1}{2.3}, t_2 = \frac{1}{2.2} - \frac{1}{3} +
  \frac{1}{2.4}, t_3 = \frac{1}{2.3} - \frac{1}{4} + \frac{1}{2.5}, \ldots, t_{n - 2} = \frac{1}{2(n - 1)} -
  \frac{1}{n - 1} + \frac{1}{2n}, t_{n - 1} = \frac{1}{2(n - 1)} - \frac{1}{n} + \frac{1}{2(n + 1)}, t_n =
  \frac{1}{2.n} - \frac{1}{n + 1} + \frac{1}{2(n + 2)}$

  $\Rightarrow S_n = \frac{1}{2.1} - \frac{1}{2} + \frac{1}{2.2} + \frac{1}{2(n + 1)}- \frac{1}{n + 1} +
  \frac{1}{2(n + 2)} = \frac{1}{4} - \frac{1}{2(n + 1)} + \frac{1}{2(n + 2)}\Rightarrow S_\infty =
  \frac{1}{4}$
\item $\startalign\NC S_n = \NC1 + 5 + 11 + 19 + \cdots + t_{n - 1} + t_n\NR\NC S_n = \NC\;\;\;\;\;\; 1 + 5 + 11 +
  \cdots + t_{n - 1} + t_n\NR\stopalign$

  Subtracting, we get $t_n = 1 = [4 + 6 + 8 + \cdots$ to $(n - 1)$ terms $] = 1 + \frac{n - 1}{2}[2.4 + (n - 2)2]
  = n^2 + n - 1\Rightarrow S_n = \displaystyle\sum_{i=1}^ni^2 + \sum_{i=1}^ni - \sum_{i=1}^n 1 = \frac{n(n +
    1)(2n + 1)}{6} + \frac{n(n + 1)}{2} - n = \frac{n(n^2 + 3n - 1)}{3}$.
\item First person gets $1$ repee, second person gets $1 + 1 = 2$ rupee, third person gets $2 + 2= 4$ rupee,
  fourth person gets $4 + 3 = 7$ rupee and so on.

  $\startalign\NC S_n  = \NC 1 + 2 + 4 + 7 + \cdots + t_n\NR\NC S_n  = \NC\;\;\;\;\;\; 1 + 2 + 4 + \cdots +
  t_{n - 1} + t_n\NR\stopalign$

  Subtracting, we get  $t_n = 1 + [1 + 2 + 3 + \cdots\;\mathrm{to}\;(n - 1)\;\mathrm{terms}] = 1 + \frac{n -
    1}{2}[2.1 + (n - 2)] = \frac{n^2 - n + 2}{2} = 67 \Rightarrow n^2 - n - 132 = 0 \Rightarrow n = 12$.
\item First term contains $1$ integer, second term contains $2$ and so on. So before $t_n$ we will have $1 +
  2 + \cdots + (n - 1)$ integers i.e. $\frac{n(n - 1)}{2}$ integers. So $t_n$ will start with $\frac{n(n -
    1) + 2}{2}$ and will have $n$ integers. So $t_n = \frac{n^2 - n + 2}{2}$. This will be the first number
  in $n$th group. So sum of $n$th group $= \frac{n}{2}[n^2 - n + 2 + n - 1] = \frac{n(n^2 + 1)}{2}$.
\item $\startalign\NC S_n = \NC 1 + 3 + 7 + 15 + \cdots + t_n\NR\NC S_n = \NC\;\;\;\;\;\;\;\;1 + 3 + 7 + \cdots +
  t_{n - 1} + t_n\NR\stopalign$

  Subtracting, we have $t_n = 1 + [2 + 4 + 8 + \cdots\;\mathrm{to}\;(n - 1)\;\mathrm{terms}] = 1 +
  \frac{2(2^{n - 1} - 1)}{2 - 1} = 2^n - 1 \Rightarrow S_n = (2 - 1) + (2^2 - 1) + (2^3 - 1) + \cdots + (2^n
  - 1) = \frac{2(2^n - 1)}{2 - 1} - n = 2^{n + 1} - 2 - n$.
\item $\startalign\NC S_n = \NC 1 + 2x + 3x^2 + \cdots + t_n\NR\NC xS_n = \NC\;\;\;\;\;\;1.x + 2x^2 + \cdots +
  t_{n - 1} + t_n\NR\stopalign$

  Subtracting we get $(1 - x)S_n = 1 + x + x^2 + \cdots\;\mathrm{to}\;n\;\mathrm{terms} - xt_n = \frac{1 -
    x^n}{1 - x} - x.nx^{n - 1} \Rightarrow S_n = \frac{1 - x^n}{(1 - x)^2} - \frac{nx^n}{1 - x}$.
\item Given $\startalign\NC S_{100} = \NC1 + 2.2 + 3.2^2 + 4.3^3 + \cdots + 100.2^{99}\NR\NC 2.S_{100} =
  \NC\;\;\;\;\;\; 1.2 + 2.2^2 + 3.2^3 + \cdots + 99.2^{99} + 100.2^{100}\NR\stopalign$

  Subtracting, we get $-S_n = 1 + [2 + 2^2  + 2^3 + \cdots\;\mathrm{to}\;99\;\mathrm{terms}] -
  100.2^{100}$

  $S_n = 100.2^{100} - \frac{2^{100} - 1}{2 - 1} = 99.2^{100} + 1$.
\item Clearly $\startalign\NC S = \NC 1 + 2^2x + 3^2x^2 + 4^2x^3 + \cdots\;\mathrm{to}\;\infty\NR\NC xS
  = \NC\;\;\;\;\;\;\;\;\;\;x + 2^2x^2 + 3^2x^3 + \cdots\;\mathrm{to}\;\infty\NR\stopalign$

  Subtracting, we get $\startalign\NC(1 - x)S = \NC 1 + 3x + 5x^2 + 7x^3 + \cdots\;\mathrm{to}\;\infty\NR\NC
  x(1 - x)S = \NC\;\;\;\;\;\;\;\;x + 3x^2 + 5x^3 + \cdots\;\mathrm{to}\;\infty\NR\stopalign$

  Again subtracting, $(1 - x)^2S = 1 + 2x + 2x^2 + 2x^3 + \cdots\;\mathrm{to}\;\infty = 1 + \frac{2x}{1 - x}
  = \frac{1 + x}{1 - x}\Rightarrow S = \frac{1 + x}{(1 - x)^2}$.
\item $S_n = 2n^2 + 4, t_n = S_n - S_{n - 1} = 2n^2 + 4 - 2(n - 1)^2 - 4 = 4n -2 \Rightarrow d = t_n - t_{n
  - 1} = 4n - 2 - 4(n - 1) + 2 = 4$ which is constant therefore the given sequence is in A.P.

  Hint: Any sequence which is of the for which sum is of the form $an^2 + bn + c$ will lead to an A.P.
\item Given $t_n = n(n - 1)(n + 1) = n^3 - n \Rightarrow S_n = \displaystyle\sum_{i=1}^ni^3 - \sum_{i=1}^ni =
  \left[\frac{n(n + 1)}{2}\right]^2 - \frac{n(n + 1)}{2} = \frac{n(n + 1)(n^2 + n - 2)}{4}$.
\item Clearly, $\displaystyle t_n = (2n - 1)^3 = 8n^3 - 12n^2 + 6n - 1 \Rightarrow S_n = 8\sum_{i=1}^ni^3 -
  12\sum_{i=1}^ni^2 + 6\sum_{i=1}^ni - \sum_{i=1}^n1 = 2n^2(n + 1)^2 - 2n(n + 1)(2n + 1) + 3n(n + 1) - n$;
  simplification is left to you.
\item Clearly, $t_n = (3n - 2)^2 = 9n^2 - 12n + 4 \Rightarrow S_n = \displaystyle9\sum_{i=1}^ni^2 -
  12\sum_{i=1}^ni + 4\sum_{i=1}^n1 = \frac{3n(n + 1)(2n + 1)}{2} - 6n(n + 1) + 4n$; simplification is
  left to you.
\item Given series is $1^2 + 3^2 + 5^2 + \cdots\;\mathrm{to}\;n\;\mathrm{terms} + 2 + 4 + 6 +
  \cdots\;\mathrm{to}\;n\;\mathrm{terms}$.

  $\Rightarrow t_n = (2n - 1)^2 + \frac{n}{2}[2.2 + (n - 1)2] = 4n^2 - 4n + 1 + n^2 + n = 5n^2 - 3n + 1$

  $\Rightarrow S_n = 5\sum_{i=1}^ni^2 - 3\sum_{i=1}^ni + \sum_{i=1}^n1 = \frac{5n(n + 1)(2n + 1)}{6} -
  \frac{3n(n + 1)}{2} + n$; simplification is left to you.
\item {\bf Case I:} When $n$ is even. Let $n = 2m$ then $S = 1^2 + 3^2 + 5^2 +
  \cdots\;\mathrm{to}\;m\;\mathrm{terms} - [2^2 + 4^2 + 6^2 + \cdots\;\mathrm{to}\;m\;\mathrm{terms}]$

  $= \sum_{i=1}^m(2i - 1)^2 - \sum_{i=1}^m(2i)^2 = -4\sum_{i=1}^mi + \sum_{i=1}^m1 = -2m(m + 1) + 4m = -2m^2
  + 2m$ and then we substitute $m = \frac{n}{2}$.

  {\bf Case II:} When $n$ is odd. Let $n = 2m + 1$, then $S = 1^2 + 3^2 + 5^2 +
  \cdots\;\mathrm{to}\;(m + 1)\;\mathrm{terms} - [2^2 + 4^2 + 6^2 + \cdots\;\mathrm{to}\;m\;\mathrm{terms}]$

  $= \displaystyle\sum_{i=1}^{m + 1}(2i - 1)^2 - \sum_{i=1}^m(2i)^2 = \frac{4(m + 1)(m + 2)(2m + 3)}{6} -
  2(m + 1)(m + 2) + (m + 1) - \frac{2m(m + 1)(2m + 1)}{3}$; put $m = \frac{n - 1}{2}$ and simplify.
\item Clearly, $t_n = (2n - 1)(2n + 1) = 4n^2 - 1\Rightarrow S_n = \displaystyle4\sum_{i=1}^ni^2 -
  \sum_{i=1}^n1 = \frac{2n(n + 1)(2n + 1)}{3} - n$; simplification is left to you.
\item Clearly, $t_n = n(n + 1) \Rightarrow\displaystyle S_n = \sum_{i=1}^ni^2 + \sum_{i=1}^ni = \frac{n(n +
  1)(2n + 1)}{6} + \frac{n(n + 1)}{2}$; simplification is left to you.
\item Clearly, $t_n = n(n + 1)^2 = n^3 + 2n^2 + n \Rightarrow \displaystyle S_n = \sum_{i=1}^ni^3 +
  2\sum_{i=1}^ni^2 + \sum_{i=1}^ni = \left[\frac{n(n + 1)}{2}\right]^2 + \frac{n(n + 1)(2n + 1)}{3} +
  \frac{n(n + 1)}{2}$; simplification is left to you.
\item Clearly, $t_n = (n + 1)n^2 = n^3 + n^2 \Rightarrow \displaystyle S_n = \left[\frac{n(n +
    1)}{2}\right]^2 + \frac{n(n + 1)(2n + 1)}{6}$;  simplification is left to you.
\item $t_n = 1 + 3 + 5 + \cdots\;\mathrm{up to}\;n\;\mathrm{terms} = \frac{n}{2}[2.1 + (n - 1)2] = n^2
  \Rightarrow \displaystyle S_n = \sum_{i=1}^ni^2 = \frac{n(n + 1)(2n + 1)}{6}$.
\item $t_n = 1^2 + 2^2 + 3^2 + \cdots\;\mathrm{up to}\;n\;\mathrm{terms} = \displaystyle\sum_{i=1}^ni^2 =
  \frac{n(n + 1)(2n + 1)}{6} = \frac{n^3 + 3n^2 + n}{6}$.

  $S_n = \frac{1}{6}\left[\sum_{i=1}^ni^3 + 3\sum_{i=1}^ni^2 + \sum_{i=1}^ni\right] =
  \frac{1}{6}\left[\frac{n^2(n + 1)^2}{4} + \frac{n(n + 1)(2n + 1)}{2} + \frac{n(n + 1)}{2}\right]$;
  simplification is left to you.
\item $t_n = n(n + 1)(2n + 1) = 2n^3 + 3n^2 + n \Rightarrow S_n = \displaystyle 2\sum_{i=1}^ni^3 +
  3\sum_{i=1}^ni^2 + \sum_{i=1}^ni = \frac{n^2(n + 1)^2}{2} + \frac{n(n + 1)(2n + 1)}{2} + \frac{n(n +
    1)}{2}$; simplification is left to you.
\item $t_n = n(n + 1)(n + 2) = n^3 + 3n^2 + 2n \Rightarrow \displaystyle S_n = \sum_{i=1}^ni^3 +
  3\sum_{i=1}^ni^2 + 2\sum_{i=1}^ni = \frac{n^2(n + 1)^2}{4} + \frac{n(n + 1)(2n + 1)}{2} + n(n + 1)$;
  simplification is left to you.
\item $t_n = n(2n + 1)^2 = 4n^3 + 4n^2 + n \Rightarrow \displaystyle S_n = 4\sum_{i=1}^ni^3 +
  4\sum_{i=1}^ni^2 + \sum_{i=1}^ni = n^2(n + 1)^2 + \frac{2n(n + 1)(2n + 1)}{3} + \frac{n(n + 1)}{2}$;
  put $n = 20$ and simplify.
\item $t_r = r(n^2 - r^2) = n^2r - r^3 \Rightarrow S = n^2\sum_{i=1}^ni - \sum_{i=1}^ni^3 = \frac{n^3(n +
  1)}{2} - \frac{n^2(n + 1)^2}{4}$; simplification is left to you.
\item $t_n = (2n + 1)^3 - (2n)^3 = 12n^2 + 6n + 1 \Rightarrow S_n = \displaystyle 12\sum_{i=1}^ni^2 +
  6\sum_{i=1}^ni + \sum_{i=1}^n1 = 2n(n + 1)(2n + 1) + 3n(n + 1) + n$; put $n = 10$ to get the answer.
\item $t_n = \frac{1}{1 + 2 + 3 + \cdots\;\mathrm{to}\;n\;\mathrm{terms}} = \frac{2}{n(n + 1)} =
  2\left[\frac{1}{n} - \frac{1}{n + 1}\right]$

  $t_1 = 2\left[1 - \frac{1}{2}\right], t_2 = 2\left[\frac{1}{2} - \frac{1}{3}\right], t_3 =
  2\left[\frac{1}{3} - \frac{1}{4}\right], \ldots, t_n = 2\left[\frac{1}{n} - \frac{1}{n  + 1}\right]$.

  Adding, $S = 2\left[1 - \frac{1}{n + 1}\right] = \frac{2n}{n + 1}$.
\item $S = \frac{1}{2.4} + \frac{1}{4.6} + \frac{1}{6.8} + \frac{1}{8.10} + \ldots = 2\left[\frac{1}{2} -
  \frac{1}{4} + \frac{1}{4} - \frac{1}{6} + \frac{1}{6} - \frac{1}{8} + \cdots\;\mathrm{to}\;\infty\right] =
  1$.
\item $\startalign\NC S = \NC2 + 6 + 12 + 20 + \cdots + t_n\NR\NC S = \NC\;\;\;\;\;\;\; 2 + 6 + 12 + \cdots +
  t_{n - 1} + t_n\NR\stopalign$

  Subtracting, $t_n = 2 + 4 + 6 + 8 + \cdots\;\mathrm{to}\;n\;\mathrm{terms} = \frac{n}{2}[2.2 + (n - 1)2] =
  n(n + 1) = n^2 + n \Rightarrow S = \displaystyle \sum_{i=1}^ni^2 + \sum_{i=1}^ni = \frac{n(n + 1)(2n +
    1)}{6} + \frac{n(n + 1)}{2}$; simplification is left to you.
\item $\startalign\NC S = \NC 3 + 6 + 11 + 18 + \cdots + t_n\NR\NC S = \NC\;\;\;\;\;\;3 + 6 + 11 + 18 +
  \cdots + t_{n - 1} + t_n\NR\stopalign$

  Subtracting,
  $t_n = 3 + [3 + 5 + 7 + \cdots\;\mathrm{to}\;(n - 1)\;\mathrm{terms}] = 3 + \frac{n - 1}{2}[2.3 + (n - 2)2]
    = 3 + n^2 - 1 = n^2 + 2$

  $S = \frac{n(n + 1)(2n + 1)}{6} + 2n$; simplification is left to you.
\item $\startalign\NC S = \NC 1 + 9 + 24 + 46 + 75 + \cdots + t_n\NR\NC S = \NC\;\;\;\;\;\;1 + 9 + 24 + 46 +
  \cdots + t_{n - 1} + t_n\NR\stopalign$

  Subtracting $t_n = 1 + 8 + 15 + 22 + 29 + \cdots\;\mathrm{to}\;n\mathrm{terms} = \frac{n}{2}[2 + (n - 1)7]
  = \frac{7n^2 - 5n}{2}$.

  $\Rightarrow S = \frac{7n(n + 1)(2n  1)}{12} - \frac{5n(n + 1)}{4}$.
\item $\startalign\NC S = \NC2 + 4 + 7 + 11 + 16 + \cdots + t_n\NR\NC S = \NC\;\;\;\;\;\;2 + 4 + 7 + 11 +
  \cdots + t_{n - 1} + t_n\NR\stopalign$

  Subtracting, $t_n = 2 + [2 + 3 + 4 + 5 + \cdots\;\mathrm{to}\;(n - 1)\;\mathrm{terms}] = 2 + \frac{n -
    1}{2}[2.2 + n - 1] = 2 + \frac{n^2 + 2n - 3}{2} = \frac{n^2 - 2n + 1}{2}$.
\item $\startalign\NC S = \NC 1 + 3 + 6 + 10 + \cdots + t_n\NR\NC S = \NC\;\;\;\;\;\;1 + 3 + 6 + \cdots +
  t_{n - 1} + t_n\NR\stopalign$

  Subtracting, $t_n = 1 + 2 + 3 + 4 + \cdots\;\mathrm{to}\;n\mathrm{terms} = \frac{n(n + 1)}{2} = \frac{n^2
    + n}{2}$

  $\Rightarrow S = \frac{n(n + 1)(2n + 1)}{12} + \frac{n(n + 1)}{4}$. Put $n = 10$ to get the answer.
\item First group contains $2$ odd numbers, second group contains $4$ odd numbers, third group contains $6$
  odd numbers so $(n - 1)$th group will contain $2n - 2$ odd numbers.

  Total no. of odd numbers till $(n - 1)$th group will be $n(n - 1)$. So last no. in $(n - 1)$th group will
  be $1 + (n^2 - n - 1)2 = 2n^2 - 2n - 1$ and hence first number in $n$th group will be $2n^2 - 2n + 1$ and
  there will be $2n$ odd numbers. So sum of $2n$ odd numbers starting from $2n^2 - 2n + 1$ is given by
  $\frac{2n}{2}[4n^2 - 4n + 2 + (2n - 1)2] = 4n^3$.
\item Groups contain $1, 3, 5, \ldots$ number of terms so $n$th group will contain $2n - 1$ numbers starting
  from $n$. So sum will be $\frac{2n - 1}{2}[2n + 2n - 2] = (2n - 1)^2$ which is square of odd positive
  integer.
\item $\startalign\NC S = \NC 2 + 5 + 14 + 41 + \cdots + t_n\NR\NC S = \NC\;\;\;\;\;\; 2 + 5 + 14 + \cdots +
  t_{n - 1} + t_n\NR\stopalign$

  Subtracting $t_n = 2 + [3 + 3^2 + \cdots\;\mathrm{to}\;(n - 1)\mathrm{terms}] = 2 + \frac{3(3^{n - 1} -
    1)}{3 - 1} = \frac{3^n + 1}{2}$.

  $\Rightarrow S = \frac{1}{2}\left[\frac{3(3^n - 1)}{2} + n\right]$.
\item $\startalign\NC S = \NC1.1 + 2.3 + 4.5 + 8.7 + \cdots + t_n\NR\NC 2S = \NC\;\;\;\;\;\;\;\;\; 2.1 + 4.3 +
  8.5 + \cdots + t_{n - 1} + 2^n(2n - 1)\NR\stopalign$

  Subtracting, $-S = 1.1 + [2.2 + 4.2 + 8.2 + \cdots\;\mathrm{to}\;(n - 1)\;\mathrm{terms}] - 2^n(2n - 1)$

  $S = 2^n(2n - 1) - 1 - 4(2^{n - 1} - 1)$.
\item Clearly, $a_{2n} - a_1 = (2n - 1)d \Rightarrow d = \frac{a_{2n} - a_1}{2n - 1}$

  Now, $a_1^2 - a_2^2 + a_3^2 - a_4^2 + \cdots + a_{2n - 1}^2 - a_{2n}^2 = (a_1 - a_2)(a_1 + a_2) + (a_3 -
  a_4)(a_3 + a_4) + \cdots + (a_{2n - 1} - a_{2n})(a_{2n - 1} + a_{2n})$

  $= -d(a_1 + a_2 + a_3 + a_4 + \cdots + a_{2n - 1} + a_{2n}) = -\frac{a_{2n} - a_1}{2n -
    1}.\frac{2n}{2}\left[a_1 + a_{2n}\right] = \frac{n}{2n - 1}(a_1^2 - a_{2n}^2)$.
\item $d = \alpha_2 - \alpha_1 = \alpha_3 - \alpha_2 = \cdots = \alpha_n - \alpha_{n - 1}$

  $\sin d\sec\alpha_1\sec\alpha_2 = \frac{\sin(\alpha_2 - \alpha_1)}{\cos\alpha_1\cos\alpha_2} =\tan\alpha_2 -
  \tan\alpha_1$. Similarly, $\sin d\sec\alpha_2\sec\alpha_3 = \tan\alpha_3 - \tan\alpha_2$ and so
  on. $\sin d\sec\alpha_{n - 1}\sec\alpha_n = \tan\alpha_n - \tan\alpha_{n - 1}$

  Adding we get L.H.S. = R.H.S.
\item L.H.S. $= \frac{1}{a_1 + a_n}\left[\frac{a_1 + a_n}{a_1a_n} + \frac{a_1 + a_n}{a_2a_{n - 1}} + \cdots
  + \frac{a_1 + a_n}{a_na_1}\right] = \frac{1}{a_1 + a_n}\left[\frac{a_1 + a_n}{a_1a_n} + \frac{a_2 +
    a_{n - 1}}{a_2a_{n - 1}} + \cdots + \frac{a_1 + a_n}{a_na_1}\right]$

  $= \frac{1}{a_1 + a_n}\left[\frac{1}{a_1} + \frac{1}{a_n} + \frac{1}{a_2} + \frac{1}{a_{n - 1}} + \cdots +
  \frac{1}{a_n} + \frac{1}{a_1}\right] = \frac{2}{a_1 + a_n}\left(\frac{1}{a_1} + \frac{1}{a_2} + \cdots +
  \frac{1}{a_n}\right)$.
\item $\frac{1}{a_1} - \frac{1}{a_2} = \frac{a_2 - a_1}{a_1a_2} = \frac{d}{a_1a_2} \Rightarrow
  \frac{1}{a_1a_2} = \frac{1}{d}\left(\frac{1}{a_1} - \frac{1}{a_2}\right)$. Similarly $\frac{1}{a_2a_3}=
  \frac{1}{d}\left(\frac{1}{a_2} - \frac{1}{a_3}\right)$ and so on.

  $\therefore S = \frac{1}{d}\left(\frac{1}{a_1} - \frac{1}{a_{n + 1}}\right) = \frac{n}{a_1a_{n 1}}$
\item $\because a_1 = 0$ then $a_2 = d, a_3 = 2d, \ldots, a_n = (n - 1)d$ where $d$ is the c.d. of the A.P.

  L.H.S. $= \frac{2}{1} + \frac{3}{2} + \frac{4}{3} + \cdots + \frac{n - 1}{n - 2} - \left(1 + \frac{1}{2} +
  \frac{1}{3} + \cdots + \frac{1}{n - 3}\right)$

  $= (1 + 1) + \left(1 + \frac{1}{2}\right) + \cdots \+ \left(1 + \frac{1}{n - 2}\right) - \left(1 +
  \frac{1}{2} + \frac{1}{3} + \cdots + \frac{1}{n - 3}\right)$

  $= n - 2 + \left[\left(1 + \frac{1}{2} + \frac{1}{3} + \cdots + \frac{1}{n - 2}\right) - \left(1 +
    \frac{1}{2} + \cdots + \frac{1}{n - 3}\right)\right]$

  $= n - 2 + \frac{1}{n - 2} = \frac{a_{n - 1}}{a_2} + \frac{a_2}{a_{n - 1}} =$ R.H.S.
\item L.H.S. $= \displaystyle\sum_{k=1}^n\frac{a_ka_{k+1}a_{k+2}}{(a_{k + 1} - d) + (a_{k + 1} + d)} = \frac{1}{2}\sum_{k
  = 1}^na_ka_{k + 2} = \frac{1}{2}\sum_{i = 1}^k(a_{k + 1}^2 - d^2) = \frac{1}{2}\sum_{k = 1}^n[(a_1 + kd)^2 -
  d^2] = \frac{1}{2}\sum_{k = 1}^n[a_1^2 + 2a_1dk + (k^2 - 1)d^2]$

  $= \displaystyle\frac{1}{2}\left[\sum_{k = 1}^na_1^2 + 2a_1d\sum_{k = 1}^nk + d^2\sum_{k = 1}^nk^2 - \sum_{k = 1}^nd^2\right] =
  \frac{1}{2}\left[na_1^2 + 2a_1d\frac{n(n + 1)}{2} + d^2\frac{n(n + 1)(2n + 1)}{6} - nd^2\right]$

  $= \frac{n}{2}\left[a_1^2 + (n + 1)a_1d + \frac{(n - 1)(2n + 5)}{6}d^2\right] =$ R.H.S.
\item Given, $x^{18} = y^{21} \Rightarrow 18\log x = 21\log y \Rightarrow \log_yx = \frac{7}{6}$

  Similarly $y^{12} = z^{28} \Rightarrow \log_zy = \frac{4}{3}$ and $x^{18} = y^{28} \Rightarrow \log_xz =
  \frac{9}{14}$

  Now it is trivial to prove that $3, 3\log_yx, \log_zx, 7\log_xz$ are in A.P.
\item Given, $I_n = \displaystyle\int_0^{\tfrac{\pi}{2}}\frac{\sin^2nx}{\sin^2x}dx$. Since we have to prove
  that $I_1, I_2, I_3, \ldots$ are in A.P. we can simply prove that $I_n, I_{n + 1}, I_{n + 2}$ are in
  A.P. which will be enough to prove the entire sequence. So it is enough to prove that $I_n + I_{n + 2} -
  2I_{n + 1} = 0$

  L.H.S. $= \displaystyle\sin_0^{\tfrac{\pi}{2}}\frac{\sin^2(n + 2)x + \sin^2nx - \sin^2(n +
    1)x}{\sin^2x}dx$

  $= \displaystyle\int_{i=0}^{\tfrac{\pi}{2}}\frac{1 - \cos(2n + 4)x + 1 - \cos2nx - 2 + 2\cos(2n +
    2)x}{2\sin^2x}dx$

  $= \displaystyle\int_{i=0}^{\tfrac{\pi}{2}}\frac{2\cos(2n + 2)x - 2\cos(2n + 2)x\cos2x}{2\sin^2x}dx$

  $= \displaystyle\int_{i=0}^{\tfrac{\pi}{2}}\frac{2\cos(2n + 2)x.2\sin^2x}{2\sin^2x}dx =
  \int_{i=0}^{\tfrac{\pi}{2}}2\cos(2n + 2)dx = \left[\frac{\sin(2n + 2)x}{n + 1}\right] = 0$.
\item Let $a_1, a_2, a_3, \ldots$ be an A.P. which are distinct primes. Clearly $a_1 \geq 1$. $d = a_2 -
  a_1\geq 1$. Now $(a_1 + 1)th$ term $= a_1 + a_1d = a_1(1 + d)$ which is a composite number. Thus, there
  cannot be such an A.P.
\item Let the four distinct integers in A.P. be $a, a + d, a + 2d, a + 3d$ where $d > 0$. Obviously, the
  term which is sum of squares of remaining terms will be $a + 3d$.

  Let $a + 3d = a^2 + (a + d)^2 + (a + 2d)^2 = 3a^2 + 6ad + 5d^2 \Rightarrow 5d^2 + a(6d - 1) + 5d^2 - 3d =
  0$

  $\Rightarrow 9(2a - 1)^2 - 20(3a^2 - a)\geq 0\;[\because d$ is real $] \Rightarrow -24a^2 - 16a + 9 \geq 0$

  Corresponding roots are $-\frac{4 \pm \sqrt{70}}{12} \Rightarrow -\frac{4 - \sqrt{70}}{12}\leq a\leq
  -\frac{4 + \sqrt{70}}{12}\therefore a = -1, 0[\because a$ is an integer $]$.

  $\Rightarrow a = 1$ other roots are not acceptable. Numbers are $-1, 0, 1, 2$.
\item Given, $t_n = p + q$ and $t_{n + 1} = p - q \Rightarrow d = -2q$. We also know that

  $t_1 + t_{2n} = t_2 + t_{2n - 1} = \cdots = t_n + t_{n + 1} = 2p$

  $t_1^3 + t_{2n}^3 = (t_1 + t_{2n})^3 - 3t_1t_{2n}(t_1 + t_{2n}) = 8p^3 - 6pt_1t_{2n} = 8p^3 + \frac{6p}{4}[(t_1 +
  t_{2n})^2 - (t_1 - t_{2n})] = 8p^3 - \frac{3p}{2}[4p^2 - (2n - 1)^2d^2] = 2p^3 + 6pq^2(2n - 1)^2$

  $S = 2np^3 + 6pq^2[1^2 + 3^2 + \cdots + (2n - 1)^2]$ (we have found $\displaystyle\sum_{i = 1}^n(2i -
  1)^2$ so we will use that result)

  $= 2np^3 + 2pq^2.n(2n + 1)(2n - 1) = 2np[p^2 + (4n^2 - 1)q^2]$.
\item Let $a$ be the first term and $d$ be the c.d. of the A.P. Then,

  $S = \frac{n}{2}[2a + (n - 1)d]$

  $S_n = a^3 + (a + d)^3 + (a + 2d)^3 + \cdots + [a + (n - 1)d]^3$

  $= na^3 + 3a^2d[1 + 2 + 3 + \cdots + (n - 1)] + 3ad^2[1^2 + 2^2 + 3^2 + \cdots + (n - 1)^2] + d^3[1^3 +
  2^3 + 3^3 + \cdots + (n - 1)^3]$

  $= na^3 + 3a^2d\frac{n(n - 1)}{2} + 3ad^2\frac{n(n - 1)(2n - 1)}{6} + d^3\frac{n^2(n - 1)^2}{4}$

  $= \frac{n}{2}[2a + (n - 1)d][a^2 + (n - 1)ad + \frac{n(n - 1)}{2}d^2] = S[a^2 + (n - 1)ad + \frac{n(n -
    1)}{2}d^2]$.

  Hence, $S$ is a factor of $S_n$.
\item Let $r$ be a positive integer greater than $1$. If possible, let $m^r = (2k + 1) + (2k + 3) + \cdots +
  (2k + 2m - 1) = \frac{m}{2}[2k + 1 + 2k + 2m - 1] = 2k + m \Rightarrow k = \frac{m^{r- 1} - m}{2}$

  Clealry for $r > 1, m^{r - 1}$ and $m$ are both odd or both even. $\therefore m^{r - 1} - m$ is an even
  number. Thus such an integer $k$ exists.

  Also, the first odd inetger $= 2k + 1 = m^{r - 1} - m + 1$.
\item Let $x$ be the first term and $d$ be the c.d. of the A.P. Then,

  $x + (x + d) + (x + 2d) + \cdots + [x + (n - 1)d] = a$

  \placeformula[eq:progression:p281-1]\startformula a = nx + \frac{dn(n - 1)}{2}]\stopformula

  Also, $x^2 + (x + d)^2 + (x + 2d)^2 + \cdots + [x + (n - 1)d]^2 = b^2$

  $= nx^2 + 2xd[1 + 2 + 3 + \cdots + (n - 1)] + d^2[1^2 + 2^2 + \cdots + (n - 1)^2]$
  \placeformula[eq:progression:p281-2]\startformula b^2 = nx^2 + xdn(n - 1) + d^2\frac{(n - 1)n(2n -
    1)}{6}\stopformula

  Sqauring \in{Eq.}[eq:progression:p281-1], we have

  $a^2 = n^2x^2 + n^2xd(n - 1) + \frac{n^2d^2(n - 1)^2}{4} = a^2$

  \placeformula[eq:progression:p281-3]\startformula nx^2 + nxd(n - 1) + \frac{nd^2(n - 1)^2}{4} =
  a^2\stopformula

  \in{Eq.}[eq:progression:p281-2] - \in{Eq.}[eq:progression:p281-3]

  $\Rightarrow d^2\frac{n(n - 1)(n + 1)}{12} = \frac{nb^2 - a^2}{n}\Rightarrow d = \pm\frac{2\sqrt{3(nb^2 -
      a^2)}}{n\sqrt{n^2 - 1}}$

  Now you can find $x$ trivially.
\item $d = a_2 - a_1 = a_3 - a_2 = \cdots = a_n - a_{n - 1}$. We have to find

  $\sin d[\csc a_1\csc a_2 + \csc a_2\csc a_3 + \cdots + \csc a_{n - 1}\csc a_n]$

  $= \sin d\left[\frac{1}{\sin a_1\sin a_2} + \frac{1}\sin a_{2}\sin a_{3} + \cdots + \frac{1}{\sin a_{n - 1}\sin
    a_n}\right]$

  $= \frac{\sin(a_2 - a_1)}{\sin a_1\sin a_2} + \frac{\sin(a_3 - a_2)}{\sin a_2\sin a_3} + \cdots +
  \frac{\sin(a_n - a_{n - 1})}{\sin a_{n - 1}\sin a_n}$

  $= \frac{\sin a_2\cos a_1 - \sin a_1\cos a_2}{\sin a_1\sin a_2} + \frac{\sin a_3\cos a_2 - \sin a_2\cos
    a_3}{\sin a_2\sin a_3} + \frac{\sin a_n\cos a_{n - 1} - \sin a_{n - 1}\cos a_n}{\sin a_{n - 1}\sin a_n}$

  $= \cot a_1 - \cot a_2 + \cot a_2 - \cot a_3 + \cdots + \cot a_{n - 1} - \cot a_n = \cot a_1 - \cot a_n$.
\item Let $d$ be common difference of the A.P.

  L.H.S. $= \frac{1}{\sqrt{a_1} + \sqrt{a_2}} + \frac{1}{\sqrt{a_2} + \sqrt{a_3}} + \cdots +
  \frac{1}{\sqrt{a_{n - 1}} + \sqrt{a_n}} = \frac{n - 1}{\sqrt{a_1} + \sqrt{a_n}}$

  $= \frac{\sqrt{a_1} - \sqrt{a_2}}{a_1 - a_2} + \frac{\sqrt{a_2} - \sqrt{a_3}}{a_2 - a_3} + \cdots +
  \frac{\sqrt{a_{n - 1}} - \sqrt{a_n}}{a_{n - 1} - a_n}$

  $= -\frac{1}{d}[\sqrt{a_1} - \sqrt{a_n}]\;[\because d = a_2 - a_1 = a_3 - a_2 = \cdots = a_n - a_{n - 1}]$

  $= -\frac{n - 1}{(n - 1)d}\frac{a_1 - a_n}{\sqrt{a_{n - 1}} + \sqrt{a_n}} = \frac{n - 1}{\sqrt{a_1} +
    \sqrt{a_n}}\;[\because a_n = a_1 + (n - 1)d]$.
\item Let $d$ be the common difference of the A.P., then

  L.H.S. $= \displaystyle\sum_{2}^n\tan^{-1}\frac{d}{1 + a_{n - 1}a_n} = \sum_{2}^n\tan^{-1}\frac{a_n - a_{n
  - 1}}{1 + a_{n - 1}a_n} = \sum_{2}^n\tan^{-1}a_n - \tan^{-1}a_{n - 1}\;\left[\because \tan^{-1}x - \tan^{-1}y =
    \tan^{-1}\frac{x - y}{1 + xy}\right]$

  $= \tan^{-1}a_2 - \tan^{-1}a_1 + \tan^{-1}a_3 - \tan^{-1}a_2 + \cdots + \tan^{-1}a_n - \tan^{-1}a_{n - 1}
  = \tan^{-1}a_n - \tan^{-1}a_1 = \tan^{-1}\frac{a_n - a_1}{1 + a_1a_n} =$ R.H.S.
\item Given, $S_n = \frac{1}{a_1a_2} + \frac{1}{a_2a_3} - \cdots + \frac{1}{a_{n - 1}a_n}$

  $= \frac{1}{d}\left[\frac{a_2 - a_1}{a_1a_2} + \frac{a_3 - a_2}{a_2a_3} + \cdots + \frac{a_n - a_{n -
      1}}{a_{n - 1}a_{n}}\right]\;[\because d = a_2 - a_1 = a_3 - a_2 = \cdots = a_n - a_{n - 1}]$

  $= \frac{1}{d}\left[\frac{1}{a_1} - \frac{1}{a_2} + \frac{1}{a_2} - \frac{1}{a_3} + \cdots + \frac{1}{a_{n
        - 1} - \frac{1}{a_n}}\right]$

  $= \frac{1}{d}\left[\frac{1}{a_1} - \frac{1}{a_n}\right] = \frac{a_n - a_1}{da_1a_n} = \frac{(n -
    1)d}{da_1a_n}\;[\because a_n = a_1 + (n - 1)d]$

  $\Rightarrow a_a_nS_n = n - 1$, which does not depend on $a$ or $d$.
\item We know that $S = \frac{n}{2}[t_1 + t_n]$ so

  $S_1 = \frac{n}{2}[a_1 + a_n] = \frac{n}{2}[2a + (n - 1)d]$

  $S_2 = \frac{n}{2}[a_{n + 1} + a_{2n}] = \frac{n}{2}[2a + (3n - 1)d]$

  $S_3 = \frac{n}{2}[a_{2n + 1} + a_{3n}] = \frac{n}{2}[2a + (5n - 1)d]$

  $\ldots\ldots$

  $S_r = \frac{n}{2}[2a + \{(2r - 1)n - 1\}d]$

  Clearly, $S_2 - S_1 = S_3 - S_2 = \cdots = S_{r + 1} - S_r = n^2d$ which is an A.P.
\item Let $d$ be the c.d. of the A.P. then $\frac{b - c}{a - b} = \frac{-d}{-d} = 1$ which is a rational
  number.
\item $\tan70^\circ = \tan(50^\circ + 20^\circ) = \frac{\tan70^\circ + \tan20^\circ}{1 -
  \tan50^\circ\tan20^\circ}$

  $\Rightarrow \tan70^\circ - \tan70^\circ\tan50^\circ\tan20^\circ = \tan50^\circ + \tan20^\circ$

  $\Rightarrow \tan70^\circ - \cot(90^\circ - 70^\circ)\tan50^\circ\tan20^\circ = \tan50^\circ +
  \tan20^\circ$

  $\Rightarrow \tan70^\circ - \tan50^\circ = \tan50^\circ + \tan20^\circ \Rightarrow \tan80^\circ =
  2\tan50^\circ + \tan20^\circ$

  Adding $\tan{20}^\circ$ to both sides, we have

  $\tan70^\circ + \tan20^\circ = 2(\tan50^\circ + \tan20^\circ)$ and thus required condition is proved.
\item Given $\log_l x, \log_m x, \log_n x$ are in A.P. Therefore $2\log_mx = \log_lx + \log_nx$

  $\Rightarrow \frac{2\log x}{\log m} = \frac{\log x}{\log l} + \frac{\log x}{\log n} \Rightarrow
  \frac{2}{\log m} = \frac{\log ln}{\log l\log n}$

  $\Rightarrow 2\log n = \frac{\log ln\log m}{\log l}$ (multiplying with $\log m\log n$ on both sides)

  $\Rightarrow \log n^2 = \log_lm\log ln = \log ln^{\log_lm} \Rightarrow n^2 = (ln)^{\log_lm}$; hence
  proved.
\item Let $b, p, h$ be base, perpendicular, hypotenuse of the triangle. Let $b$ be smallest then $2p = h +
  b \Rightarrow h = 2p - b$

  We know that for a right angle triangle $h^2 = b^2 + p^2$. Substituting for $h$,

  $4p^2 - 4bp + b^2 = b^2 + p^2 \Rightarrow 3p^2 = 4bp \Rightarrow 3p = 4b \Rightarrow h^2 = \frac{16b^2}{9}
  + b^2 \Rightarrow h = \frac{5b}{3}$

  $\Rightarrow b:p:h = 3:4:5$.
\item Let $5^x = t$ then for condition for A.P. gives us $a = 5t + \frac{5}{t} + t^2 + \frac{1}{t^2}$

  We know that $x + \frac{1}{x} \geq 2 \therefore a \geq 12$.
\item Given $\log 2, \log(2^x - 1), \log(2^x + 3)$ are in G.P. Therefore, $2\log(2^x - 1) = \log 2 +
  \log(2^x + 3)$

  $\Rightarrow (2^x - 1)^2 = 2.2^2 + 6 \Rightarrow 2^{2x} - 4.2^x - 5 = 0 \Rightarrow 2^x = 5, -1$ however,
  $2^x \neq -1$ so $2^x = 5 \Rightarrow x = \log_25$.
\item Let $d$ be the c.d. of the A.P. $\therefore \log_yx = 1 + d \Rightarrow x = y^{1 + d}, \log_zy = 1 +
  2d \Rightarrow y = z^{1 + 2d}, -15\log_xz = 1 + 3d \Rightarrow z = x^{-\tfrac{1 + 3d}{15}}$

  $\because x = y^{1 + d} = z^{(1 + 2d)(1 + d)} = x^{-\tfrac{(1 + d)(1 + 2d)(1 + 3d)}{15}} \Rightarrow (1 +
  d)(1 + 2d)(1 + 3d) = -15 \Rightarrow (d + 2)(6d^2 - d + 8) = 0$

  Discriminant of $6d^2 -d + 8$ is less than $0$ and thus $d = -2$.

  $\Rightarrow x = z^3, y = z^{-3}$.
\item Let $\sqrt{2}, \sqrt{3}, \sqrt{5}$ be $p$th, $q$th and $r$th term of an A.P. whose c.d. is $d$.

  $\sqrt{3} - \sqrt{2} = (q - p)d$ and $\sqrt{5} - \sqrt{3} = (r - q)d$. Dividing, we get

  $\frac{\sqrt{3} - \sqrt{2}}{\sqrt{5} - \sqrt{3}} = \frac{q - p}{r - q} = x$, which will be a rational
  number as $p, q, r$ are integers.

  Sqauring $5 - 2\sqrt{6} = x^2(8 - 2\sqrt{15})\Rightarrow \sqrt{15}x^2 - \sqrt{6} = (8x^2 - 5)/2 = y$
  (which will again be a rational number)

  Squaring again $15k^4 + 6 - 2\sqrt{90}k^2 = y^2 \Rightarrow 15k^4 + 6 - y^2 = 2\sqrt{90}k^2$

  L.H.S. is a rational number while R.H.S. is irrational thus our assumption is wrong.
\item Area of $r$th circle $A_r = \pi r^2$ and area of $(r + 1)$th circle is $A_{r + 1} = \pi(r + 1)^2$ so
  the difference is $D_r = \pi(2r + 1)$ therefore c.d. $= D_{r + 1} - D_r = 2\pi$ which is a constant and
  hence the successive areas of each color is in A.P.
\item $\because x, y, z$ are in A.P. $\therefore 2y = x + z$. Similarly, $2\tan^{-1}y = \tan^{-1}x +
  \tan^{-1}z$

  $\Rightarrow \frac{2y}{1 - y^2} = \frac{x + z}{1 - xz} \Rightarrow \frac{x + z}{1 - \frac{(x + z)^2}{4}} =
  \Rightarrow 1 - zx = 1 - \frac{(x + z)^2}{4} \Rightarrow (z - x)^2 = 0 \Rightarrow x = z = y$.
\item From given conditiion $\frac{\cos^4\theta}{\cos^2\alpha} + \frac{\sin^4\theta}{\sin^2\alpha} = 1 =
  \cos^2\theta + \sin^2\theta$

  $\Rightarrow \frac{\cos^4\theta}{\cos^2\alpha}(\cos^2\theta - \cos^2\alpha) =
  \frac{\sin^2\theta}{\sin^2\alpha}(\\sin^2\alpha - \sin^2\theta) =
  \frac{\sin^2\theta}{\sin^2\alpha}(\\sin^2\theta - \sin^2\alpha)$

  $\Rightarrow \frac{\cos^2\theta}{\cos^2\alpha} = \frac{\sin^2\theta}{\sin^2\alpha}$ and thus we prove the
  required condition because $\frac{\cos^{2n + 2}\theta}{\cos^{2n\alpha}} = \cos^2\theta$.
\item $a_{n + 1} - a_n = \displaystyle\int_0^\pi \frac{\sin(2n + 2)x - \sin2nx}{\sin x}dx = \int_0^\pi
  \frac{2\cos(2n + 1)x\sin x}{\sin x}dx$

  $= \left[\frac{2\sin(2n + 1)x}{2n + 1}\right]_0^\pi = 0$

  Hence, c.d. is $0$ making all terms equal and in A.P.
\item $l_n + l_{n + 2} = \displaystyle\int_{0}^{\tfrac{\pi}{4}}(\tan^nx + \tan^{n + 2}xdx) =
  \left[\frac{\tan^{n + 1}x}{n + 1}\right]_0^{\tfrac{\pi}{4}} = \frac{1}{n + 1}$.

  Thus, $\frac{1}{l_2 + l_4} = 3, \frac{1}{l_3 + l_5} = 4, \frac{1}{l_4 + l_5} = 5, \cdots$, which is an
  A.P. with a c.d. of $1$.
\item $I_{n + 1} - I_n = \displaystyle\int_{0}^\pi\frac{\cos 2nx - \cos(2n + 2)x}{\sin^2x}dx =
  2\int_{0}^\pi\frac{\sin x\sin(2n + 1)x}{\sin^2x}dx$

  $\displaystyle D_n = 2\int_0^{\pi}\frac{\sin(2n + 1)x}{\sin x}dx$

  $\displaystyle D_{n + 1} - D_n = 2\int_{0}^\pi\frac{\sin(2n + 3)x - sin(2n + 1)x}{\sin x}dx =
  4\int_0^\pi\frac{\sin x\cos(2n + 2)x}{\sin x}dx = 2\left[\frac{\sin2(n + 1)x}{n + 1}\right]_0^\pi = 0$

  $\Rightarrow D_1 = \pi \Rightarrow I_{n + 1} - I_n = \pi$ which is a constant and hence $I_1, I_2, I_3,
  \ldots$ are in A.P.
\item $\because \alpha, \beta, \gamma$ are in A.P. $\therefore 2\beta = \gamma + \alpha$

  $2\sin(\alpha + \gamma) = \sin(\beta + \gamma) + \sin(\alpha + \beta) \Rightarrow 2\sin2\beta =
  2\sin\left(\frac{\alpha + \beta + 2\beta}{2}\right).\cos\frac{\gamma - \alpha}{2}$

  $\Rightarrow \cos\frac{\gamma - \alpha}{2} = 1 = \cos 0 \Rightarrow \gamma = \alpha = \beta$
  and hence $\tan\alpha = \tan\beta = \tan\gamma$.
\item Let $d$ be the c.d then we have $2b = a + c$ and $abc = 4 \Rightarrow ac(a + c) = 4$. We know that A.M. $\geq$ G.M
  $\Rightarrow \frac{a + c}{2}\geq \sqrt{ac} \Rightarrow \frac{(a + c)^2}{4}(a + c) \geq 4 \Rightarrow b^3
  \geq 4$ and hence proved.
\item Let $S = \log a + \log\frac{a^3}{b} + \log \frac{a^5}{b^2} + \log \frac{a^7}{b^3} + \cdots$

  $= (\log a + 3\log a + 5\log a + \cdots) - (\log b + 2\log b + \cdots) = \frac{n}{2}[2\log a + (n - 1)2\log
  a] - \frac{n - 1}{2}[2\log b + (n - 2)\log b] = \frac{n}{2}[2n\log a] - \frac{n - 1}{2}[2n\log b]$

  $= \log a^{n^2} - \log b^{n(n - 1)} = \log\frac{a^{n^2}}{b^{(n(n - 1))}}$.
\item $b = a + d \Rightarrow d = b - a$ and $n = \frac{c - a}{b - a} + 1 = \frac{b + c - 2a}{b - a}$

  $S_n = \frac{n}{2}[a + c] = \frac{(b + c - 2a)(a + c)}{2(b - a)}$.
\item Let $a$ be the first term and $d$ be the c.d. of the A.P.

  $S_{n + 3} = \frac{n + 3}{2}[2a + (n + 2)d]$ and $3(S_{n + 2} - S_{n + 1}) + S_n = 3t_{n + 2} +
  \frac{n}{2}[2a + (n - 1)d] = 3[a + (n + 1)d] + \frac{n}{2}[2a + (n - 1)d]$

  $= \frac{1}{2}[2an + n(n - 1)d + 6a + 6(n + 1)d] = \frac{1}{2}[2a(n + 3) + (n^2 + 5n + 6)d] = S_{n + 3}$.
\item Observe that $2ab = (a + b)^2 - (a^2 + b^2), 2(ab + bc + ca) = (a + b + c)^2 - (a^2 + b^2 +
  c^2)$. Similarly it can be observed that $\displaystyle2\sum_{r < s}a_ra_s = \left(\sum_{i = 1}^na_i\right)^2 -
  \sum_{i = 1}^n a_i^2$

  Now, $\left(\sum_{i=1}^na_i\right)^2 = \left[\frac{n}{2}(2a_1 + (n - 1)d)\right]^2$

  \placeformula[eq:progression:307-1]\startformula \left(\sum_{i=1}^na_i\right)^2= \frac{n}{2}[4a_1^2 + 4a_1(n -
    1)d + (n - 1)^2d^2]\stopformula

  and $\sum_{i=1}^n a_i^2 = a_1^2 + (a_1 + d)^2 + (a_1 + 2d)^2 + \cdots + [a_1 + (n - 1)d]^2$

  \placeformula[eq:progression:307-2]\startformula\sum_{i=1}^n a_i^2= na_1^2 + a_1dn(n - 1) + \frac{d^2(n -
      1)n(2n - 1)}{6}\stopformula

  Adding \in{Eq.}[eq:progression:307-1] and \in{Eq.}[eq:progression:307-2], we get the desired answer.
\item Let there be $n$ rows in the equilateral triangle. Then $S = \frac{n(n + 1)}{2}$. Now according to
  given facts, $\frac{n(n + 1)}{2} + 669 = (n - 8)^2 \Rightarrow n = 55 \Rightarrow S = 1540$.
\item Required sum $= \frac{(1 + 2 + 3 + \cdots + n)^2 - (1^2 + 2^2 + 3^2 + \cdots + n^2)}{2}$

  $= \frac{\tfrac{n^2(n + 1)^2}{4} - \tfrac{n(n + 1)(2n + 1)}{6}}{2} = \frac{\tfrac{n(n +
    1)}{2}\left(\tfrac{n(n + 1)}{2} - \tfrac{2n + 1}{3}\right)}{2} = \tfrac{1}{24}n(n^2 - 1)(3n + 2)$.
\item Let $a$ be the first term and $d$ be the c.d. for the given A.P. Let $S, S'$ represent the sum for
  first $24$ days and last $18$ days. Then,

  $S = \frac{24}{2}[2a + 23d], S' = \frac{18}{2}[2(a + 24d) + 17d]$ and

  $\frac{24}{2}[2a + 23d] + \frac{18}{2}[2a + 65d] = \frac{42}{2}[2a + 41d]$ and $S = S' \Rightarrow
  \frac{24}{2}[2a + 23d] = \frac{18}{2}[2a + 65d]$

  Solving these two equations yield the answer as $12096$.
\item Let $a$ be the first term and $d$ be the c.d. for the given A.P. Then,

  $S_n = \frac{n}{2}[2a + (n - 1)d] = n^2p$ and $S_m = \frac{m}{2}[2a + (m - 1)d] = m^2p$

  $\Rightarrow 2a + (n - 1)d = 2np$ and $2a + (m - 1)d = 2mp \Rightarrow (n - m)d = 2p(n - m) \Rightarrow d
  = 2p$

  Substituting this in equation for $S_n, 2a + 2(n - 1)p = 2np \Rightarrow a = p$

  $\Rightarrow S_p = \frac{p}{2}[2p + 2(p - 1)p] = p^3$.
\item Let $S_1, S_2, \ldots, S_n$ denote the sum of A.P. with c.d. $1, 2, \ldots, n$. Then,

  $t_r = 1 + (n - 1)r$

  $S_1 + S_2 + \cdots + S_n = \displaystyle\sum_{r=1}^nt_r = n + (n - 1)\frac{n(n + 1)}{2} = \frac{n}{2}(n^2
  + 1)$.
\item $S_r = \frac{n}{2}[2r + (n - 1)(2r - 1)] = \frac{n}{2}[2r + 2rn - 2r - n + 1] = \frac{n}{2}[2rn - n +
  1]$

  $S_1 + S_2 + \cdots + S_m = \displaystyle\sum_{r=1}^mS_r = \frac{n^2m(m + 1)}{2} - \frac{n(n - 1)m}{2} =
  \frac{1}{2}[m^2n^2 + mn^2 - mn^2 + mn] = \frac{mn}{2}(mn + 1)$.
\item Given below is the diagram for the problem:
  \startplacefigure[location={left, none}]
    \startMPcode
      draw (-1.5cm, -1.5cm)--(1.5cm, -1.5cm)--(1.5cm, 1.5cm)--(-1.5cm, 1.5cm)--cycle;
      draw(-0.6cm, -1.8cm)--(0.6cm, 1.8cm);
      draw(-0.2cm, -0.6cm)--(-0.2cm, -1.5cm);
      draw(-0.2cm, -0.6cm)--(-1.5cm, -0.6cm);
      draw(-0.2cm, -0.6cm)--(-0.2cm, 1.5cm);
      draw(-0.2cm, -0.6cm)--(1.5cm, -0.6cm);
      label.llft("$A(-1, -1)$", (-1.5cm, -1.5cm));
      label.lrt("$B(1, -1)$", (1.5cm, -1.5cm));
      label.urt("$C(1, 1)$", (1.5cm, 1.5cm));
      label.ulft("$D(-1, 1)$", (-1.5cm, 1.5cm));
      label.lrt("$N$", (-0.5cm, -1.5cm));
      label.bot("$1$", ((-1.5cm, -1.5cm) + (-0.5cm, -1.5cm))/2);
      label.bot("$2$", ((1.5cm, -1.5cm) + (-0.5cm, -1.5cm))/2);
    \stopMPcode
  \stopplacefigure
  Let the inclines straight line passing through origin cuts $AB$ at $N$ such that $AN:NB = 1:2$. Let the
  coordinates of $A, B, C, D$ are $(-1, -1), (1, -1), (1, 1), (-1, 1)$. Then $N = (-1/3, -1)$. Thus equation
  of line would be $y = 3x$. Let $(x_1, y_1)$ be the point from where we have drawn perpendiculars to the
  sides. Then length of $\perp$ to $AB = \frac{3x_1 + 1}{2}$, length of $\perp$ to $AD = x_1 + 1$, length
  of $\perp$ to $BC = \frac{3x_1 - 1}{2}$ and length of $\perp$ to $CD = x_1 - 1$. It is now trivial to
  observe that these lengths are in A.P.
\item Let $p, b, h$ be the perpedicular, base, hypotenuse of the right angle triangle such that $b < p < h$
  and $r$ be the common ratio of the G.P. such that $r > 1$. Clearly $h^2 = p^2 + b^2 \Rightarrow b^2r^4 =
  b^2r^2 + b^2 \Rightarrow r^2 = \frac{1 + \sqrt{5}}{2}$.

  Clearly, the greater acute angle will be opposite to $p$ which we let as $\theta$, then

  $\cos\theta = \frac{b}{h} = \frac{1}{r^2} = \frac{1}{1 + \sqrt{5}}$.
\item Let $27, 8, 12$ be the $p$th, $q$th, $k$th terms respectively of a G.P. whose first term is $a$ and
  common ratio is $r$ then $27 = ar^{p - 1}, 8 = ar^{q - 1}, 12 = ar^{k - 1}$.

  $\Rightarrow \frac{27}{8} = r^{p - q} = \left(\frac{3}{2}\right)^3, \frac{12}{8} = r&{k - q} = \frac{3}{2}
  \Rightarrow r^{p - q} = r^{3(k - q)}\Rightarrow p + 2q - 3k = 0$.

  The system of solutions of this equation is $p = 4t, q = t, k = 2t$ where $t\in \mathbb{P}$.
\item Let $10, 11, 12$ be the $p$th, $q$th, $k$th terms respectively of a G.P. whose first term is $a$ and
  common ratio is $r$ then $10 = ar^{p - 1}, 11 = ar^{q - 1}, 12 = ar^{k - 1}$.

  $\Rightarrow \frac{11}{10} = r^{q - p}$ and $\frac{12}{11} = r^{k - q} \Rightarrow
  \left(\frac{11}{10}\right)^{k - q} = r^{(q - p)(k - q)}$ and $\left(\frac{12}{11}\right)^{q - p} == r^{(k
    - q)(q - p)}$

  $\Rightarrow \left(\frac{11}{10}\right)^{k - q} = \left(\frac{12}{11}\right)^{q - p} \Rightarrow (11)^{k -
    q + q - p} = 10^{k - q}12^{q - p} = 5^{k - q}w^{k + q - 2p}3^{q - p}$

  This is possible only if $k - p = 0, k - q = 0, k + q - 2p = 0$ and $q - p = 0$ i.e. $p = q = k = 0$ which
  is not possible as they are distinct.
\item We have $I_n = \displaystyle\int_0^{\frac{\pi}{2}}\cos^nx\cos(nx)dx, I_{n + 1} =
  \displaystyle\int_0^{\frac{\pi}{2}}\cos^{n + 1}x\cos[(n + 1)x]dx$

  $I_{n + 1} = \displaystyle\int_0^{\frac{\pi}{2}}\cos^nx[\cos x\cos[(n + 1)x]dx$

  $\cos nx = \cos[(n + 1)x - x] = \cos(n + 1)x\cos x + \sin(n + 1)x\sin x \Rightarrow \cos(n + 1)x\cos x =
    \cos nx - \sin(n + 1)x\sin x$

  $I_{n + 1} = \displaystyle\int_0^{\frac{\pi}{2}}\cos^nx[\cos nx - \sin(n + 1)x\sin x]dx = I_n -
    \int_0^{\frac{\pi}{2}}\cos^nx\sin x\sin(n + 1)xdx$

  $\displaystyle = I_n + \left[\frac{\cos^{n + 1}x\sin(n + 1)x}{n + 1}\right]_0^{\tfrac{\pi}{2}} -
    \int_{0}^{\tfrac{\pi}{2}}\cos^{n + 1}x\cos(n + 1)xdx$ [we take $u = \sin(n + 1)x$ and $v=\cos^nx\sin x$]

  $= I_n + 0 - 0 - I_{n + 1} \Rightarrow \frac{I_{n + 1}}{I_n} = 2$ and thus, $I_1, I_2, I_3, \ldots$ are in
  G.P.
\item $I_1, I_2, I_3, \ldots$ will be both in A.P. and G.P.if and only if $I_1 = I_2 = I_3 = \cdots = I_n$

  $I_{n + 1} - I_n = \displaystyle\int_{0}^\pi\frac{\sin(2n + 1)x}{\sin x}dx - \int_{0}^\pi \frac{\sin(2n -
  1)x}{\sin x}dx = \int_0^{\pi}\frac{\sin(2n + 1)x - \sin(2n - 1)x}{\sin x}dx$

  $= \displaystyle\int_0^\pi\frac{2\cos2nx\sin x}{\sin x}dx = 2\int_0^\pi \cos 2nx dx =
  \frac{2}{2n}\left[\sin2nx\right]_0^\pi = 0$

  So $I_{n + 1} = I_n$ also, $I_1 = \displaystyle\int_0^\pi\frac{\sin x}{\sin x}dx = \pi$. Hence, $I_1 = I_2
  = I_3 = \cdots = I_n = \pi$ which proves that the terms are both in A.P. and G.P.
\item Let $a, ar, ar^2$ be the sides of the triangle. If $r > 1$ then from the properrties of the triangle
  we have $ar^2 < a + ar \Rightarrow r^2 - r - 1 < 0 \Rightarrow r < \frac{1 + \sqrt{5}}{2}$. If $r < 1$ the
  the triangle will be formed if $ar + ar^2 < a \Rightarrow r^2 + r - 1 > 0 \Rightarrow r > \frac{-1 + xs
    5}{2}$. Hence we have required inequality.
\item $111\ldots 1(91\;\mathrm{digits}) = 10^{90} + 10^{89} + \cdots + 10 + 1 = \frac{10^{91} - 1}{10 - 1}$.

  Since $91 = 13\times 7$ we use $7$ to multiply and divide with $10^7 - 1$ which gives us

  $\frac{10^{91} - 1}{10^7 - 1}.\frac{10^7 - 1}{10 - 1} = (10^{84} + 10^{83} + \cdots + 10 + 1)(10^6 + 10^5 +
  \cdots + 10 + 1)$, which is a composite number.
\item $f(a + k) = f(a) + f(k)\;\because f(x + y) = f(x)f(y)\;\forall\;x, y\in\mathbb{N}$

  $\Rightarrow \displaystyle \sum_{k = 1}^nf(a + k) = \sum_{k=1}^nf(a)f(k) = f(a)[f(1) + f(2) + \cdots +
  f(n)]$

  Given, $f(1) = 2, f(2) = f(1) + f(1) = f(1)f(1) = 2^2, f(3) = f(1) + f(2) = f(1)f(2) = 2^3, \cdots, f(n) =
  2^n$ and $f(a) = 2^a$

  $\Rightarrow \displaystyle \sum_{k = 1}^nf(a + k) = 16[2^n - 1] \Rightarrow 2^a[2 + 2^2 + \cdots + 2^n] =
  2^a2(2^n - 1) = 16(2^n - 1)\Rightarrow a = 3$.
\item Number of students giving wrong answers to at least $i$ questions $= 2^{n - i}$.

  Number of students giving wrong answers to at least $i + 1$ questions $= 2^{n - i - 1}$.

  $\therefore$ Number of students giving wrong answers to exactly $i$ questions $= 2^{n - i} - 2^{n - i -
    1}$. Also, total no. of students giving wrng answers to exactly $n$ questions $= 2^{n - n} = 1$

  $\therefore$ Total no. of wrong answers $= 1(2^{n - 1} - 2^{n - 2}) + 2.(2^{n - 2} - 2^{n - 3}) + \cdots +
  (n - 1)(2^1 - 2^0) + n(2^0) = 2^{n - 1} + 2^{n - 2} + \cdots + 2^0 = 2^n - 1 = 2047 \Rightarrow n = 11$.
\item $S_1 = \frac{1}{1 - \tfrac{1}{2}} = 2, S_2 = \frac{2}{1 - \tfrac{1}{3}} = 3, S_3 = \frac{3}{1 -
  \tfrac{1}{4}}= 4, \cdots$ and so on.

  We have $S_1^2 + S_2^2 + \cdots + S_{2n - 1}^2 = 2^2 + 3^2 + \cdots + (2n - 1)^2 = 1^2 + 2^2 + 3^2 + \cdots +
  (2n)^2 - 1 = \frac{2n(2n + 1)(4n + 1)}{6} - 1 = \frac{n(n + 1)(6n + 1)}{3} - 1$.
\item   \startplacefigure[location={left, none}]
  \startMPinclusions
    vardef square_fractal(expr a, b, c, d) =
      draw a -- b -- c -- d -- cycle;
    enddef;
  \stopMPinclusions
  \startMPcode
    pair a; pair b; pair c; pair d; pair e;
    a := (0cm, 0cm); b := (0cm, 1.5cm); c := (1.5cm, 1.5cm); d := (1.5cm, 0cm);
    label.llft("$A$", a);
    label.ulft("$B$", b);
    label.urt("$C$", c);
    label.lrt("$D$", d);
    square_fractal(a, b, c, d);
    for i:=0 step 1 until 8:
      e := a;
      a := (a + b)/2; b := (b + c)/2; c := (c + d)/2; d := (e + d)/2;
      square_fractal(a, b, c, d);
    endfor;
  \stopMPcode
\stopplacefigure
Let $ABCD$ be the first square and length of sides are $a$. Clearly, sides of second square $=
\sqrt{\frac{a^2}{4} + \frac{a^2}{4}} = \frac{a}{\sqrt{2}}$ $\therefore$. Area of second square $=
\frac{a^2}{2}$. Area of third square $= \frac{a^2}{4}$ and so on.

Total area of innser squares = $\frac{\tfrac{a^2}{2}}{1 - \frac{1}{2}} = a^2 =$ Sum of first square.
\item Let $y = 7 + 2x\log 25 - 5^{x - 1} - 5^{2 - x} \Rightarrow \frac{dy}{dx} = 4\log 5 - 5^{x - 1}\log 5 +
  5^{2 - x}\log 5 = \frac{\log 5}{5^{x + 1}}(5^x - 25)(5^x + 5)$

  Now $y' > 0$ if $x > 2$ and $y' < 0$ if $x < 2$. Since $y$ has only one local maxima at $x = 2$ and has no
  local minima, therefore $y$ has greatest value at $x = 2 \Rightarrow a = 2$ which is first term of G.P.

  $r = \displaystyle\lim_{x\to 0}\int_{0}^x\frac{t^2}{x^2\tan(\pi + x)}dt = \lim_{x\to
    0}\frac{\int_0^xt^2dt}{x^2\tan x}$

  $= \lim_{x\to 0}\frac{x^3}{3x^2\tan x} = \frac{1}{3}\;\therefore \displaystyle\lim_{n \to \infty}\sum_{n
    =1}^nar^{n - 1} = \frac{2}{1 - \tfrac{1}{3}} = 3$.
\item Let $x$ be the first term and $y$ be the common ratio of the G.P. Then $a = xy^{p-1}, b = xy^{q-1}, c
  = xy^{r-1}$

  $(\log a).\vec{i} + (\log b)\vec{j} + (\log c)\vec{k} = (\log x - 1).(\vec{i} + \vec{j} + \vec{k}) + p\log
  y.\vec{i} + q\log y.\vec{j} + r\log y.\vec{k}$

  Dot products of given vectors $= (\log x - 1)(q - r + r - p + p - q) + \log y[p(q - r) + q(r - p) + r(p -
    q)] = 0$

  And therefore the vectors are perpendicular to each other.
\item Pollution after first day $= 20(1 - .8) = 4\%$ and after second day $= 4(1 - .8) = .8$. Let us say
  that it takes $n$ days then $20(1 - .8)^n < .01 \Rightarrow \frac{1}{5^n} < \frac{1}{2000} \Rightarrow 5^n
  > 2000 \Rightarrow n = 5$
\item Let the sides of the triangle are $a, ar, ar^2$ where $a > 0, r > 1$ then from properties of the triangle

  $ar^2 < ar + a \Rightarrow r^2 - r - 1 < 0 \Rightarrow r = \frac{1 \pm \sqrt{5}}{2} \Rightarrow r >
  \frac{-1 + \sqrt{5}}{2}$

  Given that largest angle is twice the smallest one. $\Rightarrow \frac{a}{\sin\theta} =
  \frac{ar^2}{\sin2\theta}$

  $\Rightarrow 2\cos\theta = r^2\Rightarrow r < \sqrt{2}$ so the range is $(1, \sqrt{2})$.
\item Let $r$ be the common ratio then $b = ar, c = ar^2 , d = ar^3$ then $\frac{ax^3 + arx^2 + ar^2x +
  ar^3}{ax^2 + ar^2} = x + r$ leaving no remainder thus given condition is satisfied.
\item Given, $(a^2 + b^2 + c^2)p^2 - 2(ab + bc + cd)p + (b^2 + c^2 + d^2)\leq 0 \Rightarrow (ap - b)^2 + (bp -
  c)^2 + (cp - d)^2\leq 0$

  However, sum of squares cannot be less than zero. $\Rightarrow p = \frac{b}{a} = \frac{c}{b} =
  =\frac{d}{c}$ thus $a, b, c, d$ are in G.P. with common ratio $p$.
\item $\because \log_yx, \log_zy, \log_xz$ are in G.P. $\therefore \left(\frac{\log y}{\log z}\right)^2 =
  \frac{\log x}{\log y}.\frac{\log z}{\log x} = \frac{\log z}{\log y} \Rightarrow \log y = \log z
  \Rightarrow y = z$

  $2x^4 = 2y^4 \Rightarrow x = y$ and $xyz = 8 \Rightarrow x^3 = 8 \Rightarrow x = 2 \Rightarrow x = y = z =
  2$.
\item If $a, b, c, d$ are both in A.P. and G.P. then $a = b = c = d\;\because b = 2\;\therefore$ number of
  such sequences is $1$.
\item We have $\log_x a, a^{x/2}, \log_b x$ are in G.P. $\therefore a^x = \log_xa \log_bx = \frac{\log a\log
  x}{\log x\log b} = \log_ba$

  Taking $\log$ of both sides with base $a$, we get $x = \log_a(\log_b a)$.
\item Let $a$ be the first term and $r$ be the common ratio of the G.P. then

  $t_{m + n} = ar^{m + n - 1} = p$ and $t_{m - n} = ar^{m - n - 1} = q$

  Dividing $r^{2n} = \frac{p}{q} \Rightarrow r = \left(\frac{p}{q}\right)^{\tfrac{1}{2n}}$

  $\Rightarrow a = p.r^{1 - m - n} = p.\left(\frac{p}{q}\right)^{\tfrac{1 - m - n}{2n}}$

  $t_m = ar^{m - 1} = p.\left(\frac{p}{q}\right)^{\tfrac{1 - m -
      n}{2n}}.\left(\frac{p}{q}\right)^{\tfrac{m - 1}{2n}} = p.\left(\frac{p}{q}\right)^{\tfrac{- n}{2n}} = \sqrt{pq}$.

  $t_n = ar^{n - 1} = p.\left(\frac{p}{q}\right)^{\tfrac{1 - m -
      n}{2n}}.\left(\frac{p}{q}\right)^{\tfrac{n - 1}{2n}} =
  p.\left(\frac{q}{p}\right)^{\tfrac{m}{2n}}$.
\item Let $a$ be the first term and $d$ be the c.d. of the A.P. then terms are $a + (p - 1)d, a + (q - 1)d,
  a + (r - 1)d$, which are in G.P. Let $a + (p - 1)d = x, a + (q - 1)d = xy, a + (r - 1)d = xy^2$ where $x$
  is the first term and $y$ is the c.r. of the G.P.

  $(p - q)d = x(1 - r)$ and $(q - r) = xr(1 - r)$. Dividing $r = \frac{q - r}{p - q}$.
\item Let $a$ be the first term and $r$ be the c.r. of the G.P. Then,

  $S_1 = a + ar^2 + ar^4 + \cdots + ar^{2n - 2} = \frac{a(r^{2n} - 1)}{r^2 - 1}, S_2 = ar + ar^3 + \cdots +
  ar^{2n - 1} = \frac{ar(ar^{2n} - 1)}{r^2 - 1}$

  Dividing $S_2/S_2 = r$, which is c.r. of the G.P.
\item $S_n = \frac{a(r^n - 1)}{r - 1} \Rightarrow rS_n = \frac{ar(r^n - 1)}{r - 1}$

  $\displaystyle\sum_{n = 1}^nS_n = S_1 + S_2 + \cdots + S_n = \frac{a(r - 1)}{r - 1} + \frac{a(r^2 - 1)}{r
  - 1} + \cdots + \frac{a(r^{n - 1} - 1)}{r - 1}$

  $\displaystyle(1 - r)\sum_{n = 1}^nS_n = a(1 - r) + a(1 - r^2) + \cdots + a(1 - r^{n - 1}) = na +
  \frac{ar(1 - r^n)}{1 - r}$

  $\Rightarrow rS_n + (1 - r)\sum_{n = 1}^nS_n = na$.
\item The series is $1 + x + xy + x^2y + x^2y^2 + \cdots = [1 + xy + x^2y^2 + \cdots] + x[1 + xy + x^2y^2 +
  \cdots]$

  $= \frac{(x^ny^n - 1)}{xy - 1} + \frac{x(x^ny^n - 1)}{xy - 1} = \frac{(x^ny^n - 1)(1 + x)}{xy - 1}$.
\item $49 = (4\times10) + 9, 4489 = (4\times10^3 + 4\times10^2) + (8\times10) + 9$ and so on.

  $t_k = 4\frac{10^k - 1}{9}.10^k + 8.\frac{10^k - 1}{9} + 1 = 4\frac{10^k - 1}{9}10^k - 4\frac{10^k - 1}{9}
  + 12\frac{10^k - 1}{9} + 1$

  $= 36\frac{10^{2k} - 2.10^k + 1}{81} +12\frac{10^k - 1}{9} + 1 = \left(6\frac{10^k - 1}{9} + 1\right)^2$.
\item $S_m = a + ar + ar^2 + \cdots + ar^{m - 1} = \frac{a(r^m - 1)}{r - 1}$. Let $S$ be required sum then

  $S = \frac{\left(\sum a_i\right)^2 - \sum a_i^2}{2} = \frac{\left(\frac{a(r^m - 1)}{r - 1}\right)^2 - [a^2 + a^2r^2 +
    \cdots + a^2r^{2(m - 1)}]}{2}$

  $2S = \frac{a^2(r^m - 1)}{r - 1}\left[\frac{r^m - 1}{r - 1} - \frac{r^m + 1}{r + 1}\right] = \frac{r}{r +
  1}.\frac{a(r^m - 1)}{r - 1}.\frac{a(r^{m - 1} - 1)}{r - 1} = \frac{r}{r + 1}S_mS_{m - 1}$.
\item $y = \log_{10}x + \log_{10}(x)^{\frac{1}{2}} + \log_{10}(x)^{\frac{1}{4}} + \cdots = \log_{10}x +
  \frac{1}{2}\log_{10}x + \frac{1}{4}\log_{10}x + \cdots$

  $y = \frac{\log_{10}x}{1 - \frac{1}{2}} = 2\log_{10}x$

  $\frac{1 + 3 + 5 + (2y - 1)}{4 + 7 + 10 + \cdots + 3y + 1} = \frac{20}{7\log_{10}x} \Rightarrow
  \frac{y^2}{\frac{y}{2}[8 + (y - 1).3]} = \frac{40}{7y}$

  $\Rightarrow y = 10, x = 10^5$.
\item Let $a = a_1$ be the first term and $r$ to be the common ratio of the G.P., then

  $S = \frac{a(r^n - 1)}{r - 1}, P = a^nr^{1 + 2+ \cdots + (n - 1)} = a^nr^{\tfrac{n(n - 1)}{2}}, T =
  \frac{1}{a}.\frac{1 - \tfrac{1}{r^n}}{1 - \tfrac{1}{r}} = \frac{1}{a}.\frac{r^n - 1}{r - 1}.\frac{1}{r^{n -
      1}}$

  Clearly, $P^2 = \left(\frac{S}{T}\right)^n$.
\item Let $x$ be the first term and $y$ be the c.r. of the G.P. Then $a = xy^{n - 1}$. The next $n$ terms
  will start from $xy^n\Rightarrow b = xy^n.y^{n - 1}$ and similarlry $c = xy^{2n}y^{n - 1}$

  It is clear that $b^2 = ac$ i.e. $a, b, c$ are in G.P.
\item $S_1 = a = \frac{a(1 - r)}{1 - r}, S_2 = \frac{a(1 - r^2)}{1 - r}, \cdots, S_n = \frac{a(1 - r^n)}{1 -
  r}$

  $S_1 + S_2 + \cdots + S_n = \frac{a}{1 - r}[1 + 1 + \cdots +\;\mathrm{to}\;n\;\mathrm{terms}] -
  \frac{ar}{1 - r}[1 + r + r^2 + \cdots + r^{n - 1}]$

  $= \frac{na}{1 - r} - \frac{ar(1 - r^n)}{(1 - r)^2}$.
\item $S_1 = a = \frac{a(1 - r)}{1 - r}, S_3 = \frac{a(1 - r^3)}{1 - r}, \cdots, S_{2n - 1} =
  \frac{a(1 - r^{2n - 1}}{1 - r}$

  $S_1 + S_3 + \cdots + S_{2n - 1} = \frac{a}{1 - r}[1 + 1 + \cdots +\;\mathrm{to}\;n\;\mathrm{terms}] -
  \frac{ar}{1 - r^2}[1 + r^2 + r^4 + \cdots + r^{2(n -1 )}]$

  $= \frac{na}{1 - r} - \frac{ar(1 - r^{2n})}{(1 - r)^2(1 + r)}$.
\item Let $a$ be the first term and $r$ be the common ratio. Then,

  $s = \frac{a}{1 - r}, \sigma = \frac{a^2}{1 - r^2}, S_n = \frac{a(1 - r^n)}{1 - r}$

  $s\left[1 - \left(\frac{s^2 - \sigma^2}{s^2 + \sigma^2}\right)^n\right] = \frac{a}{1 - r}\left[1 -
  \left(\frac{\tfrac{a^2}{(1 - r)^2} - \tfrac{a^2}{1 - r^2}}{\tfrac{a^2}{(1 - r)^2} + \tfrac{a^2}{1 -
      r^2}}\right)^n\right] = \frac{a}{1 - r}\left[1 - \left(\frac{\tfrac{1}{1 - r} - \tfrac{1}{1 +
      r}}{\tfrac{1}{1 - r} + \tfrac{1}{1 + r}}\right)^n \right]$.

  $= \frac{a(1 - r^n)}{1 - r} = S_n$.
\item $\sum_{i < j}a_ia_j = \frac{1}{2}[(a_1 + a_2 + \ldots + a_n)^2 - (a_1^2 + a_2^2 + \ldots + a_n^2)]$

  $= \frac{1}{2}\left[(a + ar + \ldots + ar^{n - 1})^2 - (a^2 + a^2r^2 + \ldots + a^2r^{2(n - 1)})\right]$

  $= \frac{1}{2}\left[\frac{a^2(1 - r^n)^2}{(1 - r)^2 - \frac{a^2(1 - r^{2n})}{1 - r^2}}\right] =
  \frac{1}{2}\left[\frac{a^2(1 - 2r^n + r^{2n})}{(1 - r)^2} - \frac{a^2(1 - r^{2n})}{1 - r^2}\right] =
  \frac{a^2r(1 - r^{n - 1})(1 - r^n)}{(1 - r)^2(1 + r)}$
\item Let $a$ be the first term and $r$ be the common ratio. Then,

  L.H.S. $= \frac{1}{a^2 - a^2r^2} + \frac{1}{a^2r^2 - a^2r^4} + \frac{1}{a^2r^4 - a^2r^6} + \ldots +
  \frac{1}{a^2r^{2(n - 2)} - a^2r^{2(n - 1)}}$

  $= \frac{1}{a^2(1 - r^2)}\left[1 + \frac{1}{r^2} + \frac{1}{r^4} + \ldots + \frac{1}{r^{2(n - 2)}}\right]
  = \frac{1}{a^2(1 - r^2)}.\frac{1 - \frac{1}{r^{2(n - 1)}}}{1 - \frac{1}{r^2}} = \frac{1}{a^2(1 -
    r^2)}.\frac{1 - r^{2n - 2}}{1 - r^2}.\frac{r^2}{r^{2n - 2}}$.
\item Let $a$ be the first term and $r$ be the common ratio. Then,

  L.H.S. $= \frac{1}{a^m + a^mr^m} + \frac{1}{a^mr^m + a^mr^{2m}} + \ldots + \frac{1}{a^mr^{m(n - 2)} +
    a^mr^{m(n - 1)}}$

  $= \frac{1}{a^m(1 + r^m)}\left[1 + \frac{1}{r^m} + \frac{1}{r^{2m}} + \ldots + \frac{1}{r^{m(n - 2)}}\right] =
  \frac{1}{a^m(1 + r^m)}.\frac{1 - \frac{1}{r^{m(n - 1)}}}{1 - \frac{1}{r^m}} = \frac{r^{mn - m} - 1}{a^m(1
    + r^m)(r^{mn - m} - r^{mn - 2m})}$.
\item Let $a$ be the first term and $r$ be the common ratio. Then,

  L.H.S. $= \sqrt{a^2r} + \sqrt{a^2r^5} + \sqrt{a^2r^9} + \ldots + \sqrt{a^2r^{4n - 3}} = a\sqrt{r}(1 + r^2
  + r^4 + \ldots + r^{2(n - 1)}) = a\sqrt{r}.\frac{(r^{2n - 1})}{r^2 - 1}$

  $\sqrt{a_1 + a_3 + \ldots + a_{2n - 1}} = \sqrt{a(1 + r^2 + \ldots + r^{2n - 2})} = \sqrt{a.\frac{r^{2n -
        1}}{r^2 - 1}}$

  $\sqrt{a_2 + a_4 + \ldots + a_{2n}} = \sqrt{ar(1 + r^2 + \ldots + r^{2n - 2})} =
  \sqrt{a\sqrt{r}.\frac{r^{2n - 1}}{r^2 - 1}}$

  $\therefore \sqrt{a_1a_2} + \sqrt{a_3a_4} + \sqrt{a_5a_6} + \ldots + \sqrt{a_{2n - 1}a_{2n}} = \sqrt{a_1 +
    a_3 + \ldots + a_{2n - 1}}\sqrt{a_2 + a_4 + \ldots + a_{2n}}$.
\item Given $1 + x + x^2 + \ldots + x^{23} = 0, 1 + x + x^2 + \ldots + x^{19} = 0$

  $\frac{x^{24} - 1}{x - 1} = 0, \frac{x^{20} - 1}{x - 1} = 0\Rightarrow x^{24} - 1 = 0, x^{20} - 1 = 0\therefore x^{20}.x^4 - 1 = 0 \Rightarrow x^4 - 1 = 0$

  Thus, roots are $-1, \pm i$.
\item $\$a$ will become $a + r.(a) = a(1 + r)$ at the end of second year, $a + ar + r(a + ar) = a + 2ar + ar^2 = a(1
  + r)^2$ at the end of third year, $a + 2ar + ar^2 + r(a + 2ar + ar^2) = a + 3ar + 3ar^2 + ar^3 = a(1 + r)^3$ and so on. So
  amount received for $\$a$ will be $a(1 + r)^{n + 1}$

  Similarly, amount receoved for $\$2a$ will be $2a(1 + r)^n$ and so on.

  Thus, total amount received will be $S = a(1 + r)^{n + 1} + 2a(1 + r)^n + 3a(1 + r)^{n - 1} + \ldots + na(1 + r)$

  $\frac{S}{1 + r} = a(1 + r)^n + 2a(1 + r)^{n - 1} + \ldots + (n - 1)(1 + r) + na$

  Writing first term of second sum against second term of first sum, second term of second sum against third term of first sum and
  so on and subtracting, we get $\frac{rS}{1 + r} = a(1 + r)^{n + 1} + a(1 + r)^{n} + a(1 + r)^{n - 1} + \ldots + a(1 + r) - na$

  $\frac{rS}{1 + r} = a(1 + r)[(1 + r)^n + (1 + r)^{n - 1}  + \ldots + 1]) - na$

  $S = \frac{a(1 + r)^2[(1 + r)^n - 1]}{r^2} - \frac{na(1 + r)}{r}$.
\item $\left(\frac{1}{3} + \frac{1}{3^2} + \frac{1}{3^3} + \ldots \infty\right) = \frac{\frac{1}{3}}{1 -
  \frac{1}{3}} = \frac{1}{2} \Rightarrow (0.16)^{\log_{2.5}\left(\frac{1}{3} + \frac{1}{3^2} + \frac{1}{3^3} + \ldots \infty\right)} =
  \left(\frac{4}{25}\right)^{\log_{\tfrac{5}{2}}\frac{1}{2}} =
  \left(\frac{1}{2}\right)^{\log_{\tfrac{5}{2}}\frac{4}{25}} = \left(\frac{1}{2}\right)^{-2} = 4$.
\item $A = 1 + r^a + r^{2a} + \ldots \;\text{to}\;\infty = \frac{1}{1 - r^a} \Rightarrow r = \left(\frac{A - 1}{A}\right)^{\frac{1}{a}}$

  $B = 1 + r^b + r^{2b} + \ldots \;\text{to}\;\infty = \frac{1}{1 - r^b} \Rightarrow r = \left(\frac{B -
  1}{B}\right)^{\frac{1}{b}}$.
\item $s_1 = \frac{1}{1 - \frac{1}{2}} = 2, s_2 = \frac{2}{1 - \frac{1}{3}} = 3,\ldots, s_n = \frac{n}{1 -
  \frac{1}{n + 1}} = n + 1$

  $s_1 + s_2 + \ldots + s_n = 2 + 3 + \ldots + (n + 1) = \frac{1}{2}n(n + 3)$.
\item $S_1 = \frac{1}{1 - \frac{1}{2}} = 2, S_2 = \frac{2}{1 - \frac{1}{3}} = 3,\ldots S_n = \frac{n}{1 -
  \frac{1}{n + 1}} = n + 1$

  $\text{General term of numerator}\;t_i = S_iS_{n - i + 1} = (i + 1)(n - i + 2) = (n + 1)i - i^2 + (n + 1)$

  $\therefore \text{Sum for numerator}\; = \displaystyle\sum_{i=1}^nt_i = \sum_{i=1}^n [(n + 1)i - i^2 + (n + 1)] =
  \frac{n(n + 1)^2}{2} - \frac{n(n + 1)(2n + 1)}{6} + n(n + 1)$

  $\text{Sum for denominator}\;= 1^2 + 2^2 + \ldots + (n + 1)^2 - 1 = \frac{(n + 1)(n + 2)(2n + 3)}{6} - 1$

  Upon simplification $\displaystyle\lim_{n\to \infty} \frac{S_1S_n + S_2S_{n - 1} + \ldots + S_nS_1}{S_1^2 + S_2^2 +
    \ldots + S_n^2} = \frac{1}{2}$.
\item $f'(x) = 3x^2 + 3$ which yields imaginary roots implying that there is no local maxima. However, $3x^2
  + 3$ is positive for all values of $x$ which means that $f(x)$ is monotonically increasing in $[-5, 3]$
  implying that maximum value will be at $x = 3$

  $f(3) = 27,$ also let $a$ to be the first term and $r$ to be the common ratio then given, $a - ar = f'(0)
  = 3.$ The sum is given as $\frac{a}{1 - r} = 27$ solving these yields $r = \frac{2}{3}, -\frac{4}{3}$ but
  the series is decreasing so $r = \frac{2}{3}$.
\item Let $S = \frac{5}{13} + \frac{55}{13^2} + \frac{555}{13^3} + \ldots \infty$

  $= \frac{5}{9}\left[\frac{10 - 1}{13} + \frac{100 - 1}{13^2} + \frac{1000 - 1}{13^3} + \ldots
  \infty\right] = \frac{5}{9}\left[\frac{10}{13} + \frac{10^2}{13^2} + \frac{10^3}{13^3} + \ldots \infty - \frac{1}{13} - \frac{1}{13^2} -
  \frac{1}{13^3} - \ldots \infty\right]$

  $= \frac{5}{9}\left[\frac{\frac{10}{13}}{1 - \frac{10}{13}} - \frac{\frac{1}{13}}{1 -
    \frac{1}{13}}\right]= \frac{5}{9}\left[\frac{10}{13}.\frac{13}{3} - \frac{1}{13}.\frac{13}{12}\right]=
  \frac{65}{36}$
\item $S = \cos x + \frac{2}{3}\cos x\sin^2x + \frac{4}{9}\cos x\sin^4x + \ldots$

  $= \frac{\cos x}{1 - \frac{2}{3}\sin^2x} = \frac{3\cos x}{3 - 2\sin^2x} = \frac{3\cos x}{2 + \cos 2x}$

  The term $\frac{3\cos x}{2 + \cos 2x}$ is finite for all $x\in \left(-\frac{\pi}{2},\frac{\pi}{2}\right)$

\item Let $a$ be the first term, $b$ be the last term and $n$ be the number of terms of A.P. and G.P.

  Then c.d. of A.P. $= \frac{b - a}{n - 1}$ and c.r. of the G.P. $= \left(\frac{b}{a}\right)^{n - 1}.$ Let $S$ be the sum of $n$
  terms of A.P. and $S'$ the sum of $n$ terms of G.P. then $S = \frac{n}{2}(a + b)$

  $S' = a(1 + r + r^2 + \ldots + r^{n - 1}), S' = a(r^{n - 1} + r^{n - 2} + \ldots + 1)$

  $\therefore S' = \frac{a}{2}[(1 + r^{n - 1}) + (r + r^{n - 2}) + (r^k + r^{n - k - 1}) + \ldots + (r^{n - 1} + 1)]$

  Now, $(r^k + r^{n - k - 1}) - (r^{n - 1} + 1) = (r^k - 1) + r^{n - 1}(r^{-k} - 1)$

  $= (r^k - 1)\left(1 - \frac{r^{n - 1}}{r^k}\right) = (r^k - 1)(1 - r^{n - k - 1})\leq 0$

  $\therefore S'\leq \frac{an}{2}(1 + r^{n - 1}) = \frac{an}{2}\left(1 + \frac{b}{a}\right) = \left(\frac{a
    + b}{2}\right)n = S$

  $\therefore S \geq S'$.
\item Given $a, a_1, a_2, a_3, \ldots$ are in G.P. so $\log a, \log a_1, \log a_2, \ldots$ are in A.P. Let the
  common difference of this A.P. be $d_1.$ Now $\log a_n = \log a + nd_1.$ Further if $d$ be the common difference of the A.P. $b,
  b_1, b_2, \ldots$ then $b_n = b + nd$

  $\therefore \frac{\log a_n - \log a}{b_n - b} = \frac{nd_1}{nd} = \frac{d_1}{d}$

  Let $\log x = \frac{d_1}{d}$ for a fixed positive real number $x.$

  $\Rightarrow \frac{\log a_n - \log a}{b_n - b} = \log x \Rightarrow b_n - b =
  \log_x\left(\frac{a_n}{a}\right)\Rightarrow \log_x a_n - \log_x a = b_n - b \Rightarrow \log_x a_n - b_n =
  \log_x a - b$
\item Given $a + md, a + nd, a + rd$ are in G.P., where $a$ is the first term and $d$ is the c.d. of A.P.

  $\Rightarrow (a + nd)^2 = (a + md)(a + rd)\Rightarrow d(n^2d + 2an) = d(am + ar + mrd)\Rightarrow (n^2 -
  mr)d = a(m + r - rn)$

  $\frac{d}{a} = \frac{m + r - 2n}{n^2 - mr}$

  Given, $m, n, r$ are in H.P. $\therefore n = \frac{2mr}{m + r} \Rightarrow m + r = \frac{2mr}{n}$

  $\therefore \frac{d}{a} = \frac{\frac{2mr}{n} - 2n}{n^2 - mr} = -\frac{2}{n}\;\therefore\; \frac{a}{d} =
  -\frac{n}{2}$
\item Let $r$ be the common ratio of the G.P., then $b = ar, c = ar^2.$ Given, $a - b, c - a, b - c$ are in H.P.

  $\therefore c - a = \frac{2(a - b)(b - c)}{a - b + b - c}$

  $(c - a)^2 = 2(a - b)(b - c)\Rightarrow (ar^2 - a)^2 = 2(a - ar)(ar - ar^2)$

  $a^2(r^2 - 1)^2 = -2a^2(1 - r)r(1 - r)\Rightarrow (r + 1)^2 = -2r \Rightarrow 1 + 4r + r^2 = 0$

  $\Rightarrow a + 4ar + ar^2 = 0 \Rightarrow a + 4b + c = 0$.
\item Let $d_1, d_2, d_3$ be the common differences of the A.P.'s.

  $\Rightarrow S_1 = \frac{n}{2}[2 + (n - 1)d_1]\Rightarrow d = \frac{2(S_1 - n)}{n(n - 1)}$

  $\text{Similalrly}\;d_2 = \frac{2(S_2 - n)}{n(n - 1)}, d_3 = \frac{2(S_3 - n)}{n(n - 1)}$

  $\because d_1, d_2, d_3$ are in H.P. $\therefore \frac{1}{d_2} - \frac{1}{d_1} = \frac{1}{d_3} - \frac{1}{d_2}$

  $\Rightarrow \frac{n(n - 1)}{2(S_2 - n)} - \frac{n(n - 1)}{2(S_1 - n)} = \frac{n(n - 1)}{2(S_3 - n)} -
  \frac{n(n - 1)}{2(S_2 - n)}$

  $\Rightarrow \frac{1}{S_2 - n} - \frac{1}{S_1 - n} = \frac{1}{S_3 -n} - \frac{1}{S_2 -n}\Rightarrow
  \frac{S_1 - S_2}{(S_1 - n)(S_2 - n)} = \frac{S_2 - S_3}{(S_3 - n)(S_2 - n)}$

  $\Rightarrow n = \frac{2S_3S_1 - S_1S_2 - S_2S_3}{S_1 - 2S_2 + S_3}$.
\item Let the digits at hundreds, tens and units places be $a, ar$ and $ar^2$ and the required number be $x,$ then
  $x = 100a + 10a + ar^2$

  Let $y = x - 400 \Rightarrow y = 100(a - 4) + 1 - ar + ar^2$ In the number $y,$ the digit at hundreds place is $a - 4.$ Clearly

  $1\leq a - 4\leq 5\;[\because\;1\leq a \leq 9 \;\text{and}\;a - 4\geq 1] \Rightarrow 5\leq a\leq 9$

  According to question $a - 4, ar, ar^2$ are in A.P. $\therefore\; 2ar = a - 4 + ar^2\Rightarrow a(r - 1)^2 = 4\Rightarrow r - 1 =
  \pm\frac{2}{\sqrt{a}}$

  $\because a$ and $ar$ are integers. $\therefore r$ is a rational number. Thus, $a$ must be a perfect square. $\therefore a = 9$

  Thus, $r = \frac{5}{3}, \frac{1}{3}$ but $r\neq \frac{5}{3}$ othereise $ar = 15\therefore r = \frac{1}{3}\therefore ar = 3, ar^2
  = 1$

  Hence required number is $931$.
\item Given $a, b, c$ are in G.P. Let $r$ be the common ratio of this G.P. then $b = ar$ and $c = ar^2$.

  Given, $\log_c a, \log_b c, \log_a b$ are in A.P.

  $\Rightarrow \frac{\log a}{\log c}, \frac{\log c}{\log b}, \frac{\log b}{\log a}\;\text{are in A.P.}$

  $\Rightarrow \frac{\log a}{\log a + 2\log r}, \frac{\log a + 2\log r}{\log a + \log r}, \frac{\log a + \log r}{\log a}\;\text{are
    in A.P.}$

  $\frac{1}{1 + 2x}, \frac{1 + 2x}{1 + x}, 1 + x\;\text{are in A.P. where}\;\frac{\log r}{\log a} = x$

  $2\left(\frac{1 + 2x}{1 + x} = \frac{1}{1 + 2x} + 1 + x\right)\Rightarrow x(2x^2 - 3x - 3) = 0$

  $2x^2 - 3x - 3 = 0[\because x\neq 0,\;\text{else}\;\log r = 0 \Rightarrow r = 1\;\text{which is not possible as}\; a, b, c\;\text{are
      distinct}]$

  $2d = 1 + x - \frac{1}{1 + 2x} = \frac{2x^2 + 3x}{1 + 2x} = \frac{3x + 3 + 3x}{1 + 2x} = 3\Rightarrow d =
  \frac{3}{2}$.
\item Let the two numbers be $a$ and $b.$ Since $n$ A.M.'s have been inserted between $a$ and $b\;\therefore\;$ common
  difference of A.P., $d = \frac{b - a}{n + 1}$

  Now $p =$ first A.M. $= 2$nd term of A.P. $= a + d = \frac{an + b}{n + 1}$

  Similarly for harmonic series $q = \frac{ab(n + 1)}{bn + a}$

  We know that $x$ will not lie between $\alpha$ and $\beta$ if $(x - \alpha)(x - \beta) > 0$

  $q - p = -\frac{n(a - b)^2}{(bn + a)(n + 1)}$

  $q - \left(\frac{n + 1}{n - 1}\right)^2p = -\frac{(n + 1)(a + b)^2n}{(n - 1)^2(bn + a)}$

  $\Rightarrow (q - p)\left[q - \left(\frac{n + 1}{n - 1}\right)^2p\right] = \frac{n^2(a - b)^2(a + b)^2}{(n
    - 1)^2(bn + a)^2} > 0$.
\item Common difference of A.P. $= q - p$ and common ratio of G.P. $= \frac{q}{p} < 1$

  $s = \frac{p}{1 - \frac{q}{p}} = \frac{p^2}{p - q}$. Let $S_n$ be the sum of $n$ terms of A.P., then

  $S_n = \frac{n}{2}[2p + (n - 1)d] = np + \frac{n(n - 1)d}{2} = np + \frac{n(n - 1)(q - p)p^2}{2p^2} = np -
  \frac{n(n -1)}{2}.\frac{p^2}{s}$.
\item $\because \log_xy, \log_zx, \log_yz$ are in G.P.

  $\Rightarrow (\log_zx)^2 = \log_xy.\log_yz \Rightarrow \left(\frac{\log x}{\log z}\right)^2 = \frac{\log
  y}{\log x}.\frac{log z}{\log y}$

  $\Rightarrow (\log x)^3 = (\log z)^3 \Rightarrow x = z\Rightarrow x = y = z = 4\;\because xyz =
  64\;\text{and}\;2y^3 = x^3 + z^3$.
\item $2(x + 2y) = x + 2x + y \Rightarrow 3y = x, (xy + 5)^2 = (y + 1)^2(x + 1)^2 \Rightarrow (3y^2 + 5) =
  \pm(y + 1)(3y + 1)$

  $\Rightarrow y = 1, \frac{-1\pm2\sqrt{2}i}{3}, x = 3, -1\pm2\sqrt{2}i$.
\item Let $a = 3$ be the first term and $d$ be the common difference of the G.P. then, given

  $(a + 9d)^2 = (a + d)(a + 33d)\Rightarrow a^2 + 18ad + 81d^2 = a^2 + 34ad + + 33d^2 \Rightarrow d =
  \frac{a}{3} = 1$

  So the A.P. is $3, 4, 5, \ldots$.
\item Given, $\sqrt{ab} = \sqrt{cd}, \frac{a^2 + b^2}{2} = \frac{c^2 + d^2}{2}\Rightarrow ab = cd, a^2 + b^2
  = c^2 + d^2$

  $\Rightarrow (a - b)^2 = (c - d)^2, (a + b)^2 = (c + d)^2\Rightarrow a = c, b = d$

  Thus, arithmetic mean of $a^n$ and $b^n$ is equal to the arithmetic mean of $c^n$ and $d^n$ for every integral value of
  $n$.
\item Let $a$ be the first term and $d$ be the common difference of A.P. and thus $d$ will be the first term and $a$
  be the common ratio of the G.P. Given,

  $155 = \frac{10}{2}[2a + (10 - 1)d] \Rightarrow 2a + 9d = 31$

  $d + ad = 9\Rightarrow a = \frac{25}{2}, 2 \Rightarrow d = \frac{2}{3}, 3$

  Thus, A.P. is $2, 5, 8, \ldots$ or $\frac{25}{2}, \frac{79}{6}, \frac{83}{6}, \ldots$ and the G.P. is $3, 6, 12, \ldots$ or
  $\frac{2}{3}, \frac{25}{3}, \frac{625}{6}, \ldots$.
\item Since $a, b, c$ are in H.P. therefore $\frac{1}{a}, \frac{1}{b}, \frac{1}{c}$ are in A.P.

  $\Rightarrow \frac{2}{b} = \frac{1}{a} + \frac{1}{c} \Rightarrow b = \frac{2ac}{a + c} \Rightarrow \frac{3}{b} - \frac{2}{c} =
  \frac{1}{a} + \frac{1}{b} - \frac{1}{c} \;\text{and}\;\frac{3}{b} - \frac{2}{a} = \frac{1}{b} + \frac{1}{c}
  - \frac{1}{a}$

  $\left(\frac{1}{a} + \frac{1}{b} - \frac{1}{c}\right)\left(\frac{1}{b} + \frac{1}{c} - \frac{1}{a}\right) = \left(\frac{3}{b} -
  \frac{2}{c}\right)\left(\frac{3}{b} - \frac{2}{a}\right)$

  $= \frac{9ac - 6ab - 6bc + 4b^2}{acb^2} = \frac{4}{ac} + \frac{9}{b^2} - \frac{6b(a + c)}{acb^2}$

  $= \frac{4}{ac} + \frac{9}{b^2} - \frac{6b}{acb^2}.\frac{2}{b}= \frac{4}{ac} - \frac{3}{b^2}$.
\item Because $a, b, c$ are in H.P. therefore $\frac{2}{b} = \frac{1}{a} + \frac{1}{c}$

  $\frac{a + b}{2a - b} + \frac{b + c}{2c - b} = \frac{\frac{1}{b} + \frac{1}{a}}{\frac{2}{b} - \frac{1}{a}} + \frac{\frac{1}{b} +
  \frac{1}{c}}{\frac{2}{b} - \frac{1}{c}} = \frac{c}{a} + \frac{c}{b} + \frac{a}{b} + \frac{a}{c} =
  \frac{c^2 + a^2}{ac} + \frac{a + c}{b}$

  $= \frac{c^2 + a^2}{ac} + \frac{(a + c)^2}{2ac} = \frac{c^2 + a^2}{ac} - 2 + \frac{(a + c)^2}{2ac} - 2 +
  4 = \frac{(c - a)^2}{ac} + \frac{(a - c)^2}{2ac} + 4 \geq 4$.
\item $b - \frac{a + b}{1 - ab} = \frac{b + c}{1 - bc} - b \Rightarrow \frac{b - ab^2 - a - b}{1 - ab} =
  \frac{b + c - b + b^2c}{1 - bc}$

  $\Rightarrow \frac{-a(1 + b^2)}{1 - ab} = \frac{c(1 + b^2)}{1 - bc}\Rightarrow -a(1 - bc) = c(1 -
  ab)\Rightarrow a + c = 2abc \Rightarrow 2b = \frac{a + c}{ac}$

  $\therefore a, b^{-1}, c$ are in H.P.
\item $x = \frac{1}{1 - a}, y = \frac{1}{1 - b}, z = \frac{1}{1 - c}$

  $a, b, c\;\text{are in A.P.}\Rightarrow 1 - a, 1 - b, 1 - c\;\text{are in A.P.}$

  $\Rightarrow \frac{1}{1 - a}, \frac{1}{1 - b}, \frac{1}{1 - c}\;\text{are in H.P.}\Rightarrow x, y,
  z\;\text{are in H.P.}$
\item Let $a^{\frac{1}{x}} = b^{\frac{1}{y}} = c^{\frac{1}{z}} = k\Rightarrow a = k^x, b = k^y, c = k^z$

  $\because a, b, c \;\text{are in G.P.} \Rightarrow b^2 = ac \Rightarrow k^{2y} = k^{x + z} \Rightarrow 2y =
  x + z$

  $\therefore x, y, z$ are in A.P.
\item $2b = a + c, m = \frac{2ln}{l + n}, b^2m^2 = acln\Rightarrow \left(\frac{a + c}{2}.\frac{2ln}{l +
  n}\right)^2 = acln$

  $\Rightarrow \frac{ln}{(l + n)^2} = \frac{ac}{(a + c)^2}\Rightarrow \frac{(a + c)^2}{ac} = \frac{(l +
  n)^2}{ln}$

  $\Rightarrow \frac{a}{c} + \frac{c}{a} = \frac{l}{n} + \frac{n}{l}\Rightarrow a:c =
  \frac{1}{n}:\frac{1}{l}$

  Now it can be proven that $a:b:c = \frac{1}{n}:\frac{1}{m}:\frac{1}{l}$.
\item The common difference of A.P. $= b - a,$ common ratio of G.P. is $b/a$ and common difference for corresponding
  A.P. of H.P. is $(a - b)/ab$

  $n + 2$th term of A.P. $= a + (n + 1)(b - a) = (n + 1)b - na$

  $n + 2$th term of G.P. $= ar^{n + 1} = \frac{b^{n + 1}}{a^n}$

  $n + 2$th term of H.P. $= \frac{1}{\frac{1}{a} + \frac{(n + 1)(a - b)}{ab}} = \frac{ab}{(n + 1)a - nb}$

  These will be in G.P. if

  $\frac{[(n + 1)b - na]ab}{(n + 1)a - nb} = \frac{b^{2n + 2}}{a^{2n}}\Rightarrow (n + 1)a^{2n + 1}b^2 -
  na^{2n + 2}b = (n + 1)ab^{2n + 2} - nb.b^{2n + 2}$

  $\Rightarrow (n + 1)ab^2[a^{2n} - b^{2n}] = nb[a^{2n + 2} - b^{2n + 2}]\Rightarrow \frac{b^{2n + 2} -
    a^{2n + 2}}{ab(b^{2n} - a^{2n})} = \frac{n + 1}{n}$.
\item $ar^n - a - nd = a\left(1 + \frac{d}{a}\right)^n - a - nd\left[\because r = \frac{a + d}{a}\right]$

  $= a\left[1 + {}^nC_1\left(\frac{d}{a}\right) + {}^nC_2\left(\frac{d}{a}\right)^2 + \ldots +
  {}^nC_n\left(\frac{d}{a}\right)^n\right] - a - nd$

  $= a\left[{}^nC_2\frac{d^2}{a^2} + {}^nC_3\frac{d^3}{a^3} + \ldots + {}^nC_n\frac{d^n}{a^n}\right] > 0 \left(\because
  \frac{d}{a}> 0\right)$.
\item $A = \frac{a + b}{2}, H=\frac{2ab}{a + b}, G = \sqrt{ab}\Rightarrow A = kH\Rightarrow (a + b)^2 = 4kab \Rightarrow A = kG^2$

  Let $b = ma\Rightarrow a^2(1 + m^2) = 4kma^2 \Rightarrow 1 + m^2 = 4km \Rightarrow m = \frac{4k \pm
    \sqrt{16k^2 - 4}}{2} = 2k\pm\sqrt{4k^2 - 1}$

  Also, $(a + b)^2 = 4kab \Rightarrow (a - b)^2 = 4kab - 4ab\because (a - b)^2 \geq 0 \therefore k \geq 1$.
\item Since $n$ means are inserted therefore total no. of terms will be $n + 2.$ Let $d$ be the c.d. of A.P. and
  $d'$ be the c.d of H.P.

  $\Rightarrow d = \frac{b - a}{n + 1}, d' = \frac{a - b}{(n + 1)ab}\Rightarrow p = a + rd = \frac{(n + 1)a
  + r(b - a)}{n + 1}, \frac{1}{q} = \frac{1}{a} + r\frac{a - b}{(n + 1)ab} \Rightarrow q =
  \frac{(n + 1)ab}{r(a - b) + (n + 1)b}$

  $\frac{p}{a} + \frac{b}{q} = \frac{(n + 1)a + r(b - a)}{a(n + 1)} + \frac{r(a - b) + (n + 1)b}{(n + 1)a} = \frac{a + b}{a}$
  which is independent of $n$ and $r$.
\item Let $s$ be the distance between $P$ and $Q.$

  Time taken by train $A = \frac{s}{2x} + \frac{s}{2y} = \frac{s(x + y)}{2xy} = \frac{s}{\text{H.M of}\;x \;\text{and}\;y}$

  Time taken by train $B = \frac{2s}{x + y} = \frac{s}{\text{A.M of}\;x \;\text{and}\;y}$

  So, second train wil reach earlier as A.M. $\geq$ H.M.
\item Let $d$ be the common difference of corresponding A.P. Also, let $H_1$ and $H_n$ be first and last
  H.M.

  $\Rightarrow d = \frac{\frac{1}{c} - \frac{1}{a}}{n + 1} = \frac{ac}{ac(n + 1)}$

  $\frac{1}{H_1} = \frac{1}{a} + \frac{a - c}{ac(n + 1)} \Rightarrow H_1 = \frac{ac(n + 1)}{nc + a}$

  $\frac{1}{H_n} = \frac{1}{a} + \frac{n(a - n)}{ac(n + 1)} \Rightarrow H_n = \frac{ac(n + 1)}{na + c}$

  $H_1 - H_n = \frac{ac(n + 1)}{nc + a} - \frac{ac(n + 1)}{na + c} = \frac{ac(n^2 - 1)(a - c)}{(n^1 + 1)ac +
    n(a^2 + c^2)}$

  Also, given that $n$ is a root of equation $x^2(1 - ac) - x(a^2+c^2) - (1 + ac) = 0$

  $\therefore n^2(1 - ac) - n(a^2 + c^2) - 1 - ac = 0 \Rightarrow n^2 - 1 = (n^2 + 1)ac + n(a^2 + c^2)
  \therefore H_1 - H_n = ac(a - c)$.
\item Let $d$ be the common difference for A.P. and $d'$ be the common difference for H.P., then

  $d = \frac{b - a}{n + 1}, d' = \frac{\frac{1}{b} - \frac{1}{a}}{n + 1} = \frac{a - b}{(n + 1)ab}$

  $A_r = a + rd = a + \frac{r(b - a)}{n + 1} = \frac{(n - r + 1)a + rb}{n + 1}$

  $\frac{1}{H_{n - r + 1}} = \frac{1}{a} + \frac{(n - r + 1)(a - b)}{(n + 1)ab} = \frac{(n - r + 1)a +
  rb}{(n + 1)ab}$

  $\Rightarrow H_{n - r + 1} = \frac{(n + 1)ab}{(n - r + 1)a + rb}\Rightarrow A_rH_{n - r + 1} = ab$.
\item Consider the equation $(x - 1)(x - 2)(x - 3)\ldots(x - 100) = 0.$ Its roots are $1, 2, 3, \ldots, 100$

  So the equation is a polynomial of $x$ of degree $100.$ Coefficient of $x^{100} = 1$

  Now sum of roots of equation taken one at a time

  $1 + 2 + 3 + \ldots + 100 = (-1)^1\frac{\text{coeff. of}\;x^{99}}{\text{coeff. of}\;x^{100}} = -\text{coeff. of}\;x^{99}$

  $\therefore \text{coeff. of}\;x^{99} = -(1 + 2 + 3 + \ldots + 100) = -5050$

  Sum of products of roots taken two at a time = coeff. of $x^{98} = \frac{1}{2}[(1 + 2 + 3 + \ldots + 100)^2 - (1^2 + 2^2 +
    \ldots + 100^2)]$

  $= \frac{1}{2}\left[5050^2 - \frac{100\times101\times102}{6}\right] = 12582075$.
\item $t_1 = 12, 40, 90, 168, 280, 432, \ldots$
  $\Delta t_1 = 28, 50, 78, 112, 152, \ldots, \Delta^2 t_1 = 22, 28, 34, 40, \ldots, \Delta^3 t_1 = 6, 6, 6,
  \ldots$

  $t_n = 12 + 28{}^{n - 1}C_1 + 22.{}^{n - 1}C_2 + 6.{}^{n - 1}C_3$

  $S_n = \displaystyle\sum_{n = 1}^n (12 + 28{}^{n - 1}C_1 + 22.{}^{n - 1}C_2 + 6.{}^{n - 1}C_3)$

  $S_n = 12n + 28.{}^nC_2 + 22.{}^nC_3 + 6.{}^nC_4$

  $= 12n + 28.\frac{n(n - 1)}{2!} + 22.\frac{n(n - 1)(n - 2)}{3!} + 6.\frac{n(n - 1)(n - 2)(n - 3)}{4!}$

  $= \frac{n}{12}(n + 1)(3n^2 + 23n + 46)$.
\item The series and the successive order differences are:

  $10, 23, 60, 169, 494, \ldots$

  $13, 37, 109, 325, \ldots$

  $24, 72, 216, \ldots$

  Here second order differences are in G.P. whose common ratio is $3.$ Let $t_n = a + bn + c.3^{n - 1}$

  $\therefore a + b + c = t_1 = 10, a + 2b + 3c = t_2 = 23, a + 3b + 9c = t_3 = 60$

  $\Rightarrow a = 3, b = 1, c = 6\Rightarrow t_n = 3 + n + 6.3^{n - 1}$

  $S_n = \displaystyle\sum_{n = 1}^n t_n = \frac{1}{2}(n^2 + 7n - 6) + 3^{n + 1}$.
\item Here one factor of the terms is in G.P. i.e. $x.$

  Now the series of the coeff. of terms together with successive order differences are
  $3, 5, 9, 15, 23, 33, \ldots$

  $2, 4, 6, 8, 10, \ldots$

  $2, 2, 2, ,2, \ldots$

  $0, 0, 0, \ldots$

  Hence third order differences are constant. Now,

  $S = 3 + 5x + 9x^2 + 15x^3 + 23x^4 + 33x^5 + \ldots \infty$

  $-3xS = -9x - 15x^2 - 27x^3 - 45x^4 - 69x^5 -\ldots$

  $3x^2S = 9x^2 + 15x^3 + 27x^4 + 45x^5 + \ldots$

  $-x^3S = -3x^3 - 5x^4 - 9x^5 - \ldots$

  Adding, we get $(1 - x)^3S = 3 - 4x + 3x^2$

  $\therefore S = \frac{3 - 4x + 3x^2}{(1 - x)^3}$.
\item Let $t_r$ denote the $r$th term of the series $\frac{1}{n(n - 1)} + \frac{2}{(n - 1)(n - 2)} + \ldots +
  \frac{n - 2}{2.3},$ then

  $t_1 = \frac{1}{n(n - 1)} = \frac{1}{n - 1} - \frac{1}{n}, t_2 = \frac{2}{n - 2} - \frac{2}{n - 1} =
  \frac{2}{n - 2} - \frac{1}{n - 1} - \frac{1}{n - 1}, t_3 = \frac{3}{n - 3} - \frac{3}{n - 2} = \frac{3}{n
    - 3} - \frac{2}{n - 2} - \frac{1}{n - 2}, \ldots, t_{n - 2} = \frac{n - 2}{2} - \frac{n - 2}{3} =
  \frac{n - 2}{2} - \frac{n - 3}{3} - \frac{1}{3}$

  $t_1 + t_2 + \ldots t_n = \frac{n - 2}{2}\left(-\frac{1}{n} - \frac{1}{n - 1} -\frac{1}{n - 2} -\ldots
  -\frac{1}{3}\right)$

  $= \frac{n + 1}{2} - \left(1 + \frac{1}{2} + \frac{1}{3} + \ldots + \frac{1}{n}\right)$

  $\therefore H_n' = \frac{n + 1}{2} - (t_1 + t_2 + \ldots + t_n) = 1 + \frac{1}{2} + \ldots + \frac{1}{n} =
  H_n$.
\item $\tan^{-1}\left(\frac{x}{1 + 1.2x^2}\right) = \tan^{-1}\left(\frac{2x - x}{1 + x. 2x}\right) =
  \tan^{-1}2x - \tan^{-1}x$

  $\tan^{-1}\left(\frac{x}{1 + 2.3x^2}\right) = \tan^{-1}\left(\frac{3x - 2x}{1 + 2x.3x}\right) =
  \tan^{-1}3x - \tan^{-1}2x$

  $\ldots$

  $\tan^{-1}\left(\frac{x}{1 + n(n + 1)x^2}\right) = \tan^{-1}\left(\frac{(n + 1)x - nx}{1 + nx.(n +
    1)x}\right) = \tan^{-1}(n + 1)x - \tan^{-1}nx$

  Adding, we get

  $L.H.S. = \tan^{-1}(n + 1)x - \tan^{-1}x = \tan^{-1}\left(\frac{nx}{1 + (n + 1)x^2}\right) = R.H.S.$
\item The $n$th term of the given series is
  $t_n = \frac{n}{1 + n^2 + n^4} = \frac{n}{(1 + n^2)^2 - n^2} = \frac{1}{2}\left(\frac{1}{1 + n^2 - n} -
  \frac{1}{1 + n^2 + n}\right)$

  $\therefore t_1 = \frac{1}{2}\left(1 - \frac{1}{3}\right), t_2 = \frac{1}{2}\left(\frac{1}{3} -
  \frac{1}{7}\right), t_3 = \frac{1}{2}\left(\frac{1}{7} - \frac{1}{13}\right), \ldots,t_n =
  \frac{1}{2}\left(\frac{1}{1 + n^2 - n} - \frac{1}{1 + n^2 + n}\right)$

  Adding, we get

  $S= \frac{1}{2}\left(1 - \frac{1}{1 + n^2 + n}\right) = \frac{n(n + 1)}{2(1 + n + n^2)}$.
\item $t_n = \tan^{-1}\frac{2n}{2 + n^2 + n^4} = \tan^{-1}\frac{2n}{1 + 1 + n^2 + n^4} =
  \tan^{-1}\frac{2n}{1 + 1 + (n^2 + 1)^2 - n^2} = \tan^{-1}\frac{2n}{1 + (n^2 + n + 1)(n^2 - n + 1)}$
  $= \tan^{-1}\frac{(n^2 + n + 1) - (n^2 - n + 1)}{1 + (n^2 + n + 1)(n^2 - n + 1)} = \tan^{-1}(n^2 + n + 1)
  - \tan^{-1}(n^2 - n + 1)$

  $\therefore t_1 = \tan^{-1}3 - \tan^{-1}1, t_2 = \tan^{-1}7 - \tan^{-1}3,\ldots,t_{n - 1} = \tan^{-1}(n^2
  -n + 1) - \tan^{-1}[(n - 1)^2 - (n - 1) + 1]$

  $t_n = \tan^{-1}(n^2 + n + 1) - \tan^{-1}(n^2 - n + 1)$

  Adding, we get $S_n = \tan^{-1}(n^2 + n + 1) - \tan^{-1}1 = \tan^{-1}\frac{n^2 + n}{n^2 + n + 2}$.
\item $t_n = \frac{n^4}{4n^2 - 1} = \frac{1}{16}\left[\frac{16n^4}{4n^2 - 1}\right] =
  \frac{1}{16}\left[\frac{16n^4 - 1 + 1}{4n^2 - 1}\right] = \frac{1}{16}\left[4n^2 + 1 + \frac{1}{(2n -
    1)(2n + 1)}\right]$

  $= \frac{1}{16}\left[4n^2 + 1 + \frac{1}{2}\left(\frac{1}{2n - 1} - \frac{1}{2n + 1}\right)\right]$

  $S_n = \displaystyle\sum t_n = \frac{1}{4}\sum n^2 + \frac{1}{16}\sum 1 + \frac{1}{32}\sum \left(\frac{1}{2n - 1} -
  \frac{1}{2n + 1}\right) = \frac{1}{4}\left[\frac{n(n + 1)(2n + 1)}{6}\right] + \frac{n}{16} +
  \frac{1}{32}\left(1 - \frac{1}{2n + 1}\right)$

  $= \frac{n}{48}(4n^2 + 6n + 5) + \frac{1}{16}\frac{n}{2n + 1} = \frac{n(4n^2 + 6n + 5)}{48} +
  \frac{n}{16(2n + 1)}$.
\item $t_k = a_ka_{k+1}\ldots a_{k + r - 1}, t_{k+1} = a_{k+1}a_{k+2}\ldots a_{k+r} \therefore a_{k+r}t_k =
  a_kt_{k+1}$

  $[a_1 + (k + r - 1)d]t_k = [a_1 + (k - 1)d]t_{k + 1} \Rightarrow [a_1 + (k - 2)d]t_k - [a_1 + (k -
    1)d]t_{k + 1} = -(1 + r)dt_k$

  Thus,

  $(a - d)t_1 - (a_1 + 0d)t_2 = -(1 + r)dt_1$

  $(a + 0d)t_2 - (a_1 + d)t_3 = -(1 + r)dt_2$

  $\ldots$

  $[a_1 + (n - 2)d]t_n - [a_1 + (n - 1)d]t_{n+1} = -(1 + r)dt_n$

  $(a - d)t_1 - [a_1 + (n - 1)d]t_{n + 1} = -(1 + r)d[t_1 + t_2 + \ldots + t_n]$

  $\therefore t_1 + t_2 + \ldots + t_n = \frac{a_na_{n + 1}\ldots a_{n +r} - a_0a_1\ldots a_r}{(r + 1)d}$.
\item Let $a$ be the first term and $d$ be the common difference of A.P. Let $t_k$ be the $k$th term of the given
  sequence. Then,

  $t_k = \frac{1}{a_ka_{k + 1}\ldots a_{k + r - 1}}, t_{k + 1} = \frac{1}{a_{k + 1}a_{k + 2}\ldots a_{k +
      r}} \Rightarrow a_kt_k = a_{k+r}t_{k + 1}$

  $[a + (k - 1)d]t_k - (a + kd)t_{k + 1} = d(r - 1)t_{k + 1} \therefore (a + 0d)t_1 - (a + d)t_2 = d(r - 1)t_2$
  $(a + d)t_2 - (a + 2d)t_3 = d(r - 1)t_3$

  $\ldots$

  $[a + (n - 2)d]t_{n - 1} - [a + (n - 1)d]t_n = d(r - 1)t_n$

  Adding, we get

  $at_1 - [a + (n - 1)d]t_n = d(r - 1)[t_2 + t_3 + \ldots + t_n]$

  $[a + (r - d)d]t_1 - [a + (n - 1)d]t_n = d(r - 1)S[t_1 + t_2 + \ldots + t_n]$

  $t_1 + t_2 + \ldots + t_n = \frac{1}{(r - 1)d}\left(\frac{a_r}{a_1a_2\ldots a_r} - \frac{a_n}{a_na_{n + 1}\ldots a_{n + r -
      1}}\right)$

  $S_n = \frac{1}{(r - 1)(a_2 - a_1)}\left(\frac{1}{a_1a_2\ldots a_{r - 1}} - \frac{1}{a_{n + 1}a_{n + 2}\ldots a_{n + r -
      1}}\right)$.
\item Let $t_i$ be the $i$th term of the series, then

  $t_i = \frac{1}{i(i + 1)(i + 2)(i + 3)}, t_{i+1} = \frac{1}{(i + 1)(i + 2)(i + 3)(i + 4)}$

  $\Rightarrow it_i = (i + 4)t_{i + 1}\Rightarrow it_i - (i + 1)t_{i + 1} = 3t_{i+1}$

  $\therefore 1.t_1 - 2t_2 = 3t_2, 2.t_2 - 3.t_3 = 3t_3, \ldots,(n - 1).t_i - nt_n = 3t_n$

  Adding, we get

  $t_1 - nt_n = 3(t_1 + t_2 + \ldots + t_n)\Rightarrow 4t_1 - nt_n = 3[t_1 + t_2 + \ldots + t_n]$

  $t_1 + t_2 + \ldots + t_n = \frac{1}{18} - \frac{1}{3(n + 1)(n + 2)(n + 3)}$.
\item $t_n = \frac{n + 2}{n(n + 1)(n + 3)} = \frac{(n + 2)^2}{n(n + 1)(n + 2)(n + 3)}$

  $= \frac{n^2 + 4n + 4}{n(n + 1)(n + 2)(n + 3)} = \frac{n(n + 4)}{n(n + 1)(n + 2)(n + 3)} + \frac{4}{n(n +
  1)(n + 2)(n + 3)}$

  $= \frac{n(n + 1) + 3n}{n(n + 1)(n + 2)(n + 3)} + \frac{4}{n(n + 1)(n + 2)(n + 3)} = \frac{1}{(n + 2)(n +
  3)} + \frac{3}{(n + 1)(n + 2)(n + 3)} + \frac{4}{n(n + 1)(n + 2)(n + 3)}$

  Now that we have found $t_n$ we can find $S_n$ like previous problem.

  $S_n = \frac{29}{36} - \frac{1}{n + 3} - \frac{3}{2(n + 2)(n + 3)} - \frac{4}{3(n + 1)(n + 2)(n + 3)}$.
\item $t_n = \frac{n}{1.3.5.7\ldots (2n - 1)(2n + 1)} = \frac{1}{2}\left[\frac{1}{1.3.5.7\ldots(2n - 1)} -
  \frac{1}{1.3.5.7\ldots(2n + 1)}\right]$

  $\therefore t_1 = \frac{1}{2}\left(1 - \frac{1}{1.3}\right),t_2 = \frac{1}{2}\left(\frac{1}{1.3} -
  \frac{1}{1.3.5}\right),\ldots,t_n = \frac{1}{2}\left(\frac{1}{1.3.5.7\ldots(2n - 1)} -
  \frac{1}{1.3.5.7\ldots(2n + 1)}\right)$

  $S_n = \frac{1}{2}\left[1 - \frac{1}{1.3.5.7\ldots (2n + 1)}\right]$.
\item $t_n = \frac{n + 1}{(2n - 1)(2n + 1)}.\frac{1}{3^n} = \frac{1}{4}\left[\frac{3}{2n - 1} - \frac{1}{2n
    + 1}\right].\frac{1}{3^n} = \frac{1}{4}\left[\frac{1}{2n - 1}.\frac{1}{3^{n - 1}} - \frac{1}{2n +
    1}.\frac{1}{3^n}\right]$

  $\therefore t_1 = \frac{1}{4}\left(\frac{1}{1.1} - \frac{1}{3}.\frac{1}{3}\right), t_2 =
  \frac{1}{4}\left(\frac{1}{3.3} - \frac{1}{5}.\frac{1}{3^2}\right), t_3 =
  \frac{1}{4}\left(\frac{1}{5}.\frac{1}{3^2} - \frac{1}{7}.\frac{1}{3^3}\right),\ldots,t_n =
  \frac{1}{4}\left(\frac{1}{2n - 1}.\frac{1}{3^{n - 1}} - \frac{1}{2n + 1}.\frac{1}{3^n}\right)$

  $S_n = \frac{1}{4}\left[1 - \frac{1}{2n + 1}.\frac{1}{3^n}\right]$.
\item $t_n = \frac{2n - 1}{3.7.11\ldots (4n - 1)} = \frac{1}{2}\left[\frac{1}{3.7.11\ldots (4n - 5)} -
    \frac{1}{3.7.11\ldots (4n + 1)}\right]$

  $t_2 = \frac{1}{2}\left(\frac{1}{3} - \frac{1}{3.7}\right), t_3 = \frac{1}{2}\left(\frac{1}{3.7} -
  \frac{1}{3.7.11}\right),\ldots,t_n = \frac{1}{2}\left(\frac{1}{3.7.11\ldots(4n - 5) -
    \frac{1}{3.7.11\ldots(4n - 1)}}\right)$

  $t_1 + t_2 + \ldots t_n = \frac{1}{3} + \frac{1}{2}\left[\frac{1}{3} - \frac{1}{3.7.11\ldots (4n -
    1)}\right]$

  $S_n = \frac{1}{2} - \frac{1}{2}.\frac{1}{3.7.11\ldots(4n - 1)}$.
\item $t_n = n(1 - a)(1 - 2a)\ldots [a - (n - 1)a], t_n = -\frac{1}{a}(1 - na - 1)(1 - a)(1 - 2a)\ldots [a -
  (n - 1)a] = -\frac{1}{a}[(1 - a)(1 - 2a)\ldots(1 - na) - (1 - a)(1 -2a)\ldots\{a + (n - 1)a\}]$

  $\therefore t_1 = -\frac{1}{a}[(1 - a) - 1], t_2 = -\frac{1}{qa}[(1 - a)(1 - 2a) - (1 - a)], \ldots$

  Adding, we get

  $S_n = \frac{1}{a}[1 - (1 - a)(1 - 2a)\ldots(1 - na)]$.
\item   $t_1 = 1, t_2 = \frac{x}{b_1} = \frac{(x + b_1) - b_1}{b_1} = \frac{x+ b_1}{b_1} - 1, t_3 =
  \frac{x(x + b_1)}{b_1b_2} = \frac{[(x + b_2) - b_2](x + b_1)}{b_1b_2} = \frac{(x + b_1)(x + b_2)}{b_1b_2}
  - \frac{x + b_1}{b_1}$

  $\ldots$

  $t_{n+1} = \frac{(x + b_1)\ldots (x+b_n)}{b_1b_2\ldots b_n} - \frac{(x+b_1)\ldots(x+b_{n -
      1})}{b_1b_2\ldots b_{n - 1}}$

  $\therefore S_n = \frac{(x + b_1)\ldots (x+b_n)}{b_1b_2\ldots b_n}$.
\item $nS_k(n) = n[1^k + 2^k + \ldots + n^k] = 1^k + (1^k + 2.2^k) + (1^k + 2^k + 3.3^k) + \ldots + (1^k +
  2^k + \ldots + n.n^k)$

  $= 1^{k + 1} + [S_k(1) + 2^{k + 1}] + [S_k(2) + 3^{k + 1}] + \ldots + [S_k(n - 1) + n^{k + 1}] = S_k(1) +
  S_k(2) + \ldots + S_k(n - 1) + S_{k + 1}(n)$.
\item $n^3 > 100 \Rightarrow n > 4, n^3 < 100000 \Rightarrow n < 22$

  So $S = 5^3 + 6^3 + \ldots + 21^3, S' = 1^3 + 2^3 + 3^3 + 4^3$

  $S' + S - S' = 1^3 + 2^2 + \ldots + 21^3 - (1^3 + 2^3 + \ldots + 4^3) = 53261$.
\item $S = a + (a + 1) + \ldots + (a + n - 1), = na + \frac{n(n - 1)}{2}$

  $S^2 = n^2a^2 + n^2(n - 1)a + \frac{n^2(n - 1)^2}{4}$

  $t = a^2 + (a + 1)^2 + \ldots + (a + n - 1)^2\Rightarrow nt = n^2a^2 + n^2(n - 1)a + n\displaystyle\sum_{i
  = 1}^{n - 1}i^2$

  Clearly, $nt - S^2$ is independent of $a$.
\item $\displaystyle\sum_{x = 5}^{n + 5}4(x - 3) = \sum_{x = 1}^{n + 5}4(x - 3) - \sum_{x = 1}^{4}4(x - 3) = \frac{4(n +
  5)(n + 6)}{2} - 12(n + 5) - \frac{4.4.5}{2} + 12.4 = 2n^2 + 10n + 8$

  $\therefore P + Q = 12$.
\item Let $S$ be the sum of series, then

  $S = 5^3 + 7^3 + 9^3 + \ldots \;\text{to}\;n\;\text{terms}\; + 2^5(3^3 + 4^3 + 5^3 + \ldots
  \;\text{to}\;n\;\text{terms})$

  $= 1^3 + 3^3 + 5^3 + \ldots \;\text{to}\;(n + 2)\;\text{terms}\; - 1^3 - 3^3 + 2^5(1^3 + 3^3 + 5^3 + \ldots \;\text{to}\;n +
  1\;\text{terms}) - 2^5$

  $= \displaystyle\sum_{i = 1}^{n + 2}(2i - 1)^3 - 28 + 2^5\sum_{i = 1}^{n + 1}(2i - 1)^3 - 32 = n(10n^3 + 96n^2 + 243n +
  540)$.
\item Let $S$ be the sum of the series and $x = \frac{2n + 1}{2n - 1},$ then

  $S = x + 3x^2 + 5x^3 + \ldots$

  $xS = x^2 + 3x^3 + \ldots + (2n - 1)x^{n + 1}$

  $(1 - x)S = x + 2x^2 + 2x^3 + \ldots = x + 2x^2(1 + x + x^2 + \ldots \text{~to~}n - 1\text{~terms}) -(2n -
  1)x^{n+1} = x + \frac{2x^2(1 - x^{n - 1})}{1 - x} - (2n - 1)x^{n+1}$
  $S = \frac{x}{1 - x} + \frac{2x^2(1 - x^{n - 1})}{(1 - x)^2} - \frac{(2n - 1)x^{n+1}}{1 - x} = \frac{x^2 -
    x + 2x^{n + 1} - 2x^2 + (x - 1).(2n - 1)x^{n + 1}}{(x - 1)^2} = n(2n + 1)$.
\item Let $S$ be the sum to $n$ terms and $x = \frac{4n + 1}{4n - 3},$ then

  $S = 1 + 5x + 9x^2 + 13x^3 + \ldots$

  $xS = x + 5x^2 + 9x^3 + \ldots + (4n + 1)x^n$

  $(1 - x)S = 1 + 4x + 4x^2 + 4x^3 + \ldots + 4x^{n - 1} - (4n + 1)x^n$

  $S = \frac{1}{x - 1} + \frac{4x(x^{n - 1} - 1)}{(x - 1)^2} - \frac{(4n + 1)x^n}{(x - 1)} = 4n^2 - 3n$.
\item $t_n = 1.10^{2n} + 2.10^{2n - 1} + 3.10^{n - 2} + \ldots + n.10^{n + 1} + (n + 1)10^n + n.10^n + (n -
  1)10^{n - 2} + \ldots + 3.10^2 + 2.10 + 1$

  $= 10^{2n}\left[1 + 2.\frac{1}{10} + 3.\frac{1}{10^2} + \ldots + n.\frac{1}{10^{n - 1}}\right] + (1 + 2.10
  + 3.10^2 + \ldots + n.10^{n - 1} + (n + 1)10^n) = 10^{2n}S_1 + S_2$
  $S_1 = 1 + 2.\frac{1}{10} + 3.\frac{1}{10^2} + \ldots + n.\frac{1}{10^{n - 1}}$

  $\frac{S_1}{10} = \frac{1}{10} + 2\frac{1}{10^2} + \ldots + (n - 1)\frac{1}{10^{n - 1}} +
  n.\frac{1}{10^n}$

  $S_1 = \frac{100}{81}\left(1 - \frac{1}{10^n}\right) - \frac{90n}{81.10^n}$

  $S_2 = 1 + 2.10 + 3.10^2 + \ldots + (n + 1)10^n$

  $10S_2 = 10 + 2.10^2 + \ldots + n.10^n + (n + 1)10^{n + 1}$

  $S_2 = \frac{1 - 10^{n + 1}}{81} + \frac{(n + 1)10^{n + 1}}{9}$

  Substituting $S_1$ and $S_2$ we obtain $t_n$ as

  $t_n = \left(\frac{10^{n + 1} - 1}{9}\right)^2$. Thus, the numbers in the sequence will be square of odd
  positive integer.
\item $t_n = \frac{2n + 1}{1^2 + 2^2 + \ldots + n^2} = \frac{2n + 1}{\frac{n(n + 1)(2n + 1)}{6}} =
  \frac{6}{n(n + 1)}$

  $\therefore t_1 = \frac{6}{1.2} = 6\left(1 - \frac{1}{2}\right), t_2 = \frac{6}{2.3} = 6\left(\frac{1}{2} -
  \frac{1}{3}\right), \ldots, t_n = \frac{6}{n(n + 1)} = 6.\left(\frac{1}{n} - \frac{1}{n + 1}\right)$

  Adding, we get

  $S = \frac{6n}{n + 1}$.
\item $t_n = \frac{1}{(1 + nx)[1 + (n + 1)x]} = \frac{1}{x}\left(\frac{1}{1 + nx} - \frac{1}{1 + (n +
  1)x}\right)$

  $t_1 = \frac{1}{x}\left(\frac{1}{1 + x} - \frac{1}{1 + 2x}\right), t_2 = \frac{1}{x}\left(\frac{1}{1 + 2x}
  - \frac{1}{1 + 3x}\right), \ldots$

  Adding, we get

  $S_n = \frac{1}{x}\left(\frac{1}{1 + x} - \frac{1}{1 + (n + 1)x}\right) = \frac{n}{(1 + x)[1 + (n +
      1)x]}$.
\item $t_n = \frac{a^{n - 1}}{(1 + a^{n - 1}x)(1 + a^nx)} = \frac{1}{(a - 1)x}\left(\frac{1}{1 + a^{n - 1}x} -
  \frac{1}{1 + a^nx}\right)$

  $t_1 = \frac{1}{(a - 1)x}\left(\frac{1}{1 + x} - \frac{1}{1 + ax}\right), t_2 = \frac{1}{(a -
    1)x}\left(\frac{1}{1 + ax} - \frac{1}{1 + a^2x}\right), \ldots$

  Adding, we get

  $S = \frac{1}{(a - 1)x}\left(\frac{1}{1 + x} - \frac{1}{1 + a^n x}\right)$.
\item $t_n = \frac{1}{\sqrt{2n - 1} + \sqrt{2n + 1}} = \frac{\sqrt{2n + 1} - \sqrt{2n - 1}}{2}$

  $\therefore t_1 = \frac{\sqrt{3}}{2} - \frac{1}{2}, t_2 = \frac{\sqrt{5}}{2} - \frac{\sqrt{3}}{2}, \ldots$

  Adding, we get

  $S = \frac{\sqrt{2n + 1} - 1}{2}$.
\item $t_k = a_ka_{k + 1}, t_{k + 1} = a_{k +1}a_{k + 2}$

  $a_{k + 2}t_k = a_kt_{k + 1}$

  $[a_1 + (k + 1)d]t_k - [a_1 + (k - 1)d]t_{k + 1} = 0$

  $[a_1 + (k - 2)d]t_k - [a_1 + (k - 1)d]t_{k + 1} = -3dt_k$

  $\therefore (a_1 - d)t_1 - (a_1 + 0d)t_2 = -3dt_1$

  $(a_1 + 0d)t_2 - (a_1 + d)t_3 = -3dt_2$

  $\ldots$

  $[a_1 + (n - 2)d]t_n - [a_1+(n - 1)]t_{n + 1} = -3dt_n$

  Adding, we get

  $-3d(t_1 + t_2 + \ldots + t_n) = (a_1 -d)t_1` - [a_1+(n - 1)]t_{n + 1}$

  $S = \frac{[a + (n - 1)d](a + nd)[a + (n + 1)d] - (a - d)a(a + d)}{3d}= \frac{n}{3}[3a^2 + 3nad + (n^2 -
    1)d^2]$.
\item $t_k = a_ka_{k+1}a_{k+2}, t_{k+1}=a_{k+1}a_{k+2}a_{k+3}$

  $a_{k+3}t_k = a_kt_{k+1}$

  $[a_1 + (k + 2)d]t_k = [a_1 + (k - 1)d]t_{k + 1}$

  $[a_1 + (k - 2)d]t_k - [a_1 + (k - 1)d]t_{k+1} = -4dt_k$

  $(a_1 - d)t_1 - (a_1 + 0d)t_2 = -4dt_1$

  $(a_1 + 0d)t_2 - (a_1 + d)t_3 = -4dt_2$

  $\ldots$

  $[a_1 + (n - 2)d]t_n - [a_1+(n - 1)]t_{n + 1} = -4dt_n$

  Adding, we get

  $-4d(t_1 + t_2 + \ldots + t_n) = (a_1 - d)t_1 - [a_1 + (n - 1)]t_{n + 1}$

  $S = \frac{[a + (n - 1)d](a + nd)[a + (n + 1)d][a + (n + 2)d] - (a - d)a(a + d)(a + 2d)}{4d}$

  $= \frac{n}{4}[4a^3 + 6(n + 1)a^2d + 2(2n^2 + 3n - 1)ad^2 + (n^3 - 2n^2 - n - 2)d^3]$.
\item $t_n = \frac{2n + 1}{n^2.(n + 1)^2} = \frac{1}{n^2} - \frac{1}{(n + 1)^2}$

  $t_1 = \frac{1}{1} - \frac{1}{2^2}, t_2 = \frac{1}{2^2} - \frac{1}{3^2}, \ldots$

  Adding, we get

  $S = 1 - \frac{1}{(n + 1)^2} = \frac{n(n + 2)}{(n + 1)^2}$.
\item $t_n = n(n + 1), S_n = \sum(n^2 + n) = \frac{n(n + 1)(2n + 1)}{6} + \frac{n(n +1)}{2}$

  $\Rightarrow S_n = \frac{n(n + 1)(n + 2)}{3}$

  We have proved in earlier that $\sigma_n = \frac{1}{18} - \frac{1}{3(n + 1)(n + 2)(n + 3)}$

  $\therefore \sigma_{n - 1} = \frac{1}{18} - \frac{1}{3n(n + 1)(n + 2)}$

  Now it is trivial to prove that $18S_n\sigma_{n - 1} - S_n = -2$.
\item $t_n = \frac{2n + 3}{n(n + 1)}.\frac{1}{3^n} = \left(\frac{3}{n} - \frac{1}{n +
  1}\right).\frac{1}{3^n}$

  $\therefore t_1 = \left(3 - \frac{1}{2}\right).\frac{1}{3}, t_2 = \left(\frac{3}{2} -
  \frac{1}{3}\right).\frac{1}{3^2}, t_3 = \left(\frac{3}{3} - \frac{1}{4}\right).\frac{1}{3^3}, \ldots$

  Adding, we get

  $S_n = 1 - \frac{1}{n + 1}.\frac{1}{3^n}$.
\item $S = \frac{1}{1^2} + \frac{1}{2^2} + \frac{1}{3^2} + \frac{1}{4^2} + \ldots\infty, S' = \frac{1}{2^2}
  + \frac{1}{4^2} + \frac{1}{6^2} + \ldots \infty\Rightarrow  4S' = \frac{1}{1^2} + \frac{1}{2^2} + \frac{1}{3^2} +
  \frac{1}{4^2} + \ldots\infty$

  $4S' = S \Rightarrow S' = \frac{S}{4}\;\therefore \frac{1}{1^2} + \frac{1}{3^2} + \frac{1}{5^2} + \ldots \infty \Rightarrow  S - S' =
  \frac{3}{4}S = \frac{\pi^2}{8}$.
\item In previous problem we have proved that $\frac{1}{2^2} + \frac{1}{4^2} + \frac{1}{6^2} + \ldots \infty =
  \frac{\pi^2}{24}$ and $\frac{1}{1^2} + \frac{1}{3^3} + \frac{1}{5^2} + \ldots \infty = \frac{\pi^2}{8}$

  $\therefore 1- \frac{1}{2^2} + \frac{1}{3^2} - \frac{1}{4^2} + \ldots \infty = \frac{\pi^2}{8} - \frac{\pi^2}{24} =
  \frac{\pi^2}{12}$.
\item $H_n = 1 + \frac{1}{2} + \frac{1}{3} + \ldots + \frac{1}{n}, = n - n + 1 + \frac{1}{2} + \frac{1}{3} +
  \ldots + \frac{1}{n}$

  $= n -\left(1 - 1\right) - \left(1 - \frac{1}{2}\right) - \left(1 - \frac{1}{3}\right) + \ldots + \left(1
  - \frac{1}{n}\right)$

  $= n - \left(\frac{1}{2} + \frac{2}{3} + \frac{3}{4} + \ldots + \frac{n - 1}{n}\right)$.
\item We can rewrite the question like $\frac{1}{x + 1} - \frac{1}{x + 1} - \frac{2}{x^2 + 1} - \frac{4}{x^4 + 1} -
  \ldots - \frac{2^n}{x^{2^{n}} + 1} = \frac{2^{n + 1}}{x^{2^{n + 1}} - 1}$

  L.H.S. $= \left(\frac{1}{x - 1} - \frac{1}{x + 1}\right) - \frac{2}{x^2 + 1} - \frac{4}{x^4 + 1} - \ldots
  - \frac{2^n}{x^{2^{n}} + 1}$

  $= \left(\frac{2}{x^2 - 1} - \frac{2}{x^2 + 1}\right) - \frac{4}{x^4 + 1} - \ldots - \frac{2^n}{x^{2^{n}}
    + 1}$

  $= \left(\frac{4}{x^4 - 1} - \frac{4}{x^4 + 1}\right) - \ldots - \frac{2^n}{x^{2^{n}} + 1}$. Progreessing
  similarly we obtain R.H.S.
\item Multiplying and dividing by $1 - \frac{1}{3},$ we get
  L.H.S. $= \frac{\left(1 - \frac{1}{3}\right)}{\left(1 - \frac{1}{3}\right)}\left(1 + \frac{1}{3}\right)\left(1 +
  \frac{1}{3^2}\right)\left(1 + \frac{1}{3^4}\right) \ldots \left(1 + \frac{1}{3^{2^{n}}}\right)$

  $= \frac{1}{\left(1 - \frac{1}{3}\right)}\left(1 - \frac{1}{3^2}\right)\left(1 + \frac{1}{3^2}\right)\left(1 +
  \frac{1}{3^4}\right) \ldots \left(1 + \frac{1}{3^{2^{n}}}\right)$

  $= \frac{1}{\left(1 - \frac{1}{3}\right)}\left(1 - \frac{1}{3^4}\right)\left(1 + \frac{1}{3^4}\right) \ldots \left(1 +
  \frac{1}{3^{2^{n}}}\right)$

  Proceeding similarly we obtain the R.H.S.
\item Since A.M $\geq$ G.M.

  $\therefore \frac{x + y}{2}\geq \sqrt{xy}, \frac{y + z}{2}\geq \sqrt{yz}, \frac{x + z}{2}\geq \sqrt{zx}$

  $\frac{(x + y)(y + z)(z + x)}{8}\geq xyz \Rightarrow (1 - x)(1 - y)(1 - z)\geq 8xyz$.
\item Since A.M $\geq$ H.M.

  $\therefore \frac{a + b + c}{3}\geq \frac{3}{\frac{1}{a} + \frac{1}{b} + \frac{1}{c}} \Rightarrow (a + b + c)\left(\frac{1}{a} +
  \frac{1}{b} + \frac{1}{c}\right)\geq 9$.
\item Taking A.M. and G.M of $7$ numbers $\frac{a}{2}, \frac{a}{2}, \frac{b}{3}, \frac{b}{3}, \frac{b}{3},
  \frac{c}{2}, \frac{c}{2},$ we get

  $\frac{2.\frac{a}{2} + 3.\frac{b}{3} + 2.\frac{c}{2}}{7}\geq
  \left[\left(\frac{a}{2}\right)^2\left(\frac{b}{3}\right)^3\left(\frac{c}{2}\right)^2\right]^{\frac{1}{7}}\Rightarrow
  \frac{3}{7}\geq \left(\frac{a^2b^3c^2}{2^23^32^2}\right)^{\frac{1}{7}}\Rightarrow \frac{3^7}{7^7}\geq
  \frac{a^2b^3c^2}{2^23^32^2}\Rightarrow a^2b^3c^2 \leq \frac{3^{10}2^4}{7^7}$.
\item $\displaystyle\sum_{i=1}^na_ib_i = \sum_{i=1}^na_i(1 - a_i) = \sum_{i=1}^na_i - \sum_{i=1}^na_i^2 = na -
  \sum_{i=1}^n(a_i - a + a)^2$

  $\displaystyle= na - \sum_{i=1}^n[(a_i - a)^2 + a^2 + 2a(ai - a)] = na - \sum_{i=1}^n(ai - a)^2 - na^2 +
  2a\sum_{i=1}^n(a_i - na)$

  $\displaystyle= na(1 - a) - \sum_{i=1}^n(a_i - a)^2 = nab - \sum_{i=1}^n(a_i - a)^2, \because na + nb =
  \sum_{i=1}^n(a_i + b_i) = n \therefore a + b = 1$.
\item Let $a_{n+1}$ be a number such that $|a_{n+1}| = |a_n + 1|$

  Squaring all the numbers, we get

  $a_1^2 = 0, a_2^2 = a_1^2 + 2a_1 + 1, a_3^2 = a_2^2 + 2a_2 + 1,\ldots, a_n^2 = a_{n-1}^2 + 2a_{n-1} + 1,
  a_{n+1}^2 = a_n^2 + 2a_n + 1$

  Adding, we get

  $a_1^2 + a_2^2 + \ldots + a_n^2 + a_{n+1}^2 = a_1^2 + a_2^2 + \ldots + a_n^2 + 2(a_1 + a_2 + \ldots + a_n)
  + n$

  $\Rightarrow 2(a_1 + a_2 + \ldots + a_n) = -n + a_{n+1}^2 \geq -n \Rightarrow (a_1 + a_2 + \ldots +
  a_n)/n\geq -1/2$.
\item We know that A.M. $\geq$ G.M.

  $\Rightarrow \frac{a + b}{2}\geq \sqrt{ab}, \frac{b + c}{2}\geq \sqrt{bc}, \frac{a + c}{2}\geq \sqrt{ac}$

  Multiplying, we get $(a + b)(b + c)(c + a) \geq 8abc$.
\item We know that A.M $\geq$ H.M.

  $\Rightarrow \frac{x + y + z}{3}\geq \frac{3}{\frac{1}{x} + \frac{1}{y} + \frac{1}{z}} \Rightarrow
  \frac{1}{x} + \frac{1}{y} + \frac{1}{z}\geq \frac{9}{a}$.
\item We know that A.M $\geq$ G.M.
  $\Rightarrow \frac{1 + 3 + 5 + \ldots + (2n - 1)}{n}\geq (1.3.5.\ldots(2n - 1))^{\frac{1}{n}}$

  $\Rightarrow \frac{n^2}{n}\geq (1.3.5.\ldots(2n - 1))^{\frac{1}{n}} \Rightarrow n^n \geq 1.3.5\ldots (2n - 1)$.
\item We consider seven numbers five of which are $2 + x$ and remaining four are $7 - x.$ Now, we know that A.M
  $\geq$ G.M.

  $\Rightarrow \frac{4.\frac{7 - x}{4} + 5.\frac{2 + x}{5}}{9}\geq \left[\left(\frac{7 - x}{4}\right)^4\left(\frac{2 +
    x}{5}\right)^5\right]^{\frac{1}{9}} \Rightarrow \frac{9}{9}\geq \left[\left(\frac{7 -
    x}{4}\right)^4\left(\frac{2 + x}{5}\right)^5\right]^{\frac{1}{9}}$

  $\Rightarrow (7 - x)^4(2 + x)^5 \leq 4^4.5^5$. So the greatest value would be $4^4.5^5$.
\item We know that A.M $\geq$ H.M.

  $\Rightarrow \frac{a + b}{2}\geq \frac{2ab}{a + b}, \frac{b + c}{2}\geq \frac{2bc}{b + c}, \frac{c + a}{2}\geq
  \frac{2ca}{c + a}$

  $\Rightarrow \frac{a + b + c}{2}\geq \frac{bc}{b + c} + \frac{ca}{c + a} + \frac{ab}{a + b}$.
\item $(a - b)^2 \geq 0, (b - c)^2\geq 0, (c - a)^2\geq 0$

  $\Rightarrow \frac{(a - b)^2}{ab}\geq 0, \frac{(b - c)^2}{bc}\geq 0, \frac{(c - a)^2}{ac}\geq 0
  \Rightarrow \frac{a^2 + b^2}{ab}\geq 2, \frac{b^2 + c^2}{bc}\geq 2, \frac{c^2 + a^2}{ca}\geq 2$

  $\Rightarrow \frac{a}{b} + \frac{b}{a} + \frac{b}{c} + \frac{c}{b} + \frac{c}{a} + \frac{a}{c}\geq
  6\Rightarrow \frac{b + c}{a} + \frac{c + a}{b} + \frac{a + b}{c}\geq 6$.
\item We know that A.M. $\geq$ H.M.
  $\frac{x_1 + x_2 + \ldots + x_n}{n}\geq \frac{n}{\left(\frac{1}{x_1} + \frac{1}{x_2} + \ldots +
  \frac{1}{x_n}\right)}$

  $\Rightarrow (x_1 + x_2 + \ldots + x_n)\left(\frac{1}{x_1} + \frac{1}{x_2} +
  \ldots + \frac{1}{x_n}\right) \geq n^2$
\item We know that A.M $\geq$ G.M.Considering $1$ and $x^{2n}\Rightarrow \frac{1 + x^{2n}}{2}\geq \sqrt{1.x^{2n}} = x^n$
  Considering $1$ and $y^{2m}\Rightarrow \frac{1 + y^{2m}}{2}\geq \sqrt{1.y^{2m}} = y^m$

  Myltiplying. we get

  $(1 + x^{2n})(1 + y^{2m})\geq 4x^ny^m\Rightarrow \frac{x^ny^m}{(1 + x^{2n})(1 + y^{2m})}\leq \frac{1}{4}$.
\item Let $b - c = x, c - a = y$ and $a - b = z,\Rightarrow x + y + z = 0.$ This also implies that $a + b - 2c = x -
  y, b + c - 2a = y - z, c + a - 2b = z - x$

  Clearly, $x + y + z = 0$

  Given, $\frac{(x - y)^2 + (y - z)^2 + (z - x)^2}{3} = \frac{x^2 + y^2 + z^2}{3} \Rightarrow x^2 + y^2 +
  z^2 - 2xy - 2yz - 2zx = 0$

  $\Rightarrow (x + y + z)^2 = 4(xy + yz + zx)\Rightarrow xy + yz + zx = 0\Rightarrow (c - a)(a - b) + (a -
  b)(b - c) + (c - a)(b - c) = 0$

  $\Rightarrow ca - bc - a^2 + ab + ab - ca - b^2 + bc + bc - c^2 - ab + ca = 0\Rightarrow ab + bc + ca -
  a^2 - b^2 - c^2 = 0\Rightarrow (a - b)^2 + (b - c)^2 + (c - a)^2 = 0\Rightarrow a = b = c$
\stopitemize
