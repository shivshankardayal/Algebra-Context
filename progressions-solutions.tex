% -*- mode: context; -*-
\chapter{Progressions}
\startitemize[n, 1*broad]
\item Given $t_n = 2n^2 + 1 \Rightarrow t_{n - 1} = 2(n - 1)^2 + 1$

  $\therefore d = t_n - t_{n - 1} = 4n - 2$, which is not constant. Hence, the sequence is not in A.P.
\item Given, $t_1 = 1, t_2 = 2$ and $t_{n+2} = t_n + t_{n + 1}$

  $\therefore t_3 = t_1 + t_2 = 3, t_4 = t_2 + t_3 = 5, t_5 = t_3 + t_4 = 8$.
\item Given $t_n = 3n + 5 \Rightarrow t_1 = 3\times1 + 5 = 8, t_2 = 3\times2 + 5 = 11, t_3 = 3\times3 + 5 = 14$. So the seuquence
  is $8, 11, 14, \ldots, 3n + 5$.
\item Given $t_n = 2n^2 + 3 \Rightarrow t_1 = 2\times1^2 + 3 = 5, t_2 = 2\times2^2 + 3 = 11, t_3 = 2\times3^2 + 5 = 23$. So the
  sequence is $5, 11, 23, \ldots, 2n^2 + 3$.
\item Given, $t_n = \frac{3n}{2n + 4}\Rightarrow t_1 = \frac{3\times1}{2\times1 + 4} = \frac{3}{6} = \frac{1}{2}, t_2 =
  \frac{3\times2}{2\times2 + 4} = \frac{6}{8} = \frac{3}{4}, t_3 = \frac{3\times3}{2\times3 + 4} = \frac{9}{10}$. So the
  sequence is $\frac{1}{2}, \frac{3}{4}, \frac{9}{10}, \cdots, \frac{3n}{2n + 4}$.
\item Given, $t_1 = 2, t_{n + 1} = \frac{2t_n + 1}{t_n + 3} \Rightarrow t_2 = \frac{2t_1 + 1}{t_1 + 3} = \frac{2\times1 + 1}{1 + 3}
  = \frac{3}{4}, t_3 = \frac{2t_2 + 1}{t_2 + 3} = \frac{2\times\tfrac{3}{4} + 1}{\tfrac{3}{4} + 3} = \frac{10}{15} = \frac{2}{3}$.
  So the sequence is $2, \frac{3}{4}, \frac{2}{3}, \cdots$.
\item Given, $t_n = 4n^2 + 1 \Rightarrow t_{n - 1} 4(n - 1)^2 + 1$

  $\therefore d = t_n - t_{n - 1} = 8n - 4$, which is not constant. Hence the sequence is not in A.P.
\item Given $t_n = 2an + b \Rightarrow t_{n - 1} = 2a(n - 1) + b$

  $\therefore d = t_n - t_{n - 1} = 2a$. which is a constant. Hence the sequence will be an A.P.
\item Given, $t_1 = 3, t_2 = 3, t_3 = 6$ and $t_{n + 2} = t_n + t_{n + 1}$

  $\therefore t_4 = t_2 + t_3 = 3 + 6 = 9$ and $t_5 = t_3 + t_4 = 6 + 9 = 15$.
\item $t_1 = 1 = a + b + c, t_2 = 5 = 4a + 2b + c$ and $t_3 = 11 = 9a + 3b + c$

  $\therefore t_2 - t_1 = 4 = 3a + b$ and $t_3 - t_2 = 6 = 5a + b$

  $\Rightarrow 2a = 2 \Rightarrow a = 1 \Rightarrow b = 1 \Rightarrow c = -1$

  $\Rightarrow t_{10} = 1\times10^2 + 1\times10 - 1 = 109$.
\item Difference between successive terms i.e. commond difference, $d = 12 - 9 = 15 - 12 = 18 - 15 = 3$ which is a constant, hence,
  the given sequence is an A.P.

  Here first term $t_1 = 9$ and $d = 3 \therefore t_{16} = 9 + (16 - 1)3 = 54$ and $t_n = 9 + (n - 1)3 = 3(n + 2)$.
\item $t_1 = \log a, t_2 = \log(ab) = \log a + \log b, t_3 = \log(ab^2) = \log a + 2\log b$

  $t_2 - t_1 = t_3 - t_2 = \log b$. Clearly, $t_1 = \log a, d = \log b$ which is constant so the sequence is an A.P.

  $\therefore t_n = \log a + (n - 1)\log b = \log(ab^{n - 1})$.
\item Given, $t_n = 5 - 6n \Rightarrow t_1 = 5 - 6 = -1$

  $S_n = \frac{n}{2}[t_1 + t_n] = n(2 - 3n)$.
\item $d = 7 - 3 = 11 -7 = 4, t_n = 407 = 3 + (n - 1)d \Rightarrow n = \frac{404}{4} + 1 = 102$.
\item Since $a, b, c, d, e$ are in A.P. $\therefore a + e = b + d = 2c = k$(say)

  $\therefore a - 4b + 6c - 4d + e = (a + e) - 4(b + d) + 3.2c = k - 4k + 3k = 0$.
\item Let $a$ be the first term and $d$ be the common difference of the given A.P.

  Given, $5t_5 = 8t_8 \Rightarrow 5a + 20d = 8a + 56d \Rightarrow 3a = -36d \Rightarrow a = -12d$

  $\Rightarrow t_{13} = a + 12d = 0$.
\item Let $n$th term be the smallest positive number. From the sequence we obtain that $t_1 = 25$ and $d = -2\frac{1}{4} = -\frac{9}{4}$.

  Then $t_n > 0 \Rightarrow 25 - (n - 1)\frac{9}{4} > 0\Rightarrow n < \frac{25\times4}{9} + 1 \Rightarrow n = 12$.
\item The given pay scale represents an A.P. with $t_1 = 700, d = 40$ and $t_n = 1500$.

  $\therefore t_n = t_1 + (n - 1)d \Rightarrow n = \frac{t_n - t_1}{d} + 1 = \frac{1500 - 700}{40} + 1 = 21$.

  Thus, the person will reach maximum payment in $21$ years.
\item Let $a$ be the first term and $d$ be the common difference of the A.P. According to the question,

  $t_7 = a + 6d = 34$ and $t_{13} = a + 12d = 64$

  Subtracting $6d = 30 \Rightarrow d = 5 \Rightarrow a = 4$. So the A.P. is $4, 9, 14, \ldots$.
\item If $55$ is the $n$th term then $n$ will have to be an integer. From the given sequence $a = 1, d = 3 - 1 = 5 - 3 = 2$.

  $55 = 1 + (n - 1)2 \Rightarrow n = 28$, which is an integer and hence, $55$ will be $28$th term of the A.P.
\item From the given sequence $a = 2000, d = 1995 - 2000 = 1990 - 1995 = -5$.

  Let $n$th term be first negative term, then, $a + (n - 1)d < 0 \Rightarrow 2000 -(n - 1)5 < 0$

  \Rightarrow $n > 401 \Rightarrow n = 402 \Rightarrow t_{402} = 2000 - (402 - 1)5 = -5$.
\item Common different of the sequence $2, 4, 6, 8, \ldots$ is $2$ and common difference of the seqquence $3, 6, 9, \ldots$ is $3$.

  Thus, common terms will have a common different which is L.C.M. of these two commond differences i.e. $6$.

  Last term of first sequence is $200$ and last term of second sequence is $240$. Clearly, last identical(common) number will be
  less than $200$. We also observe that $6$ is the first identical term. Let there be $n$ such terms. Then

  $6 + (n - 1)6 \leq 200 \Rightarrow n\leq \frac{194}{6} + 1 \Rightarrow n = 33$. Thus there will be $33$ identical terms in the two
  given A.P.
\item Clearly the first number of three digits divisible by $5$ is $100$; while the last such number is $995$. Since these numbers
  are all divisible by $5$ they will form an A.P. with common difference $5$.

  Clearly, $t_1 = 100, t_n = 995, d = 5$ and we have to find $n$.

  $t_n = 995 = 100 + (n - 1)5\Rightarrow n = 180$.
\item Given sequence is $4, 9, 14, \ldots$. So $a = 4, d = 9 - 4 = 14 - 9 = 5$. Let $105$ be $n$th term of this A.P. then $n$ has
  to be an integer for this assumption to be true.

  $105 = 4 + (n - 1)5 \Rightarrow n = \frac{106}{5}$ which is not an integer and therefore $105$ is not a term in the given A.P.
\item This problem is same as problem $21$ and has been left as an exercise.
\item This problem is same as problem $22$ and has been left as an exercise.
\item Let $a$ be the first term and $d$ be the common difference of the A.P. Given,

  $mt_m = nt_n \Rightarrow ma + (m - 1)md = na + (n - 1)nd \Rightarrow (m - n)a = (n^2 - n - m^2 + m)d$

  $\Rightarrow a = -(m + n - 1)d\,\therefore t_{m + n} = a + (m + n - 1)d = 0$.
\item Let $x$ be the first term and $y$ be the common difference of the A.P. Then,

  $a = x + (p - 1)y, b = x + (q - 1)y, c = x + (r - 1)y$

  We have to prove that $a(q - r) + b(r - p) + c(p - q) = 0$.

  Substituting the values of $a, b$ and $c$ in the above equation

  L.H.S. $= [x + (p - 1)y](q - r) + [x + (q - 1)y](r - p) + [x + (r - 1)y](p - q)$

  $= x(q - r + r - p + p - q) + y[(p - 1)(q - r) + (q - 1)(r - p) + (r - 1)(p - q)]$

  $= 0 =$ R.H.S.
\item First number after $100$ which is divisible by $7$ is $105$. The last number divisible by $7$ before $1000$ is $994$.

  Let $n$ be the numbers divisible by $7$ between $100$ and $1000$. Then $994 = 105 + (n - 1)7$

  $\Rightarrow n = 128$. Then no. of numbers not divisible by $7$ is $1000 - 100 - 128 = 772$.
\item Let $x$ be the first term and $y$ be the common difference of the A.P. Then,

  $a = x + (p - 1)y, b = x + (q - 1)y, c = x + (r - 1)y$

  We have to prove that $(a - b)r + (b - c)p + (c - a)q = 0$

  Substituting the values of $a, b$ and $c$ in the above equation

  L.H.S. $= (p - q)yr + (q - r)yp + (r - p)yq = 0 =$ R.H.S.
\item Let the numbers in A.P. be $a - d, a$ and $a + d$. Given their sum is $27$ and sum of squares is $293$.

  $\therefore a - d + a + a + d = 27 \Rightarrow a = 9$

  $\therefore (a - d)^2 + a^2 + (a + d)^2 = 293 \Rightarrow 3a^2 + 2d^2 = 293 \Rightarrow 3\times81 + 2d^2 = 293$

  $\Rightarrow 2d^2 = 50 \Rightarrow d = \pm5$

  So the numbers are $4, 9, 14$ or $14, 9, 4$.
\item Let the numbers in A.P. be $a - 3d, a - d, a + d, a + 3d$. Given their sum is $24$ and product is $945$.

  $\therefore a - 3d + a - d + a + d + a + 3d = 24 \Rightarrow 4a = 24 \Rightarrow a = 6$

  Also, $(a - 3d)(a - d)(a + d)(a + 3d) = 945 \Rightarrow (a^2 - 9d^2)(a^2 - d^2) = 945$

  $\Rightarrow a^4 - 10a^2d^2 + 9d^4 = 945 \Rightarrow 9d^4 - 360d^2 + 1296 - 945 = 0$

  $\Rightarrow 9d^4 - 360d^2 + 351 = 0 \Rightarrow d^4 - 40d^2 + 39 = 0$

  $\Rightarrow (d^2 - 1)(d^2 - 39) = 0$. Since the numbers are integers $\Rightarrow d^2 \neq 39$.

  $\Rightarrow d = \pm 1$. So the numbers are $3, 5, 7, 9$ or $9, 7, 5, 3$.
\item Let $a$ be the first term and $d$ be the common ratio of the A.P. Given,

  $t_p = a + (p - 1)d = q$ and $t_q = a + (q - 1)d = p$

  $\Rightarrow (p - q)d = q - p \Rightarrow d = -1 \Rightarrow a = p + q - 1$

  $\Rightarrow t_{p + q} = a + (p + q - 1)d = p + q - 1 - (p + q - 1) = 0$.
\item Let $a$ be the first term and $d$ be the common ratio of the A.P.

  $\Rightarrow t_m = a + (m - 1)d, t_{2n + m} = a + (2n + m - 1)d$

  $\Rightarrow t_m + t_{2n + m} = 2a + (2m + 2n - 2)d = 2[a + (m + n - 1)d] = 2t_{m + n}$
\item Let the three numbers be $a - d, a, a + d$. Given that their sum is $15$ and sum of their square is $83$.

  $\Rightarrow a - d + a + a + d = 15 \Rightarrow 3a = 15 \Rightarrow a = 5$

  $\Rightarrow (a - d)^2 + a^2 + (a + d)^2 = 83 \Rightarrow 3a^2 + 2d^2 = 83 \Rightarrow 3\times5^2 + 2d^2 = 83^2$

  $\Rightarrow d = \pm2$. So the numbers are $3, 5, 7$ or $7, 5, 3$.
\item This problem is similar to previous problem and has been left as an exercise.
\item Let the three numbers be $a - d, a, a + d$. Given their sum as $12$ and sum of cubes as $408$.

  $\therefore a - d + a + a + d = 12 \Rightarrow 3a = 12 \Rightarrow a = 4$

  $\therefore (a - d)^3 + a^3 + (a + d)^3 = 3a^3 + 6ad^2 = 408 \Rightarrow 24d^2 = 216 \Rightarrow d = \pm 3$

  Hence, the numbers are $1, 4, 7$ or $7, 4, 1$.
\item Let the numbers in A.P. be $a - 3d, a - d, a + d, a + 3d$. Given their sum is $24$ and product of first and
  fourth to product of second and third is $2:3$.

  $\therefore a - 3d + a - d + a + d + a + 3d = 20 \Rightarrow 4a = 20 \Rightarrow a = 5$

  $\therefore \frac{(a - 3d)(a + 3d)}{(a - d)(a + d)} = \frac{2}{3}$

  $\Rightarrow 3a^2 - 27d^2 = 2a^2 - 2d^2 \Rightarrow a^2 = 25d^2 \Rightarrow d = \pm1$.

  Therefore numbers are $2, 4, 6, 8$ or $8, 6, 4, 2$.
\item Let the three numbers be $a - d, a, a + d$. Given their sum is $-3$ and product is $8$.

  $\therefore a - d + a + a + d = -3 \Rightarrow 3a = -3 \Rightarrow a = -1$

  $\therefore (a - d).a.(a + d) = 8 \Rightarrow a^2 - d^2 = -8 \Rightarrow d = \pm3$

  Hence the numbers are $-4, -1, 2$ or $2, -1, -4$.
\item This problem is similar to problem 38 and has been left as an exercise.
\item Given $\frac{b + c - a}{a}, \frac{c + a - b}{b}, \frac{a + b - c}{c}$ are in A.P.

  Adding $2$ to each term will give us another A.P. [refer properties of A.P.]

  $\therefore \frac{a + b + c}{a}, \frac{a + b + c}{b}, \frac{a + b + c}{c}$ will be in A.P.

  Dividing each term with $a + b + c$ will yield another A.P.

  $\therefore \frac{1}{a}, \frac{1}{b}, \frac{1}{c}$ will be in A.P.
\item Given $a, b, c$ are in A.P.

  Dividing each term by $abc$ will yield another A.P.

  $\therefore \frac{1}{bc}, \frac{1}{ca}, \frac{1}{ab}$ will be in A.P.

  Multiplying each term with $abc + 1$ will yield another A.P.

  $\therefore a + \frac{1}{bc}, b + \frac{1}{ca}, c + \frac{1}{ab}$ will be in A.P.
\item Given $a, b, c$ are in A.P. $\therefore b - a = c - b$

  $\Rightarrow \frac{1}{b - a} = \frac{1}{c - b} \Rightarrow \frac{ab + bc + ca}{b - a} = \frac{ab + bc + ca}{c - b}$

  $\Rightarrow ab(b - a) + c(b^2 - a^2) = bc(c - a) + a(c^2 - b^2)$

  $\Rightarrow b^2a + b^2c - a^2b - a^2c = c^2a + c^2b - b^2c - b^2a \Rightarrow b^2(a + c) - a^2(b + c) = c^2(a + b) - b^2(c + a)$

  $\therefore a^2(b + c), b^2(c + a), c^2(a + b)$ are in A.P.
\item We will prove this in reverse. We assume that $\frac{1}{\sqrt{b} + \sqrt{c}}, \frac{1}{\sqrt{c} + \sqrt{a}},
  \frac{1}{\sqrt{a} + \sqrt{b}}$ are in A.P.

  $\Rightarrow \frac{1}{\sqrt{c} + \sqrt{a}} - \frac{1}{\sqrt{b} + \sqrt{c}} = \frac{1}{\sqrt{a} + \sqrt{b}} +
  \frac{1}{\sqrt{c} + \sqrt{a}}$

  $\Rightarrow \frac{2}{\sqrt{c} + \sqrt{a}} = \frac{1}{\sqrt{b} + \sqrt{c}} + \frac{1}{\sqrt{a} + \sqrt{b}}$

  $\Rightarrow \frac{2}{\sqrt{c} + \sqrt{a}} = \frac{\sqrt{a} + \sqrt{b} + \sqrt{b} + \sqrt{c}}{(\sqrt{b} + \sqrt{c})(\sqrt{a}
    + \sqrt{b})}$

  $\Rightarrow 2(\sqrt{b} + \sqrt{c})(\sqrt{a} + \sqrt{b}) = (\sqrt{c} + \sqrt{a})(\sqrt{a} + 2\sqrt{b} + \sqrt{c})$

  $\Rightarrow 2(\sqrt{ab} + b + \sqrt{ac} + \sqrt{bc}) = \sqrt{ac} + 2\sqrt{bc} + c + a + 2\sqrt{ab} + \sqrt{ac}$

  $\Rightarrow 2b = a + c$, which implies that $a, b, c$ are in A.P. So the reverse is also true.
\item Given $a, b, c$ are in A.P.

  Dividing each term by $abc$ will yield another A.P.

  $\Rightarrow \frac{1}{bc}, \frac{1}{ca}, \frac{1}{ab}$ will be in A.P.

  Multiplying each term with $ab + bc + ca$ will yield another A.P.

  $\Rightarrow \frac{ab + ca}{bc} + 1, \frac{ab + bc}{ca} + 1, \frac{bc + ca}{ab} + 1$ will be in A.P.

  Subtracting $1$ from each term yields desired terms in A.P.
\item We have to prove that $\frac{1}{b - c}, \frac{1}{c - a}, \frac{1}{a - b}$ are in A.P.

  i.e. $\frac{1}{c - a} - \frac{1}{b - c} = \frac{1}{a - b} - \frac{1}{c - a}$

  $\Rightarrow \frac{b -2c + a}{(c - a)(b - c)} = \frac{c - 2a + b}{(a - b)(c - a)}$

  $\Rightarrow (a + b - 2c)(a - b) = (b + c - 2a)(b - c)$

  Now, given that $(b - c)^2, (c - a)^2, (a - b)^2$ are in A.P.

  $\Rightarrow (c - a)^2 - (b - c)^2 = (a - b)^2 - (c - a)^2$

  $\Rightarrow (b - a)(2c - a - b) = (c - b)(2a - b - c)$

  Thus, we have proven the desierd result.
\item Given $a, b, c$ are in A.P.

  Subtracting $a, b, c$ from each term will yield another A.P.

  $\Rightarrow -(b + c), -(c + a), -(a + b)$ will be in A.P.

  Multiplying each term with $-1$ will yield the desired A.P.
\item We have to prove that $\frac{1}{b + c}, \frac{1}{c + a}, \frac{1}{a + b}$ are in A.P.

  i.e. $\frac{1}{c + a} - \frac{1}{b + c} = \frac{1}{a + b} - \frac{1}{c + a}$

  $\Rightarrow \frac{b - a}{(b + c)} = \frac{c - b}{(a + b)}$

  $\Rightarrow b^2 - a^2 = c^2 - b^2 \Rightarrow a^2, b^2, c^2$ are in A.P.

  Thus, we have proven the desired result in reverse.
\item Given that $a, b, c$ are in A.P. $\Rightarrow b - a = c - b = k$ (say)

  $\Rightarrow c - a = 2k \Rightarrow 2(a - b) = a - c = 2(b - c) = -2k$.
\item Given that $a, b, c$ are in A.P. Let $b = a + d \Rightarrow c = a + 2d$

  Now, $(a - c)^2 = 4d^2, 4(b^2 - ac) = 4[(a + d)^2 - a(a + 2d)] = 4d^2$

  $\Rightarrow (a - c)^2 = 4(b^2 - ac)$
\item Let $n = 2m + 1$ where $m\in N. \Rightarrow S_1 = \frac{n}{2}[t_1 + t_n]$ where $d$ is the commond difference.

  For $S_2$ the no. of terms will be $m. \Rightarrow S_2 = \frac{m}{2}[t_2 + t_{n - 1}]$

  We know that $t_1 + t_n = t_2 + t_{n - 1}$

  $\therefore \frac{S_1}{S_2} = \frac{n}{m} = \frac{n}{\tfrac{n - 1}{2}} = \frac{2n}{n - 1}$.
\item The degree is the highest power of $x$ which will be $1 + 6 + 11 + \cdots + 101$.

  Clearly, the above sequence is an A.P. having first term $1$, common difference $5$ and last term as $101$.

  $n = \frac{t_n - t_1}{d} + 1 = \frac{101 - 1}{5} + 1 = 21$.

  $\Rightarrow S = \frac{21}{2}[t_1 + t_n] = \frac{21}{2}[1 + 101] = 21\times 51 = 1071$

  Therefore, the degree of the polynomial will be $1071$.
\item Consider an A.P. with first term as $a$, commond difference as $d$ and no. of terms as $n$. Then sum is given by

  $S = \frac{n}{2}[2a + (n - 1)]d = \frac{n^2d^2}{2} + \frac{(2a - d)n}{2}$

  which is of the form $An^2 + Bn$ where $A = \frac{d^2}{2}$ and $B = \frac{2a - d}{2}$.
\item Let the common difference of the A.P. be $d$.

  L.H.S. $= a_1^2 - a_2^2 + a_3^2 - a)4^2 + \cdots + a_{2n - 1}^2 -a_{2n}^2$

  $= (a_1 - a_2)(a_1 + a_2) + (a_3 - a_4)(a_3 + a_4) + \cdots + (a_{2n - 1} - a_{2n})(a_{2n - 1} + a_{2n})$

  $= -d(a_1 + a_2 + a_3 + a_4 + \cdots + a_{2n - 1} + a_{2n})$

  $= -\frac{2nd}{2}[a_1 + a_{2n}]$

  $= \frac{n}{2n - 1}(a_1^2 - a_{2n}^2)\left[\because d = \frac{a_{2n} - a_1}{2n - 1}\right]$
\item We know that sum of equidistant terms from start and end of an A.P. is equal.

  $\therefore a_1 + a_{24} = a_5 + a_{20} = a_{10} + a_{15} = k$ (say)

  $\therefore a_1 + a_5 + a_{10} + a_{15} + a_{24} = 3k = 225 \Rightarrow k = 75$

  Sum of first $24$ terms $S = a_1 + a_2 + \cdots + a_{24} = \frac{24}{2}[a_1 + a_{24}] = 12\times75 = 600$.
\item Let $a$ be the first term and $d$ be the common difference. Also let $S_1$ denote the sum of first $3n$ terms and $S_2$
  denote the sum of next $n$ terms.

  $S_1 = \frac{3n}{2}[2a + (3n - 1)d], S_2 = \frac{n}{2}[2a + 6nd + (n - 1)d][\because t_{3n+1} = a + 3nd]$

  Given, $S_1 = S_2 \Rightarrow \frac{3n}{2}[2a + (3n - 1)d] = \frac{n}{2}[2a + 6nd + (n - 1)d]$

  $\Rightarrow 6a + (9n - 3)d = 2a + (7n - 1)d \Rightarrow 2a + (n - 2)d = 0$

  Let $S_3$ be sum of first $2n$ terms and $S_4$ be sum of next $2n$ terms, then

  $\frac{S_3}{S_4} = \frac{\tfrac{2n}{2}[2a + (2n - 1)d]}{\tfrac{2n}{2}[2a + 4nd + (2n - 1)]d}$

  $\Rightarrow = \frac{nd}{5nd} = \frac{1}{5}[\because 2a + (n - 1)d = 0xs]$
\item Given $S_n = 5n^2 + 3n \Rightarrow t_n = S_n - S_{n - 1} = 5n^2 + 3n - 5(n - 1)^2 - 3(n - 1)$

  $= 10n - 5 + 3 = 10n - 2 \Rightarrow d = t_n - t_{n - 1} = 10n - 2 - 10(n - 1) + 2 = 10$,

  Since common difference is a constant the series is in A.P.
\item Common difference of the series $d = (a^2 + b^2) - (a + b)^2 = (a - b)^2 - (a^2 + b^2) = -2ab$

  $S = \frac{n}{2}[2(a + b)^2 - (n - 1)2ab] = \frac{n}{2}[2a^2 + 2b^2 - 2(n + 1)ab]$

  $= n[a^2 + b62 - (n + 1)ab]$.
\item There will be two cases. First $n$ being odd and second $n$ being even.

  {\bf Case I:} When $n$ is odd i.e. $n = 2m + 1$, where $m = 0, 1, 2, \ldots$

  $S = 1 + 5 + 9 + \cdots$ up to $m + 1$ terms $- 3 - 7 - 11$ up to $m$ terms

  $= \frac{m + 1}{2}[2 + 4m] - \frac{m}{2}[6 + 4m - 4] = (m + 1)(1 + 2m) - m(2m + 1)$

  $= 2m^2 + 3m + 1 - 2m^2 - m = 2m + 1 = n$.

  {\bf Case II:} When $n$ is even i.e. $n = 2m$, where $m =1, 2, 3, \ldots$

  $S = 1 + 5 + 9 + \cdots$ up to $m$ terms $- 3 - 7 - 11$ up to $m$ terms

  $= \frac{m}{2}[2 + 4m - 4] - \frac{m}{2}[6 + 4m - 4] = -2m = -n$.
\item Let there be $n$ sides of the polygon. From geometry, we know that sum of angles of the polygon $= (n - 2)180^\circ$

  From the formula for sum of an A.P. $S = \frac{n}{2}[2\times120^\circ + (n - 1)5^\circ] = (n - 2)180^\circ$

  $\frac{n}[240^\circ + (n - 1)5^\circ] = (n - 2)360^\circ\Rightarrow n[48^\circ + (n - 1)] = (n - 2)72^circ$

  $\Rightarrow n^2 - 25n + 144 = 0 \Rightarrow n = 9, 16$
\item To water first tree the gardener will have to travel $10$ m. To water second tree he will
  have tp travel back $10$ m to well and then $15$ m to the tree i.e. $25$ m. Similarly, for third tree he will
  have to travel $15$ m to well and $20$ m i.e a total of $35$ m.

  Thus, total distance travelled will be $10 + 25 + 35 + \cdots$

  Clearly, $25$ will be the first term of the A.P. and there will be $24$ such terms because distance travelled
  for first tree is noty part of the A.P. Note that common difference would be $10$.

  Total distance travelled $= 10 + \frac{24}{2}[2\times25 + (24 - 1)10] = 10 + 3360 = 3370$ m.
\item Let $d$ be the common difference. Given $S_p = 0 \Rightarrow \frac{p}{2}[2a + (p - 1)d] = 0$

  $\Rightarrow 2a + (p - 1)d = 0 \Rightarrow d = \frac{2a}{1 - p}$

  $p + 1$th term $t_{p + 1} = a + pd$, so the sum of next $q$ terms $S = \frac{q}{2}[2a + 2pd + (q - 1)d]$

  $= \frac{q}{2}[2a + (2p + q - 1)d] = \frac{q}{2}\left[2a + (2p + q - 1).\frac{2a}{1 - p}\right]$

  $= \frac{q}{2}\left[\frac{2a.(p + q)}{1 - p}\right] = -\frac{a(p + q)}{p - 1}q$.
\item Sum of first $p$ terms, $S_p = \frac{p}{2}[2a + (p - 1)d]$; sum of first $q$ terms $S_q = \frac{q}{2}[2a + (q - 1)d]$

  $2ap + (p^2 - p)d = 2aq + (q^2 - q)d \Rightarrow 2a(p - q) = (q^2 - p^2 + p - q)d$

  $2a = (1 - p - q)d$

  Sum of $(p + q)$ terms, $S_{p + q} = \frac{p + q}{2}[2a + (p + q - 1)d] = \frac{p + q}{2}[(1 - p - q)d + (p + q - 1)d] = 0$.
\item Sum of latter half of $2n$ terms means $n + 1$th term to $2n$th term. $t_{n + 1} = a + nd$ and $t_{2n} = a + (2n - 1)d$
  where $a$ and $d$ are the first term and common difference respectively.

  Sum of latter half of terms, $S = \frac{n}{2}[t_{n + 1} + t_{2n}] = \frac{n}{2}[2a + (3n - 1)d]$

  Sum of first $3n$ terms, $S_{3n} = \frac{3n}{2}[2a + (3n - 1)d]$

  Clearly, $S/S_{3n} = 1:3$.
\item Let $S_r$ be the $r$th A.P. whose first term is $r$ and common difference is also $r$.

  $S_r = \frac{n}{2}[2r + (n - 1)r] = \frac{n}{2}[(n + 1)r] = \frac{n(n + 1)r}{2}$

  $S_1 + S_2 + S_3 + \cdots + S_p = \displaystyle\sum_{r=1}^pS_r$

  $= \frac{n(n + 1)}{2}\displaystyle\sum_{r = 1}^pr = \frac{np}{4}(n + 1)(p + 1)\left[\because\displaystyle\sum_{i=1}^ni =
  \frac{n(n + 1)}{2}\right]$.
\item Let $x$ be the first term and $y$ be the common difference of the A.P.

  Then, according to the question $a = \frac{p}{2}[2x + (p - 1)y], b = \frac{q}{2}[2x + (q - 1)y], c = \frac{r}{2}[2x + (r - 1)y]$

  We have to prove that $\frac{a}{p}(q - r) + \frac{b}{q}(r - p) + \frac{c}{r}(p - q) = 0$

  L.H.S. $= x(q - r + r - p + p - q) + \frac{y}{2}[(p - 1)(q - r) + (q - 1)(r - p) + (r - 1)(p - q)]$

  $= 0$.
\item Let $a$ be the first term and $d$ be the common difference of the A.P.

  Given, $S_m = \frac{1}{2}S_{m + n}\Rightarrow \frac{m}{2}[2a + (m - 1)d] = \frac{1}{2}.\frac{m + n}{2}[2a + (m + n - 1)d]$

  Let $2a + (m - 1)d = x$, then the above equation can be written as

  $mx = \frac{m + n}{2}[x + nd] \Rightarrow 2mx = (m + n)[x + nd]\Rightarrow mx = n(x + nd) + mnd$

  $\Rightarrow (m - n)x = (m + n)nd$

  Similarly, $(m - p)x = (m + p)pd$

  Dividing, we get

  $(m - n)(m + p)p = (m + n)(m - p)n$

  Dividing both sides with $mnp$ we arrive at the desired result.
\item Let $a$ be the first term and $d$ be the common difference of the A.P. For odd terms, the no. of terms will be $n + 1$, first
  term will be $a$ and common difference will be $2d$.

  $\therefore S_{odd} = \frac{n + 1}{2}[2a + 2nd]$

  For even terms, the no. of terms will be $n$, first term will be $a + d$ and common difference will be $2d$.

  $\therefore S_{even} = \frac{n}{2}[2a + 2d + 2(n - 1)d] = \frac{n}{2}[2a + 2nd]$

  $\therefore \frac{S_{odd}}{S_{even}} = \frac{n + 1}{n}$.
\item Let $a_1$ and $a_2$ be the first terms and $d_1$ and $d_2$ be the common differences of the two series in A.P.

  Given, $\frac{\tfrac{n}{2}[2a_1 + (n - 1)d_1]}{\tfrac{n}{2}[2a_2 + (n - 1)d_2]} = \frac{3n - 12}{5n + 21}$

  $\Rightarrow \frac{2a_1 + (n - 1)d_1}{2a_2 + (n - 1)d_2} = \frac{3n - 13}{5n + 21}$

  We need to find ratio of the $24$th terms i.e. $\frac{a_1 + 23d_1}{a_2 + 23d_2} = \frac{2a_1 + 46d_1}{2a_2 + 46d_2}$

  Putting $n = 47$ in the ratio of sums, we have

  $\frac{2a_1 + 46d_1}{2a_2 + 46d_2} = \frac{3\times 47 - 13}{5\times47 + 21} = \frac{1}{2}$
\item Let $a$ be the first term and $d$ be the common difference of the A.P.

  Given, $t_m = a + (m - 1)d = \frac{1}{n},\,t_n = a + (n - 1)d = \frac{1}{m}$

  Subtracting, we get $(m - n)d = \frac{m - n}{mn} \Rightarrow d = \frac{1}{mn} \Rightarrow a = \frac{1}{mn}$

  $\therefore S_{mn} = \frac{mn}{2}\left[\frac{2}{mn} + \frac{mn - 1}{mn}\right] = \frac{mn + 1}{2}$.
\item Let $a$ be the first term and $d$ be the common difference of the A.P.

  Given, $S_m = n = \frac{m}{2}[2a + (m - 1)d] \Rightarrow 2a + (m - 1)d = \frac{2n}{m}$

  and $S_n = m = \frac{n}{2}[2a + (n - 1)d] \Rightarrow 2a + (n - 1)d = \frac{2m}{n}$

  $\Rightarrow d = -\frac{2(m + n)}{mn}\Rightarrow a = \frac{m^2 + n^2 + mn - m - n}{mn}$

  $\Rightarrow S_{m + n} = \frac{m + n}{2}[2a + (m + n - 1)d] = -(m + n)$.
\item Let $a$ be the first term and $d$ be the common difference of the A.P.

  $\therefore S = \frac{2n + 1}{2}[2a + 2nd]$

  For $S_1$ first term would be $a$, common difference would be $2d$ and no. of terms would be $n + 1$.

  $\therefore S_1 = \frac{n + 1}{2}[2a + 2nd]$

  $\therefore \frac{S}{S_1} = \frac{2n + 1}{n +1}$.
\item Let $d$ be the common difference, then $b = a + 2d \Rightarrow d = \frac{b - a}{2}$

  $c = a + (n - 1)d \Rightarrow n - 1 = \frac{c - a}{d} = \frac{2(c - a)}{b - a}$

  $\Rightarrow n = \frac{2(c - a)}{b - a} + 1$

  $\therefore S = \frac{n}{2}[2a + (n - 1)d] = \frac{1}{2}\left[\frac{2(c - a)}{b - a} + 1\right]\left[2a + \frac{2(c - a)}{b -
    a}.\frac{b - a}{2}\right]$

  $= \frac{c + a}{2} + \frac{c^2 - a^2}{b - a}$.
\item Let $a_1, a_2$ be the first terms and $d_1, d_2$ be the common differences of the two series in A.P.

  According to the question $\frac{2a_1 + (n - 1)a_1}{2a_2 + (n - 1)d_2} = \frac{3n + 8}{7n + 15}$.

  We have to find ratio of $12$th terms i.e. $\frac{a_1 + 11d_1}{a_2 + 11d_2} = \frac{2a_1 + 22d_1}{2a_2 + 22d_2}$

  Putting $n = 23$ in previous equation, we get

  $\frac{2a_1 + 22d_1}{2a_2 + 22d_2} = \frac{77}{176} = \frac{7}{16}$.
\item Let $a$ be the first term and $d$ be the common difference of the A.P.

  Given, $\frac{S_m}{S_n} = \frac{\tfrac{m}{2}[2a + (m - 1)d]}{\tfrac{n}{2}[2a + (n - 1)d]} = \frac{m^2}{n^2}$

  $\Rightarrow \frac{2a + (m - 1)d}{2a + (n - 1)d} = \frac{m}{n}$

  $\Rightarrow 2a(n - m) + [(m - 1)n - (n - 1)m]d = 0 \Rightarrow a = \frac{d}{2}$

  We have to find $\frac{t_m}{t_n} = \frac{a + (m - 1)d}{a + (n - 1)d} = \frac{2m - 1}{2n - 1}$
\item Let $n$ be the no. of terms. Clearly, common ratio $r = \frac{20}{5} = \frac{80}{20} = 4$

  Then $t_n = 5120 = 5.r^{n - 1} \Rightarrow 4^{n - 1} = 1024 = 4^5 \Rightarrow n = 6$.
\item Let $n$ be the no. of terms. Clearly, common ratio $r = \frac{0.06}{0.03} = \frac{0.12}{0.06} = 2$

  Then $t_n = 3.84 = 0.03r^{n - 1} \Rightarrow 2^{n - 1} = 128 \Rightarrow n = 8$.
\item From the question we deduce that it is a G.P. with $a = 1, r = 2, n = 20$. We have to find $t_{20}$.

  $t_{20} = 1.2^{20 - 1} = 524288$.
\item This is a G.P. with $a = 20000, r = 1.02, n = 11$. We have to find $t_{11}$.

  $t_{11} = 20000\times(1.02)^{11 - 1} = 24380$.
\item Given, $S_n = 2^n - 1 \Rightarrow t_n = S_n - S_{n - 1} = 2^n - 1 - (2^{n - 1} - 1) = 2^{n - 1}$

  $r = \frac{t_n}{t_{n - 1}} = \frac{2^{n - 1}}{2^{n - 2}} = 2$, which is a constant and hence the sequence is in G.P.
\item Let the first term of the G.P. be $a$ and common ratio is $r$.

  Then $t_2 = ar = 24$ and $t_5 = ar^4 = 81$, Dividing, we have $r^3 = \frac{81}{24} = \frac{27}{8}$

  $\Rightarrow r = \frac{3}{2} \Rightarrow a = 16$.

  Hence the G.P. is $16, 24, 36, 54, 81, \ldots$.
\item Let the first term of the G.P. be $a$ and common ratio is $r$.

  Given $t_7 = 8t_4 \Rightarrow ar^6 = 8ar^3 \Rightarrow r = 2$. Also given, $t_5 = 48 \Rightarrow ar^4 = 48$

  $\Rightarrow a = 3$. Hence, the G.P. is $3, 6, 12, 24, \ldots$.
\item Let the first term of the G.P. be $a$ and common ratio is $r$.

  Given, $t_5 = ar^4 = 48$ and $t_8 = ar^7 = 384 \Rightarrow r^3 = 8 \Rightarrow r = 2$

  $\Rightarrow a = 3$. Hence, the G.P. is $3, 6, 12, 24, \ldots$.
\item Let the first term of the G.P. be $a$ and common ratio is $r$.

  Given $t_6 = ar^5 = \frac{1}{16}$ and $t_{10} = ar^9 = \frac{1}{256} \Rightarrow r = \pm\frac{1}{2}$

  $\Rightarrow a = \pm 2$. Hence the G.P. is $2, 1, \frac{1}{2}, \ldots$ or $-2, 1, -\frac{1}{2}, \ldots$.
\item Let the first term of the G.P. be $x$ and common ratio is $y$. Then

  $a = xy^{p - 1}, b = xy^{q - 1},  c = xy^{r - 1}$

  Taking $\log$ of both sides for these three terms

  $\log a = \log x + (p - 1)\log y, \log b = \log x + (q - 1)\log y, \log c = \log x + (r - 1)\log y$

  Clearly, $(q - r)\log a + (r - p)\log b + (p - q)\log r = 0$.
\item Let the first term of the G.P. be $x$ and common ratio is $r$.

  Given, $t_{p + q} = a = xr^{p + q - 1}$ and $t_{p - q} = b = xr^{p - q - 1}$

  Multiplying the two terms, we have

  $x^2r^{2p - 2} = (xr^{p - 1})^2 = t_p^2 = ab \Rightarrow t_p = \sqrt{ab}$.
\item Let $a$ be the first term and $b$ be the common ratio. Then,

  $x = ab^{p - 1}, y = ab^{q - 1}, z = ab^{r -1}$

  We have to prove that $x^{q - r}.y^{r - p}.z^{p - q} = 1$

  L.H.S. $= (ab^{p - 1})^{q - r}.(ab^{q - 1})^{r - p}.(ab^{r - 1})^{p - q}$

  $= a^{(q - r + r - p + p - q)}b^{[(p - 1)(q - r) + (q - 1)(r - p) + (r - 1)(p - q)]}$

  $= a^0b^0 = 1 =$ R.H.S.
\item Let $r$ be the common ratio and first term is given as $1$.

  $t_3 + t_5 = 90 \Rightarrow r^4 + r^2 = 90 \Rightarrow r^2 = 9 \Rightarrow r = pm 3$.

  $r^2$ cannot be $-10$ as that would mean that it is an imaginary number.
\item Let $a$ be the first term and $r$ be the common ratio of the G.P.

  Gibem $t_5 = ar^4 = 2$ and we have to find the product of the first nine terms. Let the required product be $S$.

  $S = a.ar.ar^2.\ldots.ar^8 = a^9r^{1 + 2 + \cdots + 8} = a^9r^{\tfrac{8.9}{2}} = a^9r^{36} = (ar^4)^9 = 2^9 = 512$.
\item Let $a$ be the first term, $r$ be the common ratio and $n$ be the number of terms.

  Given, $t_4 = ar^3 = 10, t_7 = ar^6 = 80, t_n = ar^{n - 1} = 2560$

  $\therefore \frac{t_7}{t_4} = r^3 = 8 \Rightarrow r = 2 \Rightarrow a = \frac{10}{8}$

  $\Rightarrow \frac{10}{8}2^{n- 1} = 2560 \Rightarrow 2^{n - 1} = 2048 \Rightarrow n = 12$.
\item Let the three numbers in G.P. be $a, ar, ar^2$. According to question, on doubling $ar$ the numbers form an A.P.

  $\Rightarrow 2ar - a = ar^2 - 2ar \Rightarrow r^2 - 4r + 1 = 0 \Rightarrow r = \frac{4\pm\sqrt{12}}{2} = 2\pm\sqrt{3}$.
\item Given, $p, q, r$ are in A.P. i.e. $q - p = r - q$.

  Let $x$ be the first term and $y$ be the common ratio of the G.P. We have to prove that $t_p, t_q, t_r$ are in G.P.

  $\Rightarrow \frac{t_q}{t_p} = \frac{t_r}{t_q} \Rightarrow \frac{xy^{q - 1}}{xy ^{p - 1}} = \frac{xy^{r - 1}}{xy^{q - 1}}$

  $\Rightarrow y^{q - p} = y^{r - q}$ which is true from the condition for A.P.
\item Let $r$ be the common ratio of the G.P. Then, $b = ar, c = ar^2, d = ar^3$

  L.H.S. $= (a.ar + ar.ar^2 + ar^2.ar^3)^2 = a^4r^2(1 + r^2 + r^4)^2$

  R.H.S. $= (a^2 + a^2r^2 + a^2r^4)(a^2r^2 + a^2r^4 + a^2r^6) = a^2(1 + r^2 + r^4).a^2r^2(1 + r^2 + r^4)$

  $= a^2r^4(1 + r^2 + r^4)^2 =$ L.H.S.
\item Given $a, b, c$ are in A.P. $\Rightarrow 2b = a + c$

  If we increase $a$ by $1$ then they are in G.P. $\Rightarrow b^2 = (a + 1)c \Rightarrow b^2 = (a + 1)(2b - a)$

  $\Rightarrow b^2 = 2ab - a^2 + 2b - a \Rightarrow (a - b)^2 = 2b - a$

  If we increase $c$ by $2$ then again they are in G. P $\Rightarrow b^2 = a(c + 2) = a(2b - a + 2)$

  $\Rightarrow b^2 = 2ab - a^2 + 2a \Rightarrow (a - b)^2 = 2a \Rightarrow 2b - a = 2a \Rightarrow 2b = 3a$

  $\Rightarrow \left(a - \frac{3a}{2}\right)^2 = 2a \Rightarrow a = 8 \Rightarrow b = 12 \Rightarrow c = 16$.
\item Let the three numbers in G.P. be $\frac{a}{r}, a, ar$. Then,

  $\frac{a}{r} + a + ar = 70$ and $10a = \frac{4a}{r} + 4ar \Rightarrow \frac{10a}{4} = \frac{a}{r} + ar$

  $\Rightarrow \frac{10a}{4} + a = 70 \Rightarrow a = 20$

  $\Rightarrow \frac{20}{r} + 20r = 50 \Rightarrow r = 2, \frac{1}{2}$

  So the numbers are $10, 20, 40$ or $40, 20, 10$.
\item Let the three numbers in G.P. be $\frac{a}{r}, a, ar$. Given that product of these numbers is $216$.

  $\Rightarrow \frac{a}{r}.a.ar = 216 \Rightarrow a^3 = 216 \Rightarrow a = 6$

  Also, given that their sum is $19 \Rightarrow \frac{6}{r} + 6 + 6r = 19$

  $\Rightarrow 6r^2 - 13r + 6 = 0 \Rightarrow r = \frac{2}{3}, \frac{3}{2}$.

  So the numbers are $9, 6, 4$ or $4, 6, 9$.
\item Let the number be $100a + 10ar + ar^2$.

  According to question $a + ar^2 = 2ar + 1$ and $a + ar = \frac{2}{3}(ar + ar^2)$

  $\Rightarrow a(r - 1)^2 = 1$ and $3 + 3r = 2r + 2r^2 \Rightarrow r = -1, \frac{3}{2}$

  If $r = -1, a = \frac{1}{4}$, but $a$ cannot be a fraction.

  If $r = \frac{3}{2} \Rightarrow a = 4$ and the number is $469$.
\item Given that three of four numbers are in A.P. and so we choose them as $a - d, a, a + d$. Also, since first number is same as
  first so the numbers are $a + d, a - d, a, a + d$. The first three are in G.P. Given $d = 6$

  $\therefore (a - d)^2 = a(a + d)\Rightarrow (a - 6)^2 = a(a + 6) \Rightarrow 18a = 36 \Rightarrow a = 2$.

  So the numbers are $8, -4, 2, 8$.
\item Let the three numbers are $a, ar, ar^2$. The sum is given as $21 \Rightarrow a + ar + ar^2 = 21$.

  Also, sum of squares is given as $189 \Rightarrow a^2 + a^2r^2 + a^2r^4 = 189$

  $\Rightarrow \frac{441(1 + r^2 + r^4)}{(1 + r + r^2)^2} = 189$

  $\Rightarrow 7(1 + 2r^2 + r^4 - r^2) = 3(r + r + r^2)^2 \Rightarrow 7(1 - r + r^2) - 3(1 + r + r^2)$

  $\Rightarrow 2r^2 - 5r + 2 = 0 \Rightarrow r = 2, \frac{1}{2}$

  When $r = 2, a = 3$ and so the numbers are $3, 6, 12$.

  When $r = \frac{1}{2}, a = 12$ and so the numbers are $12, 6, 3$.
\item Let the terms in G.P. be $\frac{a}{r}, a, ar$. Given that the product of these is $-64$.

  $\therefore \frac{a}{r}.a.ar = -64 \Rightarrow a^3 = -64 \Rightarrow a = -4$.

  Also given that the first term is four times the third. $\Rightarrow \frac{a}{r} = 4.ar \Rightarrow r^2 = \frac{1}{4} \Rightarrow r = \pm\frac{1}{2}$.

  If $r = \frac{1}{2},$ the terms will be $-8, -4, -2$. If $r = -\frac{1}{2}$, the terms will be $8, -4, 2$.
\item Let the numbers be $a - d, a, a + d$. Given that sum is $15. \Rightarrow a - d + a + a + d = 15 \Rightarrow a = 5$.

  Also given that if $1, 4, 19$ are added to them then they are in G.P.

  $\Rightarrow (5 + 4)^2 = (5 - d + 1)(5 + d + 19) \Rightarrow 81 = (6 - d)(24 + d)$

  $\Rightarrow d^2 + 18d - 63 = 0 \Rightarrow d = -21, 3$.

  If $d = -15$, the numbers will be $26, 5, -16$ and if $d = 3$ the numbers will be $2, 5, 8$.
\item Let the two sets of three numbers in G.P. are $a_1, a_1r_1, a_1r_1^2$ and $a_2, a_2r_2, a_2r_2^2$.

  Given that the difference is also in G.P.

  $\Rightarrow (a_1r_1 - a_2r_2)^2 = (a_1r_1^2 - a_2r_2^2)(a_1 - a_2)$

  $\Rightarrow a_1^2r_1^2 + a_2^2r_2^2 - 2a_1a_2r_1r_2 = a_1^2r_1^2 - a_1a_2r_2^2 - a_1a_2r_1^2 + a_2^2r_2^2$

  $\Rightarrow 2a_1a_2r_1r_2 = a_1a_2r_2^2 + a_1a_2r_1^2 \Rightarrow 2r_1r_2 = r_1^2 + r_2^2 \Rightarrow (r_1 - r_2)^2 = 0$

  $\Rightarrow r _1 = r_2$ which implies that they have same common ratio.
\item Let $r$ be the common ratio. Then $b = ar, c = ar^2, d = ar^3$

  L.H.S. $= (b - c)^2 + (c - a)^2 + (d - b)^2 = (ar - ar^2)^2 + (ar^2 - a)^2 + (ar^3 - ar)^2$

  $= a^2(r - r^2)^2 + a^2(r^2 - 1)^2 + a^2(r^3 - r)^2 = a^2(r^2 + r^4 - 2r^3 + r^4 + 1 - 2r^2 + r^6 + r^2 - 2r^4)$

  $= a^2(r^6 - 2r^3 + 1) = (ar^3 - a)^2 = (d - a)^2 = $ R.H.S.
\item This problem can be solved like previous problem.
\item Given that $x, y, z$ are in G.P. Let $p$ be the first term and $r$ be the common ratio of this G.P.

  Also given, $a^x = b^y = c^z \Rightarrow x\log a = y\log b = z\log c$

  $\Rightarrow \frac{\log a}{\log b} = \frac{y}{x}$ and $\frac{\log b}{\log c} = \frac{z}{y}$. Clearly, $\frac{y}{x} = \frac{z}{y} =
  r \Rightarrow \log_ba = \log_cb$.
\item Let $\frac{a}{r}, a, ar$ be the terms in G.P. Given that continued product is $216$ i.e.

  $\frac{a}{r}.a.ar = 216 \Rightarrow a^3 = 216 \Rightarrow a = 6$

  Sum of products when taken in pair is given as $156$.

  $\Rightarrow \frac{a}{r}.a + a.ar + \frac{a}{r}.ar = 156 \Rightarrow \frac{1}{r} + r + 1 = \frac{26}{6}$

  $\Rightarrow 6r^2 -20r + 6 = 0 \Rightarrow r = \frac{1}{3}, 3$

  So the numbers are $18, 6, 2$ or $2, 6, 18$.
\item Let $r$ be the common ratio. Then, $\frac{(b + c)^2}{(a + b)^2} = \frac{(ar + ar^2)^2}{(a + ar)^2} = r^2$.

  Similarly, $\frac{(c + d)^2}{(b + c)^2} = r^2 = \frac{(b + c)^2}{(a + b)^2}$.

  Thus, $(a + b)^2, (b + c)^2, (c + d)^2$ are also in G.P.
\item This problem can be solved like previous problem.
\item This problem can be solved like previous problem.
\item This problem can be solved like previous problem.
\item Let $r$ be the common ratio. Then, $a(b - c)^3 = a(ar - ar^2)^3 = a^4r^3(1 - r)^3$ and $d(a - b)^3 =
  ar^3(a - ar)^3 = a^4r^3(1 - r)^3$.

  Thus, $a(b - c)^3 = d(a - b)^3$.
\item We have to prove that $(a + b + c + d)^2 = (a + b)^2 + (c + d)^2 + 2(b + c)^2$ where $a, b, c, d$ are
  in G.P.

  Now, $(a + b + c + d)^2 = (a + b)^2 + (c + d)^2 + 2(a + b)(c + d)$ so it is enough to prove that $(a +
  b)(c + d) = (b + c)^2$.

  $(a + b)(c + d) = (a + ar)(ar^2 + ar^3) = a^2r^2(1 + r)^2$ and $(b + c)^2 = (ar + ar^2)^2 = a^2r^2(1 +
  r)^2$ which proves the required equality.
\item Let $r$ be the common ratio. L.H.S. $= a^2b^2c^2\left(\frac{1}{a^3} + \frac{1}{b^3} +
  \frac{1}{c^3}\right) = \frac{b^2c^2}{a} + \frac{a^2c^2}{b} + \frac{a^2b^2}{c}$

  $= a^3r^6 + a^3r^3 + a^3 = a^3 + b^3 + c^3 =$ R.H.S.
\item Let $r$ be the common ratio. L.H.S. $=(a^2 - b^2)(b^2 + c^2) = (a^2 - ar^2)(a^2r^2 + a^2r^4) = r^2(a^2
  - a^2r^2)(a^2 + a^2r^2) = (a^2r^2 - a^2r^4)(a^2 + a^2r^2) = (b^2 - c^2)(a^2 + b^2) =$ R.H.S.
\item Let $r$ be the common ratio. Given $a, b, c$ are in G.P. i.e. $a, ar, ar^2$ are in G.P.

  Taking $\log$ of $a, b, c$, we have

  $\log a, \log a + \log r, \log a + 2\log r$ are in A.P. with $\log a$ being the first term and $\log r$ be
  the common difference.
\item Given series is $1 + \frac{1}{2} + \frac{1}{4} + \frac{1}{8} + \cdots$ to $n$ terms. Let $S$ be the
  sum, $a = 1, r = \frac{1}{2}$, then

  $S = \frac{a(1 - r^n)}{1 - r} = 2\left(\frac{2^n - 1}{2^n}\right)$
\item Given series is $1 + 2 + 4 + 8 + \cdots$ to $12$ terms. First term $a = 1$, common ratio $r = 2$
  and no. of terms $n = 12$. Let $S$ be the sum of the series. Then,

  $S = \frac{a(r^n - 1)}{r - 1} = \frac{1(2^{12} - 1)}{2 - 1} = 4095$.
\item Given series is $1 - 3 + 9 - 27 + \cdots$ to $9$ terms. First terms $a = 1$, common ratio $r = -3$ and
  no. of terms $n = 9$. Let $S$ be the sum of the series. Then,

  $S = \frac{a(1 - r^n)}{1 - r} = \frac{1 - (-3)^9}{1 - (-3)} = 4921$
\item This problem is similar to $115$, and has been left as an exercise.
\item Given series is $(a + b) + (a^2 + 2b) + (a^3 + 3b) + \cdots $ to $n$ terms. We can rewrite the series
  as $a + a^2 + a^3 + \cdots$ to $n$ terms + $b + 2b + 3b + \cdots$ to $n$ terms.

  We know that $a + a^2 + a^3 + \cdots$ to $n$ terms $= \frac{a(a^n - 1)}{a - 1}$ and for the second series
  applying the A.P. formula, $b + 2b + 3b + \cdots$ to $n$ terms $= \frac{n}{2}[2b + (n - 1)b] =
  \frac{n}{2}[(n + 1)b] = \frac{n(n + 1)b}{2}$.
\item Clearly the given situation forms a G.P. with $a = 1$, common ratio $r = 2$ and $n = 120$. Let $S$ be
  the sum which he gets at the end of $120$ days. Then,

  $S = \frac{a(r^n - 1)}{r - 1} = 2^{120} - 1 = 1329227995784915872903807060280344575$.
\item Given series is $S = 8 + 88 + 888 + \cdots = \frac{8}{9}[9 + 99 + 999 + \cdots]$

  $= \frac{8}{9}[(10 - 1) + (100 - 1) + (1000 - 1) + \cdots]$

  $= \frac{8}{9}\left[\frac{10(10^n - 1)}{10 - 1} - n\right] = \frac{8}{81}[10^{n + 1} - 10 - 9n]$.
\item This problem can be solved like previous problem.
\item This problem can be solved like previous problem.
\item This problem can be solved like previous problem.
\item Let $S = 1 - \frac{1}{2} + \frac{1}{4} - \frac{1}{8} + \cdots$ to $n$ terms. Clearly, $a = 1$ and $r =
  -\frac{1}{2}$.

  $\Rightarrow S = \frac{a(1 - r^n)}{1 - r} = \frac{1 - (-1)^n\frac{1}{2^n}}{1 -(-)\frac{1}{2}} =
  \frac{2}{3}.\frac{2^n - (-1)^n}{2^n}$.
\item When we make $1000$ per day for $31$ days total amount received will be $31,000$.

  When we receive $1$ for the first day and doubling every day then that would be a G.P. with $a = 1, r =
  2, n = 31 \Rightarrow S = \frac{a(r^n - 1)}{r - 1} = 2^{31} - 1 = 2,147,483,647$ which is clearly way more
  than we make in the first case so we will happily take the second option.
\item We assume that $n$ terms of the series $1 + 3 + 3^2 + \cdots$ make for $3280$. Then

  $S = \frac{1(3^n - 1)}{3 - 1} \Rightarrow 3^n = 6561 \Rightarrow n = 8$.
\item Let $S = 1 + 3 + 3^2 + \cdots + 3^{n - 1}\Rightarrow S = \frac{3^n - 1}{3 - 1} > 1000 \Rightarrow 3^n
  > 2001 \Rightarrow n = 7$.
\item Let the sum be $S$. Clearly it is a G.P. with $a = 1, r = \frac{1}{2}$. We know that when $|r| < 1$
  the sum of an infinite G.P. is given by $S = \frac{a}{1 - r}$. Thus, $S = \frac{1}{1- \frac{1}{2}} = 2$.
\item Clearly, it is a G.P. with $a = 1, r = 3$ and $n = 20$. Thus sum is given by $S = \frac{3^{20} - 1}{3
  - 1} = 1,743,392,200$.
\item We can represent the given series as three series like $(x^2 + x^4 + x^6 + \cdots)$ to $n$ terms $+
  \left(\frac{1}{x^2} + \frac{1}{x^4} + \frac{1}{x^6} + \cdots\right)$ to $n$ terms $+ 2 + 5 + 8 + \cdots$
  to $n$ terms. Let the sum be $S$.

  $S = x^2\frac{(x^2)^n - 1}{x^2 - 1} + \frac{1}{x^2}.\frac{\frac{1}{(x^2)^n - 1}}{\frac{1}{x^2} - 1}
  + \frac{n}{2}[3n + 1]$.
\item Let $n$ be the no. of terms required to make the sum of given G.P. with $a = 1, r = 2$ equal to $511$.

  $511 = \frac{2^n - 1}{2 - 1} \Rightarrow 2^n = 512 \Rightarrow n = 9$.
\item Let the sum be $S. S = 1 + 2 + 2^2 + \cdots + 2^{n - 1} = \frac{2^n - 1}{2 - 1}\geq 300 \Rightarrow
  2^n\geq 301 \Rightarrow n = 9$.
\item Let $r$ be the common ratio. $a_n = ar^{n - 1} = 96. S = \frac{a_1(r^n - 1)}{r - 1} = \frac{a_nr -
  a_1}{r - 1} = \frac{96r - 3}{r - 1} = 189 \Rightarrow 32r - 1 = 63r - 63 \Rightarrow r = 2 \Rightarrow n =
  6$.
\item $0.4\dot{2}\dot{3} = 0.4232323\ldots$ to $\infty = \frac{4}{10} + \frac{23}{1000} + \frac{23}{100000}
  + \cdots$ to $\infty$

  $= \frac{4}{10} + \frac{23}{100}\left[1 + \frac{1}{100} + \frac{1}{10000} +
    \cdots\mathrm{\;to\;}\infty\right]$
  $= \frac{4}{10} + \frac{23}{100}\frac{1}{1 - \frac{1}{100}} = \frac{419}{990}$.
\item Given series can be written as $S = \frac{1}{5} + \frac{1}{5^2} + \cdots$ to $\infty + \frac{1}{7} +
  \frac{1}{7^2} + \cdots $ to $\infty$

  $= \frac{1}{5}.\frac{1}{1 - \frac{1}{5}} + \frac{1}{7}.\frac{1}{1 - \frac{1}{7}} = \frac{1}{4} +
  \frac{1}{6} = \frac{5}{12}$.
\item Let the sum be $S$, then $S = (10 + 1) + (100 + 3) + (1000 + 5) + \cdots$ to $n$ terms

  $= \frac{10(10^n - 1)}{10 - 1} + \frac{n}{2}[2 + (n - 1)2] = \frac{10}{9}(10^n - 1) + n^2$.
\item The general term of the series is $t_n = \left(x^n + \frac{1}{x^n}\right)^2 = x^{2n} +
  \frac{1}{x^{2n}} + 2$ so we can write it as three series and solve like problem $132$.
\item Let $a$ be the first term and $r$ be the common ratio of the G.P. Then,

  $S = \frac{a(r^n - 1)}{r - 1}, P = a.ar.ar^2\ldots ar^{n - 1} = a^nr^{\tfrac{n(n - 1)}{2}}, R =
  \frac{1}{a}\frac{1 - \frac{1}{r^n}}{1 - \frac{1}{r}} = \frac{1}{a}\frac{r^n - 1}{r - 1}.\frac{1}{r^{n -
  1}}$

  $P^2 = a^{2n}r^{n(n - 1)}, \frac{S}{R} = a^2.r^{n - 1}\therefore  \left(\frac{S}{R}\right)^n = P^2$.
\item Clearly, the given series is a G.P. with $a = 1, r = \frac{x}{1 + x} \Rightarrow S = \frac{1}{1 -
  \frac{x}{1 + x}} = 1 + x$.
\item We consider the $n$-th term. $t_n = ar^{n - 1}$, where $a$ is the first term. Sum of all succeeding
  terms $S = \frac{ar^n}{1 - r}\;\;\therefore \frac{t_n}{S} = \frac{1 - r}{r}$. Hence proven.
\item $S_1 = \frac{1}{1 - \frac{1}{2}} = 2, S_2 = \frac{2}{1 - \frac{1}{3}} = 3, S_3 = \frac{3}{1 -
  \frac{1}{4}}, \ldots, S_p = \frac{p}{1 - \frac{1}{p + 1}} = p + 1$.

  Clearly, $S_1, S_2, \ldots, S_p$ forms an A.P. with $2$ as first term and $1$ as c.d.

  $S_1 + S_2 + \cdots + S_p = \frac{p}{2}[2.2 + (p - 1)] = \frac{p(p + 3)}{2}$.
\item $x = \frac{1}{1 - a} \Rightarrow a = 1 - \frac{1}{x} = \frac{x - 1}{x}$ and similarly $b = \frac{y -
  1}{y}$.

  $1 + ab + a^2b^2 + \cdots$ to $\infty = \frac{1}{1 - ab} = \frac{1}{1 - \frac{x - 1}{x}.\frac{y - 1}{y}} =
  \frac{xy}{x + y - 1}$.
\item Let $S$ be the sum, then $S = \frac{1}{1 - r} + \frac{a}{1 - r} + \frac{a^2}{1 - r} + \cdots$ to
  $\infty$

  $\Rightarrow S = \frac{a}{1 - r}.\frac{1}{1 - a} = \frac{a}{(1 - r)(1 - a)}$.
\item When the ball is dropped it will first travel $120$ mts. Then it will bounce back $120.\frac{4}{5} =
  96$ m and fall $96$ m as well. It will then bounce back $96.\frac{4}{5}$ m and fall the same distance as
  well.

  Thus, total distance travelled $120 + 120\times2\times\frac{4}{5} + 120\times2\times\frac{4^2}{5^2} +
  \cdots$ to $\infty$

  $= 120 + 192 \left[1 + \frac{4}{5} + \frac{4^2}{5^2} + \cdots\right]$ to $\infty = 120 + 192.\frac{1}{1 -
    \frac{4}{5}} = 120 + 960 = 1080$ meters.
\item Let $r$ be the common ratio. Then $b = ar^{n - 1} \Rightarrow (ab)^n = a^{2n}r^{n(n - 1)}$

  $p = a.ar.ar^2.ar^3.\ldots ar^{n - 1} = a^nr^{1 + 2 + 3 + \cdots + (n - 1)} = a^nr^{\tfrac{n(n - 1)}{2}}$

  $\Rightarrow p^2 = (ab)^n$.
\item Let the first terms are $a$ and $b$; and the common ratio is $r$. Ratio of sums would be $a:b$ which
  is equal to $ar^{n -1}:br^{n -1}$ i.e. ratio of $n$th terms.
\item Let $a$ be the first term. Then, $S_1 = \frac{a(r^n - 1)}{r - 1}, S_2 = \frac{a(r^{2n} - 1)}{r - 1}$
  and $S_3 = \frac{a(r^{3n} - 1)}{r - 1}$.

  $S_2 - S_1 = \frac{a(r^{2n} - r^n)}{r - 1} = \frac{ar^n(r^n - 1)}{r - 1}$

  $S_1(S_3 - S_2) = \frac{a(r^n - 1)}{r - 1}\left(\frac{ar^{2n}(r^n - 1)}{r - 1}\right) =
  \frac{a^2r^{2n}(r^n - 1)^2}{(r - 1)^2}$

  $\Rightarrow S_1(S_3 - S_2) = (S_2 - S_1)^2$.
\item $S_1 = a, S_2 = \frac{a(r^2 - 1)}{r - 1}, S_2 = \frac{a(r^3 - 1)}{r - 1}, \cdots, S_{2n - 1} =
  \frac{a(r^{2n - 1} - 1)}{r - 1}$

  $S_1 + S_2 + S_3 + \cdots + S_{2n - 1} = \frac{a}{r - 1}\left[r + r^2 + r^3 + \cdots + r^{2n - 1} -(1 + 1
    + \cdots + \mathrm{to}\;2n - 1 \mathrm{terms})\right]$

  $= \frac{a}{r - 1}\left[\frac{r(r^{2n - 1} - 1)}{r - 1} - (2n - 1)\right]$.
\item Given, $S_n = a.2^n - b; t_n = S_n - S_{n- 1} = a.2^n - b - a.2^{n - 1} + b = a.2^{n - 1}; r =
  \frac{t_n}{t_{n - 1}} = 2$ which is a constant independent of $n$ hence the given series is in G.P.
\item Given $x\geq 0 \therefore \frac{2x}{1 + x^2} < 1$ therefore we can apply the sum formula of a G.P. for
  infinite terms.

  Let $S$ be the required sum, then $S = \frac{1}{1 + x^2}.\frac{1}{1 - \frac{2x}{1 + x^2}} = \frac{1}{(1 -
    x)^2}$.
\item Let $a$ be the first term and $r$ be the common ratio. Then given, $a + ar = 24$ and $S_\infty =
  \frac{a}{1 - r} = 32$

  $a = \frac{24}{1 + r}$ and $a = 32(1 - r) \Rightarrow 1 - r^2 = \frac{24}{32} = \frac{3}{4} \Rightarrow r
  = \pm \frac{1}{2}$

  If $r = \frac{1}{2}$ then series is $16, 8, 4, \ldots$. If $r = -\frac{1}{2}$ then series is $48, -24, 12,
  -6, \ldots$.
\item Let $a$ be the first term and $r$ be the common ratio. Sum of this G.P. $\frac{a}{1 - r} = 4$ and sum
  of squares of terms $\frac{a^2}{1 - r^2} = \frac{16}{3}$.

  $\Rightarrow \frac{16(1 - r)^2}{1 - r^2} = \frac{16}{3}\Rightarrow \frac{1 - r}{1 + r} = \frac{1}{3}
  \Rightarrow r = \frac{1}{2} \Rightarrow a = 2$. So the G.P. is $2, 1, \frac{1}{2}, \frac{1}{4}, \ldots$.
\item $p(x) = \tfrac{\tfrac{x^{2n} - 1}{x^2 - 1}}{\tfrac{x^n - 1}{x - 1}} = \frac{x^n + 1}{x + 1}$ so clearly
  $n$ is an odd number for $p(x)$ to be a polynomial in $x$.
\item $x = a + \frac{a}{r} + \frac{a}{r^2} + \cdots$ to $\infty = \frac{a}{1 - \frac{1}{r}} = \frac{ar}{r -
  1}$. Similarly $y = \frac{br}{r + 1}$ and $z = \frac{cr^2}{r^2 - 1},\;\therefore \frac{xy}{z} =
  \frac{ab}{c}$.
\item Let $a$ be the first term, $r$ be the common ratio and $2n$ be the no. of terms. Then sum of all terms
  $S = \frac{a(r^{2n} - 1)}{r - 1}$ and sum of odd terms $S_{\mathrm{odd}} = \frac{a(r^{2n} - 1)}{r^2 - 1}$.

  Given, $S = 5S_{\mathrm{odd}}\Rightarrow r = 4$.
\item $S_n = 3 - \frac{3^{n + 1}}{4^{2n}} \Rightarrow t_n = S_n - S_{n - 1} = \frac{3^n}{4^{2(n - 1)}} -
  \frac{3^{n + 1}}{4^{2n}} = \frac{16.3^n - 3^{n + 1}}{4^{2n}} = \frac{3^n.13}{4^{2n}}$.

  $\Rightarrow r = \frac{t_n}{t_{n - 1}} = \frac{3}{16}$.
\item Let $a$ be the first term and $r$ be the common ratio; then $t_n = ar^{n -1}$. Let the sum of all
  terms succeeding $t_n$ be $S$. Then $S = \frac{ar^n}{1 - r}$.

  $\frac{t_r}{S} = \frac{1 - r}{r}$. If $\frac{1 - r}{r} > 1$ then $r < \frac{1}{2}$, if $\frac{1 - r}{r} =
  1$ then $r = \frac{1}{2}$ and $\frac{1 - r}{r} < 1$ then $r > \frac{1}{2}$.
\item $666\ldots n\;\mathrm{digits} = \frac{6}{9}(10^n - 1) = \frac{2}{3}(10^n - 1)$.

  $888\ldots n\;\mathrm{digits} = \frac{8}{9}(10^n - 1). \Rightarrow$ L.H.S. $= \frac{4}{9}(10^{2n} - 2.10^n + 1 -
  2.10^n -2) = \frac{4}{9}(10^{2n} - 1)$

  R.H.S. $= 444\ldots 2n\;\mathrm{digits} = \frac{4}{9}(10^{2n} - 1) =$ L.H.S.
\item Let $S = (x + y) + (x^2 + xy + y^2) + (x^3 + x^2y + xy^2 + y^3) + \cdots$ to $n$ terms

  $S = \frac{1}{x - y}[(x^2 - y^2) + (x^3 - y^3) + (x^4 + y^4) + \cdots]$ to $n$ terms

  $= \frac{1}{x - y}\left[\frac{x^2(x^n - 1)}{x - 1} - \frac{y^2(y^n - 1)}{y - 1}\right]$.
\item $S = \frac{1}{1 - r}\Rightarrow r = \frac{S - 1}{S}$. Let $S' =
  \displaystyle\sum_{n=0}^{\infty}r^{2n}$ then
  $S' = \frac{1}{1 - r^2} = \frac{S^2}{2S - 1}$.
\item Let $a$ be the first term and $r$ be the common ratio. Then $t_m = ar^{m - 1} = \frac{1}{n^2}$ and
  $t_n = ar^{n - 1} = \frac{1}{m^2} \Rightarrow \frac{t_m}{t_n} = r^{m - n} = \frac{m^2}{n^2} \Rightarrow r
  = \sqrt[m - n]{\frac{m^2}{n^2}}$.

  $ar^{m - 1} = \frac{1}{n^2} \Rightarrow a = \frac{1}{n^2}\left(\frac{n^2}{m^2}\right)^{\tfrac{m - 1}{m -
      n}}$

  $\Rightarrow t_{\tfrac{m + n}{2}} = ar^{\tfrac{m + n - 2}{2}} =
  \frac{1}{n^2}\left(\frac{n^2}{m^2}\right)^{\tfrac{m - 1}{m - n}}.\left(\frac{m^2}{n^2}\right)^{\tfrac{m +
      n -2}{2(m - n)}} = \frac{1}{mn}$.

  This can be alternatively computed with G.M. formula i.e. $t_{\tfrac{m + n}{2}} = \sqrt{t_mt_n} =
  \frac{1}{mn}$.
\item Given condition is $c > 4b - 3a \Rightarrow c - 4b + 3a > 0 \Rightarrow r^2 - 4r + 3 < 0\;[\because a
  < 0] \Rightarrow r> 3$ or $r < 1$.
\item Given, $(1 - k)(1 + 2x + 4x^2 + 8x^3 + 16x^4 + 32x^5) = 1 - k^6 \Rightarrow (1 - k)\frac{64x^6 - 1}{x
  - 1} = 1 - k^6 \Rightarrow k = 2x \Rightarrow \frac{k}{x} = 2$.
\item Given, $(a^2 + b^2 + c^2)(b^2 + c^2 + d^2)\leq(ab + bc + cd)^2 \Rightarrow (b^2 - ac)^2 + (c^2 - ad)^2
  + (ad - bc)^2 \leq 0$

  Since $a, b, c, d$ are non-zero real numbers therefore the above conditiion leads to equality if and only
  if $b^2 = ac, c^2 = ad, ad = bc$ i.e. $a, b, c, d$ are in G.P.
\item This problem is generalization of previous problem and can be solved similarly.
\item Let $r$ be the common ratio, then $\beta = \alpha r, \gamma = \alpha r^2, \delta = \alpha r^3$.

  From roots of quadratic equaiton $\alpha + \beta = 3, \alpha\beta = a, \gamma + \delta = 12, \gamma\delta
  = b$

  $\frac{\gamma + \delta}{\alpha + \beta} = r^2 = 4 \Rightarrow r = 2$ because G.P. is increasing so we
  discard the negative root.

  $\Rightarrow \alpha = 1 \Rightarrow a = 2, \Rightarrow b = 32$.
\item Let $a$ be the first term of the A.P. Then $t_{2n + 1} = a + 4n$. So the first term of the G.P. is $a
  + 4n$.

  Middle term of A.P. $t_{n + 1} = a + 2n$ and middle term of G.P. $= \frac{a + 4n}{2^n}$

  Given, $a + 2n = \frac{a + 4n}{2^n}$ thus, $a$ can found and hence $a + 4n$ which is the mid term can be
  deduced.
\item $f(x) = 2x + 1, f(2x) = 4x + 1, f(4x) = 8x + 1$. Given that $f(x), f(2x), f(4x)$ are in G.P.

  $\Rightarrow \frac{f(2x)}{f(x)} = \frac{f(4x)}{f(2x)} \Rightarrow (4x + 1)^2 = (2x + 1)(8x + 1)
  \Rightarrow 8x + 1 = 10x + 1 \Rightarrow x = 0$.
\item Let $r$ be the common ratio then $a + b + c = xb \Rightarrow 1 + r + r^2 = xr \Rightarrow x = \frac{1
  + r + r^2}{r} = \frac{1}{r} + 1 + r$. We know that if $r > 0, r + \frac{1}{r} > 2 \Rightarrow x > 3$ and
  if $r < 0, r + \frac{1}{r} < -2 \Rightarrow x < -1$.
\item $x = \frac{1}{1 - a}, y = \frac{1}{1 - b}, z = \frac{1}{1 - c} \Rightarrow \frac{1}{x} = 1 - a,
  \frac{1}{y} = 1 - b, \frac{1}{z} = 1 - c$

  Thus, $\because a, b, c$ are in A.P. where $|a|, |b|, |c| < 1\;\therefore x, y, z$ are also in A.P.
\item $p = \frac{1}{1 + \tan^2x} = \cos^2x; q = \frac{1}{1 + \cot^2y} = \sin^2y$

  $\displaystyle\sum_{k = 0}^{\infty}\tan^{2k}x\cot^{2k}y = \frac{1}{1 - \tan^2x\cot^2y}$

  $\frac{1}{\frac{1}{p} + \frac{1}{q} - \frac{1}{pq}} = \frac{\cos^2x\sin^2y}{\cos^2x + \sin^2y - 1}$

  Dividing numerator and denominator with $\cos^2x\sin^2y$, we get

  $= \frac{1}{\csc^2y + \sec^2x - \csc^2y\sec^2x} = \frac{1}{\tan^2x + \cot^2y + 2 - 1 -\tan^2x - cot^2y
    - \tan^2xcot^2y} = \displaystyle\sum_{k = 0}^{\infty}\tan^{2k}x\cot^{2k}y$.
\item We know that area of an equilateral triangle is $\frac{\sqrt{3}}{4}a^2$, where $a$ is one of the
  sides. In this case $\Delta = \frac{3}{4}$.

  Now the area of sides joining mid-point will have side $\frac{a}{2}$ and terefore area will be
  $\frac{1}{4}$th of the original triangle. This ratio of $\frac{1}{4}$ will continue and areas of all
  triangles will form a G.P. with common ratio of $\frac{1}{4}$. Thus sum of areas of all these triangles $=
  \frac{\tfrac{3}{4}}{1 - \tfrac{1}{4}} = 1$.
\item $1 + |\cos x| + |\cos^2x| + |\cos^3x| + \cdots$ to $\infty = \frac{1}{1 - |\cos x|} = p$(let).

  $\Rightarrow e^{p.\log_e4} = 4^p$. Now given equation is $t^2 - 20t + 64 = 0 \Rightarrow t = 4, 16
  \Rightarrow p = 1, 2 \Rightarrow |\cos x| = 0, 1/2 \Rightarrow x = \pi/2, \pi/3, 2\pi/3$.
\item $1 + |\cos x| + |\cos^2x| + |\cos^3x| + \cdots$ to $\infty = \frac{1}{1 - |\cos x|} \Rightarrow
  \frac{1}{1 - |\cos x|} = 2 \Rightarrow |\cos x| = \frac{1}{2}\Rightarrow \cos x = \pm\frac{1}{2}
  \Rightarrow S = \left\{\frac{\pi}{3}, \frac{2\pi}{3}\right\}$.
\item $\sin^2x + \sin^4x + \cdots$ to $\infty = \frac{\sin^2x}{1 - \sin^2x} = \tan^2x$

  Roots of $x^2 - 9x + 8 = 0$ are $1, 8$ i.e. $2^0, 2^3 \Rightarrow \tan x = 0, \sqrt{3}$ (rejecting
  $-\sqrt{3}$ as for $0< x< \frac{\pi}{2}, \tan x$  cannot be negative.)

  $\frac{\cos x}{\cos x + \sin x} = \frac{1}{1 + \tan x} = 1, \frac{1}{1 + \sqrt{3}}$.
\item $S_\lambda = \frac{\lambda}{\lambda - 1}$ [Hint: It is a G.P.]
  $\displaystyle\sum_{\lambda=1}^n(\lambda - 1)S_\lambda = \sum_{\lambda=1}^n\lambda = \frac{n(n + 1)}{2}$.
\item Let $2^{ax + 1}, 2^{bx + 1}, 2^{cx + 1}$ are in G.P. $\Rightarrow \frac{2^{bx + 1}}{2^{ax + 1}} =
  \frac{2^{cx + 1}}{2^{bx + 1}} \Rightarrow (b - a)x = (c - b)x \Rightarrow b - a = c - b$

  which implies that $a, b, c$ are in A.P. which is a given and hence we have proven required condition in
  reverse.
\item Given $\frac{a + be^x}{a - be^x} = \frac{b + ce^x}{b - ce^x} \Rightarrow ab - ace^x + b^2e^x -
  bce^{2x} = ab + ace^x - b^2e^x - bce^{2x} \Rightarrow 2ace^x = b^2e^x \Rightarrow 2ac = b^2$, which
  implies $a, b, c$ are in G.P. Similarly it can be proven that $b, c, d$ are in G.P. making $a, b, c, d$
  are in G.P.
\item Given, $2\tan^{-1}y = \tan^{-1}x + \tan^{-1}z \Rightarrow \frac{2y}{1 - y^2} = \frac{x + z}{1 - zx}$

  But we are also given that $y^2 = zx \Rightarrow 2y = x + z \Rightarrow x, y, z$ are in A.P. Now $4y^2 =
  (x + z)^2 = 2(x + z) \Rightarrow x = z = y$ but the common values are not necessarily $0$.
\item Given, $b - c = a - b\;[\because a, b, c$ are in A.P.$]$. From second condition $(c - b)^2 = (b - a)a
  \Rightarrow (a - b)^2 = (a - b)a \Rightarrow 2a = b \Rightarrow 3a = c \Rightarrow a:b:c = 1:2:3$.
\item Since $a, b, c$ are in G.P. $\Rightarrow b^2 = ac$. From second condition, $2(\log 2b - \log 3c) =
  \log 3c - \log 2b \Rightarrow 3\log 2b = 3\log 3c \Rightarrow 2b = 3c \Rightarrow b = \frac{2a}{3}, c =
  \frac{4a}{9}$. Clearly, $a$ is the greatest side. Using $\cos$ rule,

  $\cos A = \frac{b^2 + c^2 - a^2}{2bc} = -\frac{1}{2}$ and thus $A > 90^\circ$ making the triangle
  obtuse-angled triangle.
\item Let $\alpha, \beta, \gamma$ are the roots. Then $\alpha + \beta + \gamma = -\frac{b}{c}, \alpha\beta +
  \beta\gamma + \gamma\alpha = \frac{c}{a}, \alpha\beta\gamma = -\frac{d}{a}$. Let $r$ be the common ratio
  of the G.P. then $\beta = \alpha r, \gamma = \alpha r^2$. Also let $\alpha = x$.

  $\frac{c^3}{b^3} = \frac{c^3}{a^3}.\frac{a^3}{b^3} = -\frac{(\alpha\beta + \beta\gamma +
    \gamma\alpha)^3}{(\alpha + \beta + \gamma)^3} = -\left(\frac{x^2r + x^2r^3 + x^2r^2}{x + xr +
    xr^2}\right)^3 = -x^3r^3 = -\alpha\beta\gamma = \frac{d}{a} \Rightarrow c^3a = b^3d$.
\item Clearly $t_n = \frac{1}{2n - 1} \Rightarrow t_{100} = \frac{1}{199}$.
\item The corresponding $p$th and $q$th term in the A.P.would be $\frac{1}{qr}$ and $\frac{1}{rp}$. Let $a$
  be the first term and $d$ be the commond difference of this A.P. Then, $a + (p - 1)d = \frac{1}{qr}$ and
  $a + (q - 1)d = \frac{1}{rp}$. Subtracting $(p - q)d = \frac{p - q}{pqr} \Rightarrow d = \frac{1}{pqr}$.

  $\Rightarrow a = \frac{1}{qr} - \frac{p - 1}{pqr} = \frac{1}{pqr}. \Rightarrow t_r = \frac{1}{pqr} +
  \frac{r - 1}{pqr} = \frac{1}{pq}$. Therefore $r$th term in H.P. would be $pq$.
\item Corrsponding $p$th, $q$th and $r$th term of the A.P. would be $\frac{1}{a}, \frac{1}{b}$ and
  $\frac{1}{c}$. Let $x$ be the first term and $y$ be the c.d. of this A.P. Then,

  $x + (p - 1)y = \frac{1}{a}, x + (q - 1)y = \frac{1}{b}, x + (r - 1)y = \frac{1}{c}$

  $(p - q)y = \frac{b - a}{ab}\Rightarrow (p - q)ab = \frac{b - a}{y}$. Similarly, $(q - r)bc = \frac{c -
  b}{y}$ and $(r - p)ca = \frac{c - a}{y}$. Clearly, $(q - r)bc + (r - p)ca + (p - q)ab = 0$.
\item We have to prove that $\frac{a - b}{b - c} = \frac{a}{c}\Rightarrow ac - bc = ab - ac \Rightarrow 2ac
  = ab + bc$ which prove that $a, b, c$ are in H.P. Thus required equality is proven in reverse.
\item Given $\frac{1}{a}, \frac{1}{b}, \frac{1}{c}, \frac{1}{d}$ are in A.P. Let $p$ be the c.d. of this
  A.P. $\Rightarrow \frac{1}{b} - \frac{1}{a} = p \Rightarrow ab = \frac{a - b}{p}$. Similarly, $bc =
  \frac{b - c}{p}, cd = \frac{c - d}{p}$. Adding these we have $ab + bc + cd = \frac{a - d}{p}$. Now
  $\frac{1}{d} - \frac{1}{a} = 3p \Rightarrow 3ad = \frac{a - d}{p}$. Thus, $ab + bc + cd = 3ad$.
\item Let $d$ be the common difference of the corresponding $A.P.$ Then, $\frac{1}{x_n} - \frac{1}{x_1} = (n
  - 1)d \Rightarrow \frac{x_1 - x_n}{d} = (n - 1)x_1x_n =$ R.H.S.

  Now, $\frac{1}{x_1} - \frac{1}{x_2} = d \Rightarrow \frac{x_1 - x_2}{d} = x_1x_2$. Similarly, $\frac{x_2
    - x_3}{d} = x_2x_3$ and so on till $\frac{x_{n - 1} - x_n}{d} = x_{n - 1}x_n$. Adding these and
  comparing with R.H.S. we get the required equality.
\item $\frac{1}{a}, \frac{1}{b}, \frac{1}{c}$ are in A.P.

  $\Rightarrow \frac{a + b + c}{a}, \frac{a + b + c}{b}, \frac{a + b + c}{c}$ are in A.P.

  $\Rightarrow \frac{a + b + c}{a} -1, \frac{a + b + c}{b} - 1, \frac{a + b + c}{c} -1$ are in A.P.

  $\Rightarrow \frac{a}{b + c}, \frac{b}{c + a}, \frac{c}{a + b}$ are in H.P.
\item $a^2, b^2, c^2$ are in A.P. $\Rightarrow a^2 + ab + bc + ca, b^2 + ab + bc + ca, c^2 + ab + bc + ca$
  are in A.P.

  $\Rightarrow (a + b)(c + a), (b + c)(a + b), (c + a)(b + c)$ are in A.P.

  Dividing each term by $(a + b)(b + c)(c + a)$, we have

  $\frac{1}{b + c}, \frac{1}{c + a}, \frac{1}{a + b}$ are in A.P.

  $\Rightarrow b + c, c + a, a + b$ are in H.P.
\item If $t_n = \frac{1}{3n - 2}$ then the sequence is $1, \frac{1}{4}, \frac{1}{7}, \frac{1}{10}, \cdots$

  Let us assume that it is in H.P. then corresponding $n$th term in A.P. is $3n - 2$. Thus, c.d. $= 3n - 2 -
  (3n - 1) - 2 = 3$ which is a constant so the sequence is in A.P. Thus our assumption is correct and given
  sequence is in H.P.
\item Let $a$ be the first term and $d$ be the c.d. of the corresponding A.P. Then,

  $a + (m - 1)d = \frac{1}{n}$ and $a + (n - 1)d = \frac{1}{m}$. Subtracting, $(m - n)d = \frac{m -
  n}{mn}\Rightarrow d = \frac{1}{mn} \Rightarrow a = \frac{1}{n} - \frac{m - 1}{mn} = \frac{1}{mn}$.

  Then $t_{m + n} = \frac{1}{mn} + (m + n - 1)\frac{1}{mn} = \frac{m + n}{mn}$ thus corrsponding term in
  H.P. would be $\frac{mn}{m + n}$. Also, $t_{mn} = \frac{1}{mn} + \frac{mn - 1}{mn} = 1$ and hence
  corresponding term in H.P. is $1$.
\item Let the three numbers in H.P. are $a, b, c$ then $\frac{1}{a}, \frac{1}{b}, \frac{1}{c}$ will be in
  A.P. Given, $a + b + c = 37, \frac{1}{a} + \frac{1}{b} + \frac{1}{c} = \frac{1}{4}$. Let $d$ be the
  c.d. of the A.P. then $\frac{3}{b} = \frac{1}{4} \Rightarrow b = 12$

  $\Rightarrow \frac{12}{1 - 12d} + 12 + \frac{12}{1 + 12d} = 37 \Rightarrow d = \frac{1}{60}$. So the
  numbers are $15, 12, 10$.
\item $\because a, b, c$ are in H.P. $\therefore b = \frac{2ac}{a + c}$.

  L.H.S. $= \frac{1}{b - a} + \frac{1}{b - c} = \frac{a + c}{ac - a^2} + \frac{a + c}{ac - c^2} = \frac{a + c}{ac} =
  \frac{1}{a} + \frac{1}{c} =$ R.H.S.
\item $\because a, b, c$ are in H.P. $\therefore b = \frac{2ac}{a + c}$.

  L.H.S. $= \frac{b + a}{b - a} + \frac{b + c}{b - c} = \frac{a^2 + 3ac}{ac - a^2} + \frac{c^2 + 3ac}{ac -
    c^2} = \frac{3ac^2 + a^2c - 3a^2c - ac^2}{ac(c - a)} = \frac{2ac^2 - 2a^2c}{ac(c - a)} = 2 =$ R.H.S.
\item Let $d$ be the c.d. of corresponding A.P., then $\frac{1}{x_2} - \frac{1}{x_1} = d \Rightarrow x_1x_2
  = \frac{x_1 - x_2}{d}$ and similarly, $x_2x_3 = \frac{x_2 - x_3}{d}, x_3x_4 = \frac{x_3 - x_4}{d}, x_4x_5
  = \frac{x_4 - x_5}{d}$.

  Adding toegther, $\frac{x_1 - x_5}{d} = x_1x_2 + x_2x_3 + x_3x_4 + x_4x_5 =
  \frac{x_1x_5}{d}\left[\frac{1}{x_1} - \frac{1}{x_5}\right] = 4x_1x_5$. Hence proved.
\item Like previous problem $x_1 - x_3 = 2x_1x_3d$ and $x_2 - x_4 = 2x_2x_4d$ so L.H.S. $= 4x_1x_2x_3x_4d^2$

  And $x_1 - x_2 = x_1x_2d$ and $x_3 - x_4 = x_3x_4d$ so R.H.S. $= 4x_1x_2x_3x_4d^2$ and thus L.H.S. =
  R.H.S.
\item Given $\frac{1}{b + c}, \frac{1}{c + a}, \frac{1}{a + b}$ are in A.P.

  Multiplying with $a + b + c$ and then subtracting $1$ from each term we get required condition.
\item Given $\frac{1}{b + c}, \frac{1}{c + a}, \frac{1}{a + b}$ are in A.P.

  Multiplying each term with $a + b+ c$ and then subtracting $ab + bc + ca$ from each term we get the
  required condition.
\item Given that $a, b, c$ are in A.P. Dividing each term by $abc$, we get that $\frac{1}{bc}, \frac{1}{ca},
  \frac{1}{ab}$ are in A.P. Multiplying each term with $ab + bc + ca$ and then subtracting $1$ from each
  term we get the desired condition.
\item Given that $\frac{1}{a}, \frac{1}{b}, \frac{1}{c}$ are in A.P. Multiplying each term with $a + b + c$
  and then subtracting $2$ from each term we get the desired condition.
\item Given that $\frac{1}{a}, \frac{1}{b}, \frac{1}{c}$ are in A.P. Multiplying each term with $a + b + c$
  and then subtracting $1$ from each term we get the desired condition.
\item Given that $t_n = 12n^2 - 6n + 5$ then $\displaystyle S_n = 12\sum_{i = 1}^ni^2 - 6\sum_{i=1}^ni + 5\sum_{i=1}^n1$

  $= 12.\frac{n(n + 1)(2n + 1)}{6} - 6\frac{n(n + 1)}{2} + 5n = n\left[4n^2 + 6n + 2 - 3n - 3 + 5\right]
  = n(4n^2 + 3n  + 4)$.
\item Clearly $t_n = (2n - 1)^2 = 4n^2 - 4n + 1 \Rightarrow \displaystyle S_n = 4\sum_{i =1}^ni^2 -
  4\sum_{i=1}^ni + \sum_{i=1}^n1$

  $= 4\frac{n(n + 1)(2n + 1)}{6}- 4\frac{n(n + 1)}{2} + n = n\left[\frac{4n^2 + 6n + 2 - 6n - 6 +
      3}{3}\right] = \frac{n(4n^2 -1)}{3}$.
\item Clearly, $t_n = n(n + 1)(n + 2) = n^3 + 3n^2 + 2n \Rightarrow \displaystyle S_n = \sum_{i= 1}^ni^3 +
  3\sum_{i=1}^n i^2 + \sum_{i=1}^ni$

  $= \left[\frac{n(n + 1)}{2}\right]^2 + 3.\frac{n(n + 1)(2n + 1)}{6} + \frac{n(n + 1)}{2} = \frac{n(n +
    1)}{2}\left[\frac{n(n + 1)}{2} + 2n + 1 + 1\right] = \frac{n(n + 1)}{2}.\frac{n^2 + 5n + 4}{2} =
  \frac{n(n + 1)^2(n + 4)}{4}$.
\item $r$th term of the series, $t_r = r(n - r + 1)\Rightarrow\displaystyle S_n = n\sum_{r=1}^nr -
  \sum_{r=1}^nr^2 + \sum_{r=1}^nr$

  $= \frac{n.n(n + 1)}{2} - \frac{n(n + 1)(2n + 1)}{6} + \frac{n(n + 1)}{2} = \frac{n(n + 1)}{2}\left[n -
    \frac{2n + 1}{3} + 1\right] = \frac{n(n + 1)}{2}\left[\frac{3n - 2n - 1 + 3}{3}\right] = \frac{n(n +
    1)(n + 2)}{6}$.
\item   If you see carefully this series is same as previous problem hence sum will be same.
  $t_n = 1 + 2 + 3 + \cdots + n = \frac{n^2 + n)}{2}\Rightarrow \displaystyle t_n =
  \frac{1}{2}\left[\sum_{i=1}^ni^2 + \sum_{i=1}^ni\right]$

  $=\frac{1}{2}\left[\frac{n(n + 1)(2n + 1)}{6} + \frac{n(n + 1)}{2}\right] = \frac{n(n +
    1)}{4}\left[\frac{2n + 1}{3} + 1\right] = \frac{n(n + 1)(n + 2)}{6}$.
\item First term contains $1$ integer, second term contains $2$ and so on. So before $t_n$ we will have $1 +
  2 + \cdots + (n - 1)$ integers i.e. $\frac{n(n - 1)}{2}$ integers. So $t_n$ will start with $\frac{n(n -
    1) + 2}{2}$ and will have $n$ integers. So $t_n = \frac{n^2 - n + 2}{2} $ and now it is trivial to find
  the sum, which will be $\displaystyle S_n = \frac{1}{2}\sum_{i=1}^ni^2 -
  \frac{1}{2}\sum_{i=1}^ni + \sum_{i=1}^n1 = \frac{n(n + 1)(2n + 1)}{12} - \frac{n(n + 1)}{2} + n$
  simplification is left to you.
\item Let $nt_n$ represent numerator and $dt_n$ be the denominator of the $n$th term $t_n$. Then $nt_n =
  \left[\frac{n(n + 1)}{2}\right]^3$ and $dt_n = \frac{n}{2}[2 + (n - 1)2] = n^2$

  $\displaystyle\Rightarrow t_n = \left(\frac{n + 1}{2}\right)^2 = \frac{n^2 + 2n + 1}{2}\Rightarrow S_n =
  \frac{1}{2}\sum_{i=1}^ni^2 + \sum_{i=1}^ni + \frac{1}{2}\sum_{i=1}^n1 = \frac{n(n + 1)(2n + 1)}{12} +
  \frac{n(n + 1)}{2} + \frac{n}{2}$. Simplify and put $n=16$ to arrive at the answer.
\item $t_n = [(2n + 1)^3 - (2n)^3] = 12n^2 + 6n + 1\Rightarrow\displaystyle S_n = 12\sum_{i=1}^ni^3 +
  6\sum_{i=1}^n i + \sum_{i=1}^n1 = 12\frac{n(n + 1)(2n + 1)}{6} + 6\frac{n(n + 1)}{2} + n = 2n(n + 1)(2n +
  1) + 3n(n + 1) + n$. Simplify and put $n=10$ to get the answer.
\item $t_1 = \frac{1}{1} - \frac{1}{2}, t_2 = \frac{1}{2} - \frac{1}{3} \cdots t_n = \frac{1}{n} -
  \frac{1}{n + 1}$. Adding $S_n = \frac{1}{1} - \frac{1}{n + 1} = \frac{n}{n + 1}$.
\item $t_n = \frac{1}{n(n + 1)(n + 2)} = \frac{1}{2}\left[\frac{1}{n(n + 1)} - \frac{1}{(n + 1)(n +
    2)}\right] = \frac{1}{2}\left[\frac{1}{n} - \frac{2}{n + 1} + \frac{1}{n + 2}\right]$

  Then, $t_1 = \frac{1}{2.1} - \frac{1}{2} + \frac{1}{2.3}, t_2 = \frac{1}{2.2} - \frac{1}{3} +
  \frac{1}{2.4}, t_3 = \frac{1}{2.3} - \frac{1}{4} + \frac{1}{2.5}, \ldots, t_{n - 2} = \frac{1}{2(n - 1)} -
  \frac{1}{n - 1} + \frac{1}{2n}, t_{n - 1} = \frac{1}{2(n - 1)} - \frac{1}{n} + \frac{1}{2(n + 1)}, t_n =
  \frac{1}{2.n} - \frac{1}{n + 1} + \frac{1}{2(n + 2)}$

  $\Rightarrow S_n = \frac{1}{2.1} - \frac{1}{2} + \frac{1}{2.2} + \frac{1}{2(n + 1)}- \frac{1}{n + 1} +
  \frac{1}{2(n + 2)} = \frac{1}{4} - \frac{1}{2(n + 1)} + \frac{1}{2(n + 2)}\Rightarrow S_\infty =
  \frac{1}{4}$
\item $\startalign\NC S_n = \NC1 + 5 + 11 + 19 + \cdots + t_{n - 1} + t_n\NR\NC S_n = \NC\;\;\;\;\;\; 1 + 5 + 11 +
  \cdots + t_{n - 1} + t_n\NR\stopalign$

  Subtracting, we get $t_n = 1 = [4 + 6 + 8 + \cdots$ to $(n - 1)$ terms $] = 1 + \frac{n - 1}{2}[2.4 + (n - 2)2]
  = n^2 + n - 1\Rightarrow S_n = \displaystyle\sum_{i=1}^ni^2 + \sum_{i=1}^ni - \sum_{i=1}^n 1 = \frac{n(n +
    1)(2n + 1)}{6} + \frac{n(n + 1)}{2} - n = \frac{n(n^2 + 3n - 1)}{3}$.
\item First person gets $1$ repee, second person gets $1 + 1 = 2$ rupee, third person gets $2 + 2= 4$ rupee,
  fourth person gets $4 + 3 = 7$ rupee and so on.

  $\startalign\NC S_n  = \NC 1 + 2 + 4 + 7 + \cdots + t_n\NR\NC S_n  = \NC\;\;\;\;\;\; 1 + 2 + 4 + \cdots +
  t_{n - 1} + t_n\NR\stopalign$

  Subtracting, we get  $t_n = 1 + [1 + 2 + 3 + \cdots\;\mathrm{to}\;(n - 1)\;\mathrm{terms}] = 1 + \frac{n -
    1}{2}[2.1 + (n - 2)] = \frac{n^2 - n + 2}{2} = 67 \Rightarrow n^2 - n - 132 = 0 \Rightarrow n = 12$.
\item First term contains $1$ integer, second term contains $2$ and so on. So before $t_n$ we will have $1 +
  2 + \cdots + (n - 1)$ integers i.e. $\frac{n(n - 1)}{2}$ integers. So $t_n$ will start with $\frac{n(n -
    1) + 2}{2}$ and will have $n$ integers. So $t_n = \frac{n^2 - n + 2}{2}$. This will be the first number
  in $n$th group. So sum of $n$th group $= \frac{n}{2}[n^2 - n + 2 + n - 1] = \frac{n(n^2 + 1)}{2}$.
\item $\startalign\NC S_n = \NC 1 + 3 + 7 + 15 + \cdots + t_n\NR\NC S_n = \NC\;\;\;\;\;\;\;\;1 + 3 + 7 + \cdots +
  t_{n - 1} + t_n\NR\stopalign$

  Subtracting, we have $t_n = 1 + [2 + 4 + 8 + \cdots\;\mathrm{to}\;(n - 1)\;\mathrm{terms}] = 1 +
  \frac{2(2^{n - 1} - 1)}{2 - 1} = 2^n - 1 \Rightarrow S_n = (2 - 1) + (2^2 - 1) + (2^3 - 1) + \cdots + (2^n
  - 1) = \frac{2(2^n - 1)}{2 - 1} - n = 2^{n + 1} - 2 - n$.
\item $\startalign\NC S_n = \NC 1 + 2x + 3x^2 + \cdots + t_n\NR\NC xS_n = \NC\;\;\;\;\;\;1.x + 2x^2 + \cdots +
  t_{n - 1} + t_n\NR\stopalign$

  Subtracting we get $(1 - x)S_n = 1 + x + x^2 + \cdots\;\mathrm{to}\;n\;\mathrm{terms} - xt_n = \frac{1 -
    x^n}{1 - x} - x.nx^{n - 1} \Rightarrow S_n = \frac{1 - x^n}{(1 - x)^2} - \frac{nx^n}{1 - x}$.
\item Given $\startalign\NC S_{100} = \NC1 + 2.2 + 3.2^2 + 4.3^3 + \cdots + 100.2^{99}\NR\NC 2.S_{100} =
  \NC\;\;\;\;\;\; 1.2 + 2.2^2 + 3.2^3 + \cdots + 99.2^{99} + 100.2^{100}\NR\stopalign$

  Subtracting, we get $-S_n = 1 + [2 + 2^2  + 2^3 + \cdots\;\mathrm{to}\;99\;\mathrm{terms}] -
  100.2^{100}$

  $S_n = 100.2^{100} - \frac{2^{100} - 1}{2 - 1} = 99.2^{100} + 1$.
\item Clearly $\startalign\NC S = \NC 1 + 2^2x + 3^2x^2 + 4^2x^3 + \cdots\;\mathrm{to}\;\infty\NR\NC xS
  = \NC\;\;\;\;\;\;\;\;\;\;x + 2^2x^2 + 3^2x^3 + \cdots\;\mathrm{to}\;\infty\NR\stopalign$

  Subtracting, we get $\startalign\NC(1 - x)S = \NC 1 + 3x + 5x^2 + 7x^3 + \cdots\;\mathrm{to}\;\infty\NR\NC
  x(1 - x)S = \NC\;\;\;\;\;\;\;\;x + 3x^2 + 5x^3 + \cdots\;\mathrm{to}\;\infty\NR\stopalign$

  Again subtracting, $(1 - x)^2S = 1 + 2x + 2x^2 + 2x^3 + \cdots\;\mathrm{to}\;\infty = 1 + \frac{2x}{1 - x}
  = \frac{1 + x}{1 - x}\Rightarrow S = \frac{1 + x}{(1 - x)^2}$.
\item $S_n = 2n^2 + 4, t_n = S_n - S_{n - 1} = 2n^2 + 4 - 2(n - 1)^2 - 4 = 4n -2 \Rightarrow d = t_n - t_{n
  - 1} = 4n - 2 - 4(n - 1) + 2 = 4$ which is constant therefore the given sequence is in A.P.

  Hint: Any sequence which is of the for which sum is of the form $an^2 + bn + c$ will lead to an A.P.
\item Given $t_n = n(n - 1)(n + 1) = n^3 - n \Rightarrow S_n = \displaystyle\sum_{i=1}^ni^3 - \sum_{i=1}^ni =
  \left[\frac{n(n + 1)}{2}\right]^2 - \frac{n(n + 1)}{2} = \frac{n(n + 1)(n^2 + n - 2)}{4}$.
\item Clearly, $\displaystyle t_n = (2n - 1)^3 = 8n^3 - 12n^2 + 6n - 1 \Rightarrow S_n = 8\sum_{i=1}^ni^3 -
  12\sum_{i=1}^ni^2 + 6\sum_{i=1}^ni - \sum_{i=1}^n1 = 2n^2(n + 1)^2 - 2n(n + 1)(2n + 1) + 3n(n + 1) - n$;
  simplification is left to you.
\item Clearly, $t_n = (3n - 2)^2 = 9n^2 - 12n + 4 \Rightarrow S_n = \displaystyle9\sum_{i=1}^ni^2 -
  12\sum_{i=1}^ni + 4\sum_{i=1}^n1 = \frac{3n(n + 1)(2n + 1)}{2} - 6n(n + 1) + 4n$; simplification is
  left to you.
\item Given series is $1^2 + 3^2 + 5^2 + \cdots\;\mathrm{to}\;n\;\mathrm{terms} + 2 + 4 + 6 +
  \cdots\;\mathrm{to}\;n\;\mathrm{terms}$.

  $\Rightarrow t_n = (2n - 1)^2 + \frac{n}{2}[2.2 + (n - 1)2] = 4n^2 - 4n + 1 + n^2 + n = 5n^2 - 3n + 1$

  $\Rightarrow S_n = 5\sum_{i=1}^ni^2 - 3\sum_{i=1}^ni + \sum_{i=1}^n1 = \frac{5n(n + 1)(2n + 1)}{6} -
  \frac{3n(n + 1)}{2} + n$; simplification is left to you.
\item {\bf Case I:} When $n$ is even. Let $n = 2m$ then $S = 1^2 + 3^2 + 5^2 +
  \cdots\;\mathrm{to}\;m\;\mathrm{terms} - [2^2 + 4^2 + 6^2 + \cdots\;\mathrm{to}\;m\;\mathrm{terms}]$

  $= \sum_{i=1}^m(2i - 1)^2 - \sum_{i=1}^m(2i)^2 = -4\sum_{i=1}^mi + \sum_{i=1}^m1 = -2m(m + 1) + 4m = -2m^2
  + 2m$ and then we substitute $m = \frac{n}{2}$.

  {\bf Case II:} When $n$ is odd. Let $n = 2m + 1$, then $S = 1^2 + 3^2 + 5^2 +
  \cdots\;\mathrm{to}\;(m + 1)\;\mathrm{terms} - [2^2 + 4^2 + 6^2 + \cdots\;\mathrm{to}\;m\;\mathrm{terms}]$

  $= \displaystyle\sum_{i=1}^{m + 1}(2i - 1)^2 - \sum_{i=1}^m(2i)^2 = \frac{4(m + 1)(m + 2)(2m + 3)}{6} -
  2(m + 1)(m + 2) + (m + 1) - \frac{2m(m + 1)(2m + 1)}{3}$; put $m = \frac{n - 1}{2}$ and simplify.
\item Clearly, $t_n = (2n - 1)(2n + 1) = 4n^2 - 1\Rightarrow S_n = \displaystyle4\sum_{i=1}^ni^2 -
  \sum_{i=1}^n1 = \frac{2n(n + 1)(2n + 1)}{3} - n$; simplification is left to you.
\item Clearly, $t_n = n(n + 1) \Rightarrow\displaystyle S_n = \sum_{i=1}^ni^2 + \sum_{i=1}^ni = \frac{n(n +
  1)(2n + 1)}{6} + \frac{n(n + 1)}{2}$; simplification is left to you.
\item Clearly, $t_n = n(n + 1)^2 = n^3 + 2n^2 + n \Rightarrow \displaystyle S_n = \sum_{i=1}^ni^3 +
  2\sum_{i=1}^ni^2 + \sum_{i=1}^ni = \left[\frac{n(n + 1)}{2}\right]^2 + \frac{n(n + 1)(2n + 1)}{3} +
  \frac{n(n + 1)}{2}$; simplification is left to you.
\item Clearly, $t_n = (n + 1)n^2 = n^3 + n^2 \Rightarrow \displaystyle S_n = \left[\frac{n(n +
    1)}{2}\right]^2 + \frac{n(n + 1)(2n + 1)}{6}$;  simplification is left to you.
\item $t_n = 1 + 3 + 5 + \cdots\;\mathrm{up to}\;n\;\mathrm{terms} = \frac{n}{2}[2.1 + (n - 1)2] = n^2
  \Rightarrow \displaystyle S_n = \sum_{i=1}^ni^2 = \frac{n(n + 1)(2n + 1)}{6}$.
\item $t_n = 1^2 + 2^2 + 3^2 + \cdots\;\mathrm{up to}\;n\;\mathrm{terms} = \displaystyle\sum_{i=1}^ni^2 =
  \frac{n(n + 1)(2n + 1)}{6} = \frac{n^3 + 3n^2 + n}{6}$.

  $S_n = \frac{1}{6}\left[\sum_{i=1}^ni^3 + 3\sum_{i=1}^ni^2 + \sum_{i=1}^ni\right] =
  \frac{1}{6}\left[\frac{n^2(n + 1)^2}{4} + \frac{n(n + 1)(2n + 1)}{2} + \frac{n(n + 1)}{2}\right]$;
  simplification is left to you.
\item $t_n = n(n + 1)(2n + 1) = 2n^3 + 3n^2 + n \Rightarrow S_n = \displaystyle 2\sum_{i=1}^ni^3 +
  3\sum_{i=1}^ni^2 + \sum_{i=1}^ni = \frac{n^2(n + 1)^2}{2} + \frac{n(n + 1)(2n + 1)}{2} + \frac{n(n +
    1)}{2}$; simplification is left to you.
\item $t_n = n(n + 1)(n + 2) = n^3 + 3n^2 + 2n \Rightarrow \displaystyle S_n = \sum_{i=1}^ni^3 +
  3\sum_{i=1}^ni^2 + 2\sum_{i=1}^ni = \frac{n^2(n + 1)^2}{4} + \frac{n(n + 1)(2n + 1)}{2} + n(n + 1)$;
  simplification is left to you.
\item $t_n = n(2n + 1)^2 = 4n^3 + 4n^2 + n \Rightarrow \displaystyle S_n = 4\sum_{i=1}^ni^3 +
  4\sum_{i=1}^ni^2 + \sum_{i=1}^ni = n^2(n + 1)^2 + \frac{2n(n + 1)(2n + 1)}{3} + \frac{n(n + 1)}{2}$;
  put $n = 20$ and simplify.
\item $t_r = r(n^2 - r^2) = n^2r - r^3 \Rightarrow S = n^2\sum_{i=1}^ni - \sum_{i=1}^ni^3 = \frac{n^3(n +
  1)}{2} - \frac{n^2(n + 1)^2}{4}$; simplification is left to you.
\item $t_n = (2n + 1)^3 - (2n)^3 = 12n^2 + 6n + 1 \Rightarrow S_n = \displaystyle 12\sum_{i=1}^ni^2 +
  6\sum_{i=1}^ni + \sum_{i=1}^n1 = 2n(n + 1)(2n + 1) + 3n(n + 1) + n$; put $n = 10$ to get the answer.
\item $t_n = \frac{1}{1 + 2 + 3 + \cdots\;\mathrm{to}\;n\;\mathrm{terms}} = \frac{2}{n(n + 1)} =
  2\left[\frac{1}{n} - \frac{1}{n + 1}\right]$

  $t_1 = 2\left[1 - \frac{1}{2}\right], t_2 = 2\left[\frac{1}{2} - \frac{1}{3}\right], t_3 =
  2\left[\frac{1}{3} - \frac{1}{4}\right], \ldots, t_n = 2\left[\frac{1}{n} - \frac{1}{n  + 1}\right]$.

  Adding, $S = 2\left[1 - \frac{1}{n + 1}\right] = \frac{2n}{n + 1}$.
\item $S = \frac{1}{2.4} + \frac{1}{4.6} + \frac{1}{6.8} + \frac{1}{8.10} + \ldots = 2\left[\frac{1}{2} -
  \frac{1}{4} + \frac{1}{4} - \frac{1}{6} + \frac{1}{6} - \frac{1}{8} + \cdots\;\mathrm{to}\;\infty\right] =
  1$.
\item $\startalign\NC S = \NC2 + 6 + 12 + 20 + \cdots + t_n\NR\NC S = \NC\;\;\;\;\;\;\; 2 + 6 + 12 + \cdots +
  t_{n - 1} + t_n\NR\stopalign$

  Subtracting, $t_n = 2 + 4 + 6 + 8 + \cdots\;\mathrm{to}\;n\;\mathrm{terms} = \frac{n}{2}[2.2 + (n - 1)2] =
  n(n + 1) = n^2 + n \Rightarrow S = \displaystyle \sum_{i=1}^ni^2 + \sum_{i=1}^ni = \frac{n(n + 1)(2n +
    1)}{6} + \frac{n(n + 1)}{2}$; simplification is left to you.
\item $\startalign\NC S = \NC 3 + 6 + 11 + 18 + \cdots + t_n\NR\NC S = \NC\;\;\;\;\;\;3 + 6 + 11 + 18 +
  \cdots + t_{n - 1} + t_n\NR\stopalign$

  Subtracting,
  $t_n = 3 + [3 + 5 + 7 + \cdots\;\mathrm{to}\;(n - 1)\;\mathrm{terms}] = 3 + \frac{n - 1}{2}[2.3 + (n - 2)2]
    = 3 + n^2 - 1 = n^2 + 2$

  $S = \frac{n(n + 1)(2n + 1)}{6} + 2n$; simplification is left to you.
\item $\startalign\NC S = \NC 1 + 9 + 24 + 46 + 75 + \cdots + t_n\NR\NC S = \NC\;\;\;\;\;\;1 + 9 + 24 + 46 +
  \cdots + t_{n - 1} + t_n\NR\stopalign$

  Subtracting $t_n = 1 + 8 + 15 + 22 + 29 + \cdots\;\mathrm{to}\;n\mathrm{terms} = \frac{n}{2}[2 + (n - 1)7]
  = \frac{7n^2 - 5n}{2}$.

  $\Rightarrow S = \frac{7n(n + 1)(2n  1)}{12} - \frac{5n(n + 1)}{4}$.
\item $\startalign\NC S = \NC2 + 4 + 7 + 11 + 16 + \cdots + t_n\NR\NC S = \NC\;\;\;\;\;\;2 + 4 + 7 + 11 +
  \cdots + t_{n - 1} + t_n\NR\stopalign$

  Subtracting, $t_n = 2 + [2 + 3 + 4 + 5 + \cdots\;\mathrm{to}\;(n - 1)\;\mathrm{terms}] = 2 + \frac{n -
    1}{2}[2.2 + n - 1] = 2 + \frac{n^2 + 2n - 3}{2} = \frac{n^2 - 2n + 1}{2}$.
\item $\startalign\NC S = \NC 1 + 3 + 6 + 10 + \cdots + t_n\NR\NC S = \NC\;\;\;\;\;\;1 + 3 + 6 + \cdots +
  t_{n - 1} + t_n\NR\stopalign$

  Subtracting, $t_n = 1 + 2 + 3 + 4 + \cdots\;\mathrm{to}\;n\mathrm{terms} = \frac{n(n + 1)}{2} = \frac{n^2
    + n}{2}$

  $\Rightarrow S = \frac{n(n + 1)(2n + 1)}{12} + \frac{n(n + 1)}{4}$. Put $n = 10$ to get the answer.
\item First group contains $2$ odd numbers, second group contains $4$ odd numbers, third group contains $6$
  odd numbers so $(n - 1)$th group will contain $2n - 2$ odd numbers.

  Total no. of odd numbers till $(n - 1)$th group will be $n(n - 1)$. So last no. in $(n - 1)$th group will
  be $1 + (n^2 - n - 1)2 = 2n^2 - 2n - 1$ and hence first number in $n$th group will be $2n^2 - 2n + 1$ and
  there will be $2n$ odd numbers. So sum of $2n$ odd numbers starting from $2n^2 - 2n + 1$ is given by
  $\frac{2n}{2}[4n^2 - 4n + 2 + (2n - 1)2] = 4n^3$.
\item Groups contain $1, 3, 5, \ldots$ number of terms so $n$th group will contain $2n - 1$ numbers starting
  from $n$. So sum will be $\frac{2n - 1}{2}[2n + 2n - 2] = (2n - 1)^2$ which is square of odd positive
  integer.
\item $\startalign\NC S = \NC 2 + 5 + 14 + 41 + \cdots + t_n\NR\NC S = \NC\;\;\;\;\;\; 2 + 5 + 14 + \cdots +
  t_{n - 1} + t_n\NR\stopalign$

  Subtracting $t_n = 2 + [3 + 3^2 + \cdots\;\mathrm{to}\;(n - 1)\mathrm{terms}] = 2 + \frac{3(3^{n - 1} -
    1)}{3 - 1} = \frac{3^n + 1}{2}$.

  $\Rightarrow S = \frac{1}{2}\left[\frac{3(3^n - 1)}{2} + n\right]$.
\item $\startalign\NC S = \NC1.1 + 2.3 + 4.5 + 8.7 + \cdots + t_n\NR\NC 2S = \NC\;\;\;\;\;\;\;\;\; 2.1 + 4.3 +
  8.5 + \cdots + t_{n - 1} + 2^n(2n - 1)\NR\stopalign$

  Subtracting, $-S = 1.1 + [2.2 + 4.2 + 8.2 + \cdots\;\mathrm{to}\;(n - 1)\;\mathrm{terms}] - 2^n(2n - 1)$

  $S = 2^n(2n - 1) - 1 - 4(2^{n - 1} - 1)$.
\item Clearly, $a_{2n} - a_1 = (2n - 1)d \Rightarrow d = \frac{a_{2n} - a_1}{2n - 1}$

  Now, $a_1^2 - a_2^2 + a_3^2 - a_4^2 + \cdots + a_{2n - 1}^2 - a_{2n}^2 = (a_1 - a_2)(a_1 + a_2) + (a_3 -
  a_4)(a_3 + a_4) + \cdots + (a_{2n - 1} - a_{2n})(a_{2n - 1} + a_{2n})$

  $= -d(a_1 + a_2 + a_3 + a_4 + \cdots + a_{2n - 1} + a_{2n}) = -\frac{a_{2n} - a_1}{2n -
    1}.\frac{2n}{2}\left[a_1 + a_{2n}\right] = \frac{n}{2n - 1}(a_1^2 - a_{2n}^2)$.
\item $d = \alpha_2 - \alpha_1 = \alpha_3 - \alpha_2 = \cdots = \alpha_n - \alpha_{n - 1}$

  $\sin d\sec\alpha_1\sec\alpha_2 = \frac{\sin(\alpha_2 - \alpha_1)}{\cos\alpha_1\cos\alpha_2} =\tan\alpha_2 -
  \tan\alpha_1$. Similarly, $\sin d\sec\alpha_2\sec\alpha_3 = \tan\alpha_3 - \tan\alpha_2$ and so
  on. $\sin d\sec\alpha_{n - 1}\sec\alpha_n = \tan\alpha_n - \tan\alpha_{n - 1}$

  Adding we get L.H.S. = R.H.S.
\item L.H.S. $= \frac{1}{a_1 + a_n}\left[\frac{a_1 + a_n}{a_1a_n} + \frac{a_1 + a_n}{a_2a_{n - 1}} + \cdots
  + \frac{a_1 + a_n}{a_na_1}\right] = \frac{1}{a_1 + a_n}\left[\frac{a_1 + a_n}{a_1a_n} + \frac{a_2 +
    a_{n - 1}}{a_2a_{n - 1}} + \cdots + \frac{a_1 + a_n}{a_na_1}\right]$

  $= \frac{1}{a_1 + a_n}\left[\frac{1}{a_1} + \frac{1}{a_n} + \frac{1}{a_2} + \frac{1}{a_{n - 1}} + \cdots +
  \frac{1}{a_n} + \frac{1}{a_1}\right] = \frac{2}{a_1 + a_n}\left(\frac{1}{a_1} + \frac{1}{a_2} + \cdots +
  \frac{1}{a_n}\right)$.
\item $\frac{1}{a_1} - \frac{1}{a_2} = \frac{a_2 - a_1}{a_1a_2} = \frac{d}{a_1a_2} \Rightarrow
  \frac{1}{a_1a_2} = \frac{1}{d}\left(\frac{1}{a_1} - \frac{1}{a_2}\right)$. Similarly $\frac{1}{a_2a_3}=
  \frac{1}{d}\left(\frac{1}{a_2} - \frac{1}{a_3}\right)$ and so on.

  $\therefore S = \frac{1}{d}\left(\frac{1}{a_1} - \frac{1}{a_{n + 1}}\right) = \frac{n}{a_1a_{n 1}}$
\item $\because a_1 = 0$ then $a_2 = d, a_3 = 2d, \ldots, a_n = (n - 1)d$ where $d$ is the c.d. of the A.P.

  L.H.S. $= \frac{2}{1} + \frac{3}{2} + \frac{4}{3} + \cdots + \frac{n - 1}{n - 2} - \left(1 + \frac{1}{2} +
  \frac{1}{3} + \cdots + \frac{1}{n - 3}\right)$

  $= (1 + 1) + \left(1 + \frac{1}{2}\right) + \cdots \+ \left(1 + \frac{1}{n - 2}\right) - \left(1 +
  \frac{1}{2} + \frac{1}{3} + \cdots + \frac{1}{n - 3}\right)$

  $= n - 2 + \left[\left(1 + \frac{1}{2} + \frac{1}{3} + \cdots + \frac{1}{n - 2}\right) - \left(1 +
    \frac{1}{2} + \cdots + \frac{1}{n - 3}\right)\right]$

  $= n - 2 + \frac{1}{n - 2} = \frac{a_{n - 1}}{a_2} + \frac{a_2}{a_{n - 1}} =$ R.H.S.
\item L.H.S. $= \displaystyle\sum_{k=1}^n\frac{a_ka_{k+1}a_{k+2}}{(a_{k + 1} - d) + (a_{k + 1} + d)} = \frac{1}{2}\sum_{k
  = 1}^na_ka_{k + 2} = \frac{1}{2}\sum_{i = 1}^k(a_{k + 1}^2 - d^2) = \frac{1}{2}\sum_{k = 1}^n[(a_1 + kd)^2 -
  d^2] = \frac{1}{2}\sum_{k = 1}^n[a_1^2 + 2a_1dk + (k^2 - 1)d^2]$

  $= \displaystyle\frac{1}{2}\left[\sum_{k = 1}^na_1^2 + 2a_1d\sum_{k = 1}^nk + d^2\sum_{k = 1}^nk^2 - \sum_{k = 1}^nd^2\right] =
  \frac{1}{2}\left[na_1^2 + 2a_1d\frac{n(n + 1)}{2} + d^2\frac{n(n + 1)(2n + 1)}{6} - nd^2\right]$

  $= \frac{n}{2}\left[a_1^2 + (n + 1)a_1d + \frac{(n - 1)(2n + 5)}{6}d^2\right] =$ R.H.S.
\stopitemize
