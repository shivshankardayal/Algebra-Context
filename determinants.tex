\chapter{Determinants}
Let $a,b,c,d$ be any four numbers, real or complex, then the symbol \startformula\startdeterminant \NC a \NC b \NR\NC  c \NC d\NR \stopdeterminant\stopformula denotes $ad
- bc$ and is called a {\it determinant} of second order. $a,b,c,d$ are called elements of the determinant and $ad - bc$ is
called value of the determinant.

As you can see, the elements of a determinant are positioned in the form of a square in its designation. The diagonal on which
elements aa and dd lie is called the principal or primary diagonal of the determinant and the diagonal which is formed on the line
of bb and cc is called the secondary diagonal. A row is constituted by elements lying in the same horizontal line and a column is
constituted by elements lying in the same vertical line. Clearly, determinant of second order has two rows and two columns and its
value is equal to the products of elements along primary diagonal minus the product of elements along the secondary diagonal. Thus,
by definition

\startformula \startdeterminant\NC 2 \NC 4\NR\NC 3 \NC 9\NR\stopdeterminant = 18 - 12 = 6\stopformula

Let $a_1, a_2, a_3, b_1, b_2, b_3, c_1, c_2, c_3$ be any nine numbers, then the symbol

\startformula \startdeterminant\NC a_1 \NC a_2 \NC a_3\NR\NC b_1 \NC b_2 \NC b_3\NR\NC c_1 \NC c_2 \NC c_3\NR\stopdeterminant\stopformula
is another way of saying
\startformula a_1\startdeterminant\NC b_2 \NC b_3\NR\NC c_2 \NC c_3\NR\stopdeterminant - a_2\startdeterminant\NC b_1 \NC b_3\NR\NC c_1 \NC c_3\NR\stopdeterminant + a_3\startdeterminant
b_1 \NC b_2\NR c_1 \NC c_2\NR\stopdeterminant\stopformula
i.e. $a_1(b_2c_3 - b_3c_2)-a_2(b_1c_3-b_3c_1) + a_3(b_1c_2-b_2c_1)$

{\bf Rule to put + or - before any element:} Find the sum of number of rows and columns in which the considered element
occus. If the sum is even put a $+$ sign before the element and if the sum is odd, put a $-$ sign before the element. Since $a_1$
occurs in first row and first column whose sum is $1 + 1 = 2$ which is an even number, therefore $+$ sign occurs for it. Since
$a_2$ occurs in first row and second column whose sum is $1+ 2 = 3$ which is an odd number, therefore $-$ sign occurs before it.

We have expanded the determinant along first row in previous case. The value of determinant does not change no matter which row or
column we expand it along. Expanding the determinant along second row, we get

\startformula \startdeterminant\NC a_1 \NC a_2 \NC a_3\NR\NC b_1 \NC b_2 \NC b_3\NR\NC c_1 \NC c_2 \NC c_3\NR\stopdeterminant= -b_1\startdeterminant\NC a_2 \NC a_3\NR\NC c_2 \NC
  c_3\NR\stopdeterminant + b_2\startdeterminant\NC a_1 \NC a_3\NR\NC c_1 \NC c_3\NR\stopdeterminant- b_3\startdeterminant\NC a_1 \NC a_2\NR\NC c_1 \NC c_2\NR\stopdeterminant\stopformula
\startformula = -b_1(a_2c_3 - a_3c_2) + b_2(a_1c_3 - a_3c_1) - b_3(a_1c_2 - a_2c_1)\stopformula
\startformula = a_1(b_2c_3 - b_3c_2)-a_2(b_1c_3-b_3c_1) + a_3(b_1c_2-b_2c_1)\stopformula

Thus, we see that value of determinant remains unchanged irrespective of the change of row and column against which it is expanded.

Usually, an element of a determinant is denoted by a letter with two suffices, first one indicating the row and second one
indicating the column in which the element occcur. Thus, $a_{ij}$ element indicates that it has occurred in $i$th row and $j$th
column. We also denote the rows by $R_1, R_2, R_3$ and so on. $R_i$ denotes the $i$th row of determinant while $R_j$ denotes $j$th
row. Columns are denoted by $C_1, C_2, C_3$ and so on. $C_i$ and $C_j$ denote $i$th and $j$th column of determinant. $\Delta$ is the
usual symbol for a determinant. Another way of denoting the determinant \startformula \startdeterminant\NC a_1\NC b_1\NC c_1\NR\NC a_2\NC b_2\NC c_2\NR\NC a_3\NC b_3\NC c_3
\NR\stopdeterminant\stopformula is $(a_1b_2c_3)$. The expanded form of determinant has $n!$ terms where $n$ is the number of rows or columns.

{\bf Ex 1.} Find the value of the determinant \startformula \Delta = \startdeterminant\NC 1 \NC 2 \NC 4\NR\NC 3 \NC 4 \NC 9\NR\NC 2 \NC 1 \NC 6\NR\stopdeterminant\stopformula

\startformula \Delta = 1\startdeterminant\NC 4 \NC 9\NR\NC 1 \NC 6\NR\stopdeterminant -2\startdeterminant\NC 3 \NC 9\NR\NC 2 \NC 6\NR\stopdeterminant + 4\startdeterminant\NC 3 \NC 4\NR\NC 2 \NC
  1\NR\stopdeterminant\stopformula

Expanding the determinant along first row $= 1(24 -9) - 2(18 - 18) + 4(3 - 8) = -5$

{\bf Ex 2.} Find the value of the determinant \startformula \Delta = \startdeterminant\NC 3 \NC 1 \NC 7\NR\NC 5 \NC 0 \NC 2\NR\NC 2 \NC 5 \NC 3\NR\stopdeterminant\stopformula

Expanding the determinant along second row, \startformula \Delta = -5\startdeterminant\NC 1 \NC 7\NR\NC 5 \NC 3\NR\stopdeterminant + 0\startdeterminant\NC 3 \NC 7\NR 2 \NC 3\NR\stopdeterminant -2\startdeterminant\NC 3 \NC 1\NR\NC 2 \NC 5\NR\stopdeterminant\stopformula

$= -5(3 - 35) -2(15 -2) = 134$

\section{Minors}
Consider the determinant \startformula \Delta = \startdeterminant\NC a_{11} \NC a_{12} \NC a_{13}\NR\NC a_{21} \NC a_{22} \NC a_{23}\NR\NC a_{31} \NC a_{31} \NC
  a_{33}\NR\stopdeterminant\stopformula

If we leave the elements belonging to row and column of a particular element $a_{ij}$ then we will obtain a second order
determinant. The determinant thus obtained is called minor of $a_{ij}$ and it is denoted by $M_{ij},$ since there are $9$ elements
in the above determinant we will have $9$ minors.

For example, the minor of element \startformula a_{21}=\startdeterminant\NC a_{12} \NC a_{13}\NR\NC a_{32} \NC a_{33}\NR\stopdeterminant = M_{21}\stopformula

The minor of element \startformula a_{32} = \startdeterminant\NC a_{11} \NC a_{13}\NR\NC a_{21} \NC a_{23}\NR\stopdeterminant = M_{32}\stopformula

If we want to write the determinant in terms of minors then following is the expression obtained if we expand it along first row

\startformula \Delta = (-1)^{1+1}a_{11}M_{11} + (-1)^{1 + 2}a_{12}M_{12} + (-1)^{1 + 3}a_{13}M_{13}\stopformula

\startformula =a_{11}M_{11} - a_{12}M_{12} + a_{13}M_{13}\stopformula

\section{Cofactors}
The minor $M_{ij}$ multiplied with $(-1)^{i+j}$ is known as cofactor of the element $a_{ij}$ and is denoted like $A_{ij}$. Thus, we
can say that, $\Delta = a_{11}A_{11} + a_{12}A_{12} + a_{13}A_{13}$

\section{Theorems on Determinants}
\starttheorem
  The value of a determinant is not changed when rows are changed into corresponsing columns.
\stoptheorem

\startproof
Let \startformula \Delta = \startdeterminant\NC a_1 \NC b_1 \NC c_1\NR\NC a_2 \NC b_2 \NC c_2\NR\NC  a_3 \NC b_3 \NC c_3\NR\stopdeterminant\stopformula

Expanding the determinant along first row, \startformula \Delta = a_1(b_2c_3 - b_3c_2) - b_1(a_2c_3 - a_3c_2) + c_1(a_2b_3 - a_3b_2)\stopformula

If $\Delta^{\prime}$ be the value of the determinant when rows of determinant $\Delta$ are changed into corresponding columns then

\startformula \Delta^{\prime} = \startdeterminant\NC a_1\NC a_2\NC a_3\NR\NC b_1\NC b_2 \NC b_3\NR\NC  c_1 \NC c_2 \NC c_3\NR\stopdeterminant\stopformula

\startformula = a_1(b_2c_3 - b_3c_2) - a_2(b_1c_3 - b_3c_1) + a_3(b_1c_2 - b_2c_1)\stopformula

\startformula = a_1(b_2c_3 - b_3c_2) - a_2b_1c_3 + a_2b_3c_1 + a_3b_1c_2 - a_3b_2c_1\stopformula

\startformula = a_1(b_2c_3 - b_3c_2) - b_1(a_2c_3 - a_3c_2) + c_1(a_2b_3 - a_3b_2)\stopformula

Thus, we see that $\Delta = \Delta^{\prime}$.
\stopproof

\starttheorem
  If any two rows or columns of a determinant are interchanged, the sign of determinant is changed, but its value remains the same.
\stoptheorem

\startproof
  Let \startformula \Delta = \startdeterminant\NC a_1 \NC b_1 \NC c_1\NR\NC a_2 \NC b_2 \NC c_2\NR\NC  a_3 \NC b_3 \NC c_3\NR\stopdeterminant,\stopformula

Expanding the determinant along first row, $\Delta = a_1(b_2c_3 - b_3c_2) - b_1(a_2c_3 - a_3c_2) + c_1(a_2b_3 - a_3b_2)$

Now $\Delta^{\prime} = \startdeterminant\NC a_3 \NC b_3 \NC c_3\NR\NC a_2 \NC b_2 \NC c_2\NR\NC a_1 \NC b_1 \NC c_1\NR\stopdeterminant [R_1 \leftrightarrow R_3]$

$= a_3(b_2c_1 - b_1c_2) - b_3(a_2c_1 - a_1c_2) + c_3(a_2b_1 - a_1b_2)$

$= a_3b_2c_1 - a_3b_1c_2 - b_3a_2c_2 + b_3a_1c_2 + c_3a_2b_1 - c_3a_1b_2x$

$= -a_1(b_2c_3 - b_3c_2) + b_1(a_2c_3 - a_3c_2) - c_1(a_2b_3 - a_3b_2)$

$= -\Delta$
\stopproof


\starttheorem
  The value of a determinant is zero if any two rows or columns are identical.
\stoptheorem

\startproof
  Let \startformula \Delta = \startdeterminant\NC  a_1 \NC b_1 \NC c_1\NR\NC a_2 \NC b_2 \NC c_2\NR\NC  a_1 \NC b_1 \NC c_1\NR\stopdeterminant\stopformula

\startformula \Delta = \startdeterminant\NC  a_1 \NC b_1 \NC c_1\NR\NC a_2 \NC b_2 \NC c_2 \NR\NC  a_1 \NC
b_1 \NC c_1\NR\stopdeterminant = - \startdeterminant\NC  a_1 \NC b_1 \NC c_1\NR\NC a_2 \NC b_2 \NC c_2 \NR\NC
a_1 \NC b_1 \NC c_1\NR\stopdeterminant = -\Delta [R_1\leftrightarrow R_3]\stopformula

Thus, $\Delta = -\Delta \Rightarrow 2\Delta = 0 \Rightarrow \Delta = 0$.
\stopproof

\starttheorem
  A common factor of all elements of any row(or of any column) may be taken outside the sign of the determinant. In other owrds, if
  all the elements of the same row(or the same column) are multiplies by a constant, then the determinant becomes multiplied by
  that number.
\stoptheorem

\startproof
  \startformula \Delta = \startdeterminant\NC  a_1 \NC b_1 \NC c_1\NR\NC a_2 \NC b_2 \NC c_2\NR\NC  a_3 \NC b_3 \NC c_3\NR\stopdeterminant\stopformula

Expanding the determinant along first row, $\Delta = a_1(b_2c_3 -
b_3c_2) - b_1(a_2c_3 - a_3c_2) + c_1(a_2b_3 - a_3b_2)$

and \startformula \Delta^{\prime} = \startdeterminant\NC  ma_1 \NC mb_1 \NC mc_1\NR\NC a_2 \NC b_2 \NC c_2
\NR\NC  a_3 \NC b_3 \NC c_3\NR\stopdeterminant\stopformula

$= ma_1(b_2c_3 - b_3c_2) - mb_1(a_2c_3 - a_3c_2) + mc_1(a_2b_3 - a_3b_2)$

$= m\Delta$
\stopproof

\starttheorem
  If every element of some row or column is the the sum of two terms, then the determinant is equal to the sum of two determinants;
  one containing only the first term in place of each term, the other only the second term. The remaining elements of both the
  determinants are the same as in the given determinant.
\stoptheorem

\startproof
  We have to prove that

\startformula \startdeterminant\NC  a_1 + \alpha_1 \NC b_1 \NC c_1\NR\NC a_2 + \alpha_2 \NC b_2 \NC c_2\NR\NC
a_3 + \alpha_3 \NC b_3 \NC c_3\NR\stopdeterminant = \startdeterminant\NC  a_1 \NC b_1 \NC c_1\NR\NC a_2 \NC
b_2 \NC c_2 \NR\NC  a_3 \NC b_3 \NC c_3\NR\stopdeterminant + \startdeterminant\NC \alpha_1 \NC b_1 \NC c_1\NR\NC \alpha_2 \NC
b_2 \NC c_2 \NR\NC \alpha_3 \NC b_3 \NC c_3\NR\stopdeterminant\stopformula

Let \startformula \Delta = \startdeterminant\NC  a_1 + \alpha_1 \NC b_1 \NC c_1\NR\NC a_2 + \alpha_2 \NC
b_2 \NC c_2\NR\NC  a_3 + \alpha_3 \NC b_3 \NC c_3\NR\stopdeterminant\stopformula

Then, \startformula \Delta = (a_1 + \alpha_1)\startdeterminant\NC  b_2 \NC c_2 \NR\NC  b_3 \NC
c_3\NR\stopdeterminant - (a_2 + \alpha_2)\startdeterminant\NC  b_1 \NC c_1\NR\NC b_3 \NC
c_3\NR\stopdeterminant + (a_3 + \alpha_3)\startdeterminant\NC  b_1 \NC c_1\NR\NC b_2 \NC
c_2\NR\stopdeterminant\stopformula

\startformula = a_1\startdeterminant\NC  b_2 \NC c_2 \NR\NC  b_3 \NC c_3\NR\stopdeterminant -
a_2\startdeterminant\NC  b_1 \NC c_1\NR\NC b_3 \NC c_3\NR\stopdeterminant + a_3\startdeterminant\NC  b_1 \NC c_1\NR\NC b_2 \NC
c_2\NR\stopdeterminant + \alpha_1\startdeterminant\NC  b_2 \NC c_2 \NR\NC  b_3 \NC c_3\NR\stopdeterminant -
\alpha_2\startdeterminant\NC  b_1 \NC c_1\NR\NC b_3 \NC c_3\NR\stopdeterminant + \alpha_3\startdeterminant\NC  b_1 \NC c_1\NR\NC b_2 \NC
c_2\NR\stopdeterminant\stopformula

\startformula = \startdeterminant\NC  a_1 \NC b_1 \NC c_1\NR\NC a_2 \NC b_2 \NC c_2 \NR\NC  a_3 \NC b_3 \NC
c_3\NR\stopdeterminant + \startdeterminant\NC \alpha_1 \NC b_1 \NC c_1\NR\NC \alpha_2 \NC
b_2 \NC c_2 \NR\NC  \alpha_3 \NC b_3 \NC c_3\NR\stopdeterminant.\stopformula
\stopproof

\starttheorem
  The value of a determinant does not change when any row or column is multiplied by a number or an expression and is then added to
  or subtracted from any other row or column.
\stoptheorem

\startproof
  We have to prove that

\startformula \startdeterminant\NC  a_1 \NC b_1 \NC c_1\NR\NC a_2 \NC b_2 \NC c_2 \NR\NC  a_3 \NC b_3 \NC
c_3\NR\stopdeterminant = \startdeterminant\NC  a_1 + mb_1 \NC b_1 \NC c_1\NR\NC a_2 + mb_2 \NC b_2 \NC c_2
\NR\NC  a_3 + mb_3 \NC b_3 \NC c_3\NR\stopdeterminant\stopformula

Let \startformula \Delta = \startdeterminant\NC  a_1 + mb_1 \NC b_1 \NC c_1\NR\NC a_2 + mb_2 \NC b_2 \NC c_2
\NR\NC  a_3 + mb_3 \NC b_3 \NC c_3\NR\stopdeterminant\stopformula

then \startformula \Delta = \startdeterminant\NC  a_1 \NC b_1 \NC c_1\NR\NC a_2 \NC b_2 \NC c_2 \NR\NC  a_3 \NC b_3 \NC
c_3\NR\stopdeterminant + \startdeterminant\NC mb_1 \NC b_1 \NC c_1\NR\NC mb_2 \NC b_2 \NC c_2
\NR\NC mb_3 \NC b_3 \NC c_3\NR\stopdeterminant\stopformula

\startformula = \startdeterminant\NC  a_1 \NC b_1 \NC c_1\NR\NC a_2 \NC b_2 \NC c_2 \NR\NC  a_3 \NC b_3 \NC
c_3\NR\stopdeterminant + m \startdeterminant\NC  b_1 \NC b_1 \NC c_1\NR\NC b_2 \NC b_2 \NC c_2 \NR\NC  b_3 \NC b_3 \NC
c_3\NR\stopdeterminant\stopformula

\startformula = \startdeterminant\NC  a_1 \NC b_1 \NC c_1\NR\NC a_2 \NC b_2 \NC c_2 \NR\NC  a_3 \NC b_3 \NC
c_3\NR\stopdeterminant + m.0 = \Delta\stopformula
\stopproof

\section{Reciprocal Determinants}
If \startformula \Delta = \startdeterminant\NC  a_1 \NC a_2 \NC a_3\NR\NC b_1 \NC b_2 \NC b_3\NR\NC c_1 \NC c_2 \NC c_3\NR\stopdeterminant\stopformula then \startformula \startdeterminant\NC  A_1 \NC A_2 \NC
    A_3\NR\NC B_1 \NC B_2 \NC B_3\NR\NC C_1 \NC C_2 \NC C_3\NR\stopdeterminant = \Delta^2\stopformula

where capital letters denote the cofactors of corresponding small letters in $\Delta$ i.e. $A_i =$ cofactor of $a_i, B_i =$
cofactor of $b_i$ and $C_i =$ cofactor of $c_i$ in the determinant $\Delta$. Here, the cofactors are sometimes called
{\it inverse} elements and determinant made from them is called {\it reciprocal} determinant.

We know that,

$a_1A_1 + a_2A_2 + a_3A_3 = \Delta, b_1B_1 + b_2B_2 + b_3C_3 =
\Delta, c_1C_1 + c_2C_2 + c_3C_3 = \Delta, a_1B_1 + a_2B_2 +
a_3B_3 = 0, b_1A_1 + b_2A_2 + b_3A_3 = 0, a_1C_1 + a_2C_2 +
a_3C_3 = 0, c_1A_1 + c_2A_2 + c_3A_3 = 0, b_1C_1 + b_2C_2 +
b_3C_3 = 0, c_1B_1 + c_2B_2 + c_3B_3 = 0$.
Let \startformula \Delta_1 = \startdeterminant\NC  A_1 \NC A_2 \NC A_3\NR\NC B_1 \NC B_2 \NC
B_3\NR\NC C_1 \NC C_2 \NC C_3\NR\stopdeterminant\stopformula
Now, \startformula \Delta\Delta_1 = \startdeterminant\NC  a_1 \NC a_2 \NC a_3\NR\NC b_1 \NC b_2 \NC
b_3\NR\NC c_1 \NC c_2 \NC c_3\NR\stopdeterminant\startdeterminant\NC  A_1 \NC A_2 \NC A_3\NR\NC B_1 \NC B_2 \NC
B_3\NR\NC C_1 \NC C_2 \NC C_3\NR\stopdeterminant\stopformula
\startformula = \startdeterminant\NC  a_1A_1 + a_2A_2 + a_3A_3 \NC a_1B_1 + a_2B_2 + a_3B_3 \NC
a_1C_1 + a_2C_2 + a_3C_3\NR\NC b_1A_1 + b_2A_2 + b_3A_3 \NC b_1B_1 + b_2B_2 + b_3C_3 \NC
b_1C_1 + b_2C_2 + b_3C_3\NR\NC c_1A_1 + c_2A_2 + c_3A_3 \NC c_1B_1 + c_2B_2 + c_3B_3 \NC
c_1C_1 + c_2C_2 + c_3C_3\NR\stopdeterminant\stopformula
\startformula = \startdeterminant\NC \Delta \NC 0 \NC 0\NR\NC 0 \NC \Delta \NC 0\NR\NC 0 \NC 0 \NC\Delta\NR\stopdeterminant\stopformula
\startformula \Delta\Delta_1= \Delta^3\stopformula
\startformula \Delta_1 = \Delta^2\stopformula

Similarly, if $\Delta$ is a determinant of the $n$-th order and $\Delta'$ is the reciprocal determinant, then \startformula \Delta' = \Delta^{n
  - 1}\stopformula which can be proven by induction.

Any minor of $\Delta'$ of order $r$ is equla to the complement of the corresponding minor of $\Delta$ multiplied with $\Delta^{r -
  1}$, provided that $\Delta \neq 0$. The proof of this is straightforward and has been left as an exercise to the reader.

\section{Two Methods of Expansions}
Let \startformula \Delta = \startdeterminant\NC  a_1 \NC b_1 \NC c_1\NR\NC a_2 \NC b_2 \NC c_2\NR\NC a_3 \NC b_3 \NC c_3\NR\stopdeterminant, \text{~and~} D= \startdeterminant\NC
    a_1 \NC b_1 \NC c_1 \NC l\NR\NC a_2 \NC b_2 \NC c_2 \NC m\NR\NC a_3 \NC b_3 \NC c_3 \NC n\NR\NC  l' \NC m' \NC n' \NC r
  \NR\stopdeterminant\stopformula

Let $A_1, B_1, \ldots$ be the cofactors of $a_1, b_1, \ldots$ in $\Delta$.

In the expansion of $D$, the sum of the terms containing $r$ is $r\Delta$: every other term contains one of the three $l, m, m$ and
one of the three $l', m', n'$.

Again, $\startdeterminant\NC  a_1 \NC l\NR\NC l' \NC r\NR\stopdeterminant$ and $\startdeterminant\NC  b_2 \NC c_2\NR\NC b_3 \NC c_3\NR\stopdeterminant$ are complementary
minors of $\Delta$;

hence, cofficients of $ll'$ in $D = -$ cofficient of $a_1r$ in $D = -$ coefficient of $a_1$ in $\Delta = -A_1$

and similarly, coefficient of $mn'$ in $D = -$ coefficient of $c_2r$ in $D = -$ coefficient of $c_2$ in $\Delta = -C_2$

Thus, we can show that \startformula D = r\Delta - [A_1ll' + B_2mm' + C_2nn' + C_2mn' + B_2m'n + A_2nl' + C_1n'l + B_1lm' + A_2l'm].\stopformula

\section{Symmetric Determinants}
A determinant of $n$th order is often wirtten in the form

\startformula \startdeterminant\NC a_{11} \NC a_{12} \NC a_{13} \NC \cdots \NC a_{1n}\NR
  \NC a_{21} \NC a_{22} \NC a_{23} \NC \cdots \NC a_{2n}\NR
  \NC a_{31} \NC a_{32} \NC a_{33} \NC \cdots \NC a_{3n}\NR
  %\multispan{9}{\dotfill}\NR
  \HF\NR
  \NC a_{n1} \NC a_{n2} \NC a_{n3} \NC \cdots \NC a_{nn}
\NR\stopdeterminant = (a_{11}a_{22} \ldots, a_{nn})\stopformula

Denoting any element by $a_{ij}$, the determinant is said to be {\it symmetric} if $a_{ij} = a_{ji}$. If $a_{ij} = -a_{ji}$, the
determinant is {\it skew-symmetric}: it is implied that all the elements in the leading diagonal are zero. For example, if
\startformula \Delta_1 = \startdeterminant\NC  a \NC h \NC g \NC l\NR\NC h \NC b \NC f \NC m\NR\NC g \NC f \NC c \NC n\NR\NC l \NC m \NC n \NC 0\NR\stopdeterminant,
  \Delta_2 = \startdeterminant\NC  0 \NC x \NC y\NR\NC -x \NC 0 \NC y\NR\NC -y \NC -x \NC 0\NR\stopdeterminant\stopformula
the determinant $\Delta_1$ is symmetric and $\Delta_2$ is skew-symmetric. We also say that $\Delta_1$ is bordered by $l, m, n$.

If $A_{ij}, A_{ji}$ are the cofactors of the elements $a_{ij}, a_{ji}$ of a symmetric determinant $\Delta$, then $A_{ij} = A{ji}$.

For $A_{ij}$ is trandformed into $A_{ji}$, by changing rows into columns. Thus, if $\Delta = (a_{11}\;a_{22}\;a_{33})$
\startformula A_{23} = -\startdeterminant\NC  a_{11} \NC a_{12}\NR\NC a_{31} \NC a_{32}\NR\stopdeterminant = -\startdeterminant\NC  a_{11} \NC a_{31}\NR\NC a_{12} \NC
a_{32}\NR\stopdeterminant = -\startdeterminant\NC  a_{11} \NC a_{13}\NR\NC a_{21} \NC a_{23}\NR\stopdeterminant = A_{32}.\stopformula

Similarly, for the skew-symmetric determinants $A_{ij} = (-)1^{n - 1}A_{ji}$, where $n$ is the order of the determinant. Also,
every skew-symmetric determinant of odd order is equal to zero (follows from the definition of skew-symmetric determinants).

\section{System of Linear Equations}

\subsection{Consistent Linear Equations}
A system of linear equations is said to be consistent if it has at least one
solution.

{\bf Example:} (i) System of equations $x + y = 2$ and $2x + 2y = 7$ is inconsistent because it has no solution i.e. no values
of $x$ and $y$ exit which can satisfy the pair of equations. (ii) On the other hand equations $x + y = 2$ and $x - y = 0$ has a
solution $x = 1, y = 1$ which satisfies the pair of equation making it a consistent system of linear equations.

\section{Cramer's Rule}
Cramer's rule is used to solve system of linear equations using determinants. Consider two equations $a_x + b_1y + c_1 = 0$ and
$a_2x + b_2y + c_2 = 0$ where $\frac{a_1}{a_2}\neq \frac{b_1}{b_2}$

Solving this by cross multiplication, we have,

\startformula \frac{x}{b_1c_2 - b_2c_1} = \frac{-y}{a_1c_2 - a_2c_1} = \frac{1}{a_1b_2- a_2b_1}\stopformula

\startformula \frac{x}{\startdeterminant\NC  b_1 \NC c_1\NR\NC b_2 \NC c_2\NR\stopdeterminant} = \frac{-y}{\startdeterminant\NC  a_1\NC c_1\NR\NC a_2 \NC c_2\NR\stopdeterminant} =
\frac{1}{\startdeterminant\NC  a_1\NC b_1\NR\NC a_2 \NC b_2\NR\stopdeterminant}\stopformula

\subsection{System of Linear Equations in Three Variables}
Let the given system of linear equations given in $x, y$ and $z$ be $a_1x + b_1y + c_1z = d_1, a_2x + b_2y + c_2z = d_2$ and $a_3x
+ b_3y + c_3z = d_3$

Let \startformula \Delta = \startdeterminant\NC  a_1\NC b_1\NC c_1\NR\NC a_2\NC b_2\NC c_2\NR\NC a_3\NC b_3\NC c_3\NR\stopdeterminant,
\Delta_1 = \startdeterminant\NC  d_1\NC b_1\NC c_1\NR\NC d_2\NC b_2\NC c_2\NR\NC d_3\NC b_3\NC c_3\NR\stopdeterminant,
\Delta_2 = \startdeterminant\NC  a_1\NC d_1\NC c_1\NR\NC a_2\NC d_2\NC c_2\NR\NC a_3\NC d_3\NC c_3\NR\stopdeterminant,
\Delta_2 = \startdeterminant\NC  a_1\NC b_1\NC d_1\NR\NC a_2\NC b_2\NC d_2\NR\NC a_3\NC b_3\NC d_3\NR\stopdeterminant.\stopformula

Let \startformula \Delta \neq 0\stopformula

\startformula \Delta_1 = \startdeterminant\NC  d_1\NC b_1\NC c_1\NR\NC d_2\NC b_2\NC c_2\NR\NC d_3\NC b_3\NC c_3\NR\stopdeterminant = \startdeterminant\NC  a_1x + b_1y + c_1z\NC b_1 \NC c_1\NR\NC a_2x
    + b_2y + c_2z \NC b_2 \NC c_2 \NR\NC a_3x + b_3y + c_3z \NC b_3 \NC c_3\NR\stopdeterminant
= \startdeterminant\NC  a_1x \NC b_1 \NC c_1\NR\NC a_2x \NC b_2 \NC c_2 \NR\NC a_3x \NC b_3 \NC c_3\NR\stopdeterminant[C_1\rightarrow C_1 - yC_2 -zC_3]\stopformula

\startformula = x\startdeterminant\NC  a_1\NC b_1\NC c_1\NR\NC a_2\NC b_2\NC c_2\NR\NC a_3\NC b_3\NC c_3\NR\stopdeterminant = x\Delta
\Rightarrow x = \frac{\Delta_1}{\Delta}\stopformula

Similalry, \startformula y = \frac{\Delta_2}{\Delta}, z = \frac{\Delta_3}{\Delta}\stopformula

This rule which gives the values of $x, y$ and $z$ is known as Cramer's rule.

\subsection{Nature of Solution of System of Linear Equations}
From previous section we have arrived at the fact that $x\Delta = \Delta_1, y\Delta = \Delta_2, z\Delta = \Delta_3$

{\bf Case I.} When $\Delta \neq 0$

In this case unique values of $x, y, z$ will be obtained and the system of equations will have a unique solution.

{\bf Case II.} When $\Delta = 0$

{\bf Sub Case I.} When at least one of $\Delta_1, \Delta_2, \Delta_3$ is non-zero.

Let $\Delta_1 \neq 0$ then $\Delta_1 = x\Delta$ will not be
satisfied for any value of $x$ because $\Delta = 0$ and hence no
value is possible in this case. Same is the case for $y$ and $z$.

Thus, no solution is feasible and system of equations become inconsistent.

{\bf Sub Case II.} When $\Delta_1 = \Delta_2 = \Delta_3 = 0$

In this case infinite number of solutions are possible.

\subsection{Condition for Consistency of Three Linear Equations in Two Unknonws}
Consider a system of linear equations in $x$ and $y$m $a_1x + b_1y + c_1 = 0, a_2x + b_2y + c_2 = 0$ and $a_3x+ b_3y + c_3 = 0$
will be consistent if the values of $x$ and $y$ obtained from any two equations satisfy the third equations.

Solving first two equations by Cramer's rule, we have

\startformula \frac{x}{\startdeterminant\NC  b_1 \NC c_1\NR\NC b_2 \NC c_2\NR\stopdeterminant} =
\frac{-y}{\startdeterminant\NC  a_1 \NC c_1\NR\NC a_2 \NC c_2\NR\stopdeterminant} =
\frac{1}{\startdeterminant\NC  a_1 \NC b_1\NR\NC a_2 \NC b_2\NR\stopdeterminant} = k(\text{say})\stopformula

Substituting these in third equation we get,

\startformula k[a_3(b_1c_2 - b_2c_1) - b_3(a_1c_2 - a_2c_1) + c_3(a_1b_2 - a_2b_1)] = 0\stopformula
\startformula a_3(b_1c_2 - b_2c_1) - b_3(a_1c_2 - a_2c_1) + c_3(a_1b_2 - a_2b_1) = 0\stopformula
\startformula \startdeterminant\NC  a_1\NC b_1\NC c_1\NR\NC a_2\NC b_2\NC c_2\NR\NC a_3\NC b_3\NC c_3\NR\stopdeterminant = 0\stopformula

This is the required condition for consistency of three linear equations in two variables. If such a system of equations is
consistent then number of solution is one i.e. a unique solution exists.

\subsection{System of Homogeneous Linear Equations}
A system of linear equations is said to be homogeneous if the sum of powers of the variables in each term is one. Let the three
homogeneous equations in three unknowns $x, y, z$ be $a_1x + b_1y + c_1z = 0, a_2x + b_2y + c_2z = 0$ and $a_3x + b_3y + c_3z = 0$

Clearly, $x = 0, y = 0, z= 0$ is a solution of above system of equations. This solution is called trivial solution and any other
solution is called non-triivial solution. Let the above system of equations has a non-trivial solution.

Let \startformula \Delta = \startdeterminant\NC  a_1\NC b_1\NC c_1\NR\NC a_2\NC b_2\NC c_2\NR\NC a_3\NC b_3\NC c_3\NR\stopdeterminant\stopformula

From first two we have

\startformula \frac{x}{\startdeterminant\NC  b_1 \NC c_1\NR\NC b_2 \NC c_2\NR\stopdeterminant} =
\frac{-y}{\startdeterminant\NC  a_1 \NC c_1\NR\NC a_2 \NC c_2\NR\stopdeterminant} =
\frac{z}{\startdeterminant\NC  a_1 \NC b_1\NR\NC a_2 \NC b_2\NR\stopdeterminant} = k(\text{say})\stopformula

Substituting these in third equation we get

\startformula k[a_3(b_1c_2 - b_2c_1) - b_3(a_1c_2 - a_2c_1) + c_3(a_1b_2 - a_2b_1)] =0\stopformula
\startformula a_3(b_1c_2 - b_2c_1) - b_3(a_1c_2 - a_2c_1) + c_3(a_1b_2 - a_2b_1) = 0\stopformula
\startformula \startdeterminant\NC  a_1\NC b_1\NC c_1\NR\NC a_2\NC b_2\NC c_2\NR\NC a_3\NC b_3\NC c_3\NR\stopdeterminant = 0\stopformula

This is the condition for system of equation to have non-trivial solutions.

\section{Use of Determinants in Coordinate Geometry}

\subsection{Are of a Triangle}
The area of a triangle whose vertices are $(x_1, y_1), (x_2, y_2)$ and $(x_3, y_3)$ is

\startformula \Delta = \frac{1}{2}\startdeterminant\NC  x_1 \NC y_1 \NC 1\NR\NC x_2 \NC y_2 \NC 1\NR\NC x_3 \NC y_3 \NC1\NR\stopdeterminant\stopformula

\subsection{Condition of Concurrency of Three Lines}
Three lines are said to be concurrent if they pass through a common point i.e. they meet at a point.

Let $a_1x + b_1y + c_1 = 0$ $a_2x + b_2y + c_2 = 0$ and $a_3x+ b_3y + c_3 = 0$ be three lines.

These lines will be concurrent if
\startformula \startdeterminant\NC  a_1\NC b_1\NC c_1\NR\NC a_2\NC b_2\NC c_2\NR\NC a_3\NC b_3\NC c_3\NR\stopdeterminant = 0\stopformula

\subsection{Condition for General Equation in Second Degree to Represent a Pair of Straight Lines}
The general second degree equation $ax^2 + 2hxy + by^2 + 2gx + 2fy + c = 0$ represent a pair of straight lines if

\startformula \startdeterminant\NC  a \NC h \NC g\NR\NC h \NC b \NC f\NR\NC g \NC f \NC c\NR\stopdeterminant = 0\stopformula

\section{Product of Two Determinants}
Let \startformula \Delta_1 = \startdeterminant\NC  a_1 \NC a_2 \NC a_3\NR\NC b_1 \NC b_2 \NC b_3\NR\NC c_1 \NC
c_2 \NC c_3\NR\stopdeterminant, \Delta_2 = \startdeterminant\NC  x_1 \NC x_2 \NC
x_3\NR\NC y_1 \NC y_2 \NC y_3\NR\NC z_1 \NC z_2 \NC z_3\NR\stopdeterminant\stopformula then
$\Delta_1\Delta_2$ is defined as

\startformula \Delta_1\Delta_2 = \startdeterminant\NC  a_1x_1 + a_2x_2 + a_3x_3 \NC a_1y_1 +
a_2y_2 + a_3y_3 \NC a_1z_1 + a_2z_2 + a_3z_3\NR\NC b_1x_1 + b_2x_2 + b_3x_3 \NC b_1y_1 +
b_2y_2 + b_3y_3 \NC b_1z_1 + b_2z_2 + b_3z_3 \NR\NC  c_1x_1 + c_2x_2 + c_3x_3 \NC
c_1y_1 + c_2y_2 + c_3y_3 \NC c_1z_1 + c_2z_2 + c_3z_3\NR\stopdeterminant\stopformula

\section{Differential Coefficient of Determinant}
Let \startformula y = \startdeterminant\NC  f_1(x) \NC f_2(x) \NC f_3(x)]\NR\NC g_1(x) \NC g_2(x) \NC
g_3(x)\NR\NC h_1(x) \NC h_2(x) \NC h_3(x)\NR\stopdeterminant,\stopformula where $f_i(x), g_i(x),
h_i(x), i= 1, 2, 3$ are differentiable functions of $x.$

Now, $y = f_1(x)[g_2(x)h_3(x) - g_3(x)h_2(x)] - f_2(x)[g_1(x)h_3(x) -
g_3(x)h_1(x)] +$ $f_3(x)[g_1(x)h_2(x) - g_2(x)h_1(x)]$

$\therefore \frac{dy}{dx} = f_1^{\prime}(x)[g_2(x)h_3(x) -
g_3(x)h_2(x)] + f_1(x)[g_2^{\prime}(x)h_3(x) - g_3^{\prime}(x)h_2(x) +
g_2(x)h_3^{\prime}(x) - g_3(x)h_2^{\prime}(x)] +
-f_2^{\prime}(x)[g_1(x)h_3(x) - g_3(x)h_1(x)] +
-f_2(x)[g_1^{\prime}(x)h_3(x) - g_1(x)h_3^{\prime}(x) +
g_1(x)h_3^{\prime}(x) - g_3(x)h_3^{\prime}(x)] +
f_3^{\prime}(x)[g_1(x)h_2(x) - g_2(x)h_1(x)] +
f_3(x)[g_1^{\prime}(x)h_2(x) - g_2^{\prime}(x)h_1(x)x +
g_1(x)h_2^{\prime}(x) - g_2(x)h_1^{\prime}(x)]$

\startformula = \startdeterminant\NC  f_1^{\prime}(x) \NC f_2^{\prime}(x) \NC
f_1^{\prime}(x)\NR\NC g_1(x) \NC g_2(x) \NC g_3(x)\NR\NC h_1(x) \NC h_2(x) \NC
h_3(x)\NR\stopdeterminant + \startdeterminant\NC  f_1(x) \NC f_2(x) \NC f_3(x)\NR\NC g_1^{\prime}(x)
\NC g_2^{\prime}(x) \NC g_3^{\prime}(x) \NR\NC h_1(x) \NC h_2(x) \NC h_3(x) \NR\stopdeterminant +
\startdeterminant\NC  f_(x) \NC f_2(x) \NC f_3(x)\NR\NC g_1(x) \NC g_2(x) \NC
g_3(x)\NR\NC h_1^{\prime}(x) \NC h_2^{\prime}(x) \NC h_3^{\prime}(x)\NR\stopdeterminant\stopformula

\section{Problems}
\startitemize[n, 1*broad]
\item Evaluate $\startdeterminant\NC  4 \NC 9 \NC 7\NR\NC 3 \NC 5 \NC 7\NR\NC 5 \NC 4 \NC 5\NR\stopdeterminant$.
\item Show that $\startdeterminant\NC  1 \NC a\NC a^2\NR\NC 1 \NC b \NC b^2\NR\NC 1 \NC c \NC c^2\NR\stopdeterminant = (a - b)(b - c)(c - a)$.
\item Evaluate $\startdeterminant\NC  1 \NC 2\NC 4\NR\NC 1 \NC 3 \NC 9\NR\NC 1 \NC 4 \NC 16\NR\stopdeterminant$ making use of relations between $2$nd and $3$rd
  column.
\item Evaluate $\startdeterminant\NC  4 \NC 9 \NC 2\NR\NC 3 \NC 5 \NC 7\NR\NC 8 \NC 1 \NC 6\NR\stopdeterminant$.
\item Evaluate $\startdeterminant\NC  18 \NC 1 \NC 17\NR\NC 22 \NC 3 \NC 19\NR\NC 26 \NC 5 \NC 21\NR\stopdeterminant$.
\item Evaluate $\startdeterminant\NC  4 \NC 9 \NC 7\NR\NC 3 \NC 5 \NC 7\NR\NC 5 \NC 4 \NC 5\NR\stopdeterminant$.
\item Evaluate $\startdeterminant\NC  1^2 \NC 2^2 \NC 3^2\NR\NC 2^2 \NC 3^2 \NC 4^2\NR\NC 3^2 \NC 4^2 \NC 5^2\NR\stopdeterminant$.
\item Let $a, b, c$ be positive and unequal. Show that the value of the determinant $\startdeterminant\NC  a \NC b \NC c\NR\NC b \NC c \NC a\NR\NC c \NC a \NC
  b\NR\stopdeterminant$ is negative.
\item Evaluate $\startdeterminant\NC  b + c\NC a + b \NC a\NR\NC c + a \NC b + c \NC b\NR\NC a + b \NC c + a \NC c\NR\stopdeterminant$.
\item Evaluate $\startdeterminant\NC  1 + a_1 \NC a_2 \NC a_3\NC a_1 \NC 1+ a_2 \NC a_3\NR\NC a_1 \NC a_2 \NC 1 + a_3\NR\stopdeterminant$.
\item Show that $\startdeterminant\NC  a + b + 2c \NC a \NC b\NR\NC c \NC b + c + 2a \NC b\NR\NC c \NC a \NC c + a + 2b\NR\stopdeterminant = 2(a + b + c)^3$.
\item Show that $\startdeterminant\NC  a - b + c \NC a + b - c \NC a - b - c\NR\NC  b - c + a \NC b + c -a \NC b - c - a\NR\NC  c - a + b \NC c + a - b \NC c -
  a -b\NR\stopdeterminant = 4(a^3 + b^3 + c^3 - 3abc)$.
\item Prove that $\startdeterminant\NC  a - b - c \NC 2a \NC 2a\NR\NC 2b \NC b - c - a\NR\NC 2c \NC 2c \NC c - a - b\NR\stopdeterminant = (a + b + c)^3$.
\item Prove that $\startdeterminant\NC  x \NC y \NC z\NR\NC x^2 \NC y^2 \NC z^2\NR\NC yz \NC zx \NC xy\NR\stopdeterminant = \startdeterminant\NC  1 \NC 1 \NC 1\NR\NC x^2 \NC y^2 \NC
    z^2\NR\NC x^3 \NC y^3 \NC z^3\NR\stopdeterminant = (x - y)(y - z)(z - x)(xy + yz + zx)$.
\item Prove that $\startdeterminant\NC  a^1 + 1 \NC ab \NC ac\NR\NC ab \NC b^2 + 1 \NC bc\NR\NC ac \NC bc \NC c^2 + 1\NR\stopdeterminant = 1 + a^2 + b^2 + c^2$.
\item Prove that $\startdeterminant\NC 1 + a_1 \NC 1 \NC 1\NR\NC 1 \NC 1 + a_2 \NC 1\NR\NC 1 \NC 1 \NC 1 + a_3\NR\stopdeterminant = a_1a_2a_3\left(1 + \frac{1}{a_1}
  + \frac{1}{a_2} + \frac{1}{a_3}\right)$.
\item If $x,y,z$ are all different and if $\startdeterminant\NC  x \NC x^2 \NC 1 + x^3\NR\NC y \NC y^2 \NC 1 + y^3\NR\NC z \NC z^2 \NC 1 + z^3\NR\stopdeterminant =
  0$, prove that $xyz = -1$.
\item Evaluate $\startdeterminant\NC  b + c \NC a \NC a\NR\NC b \NC c + a \NC b\NR\NC c \NC c \NC a + b\NR\stopdeterminant$.
\item Show that $\startdeterminant\NC (b + c)^2 \NC a^2 \NC a^2\NR\NC b^2 \NC (c + a)^2 \NC b^2\NR\NC c^2 \NC c^2 \NC (a + b)^2\NR\stopdeterminant = 2abc(a + b +
  c)^3$.
\item Solve the equation $\startdeterminant\NC 15 -x \NC 1 \NC 10\NR\NC 11 - 3x \NC 1\NC 16\NR\NC 7 - x \NC 1 \NC 13\NR\stopdeterminant = 0$.
\item If $a + b + c = 0$, solve the equation $\startdeterminant\NC  a - x \NC c \NC b\NR\NC c \NC b - x \NC a\NR\NC b \NC a \NC c - x\NR\stopdeterminant = 0$.
\item If $D_1 = \startdeterminant\NC  a \NC b \NC c\NR\NC d \NC e \NC f\NR\NC g \NC h \NC k\NR\stopdeterminant, D_2 = \startdeterminant\NC  a \NC g \NC x\NR\NC b \NC h \NC y\NR\NC c \NC k \NC
    z\NR\stopdeterminant$ and $d=tx, e = hy, f = tz$, prove without expanding that $D_1 = -tD_2$
\item Show without expanding thet $\startdeterminant\NC  a \NC bc \NC abc \NR\NC b \NC ca \NC abc\NR\NC c \NC ab \NC abc\NR\stopdeterminant = \startdeterminant\NC  a \NC
    a^2 \NC a^3\NR\NC b \NC b^2 \NC b^3\NR\NC c \NC c^2 \NC c^3\NR\stopdeterminant$.
\item If $a, b, c$ are positive and are the $p$th, $q$th, $r$th terms of a G.P., respectively, then show without expanding thet
  $\startdeterminant\NC \log a \NC p \NC 1\NR\NC \log b \NC q \NC 1\NR\NC \log c \NC r \NC 1\NR\stopdeterminant = 0$.
\item Evaluate $\startdeterminant\NC 1 \NC 1 \NC 1\NR\NC  1 \NC 1 + x \NC 1\NR\NC  1 \NC 1 \NC 1 +y\NR\stopdeterminant$.
\item Evaluate $\startdeterminant\NC 1 \NC 1 \NC 1\NR\NC a \NC b \NC c\NR\NC a^3 \NC b^3 \NC c^3\NR\stopdeterminant$.
\item Evaluate $\startdeterminant\NC 1 \NC b + c \NC b^2 + c^2\NR\NC 1 \NC c + a \NC c^2 + a^2\NR\NC  1 \NC a + b \NC a^2 + b^2\NR\stopdeterminant$.
\item Evaluate $\startdeterminant\NC 1 \NC a \NC a^2 - bc\NR\NC  1 \NC b \NC b^2 - ac\NR\NC  1 \NC c \NC c^2 - ab\NR\stopdeterminant$.
\item Evaluate $\startdeterminant\NC 1 \NC bc \NC bc(b + c)\NR\NC  1 \NC ca \NC ca(c + a)\NR\NC  1\NC ab \NC ab(a + b)\NR\stopdeterminant$.
\item Prove that $\startdeterminant\NC 1 \NC a \NC b + c\NR\NC 1 \NC b \NC c + a\NR\NC 1 \NC c \NC c + a\NR\stopdeterminant = 0$.
\item If $a, b, c$ are the $p$th, $q$th, $r$th terms respectively of an H.P., show that $\startdeterminant\NC  bc \NC p \NC 1\NR\NC  ca \NC q \NC
  1\NR\NC  ab \NC r \NC 1\NR\stopdeterminant = 0$.
\item If $\startdeterminant\NC  x^2 + 3x \NC x - 1 \NC x + 3\NR\NC  x + 1 \NC 1 - 2x \NC x - 4\NR\NC  x - 2 \NC x + 4 \NC 3x\NR\stopdeterminant = px^4 + qx^3 + rx^2 +
  sx + t$ be an indentity in $x$, where $p, q, r, s$ and $t$ are constants, find the value of $t$.
\item Prove that $\startdeterminant\NC  a \NC b \NC c\NR\NC a^2 \NC b^2 \NC c^2\NR\NC a^3 \NC b^3 \NC c^3\NR\stopdeterminant = abc(a - b)(b - c)(c - a)$.
\item If $a, b, c$ are in A.P., show that $\startdeterminant\NC  x + 1 \NC x + 2 \NC x + a\NR\NC  x + 2 \NC x + 3 \NC x + b \NR\NC  x + 3 \NC x + 4 \NC x +
  c\NR\stopdeterminant = 0$.
\item If $\omega$ is a complex cube root of unity, prove that $\startdeterminant\NC  1 \NC \omega \NC \omega^2\NR\NC \omega \NC \omega^2 \NC 1\NR\NC \omega
  \NC 1 \NC \omega^2\NR\stopdeterminant = 0$
\item Evaluate $\startdeterminant\NC  k \NC k \NC k\NR\NC 1 \NC 2 \NC 3\NR\NC  1\NC 3 \NC 6\NR\stopdeterminant$.
\item Evaluate $\startdeterminant\NC  a^2 + x \NC b^2 \NC c^2\NR\NC a^2 \NC b^2 + x \NC c^2\NR\NC a^2 \NC b^2 \NC c^2 + x\NR\stopdeterminant$.
\item Evaluate $\startdeterminant\NC  a \NC b + c \NC a^2\NR\NC b \NC c + a \NC b^2\NR\NC c \NC a + b \NC c^2\NR\stopdeterminant$.
\item Evaluate $\startdeterminant\NC  b + c \NC a - b \NC a\NR\NC c + a \NC b - c \NC b\NR\NC a + b \NC c - a \NC c\NR\stopdeterminant$.
\item Show that $\startdeterminant\NC  a + b \NC b + c \NC c + a\NR\NC b + c \NC c + a \NC a + b\NR\NC c + a \NC a + b \NC b + c\NR\stopdeterminant = -2(a^3 + b^3 +
  c^3 - 3abc)$.
\item Show that $\startdeterminant\NC  x + a \NC x + b \NC x + c\NR\NC y + a \NC y + b \NC y + c\NR\NC z + a\NC z + b \NC z + c\NR\stopdeterminant = 0$.
\item Show that $\startdeterminant\NC  0 \NC p - q \NC p - r\NR\NC  q - p \NC 0 \NC q - r\NR\NC  r - p \NC r - q \NC 0\NR\stopdeterminant = 0$.
\item Show that $\startdeterminant\NC  a \NC a + b \NC a + 2b\NR\NC a + 2b \NC a \NC a + b\NR\NC a + b \NC a + 2b \NC a\NR\stopdeterminant = 9b^2(a + b)$
\item Show that $\startdeterminant\NC  a \NC b - c \NC c + b\NR\NC  a + c \NC b \NC c - a\NR\NC  a - b \NC b + a \NC c\NR\stopdeterminant$ and $(a + b + c)$ have the
  same sign.
\item Evaluate $\startdeterminant\NC  b^2 + c^2 \NC ab \NC ac\NR\NC ab \NC c^2 + a^2 \NC bc\NR\NC ca \NC cb \NC a^2 + b^2\NR\stopdeterminant$.
\item Show that $\startdeterminant\NC  (b + c)^2 \NC c^2 \NC b^2\NR\NC c^2 \NC (c + a)^2 \NC a^2\NR\NC b^2 \NC a^2 \NC (a + b)^2\NR\stopdeterminant = 2(ab + bc +
  ca)^3$.
\item Show that $\startdeterminant\NC  (a + b)^2 \NC ca \NC bc\NR\NC ca \NC (b + c)^2 \NC ab\NR\NC bc \NC ab \NC (c + a)^2\NR\stopdeterminant = 2abc(a + b + c)^3$.
\item Show that $\startdeterminant\NC  \frac{a^2 + b^2}{c} \NC c \NC c\NR\NC a \NC \frac{b^2 + c^2}{a} \NC a\NR\NC b \NC b \NC \frac{c^2 + a^2}{b}\NR\stopdeterminant
  = 4abc$.
\stopitemize
Solve the following equations:

\startitemize[n, continue, 1*broad]
\item $\startdeterminant\NC  a \NC a \NC x \NR\NC  a \NC a \NC a \NR\NC  b \NC x \NC b\NR\stopdeterminant = 0$.
\item $\startdeterminant\NC  x \NC 2 \NC 3\NR\NC 6 \NC x + 4 \NC 4\NR\NC 7 \NC 8 \NC x + 8\NR\stopdeterminant = 0$.
\item $\startdeterminant\NC  x \NC 2 \NC 3\NR\NC  4 \NC x \NC 1\NR\NC  x \NC 2 \NC 5\NR\stopdeterminant = 0$.
\item $\startdeterminant\NC  x + a \NC b \NC c\NR\NC  a \NC x + b \NC c\NR\NC  a \NC b \NC x + c\NR\stopdeterminant = 0$.
\item $\startdeterminant\NC  3 + x \NC 5 \NC 2 \NR\NC  1 \NC 7 + x \NC 6 \NR\NC  2 \NC 5 \NC 3 + x\NR\stopdeterminant = 0$.
\stopitemize

Show without expanding at any stage that:

\startitemize[n, continue, 1*broad]
\item $\startdeterminant\NC  a + b \NC b + c \NC c + a\NR\NC b + c \NC c + a \NC a + b\NR\NC b + c \NC c + a \NC a + b\NR\stopdeterminant = 2\startdeterminant\NC  a \NC b \NC
  c\NR\NC b \NC c \NC a\NR\NC  c \NC a \NC b\NR\stopdeterminant$.
\item $\startdeterminant\NC  b + c \NC c + a \NC a + b\NR\NC q + r \NC r + p \NC p + q \NR\NC  y + z \NC  z + x \NC x + y\NR\stopdeterminant = 2\startdeterminant\NC  a \NC b
  \NC c\NR\NC p \NC q \NC r\NR\NC x \NC y \NC z\NR\stopdeterminant$.
\item $\startdeterminant\NC  1 \NC \cos\alpha - \sin\alpha \NC \cos\alpha + \sin\alpha\NR\NC  1 \NC \cos\beta - \sin\beta \NC \cos\beta +
  \sin\beta\NR\NC  1 \NC \cos\gamma - \sin\gamma \NC \cos\gamma + \sin\gamma\NR\stopdeterminant = 2\startdeterminant\NC  1 \NC \cos\alpha \NC \sin\alpha\NR\NC 1
  \NC \cos\beta \NC \sin\beta\NR\NC 1 \NC \cos\gamma \NC \sin\gamma\NR\stopdeterminant$.
\item $\startdeterminant\NC  (a - 1)^2 \NC a^2 + 1 \NC a\NR\NC (b - 1)^2 \NC b^2 + 1 \NC b\NR\NC (c - 1)^2 \NC c^2 + 1 \NC c\NR\stopdeterminant = 0$
\item $\startdeterminant\NC  0 \NC c \NC b\NR\NC  -c \NC 0 \NC a\NR\NC  - b \NC -a \NC 0\NR\stopdeterminant = 0$.
\item $\startdeterminant\NC  1 \NC a \NC bc\NR\NC  1 \NC b \NC ca\NR\NC  1 \NC c \NC ab\NR\stopdeterminant = \startdeterminant\NC  1 \NC a \NC a^2\NR\NC 1 \NC b \NC b^2\NR\NC  1 \NC c \NC
    c^2\NR\stopdeterminant$.
\item $\startdeterminant\NC  a \NC b \NC c\NR\NC x \NC y \NC z\NR\NC yz \NC zx \NC xy\NR\stopdeterminant = \startdeterminant\NC  ax \NC by \NC cz\NR\NC  x^2 \NC y^2 \NC z^2\NR\NC  1 \NC 1 \NC
  1\NR\stopdeterminant$.
\item $\startdeterminant\NC  a \NC b \NC c\NR\NC  x \NC y \NC z\NR\NC  p \NC q \NC r\NR\stopdeterminant = \startdeterminant\NC  y \NC b \NC q\NR\NC x \NC a \NC p\NR\NC  z \NC c \NC r\NR\stopdeterminant
    = \startdeterminant\NC  x \NC y \NC z\NR\NC  p \NC q \NC r\NR\NC  a \NC b \NC c\NR\stopdeterminant$.
\item find the value of the following determinant $\startdeterminant\NC  m! \NC (m + 1)! \NC (m + 2)!\NR\NC (m + 1)! \NC (m + 2)! \NC (m + 3)!\NR\NC (m +
  2)! \NC (m + 3)! \NC (m + 4)!\NR\stopdeterminant$.
\item Solve the following system of equations using Cramer'r rule: $x + y = 4,\;2x - 3y = 9$.
\item Solve the following system of equations using Cramer'r rule: $2x - y + 3z = 0,\;x + y + z = 6,\;x - y + z = 2$.
\item Determine the nature of solution for the equations: $2x + 3y = 6,\;4x + 6y = 10$.
\item Show that the following system of equations is consistent $x + y - z = 1,\;2x + 3x + z = 4,\;4x + 3y + z = 16$.
\item Determine the nature of solution for the equations: $x + y = 2,\;2x + 2y = 4$.
\item Determine whether the following system of equations is consistent: $2x + y = 13,\;6x + 3y = 18,\;x - y = -3$.
\item Show that the system of following euqations has non-trivial solutions: $x + y - 6z = 0,\;3x - y - 2x = 0,\;x - y + 2x = 0$.
\item For what value of $k$ the following system of equations possess non-trivial solution. Also, find all the solutions of the
  system for that value of $k,\;x + y - kz = 0,\;3x - y - 2x = 0,\;x - y + 2x = 0$.
\stopitemize

Solve the following equations by Cramer's rule:

\startitemize[n, continue, 1*broad]
\item $x - 2y = 0;\ 7x + 6y = 40$.
\item $x + y + z = 9;\ 3x + 2y - 3z = 0;\ z - x = 2$.
\item $x - y + z = 0;\ 2x + 3y - 5z = -1;\ 3x - 4y + 2z = -1$.
\item $2x + 3y - 3z = 0;\ 5x - 2y + 2z = 19;\ x + 7y - 5z = 5$.
\item $x + y + z = 1;\ ax + by + cz = k;\ a^2x + b^2y + c^2z = k^2$ where $a\neq b\neq c$.
\item $3x + 2y - 2z = 1;\ -x + y - 4z = 1;\ 2x - 3y + 4z = 8$.
\stopitemize

Determine whether the following system of equations have no solution, unique solution or infinite number of solution:

\startitemize[n, continue, 1*broad]
\item $3x + 9y = 5;\ 9x + 27y = 10$.
\item $5x - 3y = 3;\ x + y = 7$.
\item $x + 2y = 5;\ 3x + 6y = 15$.
\item $2x + 3y + z = 5;\ 3x + y + 5z = 7;\ x + 4y - 2z = 3$.
\item $x + y - z = -2;\ 6x + 4y + 6z = 26;\ 2x + 7y + 4z = 31$.
\item $x + 4y = 9;\ 2x + 8y = 18;\ y - 2x = 0$.
\stopitemize

\startitemize[n, continue, 1*broad]
\item Find the value of $k$ such that following system of equations possess a non-trivial solution over the set of rationals
  $Q$. For that value of $k$ find all the solutions of the system: $x + ky _ 3z = 0;\ x + ky - 2z = 0;\ 2x + 3y - 4z = 0$.
\item If $a, b, c$ are different, show that the following system of equations has non-trivial solutions only when $a + b + c =
  0,\ ax + by + cz = 0;\ bx + cy + az = 0;\ cz + ay + bz = 0$.
\item  what value of $\lambda$ the following system of equations has non-trivial solutions: $3x - y + 4z = 0;\ x _ 2y - 3z = 0;\ 6x
  + 5y - \lambda z = 0$.
\item For a positive integer $n$, if $D = \startdeterminant\NC  n! \NC (n + 1)! \NC (n + 2)!\NR\NC (n + 1)! \NC (n + 2)1 \NC (n + 3)! \NR\NC (n + 2)! \NC (n +
  3)! \NC (n + 4)!\NR\stopdeterminant$, then show that $\frac{D}{(n!)^3} - 4$ is divisible by $n$.
\item Let the three digit numbers $A28, 3B9, 62C,$ where $A, B, C$ are integers between $0$ and $9$, be divisible by a fixed
  integer $k$, show that the determinant $\startdeterminant\NC  A \NC 2 \NC 6\NR\NC  8 \NC 9 \NC C \NR\NC  2 \NC B\NC 2\NR\stopdeterminant$ is divisible by $k$.
\item Evaluate $\startdeterminant\NC {}^xC_1 \NC {}^xC_2 \NC {}^xC_3\NR\NC {}^yC_1 \NC {}^yC_2 \NC {}^yC_3\NR\NC {}^zC_1 \NC {}^zC_2 \NC {}^zC_3\NR\stopdeterminant$
\item If $a\neq p, b\neq q, c\neq r$ and $\startdeterminant\NC  p \NC b \NC c\NR\NC a \NC q \NC c\NR\NC a \NC b \NC r\NR\stopdeterminant = 0$, then find the values of
  $\frac{p}{p - a} + \frac{q}{q - b} + \frac{r}{r - c}$.
\item Show that $\startdeterminant\NC (x - a)^2 \NC b^2 \NC c^2 \NR\NC a^2 \NC (x - b)^2 \NC c^2\NR\NC a^2 \NC b^2 \NC (x - c)^2\NR\stopdeterminant = x^2(x - 2a)(x -
  2b)(x - 2c)$

  $\left(x + \frac{a^2}{x - 2a} + \frac{b^2}{x - 2b} + \frac{c^2}{x - 2c}\right)$.
\item If $a > 0, d > 0$, find the value of the determinant
  \startformula \startdeterminant\NC \frac{1}{a} \NC \frac{1}{a(a + d)} \NC \frac{1}{(a + d)(a +
    2d)}\NR\NC \frac{1}{a + d} \NC \frac{1}{(a + d)(a + 2d)} \NC \frac{1}{(a + 2d)(a + 3d)}\NR\NC \frac{1}{a + 2d} \NC \frac{1}{(a + 2d)(a + 3d)}
  \NC \frac{1}{(a + 3d)(a + 4d)}\NR\stopdeterminant.\stopformula
\item Show that $\startdeterminant\NC \frac{1}{a + x} \NC \frac{1}{a + y} \NC \frac{1}{a + z}\NR\NC \frac{1}{a + y} \NC \frac{1}{b + y} \NC
  \frac{1}{b + z}\NR\NC \frac{1}{c + x} \NC \frac{1}{c + y} \NC \frac{1}{c + z}\NR\stopdeterminant = $ \startformula \frac{(a - b)(b - c)(c - a)(x - y)(y -
  z)(z - x)}{(a + x)(b + x)(c + x)(b + x)(b + y)(b + z)(c + x)(c + y)(c + z)}.\stopformula
\item If $2s = a + b + c$, show that \startformula \startdeterminant\NC  a^2 \NC (s - a)^2 \NC (s - a)^2\NR\NC (s - b)^2 \NC s^2 \NC (s - b)^2\NR\NC (s - c)^2 \NC (s -
  c)^2 \NC s^2\NR\stopdeterminant = 2s^3(s - a)(s - b)(s - c).\stopformula
\item Show that \startformula \startdeterminant\NC  ax - by - cz \NC ay + bx \NC cx + az\NR\NC ay + bx \NC by - cz - ax \NC bz + cy\NR\NC cx + az \NC bz + cy \NC cz - ax -
  by\NR\stopdeterminant = (x^2 + y^2 + z^2)(a^2 + b^2 + c^2)(ax + by + cz).\stopformula
\item Find the value of $\theta$ between $0$ and $\pi/2$ and satisfying the equation: \startformula \startdeterminant\NC 1 + \cos^2\theta \NC
  \sin^2\theta \NC 4\sin\theta\NR\NC \cos^2\theta \NC 1 + \sin^2\theta \NC 4\sin\theta\NR\NC \cos^2\theta \NC \sin^2\theta \NC 1 +
  4\sin\theta\NR\stopdeterminant = 0.\stopformula
\item If $a^2 + b^2 + c^2 = 1$, then prove that \startformula \startdeterminant\NC  a^2 + (b^2 + c^2)\cos\phi \NC ab(1 - \cos\phi) \NC ac(1 -
  \cos\phi)\NR\NC ab(1 - \cos\phi) \NC b^2 + (c^2 + a^2)\cos\phi \NC bc(1 - \cos\phi)\NR\NC ca(1 - \cos\phi) \NC bc(1 - \cos\phi) \NC c^2 + (a^2 +
  b^2)\cos\phi\NR\stopdeterminant = \cos^2\phi.\stopformula
\item If none of the $a,b,c$ is zero, show that $\startdeterminant\NC -bc \NC b^2 + ac \NC c^2 + bc\NR\NC a^2 + ac \NC -ac \NC c^2 + ac\NR\NC a^2 + ab \NC
  b^2 + ab \NC -ab\NR\stopdeterminant = (ab + bc + ca)^3$.
\item If $u, v$ are functions of $x$, and $y = \frac{u}{v}$, show that $v^2\frac{d^2x}{dy^2} = \startdeterminant\NC  u \NC v \NC 0\NR\NC u' \NC b' \NC
  v\NR\NC u'' \NC v'' \NC 2v'\NR\stopdeterminant$ where primes denote derivatives.
\item If $a\neq 0$ and $a\neq 1$, show that $\startdeterminant\NC  x + 1 \NC x \NC x\NR\NC x \NC x + a \NC x\NR\NC  x \NC x\NC x + a^2\NR\stopdeterminant = a^3\left[1
    + \frac{x(a^3 - 1)}{a^2(a - 1)}\right]$.
\item If $p + q + r = 0$, prove that $\startdeterminant\NC  pa \NC qb \NC rc \NR\NC  qc \NC ra \NC pb\NR\NC  rb \NC pc \NC qa\NR\stopdeterminant =
  pqr\startdeterminant\NC  a \NC b \NC c\NR\NC  c \NC a \NC b\NR\NC b \NC c \NC a\NR\stopdeterminant$.
\item Show without expanding that $\startdeterminant\NC 1 \NC a \NC a^2\NR\NC 1 \NC b \NC b^2\NR\NC 1 \NC c \NC c^2\NR\stopdeterminant = \startdeterminant\NC 1 \NC bc \NC b +
  c\NR\NC  1 \NC ca \NC c + a\NR\NC  1 \NC ab \NC a + b\NR\stopdeterminant$.
\item Show without expanding that $\startdeterminant\NC  x^2 + x \NC x + 1 \NC x - 2\NR\NC  2x^2 + 3x - 1 \NC 3x \NC 3x - 3\NR\NC  x^2 + 2x + 3 \NC 2x - 1 \NC
  2x - 1\NR\stopdeterminant = aA + B$, where $A$ and $B$ are determinants of $3$rd order not involving $x$.
\item If $D_r = \startdeterminant\NC  r \NC x \NC \frac{n(n + 1)}{2}\NR\NC  2r - 1 \NC y \NC \frac{n(3n - 1)}{2}\NR\NC  3r - 2 \NC z \NC \frac{n(3n -
    1)}{2}\NR\stopdeterminant$ show that $\displaystyle\sum_{r=1}^nD_r = 0$.
\item Without expanding the determinant, show that the value of $\startdeterminant\NC -5 \NC 3 + 5i \NC \frac{3}{2} - 4i\NR\NC  3 - 5i \NC 8 \NC 4 +
  5i\NR\NC \frac{3}{2} + 4i \NC 4 - 5i \NC 9\NR\stopdeterminant$ is real.
\item Prove that $\startdeterminant\NC -2a \NC a + b \NC b + c\NR\NC b + a \NC -2b \NC b + c\NR\NC c + a \NC c + b \NC -2c\NR\stopdeterminant = 4(a + b)(b + c)(c +
  a).$
\item $f_r(x), g_r(x), h_r(x)$, where $r = 1,2,3$ are polynomials in $x$ such that $f_r(a) = g_r(a) = h_r(a)$ and \startformula F(x)
  = \startdeterminant\NC  f_1(x) \NC f_2(x) \NC f_3(x)\NR\NC g_1(x) \NC g_2(x) \NC g_3(x)\NR\NC h_1(x) \NC h_2(x) \NC h_3(x)\NR\stopdeterminant\stopformula then find $F'(x)$.
\item Let $\alpha$ be a repeated root of a quadratic equation $f(x) = 0$ and $A(x), B(x), C(x)$ be polynomials of degree $3,4,5$
  respectively. Show that $\Delta(x) = \startdeterminant\NC  A(x) \NC B(x) \NC C(x)\NR\NC A(\alpha) \NC B(\alpha) \NC C(\alpha)\NR\NC A'(\alpha) \NC
    B'(\alpha) \NC C'(\alpha)\NR\stopdeterminant$ is divisible by $f(x)$, where prime denotes a derivative.
\item Prove that $\startdeterminant\NC \cos(\theta + \alpha) \NC \cos(\theta + \beta) \NC \cos(\theta + \gamma)\NR\NC \sin(\theta + \alpha) \NC
  \sin(\theta + \beta) \NC \sin(\theta + \gamma)\NR\NC \sin(\beta - \gamma) \NC \sin(\gamma - \alpha) \NC \sin(\alpha - \beta)\NR\stopdeterminant$
  is independent of $\theta$.
\item If $f, g, h$ are differential functions of $x$ and $\Delta = \startdeterminant\NC  f \NC g \NC h\NR\NC  f' \NC g' \NC h'\NR\NC (x^2f)'' \NC (x^2g)'' \NC
  (x^2h)''\NR\stopdeterminant$ prove that $\Delta' = \startdeterminant\NC  f \NC g \NC h\NR\NC f' \NC g' \NC h'\NR\NC (x^3f'')' \NC (x^3y'')' \NC
    (x^3h'')'\NR\stopdeterminant$
\item If $f(x) = \startdeterminant\NC  x^n \NC \sin x \NC \cos x\NR\NC n! \NC \sin\frac{n\pi}{2} \NC \cos\frac{n\pi}{2}\NR\NC a \NC a^2 \NC a^3\NR\stopdeterminant$,
  then show that $\frac{d^nf(x)}{dx^n} = 0$, where $x = 0$.
\item Prove that $\startdeterminant\NC \cos(A - P) \NC \cos(A - Q) \NC \cos(A - R)\NR\NC \cos(B - P) \NC \cos(B - Q) \NC \cos(Q - R)\NR\NC \cos(C - P) \NC
  \cos(C - Q) \NC \cos(C- R)\NR\stopdeterminant = 0$.
\item Prove that $\startdeterminant\NC  2bc - a^2 \NC c^2 \NC b^2\NR\NC c^2 \NC 2bc - b^2 \NC a^2\NR\NC b^2 \NC a^2 \NC 2bc - c^2\NR\stopdeterminant = (a^3 + b^3 +
  c^3 - 3abc)^2$.
\item Prove that $\startdeterminant\NC  1 \NC \cos(\beta - \alpha) \NC \cos(\gamma - \alpha)\NR\NC \cos(\alpha - \beta) \NC 1 \NC \cos(\gamma -
  beta)\NR\NC \cos(\alpha - \gamma) \NC \cos(\beta - \gamma) \NC 1\NR\stopdeterminant = 0.$
\item For what value of $m$ does the system of equation $3x + my = m$ and $2x - 5y = 20$ has a solution satisfying the conditions
  $x >0,\ y>0$.
\item Prove that the system of equation $3x - y + 4z = 0,\ x + 2y - 3z = -2,\ 6x + 5y + \lambda z = -3$ has at least one solution
  for any real $\lambda$. Find the set of solutions when $\lambda = -5$.
\item For what value of $p$ and $q$, the system of equations $2x + py + 6z = 8,\ x + 2y + qz = 5,\ x + y + 3z = 4$ has (a) no
  solution (b) a unique solution, and (c) infinite solutions.
\item Let $\lambda$ and $\alpha$ be real. Find the set of all values of $\lambda$ for which the system of equations: $\lambda x +
  y\sin\alpha - z\cos\alpha = 0,\ x + y\cos\alpha + z\sin\alpha = 0,\ -x + y\sin\alpha - z\cos\alpha = 0$.
\item Evaluate $\startdeterminant\NC  a \NC b + c \NC a^2\NR\NC b \NC c + a \NC b^2\NR\NC c \NC a + b \NC c^2\NR\stopdeterminant$.
\item Evaluate $\startdeterminant\NC \sqrt{13} + \sqrt{3} \NC 2\sqrt{5} \NC \sqrt{5}\NR\NC  \sqrt{15} + \sqrt{26} \NC 5 \NC \sqrt{10}\NR\NC  3 + \sqrt{65}
  \NC \sqrt{15} \NC 5\NR\stopdeterminant$.
\item Evaluate $\startdeterminant\NC  x \NC x(x^2 + 1) \NC x + 1\NR\NC y \NC y(y^2 + 1) \NC y + 1\NR\NC z \NC z(z^2 + 1) \NC z + 1\NR\stopdeterminant$.
\item If $x, y, z$ are respectively $l$th, $2m$th, $3n$th terms of an H.P., then find the value of $\startdeterminant\NC  yz \NC zx \NC xy
  \NR\NC  l \NC 2m \NC 3n\NR\NC 1 \NC 1 \NC 1\NR\stopdeterminant$.
\item Show that $\startdeterminant\NC 1 \NC a^2 \NC a^3\NR\NC 1 \NC b^2 \NC b^3\NR\NC 1 \NC c^2 \NC c^3\NR\stopdeterminant = (ab + bc + ca)\startdeterminant\NC 1 \NC a \NC
    a^2\NR\NC 1 \NC b \NC b^2\NR\NC 1 \NC c \NC c^2\NR\stopdeterminant$.
\item Evaluate $\startdeterminant\NC (b + c)^2 \NC a^2 \NC bc\NR\NC (c + a)^2 \NC b^2 \NC ca\NR\NC  (a + b)^2 \NC c^2 \NC ab\NR\stopdeterminant$.
\item Prove that $\startdeterminant\NC  x^2 \NC x^2 - (y - z)^2 \NC yz\NR\NC y^2 \NC y^2 - (z - x)^2 \NC zx\NR\NC z^2 \NC z^2 - (x - y)^2 \NC xy\NR\stopdeterminant =
  (x - y)(y - z)(z - x)(x + y + z)(x^2 + y^2 + z^2)$.
\item If $a_1b_1c_1,\ a_2b_2c_2,\ a_3b_3c_3$ are three $3$ digit numbers such that each of them is divisible by $k$, then prove
  that the determinant $\startdeterminant\NC  a_1 \NC b_1 \NC c_1\NR\NC a_2 \NC b_2 \NC c_2\NR\NC a_3 \NC b_3 \NC c_3\NR\stopdeterminant$ is divisible by $k$.
\item If $a_i,\ b_i,\ c_i\in\mathbb{R}(i = 1,\ 2,\ 3)$ and $x\in R$, show that $\startdeterminant\NC  a_1 + b_1x \NC a_1x + b_1 \NC c_1\NR\NC a_2
  + b_2x \NC a_2x + b_2 \NC c_2\NR\NC a_3 + b_3x \NC a_3x + b_3 \NC c_3\NR\stopdeterminant = (1 - x^2)\startdeterminant\NC  a_1 \NC b_1 \NC c_1\NR\NC a_2 \NC b_2 \NC
    c_2\NR\NC a_3 \NC b_3 \NC c_3\NR\stopdeterminant$.
\item If $a,\ b,\ c$ are the roots of the equation $px^3 + qx^2 + rx + s = 0$, then find the value of $\startdeterminant\NC  1 + a \NC 1 \NC
  1\NR\NC 1 \NC 1 + b \NC 1\NR\NC 1 \NC 1 \NC 1 + c\NR\stopdeterminant$.
\item If $a < b < c$, prove that $\startdeterminant\NC 1 \NC a \NC a^4\NR\NC 1 \NC b \NC b^4\NR\NC 1 \NC c \NC c^4\NR\stopdeterminant > 0$.
\item If $a,\ b,\ c$ are distinct and $\startdeterminant\NC  a \NC a^3 \NC a^4 -1\NR\NC b \NC b^3 \NC b^4 - 1\NR\NC c \NC c^3 \NC c^4 - 1\NR\stopdeterminant = 0$,
  show that $abc(ab + bc + ca) = a + b + c$.
\item Show that $x_1,\ x_2,\ x_3\neq 0$, $\startdeterminant\NC  x_1 + a_1b_1 \NC a_1b_2 \NC a_1b_3\NR\NC a_2b_1 \NC x_2 + a_2b_2 \NC a_2b_3\NR\NC a_3b_1 \NC
  a_3b_2 \NC x_3 + a_3b_3\NR\stopdeterminant = x_1x_2x_3$

  $\left(1 + \frac{a_1b_1}{x} + \frac{a_2b_2}{x} + \frac{a_3b_3}{x_3}\right)$.
\item Show that $\startdeterminant\NC \frac{1}{a + x} \NC \frac{1}{a + y} \NC 1\NR\NC \frac{1}{b + x} \NC \frac{1}{b + y} \NC 1\NR\NC \frac{1}{c + x} \NC
  \frac{1}{c + y} \NC 1\NR\stopdeterminant  = \frac{(a - b)(b - c)(c - a)(x - y)}{(a + x)(b + x)(c + x)(a + y)(b + y)(c + y)}$.
\item Show that $\startdeterminant\NC  a^2 \NC bc \NC ac + c^2\NR\NC a^2 + ab \NC b^2 \NC ac\NR\NC ab \NC b^2 + bc \NC c^2\NR\stopdeterminant = 4a^2b^2c^2$.
\item Show that $\startdeterminant\NC 1 + a^2 - b^2 \NC 2ab \NC -2b\NR\NC 2ab \NC 1 - a^2 + b^2 \NC 2a\NR\NC 2b \NC -2a \NC 1 - a^2 - b^2\NR\stopdeterminant = (1 +
  a^2 + b^2)^3$.
\item If $a,\ b,\ c$ are sides of a triangle, show that $\startdeterminant\NC  a^2 \NC (s - a)^2 \NC (s - b)^2\NR\NC (s - b)^2 \NC b^2 \NC (s -
  b)^2\NR\NC (s - c)^2 \NC (s - c)^2 \NC c^2\NR\stopdeterminant = \frac{1}{2}P^2A^2$, where $P$ denotes the perimeter of the triangle, $A$ its
  area and $s = \frac{P}{2}$.
\item Show that $\startdeterminant\NC (x - a)^2 \NC ab \NC ac\NR\NC ba \NC (x - b)^2 \NC bc\NR\NC ca \NC cb \NC (x - c)^2\NR\stopdeterminant = x^2(x - 2a)(x - 2b)(x
  - 2c)$

  $\left(x + \frac{a^2}{x - 2a} + \frac{b^2}{x - 2b} + \frac{c^2}{x - 2c}\right)$.
\item If $x,\ y,\ z$ are unequal and $\startdeterminant\NC  x^3 \NC (x + a)^3 \NC (x - a)^3\NR\NC y^3 \NC (y + a)^3 \NC (y - a)^3\NR\NC z^3 \NC (z + a)^3 \NC (z
  - z)^3\NR\stopdeterminant = 0$, prove that $a^2(x + y + z) = 3xyz$.
\item Show that $\startdeterminant\NC (1 - x) \NC a \NC a^2\NR\NC a \NC a^2 - x \NC a^3\NR\NC a^2 \NC a^3 \NC a^4 - x\NR\stopdeterminant = x^2(1 + a^2 + a^3) - x^3$.
\item If $y = \sin px$ and $y_n = \frac{d^nx}{dy^n}$, find the value of $\startdeterminant\NC  y \NC y_1 \NC y_2\NR\NC y_3 \NC y_4 \NC y_5\NR\NC y_6 \NC y_7
  \NC y_8\NR\stopdeterminant$.
\item Evaluate $\startdeterminant\NC \cos^2\theta \NC \cos\theta\sin\theta \NC -\sin\theta\NR\NC \cos\theta\sin\theta \NC \sin^2\theta \NC
  \cos\theta\NR\NC \sin\theta \NC -\cos\theta \NC 0\NR\stopdeterminant$.
\item Evaluate $\startdeterminant\NC \cos\alpha \NC \sin\alpha\cos\beta \NC \sin\alpha\sin\beta\NR\NC -\sin\alpha \NC \cos\alpha\cos\beta \NC
  \cos\alpha\sin\beta\NR\NC 0 \NC -\sin\beta \NC \cos\beta\NR\stopdeterminant$.
\item Solve the equation $\startdeterminant\NC  a^2 + x \NC ab \NC ac\NR\NC ab \NC b^2 + x \NC bc\NR\NC ac \NC bc \NC c^2 + x\NR\stopdeterminant = 0$.
\item Solve the equation for $x, \startdeterminant\NC {}^xC_r \NC {}^{n - 1}C_r \NC {}^{n - 1}C_{r - 1}\NR\NC {}^{x + 1}C_r \NC {}^nC_r * {}^nC_{r -
    1}\NR\NC {}^{x + 2}C_r \NC {}^{n + 1}C_r \NC {}^{n + 1}C_{r - 1}\NR\stopdeterminant = 0\ \forall n,r > 1$.
\item Solve the equation $\startdeterminant\NC  u + a^2x \NC w' + abx \NC v' + acx\NR\NC w' + abx \NC v + b^2x \NC u' + bcx\NR\NC  v' + acx \NC u' + bcx \NC w
  + c^2x\NR\stopdeterminant = 0$ expressing the result by means of determinants.
\item If $f(a,\ b) = \frac{f(b) - f(a)}{b - a}$ and $f(a, b, c) = \frac{f(b, c) - f(a, b)}{c - a}$, show that \startformula f(a, b, c)
  = \startdeterminant\NC  f(a) \NC f(b) \NC f(c)\NR\NC 1 \NC 1 \NC 1\NR\NC a \NC b \NC c\NR\stopdeterminant\div\startdeterminant\NC 1 \NC 1 \NC 1\NR\NC a \NC b \NC c\NR\NC a^2 \NC b^2 \NC
    c^2\NR\stopdeterminant.\stopformula
\item If $A, B, C$ are the angles of a $\triangle ABC$, then prove that $\startdeterminant\NC  e^{2iA} \NC e^{-iC} \NC e^{-iB}\NR\NC e^{-iC} \NC
  e^{2iB} \NC e^{-iA}\NR\NC e^{-iB} \NC e^{-iA} \NC e^{2iC}\NR\stopdeterminant$ is purely real.
\item If $A, B, C$ are the angles of a $\triangle ABC$ such that $A\geq B\geq C$, find the minimum value of $\Delta$, where
  $\Delta = \startdeterminant\NC \sin^2A \NC \sin A\cos A \NC \cos^2A\NR\NC \sin^2B \NC \sin B\cos B \NC \cos^2B\NR\NC \sin^2C \NC \sin C\cos C \NC
  \cos^2C\NR\stopdeterminant$. Also, show that $\Delta = \frac{1}{4}[\sin(2A - 2B) + \sin(2B - 2C) + \sin(2C - 2A)]$.
\item Evaluate $\startdeterminant\NC  a^2 \NC a \NC 1\NR\NC \cos nx \NC \cos(n + 1)x \NC \cos(n + 2)x\NR\NC \sin nx \NC \sin(n + 1)x \NC \sin(n +
  2)x\NR\stopdeterminant$.
\item If $0<x<\frac{\pi}{2}$, the find the values of $x$ for which $\startdeterminant\NC  1 + \sin^2x \NC \cos^2x \NC 4\sin2x\NR\NC \sin^2x \NC 1 +
  \cos^2x \NC 4\sin2x\NR\NC \sin^2x \NC \cos^2x \NC 1 + 4\sin2x\NR\stopdeterminant$ has maximum value.
\item If $A, B, C$ are the angles of a triangle, show that $\startdeterminant\NC -1 \NC \cos C \NC \cos B\NR\NC \cos C \NC -1 \NC \cos A\NR\NC \cos B \NC
  \cos A \NC 1\NR\stopdeterminant = 0$.
\item If $A, B, C$ are the angles of an isoscceles triangle, evaluate \startformula \startdeterminant\NC 1 \NC 1 \NC 1\NR\NC 1 + \sin A \NC 1 + \sin B \NC 1 +
  \sin C\NR\NC \sin A + \sin^2A \NC \sin B + \sin^2B \NC \sin C + \sin^2C\NR\stopdeterminant.\stopformula
\item For positive numbers $x, y, z\neq 1$, show that the numeric value of the determinant \startformula \startdeterminant\NC 1 \NC \log_xy \NC
  \log_xz\NR\NC \log_yx \NC 1 \NC \log_yz\NR\NC \log_zx \NC \log_zy \NC 1\NR\stopdeterminant = 0.\stopformula
\item If $a, b, c > 0$ and $x, y, z\in\mathbb{R}$, then show without expanding that \startformula \startdeterminant\NC (a^x + a^{-x})^2 \NC (a^x -
  a^{-x})^2 \NC 1\NR\NC (b^y + b^{-y})^2 \NC (b^y - b^{-y})^2 \NC 1\NR\NC (c^z + c^{-z})^2 \NC (c^z - c^{-z})^2 \NC 1\NR\stopdeterminant = 0.\stopformula
\item Without expanding the determinants, prove that $\startdeterminant\NC 103 \NC 115 \NC 114\NR\NC 111 \NC 108 \NC 106\NR\NC  104 \NC 113 \NC
  116\NR\stopdeterminant + \startdeterminant\NC 113 \NC 116 \NC 104\NR\NC 108 \NC 106 \NC 111\NR\NC 115 \NC 114 \NC 103\NR\stopdeterminant = 0$.
\item Evaluate $\displaystyle\sum_{n = 1}^NU_n$ if $U_n = \startdeterminant\NC  n \NC 1 \NC 5\NR\NC n^2 \NC 2N + 1 \NC 2N + 1\NR\NC n^3 \NC 3N^2 \NC
  3N\NR\stopdeterminant$.
\item If $A, B,C$ are the angles of a triangle, then show without expanding that \startformula \startdeterminant\NC \sin(A + B + C) \NC \sin B \NC \cos
  C\NR\NC -\sin B \NC 0 \NC \tan A\NR\NC \cos(A + B) \NC -\tan A \NC 0\NR\stopdeterminant = 0.\stopformula
\item Evaluate without expanding $\startdeterminant\NC  b^2 - ab \NC b - c \NC bc - ac\NR\NC  ab - a^2 \NC a - b \NC b^2 - ab\NR\NC  bc - ac \NC c - a \NC ab -
  a^2\NR\stopdeterminant$
\item Let $\Delta_i = \startdeterminant\NC  i - 1 \NC n \NC 6\NR\NC (i - 1)^2 \NC 2n^2 \NC 4n - 2\NR\NC  (i - 1)^3 \NC 3n^3 \NC 3n^2 - 2n\NR\stopdeterminant$. Show
  that $\displaystyle\sum_{n = 1}^n\Delta_i = k$, a constant.
\item Let $m\in\mathbb{P}$ and $\Delta_r = \startdeterminant\NC  2r - 1 \NC {}^mC_r \NC 1\NR\NC m^2 - 1 \NC 2^m \NC m + 1\NR\NC \sin^2m^2 \NC \sin^2m \NC
  \sin^2(m  + 1)\NR\stopdeterminant$, then find the value of $\displaystyle\sum_{r = 0}^m\Delta_r$.
\item Show that $\startdeterminant\NC {}^{x}C_r \NC {}^xC_{r + 1} \NC {}^xC_{r + 2}\NR\NC {}^{y}C_r \NC {}^yC_{r + 1} \NC {}^yC_{r + 2}\NR\NC {}^{z}C_r \NC
  {}^zC_{r + 1} \NC {}^zC_{r + 2}\NR\stopdeterminant = \startdeterminant\NC {}^{x}C_r \NC {}^{x + 1}C_{r + 1} \NC{}^{x + 2}C_{r + 1}\NR\NC {}^{y}C_r \NC
  {}^{y + 1}C_{r + 1} \NC{}^{y + 2}C_{r + 1}\NR\NC {}^{z}C_r \NC {}^{z + 1}C_{r + 1} \NC{}^{z + 2}C_{r + 1}\NR\stopdeterminant$
\item If $\Delta_r = \startdeterminant\NC  r \NC n + 1\NC 1\NR\NC r^2 \NC 2n - 1 \NC \frac{2n + 1}{3}\NR\NC r^3 \NC 3n + 2 \NC \frac{n(n + 1)}{2}\NR\stopdeterminant$,
  show that $\displaystyle\sum_{r=1}^n\Delta_r = 0$.
\item If $\Delta_r = \startdeterminant\NC 2^{r - 1} \NC 2.3^{r - 1} \NC 4.5^{r - 1}\NR\NC x \NC y \NC z\NR\NC 2^{n} - 1 \NC 3^n - 1 \NC 5^n - 1\NR\stopdeterminant$,
  show that $\displaystyle\sum_{r=1}^n\Delta_r = 0$.
\item Show without expanding that $\startdeterminant\NC  x^2 \NC (x - 1)^2 \NC (x - 2)^2\NR\NC (x - 1)^2 \NC (x - 2)^2 \NC (x - 3)^2\NR\NC (x - 2)^2 \NC (x -
  3)^2 \NC (x- 4)^2\NR\stopdeterminant$ is independent of $x$.
\item Show without expanding that $\startdeterminant\NC 2 \NC 1 + i \NC 3\NR\NC  1- i \NC 0 \NC 2 + i\NR\NC 3 \NC 2 - i \NC 1\NR\stopdeterminant$ is purely real.
\item Show without expanding that $\startdeterminant\NC  x - 3 \NC 2x + 1 \NC 2\NR\NC 3x + 2 \NC x + 2 \NC 1\NR\NC 5x + 1 \NC 5x + 4 \NC 5\NR\stopdeterminant$ is
  independent of $x$.
\item If $a$ and $x$ are real numbers and $n$ is a positive integer, then show without expanding that $\startdeterminant\NC  a^n - x \NC
  a^{n + 1} - x \NC a^{n + 2} - x\NR\NC a^{n + 3} - x \NC a^{n + 4} - x \NC a^{n + 5} - x\NR\NC a^{n + 6} - x \NC a^{n + 7} - x \NC a^{n + 8} -
  x\NR\stopdeterminant = 0$.
\item Find $\displaystyle\sum_{r=2}^n(-2)^r\startdeterminant\NC {}^{n - 2}C_{r - 2} \NC {}^{n - 2}C_{r - 1} \NC {}^{n - 2}C_r\NR\NC -3 \NC 1 \NC 1\NR\NC 2 \NC
  1\NC 0\NR\stopdeterminant, n > 2$.
\item If $a, b, c$ are non-zero real numbers, show without expanding that $\startdeterminant\NC  b^2c^2 \NC bc \NC b + c\NR\NC c^2a^2 \NC ca \NC c +
  a\NR\NC  a^2b^2 \NC ab \NC a + b\NR\stopdeterminant = 0$.
\item Prove that $\startdeterminant\NC  b + a - a - d \NC bc - ad \NC bc(a + d) - ad(b + d)\NR\NC c + a - b - d \NC ca - bd \NC ca(b + d) - bd(c +
  a)\NR\NC  a + b - c - d \NC ab - cd \NC ab(c + d) - cd(a + b)\NR\stopdeterminant = -2(b - c)(c - a)(a - b)(a - d)(b - d)(c - d)$.
\item Prove that $\startdeterminant\NC  bc - a^2 \NC ca - b^2 \NC ab - c^2\NR\NC ca + ab - bc \NC bc + ab - ca \NC bc + ca - ab\NR\NC (a + b)(a + c) \NC (b +
  c)(b + a) \NC (c + a)(c + b)\NR\stopdeterminant= 3(b - c)(c - a)(a - b)(a + b + c)(ab + bc + ca)$.
\item Prove that $\startdeterminant\NC 1 \NC (m + n - l - p)^2 \NC (m + n - l - p)^4\NR\NC 1 \NC (n + l - m - p)^2 \NC (n + l - m - p)^4\NR\NC 1 \NC (l + m -
  n - p)^2 \NC (l + m - n - p)^4\NR\stopdeterminant = 64(l - m)(l - n)(l - p)(m - n)(m -p)(n - p)$.
\item If $u, v, w$ are differentiable functions of $f$ and suffixes denote the derivatives w.r..t $t$, prove that
  $\frac{d}{dt}\startdeterminant\NC  u_1 \NC v_1 \NC w_1\NR\NC u_2 \NC v_2 \NC w_2\NR\NC u_3 \NC v_3 \NC w_3\NR\stopdeterminant = \startdeterminant\NC  u_1 \NC v_1 \NC w_1\NR\NC u_2
    \NC v_2 \NC w_2\NR\NC u_4 \NC v_4 \NC w_4\NR\stopdeterminant$.
\item If $Y = sX$ and $Z= tX$, all the variables being differentiable functions of $x$, prove that $\startdeterminant\NC  X \NC Y \NC Z\NR\NC X_1
  \NC Y_1 \NC Z_1\NR\NC X_2 \NC Y_2 \NC Z_2\NR\stopdeterminant = X^3\startdeterminant\NC  s_1 \NC t_1\NR\NC s_2 \NC t_2\NR\stopdeterminant$, where suffixes denote the
  derivatives w.r.t. $x$.
\item If $f(x), g(x), h(x)$ are polynomials in $x$, find the condition that $\startdeterminant\NC  f(x) \NC g(x) \NC h(x)\NR\NC f(\alpha) \NC
  g(\alpha) \NC h(\alpha)\NR\NC  f(\beta) \NC g(\beta) \NC h(\beta)\NR\stopdeterminant$, which is a polynomial of degree $3$, is expressible as $a(x
  - \alpha)^2(x - \beta)$.
\item Show that $\startdeterminant\NC \sin(x + \alpha) \NC \cos(x + \alpha) \NC a + x\sin\alpha\NR\NC \sin(x + \beta) \NC \cos(x + \beta) \NC b +
  x\sin\beta\NR\NC \sin(x + \gamma) \NC \cos(x + \gamma) \NC c + x\sin\gamma\NR\stopdeterminant$ is independent of $x$.
\item If $l_r\vec{i}, m_r\vec{j}, n_r\vec{k}, r = 1, 2 ,3$ be three mutually perpendicular unit vectors, show that
  $\startdeterminant\NC  l_1 \NC l_2 \NC l_3\NR\NC m_1 \NC m_2 \NC m_3\NR\NC n_1 \NC n_2 \NC n_3\NR\stopdeterminant = \pm 1$.
\item Let $\Delta = \startdeterminant\NC  a_1 \NC b_1 \NC c_1\NR\NC a_2 \NC b_2 \NC c_2\NR\NC a_3 \NC b_3 \NC c_3\NR\stopdeterminant$ and $A_i, B_i, C_i$ be the
  cofactors of $a_i, b_i, c_i$ respectively and $\alpha_i, \beta_i, \gamma_i$ be the cofactors of $A_i, B_i, C_i$ respectively,
  where $i = 1, 2, 3$, show that $\startdeterminant\NC  A_1 \NC B_1 \NC C_1\NR\NC A_2 \NC B_2 \NC C_2\NR\NC A_3 \NC B_3 \NC
    C_3\NR\stopdeterminant\startdeterminant\NC \alpha_1 \NC \beta_1 \NC \gamma_1\NR\NC \alpha_2 \NC \beta_2 \NC \gamma_2\NR\NC \alpha_3 \NC \beta_3 \NC
      \gamma_3\NR\stopdeterminant = \Delta^6$
\item Using determinants, solve the equations: $x + 2y + 3z = 6, 2x + 4y + z = 17, 3x + 2y + 9z = 2$.
\item Solve the system of equations $ax + by + ca = d, a^2x + b^2y + c^2a = d^2, a^3x + b^3y + c^3a = d^3$. Will the solution
  always exist and be unique?
\item Determine the coefficients $a, b, c$ of the quadratic function where $f(x) = ax^2 + bx + c$, if $f(1) = 0, f(2) = -2$ and
  $f(3) = -6$.
\item Determine the coefficients $a, b, c$ of the quadratic function where $f(x) = ax^2 + bx + c$, if $f(0) = 6, f(2) = 11, f(-3) =
  6$. Also, find $f(1)$.
\item Solvve $(b + c)(y + z) - ax = b - c, (c + a)(z + x) - by = c - a, (a + b)(x + y) - cz = a - b$, where $a + b + c\neq 0$.
\item Examine the consistency of the system of equations $7x - 7y + 5z = 3, 3x + y + 5z = 7$ and $2x + 3y + 5z = 5$.
\item Find the value of $k$ for which the following system of equations is consistent $x + y = 3, (1 + k)x + (2 + k)y = 8, x - (1 +
  k)y + (2 + k) = 0$.
\item Find the value of $k$ for which the following system of equations is consistent $(k + 1)^3x + (k + 2)^3y = (k + 1)^3, (k +
  1)x + (k + 2)y = k + 3, x + y = 1$.
\item Find the values of $c$ for which the system of equations $2x + 3y = 4; (c + 2)x + (c + 4)y = c + 6, (c + 2)^2x + (c + 4)^2 =
  (c + 6)^2$ are consistent and find the solutions for all such values of $c$.
\item Find the values of $\lambda$ for which the system of equations $x + y - 2z = 0, 2x - 3y + z = 0, x - 5y + 4z = \lambda$ are
  consistent and find the solutions for all such values of $\lambda$.
\item Find the values of $\lambda$ and $\mu$ for which the following system of equations $x + y + z = 0, x + 2y + 3z = 14, 2x + 5y
  + \lambda z = \mu,\ \lambda, \mu\in R$ has (a) unique solution (b) infinite solutions.
\item If $bc + qr = ca + rp = ab + pq = -1$, show that $\startdeterminant\NC  qp \NC a \NC p\NR\NC bq \NC b\NC q\NR\NC cr \NC c \NC r\NR\stopdeterminant = 0$.
\item Find all values of $k$ for which the following system possesses a non-trivial solution: $x + ky + 3z = 0; kx + 2y + 2z = 0,
  2x + 3y + 4z = 0$
\item If $x = cy + bz, y = az + cx, z = bx + ay$, where $a, b, c$ are not all zero. Prove that $a^2 + b^2 + c^2 + 2abc = 1$.
  Further if at least one of $a, b, c$ is a proper fraction, prove that (a) $a^2 + b^2 + c^2 < 3$ (b) $abc > -1$
\item If $a = \frac{x}{y - z}, b = \frac{y}{z - x}, c = \frac{z}{x - y}$, where $x, y, z$ are not all zero, prove that $1 + ab + bc
  + ca = 0$.
\item Consider the system of linear equations, in $x, y, z: (\sin3\theta)x - y + z = 0, (\cos2\theta) + 4y + 3z = 0, 2x + 7y + 7z =
  0$. Find the value of $\theta$ for which this system has non-trivial solution.
\item If $a, b, c$ are in G.P. with common ratio $r_1; \alpha, \beta, \gamma$ are in G.P. with common ratio $r_2$, then find the
  conditions that $r_1$ must satisfy in order that the equations $ax + \alpha y + z = 0, bx + \beta y + z = 0, cx + \gamma y + z =
  0$ have only trivial solutions.
\item Prove that $\startdeterminant\NC  x \NC l \NC m \NC 1\NR\NC \alpha \NC x \NC n \NC 1\NR\NC \alpha \NC \beta \NC x \NC 1\NR\NC \alpha \NC \beta \NC \gamma \NC 1\NR\stopdeterminant
  = (x - \alpha)(x - \beta)(x - \gamma)$, where $l, m, n$ have any values whatever.
\item Prove that $\startdeterminant\NC  a \NC b \NC c \NC d\NR\NC -b \NC a \NC -d \NC c\NR\NC -c \NC d \NC a \NC -b\NR\NC -d \NC -c \NC b \NC a\NR\stopdeterminant = (a^2 + b^2 + c^2 +
  d^2)^2$.
\item If $u = ax^4 + 4bx^3 + 6cx^2 + 4dx + e$ and $u_{11} = ax^2 + 2bx + c, u_{12} = bx^2 + 2cx + d, u_{22} = cx^2 + 2dx + e$,
  prove that $\startdeterminant\NC  a \NC b \NC c \NC u_{11}\NR\NC b \NC c \NC d \NC u_{12}\NR\NC c \NC d \NC e \NC u_{22}\NR\NC u_{11} \NC u_{12} \NC u_{22} \NC 0\NR\stopdeterminant
    = -u\startdeterminant\NC  a \NC b \NC c\NR\NC b \NC c \NC d\NR\NC c \NC d \NC e\NR\stopdeterminant.$
\item If $u = ax^2 + 2bxy + cy^2, u' = a'x^2 + 2b'xy + c'y^2$, prove that $\startdeterminant\NC  y^2 \NC -xy \NC x^2\NR\NC a \NC b \NC c\NR\NC a' \NC b' \NC
  c'\NR\stopdeterminant = \startdeterminant\NC  ax + by \NC bx + cy\NR\NC a'x + b'y \NC b'x + c'y\NR\stopdeterminant = -\frac{1}{y}\startdeterminant\NC  u \NC u'\NR\NC ax +
      by \NC a'x + b'y\NR\stopdeterminant$.
\item Prove that $\startdeterminant\NC  a \NC b \NC ax + by\NR\NC b \NC c \NC bx + cy\NR\NC ax + by \NC bx + cy \NC 0\NR\stopdeterminant = -(ac - b^2)(ax^2 + 2bxy +
  cy^2)$.
\item Prove that the determinant $\startdeterminant\NC  a - x \NC b \NC c \NC d\NR\NC b \NC c - x \NC d \NC a\NR\NC c \NC d \NC a - x \NC b\NR\NC d \NC a \NC b \NC c -
  x\NR\stopdeterminant = (x - a - b - c -d)(x - a + b - c + d)[x^2 - (a - c)^2 - (b - d)^2]$.
\stopitemize
