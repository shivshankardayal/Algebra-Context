!\chapter{Inequalities}
Ineuqalities come up in different branches of mathematics; for example in algebra, geometry and trigonometry. They are very
useful in establishing many relations among various quantities. Certain inequalities are very useful in studying properties of
many common expressions which lead to interesting observations. In this chapter we will only study algebraic inequalitites. The
problems given are quite basic and simple. We start with some useful theorems for these inequalities.

There are some facts which are the very important for proving inequalities. Some of them are as follows:

\startitemize[n]
\item If $x\geq y$ and $y\geq z$ then $x\geq z$, for any $x, y, z\in\mathbb{R}$.
\item If $x\geq a$ and $y\geq b$ then $x + a\geq y + b$, for any $x, y, a, b\in\mathbb{R}$.
\item If $x\geq y$ then $x + z\geq y + z$, for any $x, y, z\in\mathbb{R}$.
\item If $x\geq y$ and $a\geq b$ then $xa\geq yb$, for any $x, y\in\mathbb{R}^+$ or $a, b\in\mathbb{R}^+$.
\item If $x\in\mathbb{R}$ then $x^2\geq 0$, with equality holding if and only if $x = 0$. More generally for $a_i\in\mathbb{R}^+$
  and $x_i\in\mathbb{R}, i = 1, 2, \ldots, n$ holds $a_ix_i^2 + a_2x_2^2 + \cdots + a_nx_n^2\geq 0$, with equality holding if and
  only if $x_1 = x_2 = \cdots = x_n = 0$.
\stopitemize

\section{Strum's Method}
Strum's method is given by the German mathematician {\sl Friedrich Otto Rudolf Sturm}. Sturm's method helps prove a large number of
different inequalities under certain conditions along with various other applications.

\starttheorem[th:strum:1]
Prove that if the product of positive numbers $x_1, x_2, \cdots, x_n (n\geq 2)$ is euqal to $1$, then $x_1 + x_2 + \cdots + x_n\geq n$.
\stoptheorem
\startproof
  If $x_1 = \cdots = x_n$, then $x_1 + \cdots + x_n = n$. So we see that the statement is true if all the numbers are equal and are
  unity. Now we consider the case when at least two numbers are different such that one is greater than $1$ and the other one is
  smaller. Let us assume that these are $x_1$ and $x_2$ which does not cause loss of generality, and that $x_1 < 1 < x_2$. Note
  that $x_1 + x_2 > 1 + x_1x_2 [\because (1 - x_1)(x2 - 1) > 0]$. If given numbers are substitued by $1, x_1x_2, x_3, \ldots, x_n$,
  then the product is equal to $1$ and $1 + x_1x_2 + x_3 + \cdots + x_n < x_1 + x_2 + \cdots + x_n$. Repeating this we will find $n
  - 1$ numbers equal to $1$ and the $n$th number equal to $x_1x_2\ldots x_n$. Thus, $x_1 + x_2 + \cdots + x_n < 1$. We see that
  equality holds if and only if $x_1 = x_2 = \cdots = x_n = 1$
\stopproof

\starttheorem[th:strum:2]
  Prove that if the sum of the numbers $x_1, x_2, \ldots, x_n (n\geq 2)$ is equal to $1$, then prove that $x_1^2 + x_2^2 + \cdots +
  x_n^2 \geq \frac{1}{n}$.
\stoptheorem
\startproof
  If $x_1 = x_2 = \ldots = x_n = \frac{1}{n}$ then $x_1^2 + x_2^2 + \cdots + x_n^2 = \frac{1}{n}$. Like previous theorem we
  consider two numbers $x_1$ and $x_2$ such that one of them is greater than $\frac{1}{n}$ while the other is smaller than
  $\frac{1}{n}$. Assume that these two numbers are $x_1$ and $x_2$, which does not cause loss of generality, and that $x_1 <
  \frac{1}{n}$ and $x_2 > \frac{1}{n}$. So we obtain a sequence of numbers $\frac{1}{n}, x_1 + x_2 - \frac{1}{n}, x_3, \ldots, x_n$
  suhc that their sum remains equal to $1$. We can easily prove that $x_1^2 + x_2^2 > \frac{1}{n^2} + \left(x_1 + x_2 -
  \frac{1}{n}\right)^2$, and hence \startformula x_1^2 + x_2^2 + \cdots + x_n^2 > \frac{1}{n^2} + \left(x_1 + x_2 - \frac{1}{n}\right)^2 +
  x_3^2 + \cdots + x_n^2.\stopformula

  Repeating this we obtain a sequence in which all terms will be equal to $\frac{1}{n}$, and sum of their square is less than the
  sum of squares of numbers $x_1, x_2, \ldots, x_n$ i.e. $x_1^2 + x_2^2 + \cdots + x_n^2 > \frac{1}{n^2} +
  \;\text{to}\;n\;\text{times}$. From this it follows that equality holds if and only if $x_1 = x_2 = \cdots = x_n$.
\stopproof

\section{A.M., G.M., H.M. and Q.M.}
\starttheorem
  {\rm\bf (A.M.-- G.M. -- H.M. -- Q.M. Inequality)} Let $x_1, x_2, \ldots, x_n$ be positive real numbers, then
  \placeformula\startformula
  \frac{n}{\frac{1}{x_1} + \frac{1}{x_2} + \cdots + \frac{1}{x_n}} \leq \sqrt[n]{x_1x_2 \ldots x_n}\leq \frac{x_1 + x_2 + \cdots
    + x_n}{n}\leq \sqrt{\frac{x_1^2 + x_2^2 + \cdots + x_n^2}{n}}.
  \stopformula
\stoptheorem

\startproof
  Consider the numbers $\frac{x_1}{\sqrt[n]{x_1x_2\cdots x_n}}, \frac{x_2}{\sqrt[n]{x_1x_2\cdots x_n}}, \cdots,
  \frac{x_n}{\sqrt[n]{x_1x_2\cdots x_n}}$, we see that product is equal to $1$. From \;\in{theorem}[th:strum:1], we have that
  \startformula \frac{x_1}{\sqrt[n]{x_1x_2\cdots x_n}} + \frac{x_2}{\sqrt[n]{x_1x_2\cdots x_n}} + \cdots + \frac{x_n}{\sqrt[n]{x_1x_2\cdots
      x_n}}\geq n \Rightarrow \frac{x_1 + x_2 + \cdots + x_n}{n}\geq \sqrt[n]{x_1x_2\cdots x_n}.\stopformula
  The above inequality is also known as Cauchy's inequality.

  In the above inequality, if we substitute $x_i = \frac{1}{x_i}$, then \startformula \frac{n}{\frac{1}{x_1} + \frac{1}{x_2} + \cdots +
    \frac{1}{x_n}} \leq \sqrt[n]{x_1x_2 \cdots x_n}.\stopformula

  Consider the numbers $\frac{x_1}{x_1 + x_2 + \cdots + x_n}, \frac{x_2}{x_1 + x_2 + \cdots + x_n}, \cdots, \frac{x_n}{x_1 + x_2 +
    \cdots + x_n}$, and note that their sum is equal to $1$. According to \;\in{theorem}[th:strum:2], we have \startformula \left(\frac{x_1}{x_1 +
    x_2 + \cdots + x_n}\right)^2 + \left(\frac{x_2}{x_1 + x_2 + \cdots + x_n}\right)^2 + \cdots + \left(\frac{x_n}{x_1 + x_2 +
    \cdots + x_n}\right)^2 \geq \frac{1}{n}\stopformula
  \startformula \Rightarrow \frac{x_1^2 + x_2^2 + \cdots + x_n^2}{n}\geq \left(\frac{x_1 + x_2 + \cdots + x_n}{n}\right)^2.\stopformula

  Hence, all the inequalities have been proven.
\stopproof

\section{Cauchy-Bunyakovsky-Schwarz Inequality}
\starttheorem
  {\rm \bf (Cauchy-Bunyakovsky-Schwarz Inequality)} Let $a_1, a_2, \ldots, a_n,$ $b_1, b_2, \ldots, b_n \in R$. Then
  \placeformula\startformula
    (a_1^2 + a_2^2 + \cdots + a_n^2)(b_1^2 + b_2^2 + \cdots + b_n)^2 \geq (a_1b_1 + a_2b_2 + \cdots a_nb_n)^2.
  \stopformula
\stoptheorem

\startproof
  Let $x_k = \sqrt{(a_1^2 + a_2^2 + \cdots + a_k^2)(b_1^2 + b_2^2 + \cdots + b_k^2)}$, where $k = 1, 2, \ldots, n$.
  In this case,

  \startformula x_{k + 1} = \sqrt{(a_1^2 + a_2^2 + \cdots + a_k^2 + a_{k + 1}^2)(b_1^2 + b_2^2 + \cdots + b_k^2 + b_{k + 1}^2)}\stopformula
  \startformula \sqrt{\left[\left(\sqrt{a_1^2 + a_2^2 + \cdots + a_k^2}\right)^2 + a_{k + 1}^2\right]\left[\left(\sqrt{b_1^2 + b_2^2 + \cdots +
        b_k}\right)^2 + b_{k + 1}^2\right]}\stopformula
  \startformula \geq \sqrt{\left(\sqrt{a_1^2 + a_2^2 + \cdots + a_k^2}.\sqrt{b_1^2 + b_2^2 + \cdots + b_k^2} + a_{k + 1}.b_{k + 1}\right)^2} =
  x_k + a_{k + 1}b_{k + 1}\stopformula
  {\it Alternative Proof.}
  \startformula (a_1^2 + a_2^2 + \cdots + a_n^2)(b_1^2 + b_2^2 + \cdots + b_n)^2 - (a_1b_1 + a_2b_2 + \cdots a_nb_n)^2 =
  \sum_{i,j=1\\i\geq j}^n(a_ib_j - b_ja_i)^2\geq 0.\stopformula
\stopproof

\subsection{Titu's Lemma}
\startlemma
  Let $a_1, a_2, \ldots, a_n, b_1, b_2, \ldots, b_n$ be positive real numbers then
  \placeformula\startformula
  \frac{a_1^2}{b_1} + \frac{a_2^2}{b_2} + \ldots
    + \frac{a_n^2}{b_n} \geq \frac{(a_1 + a_2 + \cdots + a_n)^2}{b_1 + b_2 + \cdots + b_n}
  \stopformula
\stoplemma

\startproof
  This is a direct consequence of {\it Cauchy-Bunyakovsky-Schwarz Inequality}. It is obtained by substituting $a_i =
  \frac{x_i}{\sqrt{y_i}}$ and $b_i = \sqrt{y_i}$ into Cauchy-Bunyakovsky-Schwarz Inequality. Equality holds if and only if $a_i =
  kb_i$ for a non-zero real constant $k$.
\stopproof

\section{Chebyshev's Inequality}
\starttheorem
  Let $a_1, a_2, \ldots, a_n, b_1, b_2, \ldots, b_n$ be real numbers such that  $a_1\leq _2\leq a_2\leq \cdots\leq a_n$ and $b_1\leq
  b_2\leq\cdots\leq b_n$ or $a_1\geq _2\geq a_2\geq \cdots\geq a_n$ and $b_1\geq b_2\geq\cdots\geq b_n$, then the inequality
  \placeformula\startformula
    \left(\frac{a_1 + a_2 + \cdots + a_n}{n}\right)\left(\frac{b_1 + b_2 + \cdots + b_n}{n}\right)\leq \frac{a_1b_1 + a_2b_2 +
      \cdots + a_nb_n}{n}
  \stopformula
  holds. The inequality is strict unless at least one of the sequences is a constant sequence.
\stoptheorem

\startproof
  We have \startformula \sum_{i=1}^n\sum_{j=1}^n(a_ib_i - a_jb_j) = \sum_{i=1}^n\left(na_ib_i - a_i\sum_{j=1}^nb_j\right) =
  n\sum_{i=1}^na_ib_i - \sum_{i=1}^na_i\sum_{j=1}^nb_j\stopformula
  Simiarly
  \startformula \sum_{i=1}^n\sum_{j=1}^n\left(a_jb_j - a_jb_i\right) = n\sum_{j=1}^na_jb_j - \sum_{j=1}^na_j\sum_{i=1}^nb_i\stopformula
  From these two equations, we get
  \startformula n\sum_{j=1}^na_jb_j - \sum_{j=1}^na_j\sum_{i=1}^nb_i = \frac{1}{2}\left[\sum_{i=1}^n\sum_{j=1}^n\left(a_ib_i - a_ib_j + a_jb_j
    - a_jb_i\right)\right]\stopformula
  \startformula = \frac{1}{2}\sum_{i=1}^n\sum_{j=1}^n(a_i - a_j)(b_i - b_j)\stopformula
  Since both the sequences are either decreasing or increasing, we will have $(a_i - a_j)(b_i - b_j)\geq 0$. Thus, we have
  \startformula n\sum_{j=1}^na_jb_j - \sum_{j=1}^na_j\sum_{i=1}^nb_i \geq 0.\stopformula
  Here equality holds if and only if for each of the indexes $i, j$ either $a_i = a_j$ or $b_i = b_j$.
\stopproof

\startremark
  If the the order of sequences $\langle a_i\rangle$ and $\langle b_i\rangle$ in the orevious theorem are reverses then the
  inequlaity reverses as well.

  {\rm The proof is similar to the proof of the theorem.}
\stopremark

\startremark
  Chebyshev's inequality can be generalized to three or more sets of real numbers, with the constraint that sets are in increasing
  or decreasing order.
\stopremark

\startremark
  If the two sequeqnces are non-increasing or non-decreasing, and let $p_1, p_2, \ldots, p_b$ be a sequence of non=negative real
  numbers such that $\sum_{i=1}^np_i$ is positive. Then the following inequality holds
  \startformula \left(\frac{\sum_{i=1}^np_ia_ib_i}{\sum_{i=1}^np_i}\right)\geq\left(\frac{\sum_{i=1}^np_ia_i}{\sum_{i=1}^np_i}\right)\left(\frac{\sum_{i=1}^np_ib_i}{\sum_{i=1}^np_i}\right).\stopformula

  {\rm The proof is similar to the theorem. This is called Chebyshev's inequality with weights.}
\stopremark

\section{Sur\'anyi's Inequality}
\starttheorem
  Let $a_1, a_2, \ldots, a_n$ be non-negative real numbers, and let $n\in P$. Then
  \placeformula\startformula
    (n - 1)(a_1^n + a_2^n + \cdots + a_n^n) + na_1a_2 \cdots a_n\geq (a_1 + a_2 + \cdots + a_n)(a_1^{n-1} + a_2^{n - 1} + \cdots + a_n^{n-1}).
  \stopformula
\stoptheorem

\startproof
  We will prove this by mathematical induction. Due to symmetry and homegeneity of the inequality we may assume $a_1\geq a_2\geq
  \cdots \geq a_n$ and $a_1 + a_2 + \cdots + a_n = 1$. For $n =1$ equality occurs. Let us assume that for $n = 1$ the inequality
  holds i.e. \startformula (k - 1)(a_1^k + a_2^k + \cdots + a_k^k) + ka_1a_2\cdots a_k\geq a_1^{k-1} + a_2^{k - 1} + \cdots + a_k^{k - 1}.\stopformula
  We need to prove that:
  \startformula k\sum_{i=1}^{k+1}a_i^{k+1} + (k+1)\prod_{i=1}^{k+1}a_i - (1 + a_{k+1})\sum_{i=1}^{k+1}a_i^k\geq 0.\stopformula
  Hence
  \startformula ka_{k+1}\prod_{i=1}^ka_i\geq a_{k+1}\sum_{i=1}^ka_i^{k-1} - (k-1)a_{k+1}\sum_{i=1}^ka_i^k.\stopformula
  Using this last inequality, it remains to prove that:
  \startformula \left(k\sum_{i=1}^{k+1}a_i^{k+1} - \sum_{i=1}^ka_i^k\right) - a_{k+1}\left(k\sum_{i=1}^ka_i^k - \sum_{i=1}^ka_i^{k - 1}\right)
  + a_{k+1}\left(\prod_{i=1}^ka_i + (k - 1)a_{k+1}^k - a_{k+1}^{k-1}\right)\geq 0.\stopformula
  We have \startformula \prod_{i=1}^k a_i + (k - 1)a_{k+1}^k - a_{k+1}^{k-1} = \prod_{i=1}^k(a_i - a_{k +1} + a_{k + 1}) + (k - 1)a_{k+1}^k - a_{k+1}^{k
    - 1}\stopformula
  \startformula \geq a_{k+1}^k + a_{k+1}^{k-1}\sum_{i=1}^k(a_i - a_{k+1}) + (k - 1)a_{k+1}^k - a_{k+1}^{k-1} = 0.\stopformula
  Also
  \startformula \left(k\sum_{i=1}^{k+1}a_i^{k+1} - \sum_{i=1}^ka_i^k\right) - a_{k+1}\left(k\sum_{i=1}^ka_i^k - \sum_{i=1}^ka_i^{k - 1}\right)
  \geq 0\stopformula
  \startformula \Rightarrow k\sum_{i=1}^ka_i^{k+1} - \sum_{i=1}^ka_i^k\geq a_{k+1}\left(k\sum_{i=1}^ka_i^k - \sum_{i=1}^ka_i^{k - 1}\right)\stopformula
  By Chebyshev's inequality, we have
  \startformula k\sum_{i=1}^ka_i^k\geq \sum_{i=1}^ka_i\sum_{i=1}^ka_i^{k-1} = \sum_{i=1}^ka_i^{k-1}\stopformula
  \startformula \Rightarrow k\sum_{i=1}^ka_i^k - \sum_{i=1}^ka_i^{k-1}\geq 0.\stopformula
  and since $a_1 + a_2 + \cdots + a_{k + 1} = 1$, by the assumption $a_1\geq a_2 \geq\cdots\geq a_{k +1}$, we deduce that \startformula a_{k +
    1}\leq \frac{1}{k}\stopformula
  So it is enough to prove that
  \startformula k\sum_{i=1}^ka_i^{k+1} - \sum_{i=1}^k a_i^k\geq \frac{1}{k}\left(k\sum_{i=1}^ka_i^k - \sum_{i=1}^ka_i^{k - 1}\right).\stopformula
  which is equivalent to
  \startformula k\sum_{i=1}^ka_i^{k+1} + \frac{1}{k}\sum_{i=1}^ka_i^{k-1}\geq 2\sum_{i=1}^ka_i^k\stopformula
  Since AM $\geq$ GM we have that
  \startformula ka_i^{k+1} + \frac{1}{k}a_i^{k - 1}\geq 2a_i^k\;\forall\;i\stopformula
  Adding this inequality for $i=1, 2, \ldots, k$ we obtain the required inequality.
\stopproof

\section{Rearrangement Inequality}
\starttheorem
  Let $a_1\leq a_2\leq\cdots\leq a_n$ and $b_1\leq b_2\leq\cdots\leq b_n$ (or $a_1\geq a_2\geq\cdots\geq a_n$ and $b_1\geq
  b_2\geq\cdots\geq b_n$) be real numbers. If $a_1', a_2', \ldots, a_n'$ is any permutation of $a_1, a_2, \ldots, a_n$ then the
  equality
  \placeformula\startformula
    \sum_{i=1}^na_ib_{n+1-i}\leq \sum_{i=1}^na_i;b_i\leq\sum_{i=1}^na_ib_i,
  \stopformula
  holds. Thus the sum $\displaystyle\sum_{i=1}^na_ib_i$ is maximum when the two sequences $\langle a_i\rangle$ and $\langle
  b_i\rangle$ are oredered similarly. And the sum is minimum when these are ordered in opposite manner.
\stoptheorem

\startproof
  We start by assuming that both $a_i$'s and $b_i$'s are non-decreasing. Suppose $\langle a_i'\rangle\neq\langle a_i\rangle$. Let
  $r$ be the largest index such that $a_r'\neq a_r$ i.e. $a_r'\neq a_r$ and $a_i'=a_i$ for $r<i\leq n$. This implies that $a_r'$ is
  from the set $\{a_1, a_2, \ldots, a_{r - 1}\}$ and $a_r'<a_r$. Further this also shows that $a_1', a_2', \ldots, a_r'$ is a
  permutation of $a_1, a_2, \ldots, a_r$. Thus we can find indices $k<r$ and $l<r$ such that $a_k' = a_r$ and $a_r' = a_l$. It
  follows that \startformula a_k' - a_r' = a_r - a_l\geq 0,\;b_r - b_k\geq 0\stopformula
  We now interchange $a_r'$ and $a_k'$ to get a permutation of $a_1'', a_2'', \ldots, a_n''$ of $a_1', a_2', \ldots, a_n'$; thus
  \startformula \startmathcases\NC a_i''=a_i', \NC \text{if}i\neq r, k\NR\NC a_r''=a_k'=a_r,\\a_k''=a_r'=a_l\NR\stopmathcases\stopformula
  Consider the sums
  \startformula S'' = a_1''b_1 + a_2''b_2 + \cdots + a_n''b_n, \;S' = a_1'b_1 + a_2'b_2 + \cdots + a_n'b_n,\stopformula
  and the difference $S'' - S':$
  \startformula \startalign\NC S'' - S' \NC = \sum_{i=1}^n(a_i'' - a_i')b_i\NR\NC \NC = (a_k'' - a_k') + (a_r'' - a_r')b_r\NR\NC\NC&=(a_r' - a_k')b_k + (a_k' -
    a_r')b_r\NR\NC\NC = (a_k' - a_r')(b_r - b_k).\stopalign\stopformula
  $\because a_k' - a_r' \geq 0$ and $b_r - b_k \geq 0$, we can say that $S''\geq S'$. We observe that the permutations $a_1'', a_2'',
  \ldots, a_n''$ of $a_1', a_2', \ldots, a_n'$ has th eproperty that $a_i'' = a_i = a_i$ for $r< i\leq n$ and $a_r'' = a_k' =
  a_r$. Hence the permutation $\langle a_i''\rangle$ in place of $\langle a_i'\rangle$ may be considered and the steps can be
  continued like above. After at most $n - 1$ such steps, we will arrive at the original permutation $\langle a_i\rangle$ from
  $\langle a_i'\rangle$. At each step the corresponding sum has the same order as $a_i$'s i.e. non-decreasing. Thus,
  \placeformula[eq:rearrangement:2]\startformula
    a_1'b_1 + a_2'b_2 + \cdots + a_n'b_n\leq a_1b_1 + a_2b_2 + \cdots + a_nb_n
  \stopformula
  For the other part, let us put $c_i = a_{n + 1 - i}', d_i = -b_{n + 1 - i}$. Then $c_1, c_2, \ldots, c_n$ is a permutation of
  $a_1, a_2, \ldots, a_n$ and $d_1\leq d_2\leq \cdots\leq d_n$. Using the inequality (\in{Equation}[eq:rearrangement:2]) for the sequences
  $\langle c_i\rangle$ and $\langle d_i\rangle$, we get
  \startformula c_1d_1 + c_2d_2 + \cdots + c_nd_n\leq a_1d_2 + a_2d_2 + \cdots + a_nd_n.\stopformula
  Thus,
  \startformula -\sum_{i=1}^na_{n+1 - i}'b_{n+1-i}\leq -\sum_{n=1}^na_ib_{n+1-i}.\stopformula
  Thus, \placeformula[rearrangement:3]\startformula a_1'b_1 + a_2'b_2 + \cdots + a_n'b_n\geq a_1b_1 + a_2b_{n - 1} + \cdots +
    a_nb_1,\stopformula
  which is the other part of the inequality.

  For the equality, we consider pairs $k, l$ with $1\leq k < l\leq n$, either $a+k' = a_l'$ or $a_k'>a_l'$ and $b_k = b_l$, then
  the equality holds for (\in{Equation}{rearrangement:2}). For (\in{Equation}[rearrangement:3]), for each $k, l$ with $1\leq k < l \leq n$, either
  $a_{n + 1 - k}' \geq a_{n + 1 - l}'$ and $b_{n + 1 - k} = b_{n  + 1 - l}$.
\stopproof

\startcorollary
  Let $\alpha_1, \alpha_2, \ldots, \alpha_n$ be real numbers and $\beta_1, \beta_2, \ldots, \beta_n$ be a permutation of $\alpha_1,
  \alpha_2, \ldots, \alpha_n$. Then \startformula \sum_{i=1}^n\alpha_i\beta_1\leq \sum_{i=1}^n\alpha_i^2.\stopformula The equality holds if and only if
  $\langle\alpha_i\rangle = \langle\beta_i\rangle$.
\stopcorollary

\startproof
  Let $\alpha_1', \alpha_2', \ldots, \alpha_n'$ be a permutation of $\alpha_1, \alpha_2, \ldots, \alpha_n$ such that
  $\alpha_1'\leq\alpha_2'\leq\ldots\leq\alpha_n'$. Then we can find a bijections $\sigma$ of $\{1,2,\ldots,n\}$ onto itself such
  that $\alpha_i' = \alpha_{\sigma(i)}, 1\leq j\leq n$; i.e. $\sigma$ is a permutation on the set $\{1,2,\ldots, n\}$. Let
  $\beta_i' = \beta_{\sigma(i)}$. Then $\beta_1', \beta_2', \ldots, \beta_n'$ is a permutation of
  $\alpha_1'\leq\alpha_2'\leq\ldots\leq\alpha_n'$. Applying the rearrangement inequality to
  $\alpha_1'\leq\alpha_2'\leq\ldots\leq\alpha_n'$ and $\beta_1', \beta_2', \ldots, \beta_n'$, we get
  \startformula \sum_{i=1}^n\alpha_i'\beta_i'\leq\sum_{i=1}^n(\alpha_i')^2 = \sum_{i=1}^n\alpha_i^2.\stopformula
  We also have \startformula \sum_{i=1}^n\alpha_i'\beta_i' = \sum_{i=1}^n\alpha_{\sigma(i)}\beta_{\sigma(i)} = \sum_{i=1}^n\alpha_i\beta_i,\stopformula
  because $\sigma$ is a bijection on $\{1,2,\ldots,n\}$. Thus,
  \startformula \sum_{i=1}^n\alpha_i\beta_i\leq\sum_{i=1}^n\alpha_i^2.\stopformula

  Say that equality holds and $\langle\alpha_i\rangle\neq\langle\beta_i\rangle$. Then
  $\langle\alpha_i'\rangle\neq\langle\beta_i'\rangle$. Let $k$ be the largest index such that $\alpha_k'\neq\beta_k'$ for $k <
  i\neq n$. Let $m$ be the least integer such that $\alpha_k' = \beta_m'$. If $m>k$, then $\beta_m' = \alpha_k'$ and hence
  $\alpha_k' = \alpha_m'$. This implies that $\alpha_k' = \alpha_{k+1}' = \cdots = \alpha_m'$ and hence $\beta_{k+1}' = \cdots =
  \beta_m'$. We now have an $m_1 > m$ such that $\alpha_k' = \beta_{m_1}'$. Using $m_1$ as pivot, we get $\alpha_k' = \alpha_{k+1}'
  = \cdots = \alpha_m' = \cdots = \alpha_m'$ and $\beta_{k+1}' = \cdots = \beta_m' = \cdots = \beta_{m_1}'$. It can be concluded
  that $\alpha_k' = \beta_l'$ for some $l<k$, thus forcing $m < k$.

  Clearly $\beta_m'\neq\beta_k'$ by our choice of $k$. We know that equality holds if and only if for any two indexes $r\neq s$,
  either $\alpha_r' = \alpha_s'$ or $\beta_r' = \beta_s'$. Since $\beta_m'\neq\beta_k'$, we must have $\alpha_m' = \alpha_k'$. But
  then we have $\alpha_m' = \alpha_{m+1}' = \cdots = \alpha_k'$. From the minimality of $m$, we see that $k - m + 1$ equal elements
  $\alpha_m', \alpha_{m+1}', \ldots, \alpha_k'$ must be among $\beta_m', \beta_{m+1}', \ldots, \beta_n'$ and since
  $\beta_k'\neq\alpha_k'$, we must have $\alpha_k' = \beta_l'$ for some $l > k$. But then using $\beta_l' = \alpha_l'$, we have
  \startformula \alpha_m' = \alpha_{m+1}' = \cdots = \alpha_k' = \cdots = \alpha_l'.\stopformula
  Thus the number of equal elements gets enlarged to $l - m + 1 > k -m + 1$. Since this process cannot be continues indefinitely,
  we conclude that $\langle\alpha_i'\rangle = \langle\beta_i'\rangle$ which will be followed
  by $\langle\alpha_i\rangle\neq\langle\beta_i\rangle$.
\stopproof

\startcorollary
  Let $\alpha_1, \alpha_2, \ldots, \alpha_n$ be positive real numbers and let $\beta_1, \beta_2, \ldots, \beta_n$ be a permutation
  of $\alpha_1, \alpha_2, \ldots, \alpha_n$. Then \startformula \sum_{i=1}^n\frac{\beta_i}{\alpha_i}\geq n.\stopformula
  Equality holds if and only if $\langle\alpha_i\rangle\neq\langle\beta_i\rangle$.
\stopcorollary
\startproof
  Let $\alpha_1', \alpha_2', \ldots, \alpha_n'$ be a permutation of $\alpha_1, \alpha_2, \ldots, \alpha_n$ suhc that
  $\alpha_1'\leq\alpha_2'\leq\ldots\alpha_n'$. Like in previous corollary, we can find a permutation $\sigma$ of $\{1,2,\ldots,
  n\}$ such that $\alpha_i' = \alpha_{\sigma(i)}$ for $1\leq i\leq n$. We defien $\beta_i' = \beta_{\sigma(i)}$. Then
  $\langle\beta_i'\rangle$ is a permutation of $\langle\alpha_i'\rangle$. Using the rearrangement theorem, we get
  \startformula \sum_{i=1}^n\beta_i'\left(-\frac{1}{\alpha_i'}\right)\leq\sum_{i=1}^n\alpha_i'\left(-\frac{1}{\alpha_i'}\right) = -n.\stopformula
  Thus, we have the desired inequality. Like previous case we camn derive the equality.
\stopproof

\section{Young's Inequality}
\starttheorem
  If $p\in [1,\infty)$ and $q = p/(p - 1)$. $q\in [1, \infty]$ and $\dfrac{1}{p} + \frac{1}{q} = 1$. If $a, b > 0$,
    then
    \placeformula\startformula
      \frac{a^p}{p} + \frac{b^q}{q}\geq ab
    \stopformula
\stoptheorem

\startproof
  Taking $\log$ of L.H.S. $\log\left(\frac{a^p}{p} + \frac{b^q}{q}\right)$

  Notice that, since $\dfrac{1}{p} + \frac{1}{q} = 1$, so the L.H.S. is just a convext combination of $a^p$ and $b^q$. Since $\log
  x$ is a concave function, we have
  \startformula \log\left(\frac{a^p}{p} + \frac{b^q}{q}\right)\geq \dfrac{\log a^p}{p} + \dfrac{b^q}{q} = \log a + \log b = \log(ab).\stopformula
  Hence, the inequality is proved(since $\log x$ is strictly increasing).

  {\it Alternative Proof.}

  Using generalized AM-GM inequality, \startformula \dfrac{x^p}{p} + \frac{b^q}{q}\geq \left[(x^p)^{1/p}(y^q)^{1/q}\right] = xy.\stopformula
\stopproof

\section{H\"{o}lder's Inequality}
\starttheorem
  Let $a_1, a_2, \ldots, a_n, b_1, b_2, \ldots, b_n$ be real numbers and $p, q$ be two positive real numbers such that $\frac{1}{p}
  + \frac{1}{q} = 1.$ (Such a pair of indices is called a pair of conjugate indices.) Then the inequality holds
  \placeformula\startformula
    \left|\sum_{i=1}^na_ib_i\right|\leq\left(\sum_{i=1}^n|a_i|^p\right)^{1/p}\left(\sum_{i=1}^n|b_i|^q\right)^{1/q}
  \stopformula
holds. Equality holds if and only if $|a_i|^p = c|b_i|^q, 1\leq i\leq n$, for some real constant $c$.
\stoptheorem

\startproof
  Following Young's inequality, consider \startformula x = \frac{|a_k|}{\left(\sum_{i=1}^n|a_i|^p\right)^{1/p}}, y =
  \frac{|b_k|}{\left(\sum_{i=1}^n|b_i|^q\right)^{1/q}}\stopformula
  so we get \startformula \frac{|a_k|^p}{p\left(\sum_{i=1}^n|a_i|^p\right)} +
  \frac{|b_k|^q}{q\left(\sum_{i=1}^n|b_i|^q\right)}\geq
  \frac{|a_k||b_k|}{\left(\sum_{i=1}^n|a_i|^p\right)^{1/p}\left(\sum_{i=1}^n|b_i|^q\right)^{1/q}}\stopformula
  Now summing over $k$, we obtain
  \startformula \frac{1}{p} + \frac{1}{q}\geq
  \frac{\sum_{i=1}^n|a_kb_k|}{\left(\sum_{i=1}^n|a_i|^p\right)^{1/p}\left(\sum_{i=1}^n|b_i|^q\right)^{1/q}}\stopformula
  Thus, we have
  \startformula \sum_{i=1}^n|a_kb_k|\leq \left(\sum_{i=1}^n|a_i|^p\right)^{1/p}\left(\sum_{i=1}^n|b_i|^q\right)^{1/q}.\stopformula
  It is now trivial to prove the condition for equality.
\stopproof

\startremark
  If we take $p = q = 3$, H\"{o}lder's inequality reduces to the Cauchy-Schwarz inequality.
\stopremark
\startremark
  If either of $p$ and $q$ is negativem H\"{o}lder's inequality is reversed.
\stopremark
\startremark
  H\"{o}lder's inequality can have a version with weights. In addition to what we have, we also consider consider weights $w_1,
  w_2, \ldots, w_n$ then following equality holds
  \startformula \sum_{i=1}^nw_i|a_ib_i|\leq\left(\sum_{i=1}^nw_i|a_i|^p\right)^{1/p}\left(\sum_{i=1}^nw_i|b_i|^q\right)^{1/q}\stopformula
\stopremark

Given below is generalized H\"{o}lder's inequaltiy and the proof is similar like above.
\starttheorem
  Let $a_{ij}, i=1, 2, \ldots, m;j = 1, 2, \ldots, n$, be positive humbers and $\alpha_1, \alpha_2, \ldots, \alpha_n$ be positive
  real numbers such that $\alpha_1 + \alpha_2 + \cdots + \alpha_n = 1$. Then
  \placeformula\startformula
    \sum_{i=1}^m\left(\prod_{j=1}^na_{ij}a_{ij}^{\alpha_j}\right)\leq \prod_{j=1}^n\left(\sum_{i=1}^ma_{ij}\right)^{\alpha_j}.
  \stopformula
\stoptheorem

\section{Minkowski's Inequality}
\starttheorem
  Let $p\geq 1$ be a real number and $a_1, a_2, \ldots, a_n, b_1, b_2, \ldots, b_n$ be real numbers. Then
  \placeformula[eq:10.7]\startformula
    \left(\sum_{i=1}^n|ai + b_i|^p\right)^{1/p}\leq\left(\sum_{i=1}^n|a_i|^p\right)^{1/p} + \left(\sum_{i=1}^n|b_i|^p\right)^{1/p}
  \stopformula
  Here equality holds if and only if $a_i=\lambda b_i$ for some constant $\lambda, 1\leq i\leq n$.
\stoptheorem

\startproof
  We assume that $p > 1$, because the result is clear for $p = 1$. Observe the following:
  \startformula \sum_{i=1}^n|ai + b_i|^p = \sum_{i=1}^n|a_i + b_i|^{p - 1}|a_i + b_i|\leq \sum_{i=1}^n|a_i + b_i|^{p - 1}|a_i| +
  \sum_{i=1}^n|a_i + b_i|^{p - 1}|b_i|.\stopformula
  Let $q$ be the conjugate index of $p$. Using H\"{o}lder's inequaity to each sum on the right hand side, we have
  \startformula \sum_{i=1}^n|a_i + b_i|^{p - 1}|a_i|\leq \left(\sum_{i=1}^n|a_i|^p\right)^{1/p}\left(\sum_{i=1}^n|a_i + b_i|^{(p -
    1)q}\right)^{1/q}.\stopformula
  Since $p, q$ are conjugate indexes, we get $(p - 1)q = p$. It follows that
  \startformula \sum_{i=1}^n|a_i + b_i|^{p - 1}|a_i|\leq\left(\sum_{i=1}^n|a_i|^p\right)^{1/p}\left(\sum_{i=1}^n|a_i + b_i|^p\right)^{1/q}.\stopformula
  Similarly,
  \startformula \sum_{i=1}^n|a_i + b_i|^{p - 1}|b_i|\leq\left(\sum_{i=1}^n|b_i|^p\right)^{1/p}\left(\sum_{i=1}^n|a_i + b_i|^p\right)^{1/q}.\stopformula
  It now follows that
  \startformula \sum_{i=1}^n|a_i + b_i|^p\leq\left[\left(\sum_{i=1}^n|a_i|^p\right)^{1/p} +
    \left(\sum_{i=1}^n|b_i|^p\right)^{1/p}\right]\left(\sum_{i=1}^n|a_i + b_i|^p\right)^{1/q}.\stopformula
  If we use $1 - (1/1) = 1/p$, we finally get the required inequaity.

  Like H\"{o}lder's inequality the equality can be proven for this using the same conditions.
\stopproof

\startremark
\stopremark For $0<p<1$, the inequality (\in{Equation}[eq:10.7]) gets reversed.

\section{Convex and Concave Functions}
Most of the inequalities discussed so far are consequencce of inequalities for a special class of functions, known as
{\it convex} and {\it concave} functions. Consider the function $f(x) = x^n\;\forall\;n>1$ defined on $\mathbb{R}$. Consider
the case of $n=2$, then on the graphs of this function, the chord joining any two points always lies above the graph. In fact
taking $a<b$, and the point $ka + (1 - k)b$ between $a$ and $b$, we see that
\startformula [ka + (1 - k)b]^2 - ka^2 - (1 - k)b^2 = -k(1 - k)(a - b)^2\leq 0.\stopformula
Thus,
\startformula f(ka + (1 - k)b)\leq kf(a) + (1 - k)f(b).\stopformula
This property is the defining property of a convex function. The family of convex functions obey a class of inequalities known as
Jensen's inequality.

Let $I$ be an interval in $\mathbb{R}$. A function $f:I\rightarrow\mathbb{R}$ is said to be convext if for all $x, y$ in $I$ and
$k$ in the interval $[0, 1]$, the following inequality holds:
\placeformula[eq:convex]\startformula
  f(kx + (1 - k)y)\leq kf(x) + (1 - k)f(y).
\stopformula

If the inequality is strict for all $x\neq y$, $f$ is said to be strictly convex on $I$. If the inequality is reverse for same
conditions then $f$ is said to be concave and similarly for strictly concave $f$.

There are other equivalent properties of a convex function. Let $x_1, x_2, x_3$ are in $I$ such that $x_1<x_2<x_3$ and we take $k =
\frac{x_3 - x_2}{x_3 - x_1}$ which gives us
\startformula 1 - k = \frac{x_2 - x_1}{x_3 - x_1},\;\text{and}\;x_2 = kx_1 + (1 - k)x_3.\stopformula
We have \startformula \startalign\NC f(x_2) \NC= f(kx_1 + (1 - k)x_3)\NR\NC\NC\leq kf(x_1) + (1 - k)f(x_3)\NR\NC\NC=\frac{x_3 - x_2}{x_3 - x_1}f(x_1) +
  \frac{x_2 - x_1}{x_3 - x_1}f(x_3).\stopalign\stopformula
We can write this as \startformula f\frac{f(x_1) - f(x_2)}{x_1 - x_2}\leq\frac{f(x_2) - f(x_3)}{x_2 - x_3},\stopformula
for all $x_1<x_2<x_3$ in $I$. We can also write this as:
\startformula \frac{f(x_1)}{(x_1 - x_2)(x_1 - x_3)} + \frac{f(x_2)}{(x_2 - x_1)(x_2 - x_3)} + \frac{f(x_3)}{(x_3 - x_1)(x_3 - x_2)}\geq 0.\stopformula
Consider $z_1 = (a, f(a))$ and $z_2 = (b, f(b))$ as two points on $f$. The equation of line joining these two points is given by
\startformula g(x) = f(a) + \frac{f(b) - f(a)}{b - z}(x - a).\stopformula
Any point between $a$ and $b$ is of the form $x = ka + (1 - k)b$. Thus,
\startformula \startalign\NC g(x) \NC = g(ka + (1 - k)b)\NR\NC\NC = f(a) + \frac{f(b) - f(a)}{b - a}(ka + (1 - k)b - a)\NR\NC\NC = f(a) + (1 - k)[f(b) -
    f(a)]\NR\NC\NC = kf(a) + (1 - k)f(b)\NR\NC\NC \geq f(ka + (1 - k)b) = f(x)\stopalign\stopformula
Thus, $(x, g(x))$ lies above $(x, f(x))$, a point on $f$.

We can look at this in another way. A subset $E$ of the plane $\mathbb{R}^2$ is said to be convex if for every pair of points $z_1$
and $z_2$ in $E$, the line joining $z_1$ and $z_2$ lies entirely in $E$. With every function $f: I\rightarrow\mathbb{R}$, we
associate a subset of $\mathbb{R}^2$ by \startformula E(f) = \{(x, y):a\leq x\leq b, f(x)\leq y\}.\stopformula

\starttheorem
  The function $f: I\rightarrow\mathbb{R}$ is convex if and only if $E(f)$ is a convex subset of $\mathbb{R}^2$.
\stoptheorem

\startproof
  Let $f$ be convex. Let $z_1 = (x_1, y_1)$ and $z_2 = (x_2, y_2)$ be two points of $E(f)$. Consider any point on the line zoining
  $z_1$ and $z_2$. Then,
  \startformula \startalign\NC z \NC = kz_1 + (1 - k)z_2\NR\NC\NC = (kx_1 + (1 - k)x_2, ky_1 + (1 - k)y_2)\stopalign\stopformula  for some $k\in[0, 1]$. We see
  that $a\leq kx_1 + (1 - k)x_2\leq b$. Moreover,
  \startformula \startalign\NC f(kx_1 + (1 - k)x_2)\NC\leq kf(x_1) + (1 - k)f(x_2)\NR\NC\NC\leq ky_1 + (1 - k)y_2.\stopalign\stopformula
  Thus it follows that $z\in E(f)$, proving that $E(f)$ is convex.

  Conversely let $E(f)$ be convex. Let $x_1, x_2$ be two points in $I$ and let $z_1 = (x_1,f(x_1))$ and $z_2 = (x_2, f(x_2))$. Then
  $z_1$ and $z_2$ are in $E(f)$. By conexity of $E(f)$, the point $kz_1 + (1 - k)z_2$ also lies in $E(f)$ for each $k\in[0,
    1]$. Thus,
  \startformula (kx_1 + (1 - k)x_2, kf(x_1) + (1 - k)f(x_1))\in E(f)\stopformula
  The definition of $E(f)$ shows that \startformula f(kx_1 + (1 - k)x_2)\leq kf(x_1) + (1 - k)f(x_2).\stopformula

  This shows that $f$ is convex on the interval $I$.
\stopproof

Following theorem gives description about slope of a function's graph.

\starttheorem
  Let $f:I\rightarrow\mathbb{R}$ be a convex function and $a\in I$ be a fixed point. Define a function $P:I\setminus\{a\}\rightarrow
  \mathbb{R}$ by \startformula P(x) = \frac{f(x) - f(a)}{x - a}.\stopformula  Then $P$ is a non-decreasing function on $I\setminus\{a\}$.
\stoptheorem

\startproof
  Let $f$ is convex on $I$ and let $x, y$ be two points in $I, x\neq a, x\neq b$ such that $x<y$. Then exactly one of the three
  possibilities will be possible:
  \startformula a<x<y;\;x<a<y\;x<y<a.\stopformula
  Consider the case $a<x<y$; other cases can be handled similarly. We can write
  \startformula x = \frac{x - a}{y - a}y + \frac{y - x}{y - a}a.\stopformula
  The convexity of $f$ shows that
  \startformula f\left(\frac{x - a}{y - a}y + \frac{y - x}{y - a}a\right)\leq \frac{x - a}{y - a}f(y) + \frac{y - x}{y - a}f(a).\stopformula
  This is equivalent to
  \startformula \frac{f(x) - f(a)}{x - a}\leq \frac{f(y) - f(a)}{y - a}.\stopformula
  Thus $P(x)\leq P(y).$ This shows that $P(x)$ is a non-decreasing function for $x\neq a$.
\stopproof

Interestingly, the converse is also true; if $P(x)$ is a non-decreasing function on $I\setminus\{a\}$ for every $a\in I$, then
$f(x)$ is convex. We fix $x<y$ in $I$ and let $a = kx + (1 - k)y$ where $k\in(0,1)$. (The cases $k = 0$ or $1$ are obvious.) In
this case
\startformula P(x) = \frac{f(x) - f(a)}{x - a} = \frac{f(x) - f(a)}{(1 - k)(x - y)}\stopformula
\startformula P(y) = \frac{f(y) - f(a)}{y - a} = \frac{f(y) - f(a)}{k(y - x)}.\stopformula
The condition $P(x)\leq P(y)$ implies that $f(a)\leq kf(x) + (1 - k)f(y).$ Hence convexity of $f$ is proven.

There is another easy way of deciding wherther a function is convex or concave for twice differentiable functions. If $f$ is convex
on an interval $I$ and if its second derivative exists on $I$, then $f$ is convex(strictly convex) on $I$ if $f''(x)\geq 0(> 0)$ for
all $x\in I$. Similarly $f$ is concave(striclty concave) on $I$ if $f''(x)\leq0(<0)$ for all $x\in I$.

When we defined conex function the inequality involved two points $x, y$; refer to (\in{Equation}[eq:convex]). Jensen's inequaity extends
this to any finite number of points.

\section{Jensen's Inequality}
\starttheorem
  Let $f:I\rightarrow\mathbb{R}$be a convex function. Let $x_1, x_2, \ldots, x_n$ are points in $I$ and $k_1, k_2, \ldots, k_n$ are
  real numbers in the interval $[0, 1]$ such that $k_1 + k_2 + \cdots + k_n = 1$. Then
  \placeformula[eq:jensen]\startformula
    f\left(\sum_{i=1}^nk_ix_i\right)\leq\sum_{i=1}^nk_if(x_i)
  \stopformula
\stoptheorem

\startproof
  We will use induction to prove this. For $n = 2$, this is the definition of a convex function. Suppose the inequality
  (\in{Equarion}[eq:jensen]) is true for all $p<n$; i.e. for $p<n$ if $x_1, x-2, \ldots, x_p$ are $p$ points in $I$ and $k_1, k_2, \ldots,
  k_p$ are real numbers in $[0,1]$ such that $\sum_{i=1}^nk-i = 1$, then
  \startformula f\left(\sum_{i=1}^pk_ix_i\right)\leq \sum_{i=1}^pk+if(x_i).\stopformula
  Now considering the conditions of the theorem,
  \startformula y_1 = \frac{\sum_{i=1}^{n-1}k_ix_i}{\sum_{i=1}^{n-1}k_i},\;y_2= x_n,\;\alpha_1 = \sum_{j=1}^{n- 1}k_i,\;\alpha_2 = k_n.\stopformula
  We observe that $\alpha_2 = 1 - \alpha_1$, and $y_1, y_2$ are in $I$. Using the conexity of $f$, we get
  \startformula \startalign\NC f(\alpha_1y_1 + \alpha_2y_2)\NC = f(\alpha_1y_1 + (1 - \alpha_1)y_2)\NR\NC\NC \leq\alpha_1f(y_1) + (1 -
    \alpha)1)f(y_2)\NR\NC\NC = \alpha_1f(y_1) + \alpha_2f(y_2).\stopalign\stopformula
  However, we have \startformula \alpha_1y_1 + \alpha_2y_2 = \sum_{i=1}^nk_ix_i.\stopformula
  Now we consider $f(y_1)$. If \startformula \mu_i = \frac{k_i}{\sum_{i=1}^{n-1}k_i}, 1\leq l\leq n - 1\stopformula
  then it can be easily verifief that $\sum_{l=1}^{n-1}\mu_l = 1$. Using the induction hypothesis, we get
  \startformula f\left(\sum_{l=1}^{n-1}\mu_lx_l\right)\leq \sum_{l=1}^{n-1}\mu_lf(x_l)\stopformula
  Since \startformula \sum_{l=1}^{n-1}\mu_lx_l = y_1,\stopformula
  we get
  \startformula f(y_1)\leq \frac{\sum_{l=1}^{m-1}k_lf(x_l)}{\sum_{i=1}^{n - 1}k_i} = \frac{\sum_{i=1}^{n-1}f(x_i)}{\alpha_1}\stopformula
  Thus we obtain
  \startformula f\startalign\NC\left(\sum_{i=1}^nk_if(x_i)\right)\NC\leq\alpha_1\left(\frac{\sum_{i=1}^{n-1}k_if(x_i)}{\sum_{i=1}^{n-1}k_i}\right) + k_nf(x_n)\NR\NC\NC = \sum_{i=1}^nk_if(x_i).\stopalign\stopformula
  Thus, the theorem is proved by induction.
\stopproof

\startremark
  If $f:I\rightarrow\mathbb{R}$ is concave, then the inequality (\in{Equarion}[eq:jensen]) gets reversed. If $x_1, x_2, \ldots, x_n$ are
  points in $I$ and $k_1, k_2, \ldots, k_n$ are real numbers in the interval $[0, 1]$, such that $k_1 + k_2 + \cdots + k_n = 1$,
  then following inequality holds:
  \placeformula\startformula
    f\left(\sum_{i=1}^nk_ix_i\right)\geq\sum_{i=1}^nk_if(x_i)
  \stopformula
\stopremark

\startremark
  Using the concavity of $f(x) = \ln x$ on $(0, \infty)$, the AM-GM inequality can be proved. If $x_1, x_2, \ldots, x_n$ are points
  in $(0, \infty)$ and $k_1, k_2, \ldots, k_n$ are real numbers in the interval $[0,1]$ such that $k_1 + k_2 + \ldots + k_n = 1$,
  then we have
  \startformula \ln\left(\sum_{i=1}^nk_ix_i\right)\geq\sum_{i=1}^nk_i\ln(x_i)\stopformula
\stopremark

\startproof
  Taking $k_i = \frac{1}{n}$ for all $i$,
  \startformula \ln\left(\sum_{i=1}^n\frac{x_i}{n}\right)\geq\frac{1}{n}\sum_{i=1}^n\ln x_i = \sum_{i=1}^n\ln\left(x_i^{1/n}\right).\stopformula
  Using the fact that $g(x) = e^x = {\mathrm{exp}}(x)$ is strictly increasing on the interval $(-\infty, \infty)$, this leads to
  \startformula \startalign\NC\frac{1}{n}\sum_{i=1}^nx_i \NC \geq{\mathrm{exp}}\left(\sum_{i=1}^n\ln\left(x_i^{1/n}\right)\right)\NR\NC\NC
    =\prod_{i=1}^n\mathrm{exp}\left(\ln\left(x_i^{1/n}\right)\right)\NR\NC\NC = \left(x_1x_2\ldots x_n\right)^{1/n}.\stopalign\stopformula

  We can also prove generalized AM-GM inequality with this method.
  \startformula \ln\left(\sum_{i=1}^nk_ix_i\geq\sum_{i=1}^nk_i\ln(x_i)\right) = \sum_{i=1}^n\ln x_i^{k_i},\stopformula
  Taking antilog
  \startformula \sum_{i=1}^nk_ix_i\geq\prod_{i=1}^nx_i^{k_i}.\stopformula
  Nowfor any $n$ positive real numbers $\alpha_1, \alpha_2, \ldots, \alpha_n$, consider
  \startformula k_i = \frac{\alpha_i}{\sum_{j=1}^n\alpha_j}\stopformula
  Observe that $k_i$ are in $[0, 1]$ and $\sum_{i=1}^nk_i = 1$. These choices of $k_i$ give
  \startformula \frac{\alpha_1x_1 + \alpha_2x_2 + \cdots + \alpha_nx_n}{\alpha_1 + \alpha_2 + \cdots +
    \alpha_n}\geq\left(x_1^{\alpha_1}x_2^{\alpha_2}\ldots x_n^{\alpha)2}\right)^{1/(\alpha_1 + \alpha_2 + \cdots + \alpha_n)},\stopformula
  which is our generalized AM-GM inequality.
\stopproof

\startremark
  Function $f(x) = x^p$ can be used to prove H\"{o}lder's inequality. We know that $f(x) = x^p$ is convex for $p\geq 1$ and concave
  for $0,p<1$ for $p\in(0, \infty)$. Let $x_1, x_2, \ldots, x_n$ be real numbers and $k_1, k_2, \ldots, k_n$ in $[0, 1]$, then we
  have
  \startformula \left(\sum_{i=1}^nk_ix_i\right)^p\leq \sum_{i=1}^nk_ix_i^p\;\text{for}\;p\geq 1\stopformula
  and
  \startformula \left(\sum_{i=1}^nk_ix_i\right)^p\geq \sum_{i=1}^nk_ix_i^p\;\text{for}\;0<p<1.\stopformula
\stopremark

\startproof
  Let $a_1, a_2, \ldots, a_n, b_1, b_2, \ldots, b_n$ be real numbers and $p>1$ and $q$ be conjugate numbers. Thus, $\frac{1}{p} +
  \frac{1}{q} = 1.$ We need to assume that $b_i\neq 0$ for all $i$; else we may delete all those $b_i$ which are zero without
  having an effect on the equality. Let
  \startformula t = \sum_{i=1}^n|b_i|^q,\;k_j = \frac{|b_j|^q}{t},\;x_j = \frac{|a_j|}{|b_j|^{q - 1}}\stopformula
  We have $k_j\in[0, 1]$ and $k_1 + k_2 + \ldots + k_n = 1$. Using the conexity of $x^p$, we have
  \startformula \left(\sum_{i=1}^nk_ix_i\right)^p\leq\sum_{i=1}^nk_ix_i^p,\stopformula
  which implies that
  \startformula \left(\sum_{j=1}^n\frac{|b_j|^q}{t}\frac{|a_j|}{|b_j|^{q - 1}}\right)^p \leq
  \sum_{j=1}^n\frac{|b_j|^1}{t}\frac{|a_j|^p}{|b_j|^{(q - 1)p}} = \frac{1}{t}\sum_{j=1}^n|a_j|^p.\stopformula
  Futher simplification yields
  \startformula \sum_{j=1}^n|a_jb_j|\leq\left(\sum_{j=1}^n|a_j|^p\right)^{1/p}t^{1 - (1/p)} =
  \left(\sum_{j=1}^n|a_j|^p\right)^{1/p}\left(\sum_{j=1}^n|b_j|^q\right)^{1/q}\stopformula
  For concave case the inequality is simply reversed.
\stopproof

\starttheorem
  Let $f:I\rightarrow\mathbb{R}$ be a convex function; $a_1\leq a_2\leq\cdots\leq a_n, b_1, b_2, \ldots, b_n$ are real numbers in
  $I$ such that $a_1 + b_1\in I$ and $a_n + b_n\in I$. Let $a_1', a_2', \ldots, a_n'$ be a permutation of $a_1, a_2, \ldots,
  a_n$. Then the follwoing inequality is true: \startformula \sum_{i=1}^nf(a_i + b_{n+1-i})\leq\sum_{i=1}^nf(a_i' + b_i)\leq\sum_{i=1}^nf(a_i +
  b_i).\stopformula
\stoptheorem

\startproof
  We will use the proof of rearrangement inequality. Assume $\langle a_i'\rangle \neq\langle a_i\rangle$ and $r$ be the largest
  index such that $a_r'\neq a_r$. Since $a_i=a_i'$ for $r<i\leq n$, we see that $a_1', a_2', \ldots, a_r'$ is a permutation of
  $(a_1, a_2, \ldots, a_r)$. Thus we can find $k<r, l<r$ such that $a_k' = a_r$ and $a_r' = a_l$. We deduce that $a_k' - a_r' = a_r
  - a_l\geq 0$ and $b_r - b_k\geq 0$. Interchanging $a_r'$ and $a_k'$ to get a permutation $(a_1'', a_2'', \ldots, a_n'')$ of $(a_1',
  a_2', \ldots, a_n')$. Thus
  \startformula a_i''= a_i'\;\text{{\rm{for}}}\; j\neq r,k,\;a_r''=a_k'=ar,\;a_k''=a_r'=a_l.\stopformula
  Let \startformula S'' = \sum_{i=1}^nf(a_i'' + b_i),\;S' = \sum_{i=1}^nf(a_i' + b_i).\stopformula
  Then, \startformula \startalign\NC S'' - S' \NC = f(a_r'' + b_r) + f(a_k'' + b_k) - f(a_r' + b_r) - f(a_k' + b_k)\NR\NC\NC = f(a_r + b_r) + f(a_l +
    b_k) - f(a_l + b_r) - f(a_r + b_k).\stopalign\stopformula
  We notice that \startformula a_l + b_k<a_r + b_k\;\text{{\rm{and}}}\;a_l + b_r < a_r + b_r.\stopformula
  These give \startformula a_l + b_k < a_r + b_k \leq a_r + br,\;a_l + b_k\leq a_l + b_r < a_r + b_r.\stopformula
  If $x_1, x_2, x_3$ are in $I$, then the convexity of $f$ implies that \startformula (x_3 - x_1)f(x_2)\leq(x_3 - x_2)f(x_1) + (x_2 -
  x_1)f(x_3).\stopformula
  Putting $x_1 = a_l + b_k, x_2 = a_r + b_k$ and $x_3 = a_r + b_r$, we get
  \startformula (a_r + b_r - a_l - b_k)f(a_r + b_k)\leq (b_r - b_k)f(a_l + b_k) + (a_r - a_l)f(a_r + b_r).\stopformula
  Similarly putting $x_1 = a_l + b_k, x_2= a_l + b_r$ and $x_3 = a_r + b_r$, we get
  \startformula (a_r + b_r - a_l -b_k)f(a_l + b_r)\leq(a_r - a_l)f(a_l + b_k) + (b_r - b_k)f(a_r + b_r).\stopformula
  Adding, we get
  \startformula (a_r + b_r - a_l - b_k)\{f(a_r + b_k) + f(a_l + b_r)\}\leq (a_r + b_r - a_l - b_k)\{f(a_l + b_k) + f(a_r + b_r)\}.\stopformula
  Since $a_l + b_k < a_r + b_r$, we arrive at
  \startformula f(a_r + b_k) + f(a_l + b_r)\leq f(a_l + b_k) + f(a_r + b_r).\stopformula
  This proves that $S'' - S' \geq 0$.

  Now we observe that the permutation $(a_1'', a_2'', \ldots, a_n'')$ has the property $a_r''=a_r$ and $a_i'' = a_i$, for $r <
  j\leq n$. We may consider the $(a_1'', a_2'', \ldots, a_n'')$ in place $(a_1', a_2', \ldots, a_n')$ and proceed as above. After
  at most $n-1$ steps we arrive at the original numbers $\langle a_i\rangle$ from $\langle a_i'\rangle$ and at each stage the
  corresponding sum in non-decreasing. Thus, finally we arrive at
  \startformula \sum_{i=1}^nf(a_i' + n_i)\leq \sum_{i=1}^nf(a_i + b_i).\stopformula
  For the other inequality we define $c_i = a_{n+1-i}$ so that $c_1\geq c_2\geq\cdots\geq c_n$. We have to show that
  \startformula \sum_{i=1}^nf(a_{n+1-i} + b_i)\leq\sum_{i=1}^nf(a_i' + b_i).\stopformula
  Setting $c_i' = a_i'$, we have
  \startformula \sum_{i=1}^nf(c_i + b_i)\leq\sum_{i=1}^nf(c_i' + b_i),\stopformula
  where $(c_1', c_2', \ldots, c_n')$ is a permutation of $(c_1, c_2, \ldots, c_n)$. We take $\langle c_i'\rangle\neq\langle
  c_i\rangle$ and let $r$ be the smallest index such that $c_r'\neq c_r$. This forces that $c_r'\in\{c_{r+1}, c_{r+2},\ldots,
  c_n\}$ and $c_r'<c_r$. We see that $(c_r', c_{r+1}', \ldots, c_n')$ is a permutation of $(c_r, c_{r+1}, \ldots, c_n)$. We can
  find $k>r, l>r$ such that $c_k' = c_r$ and $c_r' = c_l$. This implies that $c_k' - c_r' = c_r - c_l\geq 0$ and $b_k - b_r\geq
  0$. Now we can interchange $c_r'$ and $c_k'$ to get a permutation $(c_1'', c_2'', \ldots, c_n'')$ of $(c_1', c_2', \ldots,
  c_n')$; thus
  \startformula c_i''=c_i'\;\text{{\rm{for}}}\;i\neq r, k,\;c_r''=c_k' = c_r,\;c_k''=c_r'=c_l.\stopformula
  We compute the difference between
  \startformula S'' = \sum_{i=1}^nf(c_i'' + b_i),\;S'=\sum_{i=1}^nf(c_i' + b_i),\stopformula
  and obtain
  \startformula \startalign\NC S'' - S'\NC = f(c_r'' + b_r) + f(c_k'' + b_k) - f(c_r' + b_r) - f(c_k' + b_k)\NR\NC\NC = f(c_r + b_r) + f(c_l + b_k) -
    f(c_l + b_r) - f(c_r + b_k).\stopalign\stopformula
  We see that
  \startformula c_l + b_r \leq c_l + b_k < c_r + b_k, c_l + b_r\leq c_r + b_r < c_r + b_k.\stopformula
  From the convexity of $f$
  \startformula (c_r + b_k - c_l - b_r)f(c_l + b_k)\leq(c_r - c_l)f(c_l + b_r) + (b_k - b_r)f(c_r + b_k),\stopformula
  and
  \startformula (c_r + b_k - c_l - b_r)f(c_r + b_r)\leq(b_k - b_r)f(c_l + b_r) + (c_r - c_l)f(c_r + b_k).\stopformula
  Adding, we get
  \startformula (c_r + b_k - c_l - b_r)\{f(c_l + b_k) + f(c_r + b_r)\}\leq(c_r + b_k - c_l - b_r)\{f(c_l + b_r) + f(c_r + b_k)\}.\stopformula
  We know that $c_r + b_k - c_l - n_r \neq 0$, so we have
  \startformula f(c_l + b_k) + f(c_r + b_r)\leq f(c_l + b_r) + f(c_r + b_k).\stopformula
  Thus, we see that $S''\leq S'$. We also see that the new sequence $\langle c_i''\rangle$ has the property: $c_r'' = c_r$ and
  $C_i'' = c_i$ for $1\leq i< r$. Now we repeat the above argument by replacing $\langle c_i'\rangle$ with $\langle
  c_i''\rangle$. At each step the sum will never increase. After at most $n-1$ steps we arrive at the sequence $\langle
  c_i\rangle$. Thus, we find that the corresponding sum does not exceed to that of $S'$. Thus we get
  \startformula \sum_{i=1}^nf(c_i + b_i)\leq\sum_{i=1}^nf(c_i' + b_i),\stopformula  which was to be proved.
\stopproof

\section{Bernoulli's Inequality}
\starttheorem
  For every real number $r\geq 1$ and real number $x\geq -1$, we have \startformula (1 + x)^r\geq 1 + rx\stopformula while for $0\leq r\leq 1$ and real
  number $x\geq -1$ we have \startformula (1 + x)^r\leq 1 + rx.\stopformula
\stoptheorem

\startproof
  Using the convexity of $f(x) = \ln(x)$ on $(0, \infty)$. Since $x\geq -1$, we have $1 + x\geq 0$. If $0 \leq r \leq 1$, we have
  \startformula \ln(1 + rx) = \ln(r(1 + x) + 1 - r)\geq r\ln(1 + x) + (1 - r)\ln(1) = r\ln(1 + x).\stopformula
  Taking antilog gives $(1 + x)^r\leq 1 + rx.$ When $1\leq r < \infty$,
  \startformula \ln(1 + x) = \ln\left(\frac{r - 1}{r} + \frac{1}{r}(1 + rx)\right)\geq\frac{r - 1}{r}\ln(1) + \frac{1}{r}\ln(1 + rx) =
  \frac{1}{r}\ln(1 + rx).\stopformula
  This gives $(1 + x)^r\geq 1 + rx.$
\stopproof

\section{Popoviciu's Inequality}
\starttheorem
  Let $f: I\rightarrow\mathbb{R}$. If $f$ is convex, then for any three p;oints $x, y, z$ in $I$:
  \placeformula\startformula
    \frac{f(x) + f(y) + f(z)}{3} + f\left(\frac{x + y + z}{3}\right)\geq \frac{2}{3}\left[f\left(\frac{x + y}{2}\right) +
      f\left(\frac{y + z}{2}\right) + f\left(\frac{z + x}{2}\right)\right]
  \stopformula
\stoptheorem

\startproof
  Without loss of generality, we can assume that $x\leq y\leq z$. If $x\leq y\leq \frac{x + y + z}{3}$, then
  \startformula \frac{x + y + z}{3}\leq \frac{x + z}{2}\leq z\;\text{{\rm{and}}}\;\frac{x + y + z}{3}\leq \frac{y + z}{2}\leq z.\stopformula
  Therefore, there exists $s, t\in[0,1]$ such that
  \startformula \frac{x + z}{2} = \left(\frac{x + y + z}{3}\right)s + z(1 - s)\stopformula
  \startformula \frac{y + z}{2} = \left(\frac{x + y + z}{3}\right)t + z(1 - t)\stopformula
  Adding, we get
  \startformula \frac{x + y - 2z}{2} = \frac{x + y - 2z}{3}(s + t)\Rightarrow s + t = \frac{3}{2}.\stopformula
  As $f$ is a convex function
  \startformula f\left(\frac{x + z}{2}\right)\leq s.f\left(\frac{x + y + z}{3}\right) + (1 - s).f(z)\stopformula
  \startformula f\left(\frac{y + z}{2}\right)\leq t.f\left(\frac{x + y + z}{3}\right) + (1 - t).f(z)\stopformula
  and
  \startformula f\left(\frac{x + y}{2}\right)\leq \frac{1}{2}f(x) + \frac{1}{2}f(y).\stopformula
  Adding together last three inequalities we get the required inequality.
  The case when $\frac{x + y + z}{3}\leq y$ is considered similarly, bearing in mind that $x\leq \frac{x + z}{2}\leq \frac{x + y +
    z}{3}$ and $x\leq\frac{y + z}{2}\leq\frac{x + y + z}{3}$.

  When $f$ is a concave function, the inequality gets reversed.
\stopproof

\section{Majorization}
{\bf Definition:}
  Given two seuquences $\langle a\rangle = (a_1, a_2, \ldots, a_n)$ and $\langle b\rangle = (b_1, b_2, \ldots, b_n)$ where $a_i,
  b_i\in\mathbb{R}\;\forall i\in\{1,2,\ldots, n\}$. We say that the sequence $\langle a\rangle$ majorizes the seuqnece $\langle
  b\rangle$, and write $\langle a\rangle\succ \langle b\rangle$, if the following conditions are fulfilled:
  \startformula a_1\geq a_2\geq \cdots\geq a_n;\stopformula
  \startformula b_1\geq b_2\geq \cdots\geq b_n;\stopformula
  \startformula a_1 + a_2 + \cdots + a_n = b_1 + b_2 + \cdots + b_n;\stopformula
  \startformula a_1 + a_2 + \cdots + a_k\geq b_1 + b_2 + \cdots + b_k\;\forall k\in\{1, 2, \ldots, n - 1\}.\stopformula

\section{Karamata's Inequality}
\starttheorem
  Let $f: [a,b]\rightarrow\mathbb{R}$ be a convex function. Suppose that $(x_1, \ldots, x_n)\succ(y_1, \ldots, y_n)$ where $x_1,
  \ldots, x_n, y_1, \ldots, y_n\in[a, b]$. Then we have:
  \placeformula\startformula
    \sum_{i=1}^nf(x_i)\geq\sum_{i=1}^nf(y_i).
  \stopformula
\stoptheorem

\startproof
  If $f(x)$ is a convex function over the interval $(a, b)$, then $\forall a\leq x_1\leq x_2\leq b$ and $g(x, y)=\frac{f(y) -
    f(x)}{y - x}, f(x_1,x)\leq g(x_2, x)$. If $x<x_1$, then
  \startformula g(x_1, x) = \frac{f(x_1) - f(x)}{x_1 - x}\leq\frac{f(x_1) - f(x)}{x_1 - x} = g(x_2- x).\stopformula
  We can argue similarly for other values of $x$.

  \noindent We define a sequence $\langle C\rangle$ such that $c_i = g(a_i, b_i)$

  \noindent We also define sequences $\langle A\rangle$ and $\langle B\rangle$ such that \startformula A_i = \sum_{j=1}^ia_j, A_0 =
  0\;\text{{\rm{and}}}\; B_i = \sum_{j=1}^ib_j, B_0 =0\stopformula
  If we assume that $a_i\geq a_{i+1}$ and similarly $b_i\geq b_{i+1}$, then we get that $c_i\geq c_{i+1}$. Now, we know that
  \startformula \sum_{i=1}^nf(a_i) - \sum_{i=1}^nf(b_I) = \sum_{i=1}^nc_i(a_i - b_i) = \sum_{i=1}^nc_i(A_i - A_{i-1} - B_i + B_{i + 1})\stopformula
  \startformula =\sum_{i=1}^nc_i(A_i - B_i) - \sum_{i=0}^{n-1}c_{i+1}(A_i - B_i) = \sum_{i=1}^n(c_i - c_{i+1})(A_i - B_i)\geq 0\stopformula
  Therefore,
  \startformula \sum_{i=1}^nf(x_i)\geq\sum_{i=1}^nf(y_i).\stopformula
\stopproof

\section{Muirhead's Inequality}
\starttheorem
  If a sequence $\langle a\rangle$ majorises a sequence $\langle b\rangle$, and $x_1, x_2, \ldots, x_n$ be a set of postiive real
  numbers then
  \placeformula\startformula
    \sum_{sym}x_1^{a_1}x_2^{a_2}\ldots x_n^{a_n}\geq \sum_{sym}x_1^{b_1}x_2^{b_2}\ldots x_n^{b_n}
  \stopformula
\stoptheorem

\startproof
  We define a sequence $\langle c\rangle$ such that $\sum_{i=1}^nc_i = 0$, the we observe
  \startformula \sum_{sym}x_1^{c_1}x_2^{c_2}\ldots x_n^{c_n}\geq n!\stopformula
  for real $x_1, x_2, \ldots x_n$.
  By AM-GM we know that
  \startformula \frac{\sum_{sym}x_1^{c_1}x_2^{c_2}\ldots x_n^{c_n}}{n!}\geq n!\sqrt{\prod_{sym}x_1^{c_1}x_2^{c_2}\ldots x_n^{c_n}}\stopformula
  \startformula \Rightarrow n!\sqrt{\prod_{sym}x_1^{c_1}x_2^{c_2}\ldots x_n^{c_n}} = n!\sqrt{\prod_{i=1}^nx_i^{(n-1)!(c_1 + c_2 + \cdots
      c_n)}} = 1\stopformula
  \startformula \Rightarrow \sum_{sym}x_1^{c_1}x_2^{c_2}\ldots x_n^{c_n} \geq n!\stopformula
  We defined out sequence $\langle c\rangle$ such that $c_i = a_ i - b_i$ which gives us $\sum c_i = \sum a_i - \sum b_i = 0$

  \noindent Thus, $\sum_{sym}x_1^{c_1}x_2^{c_2}\ldots x_n^{c_n} - n! \geq 0.$
  Multiplying with $\sum_{sym}\prod_{i=1}^nx_i^{b_i}$, we get
  \startformula \left(\sum_{sym}\prod_{i=1}^nx_i^{b_i}\right)\left(\sum_{sym}x_1^{c_1}x_2^{c_2}\ldots x_n^{c_n} - 1\right)\stopformula
  \startformula = \sum_{sym}\prod_{i=1}^nx_i^{b_i + c_i} - \prod_{i=1}^nx_i^{b_1}\geq 0\stopformula
  \startformula \Rightarrow \sum_{sym}\prod_{i=1}^nx_i^{a_i} - \prod_{i=1}^nx_i^{b_1}\geq 0\stopformula
  Hence, it is proved.
\stopproof

\section{Schur's Inequality}
\starttheorem
  Let $x, y, z$ be non-negative real numbers. For any $r > 0$, we have
  \placeformula\startformula
    \sum_{cyc}x^r(x - y)(x - z) \geq 0
  \stopformula
  with equality if and only if $x = y = z$, or if two of $x, y, z$ are equal and the third is $0$.
\stoptheorem

\startproof
  When $r = 1$, the following case arises:
  \startformula x^3 + y^3 + z^3 + 3xyz \geq xy(x + y) + yz(y + z) + zx(z + x).\stopformula
  Because L.H.S. is cyclic in $x, y, z$ without loss of generality we can assume $x\geq y \geq z$. Rewriting L.H.S., we have
  \startformula (x - y)[x^r(x - z) - y^r(y - z)] + z^r(z - x)(z - y).\stopformula
  We see that $x^r \geq y^r$ and $x - z\geq y - z$. Thus the expression inside brackets is non-negative. $(x - y)$ is also
  non-negative. $z^r$ and $(z - x)(z - y)$ are also non-negtive. Thus entire expression is non-negative and hence the inequality is
  proven.
\stopproof

\noindent Velentin Vornicu has given a general form of Schur's inequality. Consider $a, b, c, x, y, z\in\mathbb{R}$, where $a\geq
b\geq c$, and either $z\geq y\geq z$ or $z\geq y\geq x$. Let $k\in\mathbb{Z}^+$, and let $f:\mathbb{R}\rightarrow\mathbb{R}_0^+$ be
either convex or monotonic, then
\placeformula\startformula
  f(x)(a - b)^k(a - c)^k + f(y)(b - a)^k(b - c)^k + f(z)(c - a)^k(c - b)^k\geq 0.
\stopformula

\section{Symmetric Functions}
Let $a_1, a_2, \ldots, a_n$ be arbitrary real numbers. Considering the polynomial $P(x) = (x + a_1)(x + a_2)\cdots(x + a_n) =
c_ox^n + c_1x^{n - 1} + \cdots + c_{n-1}x + c_n$. The the coefficients $c_0, c_1, \ldots, c_n$ can be expressed as functions of
$a_1, a_2, a_n$ like $c_0 = 1, c_1 = a_1 + a_2 + \cdots + a_n, c_2 = a_1a_2 + a_2a_3 + \cdots, a_{n-1}a_n, c_3 = a_1a_2a_3 +
a_2a_3a_4 + \cdots + a_{n-2}a_{n-1}a_n, \ldots, c_n = a_1a_2\ldots a_n$.

These are also called {\it elementary symmetric sum} and the first elementary symmetric sum of $f(x)$ is often written as
$\sum_{sym}f(x)$ while the $n$th can be written as $\sum_{sym}^nf(x)$.

The {\it symmetric sum} $\sum_{sym}f(x_1, x_2, \ldots, x_n)$ of a function $f(x_1, x_2, \ldots, x_n)$ of $n$ variables is
defined to be $\sum_{\sigma}f(x_{\sigma(1)}, x_{\sigma(2)}, \ldots, x_{\sigma(n)})$, where $\sigma$ ranges over over all
permutations of $(1, 2, \ldots, n)$. More generally symmetric sum of $n$ variables is a sum that is unchanged by any permutatoin of
its variables. Any symmetric sum can be written as a polynomial of elementary symmetric sums.

A {\it symmetric function} of $n$ variables is a function that does not change by any permutation of its variables. Therefore,

\startformula \sum_{sym}f(x_1, x_2, \ldots, x_n) = n!f(x_1, x_2, \ldots, x_n)\stopformula

We define {\it symmetric average} $p_k$ as $\frac{c_k}{\binom{n}{k}}$.

\section{Newton's Inequality}
\starttheorem
  For non-negative $x_1, x_2, \ldots, x_n$ and $0<k< n$m
  \placeformula\startformula
    d_k^2\geq d_{k-1}d_{k + 1},
  \stopformula
  equality holds when all $x_i$'s are equal.
\stoptheorem

\startproof
  We will prove this by mathematical induction. A proof by calculus is also possible but we will not prove by that method.

  For $n = 2$, the inequality becomes AM-GM inequaltiy. Let the inequality hold for $n = m - 1$ for some positive integer $m\geq
  3$.

  Let $d_k'$ be the symmetric averages of $x_1, x_2, \ldots, x_{m-1}$. Note that $d_k = \frac{n-k}{n}{d'}_k +
  \frac{k}{n}{d'}_{k-1}x_m$.
  \startformula d_{k-1}d_{k+1} = \left(\frac{n-k+1}{n} {d'}_{k-1} + \frac{k-1}{n} {d'}_{k-2} x_m \right)\left(\frac{n-k-1}{n} {d'}_{k+1} +
  \frac{k+1}{n} {d'}_k x_m \right)\stopformula
  \startformula = \frac{(n-k+1)(n-k-1)}{n^2} {d'}_{k-1}{d'}_{k+1} + \frac{(k-1)(n-k-1)}{n^2} {d'}_{k-2} {d'}_{k+1} x_m\stopformula
  \startformula + \frac{(n-k+1)(k+1)}{n^2} {d'}_{k-1}{d'}_k x_m + \frac{(k-1)(k+1)}{n^2} {d'}_{k-2}{d'}_k x_m^2\stopformula
  \startformula \le  \frac{(n-k+1)(n-k-1)}{n^2} {d'}_k^2 + \frac{(k-1)(n-k-1)}{n^2} {d'}_{k-2} {d'}_{k+1} x_m\stopformula
  \startformula + \frac{(n-k+1)(k+1)}{n^2} {d'}_{k-1}{d'}_k x_m + \frac{(k-1)(k+1)}{n^2} {d'}_{k-1}^2 x_m^2\stopformula
  \startformula \le  \frac{(n-k+1)(n-k-1)}{n^2} {d'}_k^2 + \frac{(k-1)(n-k-1)}{n^2} {d'}_{k-1} {d'}_{k} x_m\stopformula
  \startformula + \frac{(n-k+1)(k+1)}{n^2} {d'}_{k-1}{d'}_k x_m + \frac{(k-1)(k+1)}{n^2} {d'}_{k-1}^2 x_m^2\stopformula
  \startformula = \frac{(n-k)^2}{n^2} {d'}_k^2 + \frac{2(n-k)k}{n^2} {d'}_k {d'}_{k-1} x_m +\frac{k^2}{n^2} {d'}_{k-1}^2 x_m^2 -
  \left(\frac{d_k}{n} - \frac{d_{k-1}x_m}{n}\right)^2\stopformula
  \startformula \le  \left(\frac{n-k}{n} {d'}_k + \frac{k}{n} {d'}_{k-1} x_m \right)^2        = d_k^2\stopformula
  Hence, it is proven by induction.
\stopproof

\section{Maclaurin's Inequality}
\starttheorem
  For non-negative $x_1, x_2, \ldots, x_n$ and $0<k< n$m
  \placeformula\startformula
    d_1\geq d_2^{1/2}\geq\cdots\geq d_n^{1/n},
  \stopformula
  equality holds when all $x_i$'s are equal.
\stoptheorem

\startproof
  Following Newton's inequality it is enough to show that $d_{n-1}^{1/(n-1)}\geq d_n^{1/n}$.

  Since this is a homogeneous inequaqlity, it can be normalized. Thus, $d_n = \prod x_i = 1$ We then transform the inequality to(by
  exponentiating both sides by $n-1$)
  \startformula \frac{\sum 1/x_i}{n}\geq 1^{(n-1)/n} = 1.\stopformula  We know that the G.M. of $\frac{1}{x_1}, \frac{1}{x_2}, \cdots,
  \frac{1}{x_n}$ is $1$ and hence the inequality is true by AM-GM.
\stopproof

\section{Aczel's Inequality}
\starttheorem
  If $a_1^2>a_2^2 + \cdots + a_n^2$ or $b_1^2 > b_2^2 + \cdots + b_n^2$, then
  \placeformula\startformula
    (a_1b_1 - a_2b_2 - \cdots - a_nb_n)^2 \geq (a_1^2 - a_2^2 - \cdots - a_n^2)(b_1^2 - b_2^2 - \cdots - b_n^2)
  \stopformula
\stoptheorem

\startproof
  Consider the function \startformula f(x) = (a_1x  - b_1)^2 - \sum_{i=2}^n(a_ix - b_i)^2\stopformula
  \startformula = (a_1^2 - a_2^2 - \cdots - a_n^2)x^2 - 2(a_1b_2 - a_2b_2 - \cdots - a_nb_n)x + (b_1^2 - b_2^2 - \cdots - b_n^2).\stopformula
  We have $f\left(\frac{b_1}{a_1}\right) = -\sum_{i=2}^n\left(a_i\frac{b_1}{a_1} - b_i\right)^2 \leq 0$, and from $a_1^2>a_2^2 +
  \cdots + a_n^2$ we get $\displaystyle\lim_{x\to\infty}f(x)\rightarrow\infty$. Therefore, $f(x)$ must have at least one root,
  $\Leftrightarrow D = (a_1b_1 - a_2b_2 - \cdots - a_nb_n)^2 - (a_1^2 - a_2^2 - \cdots - a_n^2)(b_1^2 - b_2^2 - \cdots - b_n^2)\geq
  0.$
\stopproof

\section{Carleman's Inequality}
\starttheorem
  Let $a_1, a_2, \ldots, a_n$ be $n$ non-negative real numbers, where $n\geq 1$ then
  \placeformula\startformula
    \sum_{i=1}^{\infty} (a_1 a_2 \cdots a_i)^{1/i} < e \sum_{i=1}^{\infty} a_i ,
  \stopformula
  unless all of $a_i$'s are equal to zero.
\stoptheorem

\startproof
  Let us define $c_n = n\left(1 + \frac{1}{n}\right)^n = \frac{(n + 1)^n}{n^{n - 1}}$. Then for all positive integers $i$,
  \startformula (c_1\ldots c_i)^{1/i} = i + 1\stopformula
  \startformula \Rightarrow \sum_{i=1}^\infty(a_1\ldots a_i)^{1/i} = \sum_{i=1}^\infty\frac{(c_1a_1\ldots c_ia_i)^{1/i}}{(c_1\ldots
    c_i)^{1/i}} = \sum_{i=1}^\infty\frac{(c_1a_1\ldots c_ia_i)^{1/i}}{i + 1}.\stopformula
  Using AM-GM inequality, we get
  \startformula \sum_{i=1}^\infty\frac{(c_1a_1\ldots c_ia_i)^{1/i}}{i + 1} \leq \sum_{i=1}^\infty\sum_{j=1}^i\frac{c_ja_j}{i(i + 1)} =
  \sum_{j=1}^\infty\sum_{i=j}^\infty\frac{c_ja_j}{i(i + 1)}.\stopformula
  Using the partial fraction for $\frac{1}{i(i + 1)}$
  \startformula \sum_{i=j}^\infty\frac{1}{i(i + 1)} = \sum_{i=j}^\infty\left(\frac{1}{i} - \frac{1}{i + 1}\right) = \frac{1}{j}.\stopformula
  \startformula \Rightarrow \sum_{j=1}^\infty\sum_{i=j}^\infty\frac{c_ja_j}{i(i + 1)} = \sum_{j=1}^\infty\left(1 + \frac{1}{j}\right)^ja_i.\stopformula
  Since $\left(1 + \frac{1}{j}\right)^j< e,\;\forall\;j\in I$ the inequality holds.
\stopproof

\section{Sum of Squares(SOS Method)}
Sum of sqaures or S.O.S. method revolves around the basic fact that sum of
squares is a non-negative quantity. As you can see it requires knowledge only of
very basic inequalitites which makes it highly desirable. By using SOS method we
rewrite inequalitites as {\it sum of squares} to prove them as non-negative
using only basic inequalities.

\startproposition[pp:sos1]
  Let $a, b, c\in\mathbb{R}$. Then $(a - c)^2\leq 2(a - b)^2 + 2(b - c)^2$.
\stopproposition

\startproof
  We have \startformula (a - c)^2\leq 2(a - b)^2 + 2(b - c)^2\stopformula
  \startformula \Leftrightarrow a^2 - 2ac + c^2\leq 2(a^2 - 2ab + b^2) + 2(b^2 - 2bc +
  c^2)\stopformula
  \startformula \Leftrightarrow a^2 + 4b^2 + c^2 0 4ab - 4bc + 2ac\geq 0\stopformula
  \startformula \Leftrightarrow (a + c - 2b)^2\geq 0,\stopformula
  which clearly holds.
\stopproof

\startproposition[pp:sos2]
  Let $a\geq b\geq c$. Then $(a - c)^2\geq (a - b)^2 + (b - c)^2$.
\stopproposition

\startproof
  We have \startformula (a - c)^2\geq (a - b)^2 + (b - c)^2\stopformula
  \startformula \Leftrightarrow a^2 - 2ac + c^2 \geq (a^2 - 2ab + b^2) + (b^2 - 2bc + c^2)\stopformula
  \startformula \Leftrightarrow b^2 + ac - ab - b \leq 0\stopformula
  \startformula \Leftrightarrow (b - a)(b - c)\leq 0,\stopformula
  which is true for $a\geq b\geq c$.
\stopproof

\startproposition[pp:sos3]
  Let $a\geq b\geq c$. Then $\frac{a - c}{b - c}\geq \frac{a}{b}$.
\stopproposition

\startproof
  Given $\frac{a - c}{b - c}\geq \frac{a}{b}$
  \startformula \Leftrightarrow b(a - c)\geq a(b - c)\Leftrightarrow ac\geq
  bc\Leftrightarrow a\geq b.\stopformula
\stopproof

\starttheorem
  Consider the expression $S = S_a(b - c)^2 + S_b(c - a)^2 + S_c(a - b)^2$,
  where $S_a, S_b, S_c$ are functions of $a, b, c$.
  \startitemize[n]
  \item If $S_a, S_b, S_c\geq 0$ then $S\geq 0$.
  \item If $a\geq b\geq c$ or $a\leq b\leq c$ and $S_b, S_b + S_a, S_b + S_c\geq
    0$ then $S\geq 0$.
  \item If $a\geq b\geq c$ or $a\leq b\leq c$ and $S_a, S_c, S_a + 2S_b, S_c +
    2S_b \geq 0$ then $S\geq 0$.
  \item If $a\geq b\geq c$ and $S_b, S_c, a^2S_b + b^2S_a\geq 0$ then $S\geq 0$.
  \item If $S_a + S_b\geq 0$ or $S_b + S_c\geq 0$ or $S_c + S_a \geq 0$ or $S_a
    + S_b + S_c \geq 0$ and $S_aS_b + S_bS_c + S_cS_a\geq 0$ then $S\geq 0$.
  \stopitemize
\stoptheorem

\startproof
  \startitemize[n]
  \item If $S_a, S_b, S_c\geq 0$ then clearly $S\geq 0$.
  \item Let us assume that $a\geq b\geq c$ or $a\leq b\leq c$ and $S_b, S_b +
    S_a, S_b + S_c\geq 0$.

    \noindent By Proposition (\in{Preposition}[pp:sos2]), it follows that $(a - c)^2 \geq (a
    - b)^2 + (b - c)^2$, so we have
    \startformula \startalign\NC S \NC= S_a(b - c)^2 + S_b(c - a)^2 + S_c(a - b)^2\NR\NC\NC\geq S_a(b
      - c)^2 + S_b[(a - b)^2 + (b - c)^2] + S_c(a - b)^2\NR\NC\NC=(b - c)^2(S_a + S_b)
      + (a - b)^2(S_b + S_c).\stopalign\stopformula
    Thus, $S\geq 0$ because $S_a + S_b, S_b + S_c\geq 0$.
    \item Let us assume that $a\geq b\geq c$ or $a\leq b\leq c$ and $S_a, S_c,
      S_a + 2S_b, S_c + 2S_b \geq 0$.

      \noindent Then if $S_b\geq 0$ clearly $S\geq 0$.

      \noindent For case when $S_b\leq 0$, by Proposition (\in{Preposition}[pp:sos1]), we
      have $(a - c)^2 \leq 2(a - b)^2 + 2(b - c)^2$. Therefore
      \startformula \startalign\NC S \NC= S_b(b - c)^2 + S_b(a - c)^2 + S_c(a - b)^2\NR\NC\NC\geq
        S_a(b - c)^2 + S_b[2(a - b)^2 + 2(b - c)^2] + S_c(a - b)^2\NR\NC\NC= (b -
        c)^2(S_a + 2S_b) + (a - b)^2(S_c + 2S_b)\stopalign\stopformula
      which is true for the given conditions.
    \item Given $a\geq b\geq c$ and $S_b, S_c, a^2S_b + b^2S_a\geq 0$

      \noindent By Proposition (\in{Preposition}[pp:sos3]), we have $\frac{a - c}{b - c}\geq
      \frac{a}{b}$. Therefore
      \startformula \startalign\NC S \NC= S_a(b - c)^2 + S_b(a - c)^2 + S_c(a - b)^2\geq S_a(b
        - c)^2 + S_b(a - c)^2\NR\NC\NC=(b - c)^2\left[S_a + S_b\left(\frac{a - c}{b -
            c}\right)^2\right]\geq (b - c)^2\left[S_a +
          S_b\left(\frac{a}{b}\right)^2\right]\NR\NC\NC = (b - c)^2\left(\frac{b^2S_a
          + a^2S_b}{b^2}\right),\stopalign\stopformula
      which is true for given conditions.
    \item We assume that $S_b + S_c\geq 0$. Then
      \startformula \startalign\NC S \NC = S_a(b - c)^2 + S_b(a - c)^2 + S_c(a - b)^2\NR\NC\NC=
      S_a(b - c)^2 +
      S_b[(c - b) + (b - a)]^2 + S_c(a - b)^2\NR\NC\NC= (S_b + S_c)(a - b)^2 + 2S_b(c
      - b)(b - a) + (S_a + S_b)(b - c)^2\NR\NC\NC=(S_b + S_c)\left(b - a +
      \frac{S_b}{S_b + S_c}(c - b)\right)^2 + \frac{S_aS_b + S_bS_c+ S_cS_a}{S_b
      + S_c}(c - b)^2&\geq 0.\stopalign\stopformula
  \stopitemize
\stopproof

Every difference $\sum_{cyc}x_1^{\alpha_1}x_2^{\alpha_2}\ldots x_n^{\alpha_n} -
\sum_{cyc}x_1^{\beta_1}x_2^{\beta_2}\ldots x_n^{\beta_n}$ where $\alpha_1 +
\alpha_2 + \cdots + \alpha_n = \beta_1 + \beta_2 + \cdots + \beta_n$ can be
written in SOS form.

Some special cases are given below:

\startitemize[n]
  \item $a^2 + b^2 + c^2 - ab - bc - ca = \frac{(a - b)^2 + (b - c)^2 + (c -
    a)^2}{2}$
  \item $a^3 + b^3 + c^3 - 3abc = \frac{a + b + c}{2}\left[(a - b)^2 + (b - c)^2
    + (c - a)^2\right]$
  \item $a^b + b^2c + c^2a - ab^2 - bc^2 - ca^2 = \frac{(a - b)^3 + (b - c)^3 + (c -
    a)^3}{3}$
  \item $a^3 + b^3 + c^3 - a^2b - b^2c - c^2a = \frac{(2a + b)(a - b)^2 + (2b +
    c)(b - c)^2 + (2c + a)(c - a)^2}{3}$
  \item $a^4 + b^4 + c^4 - a^3b - b^3c - c^3b = \\\\\frac{(3a^2 + 2ab + b^2)(a -
    b)^2 + (3b^2 + 2bc + c^2)(b - c)^2 + (3c^2 + 2ca + a^2)(c - a)^2}{4}$
  \item $a^3b + b^3c + c^3a - ab^3 - bc^3 - ca^3 = \frac{a + b + c}{3}\left[(b -
    a^3) + (c - b)^3 + (a - c)^3\right]$
  \item $a^4 + b^4 + c^4 - a^2b^2 - b^2c^2 - c^2a^2 = \frac{(a^2 - b^2)^2 + (b^2
    - c^2)^2 + (c^2 - a^2)^2}{2}$
\stopitemize

\starttheorem
  Consider two polynomials having the same degree and same number of variables $A$ and $B$. The difference of these two polynomilas
  can be written in SOS form:
  \startformula \sum_{cyc}a_1^{\alpha_1}a_2^{\alpha_2}\ldots a_n^{\alpha_n} - \sum_{cyc}a_1^{\beta_1}a_2^{\beta_2}\ldots a_n^{\beta_n} = \sum
  P_{ij}(a)(a_i - a_j)^2,\stopformula
  where $\alpha_1 + \alpha_2 + \cdots + \alpha_n = \beta_1 + \beta_2 + \cdots + beta_n = m$ and $a = (a_1, a_2, \ldots,
  a_n)$.
\stoptheorem

\startproof
  We need to prove the following lemma first.
  \startlemma
    If $a = (a_1, a_2, \ldots, a_n)$ and $\alpha_1 + \alpha_2 + \alpha_n = m$, then:
    \startformula \sum_{cyc}a_1^n - \sum_{cyc}a_1^{\alpha_1}a_2^{\alpha_2}\ldots a_n^{\alpha_n} = \sum P_{ij}(a)(a_i - a_j)^2\stopformula
  \stoplemma
  We prove this lemma by induction over $k$, which will be the number of elements except $0$ belonging to the set ${\alpha_1,
    \alpha_2, \ldots, \alpha_n}$.

  If $k = 1$, the theorem is obviously true.

  If $k = 2$, the expression becomes $\sum_{cyc}a_1^m - \sum_{a_1}^ta_2^{m - t} = \sum P_{ij}(a)(a_i - a_j)^2$

  We observe that $ta^m + (m - t)b^m - ma^tb^{m - t} = P(a, b)(a - b)^2$. We also observe that $f(x) = tx^n + (m - t) - mx^t = 0$
  has one repeated root which is $1$ because $f(1) = f'(1) = 0$. Therefore $f(x)$ can be written like $Q(x)(x - 1)^2$ where degree
  of $Q$ will be $m - 2$.

  Let $x = \frac{a}{b}$, then we have: $b^mf\left(\frac{a}{b}\right) = ta^m + (m - t)b^m - ma^tb^{m - 1} = b^{m -
    2}Q\left(\frac{a}{b}\right)(a - b)^2$.

  However, $b^{m - 2}$ is a polynomial having $2$ variables $a, b$ because $Q$ is a $m - 2$ degree polynomial. If our proposition
  is already true with $k$, the number of elements except for $0$ in the set of $\alpha$, with $k + 1$ we can transform this into
  the case of $k$ as given below:

  $a_1^{\alpha_1}a_2^{\alpha_2}\ldots a_{k+1}^{\alpha_{k +1}} = \frac{\alpha_1a_1^{\alpha_1 + \alpha_2} + \alpha_2a_2^{\alpha_1 +
      \alpha_2} - (\alpha_1 + \alpha_2)a_1^{\alpha_1}a_2^{\alpha_2}}{\alpha_1 + \alpha_2}.a_3^{\alpha_3}\ldots a_{k+1}^{\alpha_{k
    +1}}\frac{\alpha_1}{\alpha_1 + \alpha_2}a_1^{\alpha_1+\alpha_2}$

  $a_3^{\alpha_3}\ldots a_{k + 1}^{\alpha_{k + 1}} + \frac{\alpha_2}{\alpha_1 + \alpha_2}a_2^{\alpha_1 + \alpha_3}a_3^{\alpha_3}\ldots a_{k
    + 1}^{\alpha_{k +1}}$

  With $k = 2$: $\frac{\alpha_1a_1^{\alpha_1 + \alpha_2} + \alpha_2a_2^{\alpha_1 + \alpha_2} - (\alpha_1 +
    \alpha_2)a_1^{\alpha_1}a_2^{\alpha_2}}{\alpha_1 + \alpha_2} = H_{12}(a)(a_1 - a_2)^2$, we have:

  $a_1^{\alpha_1}a_2^{\alpha_2}\ldots a_{k+1}^{k + 1} = Q_{12}(a)(a_1 - a_2)^2 + \frac{\alpha_1}{\alpha_1 +
    \alpha_2}a_1^{\alpha_1+\alpha_2}a_3^{\alpha_3}\ldots a_{k + 1}^{\alpha_{k + 1}} + \frac{\alpha_2}{\alpha_1 + \alpha_2}a_2^{\alpha_1 +
    \alpha_3}a_3^{\alpha_3}\ldots a_{k + 1}^{\alpha_{k +1}}$

  $\therefore \sum_{cyc}a_1^m - \sum_{cyc}a_1^{\alpha_1}a_2^{\alpha_2}\ldots a_{k + 1}^{\alpha_{k + 1}} = -\sum_{cyc}Q_{12}(a)(a_1
  - a_2)^2 + \sum_{cyc}a_1^m - \frac{\alpha_1}{\alpha_1 +
    \alpha_2}$

  $\sum a_1^{\alpha_1+\alpha_2} a_3^{\alpha_3}\ldots a_{k + 1}^{\alpha_{k + 1}} + \sum\frac{\alpha_2}{\alpha_1 +
    \alpha_2}a_2^{\alpha_1 +\alpha_3}a_3^{\alpha_3}\ldots a_{k + 1}^{\alpha_{k +1}} \sum_{a_1}^m - \sum_{cyc}a_1^{\alpha_1}\ldots
  a_{k +1}^{\alpha_{k +1}}$

  $= -\sum Q_{12}(a)(a_1 - a_2)^2 + \frac{\alpha_1}{\alpha_1 + \alpha_2}\left(\sum_{cyc} a_1^m - \sum_{cyc}a_1^{\alpha_1 +
    \alpha_2}a_3^{\alpha_3}\ldots a_{k+1}^{\alpha_{k+1}}\right) + $

  $\frac{\alpha_2}{\alpha_1 + \alpha_2}\left(\sum_{cyc} a_1^m - \sum_{cyc}a_2^{\alpha_1 + \alpha_2}a_3^{\alpha_3}\ldots a_{k+1}^{\alpha_{k+1}}\right)$

  So we see that these can be written in SOS form recursively. Hence proved.
\stopproof

\section{Problems}
Prove the following inequalities:

\startitemize[n, 2*broad]
\item $a^2 + b^2 \geq 2ab$.
\item $\sqrt{ab}\geq \frac{2}{\frac{1}{a} + \frac{1}{b}}$, where $a>0, b>0$.
\item $\sqrt{\frac{a^2 + b^2}{2}}\geq \frac{a + b}{2}$
\item $\frac{a + b}{2}\geq \frac{2}{\frac{1}{a} + \frac{1}{b}}$, where $a>0, b>0$.
\item $a + b > 1 + ab$, where $b < 1 < a$.
\item $a^2 + b^2 > c^2 + (a + b - c)^2$, where $b < c< a$.
\item $2\leq \frac{a}{b} + \frac{b}{a}$, where $ab > 0$.
\item $\frac{a}{b} + \frac{b}{a}\leq -2$, where $ab < 0$.
\item $x_1\leq \frac{x_1 + \cdots + x_n}{n}\leq x_n$, where $x_1\leq \cdots\leq x_n$.
\item $\frac{x_1}{y_1}\leq \frac{x_1 + \cdots + x_n}{y_1 + \cdots + y_n}\leq x_n$, where $\frac{x_1}{y_1}\leq\cdots\leq
  \frac{x_n}{y_n}$ and $y_i> 0, i=1, \ldots, n$.
\item $x_1\leq(x_1\ldots x_n)^{\tfrac{1}{n}}\leq x_n$, where $n\geq 2, 0\leq x_1\leq\ldots\leq x_n$.
\item $|a_1| + \cdots + |a_n|\geq |a_1 + a_2 + \cdots + a_n|$.
\item $\frac{a_1 + \cdots + a_n}{n}\geq \frac{n}{\frac{1}{a_1} + \cdots + \frac{1}{a_n}}$, where $n\geq 2, a_i> 0, i=1, \ldots, n$.
\item $a + b\sqrt{\frac{a + b}{2}} \geq a\sqrt{b} + b\sqrt{a}$, where $a > 0, b > 0$.
\item $\frac{1}{2}(a + b) + \frac{1}{4}\geq \sqrt{\frac{a + b}{2}}$, where $a > 0, b > 0$.
\item $a(x + y - a)\geq xy$, where $x\leq a\leq y$.
\item $\frac{1}{x - 1} + \frac{1}{x + 1} > \frac{2}{x}$, where $x > 1$.
\item $\frac{1}{3k + 1} + \frac{1}{3k + 2} + \frac{1}{3k + 3} > \frac{1}{2k + 1} + \frac{1}{2k + 2}$, where $k\in\mathbb{N}$.
\item $\frac{ab}{(a + b)^2}\leq \frac{(1 - a)(1 - b)}{[(1 - a) + (1 - b)]^2}$, where $o<\leq\frac{1}{2}, 0<b\leq\frac{1}{2}$.
\item $\frac{1}{\sqrt{3k + 1}}.\frac{2k + 1}{2k + 2} < \frac{1}{\sqrt{3k + 4}}$, where $k\in\mathbb{N}$.
\item $2^{n - 1}\geq n$, where $n\in\mathbb{N}$.
\item $\frac{1}{3} + \frac{2}{3}.\frac{1}{5} + \frac{2}{3}.\frac{4}{5} + \frac{1}{7} + \cdots +
  \frac{2}{3}.\frac{4}{5}.\frac{6}{7}\cdots \frac{100}{101}.\frac{1}{103} < 1$.
\item $\frac{1 - a}{1 - b} + \frac{1 - b}{1 - a}\leq \frac{a}{b} + \frac{b}{a}$, where $0 < a, b\leq \frac{1}{2}$.
\item $\displaystyle\sum_{i=1}^n\frac{1}{1 - a_i}\sum_{i=1}^m(1 - a_i)\leq \sum_{i=1}^n\frac{1}{a_i}\sum_{i=1}^na_i$, where $0 < a_1, \ldots,
  a_n\leq \frac{1}{2}$.
\item $1 + \frac{1}{2^3} + \cdots + \frac{1}{n^3} < \frac{5}{4}$, where $n\in\mathbb{N}$.
\item $\frac{1}{1 + a + b}\leq 1 - \frac{a + b}{2} + \frac{ab}{3}$, where $0\leq a\leq 1, 0\leq b\leq$.
\item $|x - y| < |1 - xy|$, where $|x| < 1, |y| < 1$.
\item $\frac{a}{bc} + \frac{b}{ca} + \frac{c}{ab}\geq \frac{2}{a} + \frac{2}{b} - \frac{2}{c}$, where $a > 0, b > 0, c > 0$.
\item $\frac{1}{a} + \frac{1}{b} - \frac{1}{c} < \frac{1}{abc}$, where $a^2 + b^2 + c^2 = \frac{5}{3}$ and $a > 0, b > 0, c > 0$.
\item $3(1 + a^2 + a^4)\geq (1 + a + a^2)^2$.
\item $(ac + bd)^2 + (ad - bc)^2\geq 144$, where $a + b = 4, c + d = 6$.
\item $x_1^2 + x_2^2 + \cdots + x_{2n}^2 + na^2 \geq a\sqrt{2}(x_1 + x_2 + \cdots + x_{2n})$.
\item $\frac{1}{a + b} + \frac{1}{b + c} + \frac{1}{a + c}\leq \frac{\sqrt{a} + \sqrt{b} + \sqrt{c}}{2\sqrt{abc}}$, where $a > 0, b
  > 0, c > 0$.
\item $a^3(b^2 - c^2) + b^3(c^2 - a^2) + c^3(a^2 - b^2) < 0$, where $0 < a < b < c$.
\item $\frac{y}{x} + \frac{y}{z} + \frac{x + z}{y}\leq \frac{(x + z)^2}{xz}$, where $0< x\leq y\leq z$.
\item $\sqrt{1 + \sqrt{a}} + \sqrt{1 + \sqrt{a + \sqrt{a^2}}} + \cdots + \sqrt{1 + \sqrt{a + \cdots + \sqrt{a^n}}} < na$ where $n
  \geq 2, a\geq 2, n\in\mathbb{N}$.
\item $[5x]\geq [x] + \frac{[2x]}{2} + \frac{[3x]}{3} + \frac{[4x]}{4} + \frac{[5x]}{5}$, where $[x]$ si the integer part of the
  real number $x$.
\item $(n!)^2 \geq n^n$, where $n\in\mathbb{N}$.
\item $x^6 + x^5 + 4x^4 - 12x^3 + 4x^2 + x + 1\geq 0$.
\item $\log^2\alpha\geq \log\beta\log\gamma$, where $\alpha>1, \beta>1, \gamma>1, \alpha^@\geq\beta\gamma$.
\item $\log_45 + \log_56 + \log_67 + \log_78> 4.4$.
\item $\frac{1}{3} + \frac{2}{3.5} + \cdots + \frac{n}{3.5\ldots(2n + 1)} < \frac{1}{2}$, where $n\in\mathbb{N}$.
\item $\frac{2^3 + 1}{2^3 - 1}\cdots\frac{n^3 + 1}{n^3 - 1}< \frac{3}{2}$, where $n\geq 2, n\in\mathbb{N}$.
\item $1.1! + 2.2! + \cdots + n.n! < (n + 1)!$, where $n\in\mathbb{N}$.
\item $\left(1 + \frac{1}{2^2}\right)\left(1 + \frac{1}{3^2}\right)\cdots\left(1 + \frac{1}{n^2}\right) < 2$, where $n\geq 2,
  n\in\mathbb{B}$.
\item $\left(1 - \frac{1}{p_1^2}\right)\left(1 - \frac{1}{p_2^2}\right)\cdots\left(1 - \frac{1}{p_n^2}\right) > \frac{1}{2}$, where
  $1 < p_1 < p_2 <\cdots < p_n, p_i\in\mathbb{N}, i = 1, 2, \ldots n$.
\item $\frac{1}{2} - \frac{1}{3} + \frac{1}{4} - \frac{1}{5} + \cdots - \frac{1}{999} + \frac{1}{1000}<\frac{2}{5}$.
\item $\frac{a + b}{1 + a + b}\leq \frac{a}{1 + a} + \frac{b}{1 + b}$, where $a\geq 0, b\geq 0$.
\item $\frac{a + b}{2 + a + b}\geq \frac{1}{2}\left(\frac{a}{1 + a} + \frac{b}{1 + b}\right)$, where $a\geq 0, b\geq 0$.
\item $\displaystyle\sum_{i=1}^n\frac{a_1 + 2a_2 + \cdots + ia_i}{i^2}\leq 2\sum_{i=1}^na_i$, where $a_i\geq 0, i=1, 2, \ldots, n$.
\item $\frac{1}{a} + \frac{1}{b} + \frac{1}{c}\leq \frac{41}{42}$, where $\frac{1}{a} + \frac{1}{b} + \frac{1}{c}< 1, a, b,
  c\in\mathbb{N}$.
\item $\frac{4x}{y + z} + \frac{y}{x + z} + \frac{z}{x + y} > 2$, where $x,y,z > 0$.
\item $1 < \frac{a}{a + b + d} + \frac{b}{a + b + c} + \frac{c}{b + c + d} + \frac{d}{a + c + d}< 2$, where $a,b,c,d > 0$.
\item $a + b > c + d$, where $a, b, c, d\geq \frac{1}{2}$ and $a^2 + b > c^2 + d, a + b^2 > c + d^2$.
\item $(b - a)(9 - a^2) + (c - a)(9 - b^2) + (c - b)(9 - c^2)\leq 24\sqrt{2}$, where $0\leq a\leq b\leq c\leq 3$.
\item If $0<a, b, c < 1$, then one of the numbers $(1 - a)b, (1 - b)c, (1 - c)a$ is not greater than $\frac{1}{4}$.
\item Let $a > 0, b > 0, c > 0$, and $a + b + c = 1$. Prove that $\sqrt{a + \frac{1}{4}(b - c)^2} + \sqrt{b + \frac{1}{4}(c - a)^2}
  + \sqrt{c + \frac{1}{4}(b - a)^2}\leq 2$.
\item Let $a > 0, b > 0, c > 0$, and $a + b + c = 1$. Prove that $\sqrt{a + \frac{1}{4}(b - c)^2} + \sqrt{b} + \sqrt{c}\leq
  \sqrt{3}$.
\item Find the smallest possible value of the expression: $\frac{a^4}{b^4} + \frac{b^4}{a^4} - \frac{a^2}{b^2} - \frac{b^2}{a^2} +
  \frac{a}{b} + \frac{b}{a}$, where $a, b > 0$.
\item $\frac{(1 - x_1)(1 - x_2)\ldots(1 - x_n)}{x_1x_2\ldots x_n}\geq (n - 1)^n$, where $n\geq 2, x_i>0,  i = 1, 2, \ldots, n$ and
  $x_1 + x_2 + \cdots + x_n = 1$.
\item $\frac{1}{1 + x_1} + \frac{1}{1 + x_2} + \cdots + \frac{1}{1 + x_n}$, where $n\geq 2, x_1\geq 1, x_2\geq 1, \ldots, x_n\geq
  1$.
\item $abc + bcd + cda + dab\leq \frac{1}{27} + \frac{176}{27}abcd$, where $a, b, c, d\geq 0$, and $a + b + c + d = 1$.
\item $0\leq xy + yz + zx - 2xyz\leq \frac{7}{27}$, where $x, y, z\geq 0$, and $x + y + z = 1$.
\item Suppose that for numbers $x_1, x_2, \ldots, x_{1997}$, the following conditions holds: (a) $-\frac{1}{\sqrt{3}}\leq x_i \leq
  \sqrt{3}, i = 1, 2, \ldots, 1997$, (b) $x_1 + x_2 + \cdots + x_{1997} = -318\sqrt{3}$. Find the greatest possible value of the
  expression $x_1^{12} + x_2^{12} + \cdots + x_{1997}^{12}$.
\item Prove that $\cos\alpha_1\cos\alpha_2\cdots\cos\alpha_n(\tan\alpha_1 + \tan\alpha_2 + \cdots + \tan\alpha_n)\leq \frac{(n-
  1)^{(n - 1)/2}}{n^{(n - 2)/2}}$, where $n\geq 2$ and $0\leq \alpha_i<\frac{\pi}{2}, i = 1, 2, \ldots, n$.
\item Prove that $\displaystyle\sum_{i=1}^nx_i^k(1 - x_i)\leq a_k$, where $k\geq 2, k\in\mathbb{N}$, and $a_k = {\rm \max}[x^k(1 -
  x) + (1 - x)^kx], x_i\geq 0, i=1, 2, \ldots, n, x_1 + x_2 + \cdots + x)n = 1, n\geq 2$.
\item $2(n - 1)(x_2x_3 + x_1x_3 + \cdots + x_1x_n + x_2x_3 + \cdots + x_2x_n + \cdots + x_{n - 1}x_n) - n^{n - 1}x_1x_2\ldots
  x_n\leq n - 2$, where, $n\geq 2, x_1, x_2, \ldots x_n\geq 0$ and $x_1 + x_2 + \cdots + x_n = 1$.
\item $\frac{x_1 + x_2 + \cdots + x_n}{n} - \sqrt[n]{x_1x_2\ldots x_n}\leq$

  $\frac{(\sqrt{x_1} - \sqrt{x_2})^2 + (\sqrt{x_1} -
  \sqrt{x_3})^2 + \cdots + (\sqrt{x_1} - \sqrt{x_n})^2 + \cdots + (\sqrt{x_{n - 1}} - \sqrt{x_n})^2}{n}$, where $n\geq 2, x_1, x_2,
  \ldots, x_n\geq 0$.
\item {\it Turkevici's Inequality}: $(n - 1)(x_1^2 + x_2^2 + \cdots + x_n^2) + \sqrt[n]{x_1^2x_2^2\ldots x_n^2}\geq (x_1 + x_2 +
  \cdots + x_n)^2$, where $n\geq 2, x_2, x_2, \ldots, x_n\geq 0$.
\item $(a + b)(b + c)(c + a)\geq 8abc$, where $a>0, b>0, c>0$.
\item $(a + b + c - d)(b + c + d - a)(c + d + a - b)(d + a + b - c)\leq (a + b)(b + c)(c + d)(d + a)$, where $a > 0, b > 0, c > 0,
  d > 0$.
\item ({\it Schur's Inequality}) $a^3 + b^3 + c^3 + 3abc \geq a^2b + ab^2 + b^2c + bc^2 + ca^2 + c^2a$, where $a > 0, b > 0, c > 0$.
\item $\left(1 + \frac{4a}{b + c}\right)\left(1 + \frac{4b}{c + a}\right)\left(1 + \frac{4c}{a + b}\right) > 25$, where $a > 0, b >
  0, c > 0$.
\item $\frac{\log(a - 1)}{\log a} < \frac{\log a}{\log(a + 1)}$, where $a > 1$.
\item ({\it Schur's Inequality}) $abc \geq (a + b - c)(c + a - b)(b + c - a)$, where $a>0, b>0, c>0$.
\item $x^8 + y^8 \geq \frac{1}{128}$, if $x + y = 1$.
\item $\left(a + \frac{1}{a}\right)^2 + \left(b + \frac{1}{b}\right)^2\geq 12.5$, if $a > 0, b > 0$ and $a + b = 1$.
\item $\left(x_1 + \frac{1}{x_1}\right)^2 + \cdots + \left(x_n + \frac{1}{x_2}\right)^2\geq \frac{(n^2 + 1)^2}{n}$, if $n\geq 2,
  x_1> 0, \ldots, x_n>0$ and $x_1 + \cdots + x_n = 1$.
\item $a^4 + b^4 + c^4 \geq abc(a + b + c)$.
\item $x^2 + y^2 \geq 2\sqrt{2}(x - y)$, if $xy = 1$.
\item $\sqrt{6a_1 + 1} + \sqrt{6a_2 + 1} + \sqrt{6a_3 + 1} + \sqrt{6a_4 + 1} + \sqrt{6a_5 + 1}\leq \sqrt{55}$, if $a_1 > 0, \ldots,
  a_5 > 0$ and $a_1 + \cdots + a_5 = 1$.
\item $6a + 4b + 5c\geq 5\sqrt{ab} + 3\sqrt{bc + 7\sqrt{ca}}$, where $a\geq 0, b\geq 0, c\geq 0$.
\item $2(a^4 + b^4) + 17 > 16 ab$.
\item $\left(\frac{1 + nb}{n + 1}\right)^{n + 1}\geq b^n$, where $n\in\mathbb{N}, b > 0$.
\item $\left(1 + \frac{1}{n}\right)^n < \left(1 + \frac{1}{n + 1}\right)^{n + 1}$, where $n\in\mathbb{N}$.
\item $\left(1 + \frac{1}{n}\right)^{n + 1} < \left(1 + \frac{1}{n + 1}\right)^{n + 2}$, where $n\in\mathbb{N}$.
\item $\left(1 + \frac{m}{n - 1}\right)^{(n - 1)/m} < \left(1 + \frac{m}{n}\right)^{n/m} < \left(1 + \frac{m - 1}{n}\right)^{n/(m
  - 1)}$, where $m > 1, n > 1$ and $m,n \in\mathbb{N}$.
\item $n!< \left(\frac{n + 1}{2}\right)^n$, where $n = 2, 3, 4, \ldots$.
\item $n(n + 1)^{1/n} < n + S_n$, where $S_n = \frac{1}{1} + \frac{1}{2} + \cdots + \frac{1}{n}, n = 2, 3, 4, \ldots$.
\item $n - S_n > (n - 1)^{1/(1 - n)}$, where $S_n = \frac{1}{1} + \frac{1}{2} + \cdots + \frac{1}{n}, n = 3, 4, \ldots$.
\item $(q^n - 1)(q^{n + 1} + 1)\geq 2nq^n(q - 1)$, where $q > 1, n \in\mathbb{N}$.
\item $a^2 + b^2 + c^2 + d^2 + ab + ac + ad + bc + bd + cd\geq 10$, where $a, b, c, d>0$, and $abcd = 1$.
\item $\left(a - 1 + \frac{1}{b}\right)\left(b - 1 + \frac{1}{c}\right)\left(c - 1 + \frac{1}{a}\right)\leq \left(\frac{1 +
  abc}{2\sqrt{abc}}\right)^3$, where $a, b, c> 0$.
\item $\left(a + \frac{1}{b} - t\right)\left(b + \frac{1}{c} - t\right)\left(c + \frac{1}{a} - t\right)\leq (a + b +
  c)\left(\frac{1}{a} + \frac{1}{b} + \frac{1}{c}\right)(1 - t)^2 + 4 - 3t$, where $a, b,c, t > 0$ and $abc = 1$.
\item $n\sqrt[n]{a_1a_2\ldots a_n} - (n - 1)\sqrt[n - 1]{a_1a_2\ldots a_{n - 1}} \leq a_n$, where $a_i > 0, i = 1, 2,
  \ldots, n, n = 3, 4, \ldots$.
\item $\sqrt[n]{a_1a_2\ldots a_n} + \sqrt[n]{b_1b_2\ldots b_n} + \cdots + \sqrt[n]{k_1k_2\ldots k_n}$

  $\leq\sqrt[n]{(a_1 + b_1 + \cdots + k_1)(a_2 + b_2 + \cdots + k_2)\cdots(a_n + b_n + \cdots + k_n)}$ where

  $a_1, a_2,\ldots, a_n, b_1, b_2, \ldots, b_n, \ldots, k_1, k_2, \ldots, k_n > 0$.
\item $a_1 + \sqrt{a_1a_2} + \cdots + \sqrt[n]{a_1a_2\ldots a_n}\leq e(a_1 + a_2 + \cdots + a_n)$, where $n\geq 2, a_1, a_2,
  \ldots, a_n\geq 0$.
\item $na^k - ka^n\leq n - 1$, where $n > k, n, k\in\mathbb{N}, a > 0$.
\item $\frac{x_1^2}{x_2} + \frac{x_2^3}{x_3^2} + \cdots + \frac{x_n^{n + 1}}{x_1^n}\geq x_1 + x_2 + \cdots + x_n$, where $n\geq 2,
  n\in\mathbb{N}, x_1 = {\rm min}(x_1, x_2, \ldots, x_n) > 0$.
\item $\frac{a^{x_1 - x_2}}{x_1 + x_2} + \frac{a^{x_2 - x_3}}{x_2 + x_3} + \cdots + \frac{a^{x_n - x_1}}{x_n +
  x_1}\geq \frac{n^2}{2\displaystyle\sum_{i=1}^nx_i}$, where $a > 0, x_i > 0, i = 1, 2, \ldots, n$.
\item $\sqrt[p]{x_1 + 1} + \sqrt[p]{x_2 + 1} + \cdots + \sqrt[p]{x_n + 1}\leq n + 1$, where $n\geq 2, x_1, x_2, x_n > 0, x_1 + x_2
  + \cdots + x_n = p, p\in\mathbb{N}, p\geq 2$.
\item $x^k(1 - x^m)\leq \frac{k^{k/m}.m}{(k + m)^{1 + k/m}}$, where $0\leq x\leq 1, k, m\in\mathbb{N}$.
\item $\frac{x}{1 - x^2} + \frac{y}{1 - y^2} + \frac{z}{1 - z^2}\geq \frac{3\sqrt{3}}{2}$, where $x, y, z > 0$ and $x^2 + y^2 + z^2
  = 1$.
\item $\frac{1}{1 - x} + \frac{1}{1 - y} + \frac{1}{1 - z}\geq \frac{9 + 3\sqrt{3}}{2}$, where $x, y, z > 0$ and $x^2 + y^2 + z^2
  = 1$.
\item Find the minimum value of the funciton $f(x) = \frac{1}{\sqrt[n]{1 + x}} + \frac{1}{\sqrt[n]{1 - x}}$ in $[0, 1)$, where
  $n\in\mathbb{N}, n > 1$.
\item Find the minimum value of the funciton $f(x) = ax^m + \frac{b}{x^n}$ in $(0, \infty)$, where $a, b > 0, m, n\in\mathbb{N}$.
\item Find in $[a, b] (0 < a < b)$ a point $x_0$ such that the function $f(x) = (x - a)^2(b^2 - x^2)$ attains its maximum value in
  $\lfloor a, b\rfloor$ at $x_0$.
\item Find the greatest possible value of the product $xyz$ given $x, y, z > 0$, and $2x + \sqrt{3}y + \pi z = 1$.
\item Find the maximum and minimum values of the function $y = \frac{x}{ax^2 + b}$, where $a, b > 0$.
\item Find the maximum value of the function $y = \frac{5\sqrt{x^2 + 6x + 8} + 12}{x + 3}$.
\item Find the maximum value of the function $y = \frac{\sqrt[3]{(x^2 + 1)^2(x^2 + 3)}}{3x^3 + 4}$.
\item Solve the system of equations: $x + y = 2, xy - z^2 = 1$.
\item Solve the system of equations: $x + y + z = 3, x^2 + y^2 + z^2 = 3$.
\item Given $a + b + c + d + e = 8, a^2 + b^2 + c^2 + d^2 + e^2 = 16$, find the greatest possible value of $e$.
\item Find the minimum value of the expression $\frac{x_1}{x_2} + \frac{x_3}{x_4} + \frac{x_5}{x_6}$ if $1\leq x_1\leq x_2\leq
  x_3\leq x_4\leq x_5\leq x_6\leq 1000$.
\item Solve the equation $x^4 + y^4 + 2 = 4xy$.
\item Find all integer solutions of the equation $\frac{xy}{z} + \frac{yz}{x} + \frac{zx}{y} = 3$.
\item Prove that $x_1^\alpha + x_2^\alpha + \cdots + x_n^\alpha\geq x_1^\beta + x_2^\beta + \cdots + x_n^\beta$, where $n\geq 2,
  x_1 > 0, x_2 > 0, \cdots, x_n > 0, \alpha > \beta \geq 0$, and $x_1x_2\ldots x_n = 1$.
\item Prove that $x_1^\alpha + x_2^\alpha + \cdots + x_n^\alpha\geq x_1^\beta + x_2^\beta + \cdots + x_n^\beta$, where $n\geq 2,
  x_1 > 0, x_2 > 0, \cdots, x_n > 0, \alpha\geq (n - 1)|\beta|$, and $x_1x_2\ldots x_n = 1$.
\item Prove that $\frac{1 + a}{1 + ab} + \frac{1 + b}{1 + bc} + \frac{1 + c}{1 + cd} + \frac{1 + d}{1 + da}\geq 4$, where $a, b, c,
  d > 0$ and $abcd = 1$.
\item Prove that $\frac{1 + ab}{1 + a} + \frac{1 + bc}{1 + b} + \frac{1 + cd}{1 + c} + \frac{1 + da}{1 + d}\geq 4$, where $a, b, c,
  d > 0$ and $abcd = 1$.
\item Prove that $2ST > \sqrt{3(S + T)[S(bd + df + fb) + T(ac + ce + ea)]}$, where $0 < a < b < c < d < e < f$ and $a + c + e = S,
  b + d + f = T$.
\item Prove that $\frac{a + \sqrt{ab} + \sqrt[3]{abc} + \sqrt[4]{abcd}}{4}\leq \sqrt[4]{a.\frac{a + b}{2}.\frac{a + b +
    c}{3}. \frac{a + b + c + d}{4}}$, where $a > 0, b > 0, c > 0, d > 0$.
\item Prove that $a^{12} + (ab)^6 + (abc)^4 + (abcd)^3\leq 1.43(a^{12} + b^{12} + c^{12} + d^{12})$.
\item $ab\leq \frac{a^p}{p} + \frac{b^q}{1}$, if $\frac{1}{p} + \frac{1}{q} =1, a, b, p, q >0$, where $p$ and $q$ are rational
  numbers.
\item $\left(1 + \frac{1}{n}\right)^n > 2$, where $n\in\mathbb{N}$.
\item $(1 + a_1)(1 + a_2)\cdots(1 + a_n)\leq 1 + \frac{S}{1!} + \cdots + \frac{S^n}{n!}$, where $n\geq 2, S = a_1 + a_2 + \cdots +
  a_n, a_i > 0, i = 1, 2, \ldots, n$.
\item $\left(1 + \frac{1}{a}\right)\left(1 + \frac{1}{b}\right)\left(1 + \frac{1}{c}\right)\geq 64$, where $a, b, c > 0$ and $a + b
  + c = 1$.
\item $\sqrt[n]{a^{2n - 1}} + \sqrt[n]{a^{2n + 1}}\geq 3a - 1$, where $n\geq 2, a > 0, n > k, n,k\in\mathbb{N}$.
\item $\frac{a^n - 1}{a^n(a - 1)}\geq n + 1 - a^{\tfrac{n(n + 1)}{2}}$, where $a > 0, a\neq 1$.
\item $na^{n + 1} + 1\geq (n + 1)a^n$, where $a > 0$.
\item $\left(\sqrt{k} + \sqrt{k + 1}\right)\left(\sqrt{k + 1} + \sqrt{k + 1}\right)\cdots \left(\sqrt{n} + \sqrt{n + 1}\right)\geq
  \left(\sqrt{n} - \sqrt{k}\right)\left(\sqrt{n} + \sqrt{k} - 1\right) + 2$, where $n > k, n, k\in\mathbb{N}$.
\item $\frac{a_1}{a_2} + \frac{a_2}{a_3} + \cdots + \frac{a_{n - 1}}{a_n} + \frac{a_n}{a_1}\geq n$, where $a_i > 0, i = 1, 2,
  \ldots, n$.
\item $a_{n + 1} + \frac{1}{a_1(a_2 - a_1)(a_3 - a_2)\cdots(a_{n + 1} - a_n)}\geq n + 1$, where $0 < a_k < a_{k + 1}, k = 1, 2,
  \ldots, n$.
\item $1 + \frac{x}{2}\leq \frac{1}{\sqrt{1 - x}}$, where $0\leq x < 1$.
\item $\frac{a^4}{b^4} + \frac{b^4}{c^4} + \frac{d^4}{e^4} + \frac{e^4}{a^4}\geq \frac{a}{b} + \frac{b}{c} + \frac{c}{d} +
  \frac{d}{e} + \frac{e}{a}$, where $abcde\neq 0$.
\item $\left(\frac{a}{b}\right)^{1999} + \left(\frac{b}{c}\right)^{1999} + \left(\frac{c}{d}\right)^{1999} +
  \left(\frac{d}{a}\right)^{1999}$, where $a, b, c, d > 0$.
\item Prove that $\sqrt{\frac{a_1 + a_2}{a_3}} + \sqrt{\frac{a_2 + a_3}{a_4}} + \cdots + \sqrt{\frac{a_{n - 1} + a_n}{a_{1}}} +
  \sqrt{\frac{a_n + a_1}{a_2}}\geq n\sqrt{2}$, where $n > 2$ and $a_1 > 0, a_2 > 0, \ldots, a_n > 0$.
\item Prove that $\frac{x}{1 + x^2} + \frac{y}{1 + y^2} + \frac{z}{1 + z^2}\leq \frac{3\sqrt{3}}{4}$, where $x^2 + y^2 + z^2 = 1$.
\item Prove that $\left(\frac{1}{a_1^2} - 1\right)\left(\frac{1}{a_2^2} - 1\right)\cdots\left(\frac{1}{a_n^2} - 1\right)\geq(n^2 -
  1)n$, where $n\geq 2, a_1 > 0, a_2 > 0, \ldots, a_n > 0$ and $a_1 + a_2 + \cdots + a_n = 1$.
\item Find the maximum and minimum value of the expression $(1 + u)(1 + v)(1 + w)$ if $0 < u \leq \frac{7}{16}, 0 < v \leq
  \frac{7}{16}, 0 < w \leq \frac{7}{16}$, and $u + v + w = 1$.
\item Find the maximum value of the expression $x^py^q$ if $x + y = a, x > 0, y > 0$ and $p, q\in\mathbb{N}$.
\item Find the maximum value of the expression $a + 2c$ if for all $x$, one has $ax^2 + bx + c \leq \frac{1}{\sqrt{1 - x^2}}$,
  where $|x|< 1$.
\item Prove that $\left(1 + \frac{a}{b}\right)\left(1 + \frac{b}{c}\right)\left(1 + \frac{c}{a}\right) \geq 2\left(1 + \frac{a + b
  + c}{\sqrt[3]{abc}}\right)$, where $a > 0, b > 0, c > 0$.
\item Prove that $\frac{1 + a_1}{1 - a_1}.\frac{1 + a_2}{1 - a_2}\cdots\frac{1 + a_{n + 1}}{1 - a_{n + 1}}$, where $-1 < a_1, a_2,
  \ldots, a_{n + 1} < 1$ and $a_1 + a_2 + \cdots + a_{n + 1}\geq n - 1$.
\item Prove that $(a + b)^3(b + c)^3(c + d)^3(d + a)^3\geq 16a^2b^2c^2d^2(a + b + c + d)^4$, where $a >0, b > 0, c > 0, d > 0$.
\item Prove that $\left[\left(1 + \frac{a}{b}\right)^2 + \left(1 + \frac{b}{c}\right)^2 + \left(1 + \frac{c}{a}\right)^2\right]
  \left[\left(1 + \frac{b}{a}\right)^2 + \left(1 + \frac{c}{b}\right)^2 + \left(1 + \frac{a}{c}\right)^2\right]\geq$

  $4\left(\frac{a + b}{c} + \frac{b + c}{a} + \frac{c + a}{b}\right)^2$, where $a > 0, b > 0, c > 0$.
\item Prove that $(a^2 + bc)^3(b^2 + ac)^3(c^2 + ab)^3\geq 64(a^3 + b^3)(b^3 + c^3)(c^3 + a^3)$, where $a > 0, b > 0, c > 0$.
\item Prove that $a + \sqrt{ab} + \sqrt[3]{abc}\leq \frac{4}{3}(a + b + c)$, where $a > 0, b > 0, c > 0$.
\item Prove that $a + \sqrt{ab} + \sqrt[3]{abc}\leq 3.\sqrt[3]{a . \frac{a + b}{2} . \frac{a + b + c}{3}}$, where $a > 0, b > 0, c
  > 0$.
\item Prove that $(ab)^{\tfrac{5}{4}} + (bc)^{\tfrac{5}{4}} + (ca)^{\tfrac{5}{4}} \leq \frac{\sqrt{3}}{9}$, where $a > 0, b > 0, c >
  0$ and $a + b + c = 1$.
\item Prove that $a^2 + b^2 + c^2 \geq 14$ if $a + 2b + 3c\geq 14$.
\item Prove that $ab + \sqrt{(1 - a^2)(1 - b^2)}\leq 1$ if $|a|\leq 1, |b|\leq 1$.
\item Prove that $\sqrt{c(a - c)} + \sqrt{c(b - c)}\leq \sqrt{ab}$ if $a > c, b > c, c > 0$.
\item Prove that $a\sqrt{a^2 + c^2} + b\sqrt{b^2 + c^2}\leq a^2 + b^2 + c^2$.
\item Prove that $\frac{1}{\sqrt{ab}} + \frac{1}{\sqrt{bc}} + \frac{1}{\sqrt{ca}}\leq \frac{1}{a} + \frac{1}{b} + \frac{1}{c}$,
  where $a > 0, b > 0, c > 0$.
\item Prove that $\sqrt{a}(a + c - b) + \sqrt{b}(a + b - c) + \sqrt{c}(b + c - a)\leq \sqrt{(a^2 + b^2 + c^2)(a + b + c)}$, where
  $a, b, c$ are lengths of sides of a triangle.
\item Prove that $(a_1 + a_2 + \cdots + a_n)\left(\frac{1}{a_1} + \frac{1}{a_2} + \cdots + \frac{1}{a_n}\right)\geq n^2$, where
  $a_1 > 0, a_2 > 0, \ldots, a_n > 0$.
\item Prove that $\frac{a_1^2 + a_2^2 + \cdots + a_n^2}{n}\geq \left(\frac{a_1 + a_2 + \cdots + a_n}{n}\right)^2$.
\item Prove that $a_1a_2 + a_2a_3 + \cdots + a_9a_{10} + a_{10}a_1\geq -1$ if $a_1^2 + a_2^2 + \cdots + a_{10}^2 = 1$.
\item Prove that $x^4 + y^4 \geq x^3y + xy^3$.
\item Prove that $\left(|a_1|^3 + |a_2|^3 + \cdots + |a_n|^3\right)^2\leq \left(a_1^2 + a_2^2 + \cdots + a_n^2\right)^3$.
\item Prove that $3\left(a^2 + b^2 + c^2 + x^2 + y^2 + z^2\right) + 6\sqrt{\left(a^2 + b^2 + c^2\right)\left(x^2 + y^2 +
  z^2\right)}\geq$

  $\left(a + b + c + x + y + z\right)^2$.
\item Prove that $a^2 + b^2 + c^2\geq ab + bc + ca$.
\item Prove that $(a_1 + a_2 + \cdots + a_n)(a_1^7 + a_2^7 + \cdots + a_n^7)\geq(a_1^3 + a_2^3 + \cdots + a_n^3)(a_1^5 + a_2^5 +
  \cdots + a_n^5)$, where $a_1 > 0, a_2 > 0, \ldots, a_n > 0$.
\item Prove that $\sqrt{a + 1} + \sqrt{2a - 3} + \sqrt{50 - 3a}\leq 12$, where $\frac{3}{2}\leq a\leq \frac{50}{3}$.
\item Prove that $a + b + c\leq abc + 2$, where $a^2 + b^2 + c^2 = 2$.
\item Prove that $2(a + b + c) - abc \leq 10$, where $a^2 + b^2 + c^2 = 9$.
\item Prove that $1 + abc\geq 3.min(a, b, c)$, where $a^2 + b^2 + c^2 = 9$.
\item Prove that $\displaystyle\left(\sum_{i=1}^na_i^{k + 1}\right)\left(\sum_{i=1}^na_i^{-1}\right)\geq
  n\left(\sum_{i=1}^na_i^k\right)$, where $k, n\in\mathbb{N}$ and $a_1 > 0, a_2 > 0, \ldots, a_n > 0$.
\item Prove that $\frac{a + b + c}{3}\geq \sqrt[3]{abc}$, where $a > 0, b > 0, c > 0$.
\item Prove that $\frac{a_1^k + a_2^k + \cdots + a_n^k}{n}\geq \left(\frac{a_1 + a_2 + \cdots + a_n}{n}\right)^k$, where $k,
  n\in\mathbb{N}$ and $a_1 > 0, a_2 > 0, \ldots, a_n > 0$.
\item Prove that $\left(1 + \frac{1}{\sin\alpha}\right)\left(1 + \frac{1}{\cos\alpha}\right) > 5$, where $0 < \alpha <
  \frac{\pi}{2}$.
\item Find the smallest possible value of the expression $(u - v)^2 + \left(\sqrt{2 - u^2} - \frac{9}{v}\right)^2$ if $0 < u <
  \sqrt{2}, v > 0$.
\item Prove that $x_1^2 + \left(\frac{x_1 + x_2}{2}\right)^2 + \cdots + \left(\frac{x_1 + x_2 + \cdots + x_n}{n}\right)^2\leq 4(x_1
  + x_2 + \cdots + x_n)^2$. This inequality is a particular case of {\it Hardy's inequality}
  $\displaystyle\sum_{k=1}^n\left(\frac{a_1 + a_2 + \cdots + a_k}{k}\right)^p \leq \left(\frac{p}{p - 1}\right)^p.\sum_{k = 1}^n
  a_k^p$, where $p > 1, a_i \geq 0, i = 1, 2, \ldots, n$.
\item Prove that $\frac{1}{a_1} + \frac{2}{a_1 + a_2} + \cdots + \frac{n}{a_1 + a_2 + \cdots + a_n} < 2\left(\frac{1}{a_1} +
  \frac{1}{a_2} + \cdots + \frac{1}{a_n}\right)$, where $a_1 > 0, a_2 > 0, \ldots, a_n > 0$.
\item Prove that $\left(\sin\alpha_1 + \sin\alpha_2 + \cdots + \sin\alpha_n\right)^2 + \left(\cos\alpha_1 + \cos\alpha_2 + \cdots +
  \cos\alpha_n\right)^2\leq n^2$.
\item Prove that $\frac{a_1 + a_2 + \cdots + a_n}{n}\geq \sqrt[n]{a_1a_2\ldots a_n}$, where $n\geq 2, a_1 > 0, a_2 > 0, \ldots, a_n
  > 0$.
\item Prove that $\sqrt{a_1b_1} + \sqrt{a_2b_2} + \cdots + \sqrt{a_nb_n}\leq \sqrt{a_1 + a_2 + \cdots + a_n}.\sqrt{b_1 + b_2 +
  \cdots + b_n}$, where $a_i\geq 0, b_i\geq 0, i = 1, 2, \ldots, n$.
\item Prove that $(x_1y_2 - x_2y_1)^2 + (x_2y_3 - x_3y_2)^2 + (x_1y_3 - x_3y_1)^2\leq (x_1^2 + x_2^2 + x_3^2)(y_1^2 + y_2^2 +
  y_3^2)$.
\item Prove that $\displaystyle\left(\sum_{i=1}^n\sqrt{a_ib_i}\right)^2\leq
  \left(\sum_{i=1}^na_ix_i\right)\left(\sum_{i=1}^n\frac{b_i}{x_i}\right)$, where $x_i > 0, a_i > 0, b_i > 0, i = 1, 2, \ldots n$.
\item Prove that $\displaystyle\left(\sum_{i=1}^nx_iy_i\right)\left(\sum_{i=1}^n\frac{x_i}{y_i}\right)\geq\left(\sum_{i =
  1}^nx_i\right)^2$, where $x_i > 0, y_i > 0, i = 1, 2, \ldots, n$.
\item Prove that $ax + by + cz + \sqrt{(a^2 + b^2 + c^2)(x^2 + y^2 + z^2)}\geq \frac{2}{3}(a + b + c)(x + y + z)$.
\item Prove that $(p_1q_1 - p_2q_2 - \cdots - p_nq_n)^2\geq(p_1^2 - p_2^2 - \cdots - p_n^2)(q_1^2 - q_2^2 - \cdots - q_n^2)$, if
  $p_1^2\geq p_2^2 + \cdots + p_n^2, q_1^2 \geq q_2^2 + \cdots + q_n^2$.
\item Prove that $\sqrt{x^2 + xy + y^2}\sqrt{y^2 + yz + z^2} + \sqrt{y^2 + yz + z^2}\sqrt{z^2 + zx + x^2} + $

  $\sqrt{z^2 + zx +  x^2}\sqrt{x^2 + xy + y^2}\geq (x + y + z)^2$.
\item Prove that $a_1(b_1 + a_2) + a_2(b_2 + a_3) + \cdots + a_n(b_n + a_1) < 1$, where $n\geq 3, a_1, a_2, \ldots, a_n > 0$ and
  $a_1 + a_2 + \cdots + a_n = 1, b_1^2 + b_2^2 + \cdots + b_n^2 = 1$.
\item Prove that $\sqrt{1 - \left(\frac{x + y}{2}\right)^2} + \sqrt{1 - \left(\frac{y + z}{2}\right)^2} + \sqrt{1 - \left(\frac{z +
    x}{2}\right)^2\geq \sqrt{6}}$, where $x, y, z \geq 0, x^2 + y^2 + z^2 = 1$.
\item Prove that $\sqrt{\frac{a}{b + c}} + \sqrt{\frac{b}{c + a}} + \sqrt{\frac{c}{a + b}}\geq 2\sqrt{1 + \frac{abc}{(a + b)(b +
    c)(c + a)}}$, where $a, b, c > 0$.
\item Prove that $\sqrt{a + (b - c)^2} + \sqrt{b + (c - a)^2} + \sqrt{c + (a - b)^2}\geq \sqrt{3}$, where $a, b, c\geq 0$ and $a +
  b + c = 1$.
\item Prove that $\sqrt{\frac{a + b}{2} - ab} + \sqrt{\frac{b + c}{2} - bc} + \sqrt{\frac{c + a}{2} - ca}\geq \sqrt{2}$, where $a,
  b, c \geq 0$ and $a + b + c = 2$.
\item Prove that $\sqrt{1 - xy}\sqrt{1 - yz} + \sqrt{1 - yz}\sqrt{1 - zx} + \sqrt{1 - zx}\sqrt{1 - xy}\geq 2$, where $x, y, z\geq
  0$ and $x^2 + y^2 + z^2 = 1$.
\item Prove that $x\sqrt{1 - yz} + y\sqrt{1 - zx} + z\sqrt{1 - xy}\geq \frac{2\sqrt{2}}{3}$, where $x, y, z\geq 0$ and $x + y + z =
  1$.
\item Prove the following identity $\displaystyle(a_1c_1 + a_2c_2 + \cdots + a_nc_n) - (a_1d_1 + a_2d_2 + \cdots + a_nd_n)(b_1c_1 +
  b_2c_2 + \cdots + b_nc_n) = \sum_{1\leq i<k\leq n}(a_ib_k - a_kb_i)(c_id_k - c_kd_i)$.
\item Prove that $(a_1c_1 + a_2c_2 + \cdots + a_nc_n) - (a_1d_1 + a_2d_2 + \cdots + a_nd_n)\geq (a_1d_1 + a_2d_2 + \cdots +
  a_nd_n)(b_1c_1 + b_2c_2 + \cdots + b_nc_n)$, where $b_id_1\geq 0 (i = 1, 2, \ldots, n)$ or $b_id_1 < 0 (i = 1, 2, \ldots, n)$ and
  $\frac{a_1}{b_1}\leq \frac{a_2}{b_2}\leq \ldots\leq \frac{a_n}{b_n}, \frac{c_1}{d_1}\leq \frac{c_2}{d_2}\leq \ldots\leq
  \frac{c_n}{d_n}$.
\item Find the maximum and minimum value of the expression $\frac{\sqrt{x^2 + y^2} + \sqrt{(x - 2)^2 + (y - 1)^2}}{\sqrt{x^2 + (y -
    1)^2} + \sqrt{(x - 2)^2 + y^2}}$.
\item Find the minimum value of the expression $\left(\frac{1}{x^n} + \frac{1}{a^n} - 1\right)\left(\frac{1}{y^n} + \frac{1}{b^n} -
  1\right)$, where $x, y, a, b > 0, x + y = 1, a + b = 1$.
\item Prove that $4\leq a^2 + b^2 + ab + \sqrt{4 - a^2}\sqrt{9 - b^2}\leq 19$, where $0\leq a\leq 2$ and $0\leq b\leq 3$.
\item Prove that $n\sqrt{m - 1} + m\sqrt{n - 1}\leq mn$, where $m\geq 1, n\geq 1$.
\item Prove that $\sqrt{m^2 - n^2} + \sqrt{2mn - n^2}\geq m$, where $m > n > 0$.
\item Prove that $x > \sqrt{x - 1} + \sqrt{x(\sqrt{x} - 1)}$, where $x\geq 1$.
\item Prove that $1 + \frac{1}{\sqrt{2}} + \cdots + \frac{1}{\sqrt{n}}\geq n\sqrt{\frac{2}{n + 1}}$, where $n\in\mathbb{N}$.
\item Prove that among seven arbitrary numbers one can find two numbers $x$ and $y$ such that $0\leq \frac{x - y}{1 + xy}
  <\frac{\sqrt{3}}{3}$.
\item Prove that $\frac{|a - b|}{\sqrt{1 + a^2}\sqrt{1 + b^2}}\leq \frac{|a - c|}{\sqrt{1 + a^2}\sqrt{1 + c^2}}\leq \frac{|b -
  c|}{\sqrt{1 + b^2}\sqrt{1 + c^2}}$.
\item {\it Huygen's inequality:} $\sqrt[n]{(a_1 + b_1)(a_2 + b_2)\cdots(a_n + b_n)}\geq \sqrt[n]{a_1a_2\ldots
  a_n}\sqrt[n]{b_1b_2\ldots b_n}$, where $a_i > 0, b_i > 0, i = 1, 2, \ldots, n$.
\item {\it Milne's inequality:} $\frac{a_1b_1}{a_1 + b_1} + \frac{a_2b_2}{a_2 + b_2} + \cdots + \frac{a_nb_n}{a_n + b_n}\leq$

  $\frac{(a_1 + a_2 + \cdots + a_n)(b_1 + b_2 + \cdots + b_n)}{(a_1 + a_2 + \cdots + a_n) + (b_1 + b_2 + \cdots + b_n)}$, where $a_i
  > 0, b_i > 0, i = 1, 2, \ldots, n$.
\item Prove that $\frac{8}{(x_1 + x_2)(y_1 + y_2) - (z_1 + z_2)^2} \leq \frac{1}{x_1y_1 - z_1^2} + \frac{1}{x_2y_2 - z_2^2}$, where
  $x_1 > 0, x_2 > 0$ and $x_1y_1 - z_1^2 > 0, x_2y_2 - z_2^2 > 0$.
\item Prove that $\sqrt{a - 1} + \sqrt{b - 1} + \sqrt{c - 1} \leq \frac{2}{3}\sqrt{abc}$, where $a\geq 1, b\geq 1, c\geq 1$.
\item Prove that $\sqrt{a - 1} + \sqrt{b - 1} + \sqrt{c - 1} + \sqrt{d - a}\leq \frac{3\sqrt{3}}{4}\sqrt{abcd}$, where $a\geq 1,
  b\geq 1, c\geq 1, d\geq 1$.
\item Prove that $\left(\frac{a^2 - b^2}{2}\right)^2 \geq \sqrt{\frac{a^2 + b^2}{2}} - \frac{a + b}{2}$, where $a, b\geq
  \frac{1}{2}$.
\item Prove that $x_1 + x_2 + \cdots + x_n \leq \frac{n}{3}$, where $x_1^3 + x_2^3 + \cdots + x_n^3 = 0$ and $x_i\in[-1, 1], i = 1,
  2, \ldots, n$.
\item Prove that $|x_1^3 + x_2^3 + \cdots + x_n^3|\leq 2n$, where $x_1 + x_2 + \cdots + x_n = 0$ and $x_i\in [-2, 2], i = 1, 2,
  \ldots, n$.
\item Prove that $1 < \frac{a}{\sqrt{a^2 + b^2}} + \frac{b}{\sqrt{b^2 + c^2}} + \frac{c}{\sqrt{c^2 + a^2}}\leq
  \frac{3\sqrt{2}}{4}$, where $a, b, c > 0$.
\item Prove that $\sqrt{1 - a} + \sqrt{1 - b} + \sqrt{1 - c} + \sqrt{1 - d}\geq \sqrt{a} + \sqrt{b} + \sqrt{c} + \sqrt{d}$, where
  $a, b, c, d > 0, a^2 + b^2 + c^2 + d^2 = 1$.
\item Prove that $\frac{a + b + c}{3} - \sqrt[3]{abc}\leq \max\left[\left(\sqrt{a} - \sqrt{b}\right)^2, \left(\sqrt{b} -
  \sqrt{c}\right)^2, \left(\sqrt{c} - \sqrt{a}\right)^2\right]$, where $a > 0, b > 0, c > 0$.
\item Given that $a^2 + b^2 = 1$. Prove that (i) $|a + b|\leq \sqrt{2}$, (ii) $|a - b|\leq \sqrt{2}$, (iii) $|ab|\leq \frac{1}{2}$,
  and (iv) $|ab^2 + a^2b|\leq \frac{1}{\sqrt{2}}$.
\item Prove that $|xy - \sqrt{(1 - x^2)(1 - y^2)}|\leq 1$, where $|x|\leq 1, |y|\leq 1$.
\item Prove that $\sqrt{1 - x^2} + \sqrt{1 - y^2}\leq 2\sqrt{1 - \left(\frac{x + y}{2}\right)^2}$, where $|x|\leq 1, |y|\leq 1$.
\item Prove that $\frac{a_1}{1 - a_1} + \frac{a_2}{1 - a_2} + \cdots + \frac{a_n}{1 - a_n}\geq \frac{n(a_1 + a_2 + \cdots + a_n)}{n
  - (a_1 + a_2 + \cdots + a_n)}$, where $0\leq a_1<1, 0\leq a_2<1, \ldots, 0\leq a_n<1$.
\item Prove that $\frac{1}{\sqrt{1 + a^2}} + \frac{1}{\sqrt{1 + b^2}} + \frac{1}{\sqrt{1 + c^2}}\leq \frac{3}{2}$, where $a, b, c
  > 0$ and $a + b + c = abc$.
\item Prove that $\frac{|x - y|}{1 + a|x - y|} + \frac{|y - z|}{1 + a|y - z|} \geq \frac{|x - z|}{1 + a|x - z|}$, where $a > 0$.
\item Prove that $\frac{2x(1 - x^2)}{(1 + x^2)^2} + \frac{2y(1 - y^2)}{(1 + y^2)^2} + \frac{2z(1 - z^2)}{(1 + z^2)^2}\leq
  \frac{x}{1 + x^2} + \frac{y}{1 + y^2} + \frac{z}{1 + z^2}$, where $x > 0, y > 0, z > 0$ and $xy + yz + zx = 1$.
\item Prove that $\sqrt{a_1 + a_2 + \cdots + a_n}\leq \sqrt{1}(\sqrt{a_1} - \sqrt{a_2}) + \sqrt{2}(\sqrt{a_2} - \sqrt{a_3}) +
  \cdots + \sqrt{n}(\sqrt{a_n} - \sqrt{a_{n + 1}})$, where $a_1\geq a_2\geq\ldots\geq a_{n + 1} = 0$.
\item Prove that $\frac{1}{\frac{1}{1 + a_1} + \frac{1}{1 + a_2} + \cdots + \frac{1}{1 + a_n}} - \frac{1}{\frac{1}{a_1} +
  \frac{1}{a_2} + \cdots + \frac{1}{a_n}}\geq \frac{1}{n}$, where $a_1 > 0, a_2 > 0, \ldots, a_n > 0$.
\item Prove that $a + b + c - 2\sqrt{abc}\geq ab + bc + ca - 2abc$, where $0\leq a\leq 1, 0\leq b\leq 1, 0\leq c\leq 1$.
\item Prove that $\sqrt{a(1 - b)(1 - c)} + \sqrt{b(1 - c)(1 - a)} + \sqrt{c(1- a)(1 - b)}\leq 1 + \sqrt{abc}$, where $0\leq a\leq
  1, 0\leq b\leq 1, 0\leq c\leq 1$.
\item Prove that $\left[(x + y)(y + z)(z + x)\right]^2\geq xyz(2x + y + z)(2y + z + x)(2z + x + y)$, where $x, y, z\geq 0$.
\item Prove that $\frac{ab(1 - a)(1 - b)}{(1 - ab)^2} < \frac{1}{4}$, where $0 < a < 1, 0 < b < 1$.
\item Prove that $\max(a_1, a_2, \ldots, a_n)\geq 2$, where $n > 3, a_1 + a_2 + \cdots + a_n \geq n, a_1^2 + a_2^2 + \cdots +
  a_n^2\geq n^2$.
\item Prove that $\sqrt{a_1 + \frac{(a_n - a_{n - 1})^2}{4(n - 2)}} + \cdots + \sqrt{a_{n - 2} + \frac{(a_n - a_{n - 1})^2}{4(n -
    2)}} + \sqrt{a_{n - 1}} + \sqrt{a_n}\leq \sqrt{n}$, where $n \geq 3, a_1, a_2, \ldots, a_n \geq 0$ and $a_1 + a_2 + \cdots +
  a_n = 1$.
\item Prove that $2\sqrt{(x^2 - 1)(y^2 - 1)}\leq 2(x - 1)(y - 1) + 1$, where $0\leq x, y\leq 1$.
\item Prove that $a^3 + b^3 + c^3 - 3abc\leq \sqrt{(a^2 + b^2 + c^2)^3}$.
\item Pprove that $\frac{1}{n - 1 + x_1} + \frac{1}{n - 1 + x_2} + \cdots + \frac{1}{n - 1 + x_n}\leq 1$, where $x_1, x_2, \ldots,
  x_n > 0$ and $x_1.x_2.\ldots.x_n = 1$.
\item Prove that $\frac{x}{\sqrt{1 - x}} + \frac{1}{\sqrt{1 - y}}\geq \frac{x + y}{\sqrt{1 - \frac{x + y}{2}}}$, where $0\leq x, y<
  1$.
\item Prove that $\frac{x_1}{\sqrt{1 - x_1}} + \frac{x_2}{\sqrt{1 - x_2}} + \cdots + \frac{x_n}{\sqrt{1 - x_n}}\geq
  \frac{\sqrt{x_1} + \sqrt{x_2} + \cdots + \sqrt{x_n}}{\sqrt{n - 1}}$, where $n\geq 2, n\in\mathbb{N}, x_1, x_2, \ldots, x_n > 0$
  and $x_1 + x_2 + \cdots + x_n = 1$.
\item Prove that $\frac{x}{\sqrt{4y^2 + 1}} + \frac{y}{\sqrt{4x^2 + 1}} \leq \frac{1}{\sqrt{2}}$, where $0\leq x,y\leq
  \frac{1}{2}$.
\item Prove that $0\leq ab + bc + ca - abc\leq 2$, where $a, b, c > 0$ and $a^2 + b^2 + c^2 + abc = 4$.
\item Prove that $a + b + c\leq 3$, whhere $a, b, c > 0$ and $a^2 + b^2 + c^2 + abc = 4$.
\item Prove that $(x - 1)(y - z)(z - 1)\leq 6\sqrt{3} - 10$, where $x, y, z > 0$ and $x + y + z = xyz$.
\item Prove that $\sqrt[3]{\frac{x + y}{2z}} + \sqrt[3]{\frac{y + z}{2x}} + \sqrt[3]{\frac{z + x}{2y}}\leq \frac{5(x + y + z) +
  9}{8}$, where $x, y, z > 0$ and $xyz = 1$.
\item Prove that among four arbitrary numbers there are two numbers $a$ and $b$ such that $\frac{1 + ab}{\sqrt{1 + a^2}\sqrt{1 +
    b^2}} > \frac{1}{2}$.
\item Given that $x + y + z = 0$ and $x^2 + y^2 + z^2 = 6$, find all possible values of the expression $x^2y + y^2z + z^2x$.
\item Let $(h_n)$ be a sequence such that $h_1 = \frac{1}{2}$ and $h_{n + 1} = \sqrt{\frac{1 - \sqrt{1 - h_n^2}}{2}}, n = 1, 2,
  \ldots$. Prove that $h_1 + h_2 + \cdots + h_n\leq 1.03$.
\item Prove that $abc\geq (a + b - c)(b + c - a)(c + a - b)$, where $a, b, c > 0$.
\item Prove that $(a_1b_1 + a_2b_2 + \cdots + a_nb_n)^2\leq (a_1^2 + a_2^2 + \cdots + a_n^2)(b_1^2 + b_2^2 + \cdots + b_n^2)$.
\item Prove that $(a + b)^2(a^2 + b^2)^2\cdots(a^n + b^n)^2\geq (a^{n + 1} + b^{n + 1})^n$, where $a, b > 0$.
\item Prove that $(a_1^\alpha + a_2^\alpha + \cdots + a_n^\alpha)^\beta\leq(a_1^\beta + a_2^\beta + \cdots + a_n^\beta)^\alpha$,
  where $0 < \beta < \alpha, a_1 > 0, a_2 > 0, \ldots, a_n > 0$.
\item {\it Nesbitt's inequality:} $\frac{a}{b + c} + \frac{b}{c + a} + \frac{c}{a + b}\geq \frac{3}{2}$, where $a, b, c > 0$.
\item Prove that $\sqrt{\frac{a}{b + c + d}} + \sqrt{\frac{b}{a + c + d}} + \sqrt{\frac{c}{a + b + d}} + \sqrt{\frac{d}{a + b + c}}
  > 2$, where $a, b, c, d > 0$.
\item Prove that $\sqrt[3]{\frac{abc + abd + acd + bcd}{4}}\leq \sqrt{\frac{ab + ac + ad + bc + bd + cd}{6}}$, where $a, b, c, d >
  0$.
\item Prove that $2\sqrt{ab + bc + ac}\leq 3\sqrt[3]{(b + c)(c + a)(a + b)}$, where $a, b, c > 0$.
\item Prove that $8(x^3 + y^3 + z^3)^2 \geq 9(x^2 + yz)(y^2 + xz)(z^2 + xy)$, where $x, y, z > 0$.
\item Prove that $4a^3 + 4b^3 + 4c^3 + 15abc\geq 1$, where $a, b, c\geq 0$ and $a + b + c = 1$.
\item Prove that $a^3 + b^3 + c^3 + abcd \geq min\left(\frac{1}{4}, \frac{1}{9} + \frac{d}{27}\right)$, where $a, b, c\geq 0$ and
  $a + b + c = 1$.
\item Prove that $\frac{a_1 + a_2 + \cdots + a_n}{n}\geq \frac{1}{n}\sqrt{\frac{a_1^2 + a_2^2 + \cdots + a_n^2}{n}} + \left(1 -
  \frac{1}{n}\right)\sqrt[n]{a_1a_2\ldots a_n}$, where $n\geq 2, a_i > 0, i = 1, 2, \ldots, n$.
\item {\it Turkevici's Inequality:} $a^4 + b^4 + c^4 + d^4 + 2abcd \geq a^2b^2 + a^2c^2 + a^2d^2 + b^2c^2 + b^2d^2 + c^2d^2$,
  where $a, b, c, d\geq 0$.
\item Prove that $\frac{a_1^3}{b_1} + \frac{a_2^3}{b_2} + \cdots + \frac{a_n^3}{b_n}\geq 1$, where $a_i, b_i > 0, i = 1, 2, \ldots,
  n$, and $(a_1^2 + a_2^2 + \cdots + a_n^2)^3 = b_1^2 + b_2^2 + \cdots + b_n^2$.
\item Prove that $\frac{a}{b} + \frac{b}{c} + \frac{c}{a}\geq \frac{a + c}{b + c} + \frac{b + a}{c + a} + \frac{c + b}{a + b}$,
  where $a, b, c > 0$.
\item Prove that $\sqrt{\frac{a_1^n}{a_1^n + \lambda a_1a_2\ldots a_n}} + \sqrt{\frac{a_2^n}{a_2^n + \lambda a_1a_2\ldots a_n}} +
  \cdots + \sqrt{\frac{a_n^n}{a_n^n + \lambda a_1a_2\ldots a_n}}\geq \frac{n}{\sqrt{1 + \lambda}}$, where $n\geq 2, a_1, 2_2,
  \ldots, a_n > 0$ and $\lambda \geq n^2 - 1$.
\item Prove that $\left(\sqrt[k]{2} - 1\right)(a_1 + a_2 + \cdots + a_n) < \sqrt[k]{2a_1^k + 2^2a_2^k + \cdots + 2^na_n^k}$, where
  $k\in\mathbb{N}, k\geq 2, a_1, a_2, \ldots, a_n > 0$.
\item Prove that $3(x^2y + y^2z + z^2x)(xy^2 + yz^2 + zx^2)\geq xyz(x + y + z)^3$, where $x, y, z > 0$.
\item Prove that $(x_1 + x_2 + \cdots + x_n + y_1 + y_2 + \cdots + y_n)^2 \geq 4n(x_1y_1 + x_2y_2 + \cdots + x_ny_n)$, where
  $x_1\leq x_2\leq\ldots\leq x_n\leq y_1\leq y_2\leq\ldots\leq y_n$.
\item Prove that $\frac{\ln z - \ln y}{z - y} < \frac{\ln z - \ln x}{z - x} < \frac{\ln y - \ln x}{y - x}$, where $0 < x < y < z$.
\item Prove that $a^bb^cc^dd^a\geq b^ac^bd^ca^d$, where $0\leq a\leq b\leq c\leq d$.
\item Prove that $\frac{x_1}{S - x_1} + \frac{x_2}{S - x_2} + \cdots + \frac{x_n}{S - x_n}\geq \frac{n}{n + 1}$, where $n\geq 2, S
  = x_1 + x_2 + \cdots + x_n, x_1, x_2, \ldots, x_n > 0$.
\item Prove that $a^3 + b^3 + c^3 + 6abc \geq \frac{1}{4}(a + b + c)^3$, where $a, b, c\geq 0$.
\item Prove that $a^2(2b + 2c - a) + b^2(2c + 2a - b) + c^2(2a + 2b - c)\geq 9abc$, where $a, b, c$ are side lengths of a triangle.
\item Prove that $\sqrt[n]{a_1a_2\ldots a_n} + \sqrt[n]{b_1b_2\ldots b_n} \leq \sqrt[n]{(a_1 + b_1)(a_2 + b_2)\ldots(a_n + b_n)}$,
  where $n\geq 2, a_i > 0, b_i > 0, i = 1, 2, \ldots, n$.
\item Prove that $\sqrt[n]{(n + 1)!} - \sqrt[n]{n!}\geq 1$, where $n\geq 2, n\in\mathbb{N}$.
\item Prove that $\sqrt[n]{F_{n + 1}} > 1 + \frac{`}{\sqrt[n]{F_n}}$, where $n\geq 2, F_1 = 1, F_2 = 2, F_{k + 2} = F_{k+1} + F_k,
  k = 1, 2, \ldots$.
\item Prove that $\sqrt[n]{C_{n+1}^n} > 2\left(1 + \frac{1}{\sqrt[n]{n + 1}}\right)$, where $n = 2, 3, \ldots$.
\item Prove that $(1 + a_1)(2 + a_2)\cdots(n + a_n)\geq n^{\frac{n}{2}}$, where $n\geq 2, n\in\mathbb{N}, a_1, a_2, \ldots, a_n >
  0$ and $a_1a_2\ldots a_n = 1$.
\item Prove that $\sqrt[n]{\frac{(a_1 + b_1)(a_2 + b_2)\ldots(a_n + b_n)}{(a_1 - c_1)(a_2 - c_2)\ldots(a_n - c_n)}}\geq
  \frac{\sqrt[n]{a_1a_2\ldots a_n} + \sqrt[n]{b_1b_2\ldots b_n}}{\sqrt[n]{a_1a_2\ldots a_n} - \sqrt[n]{c_1c_2\ldots c_n}}$, where
  $n\geq 2, n\in\mathbb{N}, b_i > 0, a_i > c_i > 0, i = 1, 2, \ldots, n$.
\item Prove that $\sqrt[3]{ab} + \sqrt[3]{cd}\leq \sqrt[3]{(a + c + d)(a + b + c)}$, where $a, b, ,c, d\geq 0$.
\item Prove that $x^(x^2 - 1)^2 + y^2(y^2 - 1)^2 \geq (x^2 - 1)(y^2 - 1)(x^2 + y^2 - 1)$.
\item Prove that $(x_1 - x_2)(x_1 - x_3)(x_1 - x_4)(x_1 - x_5) + (x_2 - x_1)(x_2 - x_3)(x_2 - x_4)(x_2 - x_5) + \cdots + (x_5 -
  x_1)(x_5 - x_2)(x_5 - x_3)(x_5 - x_4)\geq 0$.
\item Prove that $0\leq ab + bc + ca - abc\leq 2$, where $a, b, c\geq 0$ and $a^2 + b^2 + c^2 + abc = 4$.
\item Prove that $x^\lambda(x - y)(x - z) + y^\lambda(y - z)(y - x) + z^\lambda(z - y)(z - x)\geq 0$, where $x, y, z > 0$.
\item Prove that $\sqrt[3]{\left(\frac{a}{b + c}\right)^2} + \sqrt[3]{\left(\frac{b}{a + c}\right)^2} + \sqrt[3]{\left(\frac{c}{a +
    b}\right)^2}\geq \frac{3}{\sqrt[3]{4}}$, where $a, b, c > 0$.
\item Prove that $(a^5 - a^2 + 3)(b^5 - b^2 + 3)(c^5 - c^2 + 3)\geq (a + b + c)^3$, where $a, b, c > 0$.
\item Prove that $abc + abd + bcd + acd - abcd\leq 3$, where $a, b, c, d > 0$ and $a^3 + b^3 + c^3 + d^3 + abcd = 5$.
\item Prove that $0\leq ab + bc + ca - abc\leq 2$, where $a, b, c\geq 0$ and $a^2 + b^2 + c^2 + abc = 4$.
\item Prove that $a^2 + b^2 + c^2 + 2abc + 1\geq (ab + bc + ca)$, where $a, b, c\geq 0$.
\item Prove that $\frac{x + y + z}{xy + yz + zx}\leq 1 + \frac{1}{48}[(x - y)^2 + (y - z)^2 + (z - x)^2]$, where $x, y, z > 0$ and
  $xy + yz + zx + xyz = 4$.
\item Let $a_1, a_2, \ldots, a_{n +1}$ be $n + 1$ positive real numbers such that $a_1 + a_2 + \cdots + a_n = a_{n + 1}$. Prove
  that $\displaystyle\sum_{i=1}^n\sqrt{a_i(a_{n + 1}) - a_i} \leq \sqrt{\sum_{i = 1}^na_{n + 1}(a_{n + 1} - a_i)}$.
\item Prove that $\frac{a}{b + 2c} + \frac{b}{c + 2a} + \frac{c}{a + 2b}\geq 1$, where $a, b, c >0$ and $a, b, c\in\mathbb{R}$.
\item Prove that $a^2 + b^2 + c^2 \geq \sqrt{3}abc$, where $a, b, c >0$ and $a, b, c\in\mathbb{R}$ such that $abc\leq a + b + c$.
\item For any positive real numbers $a, b, c$ prove that $\frac{2}{b(a + b)} + \frac{2}{c(b + c)} + \frac{2}{a(c + a)}\geq
  \frac{27}{(a + b + c)^2}$.
\item Let $a, b, c$ be three sides of a triangle such that $a + b + c = 2$. Prove that $1\leq ab + bc + ca - abc\leq 1 +
  \frac{1}{27}$.
\item If $a, b, c$ be positive real numbers such that $a + b + c = 1$. Prove that $\sqrt{ab + c} + \sqrt{bc + a} + \sqrt{ca +
  b}\geq 1 + \sqrt{ab} + \sqrt{bc} + \sqrt{ca}$.
\item If $a, b, c, d$ are positive real numbers, prove that $\sqrt{\frac{a^2 + b^2 + c^2 + d^2}{4}}\geq$

  $\sqrt[4]{\frac{abc + bcd + cda + abd}{4}}$.
\item Let $a, b, c$ be the sides of a triangle such that $a + b + c = 2$. Prove that $a^2 + b^2 + c^2 + 2abc < 2$.
\item If $a, b, c$ are positive real numbers such that $a^2 + b^2 + c^2 = 1$, prove that $\left(\frac{1}{a} + \frac{1}{b} +
  \frac{1}{c}\right) + a + b + c\geq 4\sqrt{3}$.
\item Find all triples $(a, b, c)$ of real numbers which satisfy the system of equations:
  \startformula a + b + c = 6,\stopformula
  \startformula \frac{1}{a} + \frac{1}{b} + \frac{1}{c} = 2 - \frac{4}{abc}.\stopformula
\item Let $a, b, c$ be real numbers such that $a^2 + b^2 + c^2 = 1$. Prove that $\frac{a^2}{1 + 2bc} + \frac{b^2}{1 + 2ca} +
  \frac{c^2}{1 + 2ab}\geq \frac{3}{5}$.
\item Let $a, b, c$ and $\alpha, \beta, \gamma$ be positive real numbers such that $\alpha + \beta + \gamma = 1$. Prove that
  $b\alpha + b\beta + c\gamma + 2\sqrt{(\alpha\beta + \beta\gamma + \gamma\alpha)(ab + bc + ca)}\leq a + b + c$.
\item Prove that for all real numbers $a$ and $b, a^2 + b^2 + 1 > a\sqrt{b^2 + 1} + b\sqrt{a^2 + 1}$.
\item For a fixed positive integer $n$, compute the minimum value of the sum $x_1 + \frac{x_2^2}{2} + \frac{x_3^3}{3} + \cdots +
  \frac{x_n^n}{n}$, where $x_1, x_2, \ldots, x_n$ are positive real numbers such that $\frac{1}{x_1} + \frac{1}{x_2} + \cdots +
  \frac{1}{x_n} = n$.
\item Let $a, b, c, d$ be positive real numbers such that $a + b + c + d\leq 1$. Prove that $\frac{a}{b} +  \frac{b}{c} +
  \frac{c}{d} + \frac{d}{a}\leq \frac{1}{64abcd}$.
\item Let $a, b, c$ be positive real numbers, all less than $1$, such that $a + b + c = 2$. Prove that $abc\geq 8(1 - a)(1 - b)(1 -
  c)$.
\item Prove that $\frac{(2a + b + c)^2}{2a^2 + (b + c)^2} + \frac{(2b + c + a)^2}{2b^2 + (c + a)^2} + \frac{(2c + a + b)^2}{2c^2 +
  (a + b)^2}\leq 8$, where $a, b, c$ are positive real numbers.
\item Prove that $abc\leq 1$, where $a, b, c$ are real numbers such that $(1 + a)(1 + b)(1 + c) = 8$.
\item Prove that $\displaystyle\sum_{i=1}^n\frac{a_i}{2 - a_i}]geq \frac{n}{2n - 1}$, where $a_1, a_2, \ldots, a_n \in\mathbb{R},
  n\geq 2$ such that $\displaystyle\sum_{i=1}^na_i = 1$.
\item Prove that $\displaystyle\sum_{i=1}^n\frac{a_i^2}{a_i + a_{i + 1}}\geq \frac{1}{2}$, where $a_1, a_2, \ldots, a_n$ are
  positive numbers such that $\displaystyle\sum_{i=1}^na_i = 1$ and $a_1 = a_{n + 1}$.
\item Prove that $\frac{1}{a} + \frac{4}{b} + \frac{9}{c} + \frac{16}{d}\geq \frac{100}{a + b + c + d}$, where $a, b, c,
  d\in\mathbb{R}$.
\item Prove that $\displaystyle\sum_{i=1}^n\frac{a_i^2}{1 - 2a_i}\geq \frac{1}{n - 2}$, where $n > 2, 0 < a_1, a_2, \ldots, a_n <
  \frac{1}{2}$ such that $\displaystyle\sum_{i=1}^na_i = 1$.
\item Prove that $x_1 + x_2 + \cdots + x_n\leq \frac{x_1}{y_1} + \frac{x_2}{y_2} + \cdots + \frac{x_n}{y_n}$, where $n\geq 2, x_1 +
  x_2 + \cdots + x_n\geq x_1y_1 + x_2y_2 + \cdots + x_ny_n$ and $x_1, x_2, \ldots, x_n, y_1, y_2, \ldots, y_n$ are positive real
  numbers.
\item If $x_1, x_2, \ldots, x_n$ are $n$ positive real numbers, prove that $\frac{x_1}{1 + x_1^2} + \frac{x_2}{1 + x_1^2 + x_2^2} +
  \cdots +$

  $\frac{x_n}{1 + x_1^2 + x_2^2 + \cdots + x_n^2} < \sqrt{2}$.
\item If $a, b, c$ are poositive real numbers, prove that $3(a^2b + b^2c + c^2a)(ab^2 + bc^2 + ca^2)\geq abc(a + b + c)^3$.
\item Let $P(x) = ax^2 + bx + c$ be a quadratic polynomial with non-negative coefficients and let $\alpha$ be a positive real
  number. Prove that $P(\alpha)P(1/\alpha)\geq P(1)^2$.
\item If $a, b, c, d, e$ are positive, real numbers, prove that $\displaystyle\sum\frac{a}{b + c}\geq \frac{5}{2}$ where sum is
  taken cyclically over $a, b, c, d, e$.
\item Let $a, b, c$ be non-negative real numbers such that $\frac{1}{a^2 + 1} + \frac{1}{b^2 + 1} + \frac{1}{c^2 + 1} = 2$. Prove
  that $ab + bc + ca\leq \frac{3}{2}$.
\item Suppose $a, b, c$ are positive real numbers. Prove that $3(a + b + c)\geq 8\sqrt[3]{abc} + \sqrt[3]{\frac{a^3 + b^3 +
    c^3}{3}}$. When does equality hold?
\item Let $c_1, c_2, \ldots, c_n$ be $n$ real numbers such that either $0\leq c_i\leq 1$ for all $i$ or $c_i\geq 1$ for all
  $i$. Prove that the inequality $\displaystyle\prod_{i=1}^n(1 - p +pc_i)\leq 1 - p + p\prod_{i=1}^nc_i$ holds, for any real $p$
  with $0\leq p\leq 1$.
\item Let $x_1, x_2, x_3, x_4$ be real numbers in the interval $(0, 1/2]$. Prove that

  $\frac{x_1x_2x_3x_4}{(1 - x_1)(1 - x_2)(1 -
    x_3)(1 - x_4)}\leq \frac{x_1^4 + x_2^4 + x_3^4 + x_4^4}{(1 - x_1)^4 + (1 - x_2)^4 + (1 - x_3)^4 + (1 - x_4)^4}$.
\item If $x_1, x_2, \ldots, x_n$ be $n$ real numbers such that $x_i\in(0, 1/2]$. Prove that
  $\displaystyle\frac{\prod_{i=1}^nx_i}{\left(\sum_{i=1}^nx_i\right)^n}\leq \frac{\prod_{i=1}^n(1 - x_i)}{\left(\sum_{i=1}^n(1 -
    x_i)\right)^n}$.
\item Consider a sequence $\langle a_i\rangle$ of real numbers satisfying $a_{i + j}\leq a_i + a_j$. Prove that $a_1 +
  \frac{a_2}{2} + \cdots + \frac{a_n}{n}\geq a_n,\;\forall n$.
\item For positive real numbers $x, y, z,$ prove that $\displaystyle\sum\frac{x}{x + \sqrt{(x + y)(x + z)}}\leq 1$, where the sum
  is taken cyclically over $x, y, z$.
\item Let $x, y$ be non-negative real numbers such that $x + y = 2$. Prove that $x^3y^3(x^3 + y^3)\leq 2$.
\item Let $\langle a_i\rangle$ and $\langle b_i\rangle$ be two sequences such that $0 < h\leq a_i\leq H$ and $0 < m\leq b_i\leq M$
  for real $h, H, m, M$. Prove that $1\leq \frac{(\sum a_i^2)(\sum b_i^2)}{(\sum a_ib_i)^2}\leq
  \frac{1}{4}\left(\sqrt{\frac{HM}{hm}} + \sqrt{\frac{hm}{HM}}\right)^2$.
\item Let $f:[0, a]\rightarrow \mathbb{R}$ be a convex function. Consider $n$ points $x_1, x_2, \ldots, x_n$ in $[0, a]$ such that
  $\sum_{i=1}^nx_i$ is also in $[0, a]$. Prove that $\displaystyle\sum_{i=1}^nf(x_i)\leq f\left(\sum_{i=1}^nx_i\right) + (n -
  1)f(0)$.
\item For any real number $n$, prove that $\displaystyle\binom{2n}{n}\sqrt{3n} < 4^n$.
\item Let $a, b, c$ be positive real numbers and let $x$ be a non-negative real number. Prove that $a^{x + 2} + b^{x + 2} + c^{x +
  2}\geq a^xbc + ab^xc + abc^x$.
\item Let $(a_1, a_2, \ldots, a_n), (b_1, b_2, \ldots, b_n)$ and $(c_1, c_2, \ldots, c_n)$ be three seuqnences of positive real
  numbers. Prove that $\displaystyle\sum_{i=1}^na_ib_ic_i\leq
  \sqrt[3]{\sum_{i=1}^na_i^3}\sqrt[3]{\sum_{i=1}^nb_i^3}\sqrt[3]{\sum_{i=1}^nc_i^3}$.
\item Prove for any three real numbers $a, b, c,$ the inequality $3(a^2 - a - 1)(b^2 - b - 1)(c^2 - c -1)\geq (abc)^2 - abc + 1$.
\item Consider a polynomial of the form $P(x) = x^n + a_{n - 1}x^{n - 1} + \cdots + a_1x + 1$, where $a_i \geq 0\;\forall 1\leq
  i\leq n - 1$. Suppose $P(x) = 0$ has $n$ real roots. Prove that $P(2)\geq 3^n$.
\item Let $a_1<a_2<\cdots<a_n$ be $n$ positive integers. Prove that $(a_1 + a_2 + \cdots + a_n)^2\leq a_1^3 + a_2^3 + \cdots +
  a_n^3$.
\item Consider a sequence $a_1, a_2, \ldots, a_n$ of positive real numbers which add up to $1$, where $n\geq 2$ is an integer.
  Prove that for any positive real numbers $x_1, x_2, \ldots, x_n$ with $\displaystyle\sum_{i=1}^nx_i = 1$, the inequality
  $\displaystyle2\sum_{i<j}x_ix_j\leq \frac{n - 2}{n + 1} + \sum_{i=1}^n\frac{a_ix_i^2}{1 - a_i}$, holds.
\item Let $x_1, x_2, x_3, x_4$ be four consecutive positive real numbers such that $x_1x_2x_3x_4 = 1$. Prove that $x_1^3 + x_2^3 +
  x_3^3 + x_4^3\geq \min\left(x_1 + x_2 + x_3 + x_4, \frac{1}{x_1} + \frac{1}{x_2} + \frac{1}{x_3} + \frac{1}{x_4}\right)$.
\item Let $\{x\}$ denote the fractional part of $x$ i.e. $\{x\} = x - \lceil x\rceil$. Prove for any positive integer $n,
  \displaystyle\sum_{i=1}^n\{\sqrt{i}\}\leq \frac{n^2 - 1}{2}$.
\item If $a, b, c$ are positive real numbers, prove that $\frac{a^2}{(a + b)(a + c)} + \frac{b^2}{(b + c)(b + a)} + \frac{c^2}{(c +
  a)(c + b)}\geq \frac{3}{4}$.
\item Let $a, b, c$ be positive real numbers such that $abc > ab + bc + ca$. Prove that $abc\geq 3(a + b + c)$.
\item Let $a_1, a_2, \ldots, a_n$ be $n$ non-negative real numbers and let $a$ denote the sum of these numbers. Prove that
  $\displaystyle\sum_{i=1}^{n - 1}a_ia_{i + 1}\leq \frac{a^2}{4}$.
\item Let $a, b, c, d$ be complex numbers such that $ac\neq 0$. Prove that $\frac{\max(|ac|, |ad + bc|, |bd|)}{\max(|a|, |b|)(|c|,
  |d|)}\geq \frac{-1 + \sqrt{5}}{2}$.
\item Let $x_1, x_2, x_3, x_4$ be non-negative real numbers such that $\displaystyle\sum_{i=1}^n\frac{1}{1 + x_i}\leq 1$. Prove
  that $x_1x_2\cdots x_n\geq (n - 1)^n$.
\item Prove that $\frac{1}{m + n - 1} - \frac{1}{(m + 1)(n + 1)}\leq \frac{4}{45}$ for any two natural numbers $m$ and $n$.
\item If $a, b$ are two positive real numbers, prove that $a^b + b^a > 1$.
\item Let $a, b$ be positive real numbers such that $a + b = 1$ and let $p$ be a positive real. Prove that $\left(a +
  \frac{1}{a}\right)^p + \left(b + \frac{1}{b}\right)^p \geq \frac{5^p}{2^{p - 1}}$.
\item Let $a, b, c$ be positive real numbers such that $abc = 1$. Prove that $\left(a - 1 + \frac{1}{b}\right)\left(b - 1 +
  \frac{1}{c}\right)$

  $\left(c - 1 + \frac{1}{a}\right)\leq 1$.
\item Let $x, y, z$ be real numbers in the interval $[-1, 2]$ such that $x + y + z = 0$. Prove that $\frac{(2 - x)(2 - y)}{(2 +
  x)(2 + y)} + \sqrt{\frac{(2 - y)(2 - z)}{(2 + y)(2 + z)}} + \sqrt{\frac{(2 - z)(2 - x)}{(2 + z)(2 + x)}}\geq 3$.
\item Let $\langle a_n\rangle$ be a sequence of distinct positive integers. Prove that $\displaystyle\sum_{i=1}\frac{a_i}{i^2}\geq
  \sum_{i = 1}^n\frac{1}{i}$, for every positive integer $n$.
\item Let $x, y, z$ be non-negative real numbers such that $x + y + z = 1$. Prove that $0\leq xy + yz + zx - 2xyz\leq
  \frac{7}{27}$.
\item Let $x_1, x_2, \ldots, x_n$ be $n$ positive real numbers. Prove that $\displaystyle\sum_{i=1}^n\frac{x_i^3}{x_i^2 + x_ix_{i +
    1} + x_{i + 1}^2}\geq \frac{1}{3}\sum_{i=1}^nx_i$, where $x_1 = x_{n + 1}$.
\item Suppose $x, y, z$ are non-negative real numbers. Prove that $x(x - z)^2 + y(y - z)^2\geq (x - z)(y - z)(x + y - z)$.
\item Prove that $\frac{a}{b} + \frac{b}{c} + \frac{c}{a}\geq \frac{c + a}{c + b} + \frac{a + b}{a + c} + \frac{b + c}{b + a}$,
  where $a, b, c$ are positive real numbers.
\item If $a, b$ are real numbers, prove that $a^2 + ab + b^2\geq 3(a + b - 1)$.
\item Define a sequence $\langle x_n\rangle$ by $x_1 = 2, x_{n + 1} = \frac{x_n^4 + 9}{10x_n}$. Prove that $\frac{4}{5}< x_n \leq
  \frac{5}{4}\;\forall\;n > 1$.
\item Let $a, b, c$ be positive real numbers such that $a^2 - ab + b^2 = c^2$. Prove that $(a - c)(b - c)\leq 0$.
\item Let $a, b, c$ be positive real numbers. Prove that $\sqrt{a^2 - ab + b^2} + \sqrt{b^2 - bc + c^2}\geq \sqrt{a^2 + ac + c^2}$.
\item For all real numbers $a$, show that $(a^3 - a + 2)^2 \geq 4a^2(a^2 + 1)(a - 2)$.
\item Let $a, b, c$ be distinct real numbers. Prove that $\left(\frac{2a - b}{a - b}\right)^2 + \left(\frac{2b - c}{b -c}\right) +
  \left(\frac{2c - a}{c - a}\right)^2\geq 5$.
\item Let $\alpha, \beta, x_1, x_2, \ldots, x_n$ be positive reals such that $\alpha + \beta = 1$, and $x_1 + x_2 + \cdots + x_n =
  1$. Prove that $\displaystyle\sum_{i=1}^n\frac{x_i^{2m + 1}}{\alpha x_i + \beta x_{i + 1}}\geq \frac{1}{n^{2m - 1}}$ for every
  positive integer $m$, where $x_{n + 1} = x_1$.
\item Given positive reals $a, b, c, d$, prove that $\sqrt{(a + c)^2 + (b + d)^2}\leq\sqrt{a^2 + b^2} + \sqrt{c^2 +
  d^2}\leq\sqrt{(a + c)^2 + (b + d)^2} + \frac{2|ad - bc|}{\sqrt{(a + c)^2 + (b + d)^2}}$.
\item With every natural number $n$, associate a real number $a_n$ by $a_n = \frac{1}{p_1} + \frac{1}{p_2} + \cdots +
  \frac{1}{p_k}$, where $\{p_1, p_2, \ldots, p_k\}$ is the set of all prime divisors of $n$. Show that for any natural number
  $N\geq 2, \displaystyle\sum_{i=2}^Na_1a_2\ldots a_n < 1$.
\item Let $n$ be a fixed integer, with $n\geq 2$. Determine the least constant $C$ such that the inequality
  $\displaystyle\sum_{1\leq i<j\leq n}x_ix_j(x_i^2 + x_j^2)\leq C\left(\sum_{1\leq i\leq n}x_i\right)^4$ holds for all real numbers
  $x_1, x_2, \ldots, x_n$. Determine when the equality holds.
\item Let $a, b, c, d$ be real numbers such that $(a^2 + b^2 - 1)(c^2 + d^2 - 1) > (ac + bd - 1)^2$. Prove that $a^2 + b^2 - 1 > 0$
  and $c^2 - d^2 - 1 > 0$.
\item Let $x_1, x_2, \ldots, x_{100}$ be $100$ positive integers such that $\frac{1}{\sqrt{x_1}} + \frac{1}{\sqrt{x_2}} + \cdots +
  \frac{1}{\sqrt{x_{100}}} = 20$. Prove that at least two of the $x_i$'s are equal.
\item Let $f(x)$ be a polynomial with integer coefficients and of degree $n > 1$. Suppose $f(x) = 0$ has $n$ real roots in the
  interval $(0, 1)$, not all equal. If $a$ is the leading coefficient of $f(x)$, prove that $|a| \geq 2^n + 1$.
\item Show that the equation $\frac{x}{y} + \frac{y}{z} + \frac{z}{w} + \frac{w}{x} = m$, has no solutions in positive reals for $m
  = 2, 3$.
\item Solve the system of equations: $x = \frac{4z^2}{1 + 4z^2}, y = z = \frac{4x^2}{1 + 4x^2}$, for real numbers $x, y, z$.
\item Suppose $a, b$ are non-zero real numbers and that all the roots of the real polynomial $ax^n - ax^{n - 1} + a_{n - 1}x^{n -
  2} + \cdots + a_2x^2 - n^2bx + b = 0$ are real and positive. Prove that all the roots are in fact equal.
\item Find all triples $(a, b, c)$ of positive integers such that product of any two leaves a remainder $1$ when divided by the
  third number.
\item Find all positive solutions of the system: $x_1 + \frac{1}{x_2} = 4, x_2 + \frac{1}{x_3} = 1, \cdots, x_{1999} +
  \frac{1}{x_{2000}} = 4, x_{2000} + \frac{1}{x_1} = 1$.
\item Find all positive solutions of the system: $x + y + z = 1, x^3 + y^3 + z^3 + xyz = x^4 + y^4 + z^4 + 1$.
\item Let $a, b$ be positive integers such that each equation $(a + b - x)^2 = a - b, (ab + 1 - x)^2 = ab - 1$ has two distinct
  real roots. Suppose the bigger of these roots are the same. Show that the smaller roots are also the same.
\item Suppose the polynomial $P(x) = x^n + nx^{n - 1} + a_2x^{n - 2} + \cdots + a_n$ has real roots $\alpha_1, \alpha_2, \ldots, \alpha_n$. If $\alpha_1^{16} + \alpha_2^{16} + \cdots + \alpha_n^{16} = n$. Find $\alpha_1,\alpha_2, \ldots, \alpha_n$.
\item Find all the solutions of the following system of inequalities:
  \startformula (x_1^2 - x_3x_5)(x_2^2 - x_3x_5)\leq 0,\stopformula
  \startformula (x_2^2 - x_4x_1)(x_3^2 - x_4x_1)\leq 0,\stopformula
  \startformula (x_3^2 - x_5x_2)(x_4^2 - x_5x_2)\leq 0,\stopformula
  \startformula (x_4^2 - x_1x_3)(x_5^2 - x_1x_3)\leq 0,\stopformula
  \startformula (x_5^2 - x_2x_4)(x_1^2 - x_2x_4)\leq 0.\stopformula
\item Solve the following system of equations, when $a$ is a real number such that $|a| > 1$:
  \startformula \startalign\NC x_1^2 \NC = ax_2 + 1,\NR\NC x_2^2 & = ax_3 + 1,\NR\NC\vdots \NC \;\;\;\l\vdots \NR\NC x_{999}^2 \NC = ax_{1000} + 1,\NR\NC x_{1000}^2 \NC =
  ax_1 + 1.\stopalign\stopformula
\item Let $a_1, a_2, \ldots, a_n$ be $n$ positive integers such that $\displaystyle\sum_{i=1}^na_i = \prod_{i=1}^na_i$. Let $K_n$
  denote this common value. Show that $K_n \geq n + s$, where $s$ is the least positive integer such that $2^s - s\geq n$.
\item Let $z_1, z_2, z_3, \ldots, z_n$ be $n$ complex numbers such that $\displaystyle_{i=1}^n|z_i| = 1$. Prove that there exists a
  subset $S$ of the set $\{z_1, z_2, \ldots, z_n\}$ such that $\displaystyle\left|\sum_{z\in S}z\right|\geq \frac{1}{4}$.
\item Let $\langle a_n\rangle$ and $\langle b_n\rangle$ be two sequences of real numbers which are not proportional. Let $\langle
  x_n\rangle$ such that $\displaystyle\sum_{i=1}^na_ix_i = 0, \sum_{i=1}^nb_ix_i = 1$. Prove that $\displaystyle\sum_{i=1}^nx_i^2
  \geq $

  $\frac{\sum_{i=1}^na_i^2}{\left(\sum_{i=1}^na_i^2\right)\left(\sum_{i=1}^nb_i^2\right) - \left(\sum_{i=1}^na_ib_i\right)^2}$.
  When does equality hold?
\item Let $x_1, x_2, \ldots, x_n$ be $n$ positive real numbers. Prove that $\displaystyle\sum_{i=1}^n\frac{x_i}{2x_i + x_{i + 1} +
  \cdots + x_{i + n - 2}}\leq n$, where $x_{n + i} = x_i$.
\item Let $x_1, x_2, \ldots, x_n$ be $n\geq 2$ positive real numbers and $k$ be a fixed integer such that $1\leq k\leq n$. Show
  that $\displaystyle\sum_{\rm cyclic}\frac{x_1 + 2x_2 + \cdots + 2x_{k - 1} + x_k}{x_k + x_{k + 1} + \cdots + x_n}\geq \frac{2n(k
    - 1)}{n - k + 1}$.
\item If $z_1$ and $z_2$ be two complex numbers such that $|z_1| \leq r, |z_2|\leq r$ and $z_1\neq z_2$. Prove that for any natural
  number $n\left|\frac{z_1^n - z_2^n}{z_1 - z_2}\right|\leq \frac{1}{2}n(n - 1)r^{n - 2}|z_1 - z_2|$.
\item A sequence $\langle a_n\rangle$ is said to be convex if $a_n - 2a_{n + 1} + a_{n + 2}\geq 0$ for all $n\geq 1$. Let $a_1,
  a_2, \ldots, a_{2n + 1}$ be a convex sequence. Show that $\frac{a_1 + a_3 + \cdots + a_{2n + 1}}{n + 1}\geq \frac{a_2 + a_4 +
    \cdots + a_{2n}}{n}$, and equality holds if and only if $a_1, a_2, \ldots, a_{2n + 1}$ is an arithmetic progression.
\item Suppose $a_1, a_2, \ldots, a_n$ are $n$ positive real numbers. For each $k$, define $x_i = a_{i + 1} + a_{i + 2} + \cdots +
  a_{i + n - 1} - (n - 2)a_i$, where $a_i = a_{i - n}$ for $i > n$. Suppose $x_k\geq 0$ for $1\leq i\leq n$. Prove that
  $\displaystyle \prod_{i=1}^na_i\geq\prod_{i = 1}^nx_i$. Show that for $n = 3$ the inequality is still true without the
  non-negativity of $x_i$'s, but for $n > 3$ these conditions are essential.
\item Let $a, c$ be positive reals and $b$ be a complex number such that $f(z) = a|z|^2 + 2Re(bz) + c\geq 0$, for all complex
  numbers $z$, where $Re(z)$ denoted the real part of $z$. Prove that $|b|^2\leq ac$, and $f(z)\leq (a + c)(1 + |z|^2)$. Show that
  $|b|^2 = ac$ only if $f(z) = 0$ for some $z\in\mathbb{C}$.
\item Suppose $x_1\leq x_2\leq \cdots\leq x_n$ be $n$ real numbers. Show that $\displaystyle\left(\sum_{i=1}^n\sum_{j=1}^n|x_i -
  x_j|\right)^2\leq \frac{2(n^2 - 1)}{3}\sum_{i=1}^n\sum_{j = 1}^n(x_i - x_k)^2$. Prove also that equality holds if and only if
  the sequence $\langle x_i\rangle$ is in A.P.
\item Suppose $\langle a_n\rangle$ is an infinite sequence of real numbers with the properties
  \startitemize[n]
  \item there is some real constant $c$ such that $0\leq a_n\leq c$, for all $n\geq 1$, and
  \item $|a_i - a_j|\geq \frac{1}{i + j}\;\forall i\neq j$.
  \stopitemize
  Prove that $c\geq 1$.
\item Let $a, b, c$ be positive reals such that $a + b + c = 1$. Prove that $a(1 + b - c)^{1/3} + b(1 + c - a)^{1/3} + c(1 + a -
  b)^{1/3}\leq 1$.
\item let $x_1, x_2, \ldots, x_n$ be $n$ positive reals which add up to $1$. Find the minimum value of $\displaystyle\sum_{i = 1}^n
  \frac{x_i}{1 + \sum_{j\neq i}x_j}$.
\item If $a, b, c, d$ are positive reals then find all possible values of $\frac{a}{a + b + d} + \frac{b}{a + b + c} + \frac{c}{b +
  c + d} + \frac{d}{a + c + d}$.
\item Let $\langle F_n\rangle$ be the Fibonacci sequence defined by $F_1 = F_2 = 1, F_{n + 2} = F_{n + 1} + F_n$, for $n\geq
  1$. Prove that $\displaystyle\sum_{i=1}^n\frac{F_i}{2^i} < 2$ for all $n\geq 1$.
\item Let $P(x) = x^n + a_{n - 1}x^{n - 1} + \cdots + a_0$ be a polynomial with real coefficients such that $|P(0)| =
  P(1)$. Suppose all the roots of $P(x) = 0$ are real and lie in the interval $(0, 1)$. Prove that the product of the roots does
  not exceed $\frac{1}{2^n}$.
\item If $x, y$ are real numbers such that $2x + y + \sqrt{8x^2 + 4xy + 32y^2} = 3 + 3\sqrt{2}$, prove that $x^2y \leq 1$.
\item Determine the maximum value of $\displaystyle\sum_{i < j}x_ix_j(x_i + x_j)$, over all $n$-tuples $(x_1, x_2, \ldots, x_n)$ of
  reals such that $x_i\geq 0$ for $1\leq i\leq n$.
\item Let $x_1, x_2, \ldots, x_n$ be positive real numbers. Prove that $\sum_{i=1}^n(x_1x_2\cdots x_i)^{1/i} <
  3\left(\displaystyle\sum_{i=1}^nx_i\right)$.
\item Let $a_1\leq a_2\leq \cdots\leq a_n$ be $n$ real numbers with the property $\displaystyle_{i=1}^na_i =
  0$. Prove that $na_1a_n\displaystyle\sum_{i=1}^na_i^2\leq 0$.
\item Let $a, b, c$ be positive real numbers. Prove that $\frac{1}{a(1 + b)} + \frac{1}{b(1 + c)} + \frac{1}{c(1 +
  a)}\geq \frac{3}{1 + abc}$.
\item Let $x, y, z$ be positive real numbers such that $x^2 + y^2 + z^2 = 2$. Prove that $x + y + z\leq 2 +
  xyz$. Find the conditions under which equality holds.
\item Let $0\leq x_1\leq x_2\leq\cdots\leq x_n$ be such that $\displaystyle\sum_{i=1}^nx_i = 1$, where $n\geq 2$ is
  an integer. If $x_n\leq \frac{2}{3}$, prove that there exists a $j$ such that $1\leq j\leq n$ and
  $\displaystyle\frac{1}{3}\leq \sum_{i=1}^jx_i\leq \frac{2}{3}$.
\item Let $x, y, z$ be non-negative real numbers such that $xy + yz + zx + xyz = 4$. Prove that $x + y + z \geq xy
  + yz + zx$.
\item Let $x, y, z$ be non-negative real numbers such that $x + y + z = 1$. Prove that $x^y + y^z + z^x \leq
  \frac{4}{27}$.
\item Let $x, y, z$ be real numbers and let $p, q, r$ be real numbers in the interval $\left(0, \frac{1}{2}\right)$
  such that $p + q + r = 1$. Prove that $pqr(x + y + z)^2\geq xyr(1 - 2r) + yzp(1 - 2p) + zxq(1 - 2q)$. When does
  equality hold?
\item Let $x_1, x_2, \ldots, x_n$ be $n$ real numbers in the interval $[0, 1]$. Prove that
  $\displaystyle\left(\sum_{i = 1}^nx_i\right) - \left(\sum_{i=1}^nx_ix_{i + 1}\right)\leq
  \left[\frac{n}{2}\right]$, where $x_{n + 1} = x_1$.
\item Suppose $x, y, z$ are positive real numbers such that $xyz\geq 1$. Prove that $\frac{x^5 - x^2}{x^5 + y^2 +
  z^2} + \frac{y^5 - y^2}{y^5 + z^2 + x^2} + \frac{z^5 - z^2}{z^5 + x^2 + y^2}\geq 0$.
\item Consider two sequences of positive real numbers, $a_1\leq a_2\leq \cdots\leq a_n$ and $b_1\leq b_2\leq \cdots
  \leq b_n$, such that $\displaystyle\sum_{i=1}^na_i\geq\sum_{i=1}^nb_i$. Suppose there exists a $j, 1\leq j\leq
  n$, such that $b_i\leq a_i$ for $1\leq i\leq j$ and $b_i\geq a_i$ for $i > j$. Prove that
  $\displaystyle\prod_{i=1}^na_i\geq\prod_{i=1}^nb_i$.
\item Let $a, b, c$ be positive real numbers such that $abc = 1$. Prove that $\frac{1}{1 + a + b} + \frac{1}{1 + b
  + c} + \frac{1}{1 + c + a}\leq\frac{1}{2 + a} + \frac{1}{2 + b} + \frac{1}{2 + c}$.
\item Let $n\geq 4$ and let $a_1, a_2, \ldots, a_n$ be real numbers such that $a_1 + a_2 + \cdots + a_n\geq n,
  a_1^2 + a_2^2 + \cdots + a_n^2\geq n^2$. Prove that $\max\{a_1, a_2, \ldots, a_n\}\geq 2$.
\item Let $x_1\leq x_2\leq \cdots \leq x_{n + 1}$ be $n + 1$ positive integers. Prove that
  $\displaystyle\sum_{i=1}^{n + 1}\frac{\sqrt{x_{i +1} - x_i}}{x_{i + 1}}< \sum_{i=1}^{n^2}\frac{1}{j}$.
\item Let $a, b, c$ be three positive real numebrs which satisfy $abc = 1$ and $a^3 > 36$. Prove that
  $\frac{2}{3}a^2 < a^2 + b^2 + c^2 - ab - bc - ca$.
\item Let $z_1, z_2, \ldots, z_n$ be $n$ complex numbers and consider $n$ positive real numbers $\lambda_1,
  \lambda_2, \ldots, \lambda_n$ which have the property that $\sum 1/\lambda_i = 1$. Prove that
  $\displaystyle\left|\sum_{i=1}^nz_i\right|^2\leq\sum_{i=1}^n\lambda_1|z_i|^2$.
\item Let $a, b, c$ be three distinct real numbers. Prove that $2\min\{a, b, c\} <\sum a - $

  $\left(\sum a^2 - \sum
  ab\right)^{1/2} < \sum a + \left(\sum a^2 - \sum ab\right)^{1/2} < 3\max\{a, b, c\}$, where the sum is cyclic
  over $a, b, c$.
\item Show that for all complex numbers $z$ with $\Re(z) > 1$, prove that $|z^{n + 1} - 1| > |z^n||z -
  1|,\;\forall\;n\geq 1$.
\item Suppose $a, b, c$ are positive real numbers such that $x = a + b - c, y = b + c - a, z = c + a - b$. Prove
  that $abc(xy + yz + zx)\geq xyz(ab + bc + ca)$.
\item Let $a, b, c$ be positive real numbers. Prove that $\displaystyle\sum\frac{a^3}{b^2 - bc + c^2}\geq
  \frac{3\sum ab}{\sum a}$, where all sums are cyclic.
\item Let $a_1, a_2, \ldots, a_n < 1$ be non-negative real numbers satisfying $a = \sqrt{\frac{a_1^2 + a_2^2 +
    \cdots + a_n^2}{n}}\geq \frac{1}{\sqrt{3}}$. Prove that $\frac{a_1}{1 - a_1^2} + \frac{a_2}{1 - a_2^2} + \cdots
  + \frac{a_n}{1 - a_n^2}\geq \frac{na}{1 - a^2}$.
\item Suppose $x, y, z$ are non-negative real numbers such that $x^2 + y^2 + z^2 = 1$. Prove that
  \startitemize[n]
  \item $1\leq \sum\frac{x}{1 - yz}\leq\frac{3\sqrt{3}}{2}$, and
  \item $1\leq \sum\frac{x}{1 + yz}\leq \sqrt{2}$.
  \stopitemize
  The sums are cyclic over $x, y$ and $z$.
\item Let $x, y, z$ be non-negative real numbers satisfying $x + y + z = 1$. Prove that $xy^2 + yz^2 + zx^2 \geq xy
  + yz + zx - \frac{2}{9}$.
\item Let $a, b, c, d$ be positive real numbers such that $a + b + c + d = 2$. Prove that
  $\displaystyle\sum_{cyclic}\frac{a^2}{(a^2 _ 1)^2}\leq \frac{16}{25}$.
\item Prove that $\frac{a}{\sqrt{a^2 + 8bc}} + \frac{b}{\sqrt{b^2 + 8ca}} + \frac{c}{\sqrt{c^2 + 8ab}}\geq 1$ for
  all positive real numbers $a, b$ and $c$.
\item If $x, y$ are real numbers such that $x^3 + y^4 \leq x^2 + y^3$, prove that $x^3 + y^3\leq 2$.
\item Let $a, b, c$ be three positive real numbers. Prove that $\sum \frac{ab}{c(c + a)}\geq\sum\frac{a}{c + a}$,
  where the sum is cyclic over $a, b$ and $c$.
\item Let $x, y$ be two real numbers, where $y$ is non-negative and $y(y + 1)\leq (x + 1)^2$. Prove that $y(y -
  1)\leq x^2$.
\item Let $x, y, z$ be positive real numbers. Prove that $\left(\frac{xy + yz + zx}{3}\right)^{1/2}\leq$

  $\left(\frac{(x + y)(y + z)(z + x)}{8}\right)^{1/3}$.
\item Let $a, b, c$ be positive real numbers such that $abc = 1$. Show that $\displaystyle\sum\frac{a^9 + b^9}{a^6
  + a^3b^3 + b^6}\geq 2$, where the sum is cyclical.
\item Let $a_1, a_2, \ldots, a_n(n > 2)$ be positive real numbers and let $s$ be their sum. Let $0< \beta\leq 1$ be
  a real number. Prove that $\displaystyle\sum_{i = 1}^n\left(\frac{s - a_i}{a_i}\right)^\beta\geq(n -
  1)^{2\beta}\sum_{i = 1}^n\left(\frac{a_i}{s - a_i}\right)^\beta$. When does equality hold?
\item For $n\geq 4$, let $a_1, a_2, \ldots, a_n$ be $n$ positive real numbers such that $\displaystyle_{i=1}^na_i^2
  = 1$. Show that $\frac{a_1}{a_2^2 + 1} + \frac{a_2}{a_3^2 + 1} + \cdots + \frac{a_n}{a_1^2 + 1}\geq
  \frac{4}{5}(a_1\sqrt{a_1} + a_2\sqrt{a_2} + cdots + a_n\sqrt{a_n})^2$.
\item Does there exist an infinite sequence $\langle x_n\rangle$ of positive real numbers such that $x_{n + 2} =
  \sqrt{x_{n + 1}} - \sqrt{x_n},\;\forall\;n\geq 2$.
\item Let $a_1, a_2, \ldots, a_n$ be $n$ positive real numbers and consider a permutation of $b_1, b_2, \ldots,
  b_n$ of it. Prove that $\displaystyle\sum_{i=1}^n\frac{a_i^2}{b_i}\geq \sum_{i=1}^na_i$.
\item Let $a_1, a_2, \ldots, a_n$ and $b_1, b_2, \ldots, b_n$ be two sequences of positive real numbers such that
  $\displaystyle\sum_{i=1}^na_i = \sum_{i=1}^n b_i = 1$. Prove that $\displaystyle\sum_{i=1}^n\frac{a_i^2}{a_i +
  b_i}\geq \frac{1}{2}$.
\item Let $x, y, z$ be positive real numbers. Prove that $\frac{y^2 - x^2}{z + x} + \frac{z^2 - y^2}{x + y} +
  \frac{x^2 - z^2}{y + z}\geq 0$.
\item Find the greatest value of $k$ such that for every triple $(a, b, c)$ of positive real numbers, the
  inequality $(a^2 - bc)^2 > k(b^2 - ca)(c^2 - ab)$ holds.
\item Let $a, b, c, d$ be positive real numbers. Prove that $\displaystyle\sum_{\rm cyclic}\frac{a}{b + 2c + d}\geq
  1$.
\item Let $a, b, c$ be positive real numbers such that $(a + b)(b + c)(c + a) = 1$. Prove that $ab + bc + ca\leq
  \frac{3}{4}$.
\item Let $x, y, z$ be non-negative real numbers such that $x + y + z = 1$. Prove that $x^2 + y^2 + z^2 + 18xyz\leq
  1$.
\item Let $a, b, c$ be three positive real numbers such that $ab + bc + ca = 1$. Prove that $\left(\frac{1}{a} +
  6b\right)^{1/3} + \left(\frac{1}{b} + 6c\right)^{1/3} + \left(\frac{1}{c} + 6a\right)^{1/3}\leq \frac{1}{abc}$.
\item Let $a_1, a_2, \ldots, a_n$ be $n > 1$ positive real numbers. For each $k, 1\leq k\leq n$, let $A_k = (a_1 +
  a_2 + \cdots + a_k)/k$. Let $g_n = (a_1a_2\cdots a_n)^{1/n}$ and $G_n = (A_1A_2\cdots A_n)^{1/n}$. Prove that
  $n\left(\frac{G_n}{A_n}\right)^{1/n} + \frac{g_n}{G_n}\leq n + 1$. Find the cases of equality.
\item Let $x, y, z$ be real numbers in the interval $[0, 1]$. Prove that $3(x^2y^2 + y^2z^2 + z^2x^2) - 2xyz(x + y
  + z)\leq 3$.
\item Let $x, y, z$ be non-negative real numbers such that $x + y + z = 1$. Prove that $7(xy + yz + zx)\leq 2 +
  9xyz$.
\item Let $x, y, z$ be real numbers in the interval $[0, 1]$. Prove that $\frac{x}{yz + 1} + \frac{y}{zx + 1} +
  \frac{z}{xy + 1} \leq 2$.
\item Let $a, b, c, d$ be positive real such that $a^3 + b^3 + 3ab = c + d = 1$. Prove that $\left(a +
  \frac{1}{a}\right)^3 + \left(b + \frac{1}{b}\right)^3 + \left(c + \frac{1}{c}\right)^3 + \left(d +
  \frac{1}{d}\right)^3 \geq 40$.
\item Let $x, y, z$ be positive real numbers such that $x + y + z = xyz$. Prove that $\frac{1}{\sqrt{1 + x^2}} +
  \frac{1}{\sqrt{1 + y^2}} + \frac{1}{\sqrt{1 + z^2}}\leq \frac{3}{2}$.
\item Let $x, y, z$ be non-negative real numbers. Prove that $x^3 + y^3 + z^3 \geq x^2\sqrt{yz} + y^2\sqrt{zx} +
  z^2\sqrt{xy}$.
\item For all positive real numbers show that $4(ab + bc + ca) - 1\geq a^2 + b^2 + c^2 \geq 3(a^3 + b^3 + c^3)$.
\item Let $a, b, c$ be positive real numbers such that $abc = 1$. Prove that $\frac{a}{(a + 1)(b + 1)} +
  \frac{b}{(b + 1)(c + 1)} + \frac{c}{(c + 1)(a + 1)}\geq \frac{3}{4}$.
\item Suppose $a, b, c$ are positive real numbers such that $a^2 + b^2 + c^2 = 1$. Prove that $\frac{1}{a^2} +
  \frac{1}{b^2} + \frac{1}{c^2}\geq 3 + \frac{2(a^3 + b^3 + c^3)}{abc}$.
\item Let $x, y, z$ be positive real numbers such that $xyz = 1$. Prove that $\frac{x^3}{(1 + y)(1 + z)} +
  \frac{y^3}{(1 + z)(1 + z)} + \frac{z^3}{(1 + x)(1 + y)}\geq \frac{3}{4}$.
\item Let $a, b, c, d$ be non-negative real numbers such that $ab + bc + cd + da = 1$. Show that $\frac{a^3}{b + c
  + d} + \frac{b^3}{c + d + a} + \frac{c^3}{d + a + b} + \frac{d^3}{a + b + c}\geq \frac{1}{3}$.
\item Find all real $k$ for which the inequality $x_1^2 + x_2^2 + x_3^2 \geq k(x_1x_2 + x_2x_3)$ holds for all real
  numbers $x_1, x_2, x_3$.
\item Let $a, b, c$ be positive real numbers such that $abc = 1$. Prove that $\frac{a}{b} + \frac{b}{c} +
  \frac{c}{a} \geq \frac{1}{a} + \frac{1}{b} + \frac{1}{c}$.
\item Let $a, b, c$ be non-negative reals such that $a + b\leq 1 + c, b + c\leq 1 + a, c + a\leq 1 + b$. Prove that
  $a^2 + b^2 + c^2 \leq 2abc + 1$.
\item If $a, b, c$ are non-negative real numbers such that $a + b + c = 1$, then show that $\frac{a}{1 + bc} +
  \frac{b}{1 + ca} + \frac{c}{1 + ab}\geq \frac{9}{10}$.
\item Let $a, b, c$ be three positive real numbers such that $a + b + c = 1$. Prove that among the three numbers $a
  - ab, b - bc, c - ca$ there is one which is at most $1/4$ and there is one which is at least $2/9$.
\item Let $x$ and $y$ be positive real numbers such that $y^3 + y \leq x - x^3$. Prove that (a) $y < x < 1$, and
  (b) $x^2 + y^2 < 1$.
\item Let $a, b, c$ be three positive real numbers such that $a + b + c = 1$. Let $k = \min\{a^3 + a^2bc, b^3 +
  ab^2c, c^3 + abc^2\}$. Prove that the roots of the equation $x^2 + x + 4k = 0$ are real.
\item If $a, b, c$ are three positive real numbers, prove that $\frac{a^2 + 1}{b + c} + \frac{b^2 + 1}{c + a} +
  \frac{c^2 + 1}{a + b}\geq 3$.
\item If $d$ is tghe largest among the positive numbers $a, b, c, d$, prove that $a(d - b) + b(d - c) + c(d -
  a)\leq d^2$.
\item If $x, y, z$ are positive real numbers, prove that $(x + y + z)^2(yz + zx + xy)^2\leq 3(y^2 + yz + z^2)(z^2 +
  zx + x^2)(x^2 + xy + y^2)$.
\item Suppose $a, b, c$ are positive real bumbers. Prove that $a^ab^bc^c\geq (abc)^{(a + b + c)/3}$.
\item Find all real $p$ and $q$ for which the equation $x^4 - \frac{8p^2}{q}x^3 + 4qx^3 - 3px + p^2 = 0$ has four positive roots.
\item Let $a_1, a_2, a_3$ be real numbers, each greater than $1$. Let $S = a_1 + a_2 + a_3$ and suppose $S < \frac{a_i^2}{a_i - 1}$
  for $i = 1, 2, 3$. Prove that $\frac{1}{a_1 + a_2} + \frac{1}{a_2 + a_3} + \frac{1}{a_3 + a_1} > 1$.
\item Let $a, b, c$ be positive real numbers such that $ab + bc + ca = \frac{1}{3}$. Prove that $\frac{a}{a^2 - bc + 1} +
  \frac{b}{b^2 - ca + 1} + \frac{c}{c^2 - ab + 1}\geq \frac{1}{a + b + c}$.
\item Suppose $a, b, c$ are positive real numbers. Prove that $\frac{a^2b(b - c)}{a + b} + \frac{b^2c(c - a)}{b + c} + \frac{c^2a(a
  - b)}{c + a}\geq 0$.
\item Let $a_1, a_2, \ldots, a_n$ be $n > 2$ positive real numbers such that $a_1 + a_2 + \cdots + a_n = 1$. Prove that
  $\displaystyle\sum_{i = 1}^n\frac{a_1a_2\cdots a_{i - 1}a_{i + 1}\cdots a_n}{a_i + n - 1}\leq \frac{1}{(n - 1)^2}$.
\item Determine the largest value of $k$ such that the inequality $\left(k + \frac{a}{b}\right)\left(k + \frac{b}{c}\right)\left(k
  + \frac{c}{ba}\right)\geq \left(\frac{b}{a} + \frac{c}{b} + \frac{a}{c}\right)$ holds for positive real numbers $a, b, c$.
\item Let $x_1, x_2, \ldots, x_n$ be $n\geq 3$ positive real numbers. Prove that $\frac{x+_1x_3}{x_1x_3 + x_2x_4} +
  \frac{x_2x_4}{x_2x_4 + x_3x_5} + \cdots + \frac{x_{n - 1}x_1}{x_{n - 1}x_1 + x_nx_2} + \frac{x_nx_2}{x_nx_2 + x_1x_3}\leq n - 1$.
\item Let $a_1, a_2, \ldots, a_{2017}$ be positive real numbers. Prove that $\displaystyle\sum_{i=1}^{2017}\frac{a_i}{a_{i + 1} +
  a_{i + 2} + \cdots + a_{i + 1008}}\geq \frac{2017}{1008}$, where indices are taken modulo $2017$.
\item Let $a, b, c$ be three positive real numbers such that $ab + bc + ca = 1$. Prove that $\sqrt{a + \frac{1}{a}} + \sqrt{b +
  \frac{1}{b}} + \sqrt{c + \frac{1}{c}}\geq 2(\sqrt{a} + \sqrt{b} + \sqrt{c})$.
\item Let $a, b, c$ be positive real numbers such that $a + b + c = 3$. Prove that $\frac{a^3 + 2}{b + 2} + \frac{b^3 + 2}{c + 2} +
  \frac{c^3 + 2}{a + 3}\geq 3$.
\item Let $a, b, c, d$ be real numbers such that $a^2 + b^2 + c^2 + d^2 = 4$. Prove that $(2 + a)(2 + b)\geq cd$.
\item Find all real $k$ such that $\frac{a + b}{2}\geq k\sqrt{ab} + (1 - k)\sqrt{\frac{a^2 + b^2}{2}}$ holds for all positive real
  numbers $a, b$.
\item Let $a, b, c, d$ be real numbers having absolute value greater than $1$ such that $abc + abd + acd + bcd + a + b + c + d =
  0$. Probvve that $\frac{1}{a - 1} + \frac{1}{b - 1} + \frac{1}{c - 1} + \frac{1}{d - 1} > 0$.
\item For all positive, real $x, y$ show that $\frac{1}{x + y - 1} - \frac{1}{(x + 1)(y + 1)}< \frac{1}{11}$.
\item Let $a, b, c$ be three positive real numbers such that $abc = 1$. Prove that $\frac{1}{b(a + b)} + \frac{1}{c(b + c)} +
  \frac{1}{a(c + a)}\geq \frac{3}{2}$.
\item Let $a, b, c$ be positive real numbers such that $a + b + c = 1$. Prove that $\frac{a^2}{(b + c)^3} + \frac{b^2}{(c + a)^3 +
  \frac{c^2}{(a + b)^3}}\geq \frac{9}{8}$.
\item Suppose $a, b, c$ are positive real numbers such that $ab + bc + ca\geq a + b + c$. Prove that $(a + b + c)(ab + bc + ca) +
  3abc \geq 4(ab + bc + ca)$.
\item Let $a, b, c, d$ be four real nubers such that $a + b + c + d = 0$. Prove that $(ab + ac + ad + bc + bd + cd)^2 + 12\geq
  6(abc + abd + acd + bcd)$.
\item Consider the expression $P = \frac{x^3y^4z^3}{(x^4 + y^4)(xy + z^2)^3} + \frac{y^3z^4x^3}{(y^4 + z^4)(yz + x^2)^3} +
  \frac{z^3x^4y^3}{(z^4 + x^4)(zx + y^2)^3}$. Find the maximum value of $P$ when $x, y, z$ vary over the set of all positive real
  numbers.
\item Let $x_1, x_2, \ldots, x_n$ be positive real numbers such that $x_1x_2\ldots x_n = 1$. Let $S = x_1^3 + x_2^3 + \cdots +
  x_n^3$. Prove that $\frac{x_1}{S - x_1^3 + x_1^2} + \frac{x_2}{S - x_2^3 + x_2^2} + \cdots + \frac{x_n}{S - x_n^3 + x_n^2}\leq
  1$.
\item Let $a_1, a_2, \ldots, a_n$ be $n > 1$ positive real numbers whose sum is $1$. Define $\displaystyle b_i =
  \frac{a_i^2}{\sum_{j = 1}^na_j^2}, 1\leq i< 2$. Prove that $\displaystyle\sum_{i = 1}^n\frac{a_i}{1 - a_i}\leq \sum_{i =
    1}^n\frac{b_i}{1 - b_i}$.
\item Suppose $a, b, c, d$ are posotive real numbers. Prove that $\displaystyle\sum_{\rm cyclic}\frac{a^4}{a^3 + a^2b + ab^2 +
  b^3}\geq \frac{a + b + c + d}{4}$.
\item Let $a, b, c$ be non-negative real numberssatisfying $a^2 + b^2 + c^2 = 1$. Prove that $\sqrt{a + b} + \sqrt{b + c} + \sqrt{c
  + a}\geq 5abc + 2$.
\item Let $x, y, z$ be positive real numbers such that $x^2 + y^2 + z^2\leq x + y + z$. Prove that $\frac{x^2 + 3}{x^3 + 1} +
  \frac{y^2 + 3}{y^3 + 1} + \frac{z^2 + 3}{z^3 + 1}\geq 6$.
\item For any three positive real numbers $a, b, c$ prove that $\frac{a^2}{a + b} + \frac{b^2}{b + c}\geq \frac{3a + 2b - c}{4}$.
\item Suppose $a, b, c$ are non-negative real numbers such that $a^3 + b^3 + c^3 + abc = 4$. Prove that $a^3b + b^3c + c^3a\leq 3$.
\item Let $a, b, c$ be positive real numbers such that $abc = 1$. Prove that $\left(a + \frac{1}{b}\right)^2 + \left(b +
  \frac{1}{c}\right)^2 + \left(c + \frac{1}{a}\right)^2\geq 3(a + b + c + 1)$.
\item Let $a, b,c $ be positive reall numbers with $abc = 1$. Prove that $\frac{a}{c(a + 1)} + \frac{b}{a(b + 1)} + \frac{c}{b(c +
  1)}\geq \frac{3}{2}$.
\item Let $a, b, c$ be positive real numbers such that $abc = 1$. Prove that $\frac{1}{1 + a^{2014}} + \frac{1}{1 + b^{2014}} +
  \frac{1}{1 + c^{2014}} > 1$.
\item For positive real numbers $a, b, c$, prove the inequality

  $\left(\frac{1}{a} + \frac{1}{b} +
  \frac{1}{c}\right)\left(\frac{1}{1 + a} + \frac{1}{1 + b} + \frac{1}{1 + c}\right)\geq \frac{9}{1 + abc}$.
\item Let $x, y, z$ be positive real numbers such that $x + y + z = 3$. Prove that $\sqrt{x} + \sqrt{y} + \sqrt{z}\geq xy + yz +
  zx$.
\item Let $a, b, c$ be positive real numbers. Prove that $\frac{9abc}{2(a + b + c)}\leq \frac{ab^2}{a + b} + \frac{bc^2}{b + c} +
  \frac{ca^2}{c + a}\leq \frac{a^2 + b^2 + c^2}{2}$.
\item For positive real numbers $a, b, c$, prove that $\frac{abc}{(1 + a)(a + b)(b + c)(c + 16)}\leq \frac{1}{81}$.
\item Let $a, b, c, d$ be positive real numbers such that $a + b + c + d = 4$. Prove that $\frac{1}{a^2 + 1} + \frac{1}{b^2 + 1} +
  \frac{1}{c^2 + 1} + \frac{1}{d^2 + 1}\geq 2$.
\item Let $a, b, c$ be positive real numbers. Prove that $\frac{1 + ab}{c} + \frac{1 + bc}{a} + \frac{1 + ca}{b}\geq \sqrt{a^2 + 2}
  + \sqrt{b^2 + 2} + \sqrt{c^2 + 2}$.
\item Let $a, b, c$ be positive real numbers such that $a + b + c = 1$. Prove that $\frac{a^2}{b^3 + c^4 + 1} + \frac{b^2}{c^3 +
  a^4 + 1} + \frac{c^2}{a^3 + b^4 + 1} > \frac{1}{5}$.
\item {\it Janous Inequality}: Let $a, b, c$ and $x, y, z$ be two sets of positive real numbers. Prove that $\frac{x(b + c)}{y + z}
  + \frac{y(c + a)}{z + x} + \frac{z(a + b)}{x + y}\geq \sqrt{3(ab + bc + ca)}$.
\item Let $x, y, z$ be positive real numbers such that $xy + yz + zx = 1$. Prove that $\frac{x}{x^2 + 1} + \frac{y}{y^2 + 1} +
  \frac{z}{z^2 + 1}\leq \frac{3\sqrt{3}}{4}$.
\item Let $x, y, z$ be positive real numbers such that $x + y + z = 1$. Prove that $\frac{1}{1 - xy} + \frac{1}{1 - yz} +
  \frac{1}{1 - zx}\leq \frac{27}{8}$.
\item Let $x, y, z$ be positive real numbers such that $x + y + z = 1$. Show that $\frac{z - xy}{x^2 + xy + y^2} +$

  $\frac{x - yz}{y^2 + yz + z^2} + \frac{y - zx}{z^2 + zx + x^2}\geq 2$.
\item Let $a, b, c$ be positive real numbers. Define $u = a + b + c, \frac{u^2 - b^2}{3} = ab + bc + ca, w = abc$, where $v\geq
  0$. Then $\frac{(u + v)^2(u - 2v)}{27}\leq w\leq \frac{(u - v)^2(u + 2v)}{27}$.
\item Let $a, b, c$ be positive real numbers. Prove that $a^4 + b^4 + c^4 \geq abc(a + b + c)$.
\item Let $a, b, c$ be real numbers such that $a^2 + b^2 + c^2 = 9$. Prove that $2(a + b + c) - abc\leq 10$.
\item Let $a, b, c$ be positive real numbers such that $a + b + c = 1$. Prove that $a^2 + b^2 + c^2 + 3abc \geq \frac{9}{4}$.
\item Determine the maximum value of $k$ such that $a + b + c\geq k$ for all positive reals $a, b, c$ with $a\sqrt{bc} + b\sqrt{ca}
  + c\sqrt{ab}\geq 1$.
\item If $a, b, c$ are real numbers such that $a + b + c = 1$, prove that $10(a^3 + b^3 + c^3) - 9(a^5 + b^5 + c^5)\geq 1$.
\item Let $a, b, c$ be positive real numbers. Prove that $24abc\leq |a^3 + b^3 + c^3 - (a + b + c)^3|\leq \frac{8}{9}(a + b +
  c)^3$. Also show that equality holds in both the inequalities if and only if $a = b = c$.
\item Find all $k > 0$ such that the inequality $\sqrt{a^2 + kb^2} + \sqrt{b^2 + ka^2}\geq a + b + (k - 1)\sqrt{ab}$ holds positive
  real numbers $a$ and $b$.
\item Let $a, b, c$ be positive real numbers such that $abc = 1$. Prove that $a + b + c\geq \sqrt{\frac{1}{3}(a + 2)(b + 2)(c +
  2)}$.
\item Let $x_1, x_2, \ldots, x_n$ be $n\geq 3$ positive real numbers such that $x_1x_2\cdots x_n = 1$. Prove that
  $\displaystyle\sum_{i = 1}^n \frac{x_i^8}{x_{i + 1}(x_i^4 + x_{i + 1}^4)}\geq \frac{n}{2}$, where $x_1 = x_{n + 1}$.
\item Let $a, b, c$ be positive real numbers such that $\frac{1}{ab} + \frac{1}{bc} + \frac{1}{ca} = 1$. Prove that $\frac{a^2 +
  b^2 + c^2 + ab + bc + ca - 3}{5}\geq \frac{a}{b} + \frac{b}{c} + \frac{c}{a}$.
\item For positive, real $x, y, z$ show that $\frac{x(2x - y)}{y(2z + x)} + \frac{y(2y - z)}{z(2x + y)} + \frac{z(2z - x)}{x(2y +
  z)}\geq 1$.
\item Suppose $\frac{z(zx + yz + y)}{xy^2 + z^2 + 1}\leq k$, for alll real numbers $x, y, z\in (-2, 2)$ with $x^2 + y^2 + z^2 + xyz
  = 4$. Find the smallest value of $k$.
\item Suppose $a, b, c$ are positive real numbers such that $a^3 + b^3 + c^3 = a^4 + b^4 + c^4$. Prove that $\frac{a}{a^2 + b^3 +
  c^3} + \frac{b}{b^2 + c^3 + a^3} + \frac{c}{c^2 + c^3 + a^3}\geq 1$.
\item Let $a, b, c$ be positive real numbers such that $a + b + c = 1$. Prove that $\frac{a^4 + 5g^4}{a(a + 2b)} + \frac{b^4 +
  5c^4}{b(b + 2c)} + \frac{c^4 + 5a^4}{c(c + 2a)}\geq 1 - (ab + bc + ca)$.
\item Let $x, y, z$ be positive real numbers. Prove that $(xy + yz + zx)\left(\frac{1}{(x + y)^2} + \frac{1}{(y + z)^2} +
  \frac{1}{(z + x)^2}\right)\geq \frac{9}{4}$.
\item Suppose $a, b, c$ are positive real numbers such that $abc = 1$. Prove that $\displaystyle\sum_{\rm cyclic}\frac{a^2 +
  bc}{a^2(b + c)}\geq ab + bc + ca$.
\item Let $a, b, c$ be non-negative real numbers. Prove that $4(a^3 + b^3 + c^3) + 15abc\geq(a + b + c)^3$.
\item Let $a, b, c$ be positive real numbers such that $a + b + c = 1$. Prove that $\frac{1}{a^4 + b + c} + \frac{1}{b^4 + c + a} +
  \frac{1}{c^4 + a + b}\leq \frac{3}{a + b + c}$.
\item Let $a, b, c$ be positive reals. Prove that $a^4(b + c) + b^4(c + a) + c^4(a + b)\leq \frac{1}{12}(a + b + c)^5$.
\item Let $a, b, c$ be positive reals such that $ab + bc + ca = 1$. Prove that $\frac{1}{a + b} + \frac{1}{b + c} + \frac{1}{c + a}
  - \frac{1}{a + b + c}\geq 2$.
\item Let $a, b, c$ be positive reals such that $ab + bc + ca = 1$. Prove that $\frac{1 + a^2b^2}{(a + b)^2} + \frac{1 + b^2c^2}{(b
  + c)^2} + \frac{1 + c^2a^2}{(c + a)^2}\geq \frac{5}{2}$.
\item Let $a, b, c$ be positive real numbers. Prove that $3 + a + b + c + \frac{1}{a} + \frac{1}{b} + \frac{1}{c} + \frac{a}{b} +
  \frac{b}{c} + \frac{c}{a}\geq 3\left[\frac{(a + 1)(b + 1)(c + 1)}{1 + abc}\right]$.
\item Let $a, b, c$ be distinct positive real numbers such that $abc = 1$. Prove that $\displaystyle\sum_{\rm cyclic}\frac{a^6}{(a -
  b)(a - c)} > 15$.
\item Let $a, b, c$ be real numbers such that $a^2 + b^2 + c^2 = 1$. Prove that $a + b + c\leq 2abc + \sqrt{2}$.
\item Let $a, b, c$ be positive real numbers. Prove that $\frac{(b + c - a)^2}{a^2 + (b + c)^2} + \frac{(c + a - b)^2}{b^2 + (c +
  a)^2} + \frac{(a + b - c)^2}{c^2 + (a + b)^2}\geq \frac{3}{5}$.
\item Let $a, b, c$ be positive real numbers such that $a + b + c = 1$. Prove that $\sqrt{\frac{1}{a} - 1}\sqrt{\frac{1}{b} - 1} +
  \sqrt{\frac{1}{b} - 1}\sqrt{\frac{1}{c} - 1} + \sqrt{\frac{1}{c} - 1}\sqrt{\frac{1}{a} - 1}\geq 6$.
\stopitemize
