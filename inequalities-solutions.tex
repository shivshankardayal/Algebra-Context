% -*- mode: context; -*-
\chapter{Inequalities}
\startitemize[n]
\item We have $a^2 + b^2 - 2ab = (a - b)^2\geq 0$. It should be noted that equality holds if and only if $a
  = b$.
\item Similar to previous problem we have $a + b - 2\sqrt{ab} = (\sqrt{a} - \sqrt{b})^2\geq 0$. Similarly,
  equality holds if and only if $a = b$.
\item Squaring $\frac{a^2 + b^2}{2}\geq \frac{a^2 + b^2 + 2ab}{4}\Rightarrow \frac{(a - b)^2}{4}\geq 0$,
  which is true. Similar to previous problems equality holds if and only if $a = b$.
\item $\frac{a + b}{2}\geq \frac{2ab}{a + b}\Rightarrow (a - b)^2\geq 0$, which is same as first problem.
\item $a + b - 1 - ab = (a - 1)(1 - b)$. We have $b < 1 < a$ making $(a - 1)(1 - b) > 0$.
\item $a^2 + b^2 - c^2 - (a + b - c)^2 = (a^2 - c^2) - [(a + b - c)^2 - b^2] = 2(a - c)(c - b) > 0$.
\item Multiplying both sides of the inequality by $ab$, where $ab > 0$, gives us $a^2 + b^2\geq 2ab$.
\item Dividing both sides of the inequality $a^2 +b^2 \geq -2ab$ by $ab$, where $ab < 0$, gives us
  the required inequality.
\item We have $x_1\leq x_2\leq \cdots\leq x_{n - 1}\leq x_n$, which gives us $nx_1\leq x_1 + x_2 + \cdots +
  x_n\leq nx_n \Rightarrow x_1\leq \frac{x_1 + x_2 + \cdots + x_n}{n}\leq x_n$.
\item We are given that $\frac{x_1}{y_1}\leq\cdots\leq \frac{x_n}{y_n}$, from which we deduce that
  $y_i\frac{x_1}{y_1}\leq x_i\leq y_i\frac{x_n}{y_n}, i = 1, \ldots, n$. Adding all these inequalities leads
  to $\frac{x_1}{y_1}(y_1 + \cdots + y_n)\leq x_1 + \cdots + x_n\leq \frac{x_n}{y_n}(y_1 +
  \cdots + y_n)$. Hence, it follows that $\frac{x_1}{y_1}\leq \frac{x_1 + \cdots + x_n}{y_1 + \cdots +
    y_n}\leq x_n$.
\item Given that $x_1\leq x_i\leq x_n, i = 1, \ldots, n$. Multiplying these inequalities $x_1^n\leq
  x_1\cdots x_n\leq x_n^n \Rightarrow x_1\leq (x_1\cdots x_n)^{\frac{1}{n}}\leq x_n$.
\item If $a_1 + \cdots + a_n \geq 0$, then $|a_1 + \cdots + a_n| = a_1 + \cdots + a_n$. Using $a\leq |a|$
  gives us $|a_1 + \cdots + a_n| = a_1 + \cdots + a_n\leq |a_1| + \cdots + |a_n|$.
\item We have $(a_1 + \cdots + a_n)\left(\frac{1}{a_1} + \cdots + \frac{1}{a_n}\right) =
  \underbrace{\Big(\left(\frac{a_1}{a_2} + \frac{a_2}{a_1}\right) + \cdots + \left(\frac{a_{n - 1}}{a_n} +
    \frac{a_n}{a_{n - 1}}\right)\Big)}_{n(n - 1)/2} + n$. We also know that $\frac{x}{y} + \frac{y}{x}\geq 2$.

  $\Rightarrow (a_1 + \cdots + a_n)\left(\frac{1}{a_1} + \cdots + \frac{1}{a_n}\right)\geq n + 2.\frac{n(n -
    1)}{2} = n^2$.
\item The given inequality is equivalent to $(a + b)\sqrt{\frac{a + b}{2}}\geq 2\sqrt{ab}\frac{\sqrt{a} +
  \sqrt{b}}{2}$, which can be obtained by multiplying the inequalities $a + b\geq 2\sqrt{ab}$ and
  $\sqrt{\frac{a + b}{2}}\geq \frac{\sqrt{a} + \sqrt{b}}{2}$.
\item We have $\frac{1}{2}(a + b) + \frac{1}{4} - \sqrt{\frac{a + b}{2}} = \left(\sqrt{\frac{a + b}{2}} -
  \frac{1}{2}\right)^2\geq 0$. Therefore, $\frac{1}{2}(a + b) + \frac{1}{4}\geq \sqrt{\frac{a + b}{2}}$.
\item Since $a(x + y − a) − xy = ax − xy + a(y − a) = (y − a)(a − x)$ and $y \geq a \geq x$, it follows that
  $(y − a)(a − x) \geq 0$. Therefore, $a(x + y − a) \geq xy$.
\item We have proven that $\frac{a + b}{2}\geq \frac{2}{\frac{1}{a} + \frac{1}{b}}$. Using this we can write
  that $\frac{\frac{1}{x - 1} + \frac{1}{x + 1}}{2}\geq \frac{2}{x - 1 + x + 1}$ or $\frac{1}{x - 1} +
  \frac{1}{x + 1}\geq \frac{2}{x}$.
\item Following from prevvious problem we have $\frac{1}{3k + 1} + \frac{1}{3k+ 3} = \frac{1}{(3k+ 2) - 1} +
  \frac{1}{(3k + 2) + 1} > \frac{2}{3k + 2}$. Therefore, $\frac{1}{3k + 1} + \frac{1}{3k + 2} + \frac{1}{3k
    + 3} > \frac{3}{3k + 2}$.

  Now we will prove that $\frac{3}{3k + 2} > \frac{1}{2k + 1} + \frac{1}{2k + 2}$.

  We find that $\frac{1}{2k + 1} + \frac{1}{2k + 2} - \frac{3}{3k + 2} = \frac{-k}{(2k + 1)(2k + 2)(3k + 3)}
  < 0$.
\item The given inequality is equivalent to $\left(\frac{2}{a + b} - 1\right)^2\leq \left(\frac{1}{a} -
  1\right)\left(\frac{1}{b} - 1\right)$. We have $\left(\frac{1}{a} - 1\right)\left(\frac{1}{b} - 1\right) -
  \left(\frac{2}{a + b} - 1\right)^2 = \frac{1}{ab} - \frac{1}{a}- \frac{1}{b} - \frac{4}{(a + b)^2} +
  \frac{4}{a + b} = \frac{1}{ab} - \frac{4}{(a + b)^2} + \frac{4}{a + b} - \frac{a + b}{ab} = \frac{(a -
    b)^2[1 - (a + b)]}{ab(a + b)^2}$ and $0\leq a\leq \frac{1}{2}, 0\leq b\leq \frac{1}{2}$. Then $\frac{(a
    - b)^2[1 - (a + b)]}{ab(a + b)^2}\geq 0$, and therefore $\left(\frac{2}{a + b} - 1\right)^2\leq
  \left(\frac{1}{a} - 1\right)\left(\frac{1}{b} - 1\right)$.
\item The given inequality is equilavent to $(2k + 1)\sqrt{3k + 4} < (2k + 2)\sqrt{3k + 1} \Rightarrow (2k +
  1)^2(3k + 4) < (2k + 2)^2(3k + 1)$, and it holds because $(2k + 2)^2(3k + 1) - (2k + 1)^2(3k + 4) = k >
  0$.
\item We know that $1 < 2 < 2^2 < \cdots < 2^{n - 1}$ and the number of positive integers $1, 2, \ldots,
  2^{n - 1}$ is equal to $n$. Thus, $2^{n - 1}\geq n$.
\item Consider a one meter long rope. Suppose we painted $\frac{1}{3}$m of this rope on first day,
  $\frac{1}{5}$ of the remaining $\frac{2}{3}$ m on second day and so on. We will find that sum of painted
  parts is less than $1$ m.

  Hence, we deduce that $\frac{1}{3} + \frac{2}{3}.\frac{1}{5} + \frac{2}{3}.\frac{4}{5}.\frac{1}{7} +
  \cdots + \frac{2}{3}.\frac{4}{5}.\frac{6}{7}\cdots \frac{100}{101}.\frac{1}{103} < 1$.
\item Observe that $1 - a\geq a$ and $1 - b\geq b$. And thus,

  $\frac{1 - a}{1 - b} + \frac{1 - b}{1 - a} = \frac{(1 - a)^2 + (1 - b)^2}{(1 - a)(1 - b)} = \frac{[(1 - a)
- (1 - b)]^2 + 2(1 - a)(1 - b)}{(1 - a)(1 - b)} = \frac{(a - b)^2}{(1 - a)(1 - b)} + 2\leq \frac{(a -
  b)^2}{ab} + 2 = \frac{a}{b}+ \frac{b}{a}$.
\item $\displaystyle\sum_{i=1}^n\frac{1}{1 - a_i}\sum_{i = 1}^n(1 - a_i) = \underbrace{\left(\frac{1 -
    a_1}{1 - a_2} + \frac{1 - a_2}{1 - a_1}\right) + \cdots + \left(\frac{1 - a_{n - 1}}{1 - a_n} + \frac{1
    - a_n}{1 - a_{n - 1}}\right)}_{n(n - 1)/2} + n$.

  Using the inequality of previous problem we have $\displaystyle\sum_{i=1}^n\frac{1}{1 - a_i}\sum_{i =
    1}^n(1 - a_i) \leq \underbrace{\left(\frac{a_1}{a_2} + \frac{a_2}{a_1}\right) + \cdots +
    \left(\frac{a_{n - 1}}{a_n} + \frac{a_n}{a_{n- 1}}\right)}_{n(n - 1)/2} + n = \sum_{i =
    1}^n\frac{1}{a_i} \sum_{i = 1}^na_i$.
\item Observe that if $n\geq 4$, then $1 + \frac{1}{2^3} + \cdots + \frac{1}{n^3} = 1 + \frac{2 - 1}{2^3} +
  \cdots + \frac{n - (n - 1)}{n^3}$

  $= \frac{5}{4} - \left(\frac{1}{2^3} - \frac{1}{3^2}\right) - \left(\frac{1}{3^3} - \frac{1}{4^2}\right) -
  \cdots - \frac{n - 1}{n^3} < \frac{5}{4}$

  because $\frac{k}{(k + 1)^3} > \frac{1}{(k + 2)^2}\;\forall\;k\in\mathbb{N}$.
\item For the given conditions $(1 - a)(1 - b)\geq 0 \Rightarrow a + b - 1\leq ab$. Thus,

  $\frac{1}{1 + a + b} - \left(1 - \frac{a + b}{2}\right) = \frac{a + b}{2(1 + a + b)}(a + b - 1)\leq
  \frac{1}{3}ab$.
\item Squaring $(x - y)^2 < (1 - xy)^2 \Rightarrow x^2 + y^2 - 2xy < 1 - 2xy + x^2y^2 \Rightarrow (1 -
  x^2)(1 - y^2) > 0$, which is true for given values of $x$ and $y$.
\item Multiplying both sides by $abc, a^2 + b^2 + c^2\geq 2bc + 2ac - 2ab \Rightarrow (a + b - c)^2\geq 0$,
  which is true.
\item Multiplpying both sides by $abc, bc + ac - ab < 1 \Rightarrow ab > bc + ca - 1$. We know that $ab \leq
  \frac{a^2 + b^2}{2}$ from A.M.-G.M. inequality and $a^2 + b^2 = \frac{5}{3} - c^2\Rightarrow ab \leq
  \frac{5}{6} - \frac{c^2}{2}$.

  Rearranging the inequality, $c(a + b)\leq \frac{11}{6} - \frac{c^2}{2}$. We know that $(a + b)^2\leq 2(a^2
  + b^2)$

  $\Rightarrow c(a + b)\leq c\sqrt{2\left(\frac{5}{4} - c^2\right)}$. Since $c(a + b)\leq \frac{5}{6} + ab$
  and $ab < 1$, then $c(a + b) < \frac{11}{6}$. Thus, $ab > c(a + b) - 1$.
\item Given $3(1 + a^2 + a^4)\geq (1 + a + a^2)^2\Rightarrow 2(1- a)^2(1 + a + a^2)\geq 0$, which is true
  because $1 + a + a^2 = \left(a + \frac{1}{2}\right)^2 + \frac{3}{4}$, which is always positive.
\item $a^2 + b^2\geq \frac{(a + b)^2}{2} = 8$ and $c^2 + d^2\geq \frac{(c + d)^2}{2} = 18$.

  Now $(ac + bd)^2 + (ad - bc)^2 = (a^2 + b^2)(c^2 + d^2)\geq 8.18 = 144$.
\item $x_1^2 + x_2^2 + \cdots + x_{2n}^2 + na^2 - a\sqrt{2}(x_1 + x_2 + \cdots + x_{2n})\geq 0$

  $\Rightarrow \left(x_1 - \frac{a}{\sqrt{2}}\right)^2 + \left(x_2 - \frac{a}{\sqrt{2}}\right)^2 + \cdots +
  \left(x_{2n} - \frac{a}{\sqrt{2}}\right)^2\geq 0$, which is true.
\item From A.M.-H.M. inequaltiy $\frac{1}{a} + \frac{1}{b}\geq \frac{4}{a + b}$. Thus,

  $\frac{1}{2}\left(\frac{1}{a} + \frac{1}{b}+ \frac{1}{c}\right)\geq \frac{1}{a + b} + \frac{1}{b + c} +
  \frac{1}{c + a}$

  Now $bc + ca + ab\leq (a + b + c)^2 \leq (\sqrt{ab} + \sqrt{bc} + \sqrt{ca})^2$

  $\Rightarrow \frac{1}{2}\left(\frac{1}{a} + \frac{1}{b} + \frac{1}{c}\right)\leq
  \frac{1}{2}\left(\frac{1}{\sqrt{a}} + \frac{1}{\sqrt{b}} + \frac{1}{\sqrt{c}}\right)$

  Hence proved.
\item We have to prove that $a^3(b^2 - c^2) + b^3(c^2 - a^2) + c^3(a^2 - b^2) < 0$.

  $a^3(b^2 - c^2) + b^3(c^2 - a^2) + c^3(a^2 - b^2) = a^2b^2(a -b) - a^2c^2(a - c) + b^2c^2(b - c)$

  $= a^2b^2(a -b) - a^2c^2(a - b + b - c) + b^2c^2(b - c) = (a - b)(b - c)[a^2b + a^2c - c^2b - c^2a]$

  $= (a - b)(b - c)[b(a^2 - c^2) + ac(a - c)] = (a - b)(b - c)(a - c)[ab + bc + ac]$, which is obviously
  less than zero.
\item Clearly, $2(a^3b + b^3c + c^3a)\geq a^3(b + c) + b^3(a + c) + c^3(a + b)$ using rearrangement
  inequality.

  Using A.M.-G.M. inequality $a^3b + b^3a\geq 2a^2b^2, b^3c + c^3b\geq 2b^2c^2, c^3a + ca^3\geq 2c^2a^2$.

  From these two inequalities we have the desired inequality.
\item The given inequality is equivalent to $\frac{y}{x} + \frac{y}{z} + \frac{x}{y} + \frac{z}{y}\leq
  \frac{x}{z} + 2 + \frac{z}{x}$.

  We know that $\frac{y}{x} + \frac{x}{y}\geq 2, \frac{y}{z} + \frac{z}{y}\geq 2$ and $\frac{x}{z} +
  \frac{z}{x}\geq 2$.

  Rewriting $\left(\frac{y}{x} - \frac{z}{x} - 1\right) + \left(\frac{y}{z} - \frac{x}{z} - 1\right) +
  \left(\frac{x}{y} + \frac{z}{y}\right)\leq 0$

  $\because x\leq y\leq z$, we have $\frac{y}{x}\leq \frac{z}{x}$ and $\frac{y}{z} \leq 1$. Also,
  $\frac{x}{y}\leq 1$ and $\frac{z}{y}\geq 1$. Hence proved.
\item $\sqrt{1 + \sqrt{a}} < \sqrt{\sqrt{a} + \sqrt{a}}$. Now $\sqrt{\sqrt{a} + \sqrt{a}} =
  \sqrt{2\sqrt{a}}$ and $\sqrt{2\sqrt{a}} < \sqrt{a\sqrt{a}} = a^{\frac{3}{4}} < a$ since $a\geq 2$.

  Similarly, $\sqrt{1 + \sqrt{a + \sqrt{a^2}}} < \sqrt{\sqrt{a + \sqrt{a^2}} + \sqrt{a + \sqrt{a^2}}}$
  and $\sqrt{\sqrt{a + \sqrt{a^2}} + \sqrt{a + \sqrt{a^2}}} = \sqrt{2\sqrt{a + a}} = \sqrt{2\sqrt{2a}}$

  $\sqrt{2\sqrt{2a}} < \sqrt{a\sqrt{2a}} = 2^{\frac{1}{4}}a^{\frac{3}{4}} <\leq a$ since $a\geq 2$

  Similarly for $k$th term, $\sqrt{1 + \sqrt{a + \cdots + \sqrt{a^k}}} < \sqrt{\sqrt{a + \cdots +
      \sqrt{a^k}} + \sqrt{a + \cdots + \sqrt{a^k}}}$

  Now $\sqrt{\sqrt{a + \cdots + \sqrt{a^k}} + \sqrt{a + \cdots + \sqrt{a^k}}} = \sqrt{2\sqrt{a + \cdots +
      \sqrt{a^k}}}$

  $=\sqrt{2\sqrt{a + \cdots + a^{\frac{k}{2^k}}}} < \sqrt{2a} < a$

  Adding all such terms we have the required inequality.
\item Let $x = n + \epsilon$, where $n$ is the integral part and $\epsilon$ is the fractional part. Then

  $5n + [5\epsilon]\geq n + \frac{2n + [2\epsilon]}{2} + \frac{3n + [3\epsilon]}{3} + \frac{4n +
  [4\epsilon]}{4} + \frac{5n + [5\epsilon]}{5}$

  $\Rightarrow [5\epsilon]\geq \frac{[2\epsilon]}{2} + \frac{[3\epsilon]}{3} + \frac{[4\epsilon]}{4} +
  \frac{[5\epsilon]}{5}$

  Now we consider different ranges of $\epsilon$. If $0\leq \epsilon< \frac{1}{5}$, the inequality becomes
  $0\geq 0$.

  If $\frac{1}{5}\leq\epsilon\leq\frac{1}{4}$, the inequality becomes $1\geq 0$.

  If $\frac{1}{4}\leq\epsilon\leq\frac{1}{3}$, the inequality becomes $1\geq \frac{9}{20}$.

  If $\frac{1}{3}\leq\epsilon\leq\frac{1}{2}$, the inequality becomes $1\geq \frac{1}{3} + \frac{1}{4} +
  \frac{1}{5}$.

  if $\frac{1}{2}\leq\epsilon\leq1$, the inequality becomes $2\geq \frac{1}{2} + \frac{1}{3} + \frac{2}{4} +
  \frac{2}{5}$.

  Thus, in all cases the inequality holds.
\item For $n = 1$, we have $(1!)^2 = 1^2 = 1$ and $1^1 = 1$. So $1\geq 1$, the inequality holds for $n = 1$.

  For $n = 2$, we have $(2!)^2 = 4$ and $2^2 = 4$, so the inequality holds true for $n = 2$.

  Let the inequality holds for $n = k$ i.e. $(k!)^2\geq k^k$.

  For $n = k + 1$, we have to prove that $[(k + 1)!]^2\geq (k + 1)^{k + 1}$.

  Now $[(k + 1)!]^2 = [(k + 1)k!]^2 = (k + 1)^2(k!)^2$ and $(k + 1)^{k + 1} = (k + 1)^k.(k + 1)$.

  So we have to show that $(k + 1)^2(k!)^2\geq (k + 1)^k.(k+ 1)$ $\Rightarrow (k + 1).k^k\geq (k + 1)^k$

  $\Rightarrow k + 1\geq \left(\frac{k + 1}{k}\right)^k$

  Now $\left(1 + \frac{1}{k}\right)^k$ has an upper bound of $e$, while $k + 1$ is an increasing
  sequence. We can prove again by induction that $k + 1\geq \left(1 + \frac{1}{k}\right)^k$.

  Hence the inequality is proved.
\item We can rewrite given expression as $\left(x^3 + \frac{x^2}{2}\right)^2 + \left(x^2 -
  \frac{x}{2}\right)^2 + (1 + x + x^2)$.

  Now we know that $1 + x + x^2 = \left(x + \frac{1}{2}\right)^2+ \frac{3}{4}$, and thus, the inequality is
  proved.
\item Given $\alpha^2\geq \beta\gamma$. Taking log of both sides $2\log\alpha\geq \log\beta + \log\gamma$.

  From A.M.-G.M. inequality we have $\log\beta + \log\gamma\geq 2\sqrt{\log\beta\log\gamma}$

  $\therefore \log\alpha\geq \sqrt{\log\beta\log\gamma}$. Squaring we prove the inequality.
\item $\log_45 > 1.1$ because $4^{1.1} \approx 4.86$. $\log_56 > 1.1$ because $5^{1.1}\approx
  5.59$. $\log_67 > 1.1$ because $6^{1.1}\approx 7.32$, and $\log_78 > 1.1$ because $7^{1.1} \approx 8.48$.

  Thus, $\log_45 + \log_56 + \log_67 + \log_78> 4.4$.
\item $\frac{n}{3.5\ldots(2n + 1)} = \frac{1}{2}.\frac{2n}{3.5\ldots(2n + 1)} =
  \frac{1}{2}\left(\frac{1}{3.5\ldots(2n - 1) - \frac{1}{3.5\ldots(2n + 1)}}\right)$

  Let $S = \frac{1}{3} + \frac{2}{3.5} + \cdots + \frac{n}{3.5\ldots(2n + 1)} = \frac{1}{2}\left[\left(1 -
    \frac{1}{3}\right) + \left(\frac{1}{3} - \frac{1}{3.5}\right) + \cdots + \left(\frac{1}{3.5.\ldots(2n -
      1)} - \frac{1}{3.5\ldots(2n + 1)}\right)\right]$

  $= \frac{1}{2}\left(1 - \frac{1}{3.5\ldots(2n + 1)}\right) < \frac{1}{2}$. Hence proved.
\item $\displaystyle\prod_{i = 2}^n\frac{i^3 + 1}{i^3 - 1} = \prod_{i = 2}^n\frac{i + 1}{i - 1}.\frac{i^2 -
  i + 1}{i^2 + i + 1}$

  Now $\displaystyle\prod_{i = 2}^n\frac{i + 1}{i - 1} = \frac{n(n + 1)}{2}$ and $\displaystyle\prod_{i =
    2}^n\frac{i^2 - i + 1}{i^2 + i + 1} = \frac{3}{n^2 + n + 1}$

  If $P$ is the given product then $P = \frac{3n(n + 1)}{2(n^2 + n + 1)}$

  Now $\displaystyle\lim_{n\to \infty} \frac{3n(n + 1)}{2(n^2 + n + 1)}= \frac{3}{2}$. Thus, $P <
  \frac{3}{2}$.
\item For $n = 1, 1.1! < (1 + 1)!$, which is true. Let the inequality be true for $n = k$ i.e. $1.1! + 2.2!
  + \cdots + k.k! < (k + 1)!$

  Now, for $n = k + 1$ the inequality becomes $1.1! + 2.2! + \cdots + k.k! + (k + 1)(k + 1)! < (k + 1)! + (k
  + 1)(k + 1)! = (k + 1)![k + 1 + 1] = (k + 2)!$

  Thus, the inequality is proven by mathematical induction.
\item $1 + \frac{1}{k^2} = \frac{k^2 + 1}{k^2} = \frac{k^2 - 1}{k^2} + \frac{2}{k^2}$.

  $\displaystyle\prod_{k = 2}^n\frac{k^2 - 1}{k^2} =
  \frac{1.3}{2^2}.\frac{2.4}{3^2}\cdots.\frac{(n - 1)(n + 1)}{n^2}$

  $= \frac{1.2.3\ldots(n - 1)}{2.3.4\ldots n}.\frac{3.4.5\ldots(n + 1)}{2.3.4\ldots n} = \frac{n + 1}{2n} =
  \frac{1}{2} + \frac{1}{2n}$

  Now $\frac{1}{2} + \frac{1}{2n}\leq \frac{1}{2} + \frac{1}{4}$ because for $n\geq 2,\;\frac{1}{2n} \leq
  \frac{1}{4}$.

  Now $\displaystyle\sum_{k=1}^{\infty}\frac{2}{k^2} = \frac{\pi^2}{3}$, which is upper bound of the
  sum. $\displaystyle\Rightarrow \sum_{k = 2}^{\infty}\frac{2}{k^2} = \frac{\pi^2}{3} - \frac{2}{1^2}$. We
  see that the sum of two components is less than $2$.

  Thus, the inequality is proven.
\item We have proven in previous example that $\displaystyle\prod_{k = 2}^n\frac{k^2 - 1}{k^2} = \frac{1}{2}
  + \frac{1}{2n}$. If we want to maximize L.H.S. then we pick continuous natural numbers starting from
  $2$. For $n\geq 2,\;\frac{1}{2n} \leq \frac{1}{4}$, however, sum is greater than $\frac{1}{2}$.
\item We can rewrite the inequality as $\frac{1}{2} - \left(\frac{1}{3} - \frac{1}{4}\right) - \cdots -
  \left(\frac{1}{999} - \frac{1}{1000}\right)$. Considering first three terms we have $\frac{1}{2} -
  \frac{1}{12} - \frac{1}{30} = \frac{23}{60}$, which is less than $\frac{2}{5}$ and further terms will make
  it lesser. Hence proved.
\item Given $\frac{a + b}{1 + a + b}\leq \frac{a}{1 + a} + \frac{b}{1 + b} \Rightarrow \frac{a}{1 + a + b} -
  \frac{a}{1 + a} + \frac{b}{1 + a + b} - \frac{b}{1 + b}\leq 0$

  $\Rightarrow \frac{a + a^2 - a - a^2 - ab}{(1 + a + b)(1 + a)} + \frac{b + b^2 - b - b^2 - ab}{(1 + a +
    b)(1 + b)}\leq 0$, which is clearly true.
\item Multiplying both sides by $2(1 + a)(1 + b)(2 + a + b)$, we get

  $2(a + b)(1 + a)(1 + b)\geq [a(1 + b) + b(1 + a)](2 + a + b)\Rightarrow (a - b)^2\geq 0$, which proves the
  inequality.
\item Let $\displaystyle S_i = \sum_{j = 1}^ia_j$ and $b_i = \frac{S_i}{i}$, which makes
  L.H.S. $\displaystyle \sum_{i=1}^n\frac{b_i}{i}$.

  Clearly, $a_i = b_i - \frac{i - 1}{i}b_{i - 1}$.

  $\displaystyle\sum_{i = 1}^na_i = b_1 + \left(b_2 - \frac{b_1}{2}\right) +\cdots + \left(b_n - \frac{n-
    1}{n}b_{n - 1}\right)$

  $= b_n + \frac{b_{n - 1}}{n} + \frac{b_{n - 2}}{n - 1} + \cdots + \frac{b_1}{2} = b_n +
  \displaystyle\sum_{i = 1}^n\frac{b_i}{i + 1}$

  Now $2\displaystyle\sum_{i = 1}^na_i - \sum_{i=1}^n\frac{b_i}{i} = \left(2 - \frac{1}{n}\right) +
  \sum_{i=1}^n\left(\frac{2}{i + 1} - \frac{1}{i}\right)b_i$

  $\frac{2}{i + 1} - \frac{1}{i} = \frac{i - 1}{i(i + 1)}$ so the sum $2\displaystyle\sum_{i = 1}^na_i -
  \sum_{i=1}^n\frac{b_i}{i} \geq 0$

  Hence proved.
\item We observe that if $a = 2, b = 3$ and $c = 7$ then $\frac{1}{a} +
  \frac{1}{b} + \frac{1}{c} = \frac{41}{42}$. If $a = 2, b = 3, c > 7$, then $\frac{1}{a} + \frac{1}{b} +
  \frac{1}{c} < \frac{41}{42}$, if $a = 2, b > 3, c > 4$, then $\frac{1}{a} + \frac{1}{b} + \frac{1}{c} <
  \frac{41}{42}$, and if $a > 2, b > 3, c > 4$, then $\frac{1}{a} + \frac{1}{b} + \frac{1}{c} <
  \frac{41}{42}$.

  Thus, we see that maximum value of $\frac{1}{a} + \frac{1}{b} + \frac{1}{c}$ is $\frac{41}{42}$.
\item {\bf{\it Nesbitt's inequality:}} From A.M.-H.M. inequality $\frac{(x + y) + (y + z) + (z + x)}{3}\geq
  \frac{3}{\frac{1}{x + y} + \frac{1}{y + z} + \frac{1}{z + x}}$

  $\Rightarrow \left[(x + y) + (y + z) + (z + x)\right]\left(\frac{1}{x + y} + \frac{1}{y + z} + \frac{1}{z
  + x}\right)\geq 9$

  $\Rightarrow 2\frac{x + y + z}{x + y} + 2\frac{x + y + z}{y + z} + 2\frac{x + y + z}{z + x}\geq 9$

  $\Rightarrow \frac{x}{y + z} + \frac{y}{z + x} + \frac{z}{x + y}\geq \frac{3}{2}$.

  Now $\frac{4x}{y + z} + \frac{y}{x + z} + \frac{z}{x + y} > 2$ is to be proven.

  $\Rightarrow \frac{3x}{y + z}\geq \frac{1}{2}$

  If $x = y = z$ then it is greater than $\frac{1}{2}$.

  If we make $y >> x$ and $z >> x$ to so that it is less than $\frac{1}{2}$ then the other terms of the
  inequality becomes greater than $2$.

  If $x >> y = z$ then it is greater than $\frac{1}{2}$.

  Thus, inequality holds in all cases.
\item Since $a, b, c, d > 0$ we can write $a + b + c, a + b + d, a + c + d, b + c + d > a + b + c + d$.

  So $\frac{a}{a + b + d} > \frac{a}{a + b + c + d}, \frac{b}{a + b + c} > \frac{b}{a + b + c + d},
  \frac{c}{b + c + d} > \frac{c}{a + b + c + d}$, and $\frac{d}{a + c + d} > \frac{d}{a + b + c + d}$

  Adding we get $\frac{a}{a + b + d} + \frac{b}{a + b + c} + \frac{c}{b + c + d} + \frac{d}{a + c + d} > 1$.


  Similarly, $\frac{a}{a + b + d} < \frac{a}{a + b}$, and $\frac{b}{a + b + c} < \frac{b}{a + b}$. Adding
  $\frac{a}{a + b + d} + \frac{b}{a + b + c} < 1$.

  Also, $\frac{c}{b + c + d} < \frac{c}{c + d}$, and $\frac{d}{a + c + d} < \frac{d}{c + d}$. Adding
  $\frac{c}{b + c + d} + \frac{d}{c + d} < 1$.

  Adding all these we prove the required inequality.
\item If $a + b\leq c + d$ then $a\leq c$ or $b\leq d$. If $a\leq c$ then from first condition $b - d> (c -
  d)(c + a) > c - a \Rightarrow a + b > c + d$.

  Similarly for other condition same can be proved. And thus, the inequality is proven.
\item $(b - a)(9 - a^2) + (c - a)(9 - b^2) + (c - b)(9 - c^2) = 9b + c(9 - b^2) + (c - b)(9 - c^2) = 18c -
  c^3 + bc(c - b)\leq 18 c - c^3 + \frac{1}{4}c^3= 18c - \frac{3}{4}c^2$

  We can use calculus to apply maxima and minima to prove the inequality.
\item We assume that all of them are greater than $\frac{1}{4}$ so $abc(1 - a)(1 - b)(1 - c) >
  \frac{1}{16}$.

  From A.M.-G.M. inequality $\frac{a + 1 - a}{1}\geq \sqrt{a(1 - a)}\Rightarrow a(1 - a)\leq
  \frac{1}{4}$. Similarly, $b(1 - b)\leq \frac{1}{4}$ and $c(1 - c)\leq \frac{1}{4}$.

  Multiplying $abc(1 - a)(1 - b)(1 - c)\leq \frac{1}{16}$, which leads to a contradiction. Thus, at least
  one of the assumed numbers is less that $\frac{1}{4}$.
\item $\sqrt{a + \frac{1}{4}(b - c)^2}\leq a + \frac{b + c}{2}$ if $a, b, c > 0$ and $a + b + c = 1$.

  Adding for all the terms we get $2(a + b + c) = 2$.
\item Using Cauchy-Schwarz inequality $\left(\sqrt{a + \frac{1}{4}(b - c)^2} + \sqrt{b} +
  \sqrt{c}\right)^2\leq (1^2 + 1^2 + 1^2)\left(a + \frac{1}{4}(b - c)^2 + b + c\right) = 3\left(1 +
  \frac{1}{4}(b^2 + c^2 - 2bc)\right) = 4\left(1 + \frac{1}{4}[(1 - a)^2 - 4bc]\right)$

  From A.M.-G.M. inequality $bc\leq \frac{(b + c)^2}{4}\Rightarrow bc\leq \frac{(1 - a)^2}{4}$.

  Thus above inequality becomes $\left(\sqrt{a + \frac{1}{4}(b - c)^2} + \sqrt{b} +
  \sqrt{c}\right)^2\leq 3\left(1 + \frac{1}{4}[(1 - a)^2 - (1 - a)^2]\right) = 3$.

  Hence proved.
\item We know that $\frac{x}{y} + \frac{y}{x}\geq 2$. Given expression is $\frac{a^4}{b^4} + \frac{b^4}{a^4}
  - \frac{a^2}{b^2} - \frac{b^2}{a^2} + \frac{a}{b} + \frac{b}{a}$

  $= \left(\frac{a^2}{b^2} + \frac{b^2}{a^2}\right) - 2 - \left(\frac{a}{b} + \frac{b}{a}\right)^2 + 2 +
  \frac{a}{b}+ \frac{b}{a}$

  Substityting $\frac{a}{b} + \frac{b}{a} = y$ we have the expression as $y^4 - 5y^2 + y + 4$. Applying
  maxima-minima we find the minimum value as $2$.
\item Since $x_1 + x_2 + \cdots + x_n = 1$, it follows by problem $9$ that there are two numbers such that
  one of them is not greater than $\frac{1}{n}$, and the other one is not less than $\frac{1}{n}$. WLOG we
  can assume that $x_1\leq x_2$. Substituting $x_1$ by $\frac{1}{n}, x_2$ by $x_1 + x_2 - \frac{1}{n}$. Then
  we obtain numbers $\frac{1}{n}, x_1 + x_2 - \frac{1}{n}, x_3, \ldots, x_n$ such that

  $\frac{(1 - x_1)\cdots(1 - x_n)}{x_1\cdots x_n}\geq \frac{\left(1 - \frac{1}{n}\right)\left(1 - x_1 - x_2
    + \frac{1}{n}\right)\cdots(1 - x_n)}{\frac{1}{n}\left(x_1 + x_2 - \frac{1}{n}\right)\cdots
    x_n}$ from problem $16$. $\frac{(1 - x_1)(1 - x_2)}{x_1x_2} = 1 + \frac{1 - (x_1 +
    x_2)}{x_1x_2}$. Repeating this step, we obtain $n$ numbers less than $\frac{1}{n}$. For these numbers
  L.H.S. of the inequality is equal to $(n - 1)^n$ and is not greater than $\frac{(1 - x)1)\cdots(1 -
    x_n)}{x_1\cdots x_n}$.
\item Let $\sqrt[n]{x_1\ldots x_n} = y$. According to problem $11$, WLOG one can assume that $x_1\leq y\leq
  x_2$, and therefore, for numbers $y, \frac{x_1x_2}{y}, x_3, \ldots, x_n$

  $\frac{1}{x_1} + \cdots + \frac{1}{1 + x_n}\geq \frac{1}{1 + y} + \frac{1}{1 + \frac{x_1x_2}{y}} + \cdots
  + \frac{1}{1 + x_n}$. Since $\frac{1}{1 + x_1} + \frac{1}{1 + x_2}\geq \frac{1}{1 + y} + \frac{1}{1 +
    \frac{x_1x_2}{y}}$ and $\frac{1}{1 + x_1} + \frac{1}{1 + x_2} = 1 + \frac{1 - x_1x_2}{1 + x_1 + x_2 +
    x_1x_2}$.

  After a finite numbere of steps, we deduce that $\frac{1}{1 + x_1} + \frac{1}{1 + x_2} + \cdots +
  \frac{1}{1 + x_n}\geq \frac{1}{1 + y} + \cdots \frac{1}{1 + y} = \frac{n}{1 + \sqrt[n]{x_1\ldots x_n}}$.
\item If $a = b = c = d = \frac{1}{4}$, then we have equality. Let $a < \frac{1}{4} < b$, then

  If $c + d - \frac{176}{27}cd < 0$, then $A = ab\left(c + d - \frac{176}{27}cd\right) + cd(a + b)\leq cd(a
  + b)\leq \left(\frac{a + b + c + d}{3}\right)^3 = \frac{1}{27}$

  If $c + d - \frac{176}{27}cd \geq 0$, then $A\leq \frac{1}{4}\left(a + b - \frac{1}{4}\right)\left(c + d -
  \frac{176}{27}cd\right) + cd\left(\frac{1}{4} + a + b - \frac{1}{4}\right)$

  Hence, we must prove inequality for numbers $a_1 = \frac{1}{4}, b_1 = a + b - \frac{1}{4}, c_1 = d, d_1 =
  d$.

  Similarly, either one can prove the inequality for numbers $a_1, b_1, c_1, d_2$ or it will be sufficient
  to prove the inequality for the case among $a_1, b_1, c_1, d_1$ two numbers are equaal to
  $\frac{1}{4}$. Continuing in this way we obtain that it is sufficient to prove the inequality for numbers
  $\frac{1}{4}, \frac{1}{4}, \frac{1}{4}, \frac{1}{4}$. In this case the inequality holds.
\item Let $x\geq y\geq z$, then $y\leq \frac{1}{2}$. Therefore, $0\leq y(x + z) + xz(1 - 2y)\leq
  y\left(\frac{1}{3} + x + z - \frac{1}{3}\right) + \frac{1}{3}\left(x + z - \frac{1}{3}\right)(1 - 2y)$

  Therefore, if we substitute the numbers $x, y, z$ by the numbers $\frac{1}{3}, y, x + z - \frac{1}{3}$ in
  the expression $xy + yz + xz - 2xyz$, then its value does not decrease. Continuing similarly, we can
  substitute the numbers $\frac{1}{3}, y, x + z - \frac{1}{3}$ by $\frac{1}{3}, \frac{1}{3},
  \frac{1}{3}$. Hence, we deduce that $xy + yz + zx - 2xyz\leq \frac{1}{9} + \frac{1}{9} + \frac{1}{9} -
  \frac{2}{27} = \frac{7}{27}$.
\item Let us first prove the following lemma.

  {\bf Lemma} If $a\leq b$ and $x > 0$, then $(a - x)^{12} + (b + x)^{12} > a^{12} + b^{12}$

  $(a - x)^{12} + (b + x)^{12} - a^{12} - b^{12} = C_{1}^^{12}x(b^{11} - a^{11}) + C_2^^{12}x^2(b^{10} - a^{10})
  + \cdots + 2x^{12} > 0$

  Let $y_i = \sqrt{3}x_i, i = 1, 2, \ldots, 1997$. We have $-1\leq y_i\leq 3$ and $y_1 + \cdots + y_{1997} =
  -954$ and $x_1^{12} + \cdots + x_{1997}^{12} = \frac{y_1^{12} + \cdots + y_{1997}^{12}}{3^6}$

  If any two numbers among the numbers $y_1, \ldots, y_{1997}$ belong to $(-1, 3)$ then aaccording to lemma,
  we can substitute these two numbers such that one of them is equal to $-1$ or $3$, then first two
  conditions hold, and $y_1^{12} + \cdots + y_{1997}^{12}$ increases.

  Therefore, the sum $y_1^{12} + \cdots + y_{1997}^{12}$ is maximum if we substitute these by either $-1,
  \ldots, -1, 3,\ldots, 3$ or $-1, \ldots, -1, 3, \ldots, 3, a$ where $a\in(-1, 3)$.

  Taking into consideration the second condition, we obtain that only the second case is possible so that $k
  = \frac{a + 2}{4} + 1735$, where $k$ is the number of $-1$\symbol[rightquote]s. Since $\frac{a + 2}{4}\in\mathbb{Z}$
  and $a\in(-1, 3)$ we must have $a = 2$.

  Therefore, the greatest value of $x_1^{12} + \cdots + x_{1997}^{12}$ is $\frac{1736 + 260.3^{12} +
    2^{12}}{3^6} = 189548$.
\item Let $\frac{\alpha_1 + \cdots + \alpha_n}{n} = \phi$. If $\alpha_1 = \cdots = \alpha_n = \phi$ then

  $\cos\alpha_1\cos\alpha_2\cdots\cos\alpha_n(\tan\alpha_1 + \tan\alpha_2 + \cdots + \tan\alpha_n) =
  \cos^n\phi.n.\tan\phi = n\sin\phi\cos^{n - 1}\phi$

  $= n\sqrt{(n - 1)^{n - 1}\sin^2\phi\left(\frac{\cos^2\phi}{n - 1}\right)^{n - 1}}\leq n\sqrt{(n - 1)^{n -
      1}\left(\frac{\sin^2\phi + \frac{\cos^2\phi}{n - 1} + \cdots + \frac{\cos^2\phi}{n - 1}}{n}\right)^n} =
  n\sqrt{(n - 1)^{n - 1}\frac{1}{n^nn}} = \frac{(n - 1)^{\frac{n - 1}{2}}}{n^{\frac{n - 2}{2}}}$.

  Let there be two numbers $\alpha_1$ and $\alpha_2$ such that $\alpha_1 < \phi < \alpha_2$.

  $\cos\alpha_1\cos\alpha_2 = \frac{1}{2}[\cos(\alpha_1 + \alpha_2) + \cos(\alpha_1 - \alpha_2)] <
  \frac{1}{2}[\cos(\alpha_1 + \alpha_2) + \cos(2\phi - (\alpha_1 + \alpha_2))]$

  $=\cos\phi\cos(\alpha_1 + \alpha_2 - \phi)$

  We have $\cos\alpha_1\cos\alpha_2\cdots\cos\alpha_n(\tan\alpha_1 + \tan\alpha_2 + \cdots + \tan\alpha_n)$

  $= \sin(\alpha_1 + \alpha_2)\cos\alpha_3 \ldots \cos\alpha_n + \cos\alpha_1\cos\alpha_2\cos\alpha_3 \ldots
  \cos\alpha_n(\tan\alpha_3 + \cdots + \tan\alpha_n) < \sin(\alpha_1 + \alpha_2)\cos\alpha_3 \ldots
  \cos\alpha_n + \cos\phi\cos(\alpha_1 + \alpha_2 - \phi)\cos\alpha_3 \ldots \cos\alpha_n(\tan\alpha_3 +
  \cdots + \tan\alpha_n)$

  $=\cos\phi\cos(\alpha_1 + \alpha_2 - \phi)\cos\alpha_3\ldots\cos\alpha_n(\tan\phi + \tan(\alpha_1 +
  \alpha_2 - \phi)+ \tan\alpha_3 + \ldots + \tan\alpha_n)$

  Continuing similarly for the numbers $\phi, \alpha_1 + \alpha_2 - \phi, \alpha_3, \ldots, \alpha_n$ we
  obtain a new sequence, two of whose terms are equal to $\phi$. Repeating these steps $n - 1$ times, we
  obtain a sequence, $n - 1$ of whose terms are equal to $\phi$ and the $n$th term is equalt to $n\phi - (n
  - 1)\phi = \phi$. Hence, we have

  $\cos\alpha_1\cos\alpha_2\cdots\cos\alpha_n(\tan\alpha_1 + \tan\alpha_2 + \cdots + \tan\alpha_n) <
  \cos^n\phi.n.\tan\phi\leq\frac{(n - 1)^{\frac{n - 1}{2}}}{n^{\frac{n - 2}{2}}}$.

  Equallity holds if and only if $\alpha_1 = \cdots = \alpha_n = \phi$, where $\phi =
  \tan^{-1}\frac{1}{\sqrt{n - 1}}$.
\item Consider $x\geq 0, y\geq 0, x + y\leq \frac{2}{3}$ and $k\geq 2, k\in\mathbb{N}$, then

  $x^k(1 - x) + y^k(1 - y)\leq(x + y)^k(1 - x - y)$

  If $x + y = 0$, then above inequality holds, while if $x + y\neq 0$, then

  $\frac{x^k}{(1 + x)^k}(1 - x) + \frac{y^k}{(x + y)^k}(1 - y)\leq \left(\frac{x}{x + y}\right)^2(1 - x) +
  \left(\frac{y}{x + y}\right)^2(1 - y) = \frac{(x + y)^2(1 - x - y) + xy[3(x + y) - 2]}{(x + y)^2}\leq 1 -
  x - y$.

  Let $x_{i + 1}\geq x_i\geq 0, i = 1, \ldots, n - 1, x_1 + \cdots + x_n = 1$ and $n\geq 3$. Then $(n -
  2)x_1 + (n - 2)x_2\leq (x_3 + \cdots + x_n) + (x_3 + \cdots + x_n) = 2 - 2x_1 - 2x_2$, and therefore,

  $x_1 + x_2\leq \frac{2}{n}\leq \frac{2}{3}$.

  Therefore, if we substitute the numbers $x_1, \ldots, x_n$ by $0, x_1 + x_2, x_3, \ldots, x_n$ then their
  sum will be equal to $1$. Note that

  $\displaystyle\sum_{i=1}^kx_i^k(1 - x_i)\leq (x_1 + x_2)^k(1 - x_1 - x_2) + x_3^k(1 - x_3) + \cdots +
  x_n^k(1 - x_n)$.

  Repeating this step a finite number of times, we end ip with case $n = 2$, i.e.,
  $\displaystyle\sum_{i=1}^nx_i^k(1 - x_i)\leq x^k(1 - x) + (1 - x)^kx$, and therefore,
  $\displaystyle\sum_{i = 1}^nx_i^k(1 - x_i)\leq a_k$.

  Note that $a_1 = \displaystyle\max_{[0;1]}[2x(1 - x)] = \frac{1}{2}, a_2 = \max_{[0;1]}[x(1 - x)] =
  \frac{1}{2}, a_3 = \max_{[0;1]}[x(1 - x)(1 - 2x(1 - x))] = \frac{1}{8}$ and so on.
\item Let $x_1\leq x_2\leq \cdots\leq x_n, n\geq 3$. Now if $x_2\cdots x_{n - 1} = \frac{2(n - 1)}{n^{n -
    1}}$, then we have

  $2(n - 1)(x_2x_3 + x_1x_3 + \cdots + x_1x_n + x_2x_3 + \cdots + x_2x_n + \cdots + x_{n - 1}x_n) - n^{n -
  1}x_1x_2\ldots x_n= 2(n - 1)[(x_1 + x_n)(x_2 + \cdots + x_{n - 1}) + x_2x_3 + \cdots + x_2x_{n - 1}] +
  x_1x_n[2(n - 1) - n^{n - 1}x_2\ldots x_{n - 1}]\leq 2(n - 1)[x(1 - x) + x_2x_3 + \cdots + x_2x_{n - 1} +
    \cdots + x_{n - 2}x_{n - 1}]$, where $x = x_2 + \cdots + x_{n - 1}$.

  From R.M.S. inequality, it follows that $\frac{x_2^2 + \cdots + x_{n - 1}^2}{n - 2}\geq \left(\frac{x}{n -
  2}\right)^2$, and therefore, $x_2x_3 + \cdots + x_2x_{n - 1} + \cdots + x_{n - 2}x_{n - 1}\leq \frac{n -
    3}{2(n - 2)}x^2$, whence

  $A\leq 2(n - 1)\left[x(1 - x) + \frac{n - 3}{2(n - 2)x^2}\right] = 4(n - 2).\frac{n - 1}{2(n - 2)}x\left(1
  - \frac{n - 1}{2(n - 2)}x\right)\leq n - 2$.

  If $x_2\ldots x_{n - 1}\leq \frac{2(n - 1)}{n^{n - 1}}$, then for $x_1 = x_2 = \cdots = x_n =
  \frac{1}{n}$, we have $A = n - 2$.

  Otherwise if $x_i \neq \frac{1}{n}$ for some value of $i$, then $x_1 < \frac{1}{n} < x_n$.

  Substituting $x_1$ by $\frac{1}{n}$ and $x_n$ by $x_1 + x_n - \frac{1}{n}$, we see that the value of the
  given expression increases.

  Continuing similalry, either we can end the proof of inequality or it will be sufficient to prove the
  inequality for $x_1 = \cdots = x_n = \frac{1}{n}$.
\item If $y_1, y_2, \ldots, y_n\geq 0$ and $y_1 + y_2 + \cdots + y_n > 0$ then according to problem $67$,
  for $x_i = \frac{y_i}{y_1 + y_2 + \cdots + y_n}, i = 1, 2, \ldots, n$ we have

  $2(n - 1)q_np_n^{n - 2} - n^{n - 1}y_1y_2\ldots y_n\leq (n - 2)p_n^n$, where $p_n = y_1 + y_2 + \cdots +
  y_n, q_n = y_1y_2 + y_1y_3 + \cdots + y_1y_n + \cdots + y_{n - 1}y_n$. Therefore, $y_1y_2\ldots y_n = 1$
  we have

  $$q_n\leq \frac{(n - 1)p_n^n + n^{n - 1}}{2(n - 1)p_n^{n - 2}}$$

  If $x_1 = 0$, then we have following inequality

  $$\frac{x_1 + \cdots + x_n}{n}\leq \frac{(\sqrt{x_2} - \sqrt{x_3})^2 + \cdots + (\sqrt{x_2} -
    \sqrt{x_n})^2 + \cdots + (\sqrt{x_{n - 1}} - \sqrt{x_n})^2 + x_2 + \cdots + x_n}{n}$$

  or $2\sqrt{x_2x_3} + 2\sqrt{x_2x_4} + \cdots + 2\sqrt{x_2x_n} + \cdots + 2\sqrt{x_{n - 2}x_n}\leq (n -
  2)(x_2 + \cdots + x_n)$.

  The last ineqeuality can be proved using $2\sqrt{ab}\leq a + b(a, b)\geq 0$.

  If $x_i > 0$, let $y_i = \frac{\sqrt{x_i}}{\sqrt[2n]{x_1x_2\ldots x_n}}, i = 1, 2, \ldots, n$, then
  $y_1y_2\ldots y_n = 1$, and we need to prove that $\frac{p_n^2 - 2q_n}{n} - 1\leq \frac{(n - 1)(p_n^2 -
    2q_n) - 2q_n}{n}$, or $q_n\leq \frac{(n - 2)p_n^2 + n}{2(n - 1)}$.

  It follows that $q_n\leq \frac{(n - 2)P_n^n + n^{n - 1}}{2(n - 1)p_n^{n - 1}}\leq \frac{(n - 2)p_n^2 +
    n}{2(n - 1)}$.
\item For numbers $y_1 = x_1^2, \ldots, y_n = x_n^2$, using the inequality of problem $68$, we deduce that

  $\frac{x_1^2 + \cdots + x_n^2}{n} - \sqrt[n]{x_1^2\cdots x_n^2}\leq \frac{(|x_1| - |x_2|)^2 + \cdots +
  (|x_1| - |x_n|)^2 + \cdots + (|x_{n - 1}| - |x_n|)^2}{n}$

  or $(n - 1)(x_1^2 + \cdots + x_n^2) + n\sqrt[n]{x_1^2\ldots x_n^2}\geq (|x_1| + \cdots + |x_n|)^2\geq (x_1
  + \cdots + x_n)^2$.
\item WLOG we can assume that $c\geq a, c\geq b$. From A.M.-G.M. inequality $\frac{a + b}{2}\geq \sqrt{ab}$
  and we know that $\frac{a}{b} + \frac{b}{a}\geq 2$.

  Note that $\frac{b}{c} + \frac{c}{a} - \frac{b}{a} - 1 = \frac{(c - a)(c - b)}{ac}\geq 0$, whence
  $\frac{b}{c} + \frac{c}{a} - \frac{b}{a} -1\geq 0$.

  Adding the inequalities we obtain $\frac{a}{b} + \frac{b}{c} + \frac{c}{a}\geq 3$.
\stopitemize
