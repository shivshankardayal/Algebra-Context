% -*- mode: context; -*-
\chapter{Inequalities}
\startitemize[n]
\item We have $a^2 + b^2 - 2ab = (a - b)^2\geq 0$. It should be noted that equality holds if and only if $a
  = b$.
\item Similar to previous problem we have $a + b - 2\sqrt{ab} = (\sqrt{a} - \sqrt{b})^2\geq 0$. Similarly,
  equality holds if and only if $a = b$.
\item Squaring $\frac{a^2 + b^2}{2}\geq \frac{a^2 + b^2 + 2ab}{4}\Rightarrow \frac{(a - b)^2}{4}\geq 0$,
  which is true. Similar to previous problems equality holds if and only if $a = b$.
\item $\frac{a + b}{2}\geq \frac{2ab}{a + b}\Rightarrow (a - b)^2\geq 0$, which is same as first problem.
\item $a + b - 1 - ab = (a - 1)(1 - b)$. We have $b < 1 < a$ making $(a - 1)(1 - b) > 0$.
\item $a^2 + b^2 - c^2 - (a + b - c)^2 = (a^2 - c^2) - [(a + b - c)^2 - b^2] = 2(a - c)(c - b) > 0$.
\item Multiplying both sides of the inequality by $ab$, where $ab > 0$, gives us $a^2 + b^2\geq 2ab$.
\item Dividing both sides of the inequality $a^2 +b^2 \geq -2ab$ by $ab$, where $ab < 0$, gives us
  the required inequality.
\item We have $x_1\leq x_2\leq \cdots\leq x_{n - 1}\leq x_n$, which gives us $nx_1\leq x_1 + x_2 + \cdots +
  x_n\leq nx_n \Rightarrow x_1\leq \frac{x_1 + x_2 + \cdots + x_n}{n}\leq x_n$.
\item We are given that $\frac{x_1}{y_1}\leq\cdots\leq \frac{x_n}{y_n}$, from which we deduce that
  $y_i\frac{x_1}{y_1}\leq x_i\leq y_i\frac{x_n}{y_n}, i = 1, \ldots, n$. Adding all these inequalities leads
  to $\frac{x_1}{y_1}(y_1 + \cdots + y_n)\leq x_1 + \cdots + x_n\leq \frac{x_n}{y_n}(y_1 +
  \cdots + y_n)$. Hence, it follows that $\frac{x_1}{y_1}\leq \frac{x_1 + \cdots + x_n}{y_1 + \cdots +
    y_n}\leq x_n$.
\item Given that $x_1\leq x_i\leq x_n, i = 1, \ldots, n$. Multiplying these inequalities $x_1^n\leq
  x_1\cdots x_n\leq x_n^n \Rightarrow x_1\leq (x_1\cdots x_n)^{\frac{1}{n}}\leq x_n$.
\item If $a_1 + \cdots + a_n \geq 0$, then $|a_1 + \cdots + a_n| = a_1 + \cdots + a_n$. Using $a\leq |a|$
  gives us $|a_1 + \cdots + a_n| = a_1 + \cdots + a_n\leq |a_1| + \cdots + |a_n|$.
\item We have $(a_1 + \cdots + a_n)\left(\frac{1}{a_1} + \cdots + \frac{1}{a_n}\right) =
  \underbrace{\Big(\left(\frac{a_1}{a_2} + \frac{a_2}{a_1}\right) + \cdots + \left(\frac{a_{n - 1}}{a_n} +
    \frac{a_n}{a_{n - 1}}\right)\Big)}_{n(n - 1)/2} + n$. We also know that $\frac{x}{y} + \frac{y}{x}\geq 2$.

  $\Rightarrow (a_1 + \cdots + a_n)\left(\frac{1}{a_1} + \cdots + \frac{1}{a_n}\right)\geq n + 2.\frac{n(n -
    1)}{2} = n^2$.
\item The given inequality is equivalent to $(a + b)\sqrt{\frac{a + b}{2}}\geq 2\sqrt{ab}\frac{\sqrt{a} +
  \sqrt{b}}{2}$, which can be obtained by multiplying the inequalities $a + b\geq 2\sqrt{ab}$ and
  $\sqrt{\frac{a + b}{2}}\geq \frac{\sqrt{a} + \sqrt{b}}{2}$.
\item We have $\frac{1}{2}(a + b) + \frac{1}{4} - \sqrt{\frac{a + b}{2}} = \left(\sqrt{\frac{a + b}{2}} -
  \frac{1}{2}\right)^2\geq 0$. Therefore, $\frac{1}{2}(a + b) + \frac{1}{4}\geq \sqrt{\frac{a + b}{2}}$.
\item Since $a(x + y − a) − xy = ax − xy + a(y − a) = (y − a)(a − x)$ and $y \geq a \geq x$, it follows that
  $(y − a)(a − x) \geq 0$. Therefore, $a(x + y − a) \geq xy$.
\item We have proven that $\frac{a + b}{2}\geq \frac{2}{\frac{1}{a} + \frac{1}{b}}$. Using this we can write
  that $\frac{\frac{1}{x - 1} + \frac{1}{x + 1}}{2}\geq \frac{2}{x - 1 + x + 1}$ or $\frac{1}{x - 1} +
  \frac{1}{x + 1}\geq \frac{2}{x}$.
\item Following from prevvious problem we have $\frac{1}{3k + 1} + \frac{1}{3k+ 3} = \frac{1}{(3k+ 2) - 1} +
  \frac{1}{(3k + 2) + 1} > \frac{2}{3k + 2}$. Therefore, $\frac{1}{3k + 1} + \frac{1}{3k + 2} + \frac{1}{3k
    + 3} > \frac{3}{3k + 2}$.

  Now we will prove that $\frac{3}{3k + 2} > \frac{1}{2k + 1} + \frac{1}{2k + 2}$.

  We find that $\frac{1}{2k + 1} + \frac{1}{2k + 2} - \frac{3}{3k + 2} = \frac{-k}{(2k + 1)(2k + 2)(3k + 3)}
  < 0$.
\item The given inequality is equivalent to $\left(\frac{2}{a + b} - 1\right)^2\leq \left(\frac{1}{a} -
  1\right)\left(\frac{1}{b} - 1\right)$. We have $\left(\frac{1}{a} - 1\right)\left(\frac{1}{b} - 1\right) -
  \left(\frac{2}{a + b} - 1\right)^2 = \frac{1}{ab} - \frac{1}{a}- \frac{1}{b} - \frac{4}{(a + b)^2} +
  \frac{4}{a + b} = \frac{1}{ab} - \frac{4}{(a + b)^2} + \frac{4}{a + b} - \frac{a + b}{ab} = \frac{(a -
    b)^2[1 - (a + b)]}{ab(a + b)^2}$ and $0\leq a\leq \frac{1}{2}, 0\leq b\leq \frac{1}{2}$. Then $\frac{(a
    - b)^2[1 - (a + b)]}{ab(a + b)^2}\geq 0$, and therefore $\left(\frac{2}{a + b} - 1\right)^2\leq
  \left(\frac{1}{a} - 1\right)\left(\frac{1}{b} - 1\right)$.
\item The given inequality is equilavent to $(2k + 1)\sqrt{3k + 4} < (2k + 2)\sqrt{3k + 1} \Rightarrow (2k +
  1)^2(3k + 4) < (2k + 2)^2(3k + 1)$, and it holds because $(2k + 2)^2(3k + 1) - (2k + 1)^2(3k + 4) = k >
  0$.
\item We know that $1 < 2 < 2^2 < \cdots < 2^{n - 1}$ and the number of positive integers $1, 2, \ldots,
  2^{n - 1}$ is equal to $n$. Thus, $2^{n - 1}\geq n$.
\item Consider a one meter long rope. Suppose we painted $\frac{1}{3}$m of this rope on first day,
  $\frac{1}{5}$ of the remaining $\frac{2}{3}$ m on second day and so on. We will find that sum of painted
  parts is less than $1$ m.

  Hence, we deduce that $\frac{1}{3} + \frac{2}{3}.\frac{1}{5} + \frac{2}{3}.\frac{4}{5}.\frac{1}{7} +
  \cdots + \frac{2}{3}.\frac{4}{5}.\frac{6}{7}\cdots \frac{100}{101}.\frac{1}{103} < 1$.
\item Observe that $1 - a\geq a$ and $1 - b\geq b$. And thus,

  $\frac{1 - a}{1 - b} + \frac{1 - b}{1 - a} = \frac{(1 - a)^2 + (1 - b)^2}{(1 - a)(1 - b)} = \frac{[(1 - a)
- (1 - b)]^2 + 2(1 - a)(1 - b)}{(1 - a)(1 - b)} = \frac{(a - b)^2}{(1 - a)(1 - b)} + 2\leq \frac{(a -
  b)^2}{ab} + 2 = \frac{a}{b}+ \frac{b}{a}$.
\item $\displaystyle\sum_{i=1}^n\frac{1}{1 - a_i}\sum_{i = 1}^n(1 - a_i) = \underbrace{\left(\frac{1 -
    a_1}{1 - a_2} + \frac{1 - a_2}{1 - a_1}\right) + \cdots + \left(\frac{1 - a_{n - 1}}{1 - a_n} + \frac{1
    - a_n}{1 - a_{n - 1}}\right)}_{n(n - 1)/2} + n$.

  Using the inequality of previous problem we have $\displaystyle\sum_{i=1}^n\frac{1}{1 - a_i}\sum_{i =
    1}^n(1 - a_i) \leq \underbrace{\left(\frac{a_1}{a_2} + \frac{a_2}{a_1}\right) + \cdots +
    \left(\frac{a_{n - 1}}{a_n} + \frac{a_n}{a_{n- 1}}\right)}_{n(n - 1)/2} + n = \sum_{i =
    1}^n\frac{1}{a_i} \sum_{i = 1}^na_i$.
\item Observe that if $n\geq 4$, then $1 + \frac{1}{2^3} + \cdots + \frac{1}{n^3} = 1 + \frac{2 - 1}{2^3} +
  \cdots + \frac{n - (n - 1)}{n^3}$

  $= \frac{5}{4} - \left(\frac{1}{2^3} - \frac{1}{3^2}\right) - \left(\frac{1}{3^3} - \frac{1}{4^2}\right) -
  \cdots - \frac{n - 1}{n^3} < \frac{5}{4}$

  because $\frac{k}{(k + 1)^3} > \frac{1}{(k + 2)^2}\;\forall\;k\in\mathbb{N}$.
\item For the given conditions $(1 - a)(1 - b)\geq 0 \Rightarrow a + b - 1\leq ab$. Thus,

  $\frac{1}{1 + a + b} - \left(1 - \frac{a + b}{2}\right) = \frac{a + b}{2(1 + a + b)}(a + b - 1)\leq
  \frac{1}{3}ab$.
\item Squaring $(x - y)^2 < (1 - xy)^2 \Rightarrow x^2 + y^2 - 2xy < 1 - 2xy + x^2y^2 \Rightarrow (1 -
  x^2)(1 - y^2) > 0$, which is true for given values of $x$ and $y$.
\item Multiplying both sides by $abc, a^2 + b^2 + c^2\geq 2bc + 2ac - 2ab \Rightarrow (a + b - c)^2\geq 0$,
  which is true.
\item Multiplpying both sides by $abc, bc + ac - ab < 1 \Rightarrow ab > bc + ca - 1$. We know that $ab \leq
  \frac{a^2 + b^2}{2}$ from A.M.-G.M. inequality and $a^2 + b^2 = \frac{5}{3} - c^2\Rightarrow ab \leq
  \frac{5}{6} - \frac{c^2}{2}$.

  Rearranging the inequality, $c(a + b)\leq \frac{11}{6} - \frac{c^2}{2}$. We know that $(a + b)^2\leq 2(a^2
  + b^2)$

  $\Rightarrow c(a + b)\leq c\sqrt{2\left(\frac{5}{4} - c^2\right)}$. Since $c(a + b)\leq \frac{5}{6} + ab$
  and $ab < 1$, then $c(a + b) < \frac{11}{6}$. Thus, $ab > c(a + b) - 1$.
\item Given $3(1 + a^2 + a^4)\geq (1 + a + a^2)^2\Rightarrow 2(1- a)^2(1 + a + a^2)\geq 0$, which is true
  because $1 + a + a^2 = \left(a + \frac{1}{2}\right)^2 + \frac{3}{4}$, which is always positive.
\item $a^2 + b^2\geq \frac{(a + b)^2}{2} = 8$ and $c^2 + d^2\geq \frac{(c + d)^2}{2} = 18$.

  Now $(ac + bd)^2 + (ad - bc)^2 = (a^2 + b^2)(c^2 + d^2)\geq 8.18 = 144$.
\item $x_1^2 + x_2^2 + \cdots + x_{2n}^2 + na^2 - a\sqrt{2}(x_1 + x_2 + \cdots + x_{2n})\geq 0$

  $\Rightarrow \left(x_1 - \frac{a}{\sqrt{2}}\right)^2 + \left(x_2 - \frac{a}{\sqrt{2}}\right)^2 + \cdots +
  \left(x_{2n} - \frac{a}{\sqrt{2}}\right)^2\geq 0$, which is true.
\item From A.M.-H.M. inequaltiy $\frac{1}{a} + \frac{1}{b}\geq \frac{4}{a + b}$. Thus,

  $\frac{1}{2}\left(\frac{1}{a} + \frac{1}{b}+ \frac{1}{c}\right)\geq \frac{1}{a + b} + \frac{1}{b + c} +
  \frac{1}{c + a}$

  Now $bc + ca + ab\leq (a + b + c)^2 \leq (\sqrt{ab} + \sqrt{bc} + \sqrt{ca})^2$

  $\Rightarrow \frac{1}{2}\left(\frac{1}{a} + \frac{1}{b} + \frac{1}{c}\right)\leq
  \frac{1}{2}\left(\frac{1}{\sqrt{a}} + \frac{1}{\sqrt{b}} + \frac{1}{\sqrt{c}}\right)$

  Hence proved.
\item We have to prove that $a^3(b^2 - c^2) + b^3(c^2 - a^2) + c^3(a^2 - b^2) < 0$.

  $a^3(b^2 - c^2) + b^3(c^2 - a^2) + c^3(a^2 - b^2) = a^2b^2(a -b) - a^2c^2(a - c) + b^2c^2(b - c)$

  $= a^2b^2(a -b) - a^2c^2(a - b + b - c) + b^2c^2(b - c) = (a - b)(b - c)[a^2b + a^2c - c^2b - c^2a]$

  $= (a - b)(b - c)[b(a^2 - c^2) + ac(a - c)] = (a - b)(b - c)(a - c)[ab + bc + ac]$, which is obviously
  less than zero.
\item Clearly, $2(a^3b + b^3c + c^3a)\geq a^3(b + c) + b^3(a + c) + c^3(a + b)$ using rearrangement
  inequality.

  Using A.M.-G.M. inequality $a^3b + b^3a\geq 2a^2b^2, b^3c + c^3b\geq 2b^2c^2, c^3a + ca^3\geq 2c^2a^2$.

  From these two inequalities we have the desired inequality.
\item The given inequality is equivalent to $\frac{y}{x} + \frac{y}{z} + \frac{x}{y} + \frac{z}{y}\leq
  \frac{x}{z} + 2 + \frac{z}{x}$.

  We know that $\frac{y}{x} + \frac{x}{y}\geq 2, \frac{y}{z} + \frac{z}{y}\geq 2$ and $\frac{x}{z} +
  \frac{z}{x}\geq 2$.

  Rewriting $\left(\frac{y}{x} - \frac{z}{x} - 1\right) + \left(\frac{y}{z} - \frac{x}{z} - 1\right) +
  \left(\frac{x}{y} + \frac{z}{y}\right)\leq 0$

  $\because x\leq y\leq z$, we have $\frac{y}{x}\leq \frac{z}{x}$ and $\frac{y}{z} \leq 1$. Also,
  $\frac{x}{y}\leq 1$ and $\frac{z}{y}\geq 1$. Hence proved.
\item $\sqrt{1 + \sqrt{a}} < \sqrt{\sqrt{a} + \sqrt{a}}$. Now $\sqrt{\sqrt{a} + \sqrt{a}} =
  \sqrt{2\sqrt{a}}$ and $\sqrt{2\sqrt{a}} < \sqrt{a\sqrt{a}} = a^{\frac{3}{4}} < a$ since $a\geq 2$.

  Similarly, $\sqrt{1 + \sqrt{a + \sqrt{a^2}}} < \sqrt{\sqrt{a + \sqrt{a^2}} + \sqrt{a + \sqrt{a^2}}}$
  and $\sqrt{\sqrt{a + \sqrt{a^2}} + \sqrt{a + \sqrt{a^2}}} = \sqrt{2\sqrt{a + a}} = \sqrt{2\sqrt{2a}}$

  $\sqrt{2\sqrt{2a}} < \sqrt{a\sqrt{2a}} = 2^{\frac{1}{4}}a^{\frac{3}{4}} <\leq a$ since $a\geq 2$

  Similarly for $k$th term, $\sqrt{1 + \sqrt{a + \cdots + \sqrt{a^k}}} < \sqrt{\sqrt{a + \cdots +
      \sqrt{a^k}} + \sqrt{a + \cdots + \sqrt{a^k}}}$

  Now $\sqrt{\sqrt{a + \cdots + \sqrt{a^k}} + \sqrt{a + \cdots + \sqrt{a^k}}} = \sqrt{2\sqrt{a + \cdots +
      \sqrt{a^k}}}$

  $=\sqrt{2\sqrt{a + \cdots + a^{\frac{k}{2^k}}}} < \sqrt{2a} < a$

  Adding all such terms we have the required inequality.
\item Let $x = n + \epsilon$, where $n$ is the integral part and $\epsilon$ is the fractional part. Then

  $5n + [5\epsilon]\geq n + \frac{2n + [2\epsilon]}{2} + \frac{3n + [3\epsilon]}{3} + \frac{4n +
  [4\epsilon]}{4} + \frac{5n + [5\epsilon]}{5}$

  $\Rightarrow [5\epsilon]\geq \frac{[2\epsilon]}{2} + \frac{[3\epsilon]}{3} + \frac{[4\epsilon]}{4} +
  \frac{[5\epsilon]}{5}$

  Now we consider different ranges of $\epsilon$. If $0\leq \epsilon< \frac{1}{5}$, the inequality becomes
  $0\geq 0$.

  If $\frac{1}{5}\leq\epsilon\leq\frac{1}{4}$, the inequality becomes $1\geq 0$.

  If $\frac{1}{4}\leq\epsilon\leq\frac{1}{3}$, the inequality becomes $1\geq \frac{9}{20}$.

  If $\frac{1}{3}\leq\epsilon\leq\frac{1}{2}$, the inequality becomes $1\geq \frac{1}{3} + \frac{1}{4} +
  \frac{1}{5}$.

  if $\frac{1}{2}\leq\epsilon\leq1$, the inequality becomes $2\geq \frac{1}{2} + \frac{1}{3} + \frac{2}{4} +
  \frac{2}{5}$.

  Thus, in all cases the inequality holds.
\stopitemize
