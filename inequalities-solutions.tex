% -*- mode: context; -*-
\chapter{Inequalities}
\startitemize[n]
\item We have $a^2 + b^2 - 2ab = (a - b)^2\geq 0$. It should be noted that equality holds if and only if $a
  = b$.
\item Similar to previous problem we have $a + b - 2\sqrt{ab} = (\sqrt{a} - \sqrt{b})^2\geq 0$. Similarly,
  equality holds if and only if $a = b$.
\item Squaring $\frac{a^2 + b^2}{2}\geq \frac{a^2 + b^2 + 2ab}{4}\Rightarrow \frac{(a - b)^2}{4}\geq 0$,
  which is true. Similar to previous problems equality holds if and only if $a = b$.
\item $\frac{a + b}{2}\geq \frac{2ab}{a + b}\Rightarrow (a - b)^2\geq 0$, which is same as first problem.
\item $a + b - 1 - ab = (a - 1)(1 - b)$. We have $b < 1 < a$ making $(a - 1)(1 - b) > 0$.
\item $a^2 + b^2 - c^2 - (a + b - c)^2 = (a^2 - c^2) - [(a + b - c)^2 - b^2] = 2(a - c)(c - b) > 0$.
\item Multiplying both sides of the inequality by $ab$, where $ab > 0$, gives us $a^2 + b^2\geq 2ab$.
\item Dividing both sides of the inequality $a^2 +b^2 \geq -2ab$ by $ab$, where $ab < 0$, gives us
  the required inequality.
\item We have $x_1\leq x_2\leq \cdots\leq x_{n - 1}\leq x_n$, which gives us $nx_1\leq x_1 + x_2 + \cdots +
  x_n\leq nx_n \Rightarrow x_1\leq \frac{x_1 + x_2 + \cdots + x_n}{n}\leq x_n$.
\item We are given that $\frac{x_1}{y_1}\leq\cdots\leq \frac{x_n}{y_n}$, from which we deduce that
  $y_i\frac{x_1}{y_1}\leq x_i\leq y_i\frac{x_n}{y_n}, i = 1, \ldots, n$. Adding all these inequalities leads
  to $\frac{x_1}{y_1}(y_1 + \cdots + y_n)\leq x_1 + \cdots + x_n\leq \frac{x_n}{y_n}(y_1 +
  \cdots + y_n)$. Hence, it follows that $\frac{x_1}{y_1}\leq \frac{x_1 + \cdots + x_n}{y_1 + \cdots +
    y_n}\leq x_n$.
\item Given that $x_1\leq x_i\leq x_n, i = 1, \ldots, n$. Multiplying these inequalities $x_1^n\leq
  x_1\cdots x_n\leq x_n^n \Rightarrow x_1\leq (x_1\cdots x_n)^{\frac{1}{n}}\leq x_n$.
\item If $a_1 + \cdots + a_n \geq 0$, then $|a_1 + \cdots + a_n| = a_1 + \cdots + a_n$. Using $a\leq |a|$
  gives us $|a_1 + \cdots + a_n| = a_1 + \cdots + a_n\leq |a_1| + \cdots + |a_n|$.
\item We have $(a_1 + \cdots + a_n)\left(\frac{1}{a_1} + \cdots + \frac{1}{a_n}\right) =
  \underbrace{\Big(\left(\frac{a_1}{a_2} + \frac{a_2}{a_1}\right) + \cdots + \left(\frac{a_{n - 1}}{a_n} +
    \frac{a_n}{a_{n - 1}}\right)\Big)}_{n(n - 1)/2} + n$. We also know that $\frac{x}{y} + \frac{y}{x}\geq 2$.

  $\Rightarrow (a_1 + \cdots + a_n)\left(\frac{1}{a_1} + \cdots + \frac{1}{a_n}\right)\geq n + 2.\frac{n(n -
    1)}{2} = n^2$.
\item The given inequality is equivalent to $(a + b)\sqrt{\frac{a + b}{2}}\geq 2\sqrt{ab}\frac{\sqrt{a} +
  \sqrt{b}}{2}$, which can be obtained by multiplying the inequalities $a + b\geq 2\sqrt{ab}$ and
  $\sqrt{\frac{a + b}{2}}\geq \frac{\sqrt{a} + \sqrt{b}}{2}$.
\item We have $\frac{1}{2}(a + b) + \frac{1}{4} - \sqrt{\frac{a + b}{2}} = \left(\sqrt{\frac{a + b}{2}} -
  \frac{1}{2}\right)^2\geq 0$. Therefore, $\frac{1}{2}(a + b) + \frac{1}{4}\geq \sqrt{\frac{a + b}{2}}$.
\item Since $a(x + y − a) − xy = ax − xy + a(y − a) = (y − a)(a − x)$ and $y \geq a \geq x$, it follows that
  $(y − a)(a − x) \geq 0$. Therefore, $a(x + y − a) \geq xy$.
\item We have proven that $\frac{a + b}{2}\geq \frac{2}{\frac{1}{a} + \frac{1}{b}}$. Using this we can write
  that $\frac{\frac{1}{x - 1} + \frac{1}{x + 1}}{2}\geq \frac{2}{x - 1 + x + 1}$ or $\frac{1}{x - 1} +
  \frac{1}{x + 1}\geq \frac{2}{x}$.
\item Following from prevvious problem we have $\frac{1}{3k + 1} + \frac{1}{3k+ 3} = \frac{1}{(3k+ 2) - 1} +
  \frac{1}{(3k + 2) + 1} > \frac{2}{3k + 2}$. Therefore, $\frac{1}{3k + 1} + \frac{1}{3k + 2} + \frac{1}{3k
    + 3} > \frac{3}{3k + 2}$.

  Now we will prove that $\frac{3}{3k + 2} > \frac{1}{2k + 1} + \frac{1}{2k + 2}$.

  We find that $\frac{1}{2k + 1} + \frac{1}{2k + 2} - \frac{3}{3k + 2} = \frac{-k}{(2k + 1)(2k + 2)(3k + 3)}
  < 0$.
\item The given inequality is equivalent to $\left(\frac{2}{a + b} - 1\right)^2\leq \left(\frac{1}{a} -
  1\right)\left(\frac{1}{b} - 1\right)$. We have $\left(\frac{1}{a} - 1\right)\left(\frac{1}{b} - 1\right) -
  \left(\frac{2}{a + b} - 1\right)^2 = \frac{1}{ab} - \frac{1}{a}- \frac{1}{b} - \frac{4}{(a + b)^2} +
  \frac{4}{a + b} = \frac{1}{ab} - \frac{4}{(a + b)^2} + \frac{4}{a + b} - \frac{a + b}{ab} = \frac{(a -
    b)^2[1 - (a + b)]}{ab(a + b)^2}$ and $0\leq a\leq \frac{1}{2}, 0\leq b\leq \frac{1}{2}$. Then $\frac{(a
    - b)^2[1 - (a + b)]}{ab(a + b)^2}\geq 0$, and therefore $\left(\frac{2}{a + b} - 1\right)^2\leq
  \left(\frac{1}{a} - 1\right)\left(\frac{1}{b} - 1\right)$.
\item The given inequality is equilavent to $(2k + 1)\sqrt{3k + 4} < (2k + 2)\sqrt{3k + 1} \Rightarrow (2k +
  1)^2(3k + 4) < (2k + 2)^2(3k + 1)$, and it holds because $(2k + 2)^2(3k + 1) - (2k + 1)^2(3k + 4) = k >
  0$.
\item We know that $1 < 2 < 2^2 < \cdots < 2^{n - 1}$ and the number of positive integers $1, 2, \ldots,
  2^{n - 1}$ is equal to $n$. Thus, $2^{n - 1}\geq n$.
\item Consider a one meter long rope. Suppose we painted $\frac{1}{3}$m of this rope on first day,
  $\frac{1}{5}$ of the remaining $\frac{2}{3}$ m on second day and so on. We will find that sum of painted
  parts is less than $1$ m.

  Hence, we deduce that $\frac{1}{3} + \frac{2}{3}.\frac{1}{5} + \frac{2}{3}.\frac{4}{5}.\frac{1}{7} +
  \cdots + \frac{2}{3}.\frac{4}{5}.\frac{6}{7}\cdots \frac{100}{101}.\frac{1}{103} < 1$.
\item Observe that $1 - a\geq a$ and $1 - b\geq b$. And thus,

  $\frac{1 - a}{1 - b} + \frac{1 - b}{1 - a} = \frac{(1 - a)^2 + (1 - b)^2}{(1 - a)(1 - b)} = \frac{[(1 - a)
- (1 - b)]^2 + 2(1 - a)(1 - b)}{(1 - a)(1 - b)} = \frac{(a - b)^2}{(1 - a)(1 - b)} + 2\leq \frac{(a -
  b)^2}{ab} + 2 = \frac{a}{b}+ \frac{b}{a}$.
\item $\displaystyle\sum_{i=1}^n\frac{1}{1 - a_i}\sum_{i = 1}^n(1 - a_i) = \underbrace{\left(\frac{1 -
    a_1}{1 - a_2} + \frac{1 - a_2}{1 - a_1}\right) + \cdots + \left(\frac{1 - a_{n - 1}}{1 - a_n} + \frac{1
    - a_n}{1 - a_{n - 1}}\right)}_{n(n - 1)/2} + n$.

  Using the inequality of previous problem we have $\displaystyle\sum_{i=1}^n\frac{1}{1 - a_i}\sum_{i =
    1}^n(1 - a_i) \leq \underbrace{\left(\frac{a_1}{a_2} + \frac{a_2}{a_1}\right) + \cdots +
    \left(\frac{a_{n - 1}}{a_n} + \frac{a_n}{a_{n- 1}}\right)}_{n(n - 1)/2} + n = \sum_{i =
    1}^n\frac{1}{a_i} \sum_{i = 1}^na_i$.
\item Observe that if $n\geq 4$, then $1 + \frac{1}{2^3} + \cdots + \frac{1}{n^3} = 1 + \frac{2 - 1}{2^3} +
  \cdots + \frac{n - (n - 1)}{n^3}$

  $= \frac{5}{4} - \left(\frac{1}{2^3} - \frac{1}{3^2}\right) - \left(\frac{1}{3^3} - \frac{1}{4^2}\right) -
  \cdots - \frac{n - 1}{n^3} < \frac{5}{4}$

  because $\frac{k}{(k + 1)^3} > \frac{1}{(k + 2)^2}\;\forall\;k\in\mathbb{N}$.
\item For the given conditions $(1 - a)(1 - b)\geq 0 \Rightarrow a + b - 1\leq ab$. Thus,

  $\frac{1}{1 + a + b} - \left(1 - \frac{a + b}{2}\right) = \frac{a + b}{2(1 + a + b)}(a + b - 1)\leq
  \frac{1}{3}ab$.
\item Squaring $(x - y)^2 < (1 - xy)^2 \Rightarrow x^2 + y^2 - 2xy < 1 - 2xy + x^2y^2 \Rightarrow (1 -
  x^2)(1 - y^2) > 0$, which is true for given values of $x$ and $y$.
\item Multiplying both sides by $abc, a^2 + b^2 + c^2\geq 2bc + 2ac - 2ab \Rightarrow (a + b - c)^2\geq 0$,
  which is true.
\item Multiplpying both sides by $abc, bc + ac - ab < 1 \Rightarrow ab > bc + ca - 1$. We know that $ab \leq
  \frac{a^2 + b^2}{2}$ from A.M.-G.M. inequality and $a^2 + b^2 = \frac{5}{3} - c^2\Rightarrow ab \leq
  \frac{5}{6} - \frac{c^2}{2}$.

  Rearranging the inequality, $c(a + b)\leq \frac{11}{6} - \frac{c^2}{2}$. We know that $(a + b)^2\leq 2(a^2
  + b^2)$

  $\Rightarrow c(a + b)\leq c\sqrt{2\left(\frac{5}{4} - c^2\right)}$. Since $c(a + b)\leq \frac{5}{6} + ab$
  and $ab < 1$, then $c(a + b) < \frac{11}{6}$. Thus, $ab > c(a + b) - 1$.
\item Given $3(1 + a^2 + a^4)\geq (1 + a + a^2)^2\Rightarrow 2(1- a)^2(1 + a + a^2)\geq 0$, which is true
  because $1 + a + a^2 = \left(a + \frac{1}{2}\right)^2 + \frac{3}{4}$, which is always positive.
\item $a^2 + b^2\geq \frac{(a + b)^2}{2} = 8$ and $c^2 + d^2\geq \frac{(c + d)^2}{2} = 18$.

  Now $(ac + bd)^2 + (ad - bc)^2 = (a^2 + b^2)(c^2 + d^2)\geq 8.18 = 144$.
\item $x_1^2 + x_2^2 + \cdots + x_{2n}^2 + na^2 - a\sqrt{2}(x_1 + x_2 + \cdots + x_{2n})\geq 0$

  $\Rightarrow \left(x_1 - \frac{a}{\sqrt{2}}\right)^2 + \left(x_2 - \frac{a}{\sqrt{2}}\right)^2 + \cdots +
  \left(x_{2n} - \frac{a}{\sqrt{2}}\right)^2\geq 0$, which is true.
\item From A.M.-H.M. inequaltiy $\frac{1}{a} + \frac{1}{b}\geq \frac{4}{a + b}$. Thus,

  $\frac{1}{2}\left(\frac{1}{a} + \frac{1}{b}+ \frac{1}{c}\right)\geq \frac{1}{a + b} + \frac{1}{b + c} +
  \frac{1}{c + a}$

  Now $bc + ca + ab\leq (a + b + c)^2 \leq (\sqrt{ab} + \sqrt{bc} + \sqrt{ca})^2$

  $\Rightarrow \frac{1}{2}\left(\frac{1}{a} + \frac{1}{b} + \frac{1}{c}\right)\leq
  \frac{1}{2}\left(\frac{1}{\sqrt{a}} + \frac{1}{\sqrt{b}} + \frac{1}{\sqrt{c}}\right)$

  Hence proved.
\item We have to prove that $a^3(b^2 - c^2) + b^3(c^2 - a^2) + c^3(a^2 - b^2) < 0$.

  $a^3(b^2 - c^2) + b^3(c^2 - a^2) + c^3(a^2 - b^2) = a^2b^2(a -b) - a^2c^2(a - c) + b^2c^2(b - c)$

  $= a^2b^2(a -b) - a^2c^2(a - b + b - c) + b^2c^2(b - c) = (a - b)(b - c)[a^2b + a^2c - c^2b - c^2a]$

  $= (a - b)(b - c)[b(a^2 - c^2) + ac(a - c)] = (a - b)(b - c)(a - c)[ab + bc + ac]$, which is obviously
  less than zero.
\item Clearly, $2(a^3b + b^3c + c^3a)\geq a^3(b + c) + b^3(a + c) + c^3(a + b)$ using rearrangement
  inequality.

  Using A.M.-G.M. inequality $a^3b + b^3a\geq 2a^2b^2, b^3c + c^3b\geq 2b^2c^2, c^3a + ca^3\geq 2c^2a^2$.

  From these two inequalities we have the desired inequality.
\item The given inequality is equivalent to $\frac{y}{x} + \frac{y}{z} + \frac{x}{y} + \frac{z}{y}\leq
  \frac{x}{z} + 2 + \frac{z}{x}$.

  We know that $\frac{y}{x} + \frac{x}{y}\geq 2, \frac{y}{z} + \frac{z}{y}\geq 2$ and $\frac{x}{z} +
  \frac{z}{x}\geq 2$.

  Rewriting $\left(\frac{y}{x} - \frac{z}{x} - 1\right) + \left(\frac{y}{z} - \frac{x}{z} - 1\right) +
  \left(\frac{x}{y} + \frac{z}{y}\right)\leq 0$

  $\because x\leq y\leq z$, we have $\frac{y}{x}\leq \frac{z}{x}$ and $\frac{y}{z} \leq 1$. Also,
  $\frac{x}{y}\leq 1$ and $\frac{z}{y}\geq 1$. Hence proved.
\item $\sqrt{1 + \sqrt{a}} < \sqrt{\sqrt{a} + \sqrt{a}}$. Now $\sqrt{\sqrt{a} + \sqrt{a}} =
  \sqrt{2\sqrt{a}}$ and $\sqrt{2\sqrt{a}} < \sqrt{a\sqrt{a}} = a^{\frac{3}{4}} < a$ since $a\geq 2$.

  Similarly, $\sqrt{1 + \sqrt{a + \sqrt{a^2}}} < \sqrt{\sqrt{a + \sqrt{a^2}} + \sqrt{a + \sqrt{a^2}}}$
  and $\sqrt{\sqrt{a + \sqrt{a^2}} + \sqrt{a + \sqrt{a^2}}} = \sqrt{2\sqrt{a + a}} = \sqrt{2\sqrt{2a}}$

  $\sqrt{2\sqrt{2a}} < \sqrt{a\sqrt{2a}} = 2^{\frac{1}{4}}a^{\frac{3}{4}} <\leq a$ since $a\geq 2$

  Similarly for $k$th term, $\sqrt{1 + \sqrt{a + \cdots + \sqrt{a^k}}} < \sqrt{\sqrt{a + \cdots +
      \sqrt{a^k}} + \sqrt{a + \cdots + \sqrt{a^k}}}$

  Now $\sqrt{\sqrt{a + \cdots + \sqrt{a^k}} + \sqrt{a + \cdots + \sqrt{a^k}}} = \sqrt{2\sqrt{a + \cdots +
      \sqrt{a^k}}}$

  $=\sqrt{2\sqrt{a + \cdots + a^{\frac{k}{2^k}}}} < \sqrt{2a} < a$

  Adding all such terms we have the required inequality.
\item Let $x = n + \epsilon$, where $n$ is the integral part and $\epsilon$ is the fractional part. Then

  $5n + [5\epsilon]\geq n + \frac{2n + [2\epsilon]}{2} + \frac{3n + [3\epsilon]}{3} + \frac{4n +
  [4\epsilon]}{4} + \frac{5n + [5\epsilon]}{5}$

  $\Rightarrow [5\epsilon]\geq \frac{[2\epsilon]}{2} + \frac{[3\epsilon]}{3} + \frac{[4\epsilon]}{4} +
  \frac{[5\epsilon]}{5}$

  Now we consider different ranges of $\epsilon$. If $0\leq \epsilon< \frac{1}{5}$, the inequality becomes
  $0\geq 0$.

  If $\frac{1}{5}\leq\epsilon\leq\frac{1}{4}$, the inequality becomes $1\geq 0$.

  If $\frac{1}{4}\leq\epsilon\leq\frac{1}{3}$, the inequality becomes $1\geq \frac{9}{20}$.

  If $\frac{1}{3}\leq\epsilon\leq\frac{1}{2}$, the inequality becomes $1\geq \frac{1}{3} + \frac{1}{4} +
  \frac{1}{5}$.

  if $\frac{1}{2}\leq\epsilon\leq1$, the inequality becomes $2\geq \frac{1}{2} + \frac{1}{3} + \frac{2}{4} +
  \frac{2}{5}$.

  Thus, in all cases the inequality holds.
\item For $n = 1$, we have $(1!)^2 = 1^2 = 1$ and $1^1 = 1$. So $1\geq 1$, the inequality holds for $n = 1$.

  For $n = 2$, we have $(2!)^2 = 4$ and $2^2 = 4$, so the inequality holds true for $n = 2$.

  Let the inequality holds for $n = k$ i.e. $(k!)^2\geq k^k$.

  For $n = k + 1$, we have to prove that $[(k + 1)!]^2\geq (k + 1)^{k + 1}$.

  Now $[(k + 1)!]^2 = [(k + 1)k!]^2 = (k + 1)^2(k!)^2$ and $(k + 1)^{k + 1} = (k + 1)^k.(k + 1)$.

  So we have to show that $(k + 1)^2(k!)^2\geq (k + 1)^k.(k+ 1)$ $\Rightarrow (k + 1).k^k\geq (k + 1)^k$

  $\Rightarrow k + 1\geq \left(\frac{k + 1}{k}\right)^k$

  Now $\left(1 + \frac{1}{k}\right)^k$ has an upper bound of $e$, while $k + 1$ is an increasing
  sequence. We can prove again by induction that $k + 1\geq \left(1 + \frac{1}{k}\right)^k$.

  Hence the inequality is proved.
\item We can rewrite given expression as $\left(x^3 + \frac{x^2}{2}\right)^2 + \left(x^2 -
  \frac{x}{2}\right)^2 + (1 + x + x^2)$.

  Now we know that $1 + x + x^2 = \left(x + \frac{1}{2}\right)^2+ \frac{3}{4}$, and thus, the inequality is
  proved.
\item Given $\alpha^2\geq \beta\gamma$. Taking log of both sides $2\log\alpha\geq \log\beta + \log\gamma$.

  From A.M.-G.M. inequality we have $\log\beta + \log\gamma\geq 2\sqrt{\log\beta\log\gamma}$

  $\therefore \log\alpha\geq \sqrt{\log\beta\log\gamma}$. Squaring we prove the inequality.
\item $\log_45 > 1.1$ because $4^{1.1} \approx 4.86$. $\log_56 > 1.1$ because $5^{1.1}\approx
  5.59$. $\log_67 > 1.1$ because $6^{1.1}\approx 7.32$, and $\log_78 > 1.1$ because $7^{1.1} \approx 8.48$.

  Thus, $\log_45 + \log_56 + \log_67 + \log_78> 4.4$.
\item $\frac{n}{3.5\ldots(2n + 1)} = \frac{1}{2}.\frac{2n}{3.5\ldots(2n + 1)} =
  \frac{1}{2}\left(\frac{1}{3.5\ldots(2n - 1) - \frac{1}{3.5\ldots(2n + 1)}}\right)$

  Let $S = \frac{1}{3} + \frac{2}{3.5} + \cdots + \frac{n}{3.5\ldots(2n + 1)} = \frac{1}{2}\left[\left(1 -
    \frac{1}{3}\right) + \left(\frac{1}{3} - \frac{1}{3.5}\right) + \cdots + \left(\frac{1}{3.5.\ldots(2n -
      1)} - \frac{1}{3.5\ldots(2n + 1)}\right)\right]$

  $= \frac{1}{2}\left(1 - \frac{1}{3.5\ldots(2n + 1)}\right) < \frac{1}{2}$. Hence proved.
\item $\displaystyle\prod_{i = 2}^n\frac{i^3 + 1}{i^3 - 1} = \prod_{i = 2}^n\frac{i + 1}{i - 1}.\frac{i^2 -
  i + 1}{i^2 + i + 1}$

  Now $\displaystyle\prod_{i = 2}^n\frac{i + 1}{i - 1} = \frac{n(n + 1)}{2}$ and $\displaystyle\prod_{i =
    2}^n\frac{i^2 - i + 1}{i^2 + i + 1} = \frac{3}{n^2 + n + 1}$

  If $P$ is the given product then $P = \frac{3n(n + 1)}{2(n^2 + n + 1)}$

  Now $\displaystyle\lim_{n\to \infty} \frac{3n(n + 1)}{2(n^2 + n + 1)}= \frac{3}{2}$. Thus, $P <
  \frac{3}{2}$.
\item For $n = 1, 1.1! < (1 + 1)!$, which is true. Let the inequality be true for $n = k$ i.e. $1.1! + 2.2!
  + \cdots + k.k! < (k + 1)!$

  Now, for $n = k + 1$ the inequality becomes $1.1! + 2.2! + \cdots + k.k! + (k + 1)(k + 1)! < (k + 1)! + (k
  + 1)(k + 1)! = (k + 1)![k + 1 + 1] = (k + 2)!$

  Thus, the inequality is proven by mathematical induction.
\item $1 + \frac{1}{k^2} = \frac{k^2 + 1}{k^2} = \frac{k^2 - 1}{k^2} + \frac{2}{k^2}$.

  $\displaystyle\prod_{k = 2}^n\frac{k^2 - 1}{k^2} =
  \frac{1.3}{2^2}.\frac{2.4}{3^2}\cdots.\frac{(n - 1)(n + 1)}{n^2}$

  $= \frac{1.2.3\ldots(n - 1)}{2.3.4\ldots n}.\frac{3.4.5\ldots(n + 1)}{2.3.4\ldots n} = \frac{n + 1}{2n} =
  \frac{1}{2} + \frac{1}{2n}$

  Now $\frac{1}{2} + \frac{1}{2n}\leq \frac{1}{2} + \frac{1}{4}$ because for $n\geq 2,\;\frac{1}{2n} \leq
  \frac{1}{4}$.

  Now $\displaystyle\sum_{k=1}^{\infty}\frac{2}{k^2} = \frac{\pi^2}{3}$, which is upper bound of the
  sum. $\displaystyle\Rightarrow \sum_{k = 2}^{\infty}\frac{2}{k^2} = \frac{\pi^2}{3} - \frac{2}{1^2}$. We
  see that the sum of two components is less than $2$.

  Thus, the inequality is proven.
\item We have proven in previous example that $\displaystyle\prod_{k = 2}^n\frac{k^2 - 1}{k^2} = \frac{1}{2}
  + \frac{1}{2n}$. If we want to maximize L.H.S. then we pick continuous natural numbers starting from
  $2$. For $n\geq 2,\;\frac{1}{2n} \leq \frac{1}{4}$, however, sum is greater than $\frac{1}{2}$.
\item We can rewrite the inequality as $\frac{1}{2} - \left(\frac{1}{3} - \frac{1}{4}\right) - \cdots -
  \left(\frac{1}{999} - \frac{1}{1000}\right)$. Considering first three terms we have $\frac{1}{2} -
  \frac{1}{12} - \frac{1}{30} = \frac{23}{60}$, which is less than $\frac{2}{5}$ and further terms will make
  it lesser. Hence proved.
\item Given $\frac{a + b}{1 + a + b}\leq \frac{a}{1 + a} + \frac{b}{1 + b} \Rightarrow \frac{a}{1 + a + b} -
  \frac{a}{1 + a} + \frac{b}{1 + a + b} - \frac{b}{1 + b}\leq 0$

  $\Rightarrow \frac{a + a^2 - a - a^2 - ab}{(1 + a + b)(1 + a)} + \frac{b + b^2 - b - b^2 - ab}{(1 + a +
    b)(1 + b)}\leq 0$, which is clearly true.
\item Multiplying both sides by $2(1 + a)(1 + b)(2 + a + b)$, we get

  $2(a + b)(1 + a)(1 + b)\geq [a(1 + b) + b(1 + a)](2 + a + b)\Rightarrow (a - b)^2\geq 0$, which proves the
  inequality.
\item Let $\displaystyle S_i = \sum_{j = 1}^ia_j$ and $b_i = \frac{S_i}{i}$, which makes
  L.H.S. $\displaystyle \sum_{i=1}^n\frac{b_i}{i}$.

  Clearly, $a_i = b_i - \frac{i - 1}{i}b_{i - 1}$.

  $\displaystyle\sum_{i = 1}^na_i = b_1 + \left(b_2 - \frac{b_1}{2}\right) +\cdots + \left(b_n - \frac{n-
    1}{n}b_{n - 1}\right)$

  $= b_n + \frac{b_{n - 1}}{n} + \frac{b_{n - 2}}{n - 1} + \cdots + \frac{b_1}{2} = b_n +
  \displaystyle\sum_{i = 1}^n\frac{b_i}{i + 1}$

  Now $2\displaystyle\sum_{i = 1}^na_i - \sum_{i=1}^n\frac{b_i}{i} = \left(2 - \frac{1}{n}\right) +
  \sum_{i=1}^n\left(\frac{2}{i + 1} - \frac{1}{i}\right)b_i$

  $\frac{2}{i + 1} - \frac{1}{i} = \frac{i - 1}{i(i + 1)}$ so the sum $2\displaystyle\sum_{i = 1}^na_i -
  \sum_{i=1}^n\frac{b_i}{i} \geq 0$

  Hence proved.
\item We observe that if $a = 2, b = 3$ and $c = 7$ then $\frac{1}{a} +
  \frac{1}{b} + \frac{1}{c} = \frac{41}{42}$. If $a = 2, b = 3, c > 7$, then $\frac{1}{a} + \frac{1}{b} +
  \frac{1}{c} < \frac{41}{42}$, if $a = 2, b > 3, c > 4$, then $\frac{1}{a} + \frac{1}{b} + \frac{1}{c} <
  \frac{41}{42}$, and if $a > 2, b > 3, c > 4$, then $\frac{1}{a} + \frac{1}{b} + \frac{1}{c} <
  \frac{41}{42}$.

  Thus, we see that maximum value of $\frac{1}{a} + \frac{1}{b} + \frac{1}{c}$ is $\frac{41}{42}$.
\item {\bf{\it Nesbitt's inequality:}} From A.M.-H.M. inequality $\frac{(x + y) + (y + z) + (z + x)}{3}\geq
  \frac{3}{\frac{1}{x + y} + \frac{1}{y + z} + \frac{1}{z + x}}$

  $\Rightarrow \left[(x + y) + (y + z) + (z + x)\right]\left(\frac{1}{x + y} + \frac{1}{y + z} + \frac{1}{z
  + x}\right)\geq 9$

  $\Rightarrow 2\frac{x + y + z}{x + y} + 2\frac{x + y + z}{y + z} + 2\frac{x + y + z}{z + x}\geq 9$

  $\Rightarrow \frac{x}{y + z} + \frac{y}{z + x} + \frac{z}{x + y}\geq \frac{3}{2}$.

  Now $\frac{4x}{y + z} + \frac{y}{x + z} + \frac{z}{x + y} > 2$ is to be proven.

  $\Rightarrow \frac{3x}{y + z}\geq \frac{1}{2}$

  If $x = y = z$ then it is greater than $\frac{1}{2}$.

  If we make $y >> x$ and $z >> x$ to so that it is less than $\frac{1}{2}$ then the other terms of the
  inequality becomes greater than $2$.

  If $x >> y = z$ then it is greater than $\frac{1}{2}$.

  Thus, inequality holds in all cases.
\item Since $a, b, c, d > 0$ we can write $a + b + c, a + b + d, a + c + d, b + c + d > a + b + c + d$.

  So $\frac{a}{a + b + d} > \frac{a}{a + b + c + d}, \frac{b}{a + b + c} > \frac{b}{a + b + c + d},
  \frac{c}{b + c + d} > \frac{c}{a + b + c + d}$, and $\frac{d}{a + c + d} > \frac{d}{a + b + c + d}$

  Adding we get $\frac{a}{a + b + d} + \frac{b}{a + b + c} + \frac{c}{b + c + d} + \frac{d}{a + c + d} > 1$.


  Similarly, $\frac{a}{a + b + d} < \frac{a}{a + b}$, and $\frac{b}{a + b + c} < \frac{b}{a + b}$. Adding
  $\frac{a}{a + b + d} + \frac{b}{a + b + c} < 1$.

  Also, $\frac{c}{b + c + d} < \frac{c}{c + d}$, and $\frac{d}{a + c + d} < \frac{d}{c + d}$. Adding
  $\frac{c}{b + c + d} + \frac{d}{c + d} < 1$.

  Adding all these we prove the required inequality.
\item If $a + b\leq c + d$ then $a\leq c$ or $b\leq d$. If $a\leq c$ then from first condition $b - d> (c -
  d)(c + a) > c - a \Rightarrow a + b > c + d$.

  Similarly for other condition same can be proved. And thus, the inequality is proven.
\item $(b - a)(9 - a^2) + (c - a)(9 - b^2) + (c - b)(9 - c^2) = 9b + c(9 - b^2) + (c - b)(9 - c^2) = 18c -
  c^3 + bc(c - b)\leq 18 c - c^3 + \frac{1}{4}c^3= 18c - \frac{3}{4}c^2$

  We can use calculus to apply maxima and minima to prove the inequality.
\item We assume that all of them are greater than $\frac{1}{4}$ so $abc(1 - a)(1 - b)(1 - c) >
  \frac{1}{16}$.

  From A.M.-G.M. inequality $\frac{a + 1 - a}{1}\geq \sqrt{a(1 - a)}\Rightarrow a(1 - a)\leq
  \frac{1}{4}$. Similarly, $b(1 - b)\leq \frac{1}{4}$ and $c(1 - c)\leq \frac{1}{4}$.

  Multiplying $abc(1 - a)(1 - b)(1 - c)\leq \frac{1}{16}$, which leads to a contradiction. Thus, at least
  one of the assumed numbers is less that $\frac{1}{4}$.
\item $\sqrt{a + \frac{1}{4}(b - c)^2}\leq a + \frac{b + c}{2}$ if $a, b, c > 0$ and $a + b + c = 1$.

  Adding for all the terms we get $2(a + b + c) = 2$.
\item Using Cauchy-Schwarz inequality $\left(\sqrt{a + \frac{1}{4}(b - c)^2} + \sqrt{b} +
  \sqrt{c}\right)^2\leq (1^2 + 1^2 + 1^2)\left(a + \frac{1}{4}(b - c)^2 + b + c\right) = 3\left(1 +
  \frac{1}{4}(b^2 + c^2 - 2bc)\right) = 4\left(1 + \frac{1}{4}[(1 - a)^2 - 4bc]\right)$

  From A.M.-G.M. inequality $bc\leq \frac{(b + c)^2}{4}\Rightarrow bc\leq \frac{(1 - a)^2}{4}$.

  Thus above inequality becomes $\left(\sqrt{a + \frac{1}{4}(b - c)^2} + \sqrt{b} +
  \sqrt{c}\right)^2\leq 3\left(1 + \frac{1}{4}[(1 - a)^2 - (1 - a)^2]\right) = 3$.

  Hence proved.
\item We know that $\frac{x}{y} + \frac{y}{x}\geq 2$. Given expression is $\frac{a^4}{b^4} + \frac{b^4}{a^4}
  - \frac{a^2}{b^2} - \frac{b^2}{a^2} + \frac{a}{b} + \frac{b}{a}$

  $= \left(\frac{a^2}{b^2} + \frac{b^2}{a^2}\right) - 2 - \left(\frac{a}{b} + \frac{b}{a}\right)^2 + 2 +
  \frac{a}{b}+ \frac{b}{a}$

  Substityting $\frac{a}{b} + \frac{b}{a} = y$ we have the expression as $y^4 - 5y^2 + y + 4$. Applying
  maxima-minima we find the minimum value as $2$.
\item Since $x_1 + x_2 + \cdots + x_n = 1$, it follows by problem $9$ that there are two numbers such that
  one of them is not greater than $\frac{1}{n}$, and the other one is not less than $\frac{1}{n}$. WLOG we
  can assume that $x_1\leq x_2$. Substituting $x_1$ by $\frac{1}{n}, x_2$ by $x_1 + x_2 - \frac{1}{n}$. Then
  we obtain numbers $\frac{1}{n}, x_1 + x_2 - \frac{1}{n}, x_3, \ldots, x_n$ such that

  $\frac{(1 - x_1)\cdots(1 - x_n)}{x_1\cdots x_n}\geq \frac{\left(1 - \frac{1}{n}\right)\left(1 - x_1 - x_2
    + \frac{1}{n}\right)\cdots(1 - x_n)}{\frac{1}{n}\left(x_1 + x_2 - \frac{1}{n}\right)\cdots
    x_n}$ from problem $16$. $\frac{(1 - x_1)(1 - x_2)}{x_1x_2} = 1 + \frac{1 - (x_1 +
    x_2)}{x_1x_2}$. Repeating this step, we obtain $n$ numbers less than $\frac{1}{n}$. For these numbers
  L.H.S. of the inequality is equal to $(n - 1)^n$ and is not greater than $\frac{(1 - x)1)\cdots(1 -
    x_n)}{x_1\cdots x_n}$.
\item Let $\sqrt[n]{x_1\ldots x_n} = y$. According to problem $11$, WLOG one can assume that $x_1\leq y\leq
  x_2$, and therefore, for numbers $y, \frac{x_1x_2}{y}, x_3, \ldots, x_n$

  $\frac{1}{x_1} + \cdots + \frac{1}{1 + x_n}\geq \frac{1}{1 + y} + \frac{1}{1 + \frac{x_1x_2}{y}} + \cdots
  + \frac{1}{1 + x_n}$. Since $\frac{1}{1 + x_1} + \frac{1}{1 + x_2}\geq \frac{1}{1 + y} + \frac{1}{1 +
    \frac{x_1x_2}{y}}$ and $\frac{1}{1 + x_1} + \frac{1}{1 + x_2} = 1 + \frac{1 - x_1x_2}{1 + x_1 + x_2 +
    x_1x_2}$.

  After a finite numbere of steps, we deduce that $\frac{1}{1 + x_1} + \frac{1}{1 + x_2} + \cdots +
  \frac{1}{1 + x_n}\geq \frac{1}{1 + y} + \cdots \frac{1}{1 + y} = \frac{n}{1 + \sqrt[n]{x_1\ldots x_n}}$.
\item If $a = b = c = d = \frac{1}{4}$, then we have equality. Let $a < \frac{1}{4} < b$, then

  If $c + d - \frac{176}{27}cd < 0$, then $A = ab\left(c + d - \frac{176}{27}cd\right) + cd(a + b)\leq cd(a
  + b)\leq \left(\frac{a + b + c + d}{3}\right)^3 = \frac{1}{27}$

  If $c + d - \frac{176}{27}cd \geq 0$, then $A\leq \frac{1}{4}\left(a + b - \frac{1}{4}\right)\left(c + d -
  \frac{176}{27}cd\right) + cd\left(\frac{1}{4} + a + b - \frac{1}{4}\right)$

  Hence, we must prove inequality for numbers $a_1 = \frac{1}{4}, b_1 = a + b - \frac{1}{4}, c_1 = d, d_1 =
  d$.

  Similarly, either one can prove the inequality for numbers $a_1, b_1, c_1, d_2$ or it will be sufficient
  to prove the inequality for the case among $a_1, b_1, c_1, d_1$ two numbers are equaal to
  $\frac{1}{4}$. Continuing in this way we obtain that it is sufficient to prove the inequality for numbers
  $\frac{1}{4}, \frac{1}{4}, \frac{1}{4}, \frac{1}{4}$. In this case the inequality holds.
\item Let $x\geq y\geq z$, then $y\leq \frac{1}{2}$. Therefore, $0\leq y(x + z) + xz(1 - 2y)\leq
  y\left(\frac{1}{3} + x + z - \frac{1}{3}\right) + \frac{1}{3}\left(x + z - \frac{1}{3}\right)(1 - 2y)$

  Therefore, if we substitute the numbers $x, y, z$ by the numbers $\frac{1}{3}, y, x + z - \frac{1}{3}$ in
  the expression $xy + yz + xz - 2xyz$, then its value does not decrease. Continuing similarly, we can
  substitute the numbers $\frac{1}{3}, y, x + z - \frac{1}{3}$ by $\frac{1}{3}, \frac{1}{3},
  \frac{1}{3}$. Hence, we deduce that $xy + yz + zx - 2xyz\leq \frac{1}{9} + \frac{1}{9} + \frac{1}{9} -
  \frac{2}{27} = \frac{7}{27}$.
\item Let us first prove the following lemma.

  {\bf Lemma} If $a\leq b$ and $x > 0$, then $(a - x)^{12} + (b + x)^{12} > a^{12} + b^{12}$

  $(a - x)^{12} + (b + x)^{12} - a^{12} - b^{12} = C_{1}^^{12}x(b^{11} - a^{11}) + C_2^^{12}x^2(b^{10} - a^{10})
  + \cdots + 2x^{12} > 0$

  Let $y_i = \sqrt{3}x_i, i = 1, 2, \ldots, 1997$. We have $-1\leq y_i\leq 3$ and $y_1 + \cdots + y_{1997} =
  -954$ and $x_1^{12} + \cdots + x_{1997}^{12} = \frac{y_1^{12} + \cdots + y_{1997}^{12}}{3^6}$

  If any two numbers among the numbers $y_1, \ldots, y_{1997}$ belong to $(-1, 3)$ then aaccording to lemma,
  we can substitute these two numbers such that one of them is equal to $-1$ or $3$, then first two
  conditions hold, and $y_1^{12} + \cdots + y_{1997}^{12}$ increases.

  Therefore, the sum $y_1^{12} + \cdots + y_{1997}^{12}$ is maximum if we substitute these by either $-1,
  \ldots, -1, 3,\ldots, 3$ or $-1, \ldots, -1, 3, \ldots, 3, a$ where $a\in(-1, 3)$.

  Taking into consideration the second condition, we obtain that only the second case is possible so that $k
  = \frac{a + 2}{4} + 1735$, where $k$ is the number of $-1$\symbol[rightquote]s. Since $\frac{a + 2}{4}\in\mathbb{Z}$
  and $a\in(-1, 3)$ we must have $a = 2$.

  Therefore, the greatest value of $x_1^{12} + \cdots + x_{1997}^{12}$ is $\frac{1736 + 260.3^{12} +
    2^{12}}{3^6} = 189548$.
\item Let $\frac{\alpha_1 + \cdots + \alpha_n}{n} = \phi$. If $\alpha_1 = \cdots = \alpha_n = \phi$ then

  $\cos\alpha_1\cos\alpha_2\cdots\cos\alpha_n(\tan\alpha_1 + \tan\alpha_2 + \cdots + \tan\alpha_n) =
  \cos^n\phi.n.\tan\phi = n\sin\phi\cos^{n - 1}\phi$

  $= n\sqrt{(n - 1)^{n - 1}\sin^2\phi\left(\frac{\cos^2\phi}{n - 1}\right)^{n - 1}}\leq n\sqrt{(n - 1)^{n -
      1}\left(\frac{\sin^2\phi + \frac{\cos^2\phi}{n - 1} + \cdots + \frac{\cos^2\phi}{n - 1}}{n}\right)^n} =
  n\sqrt{(n - 1)^{n - 1}\frac{1}{n^nn}} = \frac{(n - 1)^{\frac{n - 1}{2}}}{n^{\frac{n - 2}{2}}}$.

  Let there be two numbers $\alpha_1$ and $\alpha_2$ such that $\alpha_1 < \phi < \alpha_2$.

  $\cos\alpha_1\cos\alpha_2 = \frac{1}{2}[\cos(\alpha_1 + \alpha_2) + \cos(\alpha_1 - \alpha_2)] <
  \frac{1}{2}[\cos(\alpha_1 + \alpha_2) + \cos(2\phi - (\alpha_1 + \alpha_2))]$

  $=\cos\phi\cos(\alpha_1 + \alpha_2 - \phi)$

  We have $\cos\alpha_1\cos\alpha_2\cdots\cos\alpha_n(\tan\alpha_1 + \tan\alpha_2 + \cdots + \tan\alpha_n)$

  $= \sin(\alpha_1 + \alpha_2)\cos\alpha_3 \ldots \cos\alpha_n + \cos\alpha_1\cos\alpha_2\cos\alpha_3 \ldots
  \cos\alpha_n(\tan\alpha_3 + \cdots + \tan\alpha_n) < \sin(\alpha_1 + \alpha_2)\cos\alpha_3 \ldots
  \cos\alpha_n + \cos\phi\cos(\alpha_1 + \alpha_2 - \phi)\cos\alpha_3 \ldots \cos\alpha_n(\tan\alpha_3 +
  \cdots + \tan\alpha_n)$

  $=\cos\phi\cos(\alpha_1 + \alpha_2 - \phi)\cos\alpha_3\ldots\cos\alpha_n(\tan\phi + \tan(\alpha_1 +
  \alpha_2 - \phi)+ \tan\alpha_3 + \ldots + \tan\alpha_n)$

  Continuing similarly for the numbers $\phi, \alpha_1 + \alpha_2 - \phi, \alpha_3, \ldots, \alpha_n$ we
  obtain a new sequence, two of whose terms are equal to $\phi$. Repeating these steps $n - 1$ times, we
  obtain a sequence, $n - 1$ of whose terms are equal to $\phi$ and the $n$th term is equalt to $n\phi - (n
  - 1)\phi = \phi$. Hence, we have

  $\cos\alpha_1\cos\alpha_2\cdots\cos\alpha_n(\tan\alpha_1 + \tan\alpha_2 + \cdots + \tan\alpha_n) <
  \cos^n\phi.n.\tan\phi\leq\frac{(n - 1)^{\frac{n - 1}{2}}}{n^{\frac{n - 2}{2}}}$.

  Equallity holds if and only if $\alpha_1 = \cdots = \alpha_n = \phi$, where $\phi =
  \tan^{-1}\frac{1}{\sqrt{n - 1}}$.
\item Consider $x\geq 0, y\geq 0, x + y\leq \frac{2}{3}$ and $k\geq 2, k\in\mathbb{N}$, then

  $x^k(1 - x) + y^k(1 - y)\leq(x + y)^k(1 - x - y)$

  If $x + y = 0$, then above inequality holds, while if $x + y\neq 0$, then

  $\frac{x^k}{(1 + x)^k}(1 - x) + \frac{y^k}{(x + y)^k}(1 - y)\leq \left(\frac{x}{x + y}\right)^2(1 - x) +
  \left(\frac{y}{x + y}\right)^2(1 - y) = \frac{(x + y)^2(1 - x - y) + xy[3(x + y) - 2]}{(x + y)^2}\leq 1 -
  x - y$.

  Let $x_{i + 1}\geq x_i\geq 0, i = 1, \ldots, n - 1, x_1 + \cdots + x_n = 1$ and $n\geq 3$. Then $(n -
  2)x_1 + (n - 2)x_2\leq (x_3 + \cdots + x_n) + (x_3 + \cdots + x_n) = 2 - 2x_1 - 2x_2$, and therefore,

  $x_1 + x_2\leq \frac{2}{n}\leq \frac{2}{3}$.

  Therefore, if we substitute the numbers $x_1, \ldots, x_n$ by $0, x_1 + x_2, x_3, \ldots, x_n$ then their
  sum will be equal to $1$. Note that

  $\displaystyle\sum_{i=1}^kx_i^k(1 - x_i)\leq (x_1 + x_2)^k(1 - x_1 - x_2) + x_3^k(1 - x_3) + \cdots +
  x_n^k(1 - x_n)$.

  Repeating this step a finite number of times, we end ip with case $n = 2$, i.e.,
  $\displaystyle\sum_{i=1}^nx_i^k(1 - x_i)\leq x^k(1 - x) + (1 - x)^kx$, and therefore,
  $\displaystyle\sum_{i = 1}^nx_i^k(1 - x_i)\leq a_k$.

  Note that $a_1 = \displaystyle\max_{[0;1]}[2x(1 - x)] = \frac{1}{2}, a_2 = \max_{[0;1]}[x(1 - x)] =
  \frac{1}{2}, a_3 = \max_{[0;1]}[x(1 - x)(1 - 2x(1 - x))] = \frac{1}{8}$ and so on.
\item Let $x_1\leq x_2\leq \cdots\leq x_n, n\geq 3$. Now if $x_2\cdots x_{n - 1} = \frac{2(n - 1)}{n^{n -
    1}}$, then we have

  $2(n - 1)(x_2x_3 + x_1x_3 + \cdots + x_1x_n + x_2x_3 + \cdots + x_2x_n + \cdots + x_{n - 1}x_n) - n^{n -
  1}x_1x_2\ldots x_n= 2(n - 1)[(x_1 + x_n)(x_2 + \cdots + x_{n - 1}) + x_2x_3 + \cdots + x_2x_{n - 1}] +
  x_1x_n[2(n - 1) - n^{n - 1}x_2\ldots x_{n - 1}]\leq 2(n - 1)[x(1 - x) + x_2x_3 + \cdots + x_2x_{n - 1} +
    \cdots + x_{n - 2}x_{n - 1}]$, where $x = x_2 + \cdots + x_{n - 1}$.

  From R.M.S. inequality, it follows that $\frac{x_2^2 + \cdots + x_{n - 1}^2}{n - 2}\geq \left(\frac{x}{n -
  2}\right)^2$, and therefore, $x_2x_3 + \cdots + x_2x_{n - 1} + \cdots + x_{n - 2}x_{n - 1}\leq \frac{n -
    3}{2(n - 2)}x^2$, whence

  $A\leq 2(n - 1)\left[x(1 - x) + \frac{n - 3}{2(n - 2)x^2}\right] = 4(n - 2).\frac{n - 1}{2(n - 2)}x\left(1
  - \frac{n - 1}{2(n - 2)}x\right)\leq n - 2$.

  If $x_2\ldots x_{n - 1}\leq \frac{2(n - 1)}{n^{n - 1}}$, then for $x_1 = x_2 = \cdots = x_n =
  \frac{1}{n}$, we have $A = n - 2$.

  Otherwise if $x_i \neq \frac{1}{n}$ for some value of $i$, then $x_1 < \frac{1}{n} < x_n$.

  Substituting $x_1$ by $\frac{1}{n}$ and $x_n$ by $x_1 + x_n - \frac{1}{n}$, we see that the value of the
  given expression increases.

  Continuing similalry, either we can end the proof of inequality or it will be sufficient to prove the
  inequality for $x_1 = \cdots = x_n = \frac{1}{n}$.
\item If $y_1, y_2, \ldots, y_n\geq 0$ and $y_1 + y_2 + \cdots + y_n > 0$ then according to problem $67$,
  for $x_i = \frac{y_i}{y_1 + y_2 + \cdots + y_n}, i = 1, 2, \ldots, n$ we have

  $2(n - 1)q_np_n^{n - 2} - n^{n - 1}y_1y_2\ldots y_n\leq (n - 2)p_n^n$, where $p_n = y_1 + y_2 + \cdots +
  y_n, q_n = y_1y_2 + y_1y_3 + \cdots + y_1y_n + \cdots + y_{n - 1}y_n$. Therefore, $y_1y_2\ldots y_n = 1$
  we have

  $$q_n\leq \frac{(n - 1)p_n^n + n^{n - 1}}{2(n - 1)p_n^{n - 2}}$$

  If $x_1 = 0$, then we have following inequality

  $$\frac{x_1 + \cdots + x_n}{n}\leq \frac{(\sqrt{x_2} - \sqrt{x_3})^2 + \cdots + (\sqrt{x_2} -
    \sqrt{x_n})^2 + \cdots + (\sqrt{x_{n - 1}} - \sqrt{x_n})^2 + x_2 + \cdots + x_n}{n}$$

  or $2\sqrt{x_2x_3} + 2\sqrt{x_2x_4} + \cdots + 2\sqrt{x_2x_n} + \cdots + 2\sqrt{x_{n - 2}x_n}\leq (n -
  2)(x_2 + \cdots + x_n)$.

  The last ineqeuality can be proved using $2\sqrt{ab}\leq a + b(a, b)\geq 0$.

  If $x_i > 0$, let $y_i = \frac{\sqrt{x_i}}{\sqrt[2n]{x_1x_2\ldots x_n}}, i = 1, 2, \ldots, n$, then
  $y_1y_2\ldots y_n = 1$, and we need to prove that $\frac{p_n^2 - 2q_n}{n} - 1\leq \frac{(n - 1)(p_n^2 -
    2q_n) - 2q_n}{n}$, or $q_n\leq \frac{(n - 2)p_n^2 + n}{2(n - 1)}$.

  It follows that $q_n\leq \frac{(n - 2)P_n^n + n^{n - 1}}{2(n - 1)p_n^{n - 1}}\leq \frac{(n - 2)p_n^2 +
    n}{2(n - 1)}$.
\item For numbers $y_1 = x_1^2, \ldots, y_n = x_n^2$, using the inequality of problem $68$, we deduce that

  $\frac{x_1^2 + \cdots + x_n^2}{n} - \sqrt[n]{x_1^2\cdots x_n^2}\leq \frac{(|x_1| - |x_2|)^2 + \cdots +
  (|x_1| - |x_n|)^2 + \cdots + (|x_{n - 1}| - |x_n|)^2}{n}$

  or $(n - 1)(x_1^2 + \cdots + x_n^2) + n\sqrt[n]{x_1^2\ldots x_n^2}\geq (|x_1| + \cdots + |x_n|)^2\geq (x_1
  + \cdots + x_n)^2$.
\item WLOG we can assume that $c\geq a, c\geq b$. From A.M.-G.M. inequality $\frac{a + b}{2}\geq \sqrt{ab}$
  and we know that $\frac{a}{b} + \frac{b}{a}\geq 2$.

  Note that $\frac{b}{c} + \frac{c}{a} - \frac{b}{a} - 1 = \frac{(c - a)(c - b)}{ac}\geq 0$, whence
  $\frac{b}{c} + \frac{c}{a} - \frac{b}{a} -1\geq 0$.

  Adding the inequalities we obtain $\frac{a}{b} + \frac{b}{c} + \frac{c}{a}\geq 3$.
\item WLOG we can assume that $a\geq b\geq c$. We have

  $\sqrt{\frac{a}{b + c}} + \sqrt{\frac{b}{c + a}} + \sqrt{\frac{c}{a + b}} + \sqrt{\frac{b + c}{a}} -
  \sqrt{\frac{b + c}{a}}\geq 2 + \sqrt{\frac{b}{c + a}} + \sqrt{\frac{c}{a + b}} - \sqrt{\frac{b +
      c}{a}}\geq 2 + \sqrt{\frac{b}{c + a}} + \sqrt{\frac{c}{2a}} - \sqrt{\frac{b + c}{a}} = 2 +
  \frac{1}{\sqrt{a}}\left(\sqrt{\frac{b}{1 + \frac{c}{a}}} + \sqrt{\frac{c}{2}} - \sqrt{b + c}\right)\geq 2
  + \frac{1}{\sqrt{a}}\left(\sqrt{\frac{b}{1 + \frac{c}{a}}} + \sqrt{\frac{c}{2}} - \sqrt{b + c}\right) = 2
  + \frac{\sqrt{c}}{\sqrt{2a(b + c)}}\left(\sqrt{b + c} - \sqrt{2c}\right)\geq 2$.
\item For positive numbers $a, b, c$ we have from A.M.-G.M. inequality $\frac{a + b}{2}\geq \sqrt{ab}$ and
  so on. Multiplying these we get the required inequality.
\item Note that three factors on the L.H.S. of the given inequality are positive. If only one factor on the
  L.H.S. is not positive, then the proof is obvious. Consider the case when all four factors are
  positive. In this case, from A.M.-G.M. inequality for the numbers $a + b + c - d$ and $b + c + d -a, a + b
  + c - d$ and $d + a + b - c, b + c + d - a$ and $c + d + a - b, c + d + a - b$ and $d + a + b - c$, we
  have the following inequalities:

  $\sqrt{(a + b + c - d)(b + c + d - a)}\leq\frac{(a + b + c - d) + (b + c + d - a)}{2} = b + c$,

  $\sqrt{(a + b + c - d)(d + a + b - c)}\leq a + b, \sqrt{(b + c + d - a)(c + d + a - b)}\leq c + d,$

  $\sqrt{(c + d + a - b)(d + a + b - c)}\leq a + d$.

  Multiplying these inequalities we obtain the given inequality.
\item The given inequality is equivalent to $abc \geq (a + b - c)(c + a - b)(b + c - a)$. See problem $76$.
\item The given inequality is equivalent to $7abc + a^3 + b^3 + c^3 > a^2b + ab^2 + c^2a + ca^2 + b^2c +
  bc^2$ and the proof follows from previous problem.
\item The given inequality is equivalent to the following inequality:

  $\log(a - 1)\log(a + 1) < \log^2a$. If $\log(a - 1)\leq 0$, then $\log(a - 1)\log(a + 1)\leq 0\leq
  \log^2a$.

  Otherwise, if $\log(a + 1) > 0$, then using A.M.-G.M. inequality, we have $2\sqrt{\log(a - 1)\log(a +
    1)}\leq \log(a - 1)\log(a + 1) = \log(a^2 - 1) < \log a^2$, which is equilavent to $\log(a - 1)\log(a +
  1) < \log a^2$.
\item Since $a > 0, b > 0, c > 0$, at least two factors on the R.H.S. of the inequality are positive. If
  only one factor is negative then the proof is obvious. Consider the case in which all three factors on
  R.H.S. are positive. Using A.M.-G.M. inequality for nonnegative numbers, we have

  $\sqrt{(a + b - c)(a + c - b)}\leq \frac{a + b - c + a + c - b}{2} = a, \sqrt{(a + b - c)(b + c - a)}\leq
  b, \sqrt{(a + c - b)(b + c - a)}\leq c$.

  Multiplying these inequalities we get required inequality.
\item We have $\frac{x^8 + y^8}{2}\geq \left(\frac{x^4 + y^4}{2}\right)^2\geq \left(\frac{x^2 +
  y^2}{2}\right)^4\geq \left(\frac{x + y}{2}\right)^8 = \frac{1}{128}$.
\item Since $a + b = 1$, from A.M.-G.M. inequality, we have $\frac{1}{ab}\geq 4$. Using inequality
  $\frac{a_1 + \cdots + a_n}{n}\leq \sqrt{\frac{a_1^2 + \cdots + a_n^2}{n}}$, we have

  $\frac{\left(a + \frac{1}{a}\right)^2 + \left(b + \frac{1}{b}\right)^2}{2}\geq \left(\frac{a + \frac{1}{a}
  + b + \frac{1}{b}}{2}\right)^2 = \left(\frac{1 + \frac{1}{ab}}{2}\right)^2\geq \left(\frac{1 +
  4}{2}\right)^2 = \frac{25}{4}$.
\item We know that $\frac{a_1 + \cdots + a_n}{n}\geq \sqrt[n]{a_1\ldots a_n}\geq \frac{n}{\frac{1}{a_1} +
  \cdots + \frac{1}{a_n}}$, whence

  $$(a_1 + \cdots + a_n)\left(\frac{1}{a_1} + \cdots + \frac{1}{a_n}\right)\geq n^2$$

  Using above inequality(since $x_1 + \cdots + x_n = 1$), we deduce that $\frac{1}{x_1} + \cdots +
  \frac{1}{x_n} \geq n^2$. Using $\frac{a_1 + \cdots + a_n}{n}\leq \sqrt{\frac{a_1^2 + \cdots + a_n^2}{n}}$,
  we have

  $\left(x_1 + \frac{1}{x_1}\right)^2 + \cdots + \left(x_n + \frac{1}{x_n}\right)^2\geq n\left(\frac{x_1 +
    \frac{1}{x_1} + \cdots + x_n + \frac{1}{x_n}}{n}\right)^2 = n\left(\frac{1 + x_1 + \cdots +
    x_n}{n}\right)^2\geq n\left(\frac{1 + n^2}{n}\right)^2 = \frac{(1 + n^2)^2}{n}$.
\item From A.M.-G.M. inequality, we have $a^4 + b^4\geq 2a^2b^2, b^4 + c^4\geq 2b^2c^2, c^4 + a^4\geq
  2c^2a^2$.

  Adding these, we have $a^4 + b^4 + c^4 \geq a^2b^2 + b^2c^2 + c^2a^2$.

  Also, from A.M.-G.M. inequality, we have $a^2b^2 + b^2c^2\geq 2ab^2c, b^2c^2 + c^2a^2\geq 2abc^2, a^2b^2 +
  c^2a^2\geq 2a^2bc$.

  Adding these, we have $a^2b^2 + b^2c^2 + c^2a^2\geq abc(a + b + c)$.

  From these two inequalities $a^4 + b^4 + c^4\geq abc(a + b + c)$.
\item Using that $xy = 1$, we have $x^2 + y^2 = (x - y)^2 + 2xy = (x - y)^2 + 2\geq 2\sqrt{2}(x - y)$.
\item From A.M.-G.M. inequality, for every value of $\lambda$, we have following inequalities

  $$(6a_i + 1)\lambda^2\geq 2\lambda\sqrt{6a_i + 1}, i = 1, 2, \ldots, 5$$

  Adding these inequalities and using the fact that $a_1 + \cdots + a_5 = 1$, for $\lambda > 0$, we have

  $\frac{11 + 5\lambda^2}{2\lambda}\geq \displaystyle\sum_{i = 1}^5\sqrt{6a_i + 1}$.

  Taking $\lambda = \sqrt{\frac{11}{5}}$, we have the desired inequality.
\item From A.M.-G.M. inequality, we have $5\sqrt{ab} + 3\sqrt{bc} + 7\sqrt{ca}\leq \frac{5(a + b)}{2} +
  \frac{3(b + c)}{2} + \frac{7(c + a)}{2} = 6a + 4b + 5c$.
\item Since $a^4 + b^4\geq 2a^2b^2$, we have $2(a^4 + b^4) + 17\geq 4a^2b^2 + 17 > 4(a^2 + b^2 + 4)\geq
  16ab$. Therefore, $2(a^4 + b^4) + 17 > 16ab$.
\item We have $1 + \underbrace{b + \cdots + b}_{n}\geq (n + 1)\sqrt[n + 1]{b^n}$, and thus, $\left(\frac{1 +
  nb}{n + 1}\right)^{n + 1}\geq b^n$.
\item Using the A.M.-G.M. inequality for the numbers $\underbrace{(1 + 1/n), \ldots, (1 + 1/n)}_{n}, 1$, we
  have $\frac{\left(1 + \frac{1}{n}\right) + \cdots + \left(1 + \frac{1}{n}\right) + 1}{n + 1} > \sqrt[n +
    1]{\left(1 + \frac{1}{n}\right)^n}$ or $\left(1 + \frac{1}{n + 1}\right)^{n + 1} > \left(1 +
  \frac{1}{n}\right)^n$.
\item This problem can be solved like previous problem.
\item Using the A.M.-G.M. inequality for numbers $\underbrace{1 + \frac{m}{n - 1}, \cdots, 1 + \frac{m}{n -
    1}, 1}_{n - 1}$, we have $\left(1 + \frac{m}{n - 1}\right)^{\frac{n - 1}{m}} < \left(1 +
  \frac{m}{n}\right)^{\frac{n}{m}}$.

  Using the A.M.-G.M. inequality for numbers $\underbrace{1 + \frac{m}{n} \cdots, 1 + \frac{m}{n}, 1}_{m -
    1}$, we have $\left(1 + \frac{m}{n}\right)^{\frac{n}{m}} < \left(1 + \frac{m - 1}{n}\right)^{\frac{n}{m
      - 1}}$.
\item Using the A.M.-G.M. inequality, we have $\sqrt[n]{n!} = \sqrt[n]{1\cdots n} < \frac{1 + 2 + \cdots +
  n}{n} = \frac{n + 1}{2}$.
\item $S_n + n = n + 1 + \frac{1}{2} + \cdots + \frac{1}{n} = 2 + \frac{3}{2} + \cdots + \frac{n + 1}{n} >
  n\sqrt[n]{2.\frac{3}{2}.\cdots \frac{n + 1}{n}} = n\sqrt[n]{n + 1}$.
\item $n - S_n = \frac{1}{2} + \cdots + \frac{n - 1}{n} > (n - 1)\sqrt[n -
  1]{\frac{1}{2}.\frac{2}{3}.\cdots.\frac{n - 1}{n}} = (n - 1)n^{\frac{1}{1 - n}}$.
\item We know that $q^n - 1 = (q - 1)(q^{n - 1} + q^{n - 2} + \cdots + 1)$ and $q > 1$. So the given
  inequality is reduced to $(q^{n - 1} + q^{n - 2} + \cdots + 1)(q^{n + 1} + 1) \geq 2nq^n$.

  Using the A.M.-G.M. inequality for $n > 1$, we have $q^{n - 1} + q^{n - 2} + \cdots + 1 \geq n.\sqrt[n]{q^{n -
  1}q^{n - 1}\ldots 1}$ and $q^{n + 1} + 1\geq 2q^{\frac{n + 1}{2}}$. Multiplying these we get desired
  inequality.
\item Using the A.M.-G.M. inequality $a^2 + b^2 + c^2 + d^2 + ab + ac + ad + bc + bd + cd\geq
  10.\sqrt[10]{(abcd)^5} = 10$.
\item Since $a, b, c > 0$, at least two factors on the L.H.S. are positive(see Problem $7$). If only one
  factor on L.H.S. is non-positive, then the proof is obvious. Consider the case in which all three factors
  on L.H.S. are positive. In this case, we note that

  $3 + 3abc = b\left(a - 1 + \frac{1}{b}\right) + c\left(b - 1 + \frac{1}{c}\right) + a\left(c - 1 +
  \frac{1}{a}\right) + bc\left(a - 1 + \frac{1}{b}\right) + ac\left(b - 1 + \frac{1}{c}\right) + ab\left(c -
  1 + \frac{1}{a}\right)$

  From the A.M.-G.M. inequality, we have

  $3 + 3abc\geq 6\sqrt[3]{a^3b^3c^3\left(a - 1 + \frac{1}{b}\right)^2\left(b - 1 +
    \frac{1}{c}\right)^2\left(c - 1 + \frac{1}{a}\right)^2}$, which is the given inequality.
\item Let $a + b + c = A$ and $\frac{1}{a} + \frac{1}{b} + \frac{1}{c} = B$, then we need to prove that

  $t^3 + (AB - A - B)t^2 + \left(\frac{a}{b} + \frac{b}{c} + \frac{c}{a} + A + B - 2AB\right)t + AB + 2 - A
  - B\geq 0$

  For $t = 1$, inequality holds, since for $t = 1$ it is equivalent to earlier problem for $abc = 1$.

  Therefore, for $t = 1$, from above it follows that $\frac{a}{b} + \frac{b}{c} + \frac{c}{a}\geq A + B -
  3$.

  Since $t > 0, A = a + b + c\geq 3.\sqrt[3]{abc} = 3, B = \frac{1}{a} + \frac{1}{b} + \frac{1}{c} \geq
  3.\sqrt[3]{\frac{1}{abc}} = 3$, and $AB + 2 - A - B = (A - 1)(B - 1) + 1 > 0$, it follows that

  $t^3 + (AB - A - B)t^2 + \left(\frac{a}{b} + \frac{b}{c} + \frac{c}{a} + A + B - 2AB\right)t + AB + 2 - A
  - B\geq t^3 + (AB - A - B)t^2 + (2A + 2B - 2AB - 3)t + AB + 2 - A - B = (t - 1)^2(t + AB + 2 - A - B)\geq
  0$.
\item $a_n + \underbrace{\sqrt[n - 1]{a_1\ldots a_{n - 1}} + \cdots + \sqrt[n - 1]{a_1\ldots a_{n - 1}}}_{n
  - 1}\geq n.\sqrt[n]{a_n\underbrace{\sqrt[n - 1]{a_1\ldots a_{n - 1}}\cdots \sqrt[n - 1]{a_1\ldots a_{n -
        1}}}_{n - 1}}$,

  and therefore, $n\sqrt[n]{a_1\ldots a_n} - (n - 1)\sqrt[n - 1]{a_1\ldots a_{n - 1}}\leq a_n$.
\item $\sqrt[n]{\frac{a_1}{a_1 + b_1 + \cdots + k_1}.\frac{a_2}{a_2 + b_2 + \cdots + k_2}\cdots
  \frac{a_n}{a_n + b_n + \cdots + k_n}} + $

  $\sqrt[n]{\frac{b_1}{a_1 + b_1 + \cdots + k_1}.\frac{b_2}{a_2 + b_2
    + \cdots + k_2}\cdots\frac{b_n}{a_n + b_n + \cdots + k_n}} + \cdots + \sqrt[n]{\frac{k_1}{a_1 + b_1 + \cdots
    + k_1}.\frac{k_2}{a_2 + b_2 + \cdots + k_2}\cdots \frac{k_n}{a_n + b_n + \cdots + k_n}}\leq$

  $\frac{1}{n}\left(\frac{a_1}{a_1 + b_1 + \cdots + k_1}.\frac{a_2}{a_2 + b_2 + \cdots + k_2}\cdots
  \frac{a_n}{a_n + b_n + \cdots + k_n}\right) +$

  $\frac{1}{n}\left(\frac{b_1}{a_1 + b_1 + \cdots +
    k_1}.\frac{b_2}{a_2 + b_2 + \cdots + k_2}\cdots\frac{b_n}{a_n + b_n + \cdots + k_n}\right) + \cdots +
  \frac{1}{n}\left(\frac{k_1}{a_1 + b_1 + \cdots + k_1}.\frac{k_2}{a_2 + b_2 + \cdots + k_2}\cdots
  \frac{k_n}{a_n + b_n + \cdots + k_n}\right) = 1$.
\item We observe that $n - k + ka^n = \underbrace{1 + \cdots + 1}_{n - k} + \underbrace{a^n + \cdots +
  a^n}_{k}\geq n.\sqrt[n]{1\cdots 1.\underbrace{a^n\cdots a^n}_{k}} = na^k$. Thus, the inequality is proved.
\item We observe that $\frac{x^k}{y^{k - 1}} + (k - 1)y\geq k.\sqrt[k]{\frac{x^k}{y^{k - 1}}y^{k - 1}} =
  kx$, where $k\in\mathbb{N}, k\geq 2$, whence

  $\frac{x^k}{y^{k - 1}}\geq kx - (k - 1)y$.

  Using above inequality, we have $\frac{x_1^2}{x_2} + \frac{x_2^3}{x_3^2} + \cdots + \frac{x_n^{n +
      1}}{x_1^n}\geq (2x_1 - x_2) + (3x_2 - 2x_3) + \cdots + [nx_{n - 1} - (n - 1)x_n] + [(n + 1)x_n - nx_1]
  = 2(x_1 + \cdots + x_n) - nx_1\geq x_1 + \cdots + x_n$, whence

  $\frac{x_1^2}{x_2} + \frac{x_2^3}{x_3^2} + \cdots + \frac{x_n^{n + 1}}{x_1^n}\geq x_1 + x_2 + \cdots +
  x_n$.
\item We observe that $\frac{a^{x_1 - x_2}}{x_1 + x_2} + \frac{a^{x_2 - x_3}}{x_2 + x_3} + \cdots +
  \frac{a^{x_n - x_1}}{x_n + x_1}\geq n.\sqrt[n]{\frac{a^{x_1 - x_2}}{x_1 + x_2} . \frac{a^{x_2 - x_3}}{x_2
      + x_3} . \cdots . \frac{a^{x_n - x_1}}{x_n + x_1}} = \frac{n}{\sqrt[n]{(x_1 + x_2)(x_2 + x_3)
      \cdots(x_n + x_1)}}\geq \frac{n}{\underbrace{(x_1 + x_2) + \cdots + (x_n + x_1)}_{n}} =
  \frac{n^2}{2\displaystyle\sum_{i = 1}^nx_i}$.
\item Using the A.M.-G.M. inequality, for the numbers $x_i + 1$ and $(p - 1)1$\symbol[rightquote]s, where $i
  = 1, 2, \ldots, n$, we deduce that, $\sqrt[p]{x_i + 1} = \displaystyle\sqrt[p]{(x_i +
    1)\underbrace{1\cdots 1}_{p - 1}}\leq \frac{x_i + 1 + (p - 1).1}{p} = 1 + \frac{x_i}{p}$, whence,
  $\sqrt[p]{x_1 + 1} + \cdots + \sqrt[p]{x_n + 1}\leq \left(1 + \frac{x_1}{p}\right) + \cdots + \left(1 +
  \frac{x_n}{p}\right) = n + \frac{x_1 + \cdots + x_n}{p} = n + 1$.
\item $x^k(1 - x^m) = [x^{km}(1 - x^m)^m]^{\frac{1}{m}} = \left(mx^k \cdots mx^k(k - kx^m)\cdots (k -
  kx^m)\right).\frac{1}{m^{\frac{m}{k}}.k}\leq \frac{1}{m^{\frac{m}{k}}.k}\left(\frac{mx^k + \cdots + mx^k +
    (k - kx^m) + \cdots + (k - kx^m)}{m + k}\right) = \frac{k^{\frac{k}{m}}.m}{(m + k)^{\frac{k}{m} + 1}}$.
\item According to previous problem we see that $x(1 - x^2)\leq \frac{2\sqrt{3}}{9}$, whence

  $\frac{x}{1 - x^2} + \frac{y}{1 - y^2} + \frac{z}{1 - z^2} = \frac{x^2}{x(1 - x^2)} + \frac{y^2}{y(1 -
  y^2)} + \frac{z^2}{z(1 - z^2)}\geq \frac{9}{2\sqrt{3}}(x^2 + y^2 + z^2) = \frac{3\sqrt{3}}{2}$.
\item We have $\frac{1}{1 - x} + \frac{1}{1 - y} + \frac{1}{1 - z} = \frac{1 + x}{1 - x^2} + \frac{1 + y}{1
  - y^2} + \frac{1 + z}{1 - z^2} = \frac{1}{1 - x^2} + \frac{1}{1 - y^2} + \frac{1}{1 - z^2} + \frac{x}{1 -
  x^2} + \frac{y}{1 - y^2} + \frac{z}{1 - z^2} \geq \frac{9}{3 - x^2 - y^2 - z^2} + \frac{x}{1 -
  x^2} + \frac{y}{1 - y^2} + \frac{z}{1 - z^2}\geq \frac{9}{2} + \frac{3\sqrt{3}}{2}$(from previous
  problem).
\item Using A.M.-G.M. inequality, we have $f(x) = (1 + x)^{-\frac{1}{n}} + (1 - x)^{-\frac{1}{n}}\geq
  2\sqrt{(1 + x)^{-\frac{1}{n}}(1 - x)^{-\frac{1}{n}}} = \frac{2}{\sqrt[2n]{1 - x^2}}\geq 2[\because
    x\in[0;1)]$.

    On the other hand $f(0) = 2$. Thus, the minimum value of the function $f(x)$ is $2$.
\item Let us represent the function $f(x)$ as $f(x) = (m + n).\frac{n.\frac{ax^m}{n} + m.\frac{b}{mx^n}}{m +
  n}$.

  Using the A.M.-G.M. inequality, we have $f(x)\geq (m + n)\sqrt[m +
    n]{\left(\frac{ax^m}{n}\right)^n\left(\frac{b}{mx^n}\right)^m} = (m + n)\sqrt[m +
    n]{\frac{a^nb^m}{n^nm^m}}$, where equality holds if $\frac{ax^m}{n} = \frac{b}{mx^n}$. Therefore, $x =
  \sqrt[m + n]{\frac{bn}{am}}$.
\item Let us represent the function $f(x)$ as $f(x) = \frac{(x - a)(x - a)\alpha(b - x)\beta(b +
  x)}{\alpha\beta}$, where $\alpha > 0, \beta > 0$.

  Using the A.M.-G.M. inequality, we have

  $4\sqrt[4]{(x - a)(x - a)\alpha(b - x)\beta(b + x)}\leq (x - a) + (x - a) + \alpha(b - x) + \beta(b + x) =
  (2 - \alpha + \beta)x + (\alpha + \beta)b - 2a$.

  Note that the R.H.S. does not depend on $x$ if $\alpha - \beta = 2$, and equality holds if $x - a =
  \alpha(b - x) = \beta(b + x)$. Thus, it follows that $\alpha = \frac{x - a}{b - x}, \beta = \frac{x - a}{b
    + x}$.

  Hence, from the equation $\alpha - \beta = 2$, we deduce that $2x^2 - ax - b^2 = 0$. This equation has
  only one positive root, $x_0 = \frac{a + \sqrt{a^2 + 8b^2}}{4}$. We can easily prove that $x_0\in[a,
    b]$. Therefore, $f(x)$ attains it maaximum value in $[a, b]$ at $x_0$.
\item Let us represent the product $xyz$ as $\frac{1}{2\sqrt{3}\pi}.2x.\sqrt{3}y.\pi z$, and from the
  A.M.-G.M. inequality, $xyz = \frac{1}{2\sqrt{3}\pi}.2x.\sqrt{3}y.\pi z\leq
  \frac{1}{2\sqrt{3}\pi}.\left(\frac{2x + \sqrt{3}y + \pi z}{3}\right)^3 = \frac{1}{54\sqrt{3}\pi}$, where
  equality holds if $2x = \sqrt{3}y = \pi z$.

  Thus, we have the maximum value as $\frac{1}{54\sqrt{3}\pi}$.
\item Note that $f(0) = 0$, and if $x < 0$, then $f(x) < 0$. On the other hand, if $x > 0$, then $f(x) >
  0$. Therefore, the function $f(x)$ attains it maximum value in $(0, \infty)$ and its minimum value in
  $(-\infty, 0)$.

  From the A.M.-G.M. inequality for the numbers $ax^2$ and $b$, we have $\frac{ax^2 + b}{2}\geq
  |x|\sqrt{ab}$, where equality holds if $ax^2 = b$.

  From the above inequality, it follows that $\frac{x}{ax^2 + b}\leq \frac{1}{2\sqrt{ab}}$. Therefore, the
  maximum value of the function $f(x)$ is equal to $\frac{1}{2\sqrt{ab}}$, and it attains this value at the
  point $x = \sqrt{\frac{b}{a}}$. Since $f(x)$ is an odd function, its minimum value is equal to
  $-\frac{1}{2\sqrt{ab}}$, which the function attains at the point $x = -\sqrt{\frac{b}{a}}$.
\item We have $y = \frac{5\sqrt{(x + 2)(x + 4)} + 12}{x + 3}$, and thus if $x\geq -2$, then $y =
  \frac{\sqrt{(25x + 50)(x + 4)} + 12}{x + 3}\leq \frac{\frac{25x + 50 + x + 4}{2} + 12}{x + 3} = 13$, and
  $y = 13$ if $25x + 50 = x + 4$, i.e. $x = -\frac{23}{12}$. If $x \leq -4$ then $y < 0$.

  Therefore, the maximum value of $y$ is $13$.
\item We have $y = \displaystyle\frac{\sqrt[3]{(x^2 + 1)(x^2 + 1)\left(\frac{2}{5}x^2 + \frac{6}{5}\right).\frac{5}{2}}}{3x^2 + 4} =
  \sqrt[3]{\frac{5}{2}}.\frac{\sqrt[3]{(x^2 + 1)(x^2 + 1)\left(\frac{2}{5}x^2 + \frac{6}{5}\right)}}{3x^2 + 4}\leq
    \sqrt[3]{\frac{5}{2}}.\frac{x^2 + 1 + x^2 + 1 + \frac{2}{5}x^2 + \frac{6}{5}}{3(3x^2 + 4)} =
    \sqrt[3]{\frac{5}{2}}.\frac{12x^2 + 16}{15(3x^2 + 4)} = \frac{4}{15}\sqrt[3]{\frac{5}{2}}$

    Note that $y = \frac{4}{15}\sqrt[3]{\frac{5}{2}}$, if $x^2 + 1= \frac{2}{5}x^2 + \frac{6}{5}$ i.e. if
    $x^2 = \frac{1}{3}$.
\item From A.M.-G.M. inequality it follows that $1\geq z^2 + 1$, and therefore, $z = 0, x = y = 1$.
\item We observe that the inequality $\frac{x^2 + y^2 + z^2}{3}\geq \left(\frac{x + y + z}{3}\right)^2$
  becomes an inequality because $x + y + z = x^2 + y^2 + z^2 = 3$. Hence, the solution is $x = y = z = 1$.
\item We have $\frac{a^2 + b^2 + c^2 + d^2}{4}\geq \left(\frac{a + b + c + d}{4}\right)^2 \Rightarrow
  \frac{16 - e^2}{4}\geq \left(\frac{8 - e}{4}\right)^2\Rightarrow 5e^2 - 16e\leq 0 \Rightarrow 0\leq e\leq
  3.2$.

  Thus, the maximum value of $e$ is $3.2$.
\item We have $\frac{x_1}{x_2} + \frac{x_3}{x_4} + \frac{x_5}{x_6}\geq \frac{1}{x_2} + \frac{x_2}{x_4} +
  \frac{x_4}{x_6}\geq 3.\sqrt[3]{\frac{1}{x_2}.\frac{x_2}{x_4}.\frac{x_4}{x_6}} = 3\sqrt[3]{\frac{1}{x_6}}
  \geq 0.3$.

  Thus, minimum value of the given expression is $0.3$.
\item Using the A.M.-G.M. inequality for the numbers $x^4, y^4, 1, 1$, we have $x^4 + y^4 + 1 + 1\geq
  4\sqrt[4]{x^4y4} = 4|x||y|\Rightarrow x^4 + y^4 + 2\geq 4xy$ and therefore, $x^4 = 1, y^4 = 1\Rightarrow
  x, y = \pm 1$.
\item We have $(xy)^2 + (yz)^2 + (zx)^2 = 3xyz$, whence $xyz > 0$. On the other hand, $xyz\in\mathbb{Z}$,
  and hence, $xyz \geq 1$. From the A.M.-G.M. inequality, we have $3xyz = (xy)^2 + (yz)^2 + (zx)^2\geq
  3xyz.\sqrt[3]{xyz}\geq 3xyz$, and hence, $xyz = 1$, and the inequality becomes equality. It follows that
  $(xy)^2 = (yz)^2 = (zx)^2$. Therefore $x^2 = y^2 = z^2 = 1$.

  $(x, y, z) = (1, 1, 1) = (1, -1, -1) = (-1, 1, -1) = (-1, -1, 1)$.
\item We observe that for $x > 0, \alpha > \beta \geq 0$, we have $x^{\alpha} - x^{\beta}\geq x^{\alpha -
  \beta} - 1$, since $(x^{\alpha - \beta} - 1)(x^{\beta} - 1)\geq 0$. Therefore,

  $x_1^\alpha + \cdots + x_n^\alpha - (x_1^\beta + \cdots + x_n^\beta) \geq (x_1^{\alpha - \beta} - 1) +
  \cdots + (x_n^{\alpha - \beta} - 1) = x_1^{\alpha - \beta} + \cdots + x_n^{\alpha - beta} - n\geq
  n.\sqrt[n]{x_1^{\alpha - \beta}\ldots x_n^{\alpha - \beta}} - n = 0$.

  Thus, $x_1^\alpha + x_2^\alpha + \cdots + x_n^\alpha\geq x_1^\beta + x_2^\beta + \cdots + x_n^\beta$.
\item If $\beta\geq 0$, then from previous probelm, we have $x_1^\alpha + x_2^\alpha + \cdots +
  x_n^\alpha\geq x_1^\beta + x_2^\beta + \cdots + x_n^\beta$.

  If $\beta < 0$, then $x_1^\beta + \cdots + x_n^\beta = \frac{1}{x_1^{-\beta}} + \cdots +
  \frac{1}{x_n^{-\beta}} = x_2^{-\beta} \ldots x_n^{-\beta} + \cdots + x_1^{-\beta}\ldots x_{n -
    1}^{-\beta}\leq \frac{x_2^{-\beta(n - 1)} + \cdots + x_n^{-\beta(n - 1)}}{n - 1} + \cdots +
  \frac{x_1^{-\beta(n - 1)} + \cdots + x_{n - 1}^{-\beta(n - 1)}}{n - 1} = x_1^{-\beta(n - 1)} + \cdots +
  x_n^{-\beta(n - 1)}\leq x_1^{\alpha} + \cdots + x_n^{\alpha}$.

  Therefore, $x_1^\alpha + x_2^\alpha + \cdots + x_n^\alpha\geq x_1^\beta + x_2^\beta + \cdots + x_n^\beta$.
\item WLOG we can assume that $\max(x, y, z) = x$, in which case, $x^2y + y^2z + z^2x\leq x^2y + xyz +
  0.5z^2x + 0.5zx^2 = 0.5x(x + z)(2y + z)$.

  Using the A.M.-G.M. inequality, we have $0.5x(x + z)(2y + z)\leq 0.5\left(\frac{x + (x + z) + (2y +
    z)}{3}\right)^3 = \frac{4}{27}$, and thus, the given inequality is proven.
\item We have $\frac{1 + a}{1 + ab} + \frac{1 + b}{1 + bc} + \frac{1 + c}{1 + cd} + \frac{1 + d}{1 + da} =
  \frac{cd + acd}{cd + abcd} + \frac{ad + adb}{ad + abcd} + \frac{1 + c}{1 + cd} + \frac{1 + d}{1 + da} = 1
  + \frac{c(1 + ad)}{1 + cd} + 1 + \frac{d(1 + ab)}{1 + da}\geq 2 + 2\sqrt{\frac{cd(1 + ab)}{1 + cd}} = 4$.
\item $\frac{1 + ab}{1 + a} + \frac{1 + bc}{1 + b} + \frac{1 + cd}{1 + c} + \frac{1 + da}{1 + d} = \frac{cd
  + abcd}{cd + acd} + \frac{1 + bc}{1 + b} + \frac{1 + cd}{1 + c} + \frac{bc + abcd}{bc + bcd}\geq (1 +
  cd)\frac{4}{cd + acd + 1 + c} + (1 + bc)\frac{4}{1 + b + bc + bcd} = (1 + cd)\frac{4b}{bcd + abcd + b +
    bc} + (1 + bc)\frac{4}{1 + b + bc + bcd} = \frac{4[b(1 + cd) + 1 + bc]}{1 + b + bc + bcd} = 4$.
\item We observe that $(c - b)(c - d) + (e - f)(e - d) + (e - f)(c - b) < 0$, and therefore, $(bd + df + fb)
  - (ac + ce + ea) < (c + e)(b + d + f - a - c - e)$, or $\alpha - \beta < \gamma(T - S)$, where $\alpha =
  bd + df + fb, \beta = ac + ce + ea, \gamma = c + e$.

  We have $S\alpha + T\beta = S(\alpha - \beta) + (S + T)\beta < S\gamma(T - S) + (S + T)(ce + a\gamma)\leq
  S\gamma(T - S) + (S + T)\left(\frac{\gamma^2}{4}\right + a\gamma) = \gamma\left(2ST - \frac{3}{4}(S +
  T)\gamma\right)$.

  Thus, $\sqrt{\frac{3}{4}(S + T)(S\alpha + T\beta)} < \sqrt{\frac{3}{4}(S + T)\gamma\left(2ST -
    \frac{3}{4}(S + T)\gamma\right)}\leq \frac{1}{2}\left(\frac{3}{4}(S + T)\gamma + \left(2ST -
  \frac{3}{4}(S + T)\gamma\right)\right) = ST$.

  Therefore, $\sqrt{3(S + T)[S(bd + df + fb) + T(ac + ce + ea)]} < 2ST$.
\item We observe that $\displaystyle\frac{a + \sqrt{ab} + \sqrt[3]{abc} + \sqrt[4]{abcd}}{\sqrt[4]{a.\frac{a +
      b}{2}.\frac{a + b + c}{3}.\frac{a + b + c + d}{4}}} = \sqrt[4]{1.\frac{2a}{a + b}.\frac{3a}{a + b +
    c}.\frac{4a}{a + b + c + d}} +$

  $\sqrt[4]{1.\frac{2a}{a + b}.\frac{3b}{a + b + c}.\frac{4b}{a + b + c + d}} +$

  $\sqrt[12]{1.1.1. \frac{2b}{a +
    b}.\frac{2b}{a + b}.\frac{2b}{a + b}.\frac{3a}{a + b + c}.\frac{3b}{a + b + c}.\frac{3c}{a + b +
    c}. \frac{4c}{a + b + c + d}. \frac{4c}{a + b + c + d}. \frac{4c}{a + b + c + d}} +$

  $\sqrt[4]{1.\frac{2b}{a + b} . \frac{3c}{a + b + c} . \frac{4d}{a + b + c + d}}\leq\frac{1}{4}\left(1 +
  \frac{2a}{a + b} + \frac{3a}{a + b + c} + \frac{4a}{a + b + c + d}\right) +$

  $\frac{1}{4}\left(1 + \frac{2a}{a + b} + \frac{3b}{a + b + c} + \frac{4b}{a + b + c + d}\right) +
  \frac{1}{12}\left(3 + \frac{6b}{a + b} + \frac{3a}{a + b + c} + \frac{3b}{a + b + c} + \frac{3c}{a + b +
    c} + \frac{12c}{a + b + c + d}\right) +$

  $\frac{1}{4}\left(1 + \frac{2b}{a + b} + \frac{3c}{a + b + c} + \frac{4d}{a + b + c + d}\right) = 4$.
\item WLOG we assume that $a, b, c, d\geq 0$. Let $x, y, z$ be arbitrary positive numbers. Then using
  A.M.-G.M. inequality, we observe that

  $a^{12} + (ab)^6 + (abc)^4 + (abcd)^3 = a^{12} + \frac{1}{x^6}(xab)^6 + \frac{1}{x^4y^8}(xya.yb.c)^4 +
  \frac{1}{x^3y^6z^9}(xyza.yzb.zc.d)^3\leq a^{12} + \frac{1}{2x^6}(x^{12}a^{12} + b^{12}) +
  \frac{1}{3x^4y^8}(x^{12}y^{12}a^{12} + y^{12}b^{12} + c^{12}) +
  \frac{1}{4x^3y^6z^9}(x^{12}y^{12}z^{12}a^{12} + y^{12}z^{12}b^{12} + z^{12}c^{12} + d^{12}) = A(a^{12} +
  b^{12} + c^{12} + d^{12})$.

  We take numbers $x, y, z$ such that $1 + \frac{x^6}{2} + \frac{x^8y^4}{3} + \frac{x^9y^6z^3}{4} =
  \frac{1}{2x^6} + \frac{y^4}{3x^4} + \frac{y^6z^3}{4x^3} = \frac{1}{4x^3y^4z^9} = A$.

  Therefore, $x^{12} = 1 - \frac{1}{A}, y^{12} = 1 - \frac{1}{2\sqrt{A(A - 1)}}, z^{12} = 1 -
  \frac{1}{3\sqrt[3]{A[\sqrt{A(A - 1)} - 0.5]^2}}$, and $\frac{256}{27}A\left(3\sqrt[3]{A[A(A - 1) - 0.5]^2}
  - 1\right)^3 = 1$.

  Let $f(A) = \frac{256}{27}A\left(3\sqrt[3]{A[A(A - 1) - 0.5]^2} - 1\right)^3 - 1$

  By applying limits on this function between the limits $1.42$ and $1.43$ we find a value such that $f(A) =
  0$ and thus the equality is proven.
\item Observing the first two terms $C_0^^n.1^n + C_1^^n.1^{n - 1}.\frac{1}{n} = 2$. Thus, first two terms
  only are enough to set the bound to $2$. Rest of the terms make it definitely greater than $2$.
\item We see that L.H.S.\ will have terms $E_1 = 1, E_2 = \sum a_i, E_3 = \sum a_ia_j$ and so on. All these
  terms are square-free i.e.\ $a_i$ does not repeat and power is always one.

  Using multinomial theorem $S_k^2 = (a_1 + a_2 + \cdots + a_k)^2 = \sum \frac{k!}{i_1!i_2!\ldots
    i_n!}a_1^{i_1}\ldots a_k^{i_n}$

  Thus, $\frac{S_k^2}{k!}$ will have all terms on L.H.S.\ and also more terms which won't be square-free
  making R.H.S.\ greater than L.H.S.
\item From A.M.-G.M. inequality we have $1 + a\geq 2\sqrt{a}, 1 + b\geq 2\sqrt{b}, 1 + c\geq
  2\sqrt{c}\Rightarrow (1 + a)(1 + b)(1 + c)\geq 8\sqrt{abc}$.

  Also $\left(1 + \frac{1}{a}\right)\left(1 + \frac{1}{b}\right)\left(1 + \frac{1}{c}\right) = \frac{1 +
    a}{a}.\frac{1 + b}{b}.\frac{1 + c}{c}\geq \frac{8\sqrt{abc}}{abc} = \frac{8}{\sqrt{abc}}$.

  Also $\frac{a + b + c}{3}\geq \sqrt{abc}\Rightarrow \frac{1}{3}\geq \sqrt[3]{abc}\Rightarrow
  \frac{1}{27}\geq abc$.

  Now $\left(1 + \frac{1}{a}\right)\left(1 + \frac{1}{b}\right)\left(1 + \frac{1}{c}\right) = 1 + \frac{ab +
    bc + ca + 2}{abc}$, and $\frac{ab + bc + ca}{3}\geq \sqrt[3]{(abc)^2}$

  $\Rightarrow \left(1 + \frac{1}{a}\right)\left(1 + \frac{1}{b}\right)\left(1 + \frac{1}{c}\right) \geq
  \Rightarrow 1 + \frac{3.\frac{1}{9} + 2}{\frac{1}{27}} = 64$.
\item $\frac{a^n - 1}{a - 1} = 1 + a + a^2 + \cdots + a^{n - 1}$.

  Thus, $\frac{a^n - 1}{a^n(a - 1)} = \displaystyle\sum_{k = 1}^na^{-k}$.

  Applying A.M.-G.M. inequality for $\displaystyle\sum_{k = 1}^na^{-k}$ and $a^{\frac{n(n + 1)}{2}}$, we have

  $\frac{\displaystyle\sum_{k = 1}^na^{-k} + a^{\frac{n(n + 1)}{2}}}{n + 1}\geq
  (a^{-1}.a^{-2}.\ldots.a^{-n}.a^{\frac{n(n + 1)}{2}}) = 1$

  $\Rightarrow \displaystyle\sum_{k = 1}^na^{-k} + a^{\frac{n(n + 1)}{2}}\geq n + 1$, and hence, the
  inequality is proven.
\item Using A.M.-G.M. inequality, we have

  $\displaystyle\frac{a^{n + 1} + a^{n + 1} + \cdots + a^{n + 1} + 1}{n + 1}\geq \sqrt[n + 1]{a^{(n +
    1)n}.1} = a^n$

  $\Rightarrow \displaystyle na^{n + 1} + 1\geq (n + 1)a^n$.
\item We observe that $\displaystyle\sqrt{i} + \sqrt{i + 1} = \frac{1}{\sqrt{i + 1} - \sqrt{i}}$

  L.H.S.\ $= \displaystyle\prod_{i = 1}^n\frac{1}{\sqrt{i + 1} - \sqrt{i}}$

  Using A.M.-G.M. inequality, we have

  $\displaystyle\frac{(\sqrt{k + 1} - \sqrt{k}) + (\sqrt{k + 2} - \sqrt{k + 1}) + \cdots + (\sqrt{n + 1} -
    \sqrt{n})}{n - k + 1}\leq \sqrt[n - k + 1]{\sqrt{\prod_{i = k}^n(\sqrt{i + 1} - \sqrt{i})}}$

  $\Rightarrow \displaystyle\left(\frac{\sqrt{n + 1} - \sqrt{k}}{n - k + 1}\right)^{n - k + 1} \leq \prod_{i
    = k}^n(\sqrt{i + 1} + \sqrt{i})$

  We also have, $\displaystyle\frac{n - k + 1}{\sqrt{n + 1} - \sqrt{k}} = \sqrt{n + 1} + \sqrt{k}$

  $\Rightarrow (\sqrt{n + 1} + \sqrt{k})^{n - k + 1}\leq \prod_{i = k}^n(\sqrt{i + 1} + \sqrt{i})$

  Since $n - k + 1\geq 2$,w e have

  $\displaystyle(\sqrt{n + 1} + \sqrt{k})^2\leq \prod_{i = k}^n(\sqrt{i + 1} + \sqrt{i})$

  Now, $(\sqrt{n + 1} + \sqrt{k})^2 = n + 1 + k + 2\sqrt{k(n + 1)}$ and R.H.S.\ $n - k - \sqrt{n} + \sqrt{k}
  + 2$

  We need to prove that $n + 1 + k + 2\sqrt{k(n + 1)}\geq n - k -\sqrt{n} + \sqrt{k} + 2$

  $\Rightarrow 2k - 1 + \sqrt{n} + 2\sqrt{k(n + 1)}\geq \sqrt{k}$, which is true for given
  conditions. Hence, the inequality is proven.
\item Using A.M.-G.M. inequality we have

  $\displaystyle\frac{\frac{a_1}{a_2} + \frac{a_2}{a_3} + \cdots + \frac{a_{n - 1}}{a_n} +
  \frac{a_n}{a_1}}{n}\geq \sqrt[n]{\frac{a_1}{a_2}.\frac{a_2}{a_3}.\cdots.\frac{a_{n - 1}}{a_n}.
  \frac{a_n}{a_1}} = 1$

  $\Rightarrow \frac{a_1}{a_2} + \frac{a_2}{a_3} + \cdots + \frac{a_{n - 1}}{a_n} + \frac{a_n}{a_1}\geq n$.
\item We see that $a_1 + (a_2 - a_1) + \cdots + (a_{n + 1} - a_n) = a_{n + 1}$. Using A.M.-G.M. for these
  terms and $\frac{1}{a_1(a_2 - a_1)\cdots(a_{n + 1} - a_n)}$

  $\displaystyle\frac{a_1 + (a_2 - a_1) + \cdots + (a_{n + 1} - a_n) + \frac{1}{a_1(a_2 - a_1)\cdots
      (a_{n + 1} - a_n)}}{n + 2}\geq$

  $\displaystyle\sqrt[n + 2]{a_1(a_2 - a_1)\cdots(a_{n + 1} - a_n).\frac{1}{a_1(a_2 -
      a_1)\cdots(a_{n + 1} - a_n)}} = 1$

  $\Rightarrow \displaystyle a_{n + 1} + \frac{1}{a_1(a_2 - a_1)\cdots (a_{n + 1} - a_n)}\geq n + 2$.
\item We have to prove that $1 + \frac{x}{2}\leq \frac{1}{\sqrt{1 - x}}$. Both sides are positive for given
  conditions.

  $\Rightarrow \left(1 + \frac{x}{2}\right)\sqrt{1 - x}\leq 1$

  Squaring both sides $\left(1 + \frac{x}{2}\right)^2(1 - x)\leq 1$.

  Applying A.M.-G.M. inequality we have

  $\displaystyle\frac{1 + \frac{x}{2} + 1 + \frac{x}{2}+ 1 - x}{3}\geq \sqrt[3]{\left(1 -
    \frac{x}{2}\right)^2(1 - x)}$

  $\Rightarrow 1\geq \sqrt[3]{\left(1 - \frac{x}{2}\right)^2(1 - x)}$

  Cubing boht sides we have $\left(1 + \frac{x}{2}\right)^2(1 - x)\leq 1$, and hence, the inequlity is
  proven.
\item We assume that all of $a, b, c, d, e$ have the same sign. Let $x_1 = \frac{a}{b}, x_2 = \frac{b}{c},
  x_3 = \frac{c}{d}, x_4 = \frac{d}{e}, x_5 = \frac{e}{a}$.

  So the inequality becomes $\displaystyle\sum_{i=1}^5x_i^4\geq \sum_{i=1}^5x_i$.

  We use the A.M.-G.M. inequality for $\frac{x^4 + 1 + 1 + 1}{4}\geq \sqrt[4]{x^4} = x\Rightarrow x^4 + 3\geq 4x
  \Rightarrow x^4 - x\geq 3(x - 1)$. Thus,

  $\displaystyle\sum_{i = 1}^5(x_i^4 - x_i)\geq 3\sum_{i = 1}^5(x_i) - 15$

  Again from A.M.-G.M. inequality $\displaystyle\sum_{i = 1}^5x_i\geq 5\sqrt[5]{\prod_{i=1}^5x_i} =
  5$. Thus, $\displaystyle\sum_{i = 1}^5x_i - 5 = 0$. And therefore,

  $\displaystyle\sum_{i = 1}^5(x_i^4 - x_i)\geq 0$. And hence, the inequality is proven.
\item We will prove that for any $x > 0$ and a positive integer $n\geq 1$, the inequality $x^n\geq nx - (n -
  1)$ holds.

  Consider $n$ positive integers $x^n, 1, 1, \ldots, 1(n - 1)$ times. By A.M.-G.M. inequality

  $\frac{x^n + \overbrace{{1 + 1 + \cdots + 1}}^{n - 1\text{ times}}}{n}\geq \sqrt[n]{x^n}$

  $\Rightarrow x^n + (n - 1)\geq nx \Rightarrow x^n \geq nx - (n - 1)$

  Setting $n = 1999$ and for $a, b, c, d > 0$

  $\left(\frac{a}{b}\right)^{1999} + \left(\frac{b}{c}\right)^{1999} + \left(\frac{c}{d}\right)^{1999} +
  \left(\frac{d}{a}\right)^{1999}\geq 1999\left(\frac{a}{b} + \frac{b}{c} + \frac{c}{d} + \frac{d}{a}\right)
  - 4.1988$

  Let $\frac{a}{b} + \frac{b}{c} + \frac{c}{d} + \frac{d}{a} = S$ so we want to prove that L.H.S $\geq S$.

  If we can show that $1999S - 7992\geq S\Rightarrow S\geq 4$ then the desired inequality is proven.

  Using A.M.-G.M. ineqeuality again $\frac{\frac{a}{b} + \frac{b}{c} + \frac{c}{d} + \frac{d}{a}}{4}\geq
  \sqrt[4]{\frac{a}{b} . \frac{b}{c} . \frac{c}{d} . \frac{d}{a}}$

  $\Rightarrow S\geq 4$. And hence, the inequality is proven.
\item Let L.H.S. be $S$, then using A.M.-G.M. inequality we have

  $\frac{S}{n}\geq \left(\sqrt{\frac{a_1 + a_2}{a_3}}.\sqrt{\frac{a_2 + a_3}{a_4}}\cdots\sqrt{\frac{a_n +
    a_1}{a_2}}\right)^{\frac{1}{n}}$

  $S\geq \left[\left(\frac{a_1 + a_2}{a_3}.\frac{a_2 + a_3}{a_4}\cdots \frac{a_n +
    a_1}{a_2}\right)\right]^{\frac{1}{2n}}$

  Using A.M.-G.M again on terms of numerator we have

  $(a_1 + a_2)(a_2 + a_3)\cdots(a_n + a_1)\geq 2^n(a_1a_2\ldots a_n)$

  Putting this back in previour inequality we have the proof.
\item We will first prove the inequality $\frac{t}{1 + t^2}\leq \frac{3\sqrt{3}}{16}(1 + t^2)$.

  When $t\leq 0$ the inequality is obviously satisfied. For $t > 0$ we rearrange the inequality as $(1 +
  t^2)^2\geq \frac{16t}{3\sqrt{3}}$.

  Consider four numbers $t^2, \frac{1}{3}, \frac{1}{3}, \frac{1}{3}$. We use A.M.-G.M. inequality on these
  to get

  $\frac{t^2 + 1}{4}\geq \sqrt[4]{t^2.\frac{1}{27}}$. Squaring both sides we have, $(1 + t^2)^2\geq
  \frac{16t}{3\sqrt{3}}$.

  Thus, $\frac{t}{1 + t^2}\leq \frac{3\sqrt{3}}{16}(1 + t^2)$ is true for all $t$.

  Putting $t = x, y, z$ and adding proves the desired inequality.
\item $\displaystyle\because \sum_{i = 1}^na_i = 1$, we have $1 - a_j = \displaystyle\sum_{i\neq j}a_i$

  Using A.M.-G.M. inequality on the $n - 1$ terms of $\displaystyle\sum_{i\neq j}a_i$, we get

  $\displaystyle\sum_{i\neq j}a_i\geq (n - 1)\left(\prod_{i\neq j}a_i\right)^{\frac{1}{n - 1}}$

  $\therefore \frac{1}{a_j} - 1\geq \frac{(n - 1)\left(\prod_{i\neq j}a_i\right)^{\frac{1}{n - 1}}}{a_j}$

  $\Rightarrow \displaystyle\prod_{j = 1}^n\left(\frac{1}{a_j} - 1\right)\geq \prod_{j = 1}^n\frac{(n -
    1)\left(\prod_{i\neq j}a_i\right)^{\frac{1}{n - 1}}}{a_j}$

  $= (n - 1)^n\displaystyle\frac{\prod_{j = 1}^n\left(\prod_{i]\neq j}a_i\right)^{\frac{1}{n -
      1}}}{\prod_{j= 1}^na_j}$

  Let $P = \displaystyle\prod_{i = 1}^na_i\Rightarrow \prod_{i\neq j}a_i = \frac{P}{a_j}$ so
  $\left(\displaystyle\prod_{i\neq j}a_i\right)^{\frac{1}{n - 1}} = \left(\frac{P}{a_j}\right)^{\frac{1}{n -
      1}}$

  So the numerator becomes $\displaystyle\prod_{j = 1}^n\left(\frac{P}{a_j}\right)^{\frac{1}{n - 1}} =
  \left(\frac{P^n}{P}\right)^{\frac{1}{n - 1}} = P$

  Thus, $\displaystyle\prod_{j = 1}^n\left(\frac{1}{a_j} - 1\right)\geq (n - 1)^n\frac{P}{P} = (n - 1)^n$

  For the other half of the product we want to show that $\displaystyle\prod_{i = 1}^n\left(\frac{1}{a_i} +
  1\right)\geq (n + 1)^n$

  Let $f(x) = \log(\frac{1}{x} + 1)$ so $f'(x) = -\frac{1}{x(1 + x)}$ and $f''(x) = \frac{1 + 2x}{x^2(1 +
    x)^2}$.

  Since $a_i > 0, x > 0$, so $f''(x) > 0$. Thus, $f(x)$ is a convex function.

  Using Jensen's inequality(which is a consequence of A.M.-G.M. inequality)

  $\displaystyle\frac{1}{n}\sum_{i = 1}^nf(a_i)\geq f\left(\frac{1}{n}\sum_{i = 1}^na_i\right)$

  Substituting $f(a_i) = \displaystyle\log\left(\frac{1}{a_i} + 1\right)$ and $\displaystyle\sum_{i =
    1}^na_i = 1$, we have

  $\displaystyle\frac{1}{n}\sum_{i = 1}^n\log\left(\frac{1}{a_i} + 1\right)\geq \log\left(\frac{1}{\sum_{i =
      1}^na_i} + 1\right) = \log\left(\frac{1}{1/n} + 1\right) = \log(n + 1)$

  $\Rightarrow \displaystyle\log\Biggl(\Bigl(\prod_{i = 1}^n\left(\frac{1}{a_i} +
  1\right)\Bigr)\Biggr)\geq \log(n + 1)$

  Since $\log$ is an increasing function we take antilog of both sides to get

  $\displaystyle\prod_{i = 1}^n\left(\frac{1}{a_i} + 1\right)\geq (n + 1)^n$

  Combining the two inequalities obtained we have proven the desireed inequality.
\item $u = 1 - (v + w)\geq 1 - \left(\frac{7}{16} + \frac{7}{16}\right) = \frac{1}{8}$. So the constraint
  becomes $\frac{1}{8}\leq u, v, z\leq \frac{7}{16}$

  By A.M.-G.M. inequality $\frac{(1 + u) + (1 + v) + (1 + w)}{3}\geq \sqrt[3]{(1 + u)(1 + v)(1 + w)}$

  $\Rightarrow (1 + u)(1 + v)(1 + w)\leq \frac{64}{27}$, which is the maximum value of the product.

  Let $1 + u = x, 1 + v = y$ and $1 + w = z$. We need to minimize the product $P = xyz$ such that $x + y + z
  = 4$ and $\frac{9}{8}\leq x, y, z\leq \frac{23}{16}$.

  A.M.-G.M. inequaltiy states that the product is maximized when $x = y = z$. Conversely the product is
  minimized when they are as far apart as possible.

  Let $k$ variables be $\frac{9}{8}$ and $m$ variables be $\frac{23}{16}$. The sum is $k.\frac{9}{8} +
  m.\frac{23}{16} = 4 \Rightarrow 18k + 23m = 64$ and $k + m = 3$.

  So for $(k, m)$ we have four possibilities and we see that with the exception of case $k = 1, m = 2$ the
  sum condition is not satisfied. And the product in this case is $\frac{4761}{2048}$.
\item Consider A.M.-G.M. inequality of $p, \frac{x}{p}$s and $q, \frac{y}{q}$s.

  $\displaystyle\frac{\underbrace{\frac{x}{p} + \cdots + \frac{x}{p}}_{p\;\rm {times}} + \underbrace{\frac{y}{q} +
    \cdots + \frac{y}{q}}_{q\;\rm {times}}}{p + q}\geq \sqrt[p + q]{\underbrace{\frac{x}{p} \cdots
  \frac{x}{p}}_{p\;\rm {times}}\underbrace{\frac{y}{q} \cdots \frac{y}{q}}_{q\;\rm {times}}} \Rightarrow
  \displaystyle\frac{x + y}{p + q}\geq \sqrt[p + q]{\frac{x^py^q}{p^pq^q}} \Rightarrow
  x^py^q\leq \frac{p^pq^qa^{p + q}}{(p + q)^{p + q}}$.
\item Putting $x = \frac{1}{\sqrt{2}}$ and $x = -\frac{1}{\sqrt{2}}$ we have

  $\frac{a}{2} + \frac{b}{\sqrt{2}} + c\leq \frac{1}{\sqrt{1 - \frac{1}{2}}} = \sqrt{2}$ and $\frac{a}{2}
  - \frac{b}{\sqrt{2}} + c\leq \frac{1}{\sqrt{1 - \frac{1}{2}}} = \sqrt{2}$

  Adding we get $a + 2c\leq 2\sqrt{2}$.
\item We have $\left(1 + \frac{a}{b}\right)\left(1 + \frac{b}{c}\right)\left(1 + \frac{c}{a}\right) = 2
  + \frac{a}{b} + \frac{b}{a} + \frac{b}{c} + \frac{c}{b} + \frac{a}{c} + \frac{c}{a}$.

  Using A.M.-G.M. inequality on $\frac{a}{b}, \frac{a}{c}, \frac{b}{c}$ we have
  $\frac{\frac{a}{b} + \frac{a}{c} + \frac{b}{c}}{3}\geq \sqrt[3]{\frac{a^2}{bc}} = \frac{a}{\sqrt[3]{abc}}$

  Taking cyclic sum of this A.M.-G.M. inequality we prove the result.
\item Let $x_i = \frac{1 + a_i}{1 - a_i}$ so the inequality becomes $\displaystyle\prod_{i = 1}^{n + 1}x_i\geq
  n^{n + 1}$.

  Now, $a_i = \frac{x_i - 1}{x_i + 1}$. So our assumption becomes $\displaystyle\sum_{i = 1}^{n + 1}\frac{x_i -
    1}{x_i + 1}\geq n - 1$

  $\Rightarrow \displaystyle\sum_{i = 1}^{n + 1}\frac{x_i - 1}{x_i + 1}\geq n - 1\Rightarrow \sum_{i = 1}^{n
  + 1}\left(1 - \frac{2}{x_i + 1}\right)\geq n - 1\Rightarrow (n + 1) - 2\sum_{i = 1}^{n + 1}\frac{1}{x_i +
    1}\geq n - 1$

  $\Rightarrow \displaystyle\sum_{i = 1}^{n + 1}\frac{1}{x_i + 1}\leq 1$.

  Considering symmetry. Suppose all $x_i = x$, then: $(n + 1).\frac{1}{x + 1}\leq 1\Rightarrow x\geq n$

  Thus, $\displaystyle\prod_{i = 1}^{n + 1}x_i\geq n^{n + 1}$, and hence, the inequality is proved.
\item Let $A = (a+b)(b+c)(c+d)(d+a)$. Using A.M.-G.M. inequality we have

  $a + b\geq 2\sqrt{ab}$ and so on. Hence, $(a+b)(b+c)(c+d)(d+a) \geq 16abcd$. So inequality becomes
  $A^3\geq 4096a^3b^3c^3d^3$.

  Using A.M.-G.M. again we have $(a + b + c + d)^4\geq 256abcd \Rightarrow abcd\leq \frac{(a + b + c +
    d)^3}{256}$

  $A^3\geq 4096a^3b^3c^3d^3\geq 4096.\frac{(a + b + c + d)^3}{256}.a^2b^2c^2d^2 = 16a^2b^2c^2d^2(a + b + c +
  d)^4$.
\item We have $\left(1 + \frac{a}{b}\right)^2 + \left(1 + \frac{b}{c}\right)^2 + \left(1
  + \frac{c}{a}\right)^2\geq \frac{1}{3}\left(1 + \frac{a}{b} + 1 + \frac{b}{c} + 1 + \frac{c}{a}\right)^2$
  and similar relation for the other term of the L.H.S.

  Using A.M.-G.M. on $1, 1, 1, \frac{a}{b}, \frac{b}{c}, \frac{c}{a}$ we have

  $\displaystyle\frac{3 + \frac{a}{b} + \frac{b}{c}
    + \frac{c}{a}}{6}\geq \sqrt[6]{1}\Rightarrow \frac{1}{3}\left(1 + \frac{a}{b} + 1 + \frac{b}{c} + 1
  + \frac{c}{a}\right)^2 \geq \frac{6^2}{3}$

  For both the terms of L.H.S. the R.H.S. of above inequality becomes $\frac{6^4}{3^2} = 4.6^2$.

  Using A.M.-G.M. on terms of R.H.S.

  $\displaystyle \frac{\frac{a}{b} + \frac{b}{c} + \frac{c}{a} + \frac{b}{a} + \frac{c}{b}
    + \frac{a}{c}}{6}\geq \sqrt[6]{1}$

  $\Rightarrow 4\left(\frac{a}{b} + \frac{b}{c} + \frac{c}{a} + \frac{b}{a} + \frac{c}{b}
  + \frac{a}{c}\right)^2\geq 4.6^2$

  And hence, the inequality is proven.
\item Using A.M.-G.M. inequality $a^2 + bc\geq 2a\sqrt{bc}$ and so on. So L.H.S. $\geq512a^6b^6c^6$

  Similarly, R.H.S. $\geq 512 (a^3b^3c^3)^{3/2}$.

  Thus, given inequality becomes $a^3b^3c^3\geq 1$. Since the ienquality is homogeneous we can assume that
  $abc = 1$ WLOG. And then the inequality holds for equality condition and otherwise.
\item Let $x = \frac{a}{a+ b + c}, y = \frac{b}{a + b + c}$, and $z = \frac{c}{a + b + c}$ so $x + y + z =
  1$

  $\Rightarrow x + \sqrt{xy} + \sqrt[3]{xyz} \leq x + \frac{x + y}{2} + \frac{1}{3} = \frac{9x + 3y +
    2}{6}$. Comparing with $\frac{4}{3}$, we get

  $3x + y\leq 2$, which is always true under the constraint $x + y + z = 1$, and is maximized when $x = y =
  z = \frac{1}{3}$. And thus, $x + \sqrt{xy} + \sqrt[3]{xyz} = 1 < \frac{4}{3}$. And thus, the inequality is
  proven.
\item Using A.M.-G.M. inequality we have $a = a, \sqrt{ab}\leq \frac{a + b}{2}, \sqrt[3]{abc}\leq \frac{a +
  b + c}{3}$

  Thus, we have $a + \sqrt{ab} + \sqrt[3]{abc}\leq a + \frac{a + b}{2} + \frac{a + b + c}{3}$

  Applying A.M.-G.M. on R.H.S. we have

  $a + \frac{a + b}{2} + \frac{a + b + c}{3}\leq \sqrt[3]{a.\frac{a + b}{2}.\frac{a + b + c}{3}}$.

  Combining we have desired inequality.
\item By A.M.-G.M. inequality we have L.H.S. $\leq \left(\frac{a + b}{2}\right)^{5/2} + \left(\frac{b +
  c}{2}\right)^{5/2} + \left(\frac{c + a}{2}\right)^{5/2}$

  Let $x = a + b, y = b + c, z = c + a\Rightarrow x + y + z = 2$

  Consider the function $f(x) = \left(\frac{x}{2}\right)^{5/2}$. This function is convex on $x > 0$, since
  the second derivative $f''(x) = \frac{15}{8\sqrt{2}x^{3/2}}> 0$.

  So by Jensen's inequality we have

  $f(x) + f(y) + f(z)\leq 3f\left(\frac{x + y + z}{3}\right) = 3f\left(\frac{2}{3}\right)
  = \frac{\sqrt{3}}{9}$. And hence, the inequality is proven.
\stopitemize
